% -*-latex-*-
% Document name: /home/gorby/tla/Tex/manpage.tex
% Creator: Terry L Anderson [tla@spot]
% Creation Date: Wed Oct 24 00:28:25 1990
\documentstyle[manpage]{tm}
\newcommand{\X}[1]{{#1}\index{{#1}}}
\title{A Latex Style for UNIX Man Pages}
\author{Terry L Anderson}% 
    \initials{TLA}% 
    \signatureextra{Software Development Environment Group\\Software Architecture Department}%
    \eaddress{tla@bartok.att.com}% 
    \department{59114}%
    \location{LC}{4N-E01}{908-580-4428}% 
    \cityaddr{Warren, NJ\quad 07059-0908}% 
\chargecase{978899-0100}
\filecase{61093}
\documentno{901026}{02}{TMS}
\keywords{man pages; \LaTeX ; Technical Memorandum}
\mercurycode{CMP}
\date{October 26, 1990}
\memotype{INTERNAL MEMORANDUM}
\abstract{Man pages have become the standard form of documentation for software
used in conjunction with the \UNIX\ operating system.  \UNIX\
Documenters Workbench software contains a special macro package for
preparing man pages.  This \LaTeX\ manpage style offers similar
assistance in preparing man pages with \LaTeX .  It offers macros for
most of the common man page structures and sections and formats the
page in the standard man page format.  The style follows many of the
recommendations of the document {\it Manual Page Standards}.}
\organizationalapproval
%\markdraft*
\markproprietary
\begin{document}
\bibliographystyle{atttm}
\makehead
\section{Introduction}
Man pages have become the standard form of documentation for software
used in conjunction with the \UNIX\ operating system.  \UNIX\
Documenters Workbench software contains a special macro package for
preparing man pages.  This \LaTeX\ manpage style offers similar
assistance in preparing man pages with \LaTeX .  It offers macros for
most of the common man page structures and sections and formats the
page in the standard man page format.  The style follows many of the
recommendations of the document {\it Manual Page
Standards}.\cite{bib:manpagestd}  

Since man pages are often printed as appendices in other documents,
the manpage.sty has been constructed in such a way that it can be
loaded as an option to other documentstyles; such as,
tm.sty.\cite{bib:tmdoc}  Manpage is implemented as an environment.
Even if the man page is to be printed as the 
entire document, another document such as tm.sty should be used as the
document style
  \begin{itemize}
    \item[]
          \verb|\documentstyle[manpage]{tm}|\\
          \verb|\begin{document}|\\
          \verb|\begin{manpage}|\\
          {\it manpage code}\\
          \verb|\end{manpage}|\\
          \verb|\end{document}|
  \end{itemize}


\section{Preamble-like Macros}
The following macros are primarily for use in the first part of
a {\tt manpage} environment.  This can be thought of very much like the
preamble before the \verb|\begin{document}|.  These cannot actually be
placed in the preamble of the document since 
\begin{enumerate}
  \item some of these macros conflict with others used in documents (an
        so must be used only inside a manpage environment
  \item a document may contain more than one man page and so more than
        one manpage environment and these macros specify things about
        a single man page
  \item and finally because these macros are defined by the
        \verb|\begin{manpage}| and so are not yet defined at the time
        the preamble is processed.
\end{enumerate}

The macros {\tt NAME} and {\tt title} are synonyms and are used to
specify the title of the man page, usually the command or function
name in all upper case characters.
  \begin{itemize}
    \item[]
          \verb|\NAME{|{\it name-in-caps}\verb|}|\\
          \verb|\title{|{\it name-in-caps}\verb|}|
  \end{itemize}

The {\it section}, usually a digit 1-8 optionally followed by a single
character, is specified by the macro
  \begin{itemize}
    \item[]
          \verb|\section{|{\it section-id}\verb|}|
  \end{itemize}

The  date of last change and the {\it chapter}, an optional string
used to identify a collection of related man pages and printed at the
center of the man page header, are specified by the macros
  \begin{itemize}
    \item[]
          \verb|\update{|{\it date-last-changed}\verb|}|\\
          \verb|\chapter{|{\it chapter-title}\verb|}|
  \end{itemize}

All of the items set by the these preamble-like macros are initially
set to null. 
They may be called in any order.
Though all are optional in the sense that no \LaTeX\ error will occur
if they are not set, {\tt NAME} or {\tt title} and {\tt section} must
be set to create a man page with the standard form.  {\tt update} and
{\tt chapter} are often omitted but their use is recommended.  They
must be called before the end of first page.  We recommend calling
them before any of the {\it body macros}.

\section{Body Macros}

The rest of the macro are for use in the body of the man page; i.e.,
after the \verb|\begin{document}|.  Most are implemented as \LaTeX\
environments and are intended to format the various sections of the
man page.  They must be called in the order in which they should
appear.    

All of the section environments are implemented using special cases of
the {\tt manblock} environment.  The {\tt manblock} environment can be
used directly to implement sections with titles other than those
provided.  For example, a section labeled ``HISTORY'' would be
formatted by
  \begin{itemize}
    \item[]
          \verb|\begin{manblock}{HISTORY}|\\
          \verb|The text of the section which can have|\\
          \verb|as many lines or paragraphs as necessary.|\\
          \verb|\end{manblock}|
  \end{itemize}
Note that the environment takes a required argument specifying the
label. 

The {\tt name} environment is used to format the {\bf NAME} section of
the man page.  The contents are normally the name of the command (or
function) or a comma separated list of names of the commands (or
functions) to be documented followed by a dash (two hyphens in \LaTeX)
and a brief phrase describing the command (or function).  For example
  \begin{itemize}
    \item[]
          \verb|\begin{name}|\\
          \verb|mail, rmail -- send mail to users or read mail|\\
          \verb|\end{name}|
  \end{itemize}

The {\tt synopsis} section shows the usage form of the command or
function along with any options.  For example,
  \begin{itemize}
    \item[]
          \verb|\begin{synopsis}|\\
          \verb|{\bf mail} [ -{\bf epqr} ] [ -{\bf f} {\it file} ]|\\
          \verb|{\bf mail} [ -{\bf t} ] {\it persons}|\\
          \verb|{\bf rmail} [ -{\bf t} ] {\it persons}|\\
          \verb|\end{synopsis}|
  \end{itemize}
There is also a special macro for formatting arguments which are
commonly printed with a minus sign plus one or more
characters in bold and optionally a string argument name in
italic.  
  \begin{itemize}
    \item[]
          \verb|\arg{|{\it str1}\verb|}{|{\it str2}\verb|}|
  \end{itemize}
The first string is printed in bold font, the second in italic.  
Some arguments have a space between the option in bold and the
substitutable parameter in italic; others do not.  The {\tt arg}
macros {\bf does not} supply the space.  If a space is desired the
first character of {\it str2} should be a space.  This could have been
used in the previous example of the synopsis section to format the
first line as  
  \begin{itemize}
    \item[]
          \verb|{\bf mail} [ \arg{-}{epqr}{} ] [ \arg{-}{f}{ file} ]|
  \end{itemize}
Note that any of the three arguments to {\tt arg} can be empty to
print nothing in that font.  The macro is intended to reduce the
number of explicit font changes that must be used and is similar to
some of the troff macros that alternate between two fonts.

The macro is implemented as 
  \begin{itemize}
    \item[]
          \verb|\def\arg#1#2{{\bf#1}\if\relax#2\ \else\{\it#2}\fi}|
  \end{itemize}
If different font choices are desired, simply redefine the macro in
the preamble using other font macros.

The {\tt description} section contains paragraphs describing the
command or function.
  \begin{itemize}
    \item[]
          \verb|\begin{description}|\\
          \verb|The description contains text with as many|\\
          \verb|lines or paragraphs are required|\\
          \verb|\end{description}|
  \end{itemize}

In some man page styles the option or parameter list is included in
the {\tt DESCRIPTION} section.  In others it is a separate labeled
section. The label OPTIONS or PARAMETERS are commonly used for this
section.  Section macros are provided for each of these.
  \begin{itemize}
    \item[]
          \verb|\begin{options}|\\
          {\it the options text}\\
          \verb|\end{options}|

          \verb|\begin{parameters}|\\
          {\it the parameters text}\\
          \verb|\end{parameters}|
  \end{itemize}
These are synonyms except for the label used.  

The {\tt OPTIONS} and the {\tt PARMATERS} sections often contain only
an option list.   Special list environments for are provided; e.g.,
  \begin{itemize}
    \item[]
          \verb|\begin{optionlist}|\\
          \verb|\item[\arg{-r}{}]causes messages to be printed in first-in, first-out order|\\
          \verb|\item[\arg{-f}{ file}causes mail to use {\it file} (e.g., {\bf mbox}) instead of the default {\it mailfile}.|\\
          \verb|\end{optionlist}|
  \end{itemize}
Using these, the {\tt OPTIONS} or {\tt PARAMETERS} sections would look like
  \begin{itemize}
    \item[]
          \verb|\begin{option}\begin{optionlist}|\\
          \verb|\item[\arg{-r}{}]causes messages to be printed in first-in, first-out order|\\
          \verb|\end{optionlist}\end{option}|

          \verb|\begin{parameters}\begin{parameterlist}|\\
          \verb|\item[\arg{-r}{}]causes messages to be printed in first-in, first-out order|\\
          \verb|\end{parameterlist}\end{parameters}|
  \end{itemize}

The name {\tt parameterlist} is a synonym for {\tt optionlist}.  The
option or parameter is used as the label.  The amount of indent for
the text section of the list is specified by \LaTeX\ dimensions {\tt
parametersindent} and {\tt optionsindent}.  These are set to 0.5in but
can be reset by 
  \begin{itemize}
    \item[]
          \verb|\parametersindent=|{\it new-size}\\
          \verb|\optionsindent=|{\it new-size}
  \end{itemize}

The {\tt files} and {\tt environment} environments, like the {\tt
options} or {\it parameters} environment usually contain only lists.  
Special lists for these are also available.
  \begin{itemize}
    \item[]
          \verb|\begin{files}\begin{filelist}|\\
          \verb|\item[|{\it path-and-file-name}\verb|]|{\it description}\\
          \verb|\end{filelist\end{files}|

          \verb|\begin{environment}\begin{environmentlist}|\\
          \verb|\item[|{\it environment-var}\verb|]|{\it description}\\
          \verb|\end{environmentlist}\end{environment}|
  \end{itemize}
The {\tt files} environment can be used without the {\tt filelist}
environment if the user wishes to do custom formatting.  These indents
are also preset, \verb|\fileindent=1in| and
\verb|environmentindent=2in|.  These can also be reset by the user.

The {\tt optionlist}, {\tt parameterlist}, and {\tt environmentlist}
labels are set in bold font by default (since this is the recommended
style, see Section\ref{sec:std}).  The {\tt filelist} labels are set
in roman font.

The other environments provided do no special formatting of their
contents and include
 \begin{itemize}
    \item[]
          \verb|\begin{exitcodes}|\\
          {\it user formatted text}\ldots{}\\
          \verb|\end{exitcodes}|

          \verb|\begin{restrictions}|\\
          {\it user formatted text}\ldots{}\\
          \verb|\end{restrictions}|

          \verb|\begin{examples}|\\
          {\it user formatted text}\ldots{}\\
          \verb|\end{examples}|

          \verb|\begin{diagnostics}|\\
          {\it user formatted text}\ldots{}\\
          \verb|\end{diagnostics}|

          \verb|\begin{extendeddescription}|\\
          {\it user formatted text}\ldots{}\\
          \verb|\end{extendeddescription}|

          \verb|\begin{seealso}|\\
          {\it user formatted text}\ldots{}\\
          \verb|\end{seealso}|

          \verb|\begin{references}|\\
          {\it user formatted text}\ldots{}\\
          \verb|\end{references}|

          \verb|\begin{caveats}|\\
          {\it user formatted text}\ldots{}\\
          \verb|\end{caveats}|

          \verb|\begin{bugs}|\\
          {\it user formatted text}\ldots{}\\
          \verb|\end{bugs}|
  \end{itemize}

Any special formatting of the text in these environment must be
provided by the user using \LaTeX\ formatting commands.

\section{Man Page Standards}\label{sec:std}
The Software Development System project has produced a manual page
standard for departments 55511 and 55513.\cite{bib:manpagestd}  Some
of the issues are specific to that project but many are general and
worth following.  There are good recommendations concerning writing
style.  They recommend against using right justification on manpages.
One can follow this recommendation by placing the {\tt manpage}
environment inside a {\tt flushleft} environment or calling the \TeX\
raggedright command.

\subsection{Fonts}
They recommend the following be set in bold font
  \begin{itemize}
    \item all parameters,
    \item all C and C preprocessor constructs,
    \item all environment variables, etc.,
    \item all error messages (except for variable portions)(e.g., {\bf
          vi: cannot open file}{\it filename}),
    \item all command names, and
    \item section and subsection headings (only when used to mark the
          start of the sectgion -- when they are used in the text of
          the manual page, they are in Roman-type style).
  \end{itemize}

All substitutable items should be set in italics; e.g., filenames and
patterns.  All other parts should be printed in Roman font.
\subsection{Section Ordering}
They recommend the following sections and recommend that they appear
in the indicated order.  They recommend against any addition sections
(although manpage.sty supports a few others that are in common use).
Not all sections are required on all man pages.
  \begin{enumerate}
    \item TITLE (heading line)
    \item NAME
    \item SYNOPSIS
    \item DESCRIPTIONS
    \item PARMETERS (rather than OPTIONS)
    \item ENVIRONMENT
    \item EXIT CODES
    \item RESTRICTIONS
    \item EXAMPLES
    \item DIAGNOSTICS
    \item EXTENDED DESCRIPTION
    \item FILES
    \item SEE ALSO
    \item REFERENCES
  \end{enumerate}

They also make useful recommendations concerning the contents of each
section.

\section{Conclusion}
The author hopes you find this manpage style useful.  Feel free to
direct bug reports, comments or suggestions to him by email, paper
mail or phone.

\makesignature
\noindent Atts.\\
%References\\
Appendixes A-B\par
%Table 1
\vspace{12pt}
\copytocov{{\nobreak\topsep=0pt\partopsep=0pt\begin{tabbing}\nobreak%
\hskip2.5in\=WWW \=\kill%
All Members Department 59114\\
%All Members Department 59112\\
All Supervision Center 5911\\
Y.\ Afek		\>MH\>3D-438\\
E.\ Ayanoglu (vax135!ender)\>HO\>4F-507\\
Sihem Ben Saad (buckaroo!sihem)	\>HO\>3K-602\\
M.\ Beutnagel		\>MH\>2D-513\\
C.\ D.\ Blewett		\>MH\>2C-470\\
D.\ S.\ Booth		\>HO\>4E-526\\
R.\ J.\ Brachman (allegra!rjb)	\>MH\>3C-439\\
B.\ Carpenter (hos1cad!wjc)	\>HO\>1L-410\\
M.\ Carroll (mozart!carroll)	\>LC\>4N-D02\\
Asheem Chandna (hou2d!asheem)	\>HR\>2K-058\\
H.\ H.\ Chefitz		\>LC\>3W-F09\\
Y.\ Chen (ulysses!chen)	\>MH\>3C-535\\
Johanan Codona (whutt!jlc)	\>WH\>15F-326\\
Manuel F.\ Costa (arch3!manuel)	\>HO\>1J-332A\\
D.\ L.\ De Bruler (iexist!dennis)  \>IHP\>1B-104\\
R.\ A.\ Deighan		\>ALC\>1C-343\\
David Dougherty (attunix!dwd)	\>SF\>E-125\\
Elias Drakopoulos (ihlpa!eliasd)	\>IH\>4A-264\\
Robert Ducharme (drutx!rld)	\>DR\>30F-016\\
Bryn Ekroot (hocus!bce)	\>HO\>3K-320\\
David Etherington (allegra!ether)	\>MH\>3C404A\\
P.\ Glassman		\>HO\>3K-324\\
Judith Grass (ulysses!grass)	\>MH\>3C-532C\\
Albert Greenberg (gauss!albert)	\>MH\>2C-119\\
Hsueh-Ming Hang (vax135!hmh)	\>HO\>4C-520\\
Mark Hartman (ihlpm!hartman)	\>IX\>1C-441\\
R.\ S.\ Hauck (ihlpe!micro)	\>IH\>6B-420\\
A.\ I.\ Hauser		\>HR\>2C-227\\
Gary Hurm (ihuxz!hurm)	\>IHP\>1E-546\\
Dan Jacobson		\>IH\>1D-213\\
Henry Kautz (allegra!kautz)	\>MH\>3C-402C\\
T.\ J.\ Kowalski	\>MH\>2C-552
\end{tabbing}}}%\vspace*{-20pt}
\restofcopytocov{{\nobreak\topsep=0pt\partopsep=0pt\begin{tabbing}\nobreak%
\hskip2.5in\=WWW \=\kill
P.\ Kravitz		\>MH\>30H-016\\
Bala Krishnamurthy	\>MH\>3D-561\\
Peter Kroon (alice!kroon)	\>MH\>2D-550\\
G.\ R.\ Kuntz		\>SF\>5345\\
Daniel Jacobson (ihlpa!danj1)	\>IH\>1D-213\\
L.\ S.\ Levy		\>MT\>3G-108\\
M.\ Y.\ Liberman	\>MH\>2D-444\\
J.\ P.\ Linderman	\>MH\>3D-435\\
Robert Lyons (vax135!lyons)	\>HO\>4E-626\\
S.\ Luke Jones (mtung!luke)	\>MT\>2E-337\\
Victor McCrary		\>MH\>2A-337\\
M.\ Murdocca		\>HO\>4G-538\\
R.\ B.\ Murray (mozart!rbm)	\>LC\>4N-W09\\
Steve North (ulysses!north)	\>MH\>3C-539\\
Kostas Oikonomou (honet7!ko)	\>HO\>3M-521\\
Harold Pomeranz (homxa!hrp)	\>HO\>3D-225A\\
Christopher Rath (melmac!car)	\>LC\>3W-G20\\
A.\ Reibman (hoqaa!alr)	\>HO\>2J-502\\
J.\ R.\ Rowland		\>LC\>3W-G25\\
E. Sampieri		\>HO\>4G-616\\
Juergen Schroeter (alice!jsh)	\>MH\>2D-545\\
T.\ Sizer		\>HO\>4G-530\\
Richard Stone (hocus!res)	\>HO\>3K-328\\
V.\ Tavernini		\>AN\>4S-125\\
S.\ K.\ Tewksbury (vax135!skt)	\>HO\>4B-505\\
Mark K.\ Trettin (ihlpy!mkt)	\>IHP\>2A-316\\
H.\ Trickey (coma!trickey)	\>MH\>2C-514\\
J.\ D.\ Weld (whutt!jdw)	\>WH\>4C-228A\\
Robert Wentworth (hoh-1!rhw)	\>HOH\>L-171\\
Alexander Wolf (ulysses!wolf)	\>MH\>3C-533
\end{tabbing}}}
\bibliography{tm}
\appendices
\def\bv{\begin{verbatim}}% These are necessary to get verbatim to 
\section{Example Man Page Source File}
\def\infil{\input man.1.tex}% begin only after input is redirected
\expandafter\bv\infil
\end{verbatim}
\section{Formatted Man Page Resulting from Example Source File}
{\it starts on next page}
\newpage
\begin{manpage}
\NAME{MAN}\section{1}\chapter{USER COMMANDS}\update{90-Oct-22}
\begin{name}
man -- display reference manual pages; find reference pages by keyword
\end{name}
\begin{synopsis}
{\bf man} [-] [\arg{-t}{}] [\arg{-M}{ path}]
  [\arg{-T}{ macro-package}] [[\arg{}{section}] \arg{}{title}
  \ldots{}]\ldots{}\\ 
{\bf man} [\arg{-M}{ path}] \arg{-k}{ keyword} \ldots{}\\
{\bf man} [\arg{-M}{ path}] \arg{-f}{ filename} \ldots{}
\end{synopsis}
\begin{description}
{\it man} displays information from the reference manuals.
It can display complete manual pages that you select by
{\it title}, or one-line summaries selected either by
{\it keyword} (\arg{-k}{}),
or by the name of an associated file
(\arg{-f}{}).

A {\it section}, when given, applies to the
{\it titles} that follow it on the command line (up to the next
{\it section}, if any).
{\bf man} looks in the indicated section of the manual for those
{\bf title}s.  {\it section} is either a digit (perhaps followed by a
single letter indicating the type of manual page), or one of the words 
{\bf new}, {\bf local}, {\bf old}, or {\bf public}.
If {\it section} is omitted, {\bf man} searches all reference sections
(giving preference to commands over functions)
and prints the first manual page it finds.
If no manual page is located, {\bf man} prints an error message.

The reference page sources are typically located in the
{\bf /usr/share/man/man}? directories.  Since these directories are
optionally installed, they may not reside on your host; you may have
to mount  {\bf /usr/share/man} from a host on which they do reside. 
If there are preformatted, up-to-date versions in corresponding
{\bf cat?} or {\bf fmt?} directories, {\bf man}
simply displays or prints those versions.  If the preformatted
version of interest is out of date or missing, {\bf man}
reformats it prior to display.  If directories for the
preformatted versions are not provided,  {\bf man}
reformats a page whenever it is requested; it uses a temporary
file to store the formatted text during display.
 
If the standard output is not a terminal, or if the `{\bf -}'
flag is given, {\bf man} pipes its output through {\bf cat} (1V).
Otherwise, {\bf man} pipes its output through {\bf more} (1)
to handle paging and underlining on the screen.
\end{description}
\begin{options}\begin{optionlist}
  \item[-t]{\bf man} arranges for the specified manual pages to be
        {\bf troff}ed to a suitable raster output device (see {\bf
        troff}(1) or {\bf vtroff}(1)).  If both the {\bf -} and {\bf
        -t} flags are given, {\bf man} updates the {\bf troff}ed
        versions of each named {\it title} (if necessary), but does
        not display them. 
  \item[\arg{-M}{ path}]Change the search path for manual pages.
        {\it path} is a colon-separated list of directories that
        contain manual page directory subtrees.  For example,\\ 
        {\bf/usr/share/man/u\_man:/usr/share/man/a\_man} makes {\bf man}
        search in the standard System V locations.  When used with the
        {\bf -k} or {\bf -f} options, the {\bf -M} option must appear
        first.  Each directory in the {\it path} is assumed to contain
        subdirectories of the form {\bf man[1-8l-p]}.
  \item[\arg{-T}{ macro-package}]\ \\{\bf man} uses {\it macro-package}
        rather than the standard \arg{-man}{} macros defined in {\bf
        /usr/share/lib/tmac/tmac.an} for formatting manual pages.
  \item[\arg{-k}{ keyword\ldots{}}]\ \\{\bf man} prints out one-line
        summaries from the {\bf whatis} database (table of contents)
        that contain any of the given {\it keyword}s.
  \item[\arg{-f}{ filename \ldots{}}]\ \\{\bf man} attempts to locate
        manual pages related to any of the given {\it filename}s.  It
        strips the leading pathname components from each {\it
        filename}, and then prints one-line summaries containing the
        resulting basename or names.
\end{optionlist}\end{options}
\begin{manblock}{MANUAL PAGES}
Manual pages are {\bf troff}(1)/{\bf nroff}(1) source files prepared
with the \arg{-man}{} macro package.  Refer to {\bf man}(7), or {\it
Formatting Document on the Sun Workstation} for more information.

When formatting a manual page, {\bf man} examines the first line to
determine whether it requires special processing.

\hspace{-.2in}{\bf Referring to Other Manual Pages}

If the first line of the manual page is a reference to another
manual page entry fitting the pattern:
\begin{itemize}
\item[]
{\bf .so} {\it man?*/sourcefile}
\end{itemize}
{\bf man} processes the indicated file in place of the current one. 
The reference must be expressed as a pathname relative to to the root
of the manual page directory subtree.

When the second or any subsequent line starts with {\bf .so}, {\bf man}
ignores it; {\bf troff}(1) or {\bf nroff}(1) processes the request in
the usual manner. 
\end{manblock}
\begin{environment}\begin{environmentlist}
\item[MANPATH]If set, its value overrides {\it /usr/share/man} as the
      default search path. (The {\bf -M} flag, in turn, overrides this
      value.)  
\item[PAGER]A program to use for interactively delivering {\bf man}'s
      output to the screen.  If not set, {\bf more -s} (see {\bf
      more}(1)) is used.
\item[TCAT]The name of the program to use to display {\bf troff}ed
      manual pages.  If not set, {\bf lpr -t} (see {\bf lpr}(1)) is
      used. 
\item[TROFF]The name of the formatter to use when the {\bf -t} flag is
      given.  If not set, {\bf troff} is used.
\end{environmentlist}\end{environment}
\begin{files}\begin{filelist}
\item[/usr/man]root of the standard manual page directory subtree
\item[/usr/share/man/man?/*]unformatted manual entries
\item[/usr/share/man/cat?/*]{\bf nroff}ed manual entries
\item[/usr/share/man/fmt?/*]{\bf troff}ed manual entries
\item[/usr/share/man/whatis]table of contents and keyword database
\item[/usr/share/lib/tmac/tmac.an]standard {\bf -man} macro package
\end{filelist}\end{files}
\begin{seealso}
{\bf cat}(1V), {\bf col}(1V), {\bf eqn}(1), {\bf more}(1), {\bf
nroff}(1), {\bf tbl}(1), {\bf troff}(1), {\bf whatis}(1), {\bf
man}(7), {\bf catman}(8)
\end{seealso}
\begin{caveats}
The manual is supposed to be reproducible either on a phototypesetter
or on an {\small ASCII} terminal.  However, on a terminal some
information (indicated by font changes, for instance) is necessarily
lost. 

Some dumb terminals cannot process the vertical motions produced
by the {\bf e} ({\bf eqn}(1)) preprocessing flag.  To prevent garbled
output on these terminals, when you use {\bf e} also use {\bf t}, to
invoke {\bf col}(1V) implicitly.  This workaround has the disadvantage
of eliminating superscripts and subscripts --- even on those
terminals that can display them.  {\small CTRL-Q} will clear a
terminal that gets confused by {\bf eqn}(1) output.
\end{caveats}
\end{manpage}
\tableofcontents
%\listoffigures
\coversheet
\end{document}
