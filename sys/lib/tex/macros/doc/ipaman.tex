\input ipamacs
\asisformat{\skipbefore{6pt}
            \everyasis{\leftindent{.5in}\rightindent{.5in}\tt}
            \skipafter{6pt}}
\font\smokefont=wsuipa17
\newbox\tagbox
\setbox\tagbox=\hbox{\tenpt\bf Pullum \&\ Ladusaw name:}
 
\labelformat{\skipbefore{4pt}
             \labelwidth{\wd\tagbox}
             \gutter{1em}
             \everylabel{\tenpt\bf}
}
\rhf{\oddpages{\twelvept\line{\lft{{\it WSUIPA Users Guide}}\rt{\pn}}\vs{\bl}}
     \evenpages{\twelvept\line{\lft{\pn}\rt{{\it WSUIPA Users Guide}}}\vs{\bl}}}
\font\mf=manfnt
\def\deg{$^\circ$}
 
\def\meta{{\mf METAFONT}}
\def\wsu{{\tenpt WSUIPA}}
\def\ipam{{\tenpt IPAMACS}}
\def\PL{{\it Phonetic Symbol Guide}}
 
\def\pos#1{\vs{10pt}\hrule width 1in\vs{1pt}\hrule width 1in
    \noindent\label{\hfill\tenbf WSUIPA Char' Position:}\rm #1\par}
\def\pl#1{\noindent\label{\hfill\tenbf Pullum \&\ Ladusaw name:}\rm #1\par}
\def\mac#1{\noindent\label{\hfill\tenbf IPAMACS Name:}\tt\char'134 #1\par}
\def\smoke#1{\noindent\label{\hfill\tenbf IPA Character:}
\smokefont\char#1\par}
\sfs{cm14}{fourteenpt}{default}
\fourteenpt
\cl{\bd Using the WSU International Phonetic Alphabet}
\cl{January 9, 1990}
\twelvept
 
\subheada{Preface}
 
This users guide is intended to serve as a help and supplement to the
Washington State University International Phonetic Alphabet (\wsu) fonts.
It is not intended to be a manual on \TeX, linguistics, or the correct usage of
the characters, accents and diacritics contained within the font.
It is assumed that the user is familiar with the version of \TeX\ he or she is
using and with the characters contained within the \wsu\ font.
Because the uses for a phonetic alphabet font are probably as varied as the
many disciplines which use the characters, this guide is intended to merely
show some of the most common ways of using the \wsu\ font. Surely each person
who implements the font will develop his or her own favorite way of working with
 and
using the font.
 
The principal
source of information regarding the characters and their shapes has come from
sources within the linguistic field---most
notably, Geoffrey K. Pullum and William A. Ladusaw, whose book, {\it Phonetic
Symbol Guide} was used extensively.
The \PL\  not only lists the International Phonetic Alphabet (IPA) characters
and diacritics, it also
gives an enlarged illustration of each character which shows its baseline,
height, depth and x-height. These character illustrations were invaluable
in the design of the \wsu\ \meta\ characters.
 
Several other people involved
in the wonderful world of linguistics also aided in the design of the \wsu\
font. Their input concerning everything
from character shape, to placement within the font, to determining which
characters should be included, is gratefully acknowledged. I would especially
like to thank Karen Mullen, Associate Professor of English, University of
Louisville, Louisville, Kentucky.  She tested, used the \wsu\ fonts, and took
the time to give several suggestions and recommendations that have improved
the font. Also, Christina A. Thiele, Managing Director, Journal Production
 Centre,
Carleton University, Ottawa, Ontario, who not only provided a wealth
of phonetic information, but also a wealth of encouragement throughout
the entire font designing process.
 
 
Included with the \wsu\ fonts are:
 
\listbegin
\lil1
The basic \wsu\ font which contains 128 phonetic characters
and/or diacritics in five different point sizes (8,
9, 10, 11 and 12) and in
three typefaces (roman, slanted and bold extended).
 
\lil1 Each size and typeface includes a TFM (\TeX\ Font Metric)
file and its related GF, PK or PXL file.
 
\lil1 A macro package (\ipam.TEX) for use with the \wsu\ font, and
 
\lil1 the {\it WSUIPA Font User's Guide}.
\listend
 
This guide is divided into three main sections: 1) an introduction to the \wsu\
font,
2) a description of the font layout and the characters included in the font
and 3) some \TeX niques for use with the \wsu\  fonts.
 
Although the \wsu\ fonts are written in \meta , the user does not need
experience with or knowledge of \meta\ to use the font. However, at least a
beginning understanding of \TeX\ and the use of fonts in the \TeX\ environment
{\it is} necessary. For information about either \TeX\ or font usage in \TeX ,
see {\it The \TeX book} by Donald Knuth.
 
\subheada{Introduction to the WSUIPA Font}
 
The \wsu\  fonts are written in \meta\ and use the same font parameter values
as their Computer Modern (CM) counterparts (cmr and cmbx). In fact, while
designing the character shapes, the computer modern code was left intact or
modified only slightly whenever possible. Therefore, the \wsu\  fonts
are totally compatible with computer modern fonts---the design and shape
of the characters are computer modern wherever possible. For the characters
that are drastically different from any computer modern characters or symbols,
such as the ``Gamma,'' ``Baby Gamma'' and the ``Esh,'' Pullum and Ladusaw's
{\it Phonetic Symbol Guide} was used as the character design
standard.
 
The \wsu\  fonts are obviously not an exhaustive collection of phonetic or
even recognized International Phonetic Alphabet characters. The characters
included in the \wsu\  font were chosen either because they were listed
as a ``major'' entry in Pullum and Ladusaw's {\it Guide\/},
or because of the recommendations from
various people with an interest in the font.
 
The {\it Guide} contains what it classifies as ``major'' and ``minor''
characters, based upon whether the symbol is an officially recognized IPA
character or is determined by Pullum and Ladusaw
to be a standard symbol in current American
transcriptual practice. The {\it Guide}
lists 78 ``major'' entries which are not easily
attainable in the CM fonts. All 78 of these characters are included in the
\wsu\  font, along with 50 ``minor'' entries from the {\it Guide}.
The selection of which
``minor'' entries to include was based on information and requests from
various phoneticians and linguists who contacted WSU during the beginning
stages of the creation of the font.
 
\subheada{The Layout of the WSUIPA Font}
 
The character-grouping pattern followed by Pullum and Ladusaw in their
{\it Guide} was adopted as the basis
for the \wsu\ font layout. The characters
are grouped together according to shape rather than usage. Therefore,
the ``a'' shapes occupy
the first positions: '00 through '04, ``b'' shapes are in positions '05
through '11 and so forth. The accents and diacritics follow the character
shapes and are in the last positions in the font. This approach was taken
because under it was assumed that under most circumstances,
the user would most likely be accessing the characters with the use of macros
rather than by typing large portions of entirely phonetic text.
Aside from keeping like shapes together, the positioning of the
\wsu\  font was fairly arbitrary.
 
\subheada{Character Description}
 
Within this section each of the \wsu\  characters are shown in a \meta\
``smoke mode'' proof. The octal character position is indicated, along with the
corresponding macro name in \ipam, and the
name of the character given in P\&L.
Only the roman characters will be illustrated since the shapes of the slanted
and bold extended are the same.
 
There are several Greek characters included in the \wsu\ font, and initially
it may appear these are merely copies of the corresponding CM Greek characters.
That is true with one very important distinction: the \wsu\ Greek characters
included are not italic. All the lowercase Greek characters included in
the CM fonts are italic which makes them unsuitable for some situations in
phonetics where a non-italic Greek character is essential.
\newpage
\raggedbottom
\vbox{ \pos{'00}\mac{inva}\pl{turned a}\smoke{'00}}
 
\vbox{ \pos{'01}\mac{scripta}\pl{script a}\smoke{'01}}
 
\vbox{ \pos{'02}\mac{nialpha}\pl{lowercase non-italic alpha}\smoke{'02}
}
 
\vbox{\pos{'03}\mac{invscripta}\pl{turned script a}\smoke{'03}}
 
\vbox{\pos{'04}\mac{invv}\pl{inverted v}\smoke{'04}}
 
\vbox{\pos{'05}\mac{crossb}\pl{crossed b}\smoke{'05}}
 
\vbox{\pos{'06}\mac{barb}\pl{barred b}\smoke{'06}}
 
\vbox{\pos{'07}\mac{slashb}\pl{slashed b}\smoke{'07}}
 
\vbox{\pos{'10}\mac{hookb}\pl{hooktop b}\smoke{'10}}
 
\vbox{\pos{'11}\mac{nibeta}\pl{non-italic lowercase beta}\smoke{'11}}
 
\vbox{\pos{'12}\mac{slashc}\pl{slashed c}\smoke{'12}}
 
\vbox{\pos{'13}\mac{curlyc}\pl{curly-tail c}\smoke{'13}}
 
\vbox{\pos{'14}\mac{clickc}\pl{stretched c}\smoke{'14}}
 
\vbox{\pos{'15}\mac{crossd}\pl{crossed d}\smoke{'15}}
 
\vbox{\pos{'16}\mac{bard}\pl{barred d}\smoke{'16}}
 
\vbox{\pos{'17}\mac{slashd}\pl{slashed d}\smoke{'17}}
 
\vbox{\pos{'20}\mac{hookd}\pl{hooktop d}\smoke{'20}}
 
\vbox{\pos{'21}\mac{taild}\pl{right-tail d}\smoke{'21}}
 
\vbox{\pos{'22}\mac{dz}\pl{d-yogh ligature}\smoke{'22}}
 
\vbox{\pos{'23}\mac{eth}\pl{eth}\smoke{'23}}
 
\vbox{\pos{'24}\mac{scd}\pl{small capital D}\smoke{'24}}
 
\vbox{\pos{'25}\mac{schwa}\pl{schwa}\smoke{'25}}
 
\vbox{\pos{'26}\mac{er}\pl{right-hook schwa}\smoke{'26}}
 
\vbox{\pos{'27}\mac{reve}\pl{reversed e}\smoke{'27}}
 
\vbox{\pos{'30}\mac{niepsilon}\pl{non-italic greek epsilon}\smoke{'30}}
 
\vbox{\pos{'31}\mac{revepsilon}\pl{reversed non-italic epsilon}\smoke{'31}}
 
\vbox{\pos{'32}\mac{hookrevepsilon}\pl{right-hook reversed non-italic epsilon}
\smoke{'32}}
 
\vbox{\pos{'33}\mac{closedrevepsilon}\pl{closed reversed non-italic epsilon}
\smoke{'33}}
 
\vbox{\pos{'34}\mac{scriptg}\pl{lowercase variant g}\smoke{'34}}
 
\vbox{\pos{'35}\mac{hookg}\pl{hooktop g}\smoke{'35}}
 
\vbox{\pos{'36}\mac{scg}\pl{small capital G}\smoke{'36}}
 
\vbox{\pos{'37}\mac{nigamma}\pl{non-italic gamma}\smoke{'37}}
 
\vbox{\pos{'40}\mac{ipagamma}\pl{IPA Gamma}\smoke{'40}}
 
\vbox{\pos{'41}\mac{babygamma}\pl{baby gamma}\smoke{'41}}
 
\vbox{\pos{'42}\mac{hv}\pl{h-v ligature}\smoke{'42}}
 
\vbox{\pos{'43}\mac{crossh}\pl{crossed h}\smoke{'43}}
 
\vbox{\pos{'44}\mac{hookg}\pl{hooktop g}\smoke{'44}}
 
\vbox{\pos{'45}\mac{hookheng}\pl{hooktop heng}\smoke{'45}}
 
\vbox{\pos{'46}\mac{invh}\pl{turned h}\smoke{'46}}
 
\vbox{\pos{'47}\mac{bari}\pl{barred i}\smoke{'47}}
 
\vbox{\pos{'50}\mac{dlbari}\pl{barred dotless i}\smoke{'50}}
 
\vbox{\pos{'51}\mac{niiota}\pl{non-italic greek iota}\smoke{'51}}
 
\vbox{\pos{'52}\mac{sci}\pl{small capital I}\smoke{'52}}
 
\vbox{\pos{'53}\mac{barsci}\pl{barred small capital I}\smoke{'53}}
 
\vbox{\pos{'54}\mac{invf}\pl{barred dotless j}\smoke{'54}}
 
\vbox{\pos{'55}\mac{tildel}\pl{l with tilde}\smoke{'55}}
 
\vbox{\pos{'56}\mac{barl}\pl{barred l}\smoke{'56}}
 
\vbox{\pos{'57}\mac{latfric}\pl{belted l}\smoke{'57}}
 
\vbox{\pos{'60}\mac{taill}\pl{l with right tail}\smoke{'60}}
 
\vbox{\pos{'61}\mac{lz}\pl{l-yogh ligature}\smoke{'61}}
 
\vbox{\pos{'62}\mac{nilambda}\pl{non-italic greek lambda}\smoke{'62}}
 
\vbox{\pos{'63}\mac{crossnilambda}\pl{crossed lambda}\smoke{'63}}
 
\vbox{\pos{'64}\mac{labdentalnas}
\pl{m with leftward tail at right}\smoke{'64}}
 
\vbox{\pos{'65}\mac{invm}\pl{turned m}\smoke{'65}}
 
\vbox{\pos{'66}\mac{legm}\pl{turned m with long right leg}\smoke{'66}}
 
\vbox{\pos{'67}\mac{nj}\pl{n with leftward hook at left}\smoke{'67}}
 
\vbox{\pos{'70}\mac{eng}\pl{eng}\smoke{'70}}
 
\vbox{\pos{'71}\mac{tailn}\pl{n with right tail}\smoke{'71}}
 
\vbox{\pos{'72}\mac{scn}\pl{small capital N}\smoke{'72}}
 
\vbox{\pos{'73}\mac{clickb}\pl{bull's eye}\smoke{'73}}
 
\vbox{\pos{'74}\mac{baro}\pl{barred o}\smoke{'74}}
 
\vbox{\pos{'75}\mac{openo}\pl{open o}\smoke{'75}}
 
\vbox{\pos{'76}\mac{niomega}\pl{non-italic lowercase greek omega}\smoke{'76}}
 
\vbox{\pos{'77}\mac{closedniomega}\pl{closed omega}\smoke{'77}}
 
\vbox{\pos{'100}\mac{oo}\pl{double o}\smoke{'100}}
 
\vbox{\pos{'101}\mac{barp}\pl{barred p}\smoke{'101}}
 
\vbox{\pos{'102}\mac{thorn}\pl{thorn}\smoke{'102}}
 
\vbox{\pos{'103}\mac{niphi}\pl{non-italic lowercase greek phi}\smoke{'102}}
 
\vbox{\pos{'104}\mac{flapr}\pl{fish hook r}\smoke{'104}}
 
\vbox{\pos{'105}\mac{legr}\pl{r with long leg}\smoke{'105}}
 
\vbox{\pos{'106}\mac{tailr}\pl{r with right tail}\smoke{'106}}
 
\vbox{\pos{'107}\mac{invr}\pl{turned r}\smoke{'107}}
 
\vbox{\pos{'110}\mac{tailinvr}\pl{turned r with right tail}\smoke{'110}}
 
\vbox{\pos{'111}\mac{invlegr}\pl{turned long-legged r}\smoke{'111}}
 
\vbox{\pos{'112}\mac{scr}\pl{small capital R}\smoke{'112}}
 
\vbox{\pos{'113}\mac{invscr}\pl{inverted small capital R}\smoke{'113}}
 
\vbox{\pos{'114}\mac{tails}\pl{s with right tail}\smoke{'114}}
 
\vbox{\pos{'115}\mac{esh}\pl{esh}\smoke{'115}}
 
\vbox{\pos{'116}\mac{curlyesh}\pl{curly-tail esh}\smoke{'116}}
 
\vbox{\pos{'117}\mac{nisigma}\pl{non-italic lowercase greek sigma}\smoke{'117}}
 
\vbox{\pos{'120}\mac{tailt}\pl{t with right tail}\smoke{'120}}
 
\vbox{\pos{'121}\mac{tesh}\pl{t-esh ligature}\smoke{'121}}
 
\vbox{\pos{'122}\mac{clickt}\pl{turned t}\smoke{'122}}
 
\vbox{\pos{'123}\mac{nitheta}\pl{non-italic lowercase greek theta}\smoke{'123}}
 
\vbox{\pos{'124}\mac{baru}\pl{barred u}\smoke{'124}}
 
\vbox{\pos{'125}\mac{slashu}\pl{slashed u}\smoke{'124}}
 
\vbox{\pos{'126}\mac{niupsilon}\pl{non-italic lowercase greek
 upsilon}\smoke{'126}}
 
\vbox{\pos{'127}\mac{scu}\pl{small capital U}\smoke{'127}}
 
\vbox{\pos{'130}\mac{barscu}\pl{barred small capital U}\smoke{'130}}
 
\vbox{\pos{'131}\mac{scriptv}\pl{script v}\smoke{'131}}
 
\vbox{\pos{'132}\mac{invw}\pl{inverted w}\smoke{'132}}
 
\vbox{\pos{'133}\mac{nichi}\pl{non-italic lowercase greek chi}\smoke{'133}}
 
\vbox{\pos{'134}\mac{invy}\pl{turned y}\smoke{'134}}
 
\vbox{\pos{'135}\mac{scy}\pl{small capital Y}\smoke{'135}}
 
\vbox{\pos{'136}\mac{curlyz}\pl{curly-tail z}\smoke{'136}}
 
\vbox{\pos{'137}\mac{tailz}\pl{z with right tail}\smoke{'137}}
 
\vbox{\pos{'140}\mac{yogh}\pl{yogh}\smoke{'140}}
 
\vbox{\pos{'141}\mac{curlyyogh}\pl{curly-tail yogh}\smoke{'141}}
 
\vbox{\pos{'142}\mac{glotstop}\pl{glottal stop}\smoke{'142}}
 
\vbox{\pos{'143}\mac{revglotstop}\pl{reversed glottal stop}\smoke{'143}}
 
\vbox{\pos{'144}\mac{invglotstop}\pl{inverted glottal stop}\smoke{'144}}
 
\vbox{\pos{'145}\mac{ejective}\pl{ejective}\smoke{'145}}
 
\vbox{\pos{'146}\mac{reveject}\pl{reversed ejective}\smoke{'146}}
 
\vbox{\pos{'147}\mac{dental\#1}\pl{subscript bridge}\smoke{'147}}
 
\vbox{\pos{'150}\mac{stress}\pl{vertical stroke (superior)}\smoke{'150}}
 
\vbox{\pos{'151}\mac{secstress}\pl{vertical stroke (inferior)}\smoke{'151}}
 
\vbox{\pos{'152}\mac{syllabic}\pl{syllabicity mark}\smoke{'152}}
 
\vbox{\pos{'153}\mac{corner}\pl{corner}\smoke{'153}}
 
\vbox{\pos{'154}\mac{upt}\pl{IPA pointer}\smoke{'154}}
 
\vbox{\pos{'155}\mac{downt}\pl{IPA pointer}\smoke{'155}}
 
\vbox{\pos{'156}\mac{leftt}\pl{IPA pointer}\smoke{'156}}
 
\vbox{\pos{'157}\mac{rightt}\pl{IPA pointer}\smoke{'157}}
 
\vbox{\pos{'160}\mac{halflength}\pl{half-length mark}\smoke{'160}}
 
\vbox{\pos{'161}\mac{length}\pl{length mark}\smoke{'161}}
 
\vbox{\pos{'162}\mac{underdots}\pl{subscript umlaut}\smoke{'162}}
 
\vbox{\pos{'163}\mac{ain}\pl{reversed apostrophe}\smoke{'163}}
 
\vbox{\pos{'164}\mac{upp}\pl{pointer}\smoke{'164}}
 
\vbox{\pos{'165}\mac{downp}\pl{pointer}\smoke{'165}}
 
\vbox{\pos{'166}\mac{leftp}\pl{pointer}\smoke{'166}}
 
\vbox{\pos{'167}\mac{rightp}\pl{pointer}\smoke{'167}}
 
\vbox{\pos{'170}\mac{overring}\pl{over-ring}\smoke{'170}}
 
\vbox{\pos{'171}\mac{underring}\pl{under-ring}\smoke{'171}}
 
\vbox{\pos{'172}\mac{open}\pl{subscript left half-ring}\smoke{'172}}
 
\vbox{\pos{'173}\mac{midtilde}\pl{superimposed (mid-) tilde}\smoke{'173}}
 
\vbox{\pos{'174}\mac{undertilde}\pl{subscript tilde}\smoke{'174}}
 
\vbox{\pos{'175}\mac{underwedge}\pl{subscript wedge}\smoke{'175}}
 
\vbox{\pos{'176}\mac{polishhook}\pl{polish hook}\smoke{'176}}
 
\vbox{\pos{'177}\mac{underarch}\pl{subscript arch}\smoke{'177}}
 
\newpage
\normalbottom
\subheada{Using the IPAMACS Macros}
 
In addition to the necessary font-related files, the \wsu\  font package
also includes a file called IPAMACS.TEX. This is a macro file which includes
font declarations and definitions for use with the \wsu\  fonts.
It also contains macro definitions for accessing each \wsu\  character
mnemonically, as well as some macro definitions
to simplify using \wsu\  accents and diacritics with Computer Modern
characters and vice versa.
 
Before you can use any of the macros from the \ipam\ file, you must input the
{\tt IPAMACS.TEX} file by entering,
\aib
\\input ipamacs
\aie
\nin at the top of your \TeX\ file. This assumes, of course, that you have
installed this file.
The \ipam\ file defines a font called
\\ipa to be the twelve point roman \wsu\
font. The \\ipa font is used in all the macro definitions in \ipam\
 to access the
characters from the \wsu\ font. You may redefine \\ipa, say to be ten point
roman, by entering
\aib
\\font\\ipatenrm=wsuipa10
\\def\\ipa\{\\ipatenrm\}
\aie
\nin within your file. From that point on, all the \wsu\ characters accessed by
using the \ipam\ macros will be the ten-point roman \wsu.
 
To access individual \wsu\ characters within sentences set in Computer Modern,
you need only to enter the \\ipa command. For example,
\asisbegin
A shibilant is a term occasionally found for a fricative
corresponding to
a ``hushing'' sound, e.g., IPA [\\esh] (more technically,
a grooved laminal fricative).
\asisend
\nin which will print as:
\vs{5pt}
\pb{
A shibilant is a term occasionally found for a fricative
corresponding to
a ``hushing'' sound, e.g., IPA [\esh] (more technically,
a grooved laminal fricative).
}
\vs{5pt}
 
The \ipam\ character macros are defined so they can be used with the CM
characters and accents without the need for delimiting curly braces. By using
the \ipam\
definitions, you can use the CM accents with a \wsu\ character in the
same way you would accent a CM character. The following input
 
\asisbegin
The superscript tilde is a nasalization marker for vowels,
thus [\\\tilde\\scripta] is a nasalized [\\scripta].
\asisend
\nin will print as:
\vs{5pt}
\pb{The superscript tilde is a nasalization marker for vowels, thus
[\~\scripta] is a nasalized [\scripta].
}
\vs{5pt}
 
The \wsu\ accents can be used with CM characters. However, only the \wsu\
``\\dental'' and ``\\underarch'' have been defined in \ipam\ and both are
diacritics which go under a character.
To use any other \wsu\ character as an accent either with a CM character or with
another \wsu\ character, you can define the \wsu\ accent
in the same way that the CM accents are defined. For example, an ``over-ring''
accent could be defined as,
 
\asisbegin
\\def\\or\#1\{\{\\edef\\next\{\\the\\font\}\%
          \\ipatwelverm\\accent"78\\next\#1\}\}
\aie
\nin and may be used like,
 
\aib
The over-ring may be used over letters with descenders as an
alternative to under-ring to indicate devoicing, e.g. [\\or g].
\asisend
\nin which will print as:
\vs{5pt}
\pb{\def\or#1{{\edef\next{\the\font}%
          \ipatwelverm\accent"78\next#1}}
The over-ring may be used over letters with descenders as an
alternative to under-ring to indicate devoicing, e.g. [\or g].}
\vs{5pt}
 
Most of the \wsu\ diacritics are intended to go under characters, like the
``underarch'' and ``dental'' mentioned above. You may define other ``under''
accents in the same way as these two. For example, an ``undercircle''
could be defined as,
 
\asisbegin
\\def\\undercirc\#1\{\\oalign\{\#1\\crcr
                 \\hidewidth\\underring\\hidewidth\}\}
\asisend
\nin With this definition, the input
\aib
A voiceless trilled r [\\undercirc r] in
certain Scottish dialects\el3
\aie
\nin will print as:
\vs{5pt}
\pb{\def\undercirc#1{\oalign{#1\crcr\hidewidth\underring\hidewidth}}
A voiceless trilled r [\undercirc r] in certain Scottish dialects\el3
}
\vs{5pt}
 
If the spacing or placement of the diacritic is not exactly what you desire
with the basic definition given above, you can add kerns where needed.
For example, if you want less space between the ``undercircle'' and
the character, you could change the definition to be,
\asisbegin\lin{-10pt}
\\def\\undercirc\#1\{\\oalign\{\#1\\crcr
           \\hidewidth\\raise.1ex\\hbox\{\\underring\}\\hidewidth\}\}
\asisend
\np
\nin and this input
\aib
A voiceless trilled r [\\undercirc r] in
certain Scottish dialects\el3
\aie
\nin will print as:
\vs{5pt}
\pb{\def\undercirc#1{\oalign{#1\crcr\hidewidth
\raise.1ex\hbox{\underring}\hidewidth}}
A voiceless trilled r [\undercirc r] in certain Scottish dialects\el3
}
\vs{5pt}
 
Your diacritic definitions may be as general or specific as you wish. At one
institution which uses the \wsu\ fonts, they prefer the ``undercircle'' to be
placed differently when it falls under an r than any other character. To
accomplish this, they use the following definition for \\undercirc.
 
\aib\lin{-10pt}
\\def\\undercirc\#1\{\\ifx\#1r
      \\oalign\{\#1\\crcr\\hidewidth\\kern.24em\\underring
            \\hidewidth\\crcr\}
      \\else\\oalign\{\#1\\crcr
            \\hidewidth\\raise.1ex\\hbox\{\\underring\}\\hidewidth\}
                      \\fi\}
\aie
 
With this definition of \\undercirc, the following input
 
\aib
To illustrate the different placement of the \\underring on
an \\undercirc s and on an \\undercirc r.
\aie
\nin will print as
\vs{5pt}
\pb{\def\undercirc#1{\ifx#1r
     \oalign{#1\crcr\hidewidth\kern.24em\underring\hidewidth\crcr}\fi}
To illustrate the different placement of the \\underring on an \\undercirc s
and on an \undercirc r.
}
\vs{5pt}
 
\ipam\
also includes a macro called \\diatop which provides an alternative way
of getting one or more characters, accents, or diacritics over one another. The
\\diatop macro takes one argument which is delimited by square brackets ([])
rather than curly braces, and has two parts. The first part of the
argument is delimited, or separated, from the second part by a vertical bar
($\vert$). \\diatop puts the first part of the argument over the second.
 
Using \\diatop instead of our previous definition for the ``overring,'' (\\or)
you could input,
 
\aib
The overring may be used over letters with descenders as
an alternative to
under-ring to indicate devoicing, e.g. [\\diatop[\\overring$\vert$g]].
\aie
\nin and the output would be,
\vs{5pt}
\pb{The overring may be used over letters with descenders as an alternative to
under-ring to indicate devoicing, e.g. [\diatop[\overring|g]].
}
\vs{5pt}
 
More than one character can be stacked over another character by using
\\diatop. For example,
 
\asisbegin
For a really special \\diatop[\{\\diatop\{\\'$\|$\\overring]\}$\|$n]
\asisend
\nin will print as:
\vs{5pt}
\pb{For a really special \diatop[{\diatop[\'|\overring]}|n]
}
\vs{5pt}
 
\nin Notice the use of curly braces to group the argument of the first \\diatop
when more than one \\diatop command is used.
 
It is also possible to get one or more accents over a character and another
accent or character under it. Assuming that \\undercirc has been defined as
described earlier, the following example
 
\asisbegin\lin{-10pt}
This is a really, really special
\\diatop[\\overring$\|$\\undercirc\{r\}]
\asisend
\nin will print as:
\vs{5pt}
\pb{\def\undercirc#1{\oalign{#1\crcr\hidewidth\underring\hidewidth}}
This is a really, really special \diatop[\overring|\undercirc{r}]
}
\vs{5pt}
 
The \ipam\ do not sent up the \wsu\ fonts to be part of a font family
definition, which means that any \wsu\ character that is accessed by
an \ipam\
macro, will always print out in the same font (wsuipa12, by default),
regardless of what size or typeface you may have been using when
the \ipam\
macro was called. In other words, whatever font \\ipa is defined to be,
determines the font \ipam\ will call.
 
In order to get slanted \wsu\ characters when \\it or \\sl is being used, or
bold \wsu\ characters when \\bf is being used, you must include the definition
of \\ipa as part of your font family definitions.
 
Without redefining \\ipa, the following input,
 
\aib\lin{0pt}
\\dots which is the case in the aforementioned
instances, however, \{\\bf this
does not preclude the other retroflex consonants:
[\\nj], [\\taill], [\\taild], and [\\tailr]\}.
\aie
\nin will print as,
 
\vs{5pt}
\pb{\dots which is the case in the aforementioned instances, however, {\bf this
does not preclude the other retroflex consonants: [\nj], [\taill], [\taild],
and [\tailr]}.
}
\vs{5pt}
 
Now, with \\ipa defined within the twelve-point font family, as illustrated
below
 
\aib\lin{-10pt}
\\font\\twelverm=cmr12
\\font\\twelvei=cmmi12
\\font\\twelvesy=cmsy12
\\font\\twelveex=cmex12
\\font\\twelveit=cmti12
\\font\\twelvebf=cmbx12
\\font\\twelveipa=wsuipa12
\\font\\twelveslipa=wslipa12
\\font\\twelvebfipa=wbxipa12
 
\\def\\twelvepoint\{\%
  \\textfont0=\\twelverm
      \\scriptfont0=\\sevenrm  \\scriptscriptfont0=\\sevenrm
  \\def\\rm\{\\fam0\\twelverm\\def\\ipa\{\\twelvermipa\}\}\%
  \\textfont1=\\twelvei
  \\scriptfont1=\\sevenrm  \\scriptscriptfont1=\\sevenrm
  \\textfont2=\\twelvesy
  \\scriptfont2=\\sevensy   \\scriptscriptfont2=\\sevensy
  \\textfont3=\\twelveex
  \\scriptfont3=\\twelveex  \\scriptscriptfont3=\\twelveex
  \\textfont\\itfam=\\twelveit
  \\def\\it\{\\fam\\itfam\\twelveit\\def\\ipa\{\\twelveslipa\}\}\%
  \\textfont\\bffam=\\twelvebf
  \\def\\bf\{\\fam\\bffam\\twelvebf\\def\\ipa\{\\twelvebfipa\}\}\%
  \\rm\}\%
\%
\\twelvepoint
\%
\aie
\nin the previous input of
\aib\lin{0pt}
\\dots which is the case in the aforementioned
instances, however, \{\\bf this
does not preclude the other retroflex consonants:
[\\nj], [\\taill], [\\taild], and [\\tailr]\}.
\aie
\nin will print as,
 
\vs{5pt}
\pb{\font\twelvebfipa=wbxipa12\font\twelvebf=cmbx12\def\bf{\fam\bffam\twelvebf
\def\ipa{\twelvebfipa}}%
\dots which is the case in the aforementioned instances, however, {\bf this
does not preclude the other retroflex consonants: [\nj], [\taill], [\taild],
and [\tailr]}.
}
\np
\parindent 0pt
\twelvept
% macros for font tables
\def\oct#1{\hbox{\rm\'{}\kern-.1em\it#1\/\kern.05em}}
\def\hex#1{\hbox{\rm\H{}\tt#1}} % hexadecimal constant
 
\abovedisplayskip=3pt\belowdisplayskip=2pt
\font\smalltextfont=cmr7
\newcount\fontcount
\newbox\charbox
\def\oddline#1{\cr
  \noalign{\nointerlineskip}
  \multispan{19}\hrulefill&
  \setbox\charbox=\hbox{\lower 2.3pt\hbox{\hex{#1x}}}
                       \smash{\box\charbox}\cr
                        \noalign{\nointerlineskip}}
\def\evenline{\cr\noalign{\hrule}}
\def\chartstrut{\lower3.5pt\vbox to14pt{}}
\def\beginchart#1{ \global\fontcount=0 #1
  \halign to\hsize\bgroup
    \chartstrut##\tabskip0pt plus10pt&
    &\hfil##\hfil&\vrule##\cr
    \lower6.5pt\null
    &&&\oct0&&\oct1&&\oct2&&\oct3&&\oct4&&\oct5&&\oct6&&\oct7&\evenline}
\def\endchart{\raise11.5pt\null&&&\hex 8&&\hex 9&&\hex A&&\hex B&
  &\hex C&&\hex D&&\hex E&&\hex F&\cr\egroup}
\def\:{\setbox\charbox=%\drawbox{
\hbox{\char\fontcount
%\llap{\vrule height.4pt width5pt depth0pt} %this will draw line at baseline
}%}%
  \ifdim\ht\charbox>7.5pt\reposition
  \else\ifdim\dp\charbox>2.5pt\reposition
  \else\ifdim\wd\charbox>0pt
       \ifdim\ht\charbox<.01pt
       \ifdim\dp\charbox<.01pt\unposition\fi\fi\fi\fi\fi
  \box\charbox\global\advance\fontcount by1 }
\def\reposition{}%\setbox\charbox=\hbox{$\vcenter{\kern2pt
%                                 \box\charbox\kern2pt}$}}
\def\unposition{\setbox\charbox=\hbox{\smalltextfont undef}}
\def\normalchart{%
  &\oct{00x}&&\:&&\:&&\:&&\:&&\:&&\:&&\:&&\:&&\oddline0
  &\oct{01x}&&\:&&\:&&\:&&\:&&\:&&\:&&\:&&\:&\evenline
  &\oct{02x}&&\:&&\:&&\:&&\:&&\:&&\:&&\:&&\:&&\oddline1
  &\oct{03x}&&\:&&\:&&\:&&\:&&\:&&\:&&\:&&\:&\evenline
  &\oct{04x}&&\:&&\:&&\:&&\:&&\:&&\:&&\:&&\:&&\oddline2
  &\oct{05x}&&\:&&\:&&\:&&\:&&\:&&\:&&\:&&\:&\evenline
  &\oct{06x}&&\:&&\:&&\:&&\:&&\:&&\:&&\:&&\:&&\oddline3
  &\oct{07x}&&\:&&\:&&\:&&\:&&\:&&\:&&\:&&\:&\evenline
  &\oct{10x}&&\:&&\:&&\:&&\:&&\:&&\:&&\:&&\:&&\oddline4
  &\oct{11x}&&\:&&\:&&\:&&\:&&\:&&\:&&\:&&\:&\evenline
  &\oct{12x}&&\:&&\:&&\:&&\:&&\:&&\:&&\:&&\:&&\oddline5
  &\oct{13x}&&\:&&\:&&\:&&\:&&\:&&\:&&\:&&\:&\evenline
  &\oct{14x}&&\:&&\:&&\:&&\:&&\:&&\:&&\:&&\:&&\oddline6
  &\oct{15x}&&\:&&\:&&\:&&\:&&\:&&\:&&\:&&\:&\evenline
  &\oct{16x}&&\:&&\:&&\:&&\:&&\:&&\:&&\:&&\:&&\oddline7
  &\oct{17x}&&\:&&\:&&\:&&\:&&\:&&\:&&\:&&\:&\evenline}
\font\chartfont=wsuipa12
{\tt WSUIPA12\bi--- WSU International Phonetic Alphabet Roman -- 12pt}
\beginchart{\chartfont}
\normalchart
\endchart
\vfill
\font\newchartfont=wslipa12
{\tt WSLIPA12\bd---WSU International Phonetic Alphabet Slanted -- 12pt}
\beginchart{\newchartfont}
\normalchart
\endchart
\np
\font\nchartfont=wbxipa12
{\tt WBXIPA12\bi---WSU International Phonetic Alphabet Bold Extended
-- 12pt}
\beginchart{\nchartfont}
\normalchart
\endchart
\vfill
\font\nwchartfont=wsuipa9
{\tt WSUIPA9\bi---WSU International Phonetic Alphabet Roman -- 9pt}
\beginchart{\nwchartfont}
\normalchart
\endchart
