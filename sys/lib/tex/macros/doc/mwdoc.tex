\documentstyle[epsf,twoside,bitmap,bitmap2]{mw}
\author{Terry L Anderson}
\title{A \LaTeX\ Style for Multiweight Design}
\markbottom{\today\quad Draft}
%\makeglossary
\def\LaTeXnine{{\rm L\kern-.36em\raise.3ex\hbox{\tiny A}\kern-.15em
    T\kern-.1667em\lower.7ex\hbox{E}\kern-.125emX}}
\begin{document}
\maketitle
\begin{copyrightpage}
\copyrightnotice{1991}{AT\&T}
  \begin{notice}
    Every effort was made to ensure that the information in this document
    was complete and accurate at the time of printing.  However,
    information is subject to change.
  \end{notice}
  \begin{security}
    This is a sample security statement.
  \end{security}
  \begin{trademarks}
    \TeX\ is a trademark of the American Mathematical Society\\
    UNIX is a registered trademark of UNIX Systems Laboratories, Inc.
  \end{trademarks}
  \begin{warranty}
    AT\&T provides a no warranty for this product.  
  \end{warranty}
  \begin{orderinginfo}
    This product is available to AT\&T employees on request to
    tla@bartok.att.com or (908) 580-4428.
  \end{orderinginfo}
  \begin{telephonesupport}
    Limited telephone support for this product is available at (908)
    580-4428.  Report error to the same number.
  \end{telephonesupport}
  \begin{emailsupport}
    Limited email support for this product is available at
    tla@bartok.att.com. Report error to the address.
  \end{emailsupport}
\end{copyrightpage}
\chapter{Introduction}
This book both documents and illustrates the \LaTeX\ style, mw, for
producing documents that follow the AT\&T Documentation Architecture
standard. 

\section{Multiweight Design} Multiweight Design, or officially {\bf
AT\&T Documentation Architecture} (AT\&T 1990), is a set of standards
for the content and format of documentation intended for ``all
postsale, external-customer documents typically delivered with or in
support of a product or service''.  The standards are optional for
presale information, internal documents (such as requirements and
design specifications) and training materials.

The standards are intended for large documents divided into chapters
and various sections with headings and a separate title page.  

\section{The Multiweight Design Style} The multiweight design style is
a \LaTeX\ style used with \TeX\ and \LaTeX\ to produce documents.  It
uses most of the same macros as {\bf tm.sty} (Anderson 1989) and so it
is possible to easily change from one style to the other to change the
printed format of a document.  There are some limitations.
Multiweight Design is a standard format for multichapter books and
mw.sty is based on the standard book.sty of \LaTeX .  Tm.sty is
intended for technical memoranda and is based on article.sty.  There
is no chapter macro in article.sty or tm.sty and so a tm.sty document
could be incorporated into a mw.sty book as a chapter with the
addition of a chapter macro.  Taking a mw.sty book and reformatting it
as a TM would take more revision.


\section{Rest of the Document}

Chapter~\ref{chap:doc} documents the features implemented in mw.sty.
Chapter~\ref{chap:example} is an example illustrating most of the
features ({\it signature elements}) in Multiweight Design and
implemented by mw.sty.  This entire book  is a further example of
Multiweight Design and illustrates most of the {\it special pages} and
other features. 

\begin{thechapterbibliography}{99}
\bibitem{bib:anderson89}Anderson, Terry L. 1989. {\it Using the
\LaTeX\ Document Style {\bf tm.sty}}.  AT\&T Bell Laboratories
Technical Memorandum, 59114-890130-01TMS.
\bibitem{bib:att1990}AT\&T. 1990. {\it AT\&T Documentation Architecture, Content and Format
Standards}.  Document number AT\&T 000-110-000 (available from
Customer Information Center, 800 432 6600).
\end{thechapterbibliography}
%
\chapter{Documentations for mw.sty\label{chap:doc}}
Mw.sty is a \LaTeX\ style file and so must be used with \TeX\ and
\LaTeX .  This chapter assumes that the reader is familiar with
\LaTeX\ and only documents the features of the mw style.  For general
information on using \TeX\ see (Knuth 1986) and on using \LaTeX\ see
(Lamport 1986).  For information about other local AT\&T styles see
(Anderson 1989).  There may also be a local user's guide for \LaTeX\
at your installation.  Ask the person in charge of your \LaTeX\
installation. 

\section{Preamble Macros}
The file should begin with 
\begin{itemize}
  \item[]
  \verb|\documentstyle[twoside]{mw}|
\end{itemize}
The twoside is optional but since the multiweight design format
specifies odd and even page offsets, it is intended for twosided
format.  Other style options that are useful include:
\begin{itemize}
  \item{\bf epsf} -- for inclusion of postscript figures.
  \item{\bf bitmap} -- for inclusion of bitmaps in figures. 
\end{itemize}

The document author is specified by
\begin{itemize}
  \item[]\verb|\author{|{\it author-name}\verb|} |
\end{itemize}

The book's title is specified by 
\begin{itemize}
  \item[]\verb|\title{|{\it book-title}\verb|}|
\end{itemize}

Optional footer material; such as, a release number, date or draft
marking can be specified using markbottom.
\begin{itemize}
  \item[]\verb|\markbottom{|{\it footer-text}\verb|| 
\end{itemize}
for example
\begin{itemize}
  \item[]\verb|\markbottom{\today\quad Draft}| 
\end{itemize}

\section{Body Macros}
A title page is created by the macro
\begin{itemize}
  \item[]\verb|\maketitle|
\end{itemize}

\subsection{Legal Information Page}

An environment \verb|copyrightpage| is available for copyright notices
and other legal statements.  It creates a page following the title
page.  The copyright notice is created by
\begin{itemize}
  \item[]\verb|\copyrightnotice{|{\it year}\verb|}{|{\it owner}\verb|}|
\end{itemize}
Owner will usually be ``AT\&T''.
The following environments are available for other legal and support
information. 
\begin{itemize}
  \item[]{\tt
        notice\\
        security\\
        tradmarks\\
        warranty\\
        orderinginfo\\
        telephonesupport\\
        emailsupport}
\end{itemize}
The Mandatory Customer Information section is not yet supported.
\subsection{Chapters and Section Headings}
Chapters are implemented with the standard \verb|\chapter| macro.
Parts are also implemented but are not part of the Multiweight Design
standard. 
\begin{note}
 For some unknown reason putting \verb|\label{}| into a chapter
        argument causes an extra line (or at least 5pt extra space)
        between chapter title and multirule.  The problem can be
        avoided by putting the label at the end of the argument.
\end{note}

Five levels of section headings are supported.  They are implemented
using the standard \LaTeX\ sectioning commands, \verb|\section|,
\verb|\subsection|, \verb|\subsubsection|, \verb|\paragraph|, and
\verb|\subparagraph|. 
\begin{note}Note that the level 5 heading,
\verb|subparagraph|, results in a run-in heading.  A terminal period
is added automatically by the macro and should NOT be supplied in the
argument.  
\end{note}

The heading titles are permitted to be up to three lines long, but
long titles tend to be unattractive and in paragraph and subparagraph
they do not stand out well.
\subsection{Lists}
Multiweight Design recognizes five types of lists

\begin{itemize}
  \item[\squarebullet]bulleted
  \item[---]dashed 
  \item[1.]numbered
  \item[a.]lettered
  \item[a string]variable
\end{itemize}
Multiweight Design permits lists to be nested only three levels deep.
Bulleted and dashed lists are implemented using the \verb|itemize|
environment.  Itemize defaults to bullets at the first nesting level,
dashes at the second and asterisks (not a multiweight design standard
list symbol) at the third.  The default symbol at any level may be
overridden by specifying the desired symbol within square brackets
in the \verb|\begin{itemize}| statement; as in,
\verb|\begin{itemize}[\spadesuit]|.  Multiweight Design discourages
symbols other than square bullets and dashes.

Autosequenced lists are implemented using the \verb|enumerate|
environment.  Enumerate defaults to arabic numerals at the first
level, lower case letters at the second level, and lower case roman
numerals at the third.  The type of numbering at any level can be
overridden by using the environments Romanenum, romanenum, Alphenum,
alphenum, numenum as a substitute for enumerate to obtain uppercase
Roman numerals, lower case Roman numberals, upper case letters, lower
case letters or arabic numerals, respectively.

Variable lists are not supported at this time.

\subsection{Figures and Tables and Screen Displays}
Figures and table floats are supported.
In order to properly position rules and captions in figures and table
according to the Multiweight Design standard, we have had to create
new figure and table macros that specify the caption as an argument
to the environment.  The new macros are
  \begin{itemize}
    \item[]
          \verb|\begin{capfigure}[htbpou]{|{\it caption}\verb|}|\\
          \verb|\end{capfigure}|\\
          \verb|\begin{captable}[htbpou]{|{\it caption}\verb|}|\\
          \verb|\end{captable}|
  \end{itemize}
The caption argument may contain \verb|\label| macros.
The contents of the figure should normally be centered, using the
centering macro or the center environment.  The capfigure and captable
macros take the same optional positioning specification as figure and
table, but with two additional options that control the placement of rules
  \begin{itemize}
    \item[o] put a rule over the float
    \item[u] put a rule under the float
  \end{itemize}
The Multiweight Design standard specifies both over and under rules
except where the figure has its own frame such as a screen display, so
`o' and `u' should normally be specified.

Where rules are not desired either capfigure and captable without the
`o' and `u' may be used or the standard figure and table environments
may be used. 

The Multiweight Design standards suggest placing screen displays in a
box with rounded corners.  This may be done using
  \begin{itemize}
    \item[]\verb|\figbox{|{\it material-to-box}\verb|}|
  \end{itemize}
  \begin{note}
    Unfortunately the \verb|\verbatim| environment cannot be used in an
    argument to this or other macros.
  \end{note}

Normally these boxed figures should be treated like other figures and
have captions, but they should not have the rules over and under the
box.  This can be done by placing the figbox inside a figure
environment or capfigure without the `o' and `u' options.  Screen
displays of only a few lines often are used only to illustrate a
discussion in the text and have not obvious title -- it seems
reasonable to omit a caption in these cases.  Often a caption would
only repeat words used in the text.

I have found such boxes improve the clarity
of other figures which contain only text; such as file or code
listings.  When several short textual figures appear on a single page,
especially without captions and with only a few lines a body text
between them, it is hard for the eye to distinguish between body text
and figure text.  This is even true when over and under rules are used
since they differ only slightly and which text is between the rules
and which is outside is not apparent.  Boxes with the vertical lines
help make the figure text stand out.  Where these are file listing it
is good practice to include the name of the file in the caption.
Captions should also be used where the contents have an appropriate
name or identity.  Small amounts of text that are used only for
illustration and have no obvious title might omit the caption.

The Multiweight Design standard specifies that screen display text be
set in Courier or similar monospaced typeface and recommends 9pt text
but no smaller than 7pt text.  An environment is provided 
  \begin{itemize}
    \item[]
          \verb|\begin{codelisting}|\\
          \verb|\end{codelisting}|
  \end{itemize}
Currently the codelisting environment simply changes to the tt font,
but later will be augmented to preserve spacing and special symbols
more like verbatim.
\subsection{Notes and Admonishments}
Notes and Admonishments use the following environments
  \begin{itemize}
    \item notes
    \item danger
    \item warning
    \item caution
  \end{itemize}
For example a note would be entered as
  \begin{itemize}
    \item[]\verb|\begin{note}|\\
          \verb|An example of a note.  This is important or explanatory|\\
          \verb|information and so we want it to stand out from the rest|\\
          \verb|of the text.|\\
          \verb|\end{note}|
  \end{itemize}
\subsection{Other Body Macros}
Footnotes are entered in the usual \LaTeX\ manner.

Many other macros from tm.sty are available.
\section{Macros for Special Pages}
Macros for 
\begin{enumerate}
  \item glossary
  \item bibliography
  \item index
  \item table of contents
  \item list of figures
  \item list of tables
\end{enumerate}
usually appear at the after all chapter bodies and while they are
optional, any that are used should be specified in
the order of the above list.

\subsection{Glossary}
A glossary uses a {\it theglossary} environment, similar to that for a
bibliography.
\begin{itemize}
  \item[]
        \verb|\begin{theglossary}|\\
        \verb|\end{theglossary}|
\end{itemize}
Individual glossary entries are similar to list items and use
\begin{itemize}
  \item[]\verb|\glossardef{|{\it word-or-phrase}|\verb|}|{\it definition-text}
\end{itemize}
Note however that it differs from list itens in that the word or
phrase to be defined goes into braces rather than brackets and is not
optional.  The multiweight design standard recommends that long
glossaries use a letter and rule to separate the glossary into
alphabetic sections.  This may be done using 
\begin{itemize}
  \item[]\verb|\alphaline{|{\it alphabetic-letter}\verb|}|
\end{itemize}
For example the following would produce an letter line, rule and a
single glossary entry.
\begin{itemize}
  \item[]
        \verb|\alphaline{A}|\\
        \verb|\glossarydef{AT\&T Practice}A type of document used to|\\
        \verb|convey maintenance, operations, and administration|\\
        \verb|information to support|
\end{itemize}

\subsection{Bibliography}
In multiweight design bibliographies can be placed at the end of each
chapter or at the end of the document.  This style supports both using
the macros
\begin{itemize}
  \item[]
        \verb|\begin{thebibliography}{99}|\\
        \verb|\begin{thechapterbibliography}{99}|
\end{itemize}
The chapter end bibliography is formatted like a section and should
appear as the last section in a chapter.  The bibliography is
formatted like a chapter and should appear after all other normal
chapters and the glossary (if present).

Individual entries use
\begin{itemize}
  \item[]
        \verb|\bibitem{|{\it tag}\verb|}|{\it bibliographic-text}
\end{itemize}
See the multiweight documentation for proper format of bibliographic
text.  In general it follows that {\it Chicago Manual of Style} format
of 
\begin{itemize}
  \item[] Author. year. {\it Title.} Publication data.
\end{itemize}

\subsection{Index}
Indexes are not currently supported.  Support will be added in a later version.

\subsection{Table of Contents and Lists of Figures and Tables}
A Table of Contents and Lists of Figures and Tables may be
automatically constructed and printed using 
\begin{itemize}
  \item[]
        \verb|\tableofcontents|\\
        \verb|\listoffigures|\\
        \verb|\listoftables|
\end{itemize}
\section{Conclusion}
For information on limitations and bugs see page~\pageref{sec:bugs}.

\begin{thechapterbibliography}{99}
\bibitem{bib:anderson89}Anderson, Terry L. 1989. {\it Using the
\LaTeX\ Document Style {\bf tm.sty}}.  AT\&T Bell Laboratories
Technical Memorandum, 59114-890130-01TMS.
\bibitem{bib:knuth}Knuth, Donald E. 1986. {\it The \TeX book}. Reading, MA:
Addison-Wesley.
\bibitem{bib:lamport}Lamport, Leslie. 1986. {\it \LaTeX : A Document
Preparation System}. Reading, MA: Addison-Wesley.
\end{thechapterbibliography}
\chapter{Examples of Multiweight Design Features\label{chap:example}}
%
\subchapter{Illustrates multiweight design signature elements and has
a subtitle long enough to take 2 lines}

This is ordinary text before a header. Often a section heading would
come first but this is here to measure the space from chapter line or
subtitle to first text.

\section{Heading Level 1}
This is an example of heading level 1. Level 1 headings are called
\verb|\section|.  It appears in 14 pt type above
a 2 pt rule.  There should be some text before a lower level heading.
Let's add more text here to see when the page break occurs. This page
has a footnote\footnote{This is an example of a footnote.  We will
make it long enough for two lines to see how the second indents.  I
don't like the large spacing between footnote symbol and the text.
Why not bring the footnote symbol in as in lists?} and we need to
check the space between the last line and the footnote rule.


\subsection{Heading Level 2}


This is an example of heading level 2.  Level 2 headings are called
\verb|\subsection|.  It appears in 11 pt type above
a 1 pt rule It is still outdented.


Let's have this section have two paragraphs to allow measuring distance 
between paragraphs and space below a regular body paragraph.


\subsubsection{Heading Level 3}


Level 3 headings are called \verb|\subsubsection|.  Level 3 heading are just
like level 2 except that they are not underlined. They are still in 11
pt type and outdented. Note that no headings are numbered. The
standards allow numbered headings but do not encourage them.


\paragraph{Heading Level 4}

Level 4 headings are called \verb|\paragraph|.  Level 4 headings are
not outdented. They are still in 11 pt type. The standard allows for
five levels of headers.


\subparagraph{Heading Level 5}Level 5 headings are called
\verb|\subparagraph|.  Heading level 5 is a run-in header.  Fortunately
the distinction between stand-alone and run-in headers is built into
\LaTeX\ so this is easy to implement.  These run-in headers look ok when
they are short but I don't like them when the title is long.


\section{Level 1 Heading that is Long Enough 
to Need Two Lines even in the table of contents.}


Let us put in a little text.  We need more text.  The text should be
longer than the title.  But I really have little to say.


\subsection{This is a Long Level 2 Heading to See Where it 
Breaks for the Second Line}


Let us put in a little text.  We need more text.  The text should be
longer than the title.  But I really have little to say.



\subsubsection{This is a Long Level 3 Heading to See Where it 
Breaks for the Second Line}


Let us put in a little text.  We need more text.  The text should be
longer than the title.  But I really have little to say.



\paragraph{This is a Long Level 4 Heading to See Where it Breaks for
the Second Line}

Level 4 headings are allowed to extend the entire width of the text
column.  Level 4 headings do not stand out as well with no underlining
or outdenting --- only bolding.  This is especially a problem when the
heading is more than a single line.


\subparagraph{This is a level 5 heading that is very long to see how
they look when they are very long such as more than a full line} This
is the body for a paragraph with a very long run-in header.  This
needs to be long enough that the paragraph is larger than the header.
I do not like the looks of this method of titling when the title is long.


This page looks very bad with so many headings, but of course using such long 
headings and so little text between them is very artificial.


\section{Figures and Tables}
\subsection{Figures}
We can include figures in pic, postscript and xbitmaps. 
Figure~\ref{fig:face} is an example of a figure whose source is an
xbitmap (included using \verb|\bitmap| which prints bitmaps converted
to \TeX\ by the filter bitmap2tex).

\begin{capfigure}{\label{fig:face}The Author's face}[htbpou]
\vskip.5in
\centerline{
\input tlaface
}\vskip.4in
\end{capfigure}

Figure~\ref{fig:tardis} is an example of a figure whose source is an
xbitmap (included using \verb|\bitmap2| which directly parses xbitmap files).
\begin{capfigure}{\label{fig:tardis}The Tardis}[htbpou]
\vskip.5in
\begin{center}
\Bitmap{tlaface.map}{1pt}
\end{center}
\vskip0.2in
\end{capfigure}

Figure~\ref{fig:attlogo} is an example of encapsulated PostScript.
\begin{capfigure}{\label{fig:attlogo}The AT\&T Logo}[htbpou]
\center{
%\special{psfile="/tools/sde/frame.sde/lib/EPSI/attlogo.epsf"}
%\epsfbox{/tools/sde/frame.sde/lib/EPSI/attlogo.epsf}
}
\end{capfigure}
Including Encapsulated PostScript requires using dvips or other dvi
supporting postscript specials.  Dvips supports  \verb|\special| with
the designation of a PostScript file and a number of options.  If
epsf.sty is available and the epsf style option is specified in the
documentstyle, one may also use \verb|\epsfbox{|{\it
psfilename}\verb|}| can be used to include an Encapsulated PostScript
figure with the space required automatically calculated.

Figure~\ref{fig:s} on page~\pageref{fig:s} is another example of encapsulated
PostScript.  It consists of two plots made by S (Becker, Chambers and
Wilks 1988).
\begin{capfigure}{\label{fig:s}Sample output of S (shown at 64\%
  reduction)}[htbpou] 
\center{
%\epsfbox{S.epsf}
}
\end{capfigure}


\subsection{Tables}
Tables are labelled at the top rather than at the bottom. 

Multiweight Design specifies text to be used when tables are not compete on a 
single page and must be continued onto a second page. I do not see how to 
enforce this in \LaTeX, although I can provide macros types
that include the text for continuing.


Table~\ref{tab:example} on page~\pageref{tab:example} is an example of
a small (regular width) table.

\begin{captable}{\label{tab:example}Example Table}[htbpou]
\begin{tabbing}
Lotus 1-2-3\quad\=Software BackPlane\quad\=\kill
{\bf Product}\>{\bf Description}\>{\bf Kind of Link}\\
20/20\>	a spreadsheet\>	outputs MIF\\
Lotus 1-2-3\>	a spreadsheet\>	outputs MIF\\
Teamwork\>	CASE\>	outputs MIF\\
StP\>	CASE\>	outputs MIF\\
Atherton\>	Software BackPlane\>	live-links? version control of documents\\
MML filter\>	ascii editing\>	allows ascii editing of Frame documents\\
\end{tabbing}
\end{captable}

This table was done with the tabbing environment.  We can also make
tables using tabular environment and box the items.

\subsection{Screen Displays and File Listings}
Multiweight Design recommends the use of rounded boxes to set off
screen displays.  It is also a nice for file listings.  For very small
figures such as those illustrating terminal interaction, figures with
captions seem to be overkill.  Figures without captions might be used
but while the above and
below lines are suitable for setting off non-text figures, they
are still somewhat ambiguous for text figures.  When
there are several on a page, with the text between about the same as
the text within them it is not clear which is figure and which is
paragraph text.  Placing them in rounded boxes works fine since the
vertical sides of the box removes the ambiguity.  Figure captions
without additional rules can be added when useful.

The following is an example of a command line.
\begin{figure}[h]
\figbox{{\tt{\bf\$} mycommand arg1 arg2}}
\end{figure}

Figure~\ref{fig:mpsprint} on page~\pageref{fig:mpsprint} is an example of
a multiple line file listing. 

\begin{figure}[htbp]
\figbox{
\begin{codelisting}
if test -r "\$1.ps"; then INF="\$1.ps"\\
   else if test -r "\$1.PS"; then INF="\$1.PS"\\
   else if test -r "\$1"; then INF="\$1"\\
   else echo "No \$1, \$1.ps or \$1.PS found";exit\\
fi fi fi\\
cat \$INF | rsh mozart lp -d4NE10P -oraw
\end{codelisting}
}
\caption{\label{fig:mpsprint}Listing of the mpsprint command}
\end{figure}

\section{Lists}


Multiweight Design recognizes five types of lists

\begin{itemize}
  \item[\squarebullet]bulleted
  \item[---]dashed< 
  \item[1.]numbered
  \item[a.]lettered
  \item[a string]variable
\end{itemize}

The line following a list should be 18 pt baseline to baseline if it
is part of the paragraph or 24 pt if it begins a new paragraph. 


The standard says that the period is the same distance form the text column as 
the left edge of box and dash but this looks too far out. The examples
in the document do not follow this. So I have followed the example
which puts the period at about 2 picas from text column.


The variable list is not yet implemented.


Nested lists are supported to four levels though the guidelines
suggests a limit of three.


\subsection{Bullet Lists}

This is an example of a bullet list.
\begin{itemize}
  \item This is level 1
  \item So is this, but lets make the item long enough for a second
        line to check the indent. 
  \begin{itemize}
    \item This is level 2
    \item This is too.  Lets make the item long enough for a second
          line to check the indent. 

          This is a second paragraph of the same item.
    \begin{itemize}
      \item This is level 3 Lets make the item long enough for a
            second line to check the indent.
      \item this is too Lets make the item long enough for a second
            line  to check the indent.

            This is a second paragraph of a level 3 item
    \end{itemize}
    \item This is level 2 again
    \begin{itemize}
      \item Again
    \end{itemize}
  \end{itemize}
  \item This is again level 1
\end{itemize}



\subsection{Numbered Lists}


This is an sample of nested numbered lists.
\begin{numenum}
  \item This is level 1
  \item So is this, but lets make the item long enough for a second
        line to check the indent. 
  \begin{numenum}
    \item This is level 2
    \item This is too.  Lets make the item long enough for a second
          line to check the indent. 

          This is a second paragraph of the same item.
    \begin{numenum}
      \item This is level 3 Lets make the item long enough for a
            second line to check the indent.
      \item this is too Lets make the item long enough for a second
            line  to check the indent.

            This is a second paragraph of a level 3 item
    \end{numenum}
    \item This is level 2 again
    \begin{numenum}
      \item Again
    \end{numenum}
  \end{numenum}
  \item This is again level 1
\end{numenum}

\subsection{Enumerated Lists}


This is an sample of nested enumerated lists, where the enumeration
type is changed automatically.
\begin{enumerate}
  \item This is level 1
  \item So is this, but lets make the item long enough for a second
        line to check the indent. 
  \begin{enumerate}
    \item This is level 2
    \item This is too.  Lets make the item long enough for a second
          line to check the indent. 

          This is a second paragraph of the same item.
    \begin{enumerate}
      \item This is level 3 Lets make the item long enough for a
            second line to check the indent.
      \item this is too Lets make the item long enough for a second
            line  to check the indent.

            This is a second paragraph of a level 3 item
    \end{enumerate}
    \item This is level 2 again
    \begin{enumerate}
      \item Again
    \end{enumerate}
  \end{enumerate}
  \item This is again level 1
\end{enumerate}

\subsection{Lettered Lists}


This is an sample of nested lettered lists.

\begin{alphenum}
\typeout{\theenumi}
  \item This is level 1
  \item So is this, but lets make the item long enough for a second
        line to check the indent. 
  \begin{alphenum}
    \item This is level 2
    \item This is too.  Lets make the item long enough for a second
          line to check the indent. 

          This is a second paragraph of the same item.
    \begin{alphenum}
      \item This is level 3 Lets make the item long enough for a
            second line to check the indent.
      \item this is too Lets make the item long enough for a second
            line  to check the indent.

            This is a second paragraph of a level 3 item
    \end{alphenum}
    \item This is level 2 again
    \begin{alphenum}
      \item Again
    \end{alphenum}
  \end{alphenum}
  \item This is again level 1
\end{alphenum}

\subsection{Mixed Lists}


Of course in practice one should not have nested lists of one kind but
nest dash  lists inside bullet lists and nest lettered lists inside
numbered lists as recommended in the guidelines. Here is an example

\begin{enumerate}
  \item This is level 1
  \item So is this, but lets make the item long enough for a second
        line to check the indent. 
  \begin{alphenum}
    \item This is level 2
    \item This is too.  Lets make the item long enough for a second
          line to check the indent. 

          This is a second paragraph of the same item.
    \begin{itemize}
      \item This is level 3 Lets make the item long enough for a
            second line to check the indent.
      \item this is too Lets make the item long enough for a second
            line  to check the indent.

            This is a second paragraph of a level 3 item
    \end{itemize}
    \item This is level 2 again
    \begin{itemize}
      \item Again
    \end{itemize}
  \end{alphenum}
  \item This is again level 1
\end{enumerate}
\section{Admonishments or Product Safety Labels}
There is a special notation for ``Admonishments'' or product safety
labels.  Admonishments 
(danger, warning, and caution statements) tell customers that the
actions they are about to perform may harm them or the equipment.  The
Multiweight Design documentation defines the three types of
admonishments and gives standards for their use.  Any admonishment
that appears on the product must also appear verbatim in the
documentation.  
\begin{danger}
This is an example of a {\it danger} admonishment. Danger indicates
the presence of a hazard that {\it will} cause death or severe
personal injury if the hazard is not avoided.
\end{danger}
\begin{warning}
This is an example of a {\it warning} admonishment.  Warning indicates
the presence of a hazard that {\it can} cause death or severe personal
injury if the hazard is not avoided.
\end{warning}
\begin{note}
Do not use danger or warning for property-damage accidents or
service-interruption accidents unless personal injury risk appropriate
to the level is also involved.
\end{note}
\begin{caution}
This is an example of a {\it caution} admonishment.  Caution indicates the
presence of a hazard that {\it will} or {\it can} cause minor personal
injury or property damage if the hazard is not avoided.
\end{caution}

\section{Notes }

There is also a special notation for ``Notes''. 

Notes
are used for ``important or explanatory information that stands out
from the rest of the text. If there is a notice label or an
instruction label on the product, repeat this information as a note in
the document.''


\begin{note}An example of a note.  This is important or explanatory
  information and so we want it to stand out fron the rest of the text.

Notes will usually be only a single paragraph but the standards to not
state that a second paragraph is not permitted.  Second paragraphs
should not be a distinct note, however.
\end{note}
\begin{note}
This is another note.  A second distinct note on the same or later
page must repeat the work 
NOTE and the icon if used, even if immediately adjacent to another
note; i.e., two notes may not be complined into a single note by using
multiple paragraphs.
\end{note}


\section{Quotations}


There are two methods of indicating quoted material. For short
quotations, usually of two lines or less, the text is placed in line
in the paragraph and set off by double quote marks. For longer
quotations, it is set in a separate paragraph with extra indenting.
The quote below is from the {\it Chicago Manual of Style} (University of Chicago Press 1982).

\begin{quote}
Quotations may be incorporated in the text in two ways: (1) run in,
that is, in the same type size as the text and enclosed in quotation
marks; or (2) set off from the text, without quotation marks.
Quotations of the latter sort may be set in smaller type, or with all
lines indented from the left, or with unjustified lines (if text lines
are justified), or with less space between lines than the text --- or
some combination of these typographical devices may be specified by
the book designer. Quotations set off from the text are called block
quotations, extracts or excerpts.
\end{quote}

This block form of quotation will be in the next release of the
standard.  It is implemented by using the \LaTeX\ {\it quote} environment.

\section{Using Bibliographies}

The following is an example of a bibliography placed at the end of a
chapter. Multiweight Design permits them to be placed there or in a
separate Bibliography section. The data must appear in the order:
author. publication date. title. location of publication., with each
item separated by a period and a space. For more details on the style,
Multiweight Design relies on the {\it Chicago Manual of Style}
(University of Chicago Press 1982).


Bibliographic citations in the text follow the style of ``(author
year)''; for example, (Anderson 1991). If the bibliography contains
more than one reference by the same author in a given year a
sequential lower case letter is appended to the year; for example,
(Anderson 1991a). The alphabetic letter is also used in the
publication date entry in the bibliography.

Remember to use italic font ("emphasis" in the C
catalog) for a book title or journal title but a journal article
title is in standard font enclosed in quotes.

\begin{thechapterbibliography}{99}

\bibitem{bib:att1990}AT\&T. 1990. {\it AT\&T Documentation Architecture, Content and Format
Standards}.  Document number AT\&T 000-110-000 (available from
Customer Information Center, 800 432 6600).

\bibitem{bib:becker}Becker, Richard A., John M.\ Chambers, and Allan
R.\ Wilks. 1988. {\it The New S Language}. Pacific Grove, CA: Wadsorth
\& Brooks/Cole Advanced Books \& Software.

\bibitem{bib:smith1991}Smith, John. 1991a. ``A Journal Article.'' {\it Journal Name}(Month):34-45.

\bibitem{bib:smith1988b}Smith, John. 1991b. {\it A Book by the Same Author}. New York:Some Publisher.

\bibitem{bib:chicago82}University of Chicago Press. 1982. {\it The
Chicago Manual of Style}. Chicago:University of Chicago Press.
\end{thechapterbibliography}
\chapter{Conclusion and a Long Chapter Title to See How Long Ones are Handled}

\section{Limitations and Bugs\label{sec:bugs}}


The style is not complete nor are all features implemented in a fully
standard manner. These will be removed in later releases. The
following are some of the known limitations and bugs.


\begin{enumerate}
  \item Chapter titles are in 17 pt type rather than 16 since 16 is not
        available. 
  \item Chapter numbers is in 25 pt type rather than 115 since 115 is
        not available.
  \item For some unknown reason putting \verb|\label{}| into a chapter
        argument causes an extra line (or at least 5pt extra space)
        between chapter title and multirule.  The problem can be
        avoided by putting the label at the end of the argument.
  \item Table of Contents and Lists of Figures and Tables do not have
        the correct heading on 2$^{nd}$ and subsequent pages.
\end{enumerate}
\section{Future}
It is our intention to complete the implementation of all standard
features for 8.5 by 11 inch, one-column pages, including
\begin{itemize}
  \item indexes
  \item correct size fonts (including 115 pt)
  \item copyright page
  \item proprietary notice at bottom of pages
  \item mandatory customer information section for legal page
\end{itemize}

At some later time support for 8.5 by 11 inch, two-column pages may be
added.  
\begin{theglossary}

\alphaline{A}
\glossarydef{AT\&T Practice}A type of document used to convey
maintenance, operations, and administration information to support
personnel.
\glossarydef{addenda sheets}A method of revising permanently bound
documents by using update sheets that list new or revised information.
See also errata sheets and page updates.
\alphaline{B}
\glossarydef{back matter}All parts of a document after the body text.
Back matter includes the appendixes, glossary, and index.
\glossarydef{baseline}An imaginary line on which all capital letters
on a document page rest.  Descenders extend beyond the baseline.  See
also descender.
\alphaline{D}
\glossarydef{descender}The part of certain letters, such as, g, p, q,
and y  that extend below the baseline.  
\alphaline{L}
\glossarydef{\LaTeXnine}A macro package for \TeX\ that simplifies
preparation of documents that have a common format and style.  Most
items are {\it tagged} by their function in the document (for example,
section, enumerated list, or table) with the formatting for that
function specified separately.  This keeps format and content separate
in the spirit of SGML.  (See also SGML and \TeX .)
\alphaline{M}
\glossarydef{multiweight design}Multiweight Design, or officially {\bf
AT\&T Documentation Architecture} (AT\&T 1990), is a set of standards
for the content and format of documentation intended for ``all
postsale, external-customer documents typically delivered with or in
support of a product or service''.  The standards are optional for
presale information, internal documents (such as requirements and
design specifications) and training materials.
\alphaline{S}
\glossarydef{SGML}Standard Generalized Markup Language --- a set of
standards for defining document markup tags, allow the specification
of document items by function rather than format, and methods for
defining the format to associate with tagged items.
\alphaline{T}
\glossarydef{\TeX}A page layout, document formatting, or typesetting
language; somewhat like troff but with a more regular syntax, better
control over scoping and more high-level features.  (See also \LaTeXnine ).
\end{theglossary}
\begin{thebibliography}{99}

\bibitem{bib:att1990}AT\&T. 1990. {\it AT\&T Documentation Architecture, Content and Format
Standards}.  Document number AT\&T 000-110-000 (available from
Customer Information Center, 800 432 6600).

\bibitem{bib:bangs1988a}Bangs, Alex J. 1988a. {\it The Motion Interpreter/Control Computer
System}. Harvard University.


\bibitem{bib:bangs1988b}Bangs, Alex J. 1988b. {\it Another Book by the Same Author}. New York:
Some Publisher.


\bibitem{bib:mwbarnwell}Barnwell, T. P., R.C. Rose, S. McGrath. {\it A Real-Time
Implementation of a 4600 BPS Self-Excited Vocoder Using the AT\&T
WE\regmark DSP32 Signal Processing Microprocessor}.  Georgia Institute
of Technology


\bibitem{bib:smith1991}Smith, John. 1991. ``A Journal Article.'' {\it Journal Name}(Month):34-45.

\bibitem{bib:att1990}AT\&T. 1990. {\it AT\&T Documentation Architecture, Content and Format
Standards}.  Document number AT\&T 000-110-000 (available from
Customer Information Center, 800 432 6600).

\bibitem{bib:bangs1988a}Bangs, Alex J. 1988a. {\it The Motion Interpreter/Control Computer
System}. Harvard University.


\bibitem{bib:bangs1988b}Bangs, Alex J. 1988b. {\it Another Book by the Same Author}. New York:
Some Publisher.


\bibitem{bib:mwbarnwell}Barnwell, T. P., R.C. Rose, S. McGrath. {\it A Real-Time
Implementation of a 4600 BPS Self-Excited Vocoder Using the AT\&T
WE\regmark DSP32 Signal Processing Microprocessor}.  Georgia Institute
of Technology


\bibitem{bib:smith1991}Smith, John. 1991. ``A Journal Article.'' {\it Journal Name}(Month):34-45.

\bibitem{bib:att1990}AT\&T. 1990. {\it AT\&T Documentation Architecture, Content and Format
Standards}.  Document number AT\&T 000-110-000 (available from
Customer Information Center, 800 432 6600).

\bibitem{bib:bangs1988a}Bangs, Alex J. 1988a. {\it The Motion Interpreter/Control Computer
System}. Harvard University.


\bibitem{bib:bangs1988b}Bangs, Alex J. 1988b. {\it Another Book by the Same Author}. New York:
Some Publisher.


\bibitem{bib:mwbarnwell}Barnwell, T. P., R.C. Rose, S. McGrath. {\it A Real-Time
Implementation of a 4600 BPS Self-Excited Vocoder Using the AT\&T
WE\regmark DSP32 Signal Processing Microprocessor}.  Georgia Institute
of Technology


\bibitem{bib:smith1991}Smith, John. 1991. ``A Journal Article.'' {\it Journal Name}(Month):34-45.


\bibitem{bib:att1990}AT\&T. 1990. {\it AT\&T Documentation Architecture, Content and Format
Standards}.  Document number AT\&T 000-110-000 (available from
Customer Information Center, 800 432 6600).

\bibitem{bib:bangs1988a}Bangs, Alex J. 1988a. {\it The Motion Interpreter/Control Computer
System}. Harvard University.


\bibitem{bib:bangs1988b}Bangs, Alex J. 1988b. {\it Another Book by the Same Author}. New York:
Some Publisher.


\bibitem{bib:mwbarnwell}Barnwell, T. P., R.C. Rose, S. McGrath. {\it A Real-Time
Implementation of a 4600 BPS Self-Excited Vocoder Using the AT\&T
WE\regmark DSP32 Signal Processing Microprocessor}.  Georgia Institute
of Technology


\bibitem{bib:smith1991}Smith, John. 1991. ``A Journal Article.'' {\it Journal Name}(Month):34-45.

\bibitem{bib:att1990}AT\&T. 1990. {\it AT\&T Documentation Architecture, Content and Format
Standards}.  Document number AT\&T 000-110-000 (available from
Customer Information Center, 800 432 6600).

\bibitem{bib:bangs1988a}Bangs, Alex J. 1988a. {\it The Motion Interpreter/Control Computer
System}. Harvard University.


\bibitem{bib:bangs1988b}Bangs, Alex J. 1988b. {\it Another Book by the Same Author}. New York:
Some Publisher.


\bibitem{bib:mwbarnwell}Barnwell, T. P., R.C. Rose, S. McGrath. {\it A Real-Time
Implementation of a 4600 BPS Self-Excited Vocoder Using the AT\&T
WE\regmark DSP32 Signal Processing Microprocessor}.  Georgia Institute
of Technology


\bibitem{bib:smith1991}Smith, John. 1991. ``A Journal Article.'' {\it Journal Name}(Month):34-45.

\bibitem{bib:att1990}AT\&T. 1990. {\it AT\&T Documentation Architecture, Content and Format
Standards}.  Document number AT\&T 000-110-000 (available from
Customer Information Center, 800 432 6600).

\bibitem{bib:bangs1988a}Bangs, Alex J. 1988a. {\it The Motion Interpreter/Control Computer
System}. Harvard University.


\bibitem{bib:bangs1988b}Bangs, Alex J. 1988b. {\it Another Book by the Same Author}. New York:
Some Publisher.


\bibitem{bib:mwbarnwell}Barnwell, T. P., R.C. Rose, S. McGrath. {\it A Real-Time
Implementation of a 4600 BPS Self-Excited Vocoder Using the AT\&T
WE\regmark DSP32 Signal Processing Microprocessor}.  Georgia Institute
of Technology


\bibitem{bib:smith1991}Smith, John. 1991. ``A Journal Article.'' {\it Journal Name}(Month):34-45.
\end{thebibliography}
\tableofcontents
\listoffigures
\listoftables
\end{document}