% minitoc.tex 24
\documentclass[12pt,a4paper]{book} 
\usepackage{minitoc}
\makeatletter % to include some useful commands
\newcommand{\BibTeX}{{\rm B\kern-.05em{\sc i\kern-.025em b}\kern-.08em
    T\kern-.1667em\lower.7ex\hbox{E}\kern-.125emX}}
\let\BiBTeX\BibTeX
% IF YOU NOT HAVE THE manfnt FONT,
% REPLACE THE NEXT LINES  BY
% \let\Virage\relax \let\virage\relax
\newfont{\manfnt}{manfnt}
\newcommand\virage{\marginpar[\raggedleft
 {{\manfnt\symbol{'177}}~~}]{\raggedright {~~\manfnt\symbol{'177}}}}
\def\@Vir{\hbox to12mm{\hbox{}\leaders%
 \hbox{{\manfnt\char127}\kern2pt}\hfil\hbox{}}}
\newcommand\Virage{\marginpar[\hfill{\@Vir~~}]{{~~\@Vir}\hfill}}
%%%%%%%%%%%%%%%%% END OF REPLACE
% 
\def\Cat#1{\hbox{$\langle${\it #1\/}$\rangle$}}
\def\<#1>{\Cat{#1}}
%%%
\makeatother
\parskip=12pt
\def\bs{{\tt\char'134}}
\setcounter{minitocdepth}{5}
\renewcommand{\textfraction}{.0}
\renewcommand{\topfraction}{.9999}
\renewcommand{\bottomfraction}{.9999}
\renewcommand{\floatpagefraction}{.9999}
\setcounter{totalnumber}{10}
\setcounter{topnumber}{10}
\setcounter{bottomnumber}{10}
\begin{document}
\dominitoc
%\dominilof
\dominilot
\faketableofcontents
%\fakelistoffigures
\fakelistoftables
\chapter{The {\tt minitoc} package}\label{o+minitoc}
\pagestyle{headings}
\minitoc
%\minilof
\bigskip
\minilot
\section{Introduction}
This package, initially written by Nigel Ward and Dan Jurafsky, has been
almost completely redesigned at CERT/ONERA by Jean-Pierre Drucbert.
It creates a mini-table of contents
(a ``minitoc''\,\footnote{The \texttt{minitoc} package introduces
                its own jargon, explained in this note.
                It should not be too difficult, however, to learn and use.}%
) at the beginning of each chapter of the document. 
It is also possible to have a mini-list of
figures (a ``minilof'') and a mini-list of tables (a ``minilot'').
The document class should of course define chapters 
(styles like {\tt book} or {\tt report}) or sections
(styles like {\tt article}). 
Thus, this package should not be used with
document classes without sectioning commands (like {\tt letter}).
When the document class defines a ``part'' sectionning level 
(i.e. classes like {\tt book}, {\tt report} and {\tt article}), you can
create a ``partial'' table of contents
(a ``parttoc'') at the beginning of each part of the document.
It is also possible to have a partial list of figures (a ``partlof'')
and a partial list of tables (a ``partlot''). 
When the document class has no {\tt\bs chapter} command 
but has a {\tt\bs section} command, you may use section
level tables of contents (``secttoc'') at the beginning of each section.
{\bf Note}\virage: you cannot
use chapter level and section level table of contents in the same document.
This restriction is intented to avoid documents full of local
tables of contents, list of figures and tables at every sectionning level.

The current version of this package is \#24.

\noindent
{\bf Note}\virage: the commands relative
to the part level are defined only if the document class defines \verb|\part|.
The commands relative
to the section level are defined only if the document class
does not define \verb|\chapter|.

\section{Usage}

To use the {\tt minitoc} package, you must introduce a command
\begin{verbatim}
\usepackage{minitoc}
\end{verbatim}
in the preamble of your document.
The mini-table of contents will appear in the chapter,
after the \verb|\chapter| command, at the point of the \verb|\minitoc|
command. The {\tt\bs minitoc} command may occur
         \emph{anywhere} inside a chapter. Of course, it is better
         to put it at the beginning of the chapter, eventually
         after some introductory material. But you can also decide
         to put it at the end of the chapter. You should use the
         same conventions in all chapters.
If you want to add the mini-table of contents for a chapter,
you must use the sequence given in Table~\ref{mtc+minitoc}
\begin{table*}[ht]\tt
\begin{center}
\fbox{\begin{tabular}{ll}
\multicolumn{2}{l}{\bs documentclass[...]\{book\}}\\
\multicolumn{2}{l}{\bs usepackage\{minitoc\}}\\
\ldots& \\
\bs setcounter\{minitocdepth\}\{2\} & \emph{default} \\
\bs setlength\{\bs mtcindent\}\{24pt\} & \emph{default} \\
\bs renewcommand\{\bs mtcfont\}\{\bs small\bs rm\} & \emph{default} \\
\bs renewcommand\{\bs mtcSfont\}\{\bs small\bs bf\} & \emph{default} \\
\ldots& \\
\bs begin\{document\}& \\
\ldots& \\
\bs dominitoc& \\
\bs dominilof& \\
\bs dominilot& \\
\bs tableofcontents & \emph{or} \bs faketableofcontents \\
\bs listoffigures & \emph{or} \bs fakelistoffigures \\
\bs listoftables & \emph{or} \bs fakelistoftables \\
\ldots& \\
\bs chapter\{...\}& \\
\bs minitoc & \emph{if you want one} \\
\bs minilof & \emph{if you want one} \\
\bs minilot & \emph{if you want one} \\
\ldots& \\
\end{tabular}}
\end{center}
\caption{Commands for a {\tt minitoc}}\label{mtc+minitoc}
\end{table*}
For each mini-table of contents, an auxiliary file will be
created with a name of the form \<document>\verb|.mtc|\<N>,
where \<N> is the absolute chapter number.
``Absolute'' means that this number is unique,
and increasing from the first chapter.
The suffix is \verb|.mlf|\<N> for mini-lists of figures and
is \verb|.mlt|\<N> for mini-lists of tables.
(If you are under MS-DOS or any operating system with short
extensions to filenames, see Section~\ref{MS-DOS}).

The section-level table of contents will appear in the section,
after the \verb|\section| command, at the point of the \verb|secttoc|
command. The {\tt\bs secttoc} command may occur
         \emph{anywhere} inside a section. Of course, it is better
         to put it at the beginning of the section, eventually
         after some introductory material. You should use the
         same conventions in all sections.
If you want to add the section-level table of contents for a section,
you must use the sequence given in Table~\ref{mtcsecttoc}
\begin{table*}[ht]\tt
\begin{center}
\fbox{\begin{tabular}{ll}
\multicolumn{2}{l}{\bs documentclass[...]\{article\}}\\
\multicolumn{2}{l}{\bs usepackage\{minitoc\}}\\
\ldots& \\
\bs setcounter\{\bs secttocdepth\}\{2\} & \emph{default} \\
\bs setlength\{\bs stcindent\}\{24pt\} & \emph{default} \\
\bs renewcommand\{\bs stcfont\}\{\bs small\bs rm\} & \emph{default} \\
\bs renewcommand\{\bs stcSSfont\}\{\bs small\bs bf\} & \emph{default} \\
\ldots& \\
\bs begin\{document\}& \\
\ldots& \\
\bs dosecttoc& \\
\bs dosectlof& \\
\bs dosectlot& \\
\bs tableofcontents & \emph{or} \bs faketableofcontents \\
\bs listoffigures & \emph{or} \bs fakelistoffigures \\
\bs listoftables & \emph{or} \bs fakelistoftables \\
\ldots& \\
\bs chapter\{...\}& \\
\bs secttoc & \emph{if you want one} \\
\bs sectlof & \emph{if you want one} \\
\bs sectlot & \emph{if you want one} \\
\ldots& \\
\end{tabular}}
\end{center}
\caption{Commands for a {\tt secttoc}}\label{mtcsecttoc}
\end{table*}
For each section-level table of contents, an auxiliary file will be
created with a name of the form \<document>\verb|.stc|\<N>,
where \<N> is the absolute section number.
The suffix is \verb|.slf|\<N> for section-level lists of figures and
is \verb|.slt|\<N> for section-level lists of tables.
(If you are under MS-DOS or any operating system with short
extensions to filenames, see Section~\ref{MS-DOS}).

If you want to add the partial table of contents for a part,
you must use the sequence given in Table~\ref{mtc+parttoc}.
\begin{table*}[ht]\tt
\begin{center}
\fbox{\begin{tabular}{ll}
\multicolumn{2}{l}{\bs documentclass[...]\{book\}}\\
\multicolumn{2}{l}{\bs usepackage\{minitoc\}}\\
\ldots& \\
\bs setcounter\{parttocdepth\}\{2\} & \emph{default} \\
\bs setlength\{\bs ptcindent\}\{0pt\} & \emph{default} \\
\bs renewcommand\{\bs ptcfont\}\{\bs normalsize\bs rm\} & \emph{default} \\
\bs renewcommand\{\bs ptcCfont\}\{\bs normalsize\bs bf\} & \emph{default} \\
\bs renewcommand\{\bs ptcSfont\}\{\bs normalsize\bs rm\} & \emph{default} \\
\ldots& \\
\bs begin\{document\}& \\
\ldots& \\
\bs doparttoc& \\
\bs dopartlof& \\
\bs dopartlot& \\
\bs tableofcontents & \emph{or} \bs faketableofcontents \\
\bs listoffigures & \emph{or} \bs fakelistoffigures \\
\bs listoftables & \emph{or} \bs fakelistoftables \\
\ldots& \\
\bs part\{...\}& \\
\bs parttoc & \emph{if you want one} \\
\bs partlof & \emph{if you want one} \\
\bs partlot & \emph{if you want one} \\
\ldots& \\
\end{tabular}}
\end{center}
\caption{Commands for a {\tt parttoc}}\label{mtc+parttoc}
\end{table*}
For each partial table of contents, an auxiliary file will be
created with a name of the form \<document>\verb|.ptc|\<N>,
where \<N> is the part number.
The suffix is \verb|.plf|\<N> for partial lists of figures and
is \verb|.plt|\<N> for partial lists of tables.
(If you are under MS-DOS or any operating system with short
extensions to filenames, see Section~\ref{MS-DOS}).

\noindent
{\bf Note}\virage: the user is responsible of requiring or not requiring
a mini-toc (lof or lot) for some chapter. Asking a minilof for a
chapter without any figure will result in an empty and ugly mini
list of figures (i.e. the title and two horizontal rules).
He is also responsible of requiring or not requiring
a partial toc (lof or lot) for some part. Asking a partlof for a
part without any figure will result in an empty and ugly part
list of figures (i.e. the title alone on a page). Analogous remarks
apply to section-level tables of contents (secttoc, sectlof and sectlot).

By default, the mini-tables and partial tables of contents
contain only references to sections and
subsections. The {\tt minitocdepth} and {\tt parttocdepth}
counters, similar to {\tt tocdepth},
allows the user to modify this behaviour. Mini or partial
lists of figures or tables are not affected by the value of
these counters.

\noindent
{\bf Note}\virage: if you are using \verb|\chapter*| and a
\begin{quote}
\verb|\addcontentsline{toc}{chapter}{...}|
\end{quote}
command to add something in the
table of contents, the numbering of minitoc files would be altered.
To avoid that problem, say
\begin{quote}
\verb|\addcontentsline{toc}{starchapter}{...}|
\end{quote}
You \emph{should not} use \verb|\minitoc| in a \verb|\chapter*|.
Similar remarks apply to the part and section level.

\subsection{Fonts and Titles}

The mini and partial tables and lists are typeset
in a {\tt verse}-like environment, and
can be split over pages.
The mini-table of contents
is typeset in the {\tt\bs mtcfont}
font, which is \verb|\small\rm| by default. Section entries are typeset
in the {\tt\bs mtcSfont} font, which is \verb|\small\bf| by default.
For subsections, subsubsections, paragraphs and subparagraphs, the
commands \verb|\mtcSSfont|, \verb|\mtcSSSfont|, \verb|\mtcPfont| and
\verb|\mtcSPfont| are available (by default, \verb|\small\rm|)
to enable the use of various fonts. Mini lists of figures and tables are
typeset in the fonts \verb|\mlffont| and \verb|\mltfont|, which are
\verb|\small\rm| by default.

Titles are typeset in the {\tt\bs mtifont} (\verb|\large\bf| by
default) font and the texts of the titles are defined by
{\tt\bs mtctitle}, {\tt\bs mlftitle} and {\tt\bs mlttitle}, which
are the strings ``Contents'', ``Figures'' and ``Tables'' by default.
These commands should be redefined by {\tt\bs renewcommand} for
languages other than english. The option files
{\tt mtcfrench.sty} and {\tt mtcenglish.sty}
(and others\,\footnote{The strings defined in
             these option files were taken from the {\bf Babel}
             system of Johannes Braams and some were
             adapted. Other languages are welcome.\label{fo+lang}}%
) are available.
You can easily prepare a similar style for your preferred language.

The partial table of contents
is typeset in the {\tt\bs ptcfont}
font, which is defined as \verb|\normalsize\rm| by default.
Chapter entries are typeset in the {\tt\bs ptcCfont} font,
which is \verb|\normalsize\bf| by default.
Section entries are typeset in the {\tt\bs ptcSfont} font,
which is \verb|\normalsize\rm| by default.
For subsections, subsubsections, paragraphs and subparagraphs, the
commands \verb|\ptcSSfont|, \verb|\ptcSSSfont|, \verb|\ptcPfont| and
\verb|\ptcSPfont| are available (by default, \verb|\normalsize\rm|) if you
want to use various fonts. Partial lists of figures and tables are
typeset in the fonts \verb|\mlffont| and \verb|\mltfont|, which are
\verb|\normalsize\rm| by default.

Titles are typeset in the {\tt\bs ptifont} (\verb|\Huge\bf| by
default) font and the texts of the titles are defined by
{\tt\bs ptctitle}, {\tt\bs plftitle} and {\tt\bs plttitle}, which
are the strings ``Table of Contents'',
``List of Figures'' and ``List of Tables'' by default.
These commands should be redefined by {\tt\bs renewcommand} for
languages other than english.
The option files
{\tt mtcfrench.sty} and {\tt mtcenglish.sty} (and others, see
footnote~\ref{fo+lang} above) are available.
You can easily prepare a similar style for your preferred language.

The section-level table of contents
is typeset in the {\tt\bs stcfont}
font, which is defined as \verb|\normalsize\rm| by default.
Subsection entries are typeset in the {\tt\bs stcSSfont} font,
which is \verb|\normalsize\bf| by default.
Subsubsection entries are typeset in the {\tt\bs stcSSSfont} font,
which is \verb|\normalsize\rm| by default.
For subsubsections, paragraphs and subparagraphs, the
commands \verb|\stcSSSfont|, \verb|\stcPfont| and
\verb|\stcSPfont| are available (by default, \verb|\normalsize\rm|) if you
want to use various fonts. Partial lists of figures and tables are
typeset in the fonts \verb|\slffont| and \verb|\sltfont|, which are
\verb|\normalsize\rm| by default.

Titles are typeset in the {\tt\bs stifont} (\verb|\normalsize\bf| by
default) font and the texts of the titles are defined by
{\tt\bs stctitle}, {\tt\bs slftitle} and {\tt\bs slttitle}, which
are the strings ``Contents'',
``Figures'' and ``Tables'' by default.
These commands should be redefined by {\tt\bs renewcommand} for
languages other than english.
The option files
{\tt mtcfrench.sty} and {\tt mtcenglish.sty} (and others, see
footnote~\ref{fo+lang} above) are available.
You can easily prepare a similar style for your preferred language.

By default, titles are on the left. The commands \verb|\dominitoc|,
\verb|\dominilof| and \verb|\dominilot| accept an optional argument
to change the default position of the
corresponding title: {\tt [l]} for left (default), {\tt [c]} for center,
{\tt [r]} for right, or {\tt [e]} (or {\tt[n]}) for empty (no title).
The change is global for all the document.

If you want to change the position of the title for only one minitoc
(or minilof or minilof), just use such an optional argument with the
command \verb|\minitoc| (or \verb|\minilof| or \verb|\minilot|).

By default, titles are on the left. The commands \verb|\doparttoc|,
\verb|\dopartlof| and \verb|\dopartlot| accept an optional argument
to change the default position of the
corresponding title: {\tt [l]} for left (default), {\tt [c]} for center,
{\tt [r]} for right, or {\tt [e]} (or {\tt[n]})for empty (no title).
The change is global for all the document.

By default, titles are on the left. The commands \verb|\dosecttoc|,
\verb|\dosectlof| and \verb|\dosectlot| accept an optional argument
to change the default position of the
corresponding title: {\tt [l]} for left (default), {\tt [c]} for center,
{\tt [r]} for right, or {\tt [e]} (or {\tt[n]}) for empty (no title).
The change is global for all the document.

If you want to change the position of the title for only one secttoc
(or sectlof or sectlof), just use such an optional argument with the
command \verb|\secttoc| (or \verb|\sectlof| or \verb|\sectlot|).

The mini-tables and lists, as partial and section-level tables and lists,
are using some space on the first pages on each chapter, part or section,
thus the page numbers are altered. After the first \LaTeX\ run, the
mini-tables and lists, partial tables and lists
and section-level tables and lists
will be empty; after the second run, they appear,
but because they modify the page numbering, page numbers
are wrong; after the third \LaTeX\ run, the mini, partial and section-level
tables and lists should be correct.

\subsection{Special Features}
\subsubsection{Horizontal Rules}
By default, most of minitocs and siblings have horizontal rules after their
titles and at their ends. The exception is the ``partoc'' in a book- or
report-like document (i.e. when \verb|\chapter| is defined). To activate
or desactivate these rules, the following commands are available:
\begin{center}
\begin{tabular}{llllccc}
&&&&\multicolumn{3}{c}{defaults for}\\
&rules in&&no rules in&\tt book&\tt report&\tt article\\\hline
\verb|\ptcrule|&parttocs&\verb|\noptcrule|&parttocs&{\bf N}&{\bf N}&{\bf Y}\\
\verb|\mtcrule|&minitocs&\verb|\nomtcrule|&minitocs&{\bf Y}&{\bf Y}&{\bf N-A}\\
\verb|\stcrule|&secttocs&\verb|\nostcrule|&secttocs%
&{\bf N-A}&{\bf N-A}&{\bf Y}\\
\end{tabular}
\end{center}
\subsubsection{Page Numbers}
By default, the page numbers are listed in each minitoc, minilof, etc.
Some authors want only the section titles (with the section numbers),
but not the page numbers. Hence the obvious declarations below are available:
\begin{quote}\tt\begin{tabular}{|ll|lc|l|}
\hline
{\bf Type}&&{\vphantom{$P^2_3$}\hfill\bf Page numbers (Default)\hfill}&&%
{\hfill\bf No page numbers\hfill}\\\hline
\rm minitoc&&\bs mtcpagenumbers&\qquad&\bs nomtcpagenumbers\\
\rm secttoc&&\bs stcpagenumbers&\qquad&\bs nostcpagenumbers\\
\rm parttoc&&\bs ptcpagenumbers&\qquad&\bs noptcpagenumbers\\
\rm minilof&&\bs mlfpagenumbers&\qquad&\bs nomlfpagenumbers\\
\rm sectlof&&\bs slfpagenumbers&\qquad&\bs noslfpagenumbers\\
\rm partlof&&\bs plfpagenumbers&\qquad&\bs noplfpagenumbers\\
\rm minilot&&\bs mltpagenumbers&\qquad&\bs nomltpagenumbers\\
\rm sectlot&&\bs sltpagenumbers&\qquad&\bs nosltpagenumbers\\
\rm partlot&&\bs pltpagenumbers&\qquad&\bs nopltpagenumbers\\
\hline
\end{tabular}\end{quote}

\subsubsection{The ``Chapter 0'' Problem}
Some documents do not begin with chapter number one, but with chapter
number zero (or even a weirder number). To make the {\tt minitoc}
option work with such documents, you must insert the command
\begin{quote}
\tt\bs firstchapteris\{\<N>\}
\end{quote}
before the \verb|\dominitoc| and analogous commands. \<N> is the number
of your first chapter. This command {\em does not\/}
modify the numbering of chapters, you must use a
\verb|\addtocounter{chapter}{-1}| command to get a first chapter
numbered~0. The \verb|\firstpartis| and \verb|\firstsectionis|
commands are analogous for parts and sections with a non standard numbering.

Since\virage{} version \#17c, these commands are obsolete, as this
problem has been solved. Thus they just give a harmless warning.

\subsection{Usage with MS-DOS}\label{MS-DOS}
Under\Virage{} MS-DOS (and other PC oriented operating systems),
the filename extensions are
limited to 3 characters. Thus, it is olnl necessary to edit very
slightly the file {\tt minitoc.sty}. You have just to uncomment
the line where \verb|\SHORTEXT| is defined. All other modifications
will be done automatically.
The {\tt.mtc\<N>}
suffix will become {\tt.M\<N>}, where \<N> is the absolute chapter
number.
The suffixes \verb|.mlf|\<N> and \verb|.mlt|\<N> become
\verb|.F|\<N> and \verb|.T|\<N>.
The {\tt.ptc\<N>}
suffix will become {\tt.P\<N>}, where \<N> is the part
number.
The suffixes \verb|.plf|\<N> and \verb|.plt|\<N> become
\verb|.G|\<N> and \verb|.U|\<N>.
The {\tt.stc\<N>}
suffix will become {\tt.S\<N>}, where \<N> is the absolute section
number.
The suffixes \verb|.slf|\<N> and \verb|.slt|\<N> become
\verb|.H|\<N> and \verb|.V|\<N>.
Of course, this implies a limit of 99~chapters in a
document, but do you really need so many chapters (or sections in an
article)? The limit of 99~parts does not seem too serious.

\section{The {\tt minitocoff} package}\label{o+minitocoff}
When a document has been prepared with the {\tt minitoc} package,
it contains many {\tt minitoc} specific commands, most of them being
\verb|\faketableofcontents| and \verb|\minitoc| commands (and their
equivalents for lists of figures and tables).
If you want to typeset
this document without any mini-table, you have just to replace the
{\tt minitoc} package by the {\tt minitocoff} package, and all these
commands will be ignored. At least two \LaTeX\ runs will be necessary
to get a correct page numbering and correct cross references.
%\end{document}
\chapter{Frequently Asked Questions}
%%%%%%%%%%%%%%%%%% extracted from minitoc.bug
Here is a list of problems and frequently asked questions about
{\tt minitoc.sty}. If your version has a number less than~24, please upgrade to
version~\#24.

\begin{enumerate}
\item
How avoid a page break near the rules before and after
the minitoc?\hfill\null\\
\emph{This problem seemed solved since version~\#8, but version~\#12
adds better fixes.}
\item
How about implementing others layouts for the minitoc? Suggestions are welcome.
\item\relax
\verb|\\|  in a contents line makes an error.\hfill\null\\
\emph{Use} \verb|\protect\linebreak|.
\item
If you reorder chapters, havoc follows\ldots\ minitocs
going in wrong chapters.\hfill\null\\
\emph{The best way seems to make one run with the {\tt minitocoff}
package replacing the {\tt minitoc} package, then restore
the\/ {\tt minitoc} package and re-execute \LaTeX\ three times
(yes, it is time consuming\ldots).
Running with the {\tt minitocoff} package ensures that auxiliary 
are cleared from ``spurious'' commands introduced by {\tt minitoc}.}
%\item
%Compatible with {\tt toch.sty}, but the {\tt minitoc} package must come
%\emph{after} the {\tt toch} package (is it a bug or a ``feature''?).
\item This option creates auxiliary files with extensions like {\tt.mtc\<N>}.
Some operating
systems allow only 3~letters extensions. What to do?\hfill\null\\
\emph{If you have an operating system which tolerates only
short file name extensions
(of 3~characters, like with MS-DOS), you must edit the file {\tt minitoc.sty}:
find the line containing} ``\/\verb|%\def\SHORTEXT|'' \emph{and remove
the leading percent sign. No other modification is needed: all is automatic!}
\item
Do not cheat with the ``{\tt chapter}'' counter, i.e.~do not
write horrible things like \verb|\setcounter{chapter}{6}|. The mechanism
would break. It is better to add {\tt\bs chapter} commands,
to create empty (but numbered in a legal way) chapters.
Since version~\#10, {\tt minitoc.sty} works with appendices.
Version \#19 allows to begin with a chapter other that number~1.
%\item % obsolete
%(8)  minitoc.sty is restricted to document styles which define
%     chapters in the standard way, like `book' and `report'.
%     Do not ask me to make minitocs for sections in articles.
%     There are ``parttocs'' if the document style defines
%     the \part command.
%\item %obsolete
%(9)  Some users have failed to make minitoc to work. They got
%     a message like:
%       Undefined command ... \@inputcheck ...
%     or:
%       Undefined command ... \reset@font ...
%     The \reset@font command has been added to latex.tex
%     on 29 September 1991 and the \@inputcheck command 
%     on 18 March 1992 and this version of latex.tex has been
%     released on 25 March 1992. If you have this message, you
%     have an old version of latex.tex. Get a recent one from
%     the archives and regenerate a latex format via initex.
\item
Some demanding users want to have minilof, minilot and
minibbl. First, minibbl is an other problem, strongly
related to the \BibTeX's dealing with {\tt.aux} files. Look
at {\tt chapterbib.sty}.  Version~\#13 has implemented basic minilofs
and minilots. Minibbls are not the aim of this package.
\item
This package creates a lot of auxiliary files and some users
argue that it is too many. A deep redesign would be
necessary to avoid that. Using only one big auxiliary file
(or one for all minitocs, one for all minilofs, \ldots)
would make the reading of such file very slow, and it would
be read for each {\tt\bs miniXXX} macro!
\item
How to do minitocs (minilofs and minilots) at levels other
than chapter? Here also, some redesign is needed.
From version~\#15, there are parttocs, partlofs and
partlots for the part level in book-like and article-like documents,
secttocs, sectlofs and sectlots for the section level in article-like
documents. Note that you can not have minitocs features at chapter
and section level in the same document, because doing so would make
an unreadable monster. The user must choose the main style of
the document accordingly to the size of it (e.g. do not write an article
of more than 130~sections: this is a report, or even a book!).

\begin{center}\tt
\begin{tabular}{|l|c|c|c|}
\hline
&\bf part&\bf chapter&\bf section\vphantom{\Large Pj}\\\hline
\tt book&$*$&$*$&\vphantom{\Large Pj}\\\hline
\tt report&$*$&$*$&\vphantom{\Large Pj}\\\hline
\tt article&$*$&&$*$\vphantom{\Large Pj}\\\hline
\end{tabular}
\end{center}
\item Since version \#23, works with document classes resetting
     chapter (or section) number at each part.
\end{enumerate}
\end{document}
