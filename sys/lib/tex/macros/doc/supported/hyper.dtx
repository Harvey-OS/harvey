% \iffalse
%% File: hyper.dtx Copyright (C) 1995 Michael Mehlich
%
%<*driver>
\documentclass{ltxdoc}
\usepackage{tabularx}
\makeatletter
\let\TX@vwarn\relax
\makeatother
\IfFileExists{hyper.sty}{%
  \message{*** Generating hypertext documentation ***}
  \usepackage{hyper}%
}{%
  \def\hyperURL##1##2##3##4{##4}
}
\def\AmS{{\protect\usefont{OMS}{cmsy}{m}{n}%
  A\kern-.1667em\lower.5ex\hbox{M}\kern-.125emS}}
\def\amslatex{{\protect\AmS-\protect\LaTeXe}}
%</driver>
\def\fileversion{V2.3}
\def\filedate{1995/02/12}
\def\docdate{1995/02/12}
%<*driver>
\begin{document}
  \title{HyperText Documents with \LaTeXe\\\fileversion}
  \author{Michael Mehlich}
  \date{\filedate}
  \maketitle
  \tableofcontents
  \DocInput{hyper.dtx}
\end{document}
%</driver>
% \fi
% \CheckSum{2040}
%%
%% \CharacterTable
%% {Upper-case    \A\B\C\D\E\F\G\H\I\J\K\L\M\N\O\P\Q\R\S\T\U\V\W\X\Y\Z
%%  Lower-case    \a\b\c\d\e\f\g\h\i\j\k\l\m\n\o\p\q\r\s\t\u\v\w\x\y\z
%%  Digits        \0\1\2\3\4\5\6\7\8\9
%%  Exclamation   \!     Double quote  \"     Hash (number) \#
%%  Dollar        \$     Percent       \%     Ampersand     \&
%%  Acute accent  \'     Left paren    \(     Right paren   \)
%%  Asterisk      \*     Plus          \+     Comma         \,
%%  Minus         \-     Point         \.     Solidus       \/
%%  Colon         \:     Semicolon     \;     Less than     \<
%%  Equals        \=     Greater than  \>     Question mark \?
%%  Commercial at \@     Left bracket  \[     Backslash     \\
%%  Right bracket \]     Circumflex    \^     Underscore    \_
%%  Grave accent  \`     Left brace    \{     Vertical bar  \|
%%  Right brace   \}     Tilde         \~}
%%
% \MakeShortVerb{|}
%
% \title{Hyper\LaTeXe}
% \maketitle
%
% \section{Introduction}
% Traditional texts only have a linear structure given by its series of
% characters. But this is not sufficient for any kind of documents one
% may think of. Thus, recently the number of documents using a more
% complex graphical structure (called ``hyper text'' documents) have 
% increased.
% 
% This package provides the generation of such documents based on the
% already given structure of \LaTeXe{}-documents without the document
% writer having to consider new commands.
%
% \section{Usage}
% 
% Just insert |\usepackage{hyper}| in the preamble of your
% \LaTeX2e--document as the last package to be loaded.\par
% If you want to get an index or a glossary you have to
% use the macros |\makeindex| resp.\ |\makeglossary| before loading
% this package.\par
% If you use the theorem package define the theorems via
% |\newtheorem| after inserting the hyper package.\medskip
% 
% You may also give some options for the package, i.e.\ instead of 
% the command above you may use |\usepackage[<option list>]{hyper}| 
% in your preamble.\medskip
%
% \noindent Currently the following options are known:
% \begin{itemize}
%   \item |yes|, |no| 
%			determine whether the hypertext marks are inserted
%                       or not (default is |yes|)
%   \item |black|, |gray|, |color|  
%		        determine the colors anchors and references
%                       are printed in (default is |black|)
%   \item |amsart|, |amsbook|, |amsdtx|, |amsldoc|, |amsproc|, |article|,
%	  |book|, |letter|, |ltxdoc|, |ltxguide|, |ltnews|, |proc|, 
%	  |report|, |slides|
%			determine the document class for which the
%                       modifications are loaded
%                       (default is the document class given via 
%                       |\documentclass| if it can be derived)
%   \item |leqno|     
%			adds modifications for the document class option
%			|leqno| (will be used automatically for known 
%			document classes)
%   \item |amsmath|, |amsthm|, |doc|, |ftnright|, |longtable|, |theorem|,
%	  |upref| 
%			add modifications for the corresponding package
%                       (will be used automatically)
%   \item |amstex|
%			gives an error message as amslatex is supported only
%			in native mode
%			(will be used automatically)
%   \item |xr|
%			adds modifications for the package |xr| and
%			redefines it's internal macro |\externaldocument|
%			addint a new parameter determining the URL--address
%			of the document for the cross references used
%			(will be used automatically)
%   \item |<other>|   
%			adds a modification file |<other>.hyp| that has to
%                       be provided by yourself
% \end{itemize}
%
% \section{Macros Provided}
% The following macros are defined by the package and may be used:
% \begin{itemize}
%    \item |\hyperanchor#1#2|
%		lets |#1| being the hypertext anchor for |#2|
%		(|#1| may neither be a natural number nor beginning
%		 with |page.| or |fn.|)
%    \item |\blindhyperanchor#1|
%		lets |#1| being a hypertext anchor
%		(|#1| may neither be a natural number nor beginning
%		 with |page.| or |fn.|)
%    \item |\hyperreference#1#2|
%		lets |#2| being a reference to the anchor |#1|,
%		(|#1| may not start with a |[|)
%    \item |\hyperURL#1#2#3#4|
%		lets |#4| be a reference to an external URL
%		with type |#1|, domain |#2|, and document |#3|, i.e.\
%		to the URL |#1://#2/#3|
%    \item |\labeltext#1#2| 
%		lets |#1| being the reference to the text |#2|
%		and prints |#2|
%    \item |\noref[#1]#2| 
%		lets |#1| being a hypertext reference to the
%		anchor |#2| without printing the reference
%		(|#1| is not optional!)
%    \item |\ref[#1]#2|
%		is the same as \LaTeXe's original |#1~\ref{#2}| except that
%		the hypertext reference is provided from the complete
%		|#1~\ref{#2}| instead of |\ref{#2}| only
%    \item |\pageref[#1]#2|
%		is the same as |\ref| execpt that we get a reference to the
%	  	page number 
%    \item |\refcurrent#1|
%		lets |#1| being a hypertext reference to the actual anchor
%    \item |\refnext#1|
%		lets |#1| being a hypertext reference to the next anchor
%		that will be generated
% \end{itemize}
%
% \noindent The following macros are redefined when using the 
% corresponding packages:
% \begin{itemize}
%   \item for |xr|:
%     \begin{itemize}
%       \item |\externaldocument(#1)[#2]#3|, |\externaldocument[#1](#2)#3|
%		are equivalent to the original |\externaldocument| of the
%		|xr|--package except for the additional parameter surrounded
%		by |(||)| determining the URL--address of the external
%		document (the URL--type |file:| is added automatically)
%		(default is |#3.dvi|)
%     \end{itemize}    
% \end{itemize}
%
% \section{How to View Hyper\LaTeXe--documents?}
%
% See URL
% \hyperURL{http}{xxx.lanl.gov}{hypertex/}{|http://xxx.lanl.gov/hypertex/|} 
% for this.
%
% \section{Bugs}
% Currently there is only one known bug. Unfortunately it is one
% that cannot be avoided. You have to care about this:
% \begin{itemize}
%   \item In indexing commands you may not use |]| without surrounding braces.
%         If you do, \TeX{} will parse the command in a wrong way.
%         To get the sorting of the index right you therefore should use
%         |\index{A[zzzzB@A{]}B}| instead of |\index{A]B}|.
% \end{itemize}
%
% \noindent Other bugs are not known by now. But surely there are some. 
% Thus, if you detect one please report it with the package version and
% a small example to\medskip
%
% |      Michael Mehlich|\par
% |      mehlich@informatik.uni-muenchen.de|
%
% \clearpage
% \section{History}
% \leavevmode
% \begin{tabularx}{\textwidth}{|c|l|X|r@{.}l|}
% \hline
% When & By & What & \multicolumn{2}{|c|}{V}
% \\ \hline\hline
% 1995/02/12 & Michael Mehlich &
% support for package |xr| added\newline
% handling of options improved
% & 2&3
% \\ \hline
% 1995/02/11 & Michael Mehlich &
% definition of |\label| corrected (bug introduced at 1995/01/31)
% & 2&2b
% \\ \hline
% 1995/02/02 & Michael Mehlich &
% macros for |amsthm| changed according to new definition in amslatex
% & 2&2a
% \\ \hline
% 1995/02/01 & Michael Mehlich &
% support for package |amslatex| added\newline
% support for package |longtable| added
% & 2&2
% \\ \hline
% 1995/01/31 & Michael Mehlich &
% support for package |theorem| added\newline
% catcodes of \verb+|+, |@|, and |!| for parsing the parameters of
% |\index| and |\glossary| changed\newline
% |\@makecaption| modified\newline
% hypertext references in |eqnarray| corrected\newline
% redefinition of |\label| changed
% & 2&1
% \\ \hline
% 1995/01/30 & Michael Mehlich &
% handling of series of page numbers for index/glossary added\newline
% |\protected@write| stopped from inserting an extra space
% & 2&0b
% \\ \hline
% 1995/01/29 & Michael Mehlich &
% references from headers to sections for bibliography, index, glossary, and 
% table of contents added
% & 2&0a
% \\ \hline
% 1995/01/26 & Michael Mehlich & 
% options to be handled are determined automatically know\newline
% protected the anchor and reference adding macros instead of {\tt"}\newline
% some internal algorithms changed\newline
% some macro--names changed
% & 2&0
% \\ \hline
% 1995/01/25 & Michael Mehlich & handling of index and glossary added & 1&3
% \\ \hline
% 1995/01/24 & Michael Mehlich & 
% handling of options changed\newline 
% color support for anchors and references added
% & 1&2
% \\ \hline
% 1995/01/24 & Michael Mehlich & 
% |\noref| added\newline 
% bug in writing wrong page numbers to auxiliary files removed 
% & 1&1
% \\ \hline
% 1995/01/20 & Michael Mehlich & 
% protected {\tt"} for compatibility with the package |german.sty|
% & 1&0g 
% \\ \hline
% 1995/01/12 & Michael Mehlich & 
% initial version
% & 1&0 
% \\ \hline
% \end{tabularx}
% \StopEventually{}
%
% \clearpage
% \section{The Realization}
%
% \subsection{The Package Header}
% We need a version of \LaTeX2e{} dated with December 1, 1994 or newer. 
% Thus, we just demand it.
%    \begin{macrocode}
%<*hyper>
\NeedsTeXFormat{LaTeX2e}[1994/12/01]
%    \end{macrocode}
% Now we introduce us.
%    \begin{macrocode}
\ProvidesPackage{hyper}[\filedate\space HyperLaTeX2e extension]
%    \end{macrocode}
%
% \subsection{Special Characters}
% First of all let us reserve some characters which are important and are
% not preserved by every other package.
%    \begin{macrocode}
\bgroup
  \catcode`\!00\relax
  \catcode`\"12\relax
  !catcode`!\12!relax
  !gdef!hyper@quote{"}
  !gdef!hyper@backslash{\}
  !global!edef!hyper@hash{!string#} 
!egroup
%    \end{macrocode}
% For compatibility with other packages this may need to be extended.
% But for now we suppose this being sufficient.
%
% \subsection{The Basic Anchor/Reference-Entries}
% We have to define some macros that allow us to insert the hypertext marks
% for anchors and references into the resulting dvi-file. This can be
% done by using the |\special|--macro.
% But we have to be careful as we will write them to files. Hence, they
% are fragile commands.\newline
% The macro |\hyperreference| inserts a html-reference to an internal
% label except when the first parameter begins with |[|. Then the 
% parameter must have the form |[#1][#2]| where |#1| is an external URL and
% |#2| an anchor within this URL. This is used when generating cross references
% between documents.
%    \begin{macrocode}
\bgroup
  \catcode`\<12\catcode`\>12\relax%
  \catcode`\:12\catcode`\=12\catcode`\/12\relax%
  \gdef\hyper{%
    \def\set@fragilehyperreference##1##2{%
      \ifvmode\leavevmode\fi%
      \special{%
        html:<a href=\hyper@quote##1\hyper@quote>%
      }%
        \hyper@ref@color{##2}%
      \special{html:</a>}%
    }%
    \def\f@hyper@ref@ext[##1][##2]\end@hyper@ref{%
      \set@fragilehyperreference{##1\hyper@hash##2}%
    }
    \def\f@hyper@ref@int##1\end@hyper@ref{%
      \set@fragilehyperreference{\hyper@hash##1}%
    }
    \def\f@hyper@ref{%
      \@ifnextchar[{\f@hyper@ref@ext}{\f@hyper@ref@int}%
    }
    \def\fragilehyperreference##1{%
      \expandafter\expandafter\expandafter\f@hyper@ref##1\end@hyper@ref%
    }
    \def\fragilehyperanchor##1##2{%
      \ifvmode\leavevmode\fi%
      \special{%
        html:<a name=\hyper@quote##1\hyper@quote>%
      }%
        \hyper@anchor@color{##2}%
      \special{html:</a>}%
    }
    \def\fragileblindhyperanchor##1{%
      \special{%
        html:<a name=\hyper@quote##1\hyper@quote></a>%
      }%
    }
    \def\fragilehyperURL##1##2##3##4{%
      \ifvmode\leavevmode\fi%
      \special{html:<a href=\hyper@quote##1://##2/##3\hyper@quote>}%
        {##4}%
      \special{html:</a>}%
    }
  }
\egroup
\hyper
%    \end{macrocode}
% Sometimes we need this package without being allowed to insert
% the hypertext marks into the dvi--file.
%    \begin{macrocode}
\def\nohyper{%
  \def\fragilehyperreference##1##2{\ifvmode\leavevmode\fi##2}%
  \def\fragilehyperanchor##1##2{\ifvmode\leavevmode\fi##2}%
  \def\fragileblindhyperanchor##1{}%
  \def\fragilehyperURL##1##2##3##4{\ifvmode\leavevmode\fi##4}%
}
%    \end{macrocode}
% Now lets protect the fragile commands.
%    \begin{macrocode}
\def\hyperreference{\protect\fragilehyperreference}
\def\hyperanchor{\protect\fragilehyperanchor}
\def\blindhyperanchor{\protect\fragileblindhyperanchor}
\def\hyperURL{\protect\fragilehyperURL}
%    \end{macrocode}
%
% \subsection{Colored Anchors/References}
% Above we have used two macros |\hyper@ref@color| and |\hyper@anchor@color|
% with which we can modify the text that is printed to the document.
% As their name indicates we want to use them to color the anchors resp.\
% references. Hence we should define some variants of these macros.\par
% \noindent The first variant prints the text in different colors.
%    \begin{macrocode}
\def\hyper@color{%
  \definecolor{refcolor}{rgb}{1,0,0}%
  \definecolor{anchorcolor}{rgb}{0,1,0}%
  \def\hyper@ref@color##1{\textcolor{refcolor}{##1}}%
  \def\hyper@anchor@color##1{\textcolor{anchorcolor}{##1}}%
}
%    \end{macrocode}
% If we do not have colors we may use gray text instead.
%    \begin{macrocode}
\def\hyper@gray{%
  \definecolor{refcolor}{gray}{0.666}%
  \definecolor{anchorcolor}{gray}{0.333}%
  \def\hyper@ref@color##1{\textcolor{refcolor}{##1}}%
  \def\hyper@anchor@color##1{\textcolor{anchorcolor}{##1}}%
}
%    \end{macrocode}
% In ready to print documents we should not use these options but print
% the text in ordinary black. This is also the default definition.
%    \begin{macrocode}
\def\hyper@black{%
  \def\hyper@ref@color##1{##1}%
  \def\hyper@anchor@color##1{##1}%
}
\hyper@black
%    \end{macrocode} 
%
% \subsection{Automatic Generation of Anchor Names}
% As we want to insert hypertext marks automatically we need a service that
% provides new names for anchors. The simplest solution is to let the 
% name be a natural number.\par
% \noindent Thus we need an initialization for providing them.
%    \begin{macrocode} 
\newcount\hypercount
\hypercount=0\relax
\def\@currenthyper{}
%    \end{macrocode}
% Now we can compute the next anchor from the last one generated.
%    \begin{macrocode}
\def\hyper@setcurrent{%
  \global\advance\hypercount1\relax%
  \edef\@currenthyper{\the\hypercount}%
}%
%    \end{macrocode}
% The automatic generation of anchor names leads to a little problem.
% We have to ensure that |\@currenthyper| is expanded at an appropriate
% time. Thus we define a macro that allow us to get an expanded
% hypertext reference that is inserted after a not yet expanded
% token list.
%    \begin{macrocode}
\def\hyper@ref@toggle#1#2#3{#3[\hyperreference{#1}{#2}]}
\def\ref@by@expanded@hyper#1\end@hyper{{#1}}
\def\callwithexpandedhyperref#1#2{%
  \expandafter\expandafter\expandafter\hyper@ref@toggle%
    \expandafter\ref@by@expanded@hyper\@currenthyper\end@hyper%
  {#2}{#1}%
}
%    \end{macrocode}
%
% \subsection{HyperText Cross--Referencing for \LaTeXe}
% For a replacement of \LaTeXe's cross--referencing features we need to
% redefine the commands |\label| and |\ref|.\par
% \noindent Let us first define a label command for writing the current
% hypertext reference to the |.aux|--file. It must print three components 
% of a label: the hypertext--anchor, the \LaTeXe--anchor, and the
% \LaTeXe--page. This is due to the fact that we need to know not only 
% the stuff we reference to but also the internal name of the anchor.
%    \begin{macrocode}
\def\hyperlabel#1{\@bsphack
  \protected@write\@auxout{}%
         {\string\newlabel{#1}%
	   {{\@currenthyper}{\@currentlabel}{\hyper@thepage}}}%
  \@esphack}
%    \end{macrocode}
% Now we need some auxiliary macros to be able to select each of the three
% parts of the second argument of |\newlabel| that has been written to the
% |.aux|--file
%    \begin{macrocode}
\def\@firstofthree#1#2#3{#1}
\def\@secondofthree#1#2#3{#2}
\def\@thirdofthree#1#2#3{#3}
%    \end{macrocode}
% When setting a reference to an anchor we also have to add the corresponding
% hypertext link into the dvi-file.
%    \begin{macrocode}
\def\hyper@setref#1#2#3#4{%
  \ifx#1\relax%
     \protect\G@refundefinedtrue%
     #4\nfss@text{\reset@font\bfseries ??}%
     \@latex@warning{Reference `#3' on page \thepage \space undefined}%
  \else%
    \ifx#2\@firstofthree%
      \hyperreference{\expandafter\@firstofthree#1}{#4}\null%
    \else\ifx#2\@secondofthree%
      \hyperreference%
        {\expandafter\@firstofthree#1}%
        {#4\expandafter#2#1\null}\null%
    \else%
      \hyperreference%
        {page.\expandafter\@thirdofthree#1}%
        {#4\expandafter#2#1\null}\null%
    \fi\fi%
  \fi%
}
%    \end{macrocode}
% Outgoing from this we can define three macros |\hypernoref|, |\hyperref|, 
% and |\hyperpageref| that correspond to the same \LaTeXe--macros where the
% hyper does not occur. But we want to do a little bit more. Thus an
% additional optional argument may be given to these macros to allow
% the user to reference from a whole text and not only from the number 
% to the anchor.
%    \begin{macrocode}
\def\@@hypernoref[#1]#2{%
  \expandafter\hyper@setref\csname r@#2\endcsname%
    \@firstofthree{#2}{#1}%
}
\def\@hypernoref[#1]{\@@hypernoref[#1]}
\def\hypernoref{\@ifnextchar[{\@hypernoref}{\@@hypernoref[]}}
\def\@@hyperref[#1]#2{%
  \expandafter\hyper@setref\csname r@#2\endcsname%
    \@secondofthree{#2}{#1}%
}
\def\@hyperref[#1]{\@@hyperref[#1~]}
\def\hyperref{\@ifnextchar[{\@hyperref}{\@@hyperref[]}}
\def\@@hyperpageref[#1]#2{%
   \expandafter\hyper@setref\csname r@#2\endcsname%
     \@thirdofthree{#2}{#1}%
}
\def\@hyperpageref[#1]{\@@hyperpageref[#1~]}
\def\hyperpageref{\@ifnextchar[{\@hyperpageref}{\@@hyperpageref[]}}
%    \end{macrocode}
% Now let's patch \LaTeXe's ref/label-system:
%    \begin{macrocode}
\let\label\hyperlabel
\let\noref\hypernoref
\let\ref\hyperref
\let\pageref\hyperpageref
%    \end{macrocode}
% Up to now we only can reference to anchors but haven't yet defined some.
% This will be done in many of the following subsections.
% \subsection{Other Cross--Referencing Features}
% Though \LaTeXe{} has a very good reference mechanism this is not enough
% for hypertext documents. The user should be able to add an anchor to
% a text of his own choice.
%    \begin{macrocode}
\def\labeltext#1#2{%
  \hyper@setcurrent%
  \bgroup%
    \def\@currentlabel{#2}%
    \label{#1}%
    \hyperanchor{\@currenthyper}{#2}%
  \egroup%
}
%    \end{macrocode}
% But that is not enought. Some of the macros of this package define anchors 
% which cannot be caught by a |\label| command. But then it is the current or 
% the next anchor generated. The following macros will allow us to reference 
% to them:
%    \begin{macrocode}
\def\refcurrent#1{\hyperreference{\@currenthyper}{#1}}
\def\hyper@next#1#2{\bgroup%
  \hypercount=#1\relax\advance\hypercount1\relax%
  \hyperreference{\the\hypercount}{#2}%
\egroup}
\def\refnext#1{\protect\hyper@next{\the\hypercount}{#1}}
%    \end{macrocode}
% But wait, there is more. \LaTeXe{} not only uses |\ref| and |\label|
% for it's referencing mechanism but also |\refstepcounter| to determine
% to what the next |\label| should reference.
% This has to be reflected for the hypertext stuff.
%    \begin{macrocode}
\let\hyper@refstepcounter\refstepcounter
\def\refstepcounter{\hyper@setcurrent\hyper@refstepcounter}
%    \end{macrocode}
% This is especially usefull for inserting links from the table of contents
% to an unnumbered section like e.g.\ the references of a document.\par
% \noindent As we will patch some latex internals below we also need
% a macro that allow us to forget the next |\refstepcounter|.
%    \begin{macrocode}
\def\ignore@next@refstepcounter{%
  \def\refstepcounter##1{%
    \def\refstepcounter{\hyper@setcurrent\hyper@refstepcounter}%
  }%
}
%    \end{macrocode}
%
% \subsection{Handling Page Numbers}
% In documents many page numbers are given. Normally you can find one on
% each page of the text. We want to reference to them from every other
% place they can be found. Thus we define our own |\thepage| command 
% dependent on the environment we are in.\par
% \noindent At first we define the standard way to expand |\thepage|
% that is |\thepage| references to the corresponding page.
%     \begin{macrocode}
\let\hyper@thepage\thepage
\def\hyper@ref@page#1{\hyperreference{page.#1}{#1}}
\def\hyper@refpage{\hyper@ref@page\hyper@thepage}
\let\thepage\hyper@refpage
%    \end{macrocode}
% This definition has to be changed if we change the style used to
% number the pages, e.g.\ roman, arabic, ...
%    \begin{macrocode}
\let\hyper@pagenumbering\pagenumbering
\def\pagenumbering#1{%
  \hyper@pagenumbering{#1}%
  \global\let\hyper@thepage\thepage%
  \global\let\thepage\hyper@refpage%
}
%    \end{macrocode}
% But we do not want them to be printed into the auxiliary files
% |.toc|, |.idx|, or |.glo|. Therefore we have to redefine the corresponding
% macros.\par
% \noindent For the |.toc| file we need to modify |\addcontentsline|.
%    \begin{macrocode}
\def\addcontentsline#1#2#3{%
  \bgroup%
    \protected@write\@auxout
        {\let\label\@gobble \let\index\@gobble \let\glossary\@gobble }%
        {\string\@writefile{#1}%
           {\protect\contentsline{#2}{#3}{\hyper@thepage}}}%
  \egroup%
}
%    \end{macrocode}
% For indices the macro to patch is |\@wrindex|.
%    \begin{macrocode}
\def\@wrindex#1{%
   \protected@write\@indexfile{}%
      {\string\indexentry{#1}{\hyper@thepage}}%
 \endgroup
 \@esphack
}
%    \end{macrocode}
% As sometimes glossaries will be printed also |\@wrglossary| has to be
% changed.
%    \begin{macrocode}
\def\@wrglossary#1{%
   \protected@write\@glossaryfile{}%
      {\string\glossaryentry{#1}{\hyper@thepage}}%
 \endgroup
 \@esphack
}
%    \end{macrocode}
% Unfortunately this is not enough to ensure the page numbers printed
% into the toc file. Up to now we sometimes would get wrong numbers.
% To resolve this problem we also have to patch |\protected@write|.
%    \begin{macrocode}
\let\hyper@protected@write\protected@write
\def\protected@write#1#2#3{%
  \begingroup%
    \let\hyper@thepage\relax%
    \hyper@protected@write{#1}{#2}{#3}%
  \endgroup%
}
%    \end{macrocode}
% But wait, we want to refer from the pages in the table of contents
% to the corresponding page. Thus we have to recall the necessary
% information when printing the table.
%    \begin{macrocode}
\let\hyper@contentsline\contentsline
\def\contentsline#1#2#3{%
  \hyper@contentsline{#1}{#2}{\hyper@ref@page{#3}}%
}
%    \end{macrocode}
% The same is valid for indices and glossaries. But they are more
% problematic. Hence we will handle them in a subsection of its own.\par
% \noindent So far the result is a couple of references from page numbers
% to pages. But we haven't defined the anchors they refer to.\par
% \noindent To resolve this we have to modify headers and footers of
% a page. This can be done for an arbitrary page style when we change
% |\@shipoutsetup|.
%    \begin{macrocode}
\def\hyper@anchorpage{\hyperanchor{page.\hyper@thepage}{\hyper@thepage}}
\let\hyper@shipoutsetup\@shipoutsetup
\def\hyper@evenhead{\bgroup%
  \let\thepage\hyper@anchorpage\@evenhead%
\egroup}
\def\hyper@oddhead{\bgroup%
  \let\thepage\hyper@anchorpage\@oddhead%
\egroup}
\def\hyper@evenfoot{\bgroup%
  \let\thepage\hyper@anchorpage\@evenfoot%
\egroup}
\def\hyper@oddfoot{\bgroup%
  \let\thepage\hyper@anchorpage\@evenfoot%
\egroup}
\def\@shipoutsetup{%
  \hyper@shipoutsetup%
     \if@twoside%
       \ifodd\count\z@%
         \let\@thehead\hyper@oddhead%
         \let\@thefoot\hyper@oddfoot%
         \let\@themargin\oddsidemargin%
       \else%
         \let\@thehead\hyper@evenhead%
         \let\@thefoot\hyper@evenfoot%
         \let\@themargin\evensidemargin%
       \fi%
     \else%
       \let\@thehead\hyper@oddhead%
       \let\@thefoot\hyper@oddfoot%
     \fi%
}
%    \end{macrocode}
% Hopefully we do not use a page number twice.\par
% \noindent But there is more to do. Testing the macros we have written
% we would get \LaTeXe--warnings where page numbers contain the whole
% reference information. This should be avoided.
%    \begin{macrocode}
\let\hyper@latex@warning\@latex@warning
\def\@latex@warning#1{%
  \bgroup%
    \let\thepage\hyper@thepage%
    \hyper@latex@warning{#1}%
  \egroup%
}
%    \end{macrocode}
%
% \subsection{Table of Contents}
% In the subsection ``Handling of Pages'' we already have inserted
% references from the page number to the corresponding page.
% Now we want to add further references from the title of an entrie
% in the table of contents to the corresponding text within the
% document.\par
% \noindent To do that we have to modify the corresponding sectioning commands.
%     \begin{macrocode}
\let\hyper@sect\@sect
\def\@sect#1#2#3#4#5#6[#7]#8{%
  \ifnum #2>\c@secnumdepth%
    \hyper@setcurrent%
  \else%
    \refstepcounter{#1}%
    \ignore@next@refstepcounter%
  \fi%
  \global\let\@currenthyper\@currenthyper%
  \callwithexpandedhyperref%
    {\hyper@sect{#1}{#2}{#3}{#4}{#5}{#6}}%
    {#7}%
    {\hyperanchor{\@currenthyper}{#8}}
}
\let\hyper@ssect\@ssect
\def\@ssect#1#2#3#4#5{%
  \hyper@setcurrent%
  \global\let\@currenthyper\@currenthyper%
  \hyper@ssect{#1}{#2}{#3}{#4}{\hyperanchor{\@currenthyper}{#5}}%
}
%</hyper>
%    \end{macrocode}
% The classes |book| and |report| additionally define a |\chapter| command.
% Unfortunately they do it in a different way. But the differences are fairly
% small. Thus we will use nearly the same definition.  
%    \begin{macrocode}
%<*book,report>
\let\hyper@chapter\@chapter
\let\hyper@schapter\@schapter
\def\@chapter[#1]#2{%
  \ifnum\c@secnumdepth>\m@ne%
%</book,report>
%<*book>
    \if@mainmatter%
%</book>
%<*book,report>
      \refstepcounter{chapter}%
      \ignore@next@refstepcounter%
%</book,report>
%<*book>
    \else%
      \hyper@setcurrent%
    \fi%
%</book>
%<*book,report>
  \else%
    \hyper@setcurrent%
  \fi%
  \global\let\@currenthyper\@currenthyper%
  \callwithexpandedhyperref%
    {\hyper@chapter}%
    {#1}%
    {\hyperanchor{\@currenthyper}{#2}}%
}
\def\@schapter#1{%
  \hyper@setcurrent%
  \global\let\@currenthyper\@currenthyper%
  \hyper@schapter{\hyperanchor{\@currenthyper}{#1}}%
}
%</book,report>
%    \end{macrocode}
% There is also a |\part|--macro defined in the classes |article|, |book|,
% |ltnews|, |ltxdoc|, |ltxguide|, |proc|, and |report|.
% Unfortunately there are two different definitions we have to modify.
% But they differ only in a constant in a conditional expression.
% Thus we can use a nearly common definition.
%    \begin{macrocode}
%<*article,book,ltnews,ltxdoc,ltxguide,proc,report>
\let\hyper@part\@part
\def\@part[#1]#2{%
  \ifnum\c@secnumdepth>%
%</article,book,ltnews,ltxdoc,ltxguide,proc,report>
%<*article,ltnews,ltxdoc,ltxguide,proc>
-1\relax%
%</article,ltnews,ltxdoc,ltxguide,proc>
%<*book,report>
-2\relax%
%</book,report>
%<*article,book,ltnews,ltxdoc,ltxguide,proc,report>
      \refstepcounter{part}%
      \ignore@next@refstepcounter%
  \else%
    \hyper@setcurrent%
  \fi%
  \global\let\@currenthyper\@currenthyper%
  \callwithexpandedhyperref%
    {\hyper@part}%
    {#1}%
    {\hyperanchor{\@currenthyper}{#2}}%
}
\let\hyper@spart\@spart
\def\@spart#1{%
  \hyper@setcurrent%
  \global\let\@currenthyper\@currenthyper%
  \hyper@spart{\hyperanchor{\@currenthyper}{#1}}%
}
%</article,book,ltnews,ltxdoc,ltxguide,proc,report>
%    \end{macrocode}
% Now we do not only get references from the table of contents to the
% corresponding sections but also from the headers and footers to them.
% There is only one exception these references are not inserted:
% If we used the |*|--ed variants of the sectioning commands the marks
% are not inserted in the headers.\par
% \noindent For bibliography, glossary, index, and table of contents 
% this can be changed by modifying |\@mkboth|.
%    \begin{macrocode}
%<*article,book,ltnews,ltxdoc,ltxguide,proc,report>
\def\hyper@mkboth@do[#1][#2]{%
  \markboth{#1}{#2}%
}
\def\hyper@mkboth#1#2{%
  \callwithexpandedhyperref%
    {\callwithexpandedhyperref{\hyper@mkboth@do}{#1}}%
    {#2}%
}
%    \end{macrocode}
% As marks are only inserted when using the pagestyle ``headings'',
% the corresponding macro has to be modified to get the desired result.
%    \begin{macrocode}
\let\hyper@ps@headings\ps@headings
\def\ps@headings{%
  \hyper@ps@headings%
  \let\@mkboth\hyper@mkboth%
}
%</article,book,ltnews,ltxdoc,ltxguide,proc,report>
%    \end{macrocode}
% Last but not least set the new page style.
%    \begin{macrocode}
%<*book,report>
\pagestyle{headings}
%</book,report>
%    \end{macrocode}
% Unfortunately, the package |doc.sty| modifies the |theindex| environment
% in a way that the above changes do not work correctly with it.
% Hence, we have to modify this environment to resolve this.
%    \begin{macrocode}
%<*doc>
\let\hyper@index@prologue\index@prologue
\def\index@prologue{%
  \let\hyper@markboth@ip\markboth%
  \def\markboth##1##2{%
    \xdef\hyper@tmpa{\refcurrent{##1}}%
    \xdef\hyper@tmpb{\refcurrent{##2}}%
    \hyper@markboth@ip{\hyper@tmpa}{\hyper@tmpb}%
    \let\markboth\hyper@markboth@ip%
  }%
  \hyper@index@prologue%
}
%</doc>
%    \end{macrocode}
%
% \subsection{Bibliography}
% We want to reference from citation points within the text,
% given by |\cite|, to the corresponding entry in the bibliography.
% To realize this we only have to modify the |\bibitem|--macro.
%    \begin{macrocode}
%<*hyper>
\def\@lbibitem[#1]#2{%
  \hyper@setcurrent%
  \item[\@biblabel{\hyperanchor{\@currenthyper}{#1}}\hfill]%
  \if@filesw%
      {\let\protect\@unexpandable@protect%
       \immediate%
       \write\@auxout{%
         \string\bibcite{#2}{\hyperreference{\@currenthyper}{#1}}%
       }%
      }%
  \fi%
  \ignorespaces%
}
\def\@bibitem{%
  \stepcounter{\@listctr}%
  \@lbibitem[\the\value{\@listctr}]%
}
%    \end{macrocode}
%
% \subsection{Equations}
% The goal is to reference to the equation number in equations and
% arrays of equations. This is really simple to do.
%    \begin{macrocode}
\def\@eqnnum{%
  \hyperanchor{\@currenthyper}{\reset@font\rmfamily (\theequation)}%
}
\let\hyper@eqnarray\eqnarray
\def\eqnarray{%
  \hyper@setcurrent%
  \def\@currenthyper{\the\hypercount}%
  \hyper@eqnarray%
}
\let\hyper@eqncr\@eqncr
\def\@eqncr{\hyper@eqncr\hyper@setcurrent}
%</hyper>
%    \end{macrocode}
% There is a document class option |leqno| that has to be redefined for us.
%    \begin{macrocode}
%<*leqno>
\def\@eqnnum{%
  \hbox to .01\p@{}%
  \rlap{\reset@font\rmfamily%
        \hskip -\displaywidth%
	\hyperanchor{\@currenthyper}{\reset@font\rmfamily (\theequation)}%
       }%
}
%</leqno>
%    \end{macrocode}
%
% \subsection{Float Captions}
% What do you think we want to do here? Simply using |\caption| to define
% an anchor that may be used to refer to the current float.
%     \begin{macrocode}
%<*hyper>
\long\def\@caption#1[#2]#3{\par%
  \addcontentsline%
    {\csname ext@#1\endcsname}%
    {#1}%
    {\protect\numberline{\csname the#1\endcsname}%
     {\ignorespaces \hyperreference{\@currenthyper}{#2}}%
    }%
  \begingroup%
    \@parboxrestore%
    \normalsize%
    \@makecaption{\csname fnum@#1\endcsname}{\ignorespaces #3}\par%
  \endgroup%
}
%</hyper>
%    \end{macrocode}
% Obviously we need to modify |\@makecaption| to define the anchor. 
% As this macro is defined only for the classes |article|, |book|, 
% |ltnews|, |ltxdoc|, |ltxguide|, |proc|,  and |report| we redefine 
% it only for them.
%    \begin{macrocode}
%<*article,book,ltnews,ltxdoc,ltxguide,proc,report>
\let\hyper@makecaption\@makecaption
\long\def\@makecaption#1#2{%
  \hyper@makecaption{\hyperanchor{\@currenthyper}{#1}}{#2}%
}
%</article,book,ltnews,ltxdoc,ltxguide,proc,report>
%    \end{macrocode}
%
% \subsection{Footnotes}
% Here we want to handle references from footnotemarks to footnotes
% at the bottom of the page.\par
% \noindent At first modify the standard definition of |\@makefnmark| to
% refer to a footnote.
%    \begin{macrocode}
%<*hyper>
\def\@makefnmark{%
  \hbox{$^{\hyperreference%
             {fn.\hyper@minipage@label.\@thefnmark.\hyper@thepage}%
             {\@thefnmark}%
          }\m@th$%
       }%
}
%</hyper>
%    \end{macrocode}
% Unfortunately, the definition of |\@makefntext| is dependent of the 
% document class we use.\par
% \noindent For |article|, |book|, |ltnews|, |ltxdoc|, |ltxguide|, |proc|, 
% and |report| we have to use the following hypertext--variant.
%    \begin{macrocode}
%<*article,book,ltnews,ltxproc,ltxguide,proc,report>
\long\def\@makefntext#1{%
  \bgroup%
    \edef\@currenthyper{%
       fn.\hyper@minipage@label.\@thefnmark.\hyper@thepage%
    }%
    \parindent 1em%
    \noindent%
    \hbox to 1.8em%
      {\hss$^{\hyperanchor{\@currenthyper}{\@thefnmark}}\m@th$}#1%
  \egroup%
}
%</article,book,ltnews,ltxproc,ltxguide,proc,report>
%    \end{macrocode}
%For |letter| we need the following.
%    \begin{macrocode}
%<*letter>
\long\def\@makefntext#1{%
  \bgroup%
    \edef\@currenthyper{%
       fn.\hyper@minipage@label.\@thefnmark.\hyper@thepage%
    }%
    \noindent%
    \hangindent 5\p@%
    \hbox to5\p@{$^{\hyperanchor{\@currenthyper}{\@thefnmark}}\m@th$}#1%
  \egroup%
}
%</letter>
%    \end{macrocode}
% And last but not least the definition for |slides| follows.
%    \begin{macrocode}
%<*slides>
\long\def\@makefntext#1{%
  \bgroup%
    \edef\@currenthyper{%
       fn.\hyper@minipage@label.\@thefnmark.\hyper@thepage%
    }%
    \noindent%
    \hangindent 10\p@%
    \hbox to10\p@{%
      $^{\hyperanchor{\@currenthyper}{\@thefnmark}}\m@th$%
    }#1%
  \egroup%
}
%</slides>
%    \end{macrocode}
% There are also footnotes within minipages. Therefore we should handle them.
%    \begin{macrocode}
%<*hyper>
\newcount\hyper@minipage@cnt
\hyper@minipage@cnt=0
\edef\hyper@minipage@label{}
\let\hyper@minipage\minipage
\let\hyper@endminipage\endminipage
\def\minipage{%
  \bgroup%
    \global\advance\hyper@minipage@cnt1\relax%
    \edef\hyper@minipage@label{\the\hyper@minipage@cnt}%
    \hyper@minipage%
}
\def\endminipage{%
    \hyper@endminipage%
  \egroup%
}
%</hyper>
%    \end{macrocode}
% Note, that there exists only one level of minipages the footnotes 
% are handled correctly by \LaTeXe. Thus we haven't to do more.\par
% \noindent Unfortunately there is another problem. The footnote--macros
% are redefined by the |\maketitle|--macro in some classes.\par
% \noindent Thus we need the following changes for the classes
% |\article|, |\book|, |ltxdoc|, |ltxguide|, and |\report|.
%    \begin{macrocode}
%<*article,book,ltxdoc,ltxguide,report>
\if@titlepage\else
\renewcommand\maketitle{\par
  \begingroup
    \renewcommand\thefootnote{\fnsymbol{footnote}}%
    \def\@makefnmark{%
      \hbox to\z@{%
        $\m@th^{%
           \hyperreference{%
              fn.\hyper@minipage@label.\@thefnmark.\hyper@thepage%
           }%
           {\@thefnmark}%
         }$\hss}%
    }%
    \long\def\@makefntext##1{%
      \bgroup%
        \edef\@currenthyper{%
           fn.\hyper@minipage@label.\@thefnmark.\hyper@thepage%
        }%
        \parindent 1em%
        \noindent%
        \hbox to1.8em%
          {\hss$\m@th^{\hyperanchor{\@currenthyper}{\@thefnmark}}$}##1%
       \egroup%
    }%
    \if@twocolumn
      \ifnum \col@number=\@ne
        \@maketitle
      \else
        \twocolumn[\@maketitle]%
      \fi
    \else
      \newpage
      \global\@topnum\z@
      \@maketitle
    \fi
    \thispagestyle{plain}\@thanks
  \endgroup
  \setcounter{footnote}{0}%
  \let\thanks\relax
  \let\maketitle\relax\let\@maketitle\relax
  \gdef\@thanks{}\gdef\@author{}\gdef\@title{}}
\fi
%</article,book,ltxdoc,ltxguide,report>
%    \end{macrocode}
% For the class |proc| another redefinition is needed.
%    \begin{macrocode}
%<*proc>
\def\maketitle{\par
 \begingroup
   \renewcommand\thefootnote{\fnsymbol{footnote}}%
    \def\@makefnmark{%
      \hbox to\z@{%
         $\m@th^{%
            \hyperreference{%
               fn.\hyper@minipage@label.\@thefnmark.\hyper@thepage%
            }%
            {\@thefnmark}%
          }$\hss
      }%
    }%
   \twocolumn[\@maketitle]%
   \@thanks
 \endgroup
 \setcounter{footnote}{0}%
 \let\maketitle\relax
 \let\@maketitle\relax
 \gdef\@thanks{}\gdef\@author{}\gdef\@title{}\let\thanks\relax}
%</proc>
%    \end{macrocode}
%
% \subsection{Lists}
% Some lists have enumerated items which may be referenced to by the
% label command. Hence we have to set the corresponding anchors.
%    \begin{macrocode}
%<*hyper>
\let\hyper@item\@item
\def\@item[#1]{%
  \if@noitemarg%
    \if@nmbrlist%
      \hyper@item[\hyperanchor{\@currenthyper}{#1}]%
    \else%
      \hyper@item[#1]%
    \fi%
  \else%
    \hyper@item[#1]%
  \fi%
}
%    \end{macrocode}
%
% \subsection{Index and Glossary}
% As already have been said, we don't write the page number together with a 
% reference to |.idx|-- resp.\ |.glo|--files. |makeindex| wouldn't accept
% such informations. But that means that we have to think of a way to get
% back the references after |makeindex| has generated the index resp.\
% glossary.\par
% \noindent We only want to insert this stuff if we really have to produce 
% an index or a glossary.
%    \begin{macrocode}
\ifnum%
  \ifx\@indexfile\@undefined0\else1\fi%
  \ifx\@glossaryfile\@undefined0\else1\fi%
  >0\relax
%    \end{macrocode}
% \noindent For the following definition we need the |defpattern|-package.
%    \begin{macrocode}
\RequirePackage{defpattern}
%    \end{macrocode}
% Now we can define hypertext variants of the macros |\index| and
% |\glossary|.\par
% \noindent First of all we handle series of page numbers and ranges.
%    \begin{macrocode}
\defpattern\hyper@page@range[#2]{\hyper@ref@page{#2}}
\defpattern\hyper@page@range[#2--#3]{%
  \hyper@ref@page{#2}--\hyper@ref@page{#3}%
}
\defpattern\hyper@page@range[#2, #3]{%
  \hyper@page@range[#2], \hyper@page@range[#3]%
}
%    \end{macrocode}
% Based on this we can define a macro |\hyperpage| with an optional executable 
% macro as provided by the indexing system to surround a page number.
%    \begin{macrocode}
\def\hyper@page@cmd[#1]#2{#1{\hyper@page@range[#2]}}
\def\hyper@page@nocmd#1{\hyper@page@range[#1]}
\def\hyperpage{\@ifnextchar[{\hyper@page@cmd}{\hyper@page@nocmd}}
%    \end{macrocode}
% After saving the original meaning of |\index| and |\glossary|
%    \begin{macrocode}
\let\hyper@index\index
\let\hyper@glossary\glossary
%    \end{macrocode}
% we can insert the page number surrounding command for |makeindex| by
%    \begin{macrocode}
\defpattern\hyper@ig@pat[#2|#3]{#2{#3|hyperpage}}
\defpattern\hyper@ig@pat[#2|#3|#4]{%
  #2{#3|hyperpage[\hyper@backslash#4]}%
}
\defpattern\hyper@ig@pat[#2|#3|(#4]{%
  #2{#3|(hyperpage[\hyper@backslash#4]}%
}
\defpattern\hyper@ig@pat[#2|#3|)#4]{%
  #2{#3|)hyperpage[\hyper@backslash#4]}%
}
\defpattern\hyper@ig@pat[#2|#3|(]{#2{#3|(hyperpage}}
\defpattern\hyper@ig@pat[#2|#3|)]{#2{#3|)hyperpage}}
%    \end{macrocode}
% and define the new |\index|-- and |\glossary|--macros as follows.
%    \begin{macrocode}
\def\hyper@ig{%
  \bgroup%
    \catcode`\|12\catcode`\@12\catcode`\!12\relax%
    \hyper@ig@%
}
\def\hyper@ig@#1#2{%
    \hyper@ig@pat[#1|#2]%
  \egroup%
}
\def\index{\hyper@ig\hyper@index}
\def\glossary{\hyper@ig\hyper@glossary}
%    \end{macrocode}
% We have reached the end of the conditional part
%    \begin{macrocode}
\fi
%    \end{macrocode}
%
% \subsection{Theorems}
% Theorems, Lemmatas, Corollaries, Examples, etc.\ are often refered to. Hence
% they have to introduce an anchor for these references.
%    \begin{macrocode}
\let\hyper@begintheorem\@begintheorem%
\def\@begintheorem#1#2{\trivlist%
   \item[\hskip%
         \labelsep{\hyperanchor{\@currenthyper}{\bfseries #1\ #2}}%
        ]\itshape%
}
\def\@opargbegintheorem#1#2#3{\trivlist%
      \item[\hskip%
            \labelsep%
               {\hyperanchor{\@currenthyper}{\bfseries #1\ #2}\ (#3)}%
           ]\itshape%
}
%</hyper>
%    \end{macrocode}
% These definitions are changed by the theorem--package. Hence, we have to
% give hypertext definitions for them in future.
%
% \subsection{Additional Patches}
% The Hyper\TeX--previewer for NeXTSTEP can't handle multiple pages with the
% same name correctly. It supposes that each link on a page number 1 is on 
% the first page with the number one. This is a problem for the title page
% often having the same number as one of the following pages. Hence we patch
% the page number of the title page from 1 to -1. 
%    \begin{macrocode}
%<*article,book,ltnews,ltxdoc,ltxguide,proc,report>
\let\hyper@titlepage\titlepage
\def\titlepage{%
  \hyper@titlepage%
  \if@compatibility%
     \global\setcounter{page}{0}%
  \else%
     \global\setcounter{page}{-1}%
  \fi%
}
%</article,book,ltnews,ltxdoc,ltxguide,proc,report>
%    \end{macrocode}
%
% \subsection{Handling Options}
% At first let us define some macros to collect and detect mutually 
% excluding options.
%    \begin{macrocode}
%<*hyper>
\def\hyper@set@option#1#2#3{%
  \ifx#1\@undefined%
    \def#1{#2}%
  \else%
    \errmessage{#3}%
  \fi}
\def\set@hyper@class#1{%
  \hyper@set@option\hyper@class{#1}%
    {Only one class is allowed in option list}%
}
\def\set@hyper@do#1{%
  \hyper@set@option\hyper@do{#1}%
    {Excluding options 'yes' and 'no' in option list found}%
}
\def\set@hyper@color#1{%
  \hyper@set@option\hyper@this@color{#1}%
    {Multiple color options in option list found}%
}
%    \end{macrocode}
% We suppose each other option given being an additional package.
% They are collected without regarding multiple occurrences.
%    \begin{macrocode}
\def\hyper@packages{}
\def\add@hyper@package#1{%
  \edef\hyper@packages{\hyper@packages(#1)}%
}
%    \end{macrocode}
% Now we define the options known to distribute them between the macros
% given above.\par
% \noindent The document classes known by the package are the following.
%    \begin{macrocode}
\DeclareOption{article}{\set@hyper@class{article}}
\DeclareOption{book}{\set@hyper@class{book}}
\DeclareOption{letter}{\set@hyper@class{letter}}
\DeclareOption{ltxdoc}{\set@hyper@class{ltxdoc}}
\DeclareOption{ltxguide}{\set@hyper@class{ltxguide}}
\DeclareOption{ltnews}{\set@hyper@class{ltnews}}
\DeclareOption{proc}{\set@hyper@class{proc}}
\DeclareOption{report}{\set@hyper@class{report}}
\DeclareOption{slides}{\set@hyper@class{slides}}
\DeclareOption{amsart}{\set@hyper@class{amsart}}
\DeclareOption{amsproc}{\set@hyper@class{amsproc}}
\DeclareOption{amsbook}{\set@hyper@class{amsbook}}
\DeclareOption{amsdtx}{\set@hyper@class{amsdtx}}
\DeclareOption{amsldoc}{\set@hyper@class{amsldoc}}
%    \end{macrocode}
% There are two options determining whether we want to insert the 
% hypertext--marks into the dvi--file.
%    \begin{macrocode}
\DeclareOption{yes}{\set@hyper@do{\hyper}}
\DeclareOption{no}{\set@hyper@do{\nohyper}}
%    \end{macrocode}
% In the subsection ``Colored Anchors/References'' we have introduced 
% three variants of colors the anchors resp.\ references may be printed in.
%    \begin{macrocode}
\DeclareOption{color}{\set@hyper@color{\hyper@color}}
\DeclareOption{gray}{\set@hyper@color{\hyper@gray}}
\DeclareOption{black}{\set@hyper@color{\hyper@black}}
%    \end{macrocode}
% All other options to be given are considered as packages.
%    \begin{macrocode}
\DeclareOption*{\add@hyper@package{\CurrentOption}}
%    \end{macrocode}
% Now let's collect the options given.
%    \begin{macrocode}
\ProcessOptions
%    \end{macrocode}
% So far now we know the options the user has send to us.\par
% \noindent If no option determining a document class has been given we should 
% try to determine which class the document is of.
%    \begin{macrocode}
\ifx\hyper@class\@undefined
  \@ifclassloaded{slides}{\def\hyper@class{slides}}{}
  \@ifclassloaded{report}{\def\hyper@class{report}}{}
  \@ifclassloaded{letter}{\def\hyper@class{letter}}{}
  \@ifclassloaded{book}{\def\hyper@class{book}}{}
  \@ifclassloaded{article}{\def\hyper@class{article}}{}
  \@ifclassloaded{proc}{\def\hyper@class{proc}}{}
  \@ifclassloaded{ltnews}{\def\hyper@class{ltnews}}{}
  \@ifclassloaded{ltxguide}{\def\hyper@class{ltxguide}}{}
  \@ifclassloaded{ltxdoc}{\def\hyper@class{ltxdoc}}{}
  \@ifclassloaded{amsart}{\def\hyper@class{amsart}}{}
  \@ifclassloaded{amsproc}{\def\hyper@class{amsproc}}{}
  \@ifclassloaded{amsbook}{\def\hyper@class{amsbook}}{}
  \@ifclassloaded{amsdtx}{\def\hyper@class{amsdtx}}{}
  \@ifclassloaded{amsldoc}{\def\hyper@class{amsldoc}}{}
\fi
%    \end{macrocode}
% If there is a document class load the corresponding |.hyp|--file
% otherwise give an error message.
%    \begin{macrocode}
\ifx\hyper@class\@undefined
  \errmessage{Document class can't be determined. %
              Please add one to the option list.%
	     }
\else
  \InputIfFileExists{\hyper@class.hyp}{}%
    {\errmessage{\hyper@class.hyp not found. %
                 Please reinstall the complete package%
                }%
    }%
\fi
%    \end{macrocode}
% If we want to print the anchors resp.\ references in colors we have to 
% load the |color|-package and defined the appropriate macros.
%    \begin{macrocode}
\ifx\hyper@this@color\@undefined
\else
  \expandafter\ifx\hyper@this@color\hyper@gray
    \RequirePackage{color}
  \else\expandafter\ifx\hyper@this@color\hyper@color
    \RequirePackage{color}
  \fi\fi
  \hyper@this@color
\fi
%    \end{macrocode}
% Afterwards we have to load modifications.\par
% \noindent At first we just handle the document class option |leqno|
% if it can be found in the option list.
%    \begin{macrocode}
\@ifclasswith\hyper@class{leqno}%
  {\InputIfFileExists{leqno.hyp}{}%
     {\errmessage{leqno.hyp not found. %
                  Please reinstall the complete package%
                 }%
     }%
  }%
  {}
%    \end{macrocode}
% We may have to handle some known packages, i.e.\ we need to load
% some modifications for them.\par
% \noindent As there is a common way to do this let us first define 
% a generic macro.
%    \begin{macrocode}
\def\hyper@loadpackage#1{%
  \@ifpackageloaded{#1}%
    {\InputIfFileExists{#1.hyp}%
       {\expandafter\edef\csname hyper@loaded@#1\endcsname{#1.hyp}}%
       {\errmessage{#1.hyp not found. %
                    Please reinstall the complete package%
                   }%
       }%
    }%
    {}
}
%    \end{macrocode}
% Now we can call this macro for each package known.
%    \begin{macrocode}
\hyper@loadpackage{amsmath}
\hyper@loadpackage{amstex}
\hyper@loadpackage{amsthm}
\hyper@loadpackage{doc}
\hyper@loadpackage{ftnright}
\hyper@loadpackage{longtable}
\hyper@loadpackage{theorem}
\hyper@loadpackage{upref}
\hyper@loadpackage{xr}
%    \end{macrocode}
% Afterwards we add each package the user has told us to do.
%    \begin{macrocode}
\def\@hyper@load@pkg(#1){%
  \expandafter\ifx\csname hyper@loaded@#1\endcsname\relax%
    \IfFileExists{#1.hyp}%
      {\RequirePackage{#1}%
       \input{#1.hyp}%
       \expandafter\edef\csname hyper@loaded@#1\endcsname{#1.hyp}%
      }%
      {\errmessage{Modification file #1.hyp not found}}%
  \fi%
  \hyper@loadpackage%
}
\def\hyper@loadpackage{\@ifnextchar({\@hyper@load@pkg}{}}
\expandafter\hyper@loadpackage\hyper@packages\relax
%    \end{macrocode}
% Last but not least we have to decide whether we really want the 
% hypertext--marks being inserted into the dvi--file.
%    \begin{macrocode}
\ifx\hyper@do\@undefined\else\hyper@do\fi
%</hyper>
%    \end{macrocode}
%
% \clearpage
% \section{Supported Packages}
%
% \subsection{\amslatex}
% \subsubsection{The Classes}
% First of all let us modify the \amslatex--classes.\par
% \noindent We change |amsart.cls|, |amsproc.cls|, and |amsbook.cls| 
% in the following way.
%    \begin{macrocode}
%<*amsart,amsproc,amsbook>
\let\hyper@makecaption\@makecaption
\def\@makecaption#1#2{%
  \hyper@makecaption{\hyperanchor{\@currenthyper}{#1}}{#2}%
}
\def\@makefnmark{%
  \hbox{$\m@th^{\hyperreference%
                  {fn.\hyper@minipage@label.\@thefnmark.\hyper@thepage}%
                  {\@thefnmark}%
               }$%
       }%
}
\def\@makefntext{\indent%
  \hbox{$\m@th^{\hyperanchor%
                  {fn.\hyper@minipage@label.\@thefnmark.\hyper@thepage}%
                  {\@thefnmark}%
               }$%
       }%
}
\let\hyper@tocwriteb\@tocwriteb
\def\hyper@tocwriteb@def[#1]{%
    \def\@tocwriteb##1##2##3{%
       \hyper@tocwriteb{##1}{##2}{#1}%
       \let\@tocwriteb\hyper@tocwriteb%
    }%
}
\def\@sect#1#2#3#4#5#6[#7]#8{%
  \ifnum#2>\c@secnumdepth%
    \hyper@setcurrent%
  \else%
    \refstepcounter{#1}%
    \ignore@next@refstepcounter%
  \fi%
  \ifnum#2>\@m\else%
    \callwithexpandedhyperref{\hyper@tocwriteb@def}{#8}%
  \fi%
  \callwithexpandedhyperref%
    {\hyper@sect{#1}{#2}{#3}{#4}{#5}{#6}}%
    {#7}%
    {\hyperanchor{\@currenthyper}{#8}}%
}
%</amsart,amsproc,amsbook>
%    \end{macrocode}
% For |amsbook.cls| we need some additional modifications for parts and chapters.
%    \begin{macrocode}
%<*amsbook>
\let\hyper@chapter\@chapter
\def\@chapter[#1]#2{%
  \refstepcounter{chapter}%
  \ignore@next@refstepcounter%
  \global\let\@currenthyper\@currenthyper%
  \callwithexpandedhyperref{\hyper@tocwriteb@def}{#2}%
  \hyper@chapter[#1]{\hyperanchor{\@currenthyper}{#2}}%
}
\let\hyper@chaptermark\chaptermark
\def\hyper@chaptermark@def[#1]{%
    \def\chaptermark##1{%
       \hyper@chaptermark{#1}%
       \let\chaptermark\hyper@chaptermark%
    }%
}
\let\hyper@schapter\@schapter
\def\@schapter#1{%
  \hyper@setcurrent%
  \global\let\@currenthyper\@currenthyper%
  \callwithexpandedhyperref{\hyper@tocwriteb@def}{#1}%
  \callwithexpandedhyperref{\hyper@chaptermark@def}{#1}%
  \hyper@schapter{\hyperanchor{\@currenthyper}{#1}}%
}
\let\hyper@part\@part
\def\@part[#1]#2{%
  \ifnum\c@secnumdepth>-2\relax%
    \refstepcounter{part}%
    \ignore@next@refstepcounter%
  \else%
    \hyper@setcurrent%
  \fi%
  \global\let\@currenthyper\@currenthyper%
  \callwithexpandedhyperref%
    {\hyper@part}%
    {#1}%
    {\hyperanchor{\@currenthyper}{#2}}%
}
\def\hyper@add@part[#1]{%
  \addcontentsline{toc}{part}{\protect\noindent#1}%
}
\def\@spart#1{%
  \hyper@setcurrent%
  \callwithexpandedhyperref{\hyper@add@part}{#1}%
  \begingroup\centering
  \fontsize{\@xxpt}{25}\bfseries
     \hyperanchor{\@currenthyper}{#1}%
     \vfil\vfil\endgroup \newpage\thispagestyle{empty}}
%</amsbook>
%    \end{macrocode}
% |amsdtx.cls| has to be modified as follows.
%    \begin{macrocode}
%<*amsdtx>
%% 
%% This is file `book.cls', generated 
%% on <1994/12/18> with the docstrip utility (2.2i).
%% 
%% The original source files were:
%% 
%% classes.dtx  (with options: `book')
%% 
%% Copyright 1994 the LaTeX3 project and the individual authors. 
%% All rights reserved. For further copyright information see the file 
%% legal.txt, and any other copyright indicated in this file. 
%% 
%% This file is part of the LaTeX2e system. 
%% ---------------------------------------- 
%% 
%% This system is distributed in the hope that it will be useful, 
%% but WITHOUT ANY WARRANTY; without even the implied warranty of 
%% MERCHANTABILITY or FITNESS FOR A PARTICULAR PURPOSE. 
%% 
%% 
%% IMPORTANT NOTICE: 
%% 
%% For error reports in case of UNCHANGED versions see bugs.txt. 
%% 
%% Please do not request updates from us directly. Distribution is 
%% done through Mail-Servers and TeX organizations. 
%% 
%% You are not allowed to change this file. 
%% 
%% You are allowed to distribute this file under the condition that 
%% it is distributed together with all files mentioned in manifest.txt. 
%% 
%% If you receive only some of these files from someone, complain! 
%% 
%% You are NOT ALLOWED to distribute this file alone. You are NOT 
%% ALLOWED to take money for the distribution or use of either this 
%% file or a changed version, except for a nominal charge for copying 
%% etc. 
%% \CharacterTable
%%  {Upper-case    \A\B\C\D\E\F\G\H\I\J\K\L\M\N\O\P\Q\R\S\T\U\V\W\X\Y\Z
%%   Lower-case    \a\b\c\d\e\f\g\h\i\j\k\l\m\n\o\p\q\r\s\t\u\v\w\x\y\z
%%   Digits        \0\1\2\3\4\5\6\7\8\9
%%   Exclamation   \!     Double quote  \"     Hash (number) \#
%%   Dollar        \$     Percent       \%     Ampersand     \&
%%   Acute accent  \'     Left paren    \(     Right paren   \)
%%   Asterisk      \*     Plus          \+     Comma         \,
%%   Minus         \-     Point         \.     Solidus       \/
%%   Colon         \:     Semicolon     \;     Less than     \<
%%   Equals        \=     Greater than  \>     Question mark \?
%%   Commercial at \@     Left bracket  \[     Backslash     \\
%%   Right bracket \]     Circumflex    \^     Underscore    \_
%%   Grave accent  \`     Left brace    \{     Vertical bar  \|
%%   Right brace   \}     Tilde         \~}
\NeedsTeXFormat{LaTeX2e}
\ProvidesClass{book}
              [1994/12/09 v1.2x
 Standard LaTeX document class]
\newcommand{\@ptsize}{}
\newif\if@restonecol
\newif\if@titlepage
\@titlepagetrue
\newif\if@openright
\newif\if@openbib
\@openbibfalse
\newif\if@mainmatter \@mainmattertrue
\DeclareOption{a4paper}
   {\setlength\paperheight {297mm}%
    \setlength\paperwidth  {210mm}}
\DeclareOption{a5paper}
   {\setlength\paperheight {210mm}%
    \setlength\paperwidth  {148mm}}
\DeclareOption{b5paper}
   {\setlength\paperheight {250mm}%
    \setlength\paperwidth  {176mm}}
\DeclareOption{letterpaper}
   {\setlength\paperheight {11in}%
    \setlength\paperwidth  {8.5in}}
\DeclareOption{legalpaper}
   {\setlength\paperheight {14in}%
    \setlength\paperwidth  {8.5in}}
\DeclareOption{executivepaper}
   {\setlength\paperheight {10.5in}%
    \setlength\paperwidth  {7.25in}}
\DeclareOption{landscape}
   {\setlength\@tempdima   {\paperheight}%
    \setlength\paperheight {\paperwidth}%
    \setlength\paperwidth  {\@tempdima}}
\DeclareOption{10pt}{\renewcommand{\@ptsize}{0}}
\DeclareOption{11pt}{\renewcommand{\@ptsize}{1}}
\DeclareOption{12pt}{\renewcommand{\@ptsize}{2}}
\DeclareOption{oneside}{\@twosidefalse \@mparswitchfalse}
\DeclareOption{twoside}{\@twosidetrue  \@mparswitchtrue}
\DeclareOption{draft}{\setlength\overfullrule{5pt}}
\DeclareOption{final}{\setlength\overfullrule{0pt}}
\DeclareOption{titlepage}{\@titlepagetrue}
\DeclareOption{notitlepage}{\@titlepagefalse}
\DeclareOption{openright}{\@openrighttrue}
\DeclareOption{openany}{\@openrightfalse}
\DeclareOption{onecolumn}{\@twocolumnfalse}
\DeclareOption{twocolumn}{\@twocolumntrue}
\DeclareOption{leqno}{% leqno.sty 14 Jun 85
\def\@eqnnum{\hbox to .01pt{}\rlap{\rm \hskip -\displaywidth(\theequation)}}
}
\DeclareOption{fleqn}{% fleqn.sty 26-Sep-89


\def\[{\relax\ifmmode\@badmath\else
 \begin{trivlist}
 \@beginparpenalty\predisplaypenalty
 \@endparpenalty\postdisplaypenalty
 \item[]\leavevmode
 \hbox to\linewidth\bgroup$ \displaystyle
 \hskip\mathindent\bgroup\fi}
\def\]{\relax\ifmmode \egroup $\hfil \egroup \end{trivlist}\else \@badmath \fi}
\def\equation{\@beginparpenalty\predisplaypenalty
 \@endparpenalty\postdisplaypenalty
\refstepcounter{equation}\trivlist \item[]\leavevmode
 \hbox to\linewidth\bgroup $ \displaystyle
\hskip\mathindent}
\def\endequation{$\hfil \displaywidth\linewidth\@eqnnum\egroup \endtrivlist}
\def\eqnarray{\stepcounter{equation}\let\@currentlabel=\theequation
\global\@eqnswtrue
\global\@eqcnt\z@\tabskip\mathindent\let\\=\@eqncr
\abovedisplayskip\topsep\ifvmode\advance\abovedisplayskip\partopsep\fi
\belowdisplayskip\abovedisplayskip
\belowdisplayshortskip\abovedisplayskip
\abovedisplayshortskip\abovedisplayskip
$$\halign to
\linewidth\bgroup\@eqnsel\hskip\@centering$\displaystyle\tabskip\z@
 {##}$&\global\@eqcnt\@ne \hskip 2\arraycolsep \hfil${##}$\hfil
 &\global\@eqcnt\tw@ \hskip 2\arraycolsep $\displaystyle{##}$\hfil 
 \tabskip\@centering&\llap{##}\tabskip\z@\cr}
\def\endeqnarray{\@@eqncr\egroup
 \global\advance\c@equation\m@ne$$\global\@ignoretrue }
\newdimen\mathindent 
\mathindent = \leftmargini

}
\DeclareOption{openbib}{\@openbibtrue}
\ExecuteOptions{letterpaper,10pt,twoside,onecolumn,final,openright}
\ProcessOptions
\input{bk1\@ptsize.clo}
\setlength\lineskip{1\p@}
\setlength\normallineskip{1\p@}
\renewcommand{\baselinestretch}{}
\setlength\parskip{0\p@ \@plus \p@}
\@lowpenalty   51
\@medpenalty  151
\@highpenalty 301
\setcounter{topnumber}{2}
\renewcommand{\topfraction}{.7}
\setcounter{bottomnumber}{1}
\renewcommand{\bottomfraction}{.3}
\setcounter{totalnumber}{3}
\renewcommand{\textfraction}{.2}
\renewcommand{\floatpagefraction}{.5}
\setcounter{dbltopnumber}{2}
\renewcommand{\dbltopfraction}{.7}
\renewcommand{\dblfloatpagefraction}{.5}
\if@twoside
  \def\ps@headings{%
      \let\@oddfoot\@empty\let\@evenfoot\@empty
      \def\@evenhead{\thepage\hfil\slshape\leftmark}%
      \def\@oddhead{{\slshape\rightmark}\hfil\thepage}%
      \let\@mkboth\markboth
    \def\chaptermark##1{%
      \markboth {\uppercase{%
        \ifnum \c@secnumdepth >\m@ne
          \if@mainmatter
            \@chapapp\ \thechapter. \ %
          \fi
        \fi
        ##1}}{}}%
    \def\sectionmark##1{%
      \markright {\uppercase{%
        \ifnum \c@secnumdepth >\z@
          \thesection. \ %
        \fi
        ##1}}}}
\else
  \def\ps@headings{%
    \let\@oddfoot\@empty
    \def\@oddhead{{\slshape\rightmark}\hfil\thepage}%
    \let\@mkboth\markboth
    \def\chaptermark##1{%
      \markright {\uppercase{%
        \ifnum \c@secnumdepth >\m@ne
          \if@mainmatter
            \@chapapp\ \thechapter. \ %
          \fi
        \fi
        ##1}}}}
\fi
\def\ps@myheadings{%
    \let\@oddfoot\@empty\let\@evenfoot\@empty
    \def\@evenhead{\thepage\hfil\slshape\leftmark}%
    \def\@oddhead{{\slshape\rightmark}\hfil\thepage}%
    \let\@mkboth\@gobbletwo
    \let\chaptermark\@gobble
    \let\sectionmark\@gobble
    }
\if@titlepage
  \newcommand{\maketitle}{\begin{titlepage}%
  \let\footnotesize\small
  \let\footnoterule\relax
  \null\vfil
  \vskip 60\p@
  \begin{center}%
    {\LARGE \@title \par}%
    \vskip 3em%
    {\large
     \lineskip .75em%
      \begin{tabular}[t]{c}%
        \@author
      \end{tabular}\par}%
      \vskip 1.5em%
    {\large \@date \par}%       % Set date in \large size.
  \end{center}\par
  \@thanks
  \vfil\null
  \end{titlepage}%
  \setcounter{footnote}{0}%
  \let\thanks\relax\let\maketitle\relax
  \gdef\@thanks{}\gdef\@author{}\gdef\@title{}}
\else
\newcommand{\maketitle}{\par
  \begingroup
    \renewcommand{\thefootnote}{\fnsymbol{footnote}}%
    \def\@makefnmark{\hbox to\z@{$\m@th^{\@thefnmark}$\hss}}%
    \long\def\@makefntext##1{\parindent 1em\noindent
            \hbox to1.8em{\hss$\m@th^{\@thefnmark}$}##1}%
    \if@twocolumn
      \ifnum \col@number=\@ne
        \@maketitle
      \else
        \twocolumn[\@maketitle]%
      \fi
    \else
      \newpage
      \global\@topnum\z@   % Prevents figures from going at top of page.
      \@maketitle
    \fi
    \thispagestyle{plain}\@thanks
  \endgroup
  \setcounter{footnote}{0}%
  \let\thanks\relax
  \let\maketitle\relax\let\@maketitle\relax
  \gdef\@thanks{}\gdef\@author{}\gdef\@title{}}
\def\@maketitle{%
  \newpage
  \null
  \vskip 2em%
  \begin{center}%
    {\LARGE \@title \par}%
    \vskip 1.5em%
    {\large
      \lineskip .5em%
      \begin{tabular}[t]{c}%
        \@author
      \end{tabular}\par}%
    \vskip 1em%
    {\large \@date}%
  \end{center}%
  \par
  \vskip 1.5em}
\fi
\newcommand*{\chaptermark}[1]{}
\setcounter{secnumdepth}{2}
\newcounter {part}
\newcounter {chapter}
\newcounter {section}[chapter]
\newcounter {subsection}[section]
\newcounter {subsubsection}[subsection]
\newcounter {paragraph}[subsubsection]
\newcounter {subparagraph}[paragraph]
\renewcommand{\thepart}         {\Roman{part}}
\renewcommand{\thechapter}      {\arabic{chapter}}
\renewcommand{\thesection}      {\thechapter.\arabic{section}}
\renewcommand{\thesubsection}   {\thesection.\arabic{subsection}}
\renewcommand{\thesubsubsection}{\thesubsection .\arabic{subsubsection}}
\renewcommand{\theparagraph}    {\thesubsubsection.\arabic{paragraph}}
\renewcommand{\thesubparagraph} {\theparagraph.\arabic{subparagraph}}
\newcommand{\@chapapp}{\chaptername}
\newcommand{\frontmatter}{\cleardoublepage
            \@mainmatterfalse\pagenumbering{roman}}
\newcommand{\mainmatter}{\cleardoublepage
       \@mainmattertrue\pagenumbering{arabic}}
\newcommand{\backmatter}{\if@openright\cleardoublepage\else\clearpage\fi
      \@mainmatterfalse}
\newcommand{\part}{\cleardoublepage
                 \thispagestyle{plain}%
                 \if@twocolumn
                     \onecolumn
                     \@tempswatrue
                   \else
                     \@tempswafalse
                 \fi
                 \hbox{}\vfil
                 \secdef\@part\@spart}

\def\@part[#1]#2{%
    \ifnum \c@secnumdepth >-2\relax
      \refstepcounter{part}%
      \addcontentsline{toc}{part}{\thepart\hspace{1em}#1}%
    \else
      \addcontentsline{toc}{part}{#1}%
    \fi
    \markboth{}{}
    {\centering
     \interlinepenalty \@M
     \reset@font
     \ifnum \c@secnumdepth >-2\relax
       \huge\bfseries \partname~\thepart
       \par
       \vskip 20\p@
     \fi
     \Huge \bfseries #2\par}%
    \@endpart}
\def\@spart#1{%
    {\centering
     \interlinepenalty \@M
     \reset@font
     \Huge \bfseries #1\par}%
    \@endpart}
\def\@endpart{\vfil\newpage
              \if@twoside
                \hbox{}%
                \thispagestyle{empty}%
                \newpage
              \fi
              \if@tempswa
                \twocolumn
              \fi}
\newcommand{\chapter}{\if@openright\cleardoublepage\else\clearpage\fi
                    \thispagestyle{plain}%
                    \global\@topnum\z@
                    \@afterindentfalse
                    \secdef\@chapter\@schapter}
\def\@chapter[#1]#2{\ifnum \c@secnumdepth >\m@ne
                       \if@mainmatter
                         \refstepcounter{chapter}%
                         \typeout{\@chapapp\space\thechapter.}%
                         \addcontentsline{toc}{chapter}%
                                   {\protect\numberline{\thechapter}#1}%
                       \else
                         \addcontentsline{toc}{chapter}{#1}%
                       \fi
                    \else
                      \addcontentsline{toc}{chapter}{#1}%
                    \fi
                    \chaptermark{#1}%
                    \addtocontents{lof}{\protect\addvspace{10\p@}}%
                    \addtocontents{lot}{\protect\addvspace{10\p@}}%
                    \if@twocolumn
                      \@topnewpage[\@makechapterhead{#2}]%
                    \else
                      \@makechapterhead{#2}%
                      \@afterheading
                    \fi}
\def\@makechapterhead#1{%
  \vspace*{50\p@}%
  {\parindent \z@ \raggedright \reset@font
    \ifnum \c@secnumdepth >\m@ne
      \if@mainmatter
        \huge\bfseries \@chapapp{} \thechapter
        \par\nobreak
        \vskip 20\p@
      \fi
    \fi
    \interlinepenalty\@M
    \Huge \bfseries #1\par\nobreak
    \vskip 40\p@
  }}
\def\@schapter#1{\if@twocolumn
                   \@topnewpage[\@makeschapterhead{#1}]%
                 \else
                   \@makeschapterhead{#1}%
                   \@afterheading
                 \fi}
\def\@makeschapterhead#1{%
  \vspace*{50\p@}%
  {\parindent \z@ \raggedright
    \reset@font
    \interlinepenalty\@M
    \Huge \bfseries  #1\par\nobreak
    \vskip 40\p@
  }}
\newcommand{\section}{\@startsection {section}{1}{\z@}%
                                   {-3.5ex \@plus -1ex \@minus -.2ex}%
                                   {2.3ex \@plus.2ex}%
                                   {\reset@font\Large\bfseries}}
\newcommand{\subsection}{\@startsection{subsection}{2}{\z@}%
                                     {-3.25ex\@plus -1ex \@minus -.2ex}%
                                     {1.5ex \@plus .2ex}%
                                     {\reset@font\large\bfseries}}
\newcommand{\subsubsection}{\@startsection{subsubsection}{3}{\z@}%
                                     {-3.25ex\@plus -1ex \@minus -.2ex}%
                                     {1.5ex \@plus .2ex}%
                                     {\reset@font\normalsize\bfseries}}
\newcommand{\paragraph}{\@startsection{paragraph}{4}{\z@}%
                                    {3.25ex \@plus1ex \@minus.2ex}%
                                    {-1em}%
                                    {\reset@font\normalsize\bfseries}}
\newcommand{\subparagraph}{\@startsection{subparagraph}{5}{\parindent}%
                                       {3.25ex \@plus1ex \@minus .2ex}%
                                       {-1em}%
                                      {\reset@font\normalsize\bfseries}}
\if@twocolumn
  \setlength\leftmargini  {2em}
\else
  \setlength\leftmargini  {2.5em}
\fi
\setlength\leftmarginii  {2.2em}
\setlength\leftmarginiii {1.87em}
\setlength\leftmarginiv  {1.7em}
\if@twocolumn
  \setlength\leftmarginv  {.5em}
  \setlength\leftmarginvi {.5em}
\else
  \setlength\leftmarginv  {1em}
  \setlength\leftmarginvi {1em}
\fi
\setlength\leftmargin    {\leftmargini}
\setlength  \labelsep  {.5em}
\setlength  \labelwidth{\leftmargini}
\addtolength\labelwidth{-\labelsep}
\@beginparpenalty -\@lowpenalty
\@endparpenalty   -\@lowpenalty
\@itempenalty     -\@lowpenalty
\renewcommand{\theenumi}{\arabic{enumi}}
\renewcommand{\theenumii}{\alph{enumii}}
\renewcommand{\theenumiii}{\roman{enumiii}}
\renewcommand{\theenumiv}{\Alph{enumiv}}
\newcommand{\labelenumi}{\theenumi.}
\newcommand{\labelenumii}{(\theenumii)}
\newcommand{\labelenumiii}{\theenumiii.}
\newcommand{\labelenumiv}{\theenumiv.}
\renewcommand{\p@enumii}{\theenumi}
\renewcommand{\p@enumiii}{\theenumi(\theenumii)}
\renewcommand{\p@enumiv}{\p@enumiii\theenumiii}
\newcommand{\labelitemi}{$\m@th\bullet$}
\newcommand{\labelitemii}{\normalfont\bfseries --}
\newcommand{\labelitemiii}{$\m@th\ast$}
\newcommand{\labelitemiv}{$\m@th\cdot$}
\newenvironment{description}
               {\list{}{\labelwidth\z@ \itemindent-\leftmargin
                        \let\makelabel\descriptionlabel}}
               {\endlist}
\newcommand{\descriptionlabel}[1]{\hspace\labelsep
                                \normalfont\bfseries #1}
\newenvironment{verse}
               {\let\\=\@centercr
                \list{}{\itemsep      \z@
                        \itemindent   -1.5em%
                        \listparindent\itemindent
                        \rightmargin  \leftmargin
                        \advance\leftmargin 1.5em}%
                \item[]}
               {\endlist}
\newenvironment{quotation}
               {\list{}{\listparindent 1.5em%
                        \itemindent    \listparindent
                        \rightmargin   \leftmargin
                        \parsep        \z@ \@plus\p@}%
                \item[]}
               {\endlist}
\newenvironment{quote}
               {\list{}{\rightmargin\leftmargin}%
                \item[]}
               {\endlist}
\newenvironment{titlepage}
    {%
      \cleardoublepage
      \if@twocolumn
        \@restonecoltrue\onecolumn
      \else
        \@restonecolfalse\newpage
      \fi
      \thispagestyle{empty}%
      \if@compatibility
        \setcounter{page}{0}
      \fi}%
    {\if@restonecol\twocolumn \else \newpage \fi
    }
\newcommand{\appendix}{\par
  \setcounter{chapter}{0}%
  \setcounter{section}{0}%
  \renewcommand{\@chapapp}{\appendixname}%
  \renewcommand{\thechapter}{\Alph{chapter}}}
\setlength\arraycolsep{5\p@}
\setlength\tabcolsep{6\p@}
\setlength\arrayrulewidth{.4\p@}
\setlength\doublerulesep{2\p@}
\setlength\tabbingsep{\labelsep}
\skip\@mpfootins = \skip\footins
\setlength\fboxsep{3\p@}
\setlength\fboxrule{.4\p@}
\@addtoreset{equation}{chapter}
\renewcommand{\theequation}{\thechapter.\arabic{equation}}
\newcounter{figure}[chapter]
\renewcommand{\thefigure}{\thechapter.\@arabic\c@figure}
\def\fps@figure{tbp}
\def\ftype@figure{1}
\def\ext@figure{lof}
\def\fnum@figure{\figurename~\thefigure}
\newenvironment{figure}
               {\@float{figure}}
               {\end@float}
\newenvironment{figure*}
               {\@dblfloat{figure}}
               {\end@dblfloat}
\newcounter{table}[chapter]
\renewcommand{\thetable}{\thechapter.\@arabic\c@table}
\def\fps@table{tbp}
\def\ftype@table{2}
\def\ext@table{lot}
\def\fnum@table{\tablename~\thetable}
\newenvironment{table}
               {\@float{table}}
               {\end@float}
\newenvironment{table*}
               {\@dblfloat{table}}
               {\end@dblfloat}
\newlength\abovecaptionskip
\newlength\belowcaptionskip
\setlength\abovecaptionskip{10\p@}
\setlength\belowcaptionskip{0\p@}
\long\def\@makecaption#1#2{%
  \vskip\abovecaptionskip
  \sbox\@tempboxa{#1: #2}%
  \ifdim \wd\@tempboxa >\hsize
    #1: #2\par
  \else
    \global \@minipagefalse
    \hbox to\hsize{\hfil\box\@tempboxa\hfil}%
  \fi
  \vskip\belowcaptionskip}
\DeclareOldFontCommand{\rm}{\normalfont\rmfamily}{\mathrm}
\DeclareOldFontCommand{\sf}{\normalfont\sffamily}{\mathsf}
\DeclareOldFontCommand{\tt}{\normalfont\ttfamily}{\mathtt}
\DeclareOldFontCommand{\bf}{\normalfont\bfseries}{\mathbf}
\DeclareOldFontCommand{\it}{\normalfont\itshape}{\mathit}
\DeclareOldFontCommand{\sl}{\normalfont\slshape}{\@nomath\sl}
\DeclareOldFontCommand{\sc}{\normalfont\scshape}{\@nomath\sc}
\DeclareRobustCommand*{\cal}{\@fontswitch{\relax}{\mathcal}}
\DeclareRobustCommand*{\mit}{\@fontswitch{\relax}{\mathnormal}}
\newcommand{\@pnumwidth}{1.55em}
\newcommand{\@tocrmarg} {2.55em}
\newcommand{\@dotsep}{4.5}
\setcounter{tocdepth}{2}
\newcommand{\tableofcontents}{%
    \if@twocolumn
      \@restonecoltrue\onecolumn
    \else
      \@restonecolfalse
    \fi
    \chapter*{\contentsname
        \@mkboth{\uppercase{\contentsname}}{\uppercase{\contentsname}}}%
    \@starttoc{toc}%
    \if@restonecol\twocolumn\fi
    }
\newcommand*{\l@part}[2]{%
  \ifnum \c@tocdepth >-2\relax
    \addpenalty{-\@highpenalty}%
    \addvspace{2.25em \@plus\p@}%
    \begingroup
      \setlength\@tempdima{3em}%
      \parindent \z@ \rightskip \@pnumwidth
      \parfillskip -\@pnumwidth
      {\leavevmode
       \large \bfseries #1\hfil \hbox to\@pnumwidth{\hss #2}}\par
       \nobreak
         \global\@nobreaktrue
         \everypar{\global\@nobreakfalse\everypar{}}
    \endgroup
  \fi}
\newcommand*{\l@chapter}[2]{%
  \ifnum \c@tocdepth >\m@ne
    \addpenalty{-\@highpenalty}%
    \vskip 1.0em \@plus\p@
    \setlength\@tempdima{1.5em}%
    \begingroup
      \parindent \z@ \rightskip \@pnumwidth
      \parfillskip -\@pnumwidth
      \leavevmode \bfseries
      \advance\leftskip\@tempdima
      \hskip -\leftskip
      #1\nobreak\hfil \nobreak\hbox to\@pnumwidth{\hss #2}\par
      \penalty\@highpenalty
    \endgroup
  \fi}
\newcommand*{\l@section}      {\@dottedtocline{1}{1.5em}{2.3em}}
\newcommand*{\l@subsection}   {\@dottedtocline{2}{3.8em}{3.2em}}
\newcommand*{\l@subsubsection}{\@dottedtocline{3}{7.0em}{4.1em}}
\newcommand*{\l@paragraph}    {\@dottedtocline{4}{10em}{5em}}
\newcommand*{\l@subparagraph} {\@dottedtocline{5}{12em}{6em}}
\newcommand{\listoffigures}{%
    \if@twocolumn
      \@restonecoltrue\onecolumn
    \else
      \@restonecolfalse
    \fi
    \chapter*{\listfigurename
      \@mkboth{\uppercase{\listfigurename}}%
              {\uppercase{\listfigurename}}}%
    \@starttoc{lof}%
    \if@restonecol\twocolumn\fi
    }
\newcommand*{\l@figure}{\@dottedtocline{1}{1.5em}{2.3em}}
\newcommand{\listoftables}{%
    \if@twocolumn
      \@restonecoltrue\onecolumn
    \else
      \@restonecolfalse
    \fi
    \chapter*{\listtablename
      \@mkboth{\uppercase{\listtablename}}{\uppercase{\listtablename}}}%
    \@starttoc{lot}%
    \if@restonecol\twocolumn\fi
    }
\let\l@table\l@figure
\newdimen\bibindent
\bibindent=1.5em
\newenvironment{thebibliography}[1]
     {\chapter*{\bibname
        \@mkboth{\uppercase{\bibname}}{\uppercase{\bibname}}}%
      \list{\@biblabel{\arabic{enumiv}}}%
           {\settowidth\labelwidth{\@biblabel{#1}}%
            \leftmargin\labelwidth
            \advance\leftmargin\labelsep
            \if@openbib
              \advance\leftmargin\bibindent
              \itemindent -\bibindent
              \listparindent \itemindent
              \parsep \z@
            \fi
            \usecounter{enumiv}%
            \let\p@enumiv\@empty
            \renewcommand{\theenumiv}{\arabic{enumiv}}}%
      \if@openbib
        \renewcommand{\newblock}{\par}
      \else
        \renewcommand{\newblock}{\hskip .11em \@plus.33em \@minus.07em}%
      \fi
      \sloppy\clubpenalty4000\widowpenalty4000%
      \sfcode`\.=\@m}
     {\def\@noitemerr
       {\@latex@warning{Empty `thebibliography' environment}}%
      \endlist}
\newcommand{\newblock}{}
\newenvironment{theindex}
               {\if@twocolumn
                  \@restonecolfalse
                \else
                  \@restonecoltrue
                \fi
                \columnseprule \z@
                \columnsep 35\p@
                \twocolumn[\@makeschapterhead{\indexname}]%
                \@mkboth{\uppercase{\indexname}}%
                        {\uppercase{\indexname}}%
                \thispagestyle{plain}\parindent\z@
                \parskip\z@ \@plus .3\p@\relax
                \let\item\@idxitem}
               {\if@restonecol\onecolumn\else\clearpage\fi}
\newcommand{\@idxitem}  {\par\hangindent 40\p@}
\newcommand{\subitem}   {\par\hangindent 40\p@ \hspace*{20\p@}}
\newcommand{\subsubitem}{\par\hangindent 40\p@ \hspace*{30\p@}}
\newcommand{\indexspace}{\par \vskip 10\p@ \@plus5\p@ \@minus3\p@\relax}
\renewcommand{\footnoterule}{%
  \kern-3\p@
  \hrule width .4\columnwidth
  \kern 2.6\p@}
\@addtoreset{footnote}{chapter}
\long\def\@makefntext#1{%
    \parindent 1em%
    \noindent
    \hbox to 1.8em{\hss\@makefnmark}#1}
\newcommand{\contentsname}{Contents}
\newcommand{\listfigurename}{List of Figures}
\newcommand{\listtablename}{List of Tables}
\newcommand{\bibname}{Bibliography}
\newcommand{\indexname}{Index}
\newcommand{\figurename}{Figure}
\newcommand{\tablename}{Table}
\newcommand{\partname}{Part}
\newcommand{\chaptername}{Chapter}
\newcommand{\appendixname}{Appendix}
\newcommand{\today}{\ifcase\month\or
  January\or February\or March\or April\or May\or June\or
  July\or August\or September\or October\or November\or December\fi
  \space\number\day, \number\year}
\setlength\columnsep{10\p@}
\setlength\columnseprule{0\p@}
\pagestyle{headings}
\pagenumbering{arabic}      % Arabic page numbers
\if@twoside
\else
  \raggedbottom
\fi
\if@twocolumn
  \twocolumn
  \sloppy
  \flushbottom
\else
  \onecolumn
\fi
\endinput
%% 
%% End of file `book.cls'.

\renewcommand\maketitle{\par
  \begingroup
    \renewcommand\thefootnote{\fnsymbol{footnote}}%
    \def\@makefnmark{%
      \hbox to\z@{%
        $\m@th^{%
           \hyperreference{%
              fn.\hyper@minipage@label.\@thefnmark.\hyper@thepage%
           }%
           {\@thefnmark}%
         }$\hss}%
    }%
    \long\def\@makefntext##1{%
      \bgroup%
        \edef\@currenthyper{%
           fn.\hyper@minipage@label.\@thefnmark.\hyper@thepage%
        }%
        \parindent 1em%
        \noindent%
        \hbox to1.8em%
          {\hss$\m@th^{\hyperanchor{\@currenthyper}{\@thefnmark}}$}##1%
       \egroup%
    }%
    \if@twocolumn
      \ifnum \col@number=\@ne
        \@maketitle
      \else
        \twocolumn[\@maketitle]%
      \fi
    \else
      \newpage
      \global\@topnum\z@
      \@maketitle
    \fi
    \thispagestyle{plain}\@thanks
  \endgroup
  \setcounter{footnote}{0}%
  \let\thanks\relax
  \let\maketitle\relax\let\@maketitle\relax
  \gdef\@thanks{}\gdef\@author{}\gdef\@title{}}
%</amsdtx>
%    \end{macrocode}
% For |amsldoc.cls| we need the following changes.
%    \begin{macrocode}
%<*amsldoc>
%% 
%% This is file `book.cls', generated 
%% on <1994/12/18> with the docstrip utility (2.2i).
%% 
%% The original source files were:
%% 
%% classes.dtx  (with options: `book')
%% 
%% Copyright 1994 the LaTeX3 project and the individual authors. 
%% All rights reserved. For further copyright information see the file 
%% legal.txt, and any other copyright indicated in this file. 
%% 
%% This file is part of the LaTeX2e system. 
%% ---------------------------------------- 
%% 
%% This system is distributed in the hope that it will be useful, 
%% but WITHOUT ANY WARRANTY; without even the implied warranty of 
%% MERCHANTABILITY or FITNESS FOR A PARTICULAR PURPOSE. 
%% 
%% 
%% IMPORTANT NOTICE: 
%% 
%% For error reports in case of UNCHANGED versions see bugs.txt. 
%% 
%% Please do not request updates from us directly. Distribution is 
%% done through Mail-Servers and TeX organizations. 
%% 
%% You are not allowed to change this file. 
%% 
%% You are allowed to distribute this file under the condition that 
%% it is distributed together with all files mentioned in manifest.txt. 
%% 
%% If you receive only some of these files from someone, complain! 
%% 
%% You are NOT ALLOWED to distribute this file alone. You are NOT 
%% ALLOWED to take money for the distribution or use of either this 
%% file or a changed version, except for a nominal charge for copying 
%% etc. 
%% \CharacterTable
%%  {Upper-case    \A\B\C\D\E\F\G\H\I\J\K\L\M\N\O\P\Q\R\S\T\U\V\W\X\Y\Z
%%   Lower-case    \a\b\c\d\e\f\g\h\i\j\k\l\m\n\o\p\q\r\s\t\u\v\w\x\y\z
%%   Digits        \0\1\2\3\4\5\6\7\8\9
%%   Exclamation   \!     Double quote  \"     Hash (number) \#
%%   Dollar        \$     Percent       \%     Ampersand     \&
%%   Acute accent  \'     Left paren    \(     Right paren   \)
%%   Asterisk      \*     Plus          \+     Comma         \,
%%   Minus         \-     Point         \.     Solidus       \/
%%   Colon         \:     Semicolon     \;     Less than     \<
%%   Equals        \=     Greater than  \>     Question mark \?
%%   Commercial at \@     Left bracket  \[     Backslash     \\
%%   Right bracket \]     Circumflex    \^     Underscore    \_
%%   Grave accent  \`     Left brace    \{     Vertical bar  \|
%%   Right brace   \}     Tilde         \~}
\NeedsTeXFormat{LaTeX2e}
\ProvidesClass{book}
              [1994/12/09 v1.2x
 Standard LaTeX document class]
\newcommand{\@ptsize}{}
\newif\if@restonecol
\newif\if@titlepage
\@titlepagetrue
\newif\if@openright
\newif\if@openbib
\@openbibfalse
\newif\if@mainmatter \@mainmattertrue
\DeclareOption{a4paper}
   {\setlength\paperheight {297mm}%
    \setlength\paperwidth  {210mm}}
\DeclareOption{a5paper}
   {\setlength\paperheight {210mm}%
    \setlength\paperwidth  {148mm}}
\DeclareOption{b5paper}
   {\setlength\paperheight {250mm}%
    \setlength\paperwidth  {176mm}}
\DeclareOption{letterpaper}
   {\setlength\paperheight {11in}%
    \setlength\paperwidth  {8.5in}}
\DeclareOption{legalpaper}
   {\setlength\paperheight {14in}%
    \setlength\paperwidth  {8.5in}}
\DeclareOption{executivepaper}
   {\setlength\paperheight {10.5in}%
    \setlength\paperwidth  {7.25in}}
\DeclareOption{landscape}
   {\setlength\@tempdima   {\paperheight}%
    \setlength\paperheight {\paperwidth}%
    \setlength\paperwidth  {\@tempdima}}
\DeclareOption{10pt}{\renewcommand{\@ptsize}{0}}
\DeclareOption{11pt}{\renewcommand{\@ptsize}{1}}
\DeclareOption{12pt}{\renewcommand{\@ptsize}{2}}
\DeclareOption{oneside}{\@twosidefalse \@mparswitchfalse}
\DeclareOption{twoside}{\@twosidetrue  \@mparswitchtrue}
\DeclareOption{draft}{\setlength\overfullrule{5pt}}
\DeclareOption{final}{\setlength\overfullrule{0pt}}
\DeclareOption{titlepage}{\@titlepagetrue}
\DeclareOption{notitlepage}{\@titlepagefalse}
\DeclareOption{openright}{\@openrighttrue}
\DeclareOption{openany}{\@openrightfalse}
\DeclareOption{onecolumn}{\@twocolumnfalse}
\DeclareOption{twocolumn}{\@twocolumntrue}
\DeclareOption{leqno}{% leqno.sty 14 Jun 85
\def\@eqnnum{\hbox to .01pt{}\rlap{\rm \hskip -\displaywidth(\theequation)}}
}
\DeclareOption{fleqn}{% fleqn.sty 26-Sep-89


\def\[{\relax\ifmmode\@badmath\else
 \begin{trivlist}
 \@beginparpenalty\predisplaypenalty
 \@endparpenalty\postdisplaypenalty
 \item[]\leavevmode
 \hbox to\linewidth\bgroup$ \displaystyle
 \hskip\mathindent\bgroup\fi}
\def\]{\relax\ifmmode \egroup $\hfil \egroup \end{trivlist}\else \@badmath \fi}
\def\equation{\@beginparpenalty\predisplaypenalty
 \@endparpenalty\postdisplaypenalty
\refstepcounter{equation}\trivlist \item[]\leavevmode
 \hbox to\linewidth\bgroup $ \displaystyle
\hskip\mathindent}
\def\endequation{$\hfil \displaywidth\linewidth\@eqnnum\egroup \endtrivlist}
\def\eqnarray{\stepcounter{equation}\let\@currentlabel=\theequation
\global\@eqnswtrue
\global\@eqcnt\z@\tabskip\mathindent\let\\=\@eqncr
\abovedisplayskip\topsep\ifvmode\advance\abovedisplayskip\partopsep\fi
\belowdisplayskip\abovedisplayskip
\belowdisplayshortskip\abovedisplayskip
\abovedisplayshortskip\abovedisplayskip
$$\halign to
\linewidth\bgroup\@eqnsel\hskip\@centering$\displaystyle\tabskip\z@
 {##}$&\global\@eqcnt\@ne \hskip 2\arraycolsep \hfil${##}$\hfil
 &\global\@eqcnt\tw@ \hskip 2\arraycolsep $\displaystyle{##}$\hfil 
 \tabskip\@centering&\llap{##}\tabskip\z@\cr}
\def\endeqnarray{\@@eqncr\egroup
 \global\advance\c@equation\m@ne$$\global\@ignoretrue }
\newdimen\mathindent 
\mathindent = \leftmargini

}
\DeclareOption{openbib}{\@openbibtrue}
\ExecuteOptions{letterpaper,10pt,twoside,onecolumn,final,openright}
\ProcessOptions
\input{bk1\@ptsize.clo}
\setlength\lineskip{1\p@}
\setlength\normallineskip{1\p@}
\renewcommand{\baselinestretch}{}
\setlength\parskip{0\p@ \@plus \p@}
\@lowpenalty   51
\@medpenalty  151
\@highpenalty 301
\setcounter{topnumber}{2}
\renewcommand{\topfraction}{.7}
\setcounter{bottomnumber}{1}
\renewcommand{\bottomfraction}{.3}
\setcounter{totalnumber}{3}
\renewcommand{\textfraction}{.2}
\renewcommand{\floatpagefraction}{.5}
\setcounter{dbltopnumber}{2}
\renewcommand{\dbltopfraction}{.7}
\renewcommand{\dblfloatpagefraction}{.5}
\if@twoside
  \def\ps@headings{%
      \let\@oddfoot\@empty\let\@evenfoot\@empty
      \def\@evenhead{\thepage\hfil\slshape\leftmark}%
      \def\@oddhead{{\slshape\rightmark}\hfil\thepage}%
      \let\@mkboth\markboth
    \def\chaptermark##1{%
      \markboth {\uppercase{%
        \ifnum \c@secnumdepth >\m@ne
          \if@mainmatter
            \@chapapp\ \thechapter. \ %
          \fi
        \fi
        ##1}}{}}%
    \def\sectionmark##1{%
      \markright {\uppercase{%
        \ifnum \c@secnumdepth >\z@
          \thesection. \ %
        \fi
        ##1}}}}
\else
  \def\ps@headings{%
    \let\@oddfoot\@empty
    \def\@oddhead{{\slshape\rightmark}\hfil\thepage}%
    \let\@mkboth\markboth
    \def\chaptermark##1{%
      \markright {\uppercase{%
        \ifnum \c@secnumdepth >\m@ne
          \if@mainmatter
            \@chapapp\ \thechapter. \ %
          \fi
        \fi
        ##1}}}}
\fi
\def\ps@myheadings{%
    \let\@oddfoot\@empty\let\@evenfoot\@empty
    \def\@evenhead{\thepage\hfil\slshape\leftmark}%
    \def\@oddhead{{\slshape\rightmark}\hfil\thepage}%
    \let\@mkboth\@gobbletwo
    \let\chaptermark\@gobble
    \let\sectionmark\@gobble
    }
\if@titlepage
  \newcommand{\maketitle}{\begin{titlepage}%
  \let\footnotesize\small
  \let\footnoterule\relax
  \null\vfil
  \vskip 60\p@
  \begin{center}%
    {\LARGE \@title \par}%
    \vskip 3em%
    {\large
     \lineskip .75em%
      \begin{tabular}[t]{c}%
        \@author
      \end{tabular}\par}%
      \vskip 1.5em%
    {\large \@date \par}%       % Set date in \large size.
  \end{center}\par
  \@thanks
  \vfil\null
  \end{titlepage}%
  \setcounter{footnote}{0}%
  \let\thanks\relax\let\maketitle\relax
  \gdef\@thanks{}\gdef\@author{}\gdef\@title{}}
\else
\newcommand{\maketitle}{\par
  \begingroup
    \renewcommand{\thefootnote}{\fnsymbol{footnote}}%
    \def\@makefnmark{\hbox to\z@{$\m@th^{\@thefnmark}$\hss}}%
    \long\def\@makefntext##1{\parindent 1em\noindent
            \hbox to1.8em{\hss$\m@th^{\@thefnmark}$}##1}%
    \if@twocolumn
      \ifnum \col@number=\@ne
        \@maketitle
      \else
        \twocolumn[\@maketitle]%
      \fi
    \else
      \newpage
      \global\@topnum\z@   % Prevents figures from going at top of page.
      \@maketitle
    \fi
    \thispagestyle{plain}\@thanks
  \endgroup
  \setcounter{footnote}{0}%
  \let\thanks\relax
  \let\maketitle\relax\let\@maketitle\relax
  \gdef\@thanks{}\gdef\@author{}\gdef\@title{}}
\def\@maketitle{%
  \newpage
  \null
  \vskip 2em%
  \begin{center}%
    {\LARGE \@title \par}%
    \vskip 1.5em%
    {\large
      \lineskip .5em%
      \begin{tabular}[t]{c}%
        \@author
      \end{tabular}\par}%
    \vskip 1em%
    {\large \@date}%
  \end{center}%
  \par
  \vskip 1.5em}
\fi
\newcommand*{\chaptermark}[1]{}
\setcounter{secnumdepth}{2}
\newcounter {part}
\newcounter {chapter}
\newcounter {section}[chapter]
\newcounter {subsection}[section]
\newcounter {subsubsection}[subsection]
\newcounter {paragraph}[subsubsection]
\newcounter {subparagraph}[paragraph]
\renewcommand{\thepart}         {\Roman{part}}
\renewcommand{\thechapter}      {\arabic{chapter}}
\renewcommand{\thesection}      {\thechapter.\arabic{section}}
\renewcommand{\thesubsection}   {\thesection.\arabic{subsection}}
\renewcommand{\thesubsubsection}{\thesubsection .\arabic{subsubsection}}
\renewcommand{\theparagraph}    {\thesubsubsection.\arabic{paragraph}}
\renewcommand{\thesubparagraph} {\theparagraph.\arabic{subparagraph}}
\newcommand{\@chapapp}{\chaptername}
\newcommand{\frontmatter}{\cleardoublepage
            \@mainmatterfalse\pagenumbering{roman}}
\newcommand{\mainmatter}{\cleardoublepage
       \@mainmattertrue\pagenumbering{arabic}}
\newcommand{\backmatter}{\if@openright\cleardoublepage\else\clearpage\fi
      \@mainmatterfalse}
\newcommand{\part}{\cleardoublepage
                 \thispagestyle{plain}%
                 \if@twocolumn
                     \onecolumn
                     \@tempswatrue
                   \else
                     \@tempswafalse
                 \fi
                 \hbox{}\vfil
                 \secdef\@part\@spart}

\def\@part[#1]#2{%
    \ifnum \c@secnumdepth >-2\relax
      \refstepcounter{part}%
      \addcontentsline{toc}{part}{\thepart\hspace{1em}#1}%
    \else
      \addcontentsline{toc}{part}{#1}%
    \fi
    \markboth{}{}
    {\centering
     \interlinepenalty \@M
     \reset@font
     \ifnum \c@secnumdepth >-2\relax
       \huge\bfseries \partname~\thepart
       \par
       \vskip 20\p@
     \fi
     \Huge \bfseries #2\par}%
    \@endpart}
\def\@spart#1{%
    {\centering
     \interlinepenalty \@M
     \reset@font
     \Huge \bfseries #1\par}%
    \@endpart}
\def\@endpart{\vfil\newpage
              \if@twoside
                \hbox{}%
                \thispagestyle{empty}%
                \newpage
              \fi
              \if@tempswa
                \twocolumn
              \fi}
\newcommand{\chapter}{\if@openright\cleardoublepage\else\clearpage\fi
                    \thispagestyle{plain}%
                    \global\@topnum\z@
                    \@afterindentfalse
                    \secdef\@chapter\@schapter}
\def\@chapter[#1]#2{\ifnum \c@secnumdepth >\m@ne
                       \if@mainmatter
                         \refstepcounter{chapter}%
                         \typeout{\@chapapp\space\thechapter.}%
                         \addcontentsline{toc}{chapter}%
                                   {\protect\numberline{\thechapter}#1}%
                       \else
                         \addcontentsline{toc}{chapter}{#1}%
                       \fi
                    \else
                      \addcontentsline{toc}{chapter}{#1}%
                    \fi
                    \chaptermark{#1}%
                    \addtocontents{lof}{\protect\addvspace{10\p@}}%
                    \addtocontents{lot}{\protect\addvspace{10\p@}}%
                    \if@twocolumn
                      \@topnewpage[\@makechapterhead{#2}]%
                    \else
                      \@makechapterhead{#2}%
                      \@afterheading
                    \fi}
\def\@makechapterhead#1{%
  \vspace*{50\p@}%
  {\parindent \z@ \raggedright \reset@font
    \ifnum \c@secnumdepth >\m@ne
      \if@mainmatter
        \huge\bfseries \@chapapp{} \thechapter
        \par\nobreak
        \vskip 20\p@
      \fi
    \fi
    \interlinepenalty\@M
    \Huge \bfseries #1\par\nobreak
    \vskip 40\p@
  }}
\def\@schapter#1{\if@twocolumn
                   \@topnewpage[\@makeschapterhead{#1}]%
                 \else
                   \@makeschapterhead{#1}%
                   \@afterheading
                 \fi}
\def\@makeschapterhead#1{%
  \vspace*{50\p@}%
  {\parindent \z@ \raggedright
    \reset@font
    \interlinepenalty\@M
    \Huge \bfseries  #1\par\nobreak
    \vskip 40\p@
  }}
\newcommand{\section}{\@startsection {section}{1}{\z@}%
                                   {-3.5ex \@plus -1ex \@minus -.2ex}%
                                   {2.3ex \@plus.2ex}%
                                   {\reset@font\Large\bfseries}}
\newcommand{\subsection}{\@startsection{subsection}{2}{\z@}%
                                     {-3.25ex\@plus -1ex \@minus -.2ex}%
                                     {1.5ex \@plus .2ex}%
                                     {\reset@font\large\bfseries}}
\newcommand{\subsubsection}{\@startsection{subsubsection}{3}{\z@}%
                                     {-3.25ex\@plus -1ex \@minus -.2ex}%
                                     {1.5ex \@plus .2ex}%
                                     {\reset@font\normalsize\bfseries}}
\newcommand{\paragraph}{\@startsection{paragraph}{4}{\z@}%
                                    {3.25ex \@plus1ex \@minus.2ex}%
                                    {-1em}%
                                    {\reset@font\normalsize\bfseries}}
\newcommand{\subparagraph}{\@startsection{subparagraph}{5}{\parindent}%
                                       {3.25ex \@plus1ex \@minus .2ex}%
                                       {-1em}%
                                      {\reset@font\normalsize\bfseries}}
\if@twocolumn
  \setlength\leftmargini  {2em}
\else
  \setlength\leftmargini  {2.5em}
\fi
\setlength\leftmarginii  {2.2em}
\setlength\leftmarginiii {1.87em}
\setlength\leftmarginiv  {1.7em}
\if@twocolumn
  \setlength\leftmarginv  {.5em}
  \setlength\leftmarginvi {.5em}
\else
  \setlength\leftmarginv  {1em}
  \setlength\leftmarginvi {1em}
\fi
\setlength\leftmargin    {\leftmargini}
\setlength  \labelsep  {.5em}
\setlength  \labelwidth{\leftmargini}
\addtolength\labelwidth{-\labelsep}
\@beginparpenalty -\@lowpenalty
\@endparpenalty   -\@lowpenalty
\@itempenalty     -\@lowpenalty
\renewcommand{\theenumi}{\arabic{enumi}}
\renewcommand{\theenumii}{\alph{enumii}}
\renewcommand{\theenumiii}{\roman{enumiii}}
\renewcommand{\theenumiv}{\Alph{enumiv}}
\newcommand{\labelenumi}{\theenumi.}
\newcommand{\labelenumii}{(\theenumii)}
\newcommand{\labelenumiii}{\theenumiii.}
\newcommand{\labelenumiv}{\theenumiv.}
\renewcommand{\p@enumii}{\theenumi}
\renewcommand{\p@enumiii}{\theenumi(\theenumii)}
\renewcommand{\p@enumiv}{\p@enumiii\theenumiii}
\newcommand{\labelitemi}{$\m@th\bullet$}
\newcommand{\labelitemii}{\normalfont\bfseries --}
\newcommand{\labelitemiii}{$\m@th\ast$}
\newcommand{\labelitemiv}{$\m@th\cdot$}
\newenvironment{description}
               {\list{}{\labelwidth\z@ \itemindent-\leftmargin
                        \let\makelabel\descriptionlabel}}
               {\endlist}
\newcommand{\descriptionlabel}[1]{\hspace\labelsep
                                \normalfont\bfseries #1}
\newenvironment{verse}
               {\let\\=\@centercr
                \list{}{\itemsep      \z@
                        \itemindent   -1.5em%
                        \listparindent\itemindent
                        \rightmargin  \leftmargin
                        \advance\leftmargin 1.5em}%
                \item[]}
               {\endlist}
\newenvironment{quotation}
               {\list{}{\listparindent 1.5em%
                        \itemindent    \listparindent
                        \rightmargin   \leftmargin
                        \parsep        \z@ \@plus\p@}%
                \item[]}
               {\endlist}
\newenvironment{quote}
               {\list{}{\rightmargin\leftmargin}%
                \item[]}
               {\endlist}
\newenvironment{titlepage}
    {%
      \cleardoublepage
      \if@twocolumn
        \@restonecoltrue\onecolumn
      \else
        \@restonecolfalse\newpage
      \fi
      \thispagestyle{empty}%
      \if@compatibility
        \setcounter{page}{0}
      \fi}%
    {\if@restonecol\twocolumn \else \newpage \fi
    }
\newcommand{\appendix}{\par
  \setcounter{chapter}{0}%
  \setcounter{section}{0}%
  \renewcommand{\@chapapp}{\appendixname}%
  \renewcommand{\thechapter}{\Alph{chapter}}}
\setlength\arraycolsep{5\p@}
\setlength\tabcolsep{6\p@}
\setlength\arrayrulewidth{.4\p@}
\setlength\doublerulesep{2\p@}
\setlength\tabbingsep{\labelsep}
\skip\@mpfootins = \skip\footins
\setlength\fboxsep{3\p@}
\setlength\fboxrule{.4\p@}
\@addtoreset{equation}{chapter}
\renewcommand{\theequation}{\thechapter.\arabic{equation}}
\newcounter{figure}[chapter]
\renewcommand{\thefigure}{\thechapter.\@arabic\c@figure}
\def\fps@figure{tbp}
\def\ftype@figure{1}
\def\ext@figure{lof}
\def\fnum@figure{\figurename~\thefigure}
\newenvironment{figure}
               {\@float{figure}}
               {\end@float}
\newenvironment{figure*}
               {\@dblfloat{figure}}
               {\end@dblfloat}
\newcounter{table}[chapter]
\renewcommand{\thetable}{\thechapter.\@arabic\c@table}
\def\fps@table{tbp}
\def\ftype@table{2}
\def\ext@table{lot}
\def\fnum@table{\tablename~\thetable}
\newenvironment{table}
               {\@float{table}}
               {\end@float}
\newenvironment{table*}
               {\@dblfloat{table}}
               {\end@dblfloat}
\newlength\abovecaptionskip
\newlength\belowcaptionskip
\setlength\abovecaptionskip{10\p@}
\setlength\belowcaptionskip{0\p@}
\long\def\@makecaption#1#2{%
  \vskip\abovecaptionskip
  \sbox\@tempboxa{#1: #2}%
  \ifdim \wd\@tempboxa >\hsize
    #1: #2\par
  \else
    \global \@minipagefalse
    \hbox to\hsize{\hfil\box\@tempboxa\hfil}%
  \fi
  \vskip\belowcaptionskip}
\DeclareOldFontCommand{\rm}{\normalfont\rmfamily}{\mathrm}
\DeclareOldFontCommand{\sf}{\normalfont\sffamily}{\mathsf}
\DeclareOldFontCommand{\tt}{\normalfont\ttfamily}{\mathtt}
\DeclareOldFontCommand{\bf}{\normalfont\bfseries}{\mathbf}
\DeclareOldFontCommand{\it}{\normalfont\itshape}{\mathit}
\DeclareOldFontCommand{\sl}{\normalfont\slshape}{\@nomath\sl}
\DeclareOldFontCommand{\sc}{\normalfont\scshape}{\@nomath\sc}
\DeclareRobustCommand*{\cal}{\@fontswitch{\relax}{\mathcal}}
\DeclareRobustCommand*{\mit}{\@fontswitch{\relax}{\mathnormal}}
\newcommand{\@pnumwidth}{1.55em}
\newcommand{\@tocrmarg} {2.55em}
\newcommand{\@dotsep}{4.5}
\setcounter{tocdepth}{2}
\newcommand{\tableofcontents}{%
    \if@twocolumn
      \@restonecoltrue\onecolumn
    \else
      \@restonecolfalse
    \fi
    \chapter*{\contentsname
        \@mkboth{\uppercase{\contentsname}}{\uppercase{\contentsname}}}%
    \@starttoc{toc}%
    \if@restonecol\twocolumn\fi
    }
\newcommand*{\l@part}[2]{%
  \ifnum \c@tocdepth >-2\relax
    \addpenalty{-\@highpenalty}%
    \addvspace{2.25em \@plus\p@}%
    \begingroup
      \setlength\@tempdima{3em}%
      \parindent \z@ \rightskip \@pnumwidth
      \parfillskip -\@pnumwidth
      {\leavevmode
       \large \bfseries #1\hfil \hbox to\@pnumwidth{\hss #2}}\par
       \nobreak
         \global\@nobreaktrue
         \everypar{\global\@nobreakfalse\everypar{}}
    \endgroup
  \fi}
\newcommand*{\l@chapter}[2]{%
  \ifnum \c@tocdepth >\m@ne
    \addpenalty{-\@highpenalty}%
    \vskip 1.0em \@plus\p@
    \setlength\@tempdima{1.5em}%
    \begingroup
      \parindent \z@ \rightskip \@pnumwidth
      \parfillskip -\@pnumwidth
      \leavevmode \bfseries
      \advance\leftskip\@tempdima
      \hskip -\leftskip
      #1\nobreak\hfil \nobreak\hbox to\@pnumwidth{\hss #2}\par
      \penalty\@highpenalty
    \endgroup
  \fi}
\newcommand*{\l@section}      {\@dottedtocline{1}{1.5em}{2.3em}}
\newcommand*{\l@subsection}   {\@dottedtocline{2}{3.8em}{3.2em}}
\newcommand*{\l@subsubsection}{\@dottedtocline{3}{7.0em}{4.1em}}
\newcommand*{\l@paragraph}    {\@dottedtocline{4}{10em}{5em}}
\newcommand*{\l@subparagraph} {\@dottedtocline{5}{12em}{6em}}
\newcommand{\listoffigures}{%
    \if@twocolumn
      \@restonecoltrue\onecolumn
    \else
      \@restonecolfalse
    \fi
    \chapter*{\listfigurename
      \@mkboth{\uppercase{\listfigurename}}%
              {\uppercase{\listfigurename}}}%
    \@starttoc{lof}%
    \if@restonecol\twocolumn\fi
    }
\newcommand*{\l@figure}{\@dottedtocline{1}{1.5em}{2.3em}}
\newcommand{\listoftables}{%
    \if@twocolumn
      \@restonecoltrue\onecolumn
    \else
      \@restonecolfalse
    \fi
    \chapter*{\listtablename
      \@mkboth{\uppercase{\listtablename}}{\uppercase{\listtablename}}}%
    \@starttoc{lot}%
    \if@restonecol\twocolumn\fi
    }
\let\l@table\l@figure
\newdimen\bibindent
\bibindent=1.5em
\newenvironment{thebibliography}[1]
     {\chapter*{\bibname
        \@mkboth{\uppercase{\bibname}}{\uppercase{\bibname}}}%
      \list{\@biblabel{\arabic{enumiv}}}%
           {\settowidth\labelwidth{\@biblabel{#1}}%
            \leftmargin\labelwidth
            \advance\leftmargin\labelsep
            \if@openbib
              \advance\leftmargin\bibindent
              \itemindent -\bibindent
              \listparindent \itemindent
              \parsep \z@
            \fi
            \usecounter{enumiv}%
            \let\p@enumiv\@empty
            \renewcommand{\theenumiv}{\arabic{enumiv}}}%
      \if@openbib
        \renewcommand{\newblock}{\par}
      \else
        \renewcommand{\newblock}{\hskip .11em \@plus.33em \@minus.07em}%
      \fi
      \sloppy\clubpenalty4000\widowpenalty4000%
      \sfcode`\.=\@m}
     {\def\@noitemerr
       {\@latex@warning{Empty `thebibliography' environment}}%
      \endlist}
\newcommand{\newblock}{}
\newenvironment{theindex}
               {\if@twocolumn
                  \@restonecolfalse
                \else
                  \@restonecoltrue
                \fi
                \columnseprule \z@
                \columnsep 35\p@
                \twocolumn[\@makeschapterhead{\indexname}]%
                \@mkboth{\uppercase{\indexname}}%
                        {\uppercase{\indexname}}%
                \thispagestyle{plain}\parindent\z@
                \parskip\z@ \@plus .3\p@\relax
                \let\item\@idxitem}
               {\if@restonecol\onecolumn\else\clearpage\fi}
\newcommand{\@idxitem}  {\par\hangindent 40\p@}
\newcommand{\subitem}   {\par\hangindent 40\p@ \hspace*{20\p@}}
\newcommand{\subsubitem}{\par\hangindent 40\p@ \hspace*{30\p@}}
\newcommand{\indexspace}{\par \vskip 10\p@ \@plus5\p@ \@minus3\p@\relax}
\renewcommand{\footnoterule}{%
  \kern-3\p@
  \hrule width .4\columnwidth
  \kern 2.6\p@}
\@addtoreset{footnote}{chapter}
\long\def\@makefntext#1{%
    \parindent 1em%
    \noindent
    \hbox to 1.8em{\hss\@makefnmark}#1}
\newcommand{\contentsname}{Contents}
\newcommand{\listfigurename}{List of Figures}
\newcommand{\listtablename}{List of Tables}
\newcommand{\bibname}{Bibliography}
\newcommand{\indexname}{Index}
\newcommand{\figurename}{Figure}
\newcommand{\tablename}{Table}
\newcommand{\partname}{Part}
\newcommand{\chaptername}{Chapter}
\newcommand{\appendixname}{Appendix}
\newcommand{\today}{\ifcase\month\or
  January\or February\or March\or April\or May\or June\or
  July\or August\or September\or October\or November\or December\fi
  \space\number\day, \number\year}
\setlength\columnsep{10\p@}
\setlength\columnseprule{0\p@}
\pagestyle{headings}
\pagenumbering{arabic}      % Arabic page numbers
\if@twoside
\else
  \raggedbottom
\fi
\if@twocolumn
  \twocolumn
  \sloppy
  \flushbottom
\else
  \onecolumn
\fi
\endinput
%% 
%% End of file `book.cls'.

\let\hyper@theindex\theindex
\def\theindex{%
  \bgroup%
    \let\hyper@makeschapterhead\@makeschapterhead%
    \def\@makeschapterhead##1{%
      \hyper@setcurrent%
      \global\let\@currenthyper\@currenthyper%
      \hyper@makeschapterhead{\hyperanchor{\@currenthyper}{##1}}%
    }
    \hyper@theindex%
  \egroup%
  \let\autoindex\@gobble%
  \let\item\@idxitem%
}
%</amsldoc>
%    \end{macrocode}
% \subsubsection{The Packages}
% Currently (may be even for ever) we do not support the
% compatibility mode |amstex|.
%    \begin{macrocode}
%<*amstex>
\errmessage{Compatibility mode amstex not supported. %
            Try native mode amsmath instead.%
	   }
%</amstex>
%    \end{macrocode}
% To add the hypertext--capability to |amsmath| some macros of it
% has to be modified.\par
% \noindent As we have redefined |\@eqnnum| and we need another
% definition for |amsmath| let us set the macro to its |amsmath|--default.
%    \begin{macrocode}
%<*amsmath>
\iftagsleft@
  \def\@eqnnum{\hbox to1sp{}\rlap{\normalfont\normalcolor
    \hskip -\displaywidth\tagform@\theequation}}
\else
  \def\@eqnnum{{\normalfont\normalcolor \tagform@\theequation}}
\fi
%    \end{macrocode}
% |amsmath| temporary redefines |\label| but needs the original definition
% that is hold in |\ltx@label|.
%    \begin{macrocode}
\let\ltx@label\label
%    \end{macrocode}
% Now let us define a blind anchor for referencing a couple of equations 
% bound together.
%    \begin{macrocode}
\let\hyper@subequations\subequations
\def\subequations{%
  \hyper@setcurrent%
  \blindhyperanchor{\@currenthyper}%
  \hyper@subequations%
}
%    \end{macrocode}
% To allow the references pointing to an equation number we have to define an
% appropriate anchor.
%    \begin{macrocode}
\let\hyper@tagform@\tagform@
\def\tagform@#1{%
  \maketag@@@{\hyperanchor{\@currenthyper}{(\ignorespaces#1\unskip)}}%
}
%    \end{macrocode}
% For user defined tags it is necessary to generate a new hypertext number.
%    \begin{macrocode}
\let\hyper@make@display@tag\make@display@tag
\def\make@display@tag{%
  \if@eqnsw\else\iftag@%
    \hyper@setcurrent%
  \fi\fi%
  \hyper@make@display@tag%
}
%    \end{macrocode}
% We should not use an anchor when using |\eqref|
%    \begin{macrocode}
\def\eqref#1{\textup{\hyper@tagform@{\ref{#1}}}}
%</amsmath>
%    \end{macrocode}
% To get hypertext references to theorems, lemmas, etc.\ |amsthm| has to
% be modified.
%    \begin{macrocode}
%<*amsthm>
\let\@begintheorem\hyper@begintheorem
\def\thmhead@plain#1#2#3{%
  \hyperanchor{\@currenthyper}{\thmname{#1}\thmnumber{ #2}}%
  \thmnote{ {\the\theoremnotefont(#3)}}%
}
\let\thmhead\thmhead@plain
\def\swappedhead#1#2#3{%
  \hyperanchor{\@currenthyper}{\thmnumber{#2}\thmname{. #1}}%
  \thmnote{ {\the\theoremnotefont(#3)}}%
}
%</amsthm>
%    \end{macrocode}
% For |upref| working correct it is sufficient to modify |\hyper@setref|.
%    \begin{macrocode}
%<*upref>
\def\hyper@setref#1#2#3#4{%
  \ifx#1\relax%
     \protect\G@refundefinedtrue%
     #4\nfss@text{\reset@font\bfseries ??}%
     \@latex@warning{Reference `#3' on page \thepage \space undefined}%
  \else%
    \ifx#2\@firstofthree%
      \hyperreference{\expandafter\@firstofthree#1}{#4}\null%
    \else\ifx#2\@secondofthree%
      \hyperreference%
        {\expandafter\@firstofthree#1}%
        {#4\protect\textup{\expandafter#2#1\hbox{}}}\null%
    \else%
      \hyperreference%
        {page.\expandafter\@thirdofthree#1}%
        {#4\protect\textup{\expandafter#2#1\hbox{}}}\null%
    \fi\fi%
  \fi%
}
%</upref>
%    \end{macrocode}
% 
% \subsection{Cross References between Documents}
% The package |xr| adds the capability of generating cross--references
% between different \LaTeXe--documents. It is a nice feature to support
% this by hypertext links. Thus we have to modify |xr| appropriately.\newline
% At first let us provide an additional parameter to |\externaldocument|
% giving the URL--address of the document we use references to. It has to
% be given with surrounding |(||)|. The default value is given by
% |documentname.dvi|.
%    \begin{macrocode}
%<*xr>
\def\externaldocument{%
  \@ifnextchar(%
    {\hyper@externaldocument}%
    {\hyper@externaldocument()}%
}
\def\hyper@externaldocument(#1){%
  \@ifnextchar[%
    {\hyper@@externaldocument(#1)}%
    {\hyper@@externaldocument(#1)[]}%
}
\def\hyper@@externaldocument(#1)[#2]#3{%
  \def\hyper@XR@docprefix{#1}%
  \ifx\hyper@XR@docprefix\@empty%
    \def\hyper@XR@docprefix{#3.dvi}%
  \fi%
  \XR@[#2]{#3}%
}
%    \end{macrocode}
% Now we have to modify the anchors we refer to from their local definition
% in the documents |.aux|--file to its external URL--definitions.
%    \begin{macrocode}
\def\hyper@xr@newlabel#1#2#3#4{%
    \newlabel%
      {\XR@prefix#4}%
      {{[file:\hyper@XR@docprefix][#1]}{#2}{[hyper@XR@docprefix][#3]}}%
}
\long\def\XR@test#1#2#3#4\XR@{%
  \ifx#1\newlabel%
    \expandafter\hyper@xr@newlabel#3{#2}%
  \else\ifx#1\@input%
     \edef\XR@list{\XR@list#2\relax}%
  \fi\fi%
  \ifeof%
    \@inputcheck\expandafter\XR@aux%
  \else%
    \expandafter\XR@read%
  \fi%
}
%</xr>
%    \end{macrocode}
%
% \subsection{Long Tables}
% The package |longtable| uses a definition of |\caption| for it's own.
% Thus we have to modify this to reflect the hypertext needs.
%    \begin{macrocode}
%<*longtable>
\let\hyper@LT@array\LT@array
\def\LT@array{%
  \refstepcounter{table}\ignore@next@refstepcounter%
  \def\@currenthyper{\the\hypercount}%
  \hyper@LT@array%
}
\def\LT@caption{%
  \noalign\bgroup\hyper@setcurrent%
  \@ifnextchar[{\egroup\LT@c@ption}{\LT@capti@n}%
}
\def\LT@c@ption[#1]#2{%
  \LT@mkcaption{\hyperanchor{\@currenthyper}{\fnum@table}: #2}%
  \def\@tempa{#1}\ifx\@tempa\empty\else%
       {\let\\\space%
          \addcontentsline{lot}{table}%
            {\protect\numberline{\thetable}%
               {\hyperreference{\@currenthyper}{#1}}%
            }%
      }%
  \fi%
}
\def\LT@@c@ption#1{%
  \LT@mkcaption{\hyperanchor{\@currenthyper}{\fnum@table}: #1}%
  {\let\\\space
     \addcontentsline{lot}{table}%
       {\protect\numberline{\thetable}%
          {\hyperreference{\@currenthyper}{#1}}%
       }%
  }%
}
%</longtable>
%    \end{macrocode}
%
% \subsection{Right Column Footnotes in Double Column Documents}
% |ftnright.sty| redefines the macro |\@makefntext|. Thus we have to 
% modify it's definition when needed.
%    \begin{macrocode}
%<*ftnright>
\long\def\@makefntext#1{%
  \bgroup%
    \edef\@currenthyper{%
       fn.\hyper@minipage@label.\@thefnmark.\hyper@thepage%
    }%
    \parindent 1em%
    \noindent\hbox to 2em{}%
    \llap{${\hyperanchor{\@currenthyper}{\@thefnmark}}.\;\;$}#1%
  \egroup%
}
%</ftnright>
%    \end{macrocode}
%
% \subsection{The Theorem Package}
% Analoguous to the theorem environment of \LaTeXe{} we have to modify
% the different theorem styles of the theorem package.
%    \begin{macrocode}
%<*theorem>
%plain
\gdef\th@plain{\normalfont\itshape%
  \def\@begintheorem##1##2{%
     \item[\hskip\labelsep%
           \hyperanchor{\@currenthyper}{\theorem@headerfont ##1\ ##2}%
          ]%
  }%
  \def\@opargbegintheorem##1##2##3{%
     \item[\hskip\labelsep%
           \hyperanchor%
             {\@currenthyper}%
             {\theorem@headerfont ##1\ ##2}\ (##3)%
          ]%
  }%
}
%break
\gdef\th@break{\normalfont\slshape%
  \def\@begintheorem##1##2{%
     \item[\rlap{\vbox{\hbox{\hskip \labelsep%
             \hyperanchor{\@currenthyper}{\theorem@headerfont ##1\ ##2}%
	   }\hbox{\strut}}}%
          ]%
  }%
  \def\@opargbegintheorem##1##2##3{%
    \item[\rlap{\vbox{\hbox{\hskip \labelsep%
            \hyperanchor%
              {\@currenthyper}%
              {\theorem@headerfont ##1\ ##2}\ (##3)%
          }\hbox{\strut}}}%
         ]%
  }%
}
%marginbreak
\gdef\th@marginbreak{\normalfont\slshape%
  \def\@begintheorem##1##2{%
    \item[\rlap{\vbox{\theorem@headerfont%
            \hbox{\llap{##2}\hskip\labelsep%
                  \hyperanchor{\@currenthyper}{##1}%
                 }%
          \hbox{\strut}}}%
         ]%
  }%
  \def\@opargbegintheorem##1##2##3{%
    \item[\rlap{\vbox{\theorem@headerfont%
            \hbox{\llap{##2}\hskip\labelsep%
                  \hyperanchor{\@currenthyper}{##1}\ (##3)
                 }%
          \hbox{\strut}}}%
         ]%
  }%
}
%changebreak
\gdef\th@changebreak{\normalfont\slshape%
  \def\@begintheorem##1##2{%
    \item[\rlap{\vbox{\hbox{\hskip\labelsep%
            \hyperanchor{\@currenthyper}{\theorem@headerfont ##2\ ##1}%
          }\hbox{\strut}}}%
         ]%
  }%
  \def\@opargbegintheorem##1##2##3{%
    \item[\rlap{\vbox{\hbox{\hskip\labelsep%
            \hyperanchor%
              {\@currenthyper}%
              {\theorem@headerfont ##2\ ##1}\ (##3)%
          }\hbox{\strut}}}%
         ]%
  }%
}
%change
\gdef\th@change{\normalfont\slshape%
  \def\@begintheorem##1##2{%
    \item[\hskip\labelsep%
          \hyperanchor{\@currenthyper}{\theorem@headerfont ##2\ ##1}%
         ]%
  }%
  \def\@opargbegintheorem##1##2##3{%
    \item[\hskip\labelsep%
          \hyperanchor%
            {\@currenthyper}%
            {\theorem@headerfont ##2\ ##1}\ (##3)%
         ]%
  }%
}
%margin
\gdef\th@margin{\normalfont\slshape%
  \def\@begintheorem##1##2{%
    \item[\theorem@headerfont \llap{##2}\hskip\labelsep%
          \hyperanchor{\@currenthyper}{##1}%
         ]%
  }%
  \def\@opargbegintheorem##1##2##3{%
    \item[\theorem@headerfont \llap{##2}\hskip\labelsep%
          \hyperanchor{\@currenthyper}{##1}\ (##3)%
         ]%
  }%
}
%</theorem>
%    \end{macrocode}
%
% \clearpage
% \section{Needed Packages (Part of this Distribution)}
%
% \subsection{Pattern--Like Definition of Macros}
% For the index and glossary definitions we have used the package |defpattern| 
% that is already part of the fp--package on |CTAN:macros/latex/contrib/other/fp|.
% This is a very usefull macro allowing the definition of other macros by
% pattern matching rules.\par
% \noindent The package has been adapted from Alan Jeffrey's |\howto| that can be
% found in ``\TeX and TUG NEWS 2(2)''.
%    \begin{macrocode}
%<*defpattern>
\ProvidesPackage{defpattern}
\message{%
  `\string\defpattern-macro %
   (adapted from \string\howto\space%
    (Alan Jeffrey in TeX and TUG NEWS 2(2))%
   )%
  '%
}
\def\xcs#1#2{\expandafter#1\csname#2\endcsname}
\newcount\actioncount
\def\defpattern#1[#2]#3{%
  \xcs\ifx{\expandafter\gobble\string#1}\relax
  \def#1{\erroraction#1}\fi
  \advance\actioncount by 1\relax
  \expandafter\defaction\expandafter{#1}{#3}{#2}%
  \def\temp##1##2##3##4##5##6##7##8##9{\def\temp{[#2]}}%
  \temp\empty\empty\empty\empty\empty\empty\empty\empty\empty
  \edef#1{\noexpand\doaction\xcs\noexpand
    {action-\the\actioncount}\temp}}
\def\defaction#1#2#3{%
  \xcs\def{action-\the\actioncount}##1[#3]{\applyto{#1}{#2}}}
\def\gobble#1{}
\def\applyto#1#2#3{#3{#1}{#2}}
\def\doaction#1[#2][#3]{#1[#3]\success[#2]\failure[#3]}
\def\success#1#2[#3]\failure[#4]{#2}
\def\failure#1#2{#1}
\def\erroraction#1[#2]{%
   \errmessage{I don't know how to \string#1[#2]}}
%</defpattern>
%    \end{macrocode}
%
% \Finale
