% \iffalse
%% File: printtim.dtx Copyright (C) 1995 Martin Schr\"oder
%
%<package>\NeedsTeXFormat{LaTeX2e}
%<package>\ProvidesPackage{printtim}
%<package>         [1995/01/24 v1.00 PrintTime Package (MS)]
%
%<*driver>
\documentclass{ltxdoc}
\usepackage{printtim}
\GetFileInfo{printtim.sty}
\setcounter{IndexColumns}{2}
\EnableCrossrefs
\CodelineIndex
\RecordChanges
\setcounter{IndexColumns}{2}
\setlength{\IndexMin}{30ex}
\setlength{\columnseprule}{.4pt}
\AtBeginDocument{\addtocontents{toc}{\protect\begin{multicols}{2}}}
\AtEndDocument{\addtocontents{toc}{\protect\end{multicols}}}
\begin{document}
\DocInput{printtim.dtx}
\end{document}
%</driver>
%
% Copyright (C) 1995 by Martin Schr\"oder.  All rights reserved.
%
% IMPORTANT NOTICE:
%
% You are not allowed to change this file.  You may however copy
% this file to a file with a different name and then change the
% copy if you obey the restrictions on file changes described in
% everyshi.ins.
%
% You are NOT ALLOWED to distribute this file alone.  You are NOT
% ALLOWED to take money for the distribution or use of this file
% (or a changed version) except for a nominal charge for copying
% etc.
%
% You are allowed to distribute this file under the condition that
% it is distributed together with all files mentioned in
% printtim.ins.
%
% If you receive only some of these files from someone, complain!
%
% However, if these files are distributed by established suppliers
% as part of a complete TeX distribution, and the structure of the
% distribution would make it difficult to distribute the whole set
% of files, *those parties* are allowed to distribute only some of
% the files provided that it is made clear that the user will get
% a complete distribution-set upon request to that supplier (not
% me).  Notice that this permission is not granted to the end
% user.
%
%
% For error reports in case of UNCHANGED versions see everyshi.ins
%
% \fi
%
% \CheckSum{30}
%
%% \CharacterTable
%% {Upper-case    \A\B\C\D\E\F\G\H\I\J\K\L\M\N\O\P\Q\R\S\T\U\V\W\X\Y\Z
%%  Lower-case    \a\b\c\d\e\f\g\h\i\j\k\l\m\n\o\p\q\r\s\t\u\v\w\x\y\z
%%  Digits        \0\1\2\3\4\5\6\7\8\9
%%  Exclamation   \!     Double quote  \"     Hash (number) \#
%%  Dollar        \$     Percent       \%     Ampersand     \&
%%  Acute accent  \'     Left paren    \(     Right paren   \)
%%  Asterisk      \*     Plus          \+     Comma         \,
%%  Minus         \-     Point         \.     Solidus       \/
%%  Colon         \:     Semicolon     \;     Less than     \<
%%  Equals        \=     Greater than  \>     Question mark \?
%%  Commercial at \@     Left bracket  \[     Backslash     \\
%%  Right bracket \]     Circumflex    \^     Underscore    \_
%%  Grave accent  \`     Left brace    \{     Vertical bar  \|
%%  Right brace   \}     Tilde         \~}
%%
%% \iffalse meta-comment
%% ===================================================================
%%  @LaTeX-package-file{
%%     author          = {Martin Schr\"oder},
%%     version         = "1.00",
%%     date            = "24 January 1995",
%%     filename        = "printtim.sty",
%%     address         = {Martin Schr\"oder
%%                        Friedrich-Humbert-Stra\ss{}e 124
%%                        D-28759 Bremen
%%     telephone       = "+49-421-628813",
%%     email           = "MS@Dream.HB.North.DE (INTERNET)",
%%     codetable       = "ISO/ASCII",
%%     keywords        = "LaTeX2e, \time",
%%     supported       = "yes",
%%     docstring       = "LaTeX package which defines a command
%%                        \PrintTime which prints the current time".
%%  }
%% ===================================================================
%% \fi
%
%  \changes{v1.00}{1995/01/24}{New}
%
%  \IndexPrologue{^^A
%     \section*{\indexname}^^A
%     \markboth{\indexname}{\indexname}^^A
%     Numbers written in \emph{italic} refer to the page where the
%     corresponding entry is described, the ones
%     \underline{underlined} to the definition, the rest to the places
%     where the entry is used.}
%
% ^^A -----------------------------
%
%  \title{\unskip
%           The \textsf{printtim} package^^A
%           \thanks{^^A
%              The version umber of this file is \fileversion,
%              last revised \filedate.\newline
%              The name \textsf{printtim} is a tribute to the $8+3$
%              file-naming convention of certain ``operating
%              systems''; strictly speaking it should be 
%              \textsf{printtime}.}^^A
%        }
%  \author{Martin Schr\"oder\\[0.5ex]
%          \normalsize  Friedrich-Humbert-Stra\ss{}e 124\\
%          \normalsize  D-28759 Bremen\\
%          \normalsize  MS@Dream.HB.North.DE (INTERNET)}
%  \date{\filedate}
%  \maketitle
%
% ^^A -----------------------------
%
%
%  \begin{abstract}
%     This package defines a new command \cs{PrintTime} which prints
%     out the time the document was processed.
%  \end{abstract}
%
%  \pagestyle{headings}
%
% ^^A -----------------------------
%
%  \tableofcontents
%
% ^^A -----------------------------
%
%  \section{Introduction}
%
%  This package provides the command \cs{PrintTime} which prints out
%  the time the document was processed.
%
% ^^A -----------------------------
%
%  \section{Usage}
%
%  \cs{PrintTime} prints out the current time: ``\PrintTime''.
%
% ^^A -----------------------------
%
%  \section{Options}
%
%  The package has no options.
%
% ^^A -----------------------------
%
%  \section{Required packages}
%
%  The package does not require any further packages.
%
% ^^A -----------------------------
%
%  \StopEventually{^^A
%     \PrintIndex\PrintChanges
%     ^^A Make sure that the index is not printed twice
%     ^^A (ltxdoc.cfg might have a second \PrintIndex command)
%     \let\PrintChanges\relax
%     \let\PrintIndex\relax
%     }
%
% ^^A -----------------------------
%
%  \section{The implementation}
%
%    \begin{macrocode}
%<*package>
%    \end{macrocode}
%
%  \begin{macro}{\PrintTime}
%  \cs{PrintTime} calculates the current hours and minutes out of 
%  \cs{time}, which is set by \TeX{} at the start of each job and
%  contains the minutes since midnight.
%  We use these formulas:
%  \begin{eqnarray*}
%     hours    & = & \cs{time}/60 \\
%     minutes  & = & (hour \times -60) - \cs{time}
%  \end{eqnarray*}
%  Since we don't want to allocate a new counter, we use 
%  \cs{count255}.
%    \begin{macrocode}
\newcommand*{\PrintTime}{%
%    \end{macrocode}
%  \cs{PrintTime} is \cs{short}, since it has no parameter and contains
%  no \cs{par}.
%    \begin{macrocode}
   \bgroup%
      \count\@cclv=\time
      \divide  \count\@cclv by 60\relax
%    \end{macrocode}
%  Now we have the hour.
%    \begin{macrocode}
      \the\count\@cclv%
      :%
      \multiply\count\@cclv by -60\relax
      \advance \count\@cclv by \time
%    \end{macrocode}
%  If the minutes are $<10$ we output a ``0'' to make the minutes a
%  two-digit number.
%    \begin{macrocode}
      \ifnum\count\@cclv<10\relax 0\fi
      \the\count\@cclv%
   \egroup%
   }
%    \end{macrocode}
%  \end{macro}
%
%    \begin{macrocode}
%</package>
%    \end{macrocode}
%
% ^^A -----------------------------
%
%  \section{Acknowledgements}
%
%  As usual Rebecca Stiels improved the quality of this documentation.
%  If you need a translator from English or Fran\c{c}ais to German, send
%  her an e-mail to \texttt{Rebecca@Andurg.HB.North.DE}.
%  And if you need a \TeX{}nician or computer scientist, send me an e-mail;
%  I'm looking for a job.
%
% ^^A -----------------------------
%
%  \Finale
