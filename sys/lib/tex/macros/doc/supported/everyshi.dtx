% \iffalse
%% File: everyshi.dtx Copyright (C) 1994 Martin Schr\"oder
%
%<package>\NeedsTeXFormat{LaTeX2e}
%<package>\ProvidesPackage{everyshi}
%<package>         [1995/01/25 v2.00 EveryShipout Package (MS)]
%
%<*driver>
\documentclass{ltxdoc}
\usepackage{everyshi}
\GetFileInfo{everyshi.sty}
\setcounter{IndexColumns}{2}
\EnableCrossrefs
\CodelineIndex
\RecordChanges
\setcounter{IndexColumns}{2}
\setlength{\IndexMin}{30ex}
\setlength{\columnseprule}{.4pt}
\AtBeginDocument{\addtocontents{toc}{\protect\begin{multicols}{2}}}
\AtEndDocument{\addtocontents{toc}{\protect\end{multicols}}}
\begin{document}
\DocInput{everyshi.dtx}
\end{document}
%</driver>
%
% Copyright (C) 1994, 1995 by Martin Schr\"oder.  All rights reserved.
%
% IMPORTANT NOTICE:
%
% You are not allowed to change this file.  You may however copy
% this file to a file with a different name and then change the
% copy if you obey the restrictions on file changes described in
% everyshi.ins.
%
% You are NOT ALLOWED to distribute this file alone.  You are NOT
% ALLOWED to take money for the distribution or use of this file
% (or a changed version) except for a nominal charge for copying
% etc.
%
% You are allowed to distribute this file under the condition that
% it is distributed together with all files mentioned in
% everyshi.ins.
%
% If you receive only some of these files from someone, complain!
%
% However, if these files are distributed by established suppliers
% as part of a complete TeX distribution, and the structure of the
% distribution would make it difficult to distribute the whole set
% of files, *those parties* are allowed to distribute only some of
% the files provided that it is made clear that the user will get
% a complete distribution-set upon request to that supplier (not
% me).  Notice that this permission is not granted to the end
% user.
%
%
% For error reports in case of UNCHANGED versions see everyshi.ins
%
% \fi
%
% \CheckSum{42}
%
%% \CharacterTable
%% {Upper-case    \A\B\C\D\E\F\G\H\I\J\K\L\M\N\O\P\Q\R\S\T\U\V\W\X\Y\Z
%%  Lower-case    \a\b\c\d\e\f\g\h\i\j\k\l\m\n\o\p\q\r\s\t\u\v\w\x\y\z
%%  Digits        \0\1\2\3\4\5\6\7\8\9
%%  Exclamation   \!     Double quote  \"     Hash (number) \#
%%  Dollar        \$     Percent       \%     Ampersand     \&
%%  Acute accent  \'     Left paren    \(     Right paren   \)
%%  Asterisk      \*     Plus          \+     Comma         \,
%%  Minus         \-     Point         \.     Solidus       \/
%%  Colon         \:     Semicolon     \;     Less than     \<
%%  Equals        \=     Greater than  \>     Question mark \?
%%  Commercial at \@     Left bracket  \[     Backslash     \\
%%  Right bracket \]     Circumflex    \^     Underscore    \_
%%  Grave accent  \`     Left brace    \{     Vertical bar  \|
%%  Right brace   \}     Tilde         \~}
%%
%% \iffalse meta-comment
%% ===================================================================
%%  @LaTeX-package-file{
%%     author          = {Martin Schr\"oder},
%%     version         = "2.00",
%%     date            = "25 January 1995",
%%     filename        = "everyshi.sty",
%%     address         = {Martin Schr\"oder
%%                        Friedrich-Humbert-Stra\ss{}e 124
%%                        D-28759 Bremen
%%     telephone       = "+49-421-628813",
%%     email           = "MS@Dream.HB.North.DE (INTERNET)",
%%     codetable       = "ISO/ASCII",
%%     keywords        = "LaTeX2e, \shipout",
%%     supported       = "yes",
%%     docstring       = "LaTeX package which defines a new hook
%%                        \EveryShipout".
%%  }
%% ===================================================================
%% \fi
%
%  \changes{v1.00}{1994/12/04}{New}
%  \changes{v1.02}{1994/12/07}{Name changed from \textsf{atshipou} to 
%                                \textsf{everyshi}}
%  \changes{v1.03}{1994/12/09}{Documentation improved}
%  \changes{v2.00}{1995/01/25}{Redesign}
%
%  \IndexPrologue{^^A
%     \section*{Index}^^A
%     \markboth{Index}{Index}^^A
%     Numbers written in \emph{italic} refer to the page where the
%     corresponding entry is described, the ones
%     \underline{underlined} to the definition, the rest to the places
%     where the entry is used.}
%
% ^^A -----------------------------
%
%  \title{\unskip
%           The \textsf{everyshi} package^^A
%           \thanks{^^A
%              The version umber of this file is \fileversion,
%              last revised \filedate.\newline
%              The name \textsf{everyshi} is a tribute to the $8+3$
%              file-naming convention of certain ``operating
%              systems''; strictly speaking it should be 
%              \textsf{everyshipout}.}^^A
%        }
%  \author{Martin Schr\"oder\\[0.5ex]
%          \normalsize  Friedrich-Humbert-Stra\ss{}e 124\\
%          \normalsize  D-28759 Bremen\\
%          \normalsize  MS@Dream.HB.North.DE (INTERNET)}
%  \date{\filedate}
%  \maketitle
%
% ^^A -----------------------------
%
%
%  \begin{abstract}
%     This package defines a new command \cs{EveryShipout} analogous 
%     to \cs{AtBeginDocument} etc., whose argument is executed before 
%     each \cs{shipout}.
%  \end{abstract}
%
%  \pagestyle{headings}
%
% ^^A -----------------------------
%
%  \tableofcontents
%
% ^^A -----------------------------
%
%  \section{Introduction}
%
%  This package provides the hook \cs{EveryShipout} whose argument is
%  executed after the output routine has constructed \cs{box255}, and
%  before \cs{shipout} is called.
%
%  An example application for this package would be a package for
%  adding text to the bottom of each page.
%  Such a package does exist: \textsf{prelim2e}.
%
% ^^A -----------------------------
%
%  \section{Usage}
%
%  \cs{EveryShipout}\marg{code} declares
%  \mbox{$\langle$\emph{code}$\rangle$} that is saved internally
%  and executed before each \cs{shipout}.
%
%  The \meta{code} is executed after \cs{box255} has been constructed
%  by the output routine and can change \cs{box255}.
%  \cs{shipout} is called \emph{after} \meta{code}.
%
%  Repeated use of this command is permitted: the code in the
%  argument is stored (and executed) in the order of their
%  declarations.
%
% ^^A -----------------------------
%
%  \section{Options}
%
%  The package has no options.
%
% ^^A -----------------------------
%
%  \section{Required packages}
%
%  The package does not require any further packages.
%
% ^^A -----------------------------
%
%  \StopEventually{^^A
%     \PrintIndex\PrintChanges
%     ^^A Make sure that the index is not printed twice
%     ^^A (ltxdoc.cfg might have a second \PrintIndex command)
%     \let\PrintChanges\relax
%     \let\PrintIndex\relax
%     }
%
% ^^A -----------------------------
%
%  \section{The implementation}
%
%    \begin{macrocode}
%<*package>
%    \end{macrocode}
%
%  \begin{macro}{\@EveryShipout@Hook}
%  \changes{v2.00}{1995/01/25}{Name changed from \cs{@shipouthook} to
%                                \cs{@EveryShipout@Hook}}
%  The code to be executed before \cs{shipout} is stored in
%  \cs{@EveryShipout@Hook}.
%    \begin{macrocode}
\let\@EveryShipout@Hook\@empty
%    \end{macrocode}
%  \end{macro}
%
%  \begin{macro}{\EveryShipout}
%  \changes{v1.01}{1994/12/06}{\cs{newcommand} instead of \cs{def}}
%  \changes{v1.02}{1994/12/07}{Name changed from \cs{AtShipout} to 
%                                \cs{EveryShipOut}}
%  \cs{EveryShipout} is modeled after \cs{AtBeginDocument}.
%    \begin{macrocode}
\newcommand{\EveryShipout}{\g@addto@macro\@EveryShipout@Hook}
%    \end{macrocode}
%  \end{macro}
%
%  We want to redefine \cs{shipout} so that first \cs{box255} is
%  constructed and after that we can do something and at last shipout
%  the (possible modified) \cs{box255}.
%  Alas, this does not work in the usual way, since \cs{shipout} is
%  a \TeX{} primitive whose argument is a \meta{box}.
%  This means that simply redefining \cs{shipout} via \cs{newcommand[1]}
%  is impossible since \meta{box} can be something like \cs{box255} or
%  something like \cs{vbox\{\ldots\}}.
%  In the first case \texttt{\#1} would be \meta{\cs{box}} (without 
%  \meta{255}); in the second case it would be \meta{\cs{vbox}} (without
%  \meta{\{\ldots\}}).
%
%  The solution we use here is borrowed from \textsf{quire.tex} by
%  Marcel R.~van der Goot.
%  It is based upon \cs{afterassignment} and \cs{aftergroup}.
%
%  \begin{macro}{\@EveryShipout@Shipout}
%  \changes{v2.00}{1995/01/25}{new}
%  \cs{@EveryShipout@Shipout} is our replacement for \cs{shipout}.
%    \begin{macrocode}
\newcommand{\@EveryShipout@Shipout}{%
   \afterassignment\@EveryShipout@Test
   \global\setbox\@cclv= %
   }
%    \end{macrocode}
%  \cs{box255} is set to whatever comes after \cs{shipout}; but after
%  that assignment \cs{@EveryShipout@Test} is called.
%  \end{macro}
%
%  \begin{macro}{\@EveryShipout@Test}
%  \changes{v2.00}{1995/01/25}{new}
%  \cs{@EveryShipout@Test} determines if \cs{shipout} is called with
%  an argument like \cs{box255} or something like \cs{vbox\{\ldots\}}.
%  In the later case we delay the call of \cs{@EveryShipout@Output} 
%  (where the original \cs{shipout} is called) via \cs{aftergroup}.
%    \begin{macrocode}
\newcommand{\@EveryShipout@Test}{%
   \ifvoid\@cclv\relax
      \aftergroup\@EveryShipout@Output
   \else
      \@EveryShipout@Output
   \fi%
   }
%    \end{macrocode}
%  \end{macro}
%
%  \begin{macro}{\@EveryShipout@Output}
%  \changes{v2.00}{1995/01/25}{new}
%  \cs{@EveryShipout@Output} does the actual work.
%  First the \meta{code} accumulated via \cs{EveryShipout} is called
%  and then the original \cs{shipout} stored in 
%  \cs{@EveryShipout@Org@Shipout} is called to finally ship out 
%  \cs{box255}.
%    \begin{macrocode}
\newcommand{\@EveryShipout@Output}{%
   \@EveryShipout@Hook%
   \@EveryShipout@Org@Shipout\box\@cclv%
   }
%    \end{macrocode}
%  \end{macro}
%
%  \begin{macro}{\@EveryShipout@Org@Shipout}
%  \changes{v2.00}{1995/01/25}{new}
%  The original \cs{shipout} is stored in \cs{@EveryShipout@Org@Shipout}
%  by \cs{@EveryShipout@Init}.
%  Here we allocate it.
%    \begin{macrocode}
\newcommand{\@EveryShipout@Org@Shipout}{}
%    \end{macrocode}
%  \end{macro}
%
%  \begin{macro}{\@EveryShipout@Init}
%  \changes{v2.00}{1995/01/25}{new}
%  \emergencystretch4em
%  \looseness-1
%  \cs{@EveryShipout@Init} stores the original \cs{shipout} in 
%  \cs{@EveryShipout@Org@Shipout} and sets \cs{shipout} to 
%  \cs{@EveryShipout@Shipout}.
%  This is done at \cs{begin\{document\}} via \cs{AtBeginDocument}.
%    \begin{macrocode}
\newcommand*{\@EveryShipout@Init}{%
   \let\@EveryShipout@Org@Shipout\shipout
   \let\shipout\@EveryShipout@Shipout
   }
\AtBeginDocument{\@EveryShipout@Init}
%    \end{macrocode}
%  \end{macro}
%
%    \begin{macrocode}
%</package>
%    \end{macrocode}
%
% ^^A -----------------------------
%
%  \section{Acknowledgements}
%  ^^A
%  Version 2.0 of \textsf{everyshi} borrows heavily from 
%  \textsf{quire.tex} of the \textsf{Midnight Macros} by Marcel R.~van 
%  der Goot (\texttt{marcel@cs.caltech.edu}).
%  The pointer to \textsf{quire} was provided by Lothar Meyer-Lerbs
%  (\texttt{\TeX{}Satz@zfn.uni-bremen.de}).
%  As usual Rebecca Stiels improved the quality of this documentation.
%  If you need a translator from English or Fran\c{c}ais to German, send
%  her an e-mail to \texttt{Rebecca@Andurg.HB.North.DE}.
%  And if you need a \TeX{}nician or computer scientist, send me an e-mail;
%  I'm looking for a job.
%
% ^^A -----------------------------
%
%  \Finale
