\documentclass{article}
% $Id: sptest.tex,v 1.2 1995/01/28 07:48:54 swift Exp $
%* preamble
\usepackage{shortvrb}
\MakeShortVerb{\|}

\usepackage{tmacros}
\usepackage{relsize}
\usepackage{markx}
\usepackage{includex}
\usepackage{lips}

\usepackage{alltt}

\makeatletter
\newcommand\debug[1]
  {\tracingoutput\@ne
  \tracingpages\@ne
  \tracingmacros#1  
  \tracingcommands#1
  \showboxbreadth\@M
  \showboxdepth\@M}
\makeatother

\setcounter{errorcontextlines}{999}

\begin{document}
%* Intro

\author{Matt Swift \texttt{<swift@bu.edu>}}
\date{27 January 1995}
\title{Pre-release testing of the \texttt{scholar} package}
\maketitle


\section{Ellipses}
The log file will contain detailed information on the spacings.

The internal code for |\ldots| changed with the December 1994 release;
I highly doubt there was a visible difference.
\begin{quote}
 \debug1
   Hello\lips. And

   \lips and \\
   Hello \lips and \\
   Hello!\lips And  \\  % (also `?', `.', `:', `,' and `:')
   Hello\lips and \\
   Hello! \lips And \\  % (also `?', `.', `:', `,' and `:')
   Hello \lips and
 \debug0
\end{quote}
\begin{center}
*\quad*\quad*
\end{center}
\begin{quote} \let\lips\ldots
 \debug1
   Hello\lips. And

   \lips and \\
   Hello \lips and \\
   Hello!\lips And  \\  % (also `?', `.', `:', `,' and `:')
   Hello\lips and \\
   Hello! \lips And \\  % (also `?', `.', `:', `,' and `:')
   Hello \lips and
 \debug0
\end{quote}

\section{Text and abbreviation macros}
\newabbrev{uni}{University Professors}{UNI}
\newbook{worst}{Worstward Ho}
\newbook{fall}{All That Fall}
\newbook{nacht}{Nacht und Tr\"aume}
\newbook{csp}{Collected Shorter Plays \emph{(}CSP\emph{)}}[CSP]
\newname{joyce}{James Joyce}{Joyce}
\newname{hill}{Geoffrey Hill}{Hill}
% \newwork{pricks}{More Pricks Than Kicks}[More Pricks]
% \newwork{how}{How It Is}

In this example, I will talk a little about my real work that I do when I'm not
distracting myself with the |scholar| package.  I am getting a Ph.D. at the
\uni Program at Boston University---that's ``\uni'' for short.  \uni's purpose
is to bring together a group of faculty (also called the \unilong) and a group
of students who don't belong in any one department, and free them from the
beauracracy of departments.  There are some great faculty in our group,
including \hill, a great poet and scholar, with whom I have been lucky enough
to work.  I can actually write better English than this (which would distress
\hill), but I'm trying to test text macros.  I'm presently writing a
dissertation on Samuel Beckett.  Although there is very little biographical
material available, it is well known that he spent several years under the wing
of \joyce, another of the great writers in English this century.  Both \joyce
and Beckett, it is curious, like other great writers, both had trouble with
their vision, and were exiles in some sense.  One of my favorite pieces by
Beckett is \worst, a short work written in the 1980's not long before his
death: ``Fail again.  Fail better.''  \worst is lyric and exalting to me.  A
work I feel is underrated is the radio play \fall (all but his three long plays
are collected in \csp).  It's extremely funny, and very touchingly
compassionate.  Because it is a radio play, it looses less from performance to
reading.  I would recommend \fall to anyone.  His later plays (and fiction) are
famously enigmatic, but with a little practice, it is not hard to see the same
lyric beauty and compassion.  Take the brief television play \nacht (in \csp of
course), which has no dialogue, only a few murmured bars of the Schubert song,
also brief, and also called \nacht---it's one of the most hauntingly beautiful
few minutes of music I've ever heard, and I particularly recommend Cheryl
Studer's recording on Deutsche Grammophone.  Every other recording I've heard
plays too fast.

\ResetAbbrevs

It's funny, but though I think the \unishort Program is great, I can't wait to
leave, and move on to something else.  I guess it isn't the greatest thing
around, though I will miss the inspiration of my advisor, and some other of the
\uni like \hill.

\section{Relative sizing}

This test won't be as literary as the last, sorry.  Just three great words in
three sentences (by Beckett, of course).

\makeatletter
\newcommand\checks[1]
  {\quad$\the\rs@size = #1$?\par}
\makeatother

Heaven.  Hell.  Helen.  \checks4 \smaller
Heaven.  Hell.  Helen.  \checks3 \smaller[1]
Heaven.  Hell.  Helen.  \checks2 \smaller[-1]
Heaven.  Hell.  Helen.  \checks3 \smaller[2]
Heaven.  Hell.  Helen.  \checks1 \relsize{-1}
Heaven.  Hell.  Helen.  \checks0 \smaller
Heaven.  Hell.  Helen.  \checks0 \relsize{5}
Heaven.  Hell.  Helen.  \checks5 \larger
Heaven.  Hell.  Helen.  \checks6 \larger[1]
Heaven.  Hell.  Helen.  \checks7 \larger[-1]
Heaven.  Hell.  Helen.  \checks6 \larger[2]
Heaven.  Hell.  Helen.  \checks8 \larger
Heaven.  Hell.  Helen.  \checks9 \larger
Heaven.  Hell.  Helen.  \checks9 \normalsize

\section{New ways to include}

This will take several pages. 

First, check that we can |\include| as usual, so the rest of this page should
be blank, and the Melville quote should be alone on the next page.

% $Id: sptest-aux.tex,v 1.1 1995/01/28 07:39:25 swift Exp $

This text is in a separate file that just contains text, and some \LaTeX\ code.
For a change, let's quote \emph{Moby Dick}, chapter~94:
\begin{quotation} % 
  Squeeze!\ squeeze!\ squeeze!\ all the morning long; I squeezed that sperm
  till I myself almost melted into it; I squeezed that sperm till a strange
  sort of insanity came over me; and I found myself unwittingly squeezing my
  co-laborers' hands in it, mistaking their hands for the gentle globules.
  Such an abounding, affectionate, friendly, loving feeling did this avocation
  beget; that at last I was continually squeezing their hands, and looking up
  into their eyes sentimentally; as much as to say,---Oh!\ my dear fellow
  beings, why should we longer cherish any social acerbities, or know the
  slightest ill-humor or envy!  Come; let us squeeze hands all round; nay, let
  us all squeeze ourselves into each other; let us squeeze ourselves
  universally into the very milk and sperm of kindness.

  Would that I could keep squeezing that sperm for ever! For now, since by many
  prolonged, repeated experiences, I have perceived that in all cases man must
  eventually lower, or at least shift, his conceit of attainable felicity; not
  placing it anywhere in the intellect or the fancy; but in the wife, the
  heart, the bed, the table, the saddle, the fire-side, the country; now that I
  have perceived all this, I am ready to squeeze case eternally.  In thoughts
  of the visions of the night, I saw long rows of angels in paradise, each with
  his hands in a jar of spermaceti.
\end{quotation}


This should be on a new page.  Now we include the same thing, but it should
come immediately here because we use |\include*|:

\include*{sptest-aux}

And this paragraph should immediately follow the Melville.  Now, we check
|\includedoc| and |\includedoc*| in a similar manner.  The included file is
going to use the |alltt| environment.  We have loaded the package in the parent
file.  The |\usepackage| command in the included file should be disabled; if
not, there will be an error when it tries to redefine the environment.  Also,
we disable several commands that will produce an error if they are not. This
should be the last text on this page.

\newcommand\tdisable{\typeout{Error with \protect\tdisable.}}
\newcommand\tdisableone[1]{\typeout{Error with \protect\tdisableone.}}
\newcommand\tdisableopt[2][]{\typeout{Error with \protect\tdisableopt.}}
\newcommand\tdisabletwo[2]{\typeout{Error with \protect\tdisabletwo.}}

\disable{\let\tdisable\relax}
\disable{\let\tdisableone\gobble}
\disable{\let\tdisableopt\gobbleopt}
\disable{\let\tdisabletwo\gobbletwo}

\makeatletter
\newcommand\checkp
  {\@ifundefined{preamblecmd}
      {%
%      \typeout{Preamblecmd undefined.}%
      }
    {%
%    \typeout{Preamblecmd is: (\preamblecmd).}%
%    \let\preamblecmd\relax
    }}
\makeatother
\checkp

\includedoc{sptest-aux1}

\checkp

And this should be on a new page.  The first line should have said it
comes from the preamble. 

The |\includedoc| family of commands puts the included file into a group.  So
|\preamblecmd| is not defined now.

Now we do the same thing over, but the file should appear immediately.  

\includedoc*{sptest-aux1}

\checkp

And this paragraph should immediately follow the Beckett.  Finally, we have to
check |\includedocskip| and |\includedocskip*|.  The difference will be that
the first line will say it comes from the parent.  This should be the last text
on the page.

\newcommand\preamblecmd
  {This def of preamblecmd comes from the parent.}

\includedocskip{sptest-aux1}

\checkp
This should be on a new page.  And once again, the quote should follow
immediately here:

\includedocskip*{sptest-aux1}

\checkp

And this text should immediately follow the quote. 

%* end

\end{document}


\section{Extra marks}

See the discussion in the source about how these marks are \emph{not} really
independent.  This test file is a good way to see the mess you can get into. 

This too will take a bunch of pages to check.  I'm using the table on page~258
of the \TeX book instead of bothering to make up my own test example.   I don't
test the macros |\FreeXMarkTwo| and |\ClaimXMarkTwo|. 

\makeatletter
\newcommand\ps@mcheck
  {\renewcommand\@oddfoot
    {\raisebox{0pt}[10pt][0pt]{\parbox[b]{\textwidth}{\markchecks}}}}
\newcommand\domark[1]
  {#1 \printmark{#1} makes the mark (\sc@themark)\par}
\makeatother

\newcommand\printmark[1]
  {d}

\newcommand{\markchecks}{}
\newcommand\cmark[2]
  {#1xmark#2 is (\csname #1xmark#2\endcsname)}
\newcommand\cmarks
  {\thispagestyle{mcheck}
  \renewcommand{\markchecks}
    {
      \cmark{top}{one}\\
      \cmark{top}{two}\\
      \cmark{first}{one}\\
      \cmark{first}{two}\\
      \cmark{bot}{one}\\
      \cmark{bot}{two}\\
      leftmark is (\leftmark)\\
      rightmark is (\rightmark)\\
      topmark is (\topmark)\\
      firstmark is (\firstmark)\\
      botmark is (\botmark)
    }

 \begin{quotation} 
      A road still carriageable climbs over the high moorland.  It cuts across
  vast turfbogs, a thousand feet above sealevel, two thousand if you prefer.
  It leads to nothing any more.  A few ruined forts, a few ruined dwellings.
  The sea is not far, just visible beyond the valleys dipping eastward, pale
  plinth as pale as the pale wall of sky.  Tarns lie hidden in the folds of the
  moor, invisible from the road, reached by faint paths, under high overhanging
  crags.  All seems flat, or gently undulating, and there at a stone's throw
  these high crags, all unsuspected by the wayfarer.  Of granite what is more.
  In the west the chain is at its highest, its peaks exalt even the most
  downcast eyes, peaks commanding the vast champaign land, the celebrated
  pastures, the golden vale.  Before the travellers, as far as eye can reach,
  the road winds on into the south, uphill, but imperceptibly.  None ever pass
  this way but beauty-spot hogs and fanatical trampers.  Under its heather mass
  the quag allures, with an allurement not all mortals can resist.  Then it
  swallows them up or the mist comes down.  The city is not far either, from
  certain points its lights can be seen by night, its light rather, and by day
  its haze.  Even the piers of the harbour can be distinguished, on very clear
  days, of the two harbours, tiny arms in the glassy sea outflung, known flat,
  seen raised.  And the islands and promontories, one has only to stop and turn
  at the right place, and of course by night the beacon lights, both flashing
  and revolving.  It is here one would lie down, in a hollow bedded with dry
  heather, and fall asleep, for the last time, on an afternoon, in the sun,
  head down among the minute life of stems and bells, and fast fall asleep,
  fast farewell to charming things.  It's a birdless sky, the odd raptor, no
  song.  End of descriptive passage.
\end{quotation}
\begin{center}
  \emph{Mercier and Camier}\\
  section~VII (beginning)\\
    Samuel Beckett
\end{center}
  \clearpage}

\clearpage

\domark{\markboth{1}{one}}

\cmarks

\domark{\markright{Two}}
\domark{\markboth{2}{two}}
\domark{\xmarkone{A}}
\domark{\xmarktwo{a}}

\cmarks

\domark{\markboth{3}{three}}
\domark{\markright{Three}}

\cmarks

\domark{\markboth{4}{four}}
\domark{\xmarkone{B}}
\domark{\xmarktwo{b}}
\domark{\xmarkone{C}}
\domark{\xmarktwo{c}}

\cmarks

\domark{\markboth{5}{five}}
\domark{\xmarkone{D}}
\domark{\xmarktwo{d}}

\cmarks

\cmarks
