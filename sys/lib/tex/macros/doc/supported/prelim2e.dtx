% \iffalse
%% File: printtim.dtx Copyright (C) 1995 Martin Schr\"oder
%
%<package>\NeedsTeXFormat{LaTeX2e}
%<package>\ProvidesPackage{prelim2e}
%<package>         [1995/01/26 v1.00 Prelim2e Package (MS)]
%
%<*driver>
\documentclass{ltxdoc}
\usepackage{prelim2e}
\GetFileInfo{prelim2e.sty}
\setcounter{IndexColumns}{2}
\EnableCrossrefs
\CodelineIndex
\RecordChanges
\setcounter{IndexColumns}{2}
\setlength{\IndexMin}{40ex}
\setlength{\columnseprule}{.4pt}
\AtBeginDocument{\addtocontents{toc}{\protect\begin{multicols}{2}}}
\AtEndDocument{\addtocontents{toc}{\protect\end{multicols}}}
\begin{document}
\DocInput{prelim2e.dtx}
\end{document}
%</driver>
%
% Copyright (C) 1995 by Martin Schr\"oder.  All rights reserved.
%
% IMPORTANT NOTICE:
%
% You are not allowed to change this file.  You may however copy
% this file to a file with a different name and then change the
% copy if you obey the restrictions on file changes described in
% everyshi.ins.
%
% You are NOT ALLOWED to distribute this file alone.  You are NOT
% ALLOWED to take money for the distribution or use of this file
% (or a changed version) except for a nominal charge for copying
% etc.
%
% You are allowed to distribute this file under the condition that
% it is distributed together with all files mentioned in
% printtim.ins.
%
% If you receive only some of these files from someone, complain!
%
% However, if these files are distributed by established suppliers
% as part of a complete TeX distribution, and the structure of the
% distribution would make it difficult to distribute the whole set
% of files, *those parties* are allowed to distribute only some of
% the files provided that it is made clear that the user will get
% a complete distribution-set upon request to that supplier (not
% me).  Notice that this permission is not granted to the end
% user.
%
%
% For error reports in case of UNCHANGED versions see everyshi.ins
%
% \fi
%
% \CheckSum{94}
%
%% \CharacterTable
%% {Upper-case    \A\B\C\D\E\F\G\H\I\J\K\L\M\N\O\P\Q\R\S\T\U\V\W\X\Y\Z
%%  Lower-case    \a\b\c\d\e\f\g\h\i\j\k\l\m\n\o\p\q\r\s\t\u\v\w\x\y\z
%%  Digits        \0\1\2\3\4\5\6\7\8\9
%%  Exclamation   \!     Double quote  \"     Hash (number) \#
%%  Dollar        \$     Percent       \%     Ampersand     \&
%%  Acute accent  \'     Left paren    \(     Right paren   \)
%%  Asterisk      \*     Plus          \+     Comma         \,
%%  Minus         \-     Point         \.     Solidus       \/
%%  Colon         \:     Semicolon     \;     Less than     \<
%%  Equals        \=     Greater than  \>     Question mark \?
%%  Commercial at \@     Left bracket  \[     Backslash     \\
%%  Right bracket \]     Circumflex    \^     Underscore    \_
%%  Grave accent  \`     Left brace    \{     Vertical bar  \|
%%  Right brace   \}     Tilde         \~}
%%
%% \iffalse meta-comment
%% ===================================================================
%%  @LaTeX-package-file{
%%    author         = {Martin Schr\"oder},
%%    version        = "1.00",
%%    date           = "26 January 1995",
%%    filename       = "prelim2e.sty",
%%    address        = {Martin Schr\"oder
%%                      Friedrich-Humbert-Stra\ss{}e 124
%%                      D-28759 Bremen},
%%    telephone      = "++49-421-628813",
%%    email          = "MS@Dream.HB.North.DE (INTERNET)",
%%    codetable      = "ISO/ASCII",
%%    keywords       = "LaTeX2e, preliminary versions, versions",
%%    dependences    = "everyshi, printtim",
%%    supported      = "yes",
%%    docstring      = "LaTeX package which allows the marking of
%%                      preliminary versions of a document"
%%                     .
%%  }
%% ===================================================================
%% \fi
%
%  \iftrue
%  \renewcommand*{\PrelimText}{%
%     \textnormal{%
%     \footnotesize%
%     \textsf{prelim2e} package --
%     Version \fileversion{} --
%     Documentation \LaTeX{}ed on \today{} at \PrintTime}}
%  \fi
%
%  \changes{v1.00}{1995/01/26}{New}
%
%  \IndexPrologue{^^A
%     \section*{\indexname}^^A
%     \markboth{\indexname}{\indexname}^^A
%     Numbers written in \emph{italic} refer to the page where the
%     corresponding entry is described, the ones
%     \underline{underlined} to the definition, the rest to the places
%     where the entry is used.}
%
%  \pagestyle{headings}
%
% ^^A -----------------------------
%
%  \title{\unskip
%           The \textsf{prelim2e} package^^A
%           \thanks{^^A
%              The version number of this file is \fileversion,
%              last revised \filedate.}^^A
%        }
%  \author{Martin Schr\"oder\\[0.5ex]
%          \normalsize  Friedrich-Humbert-Stra\ss{}e 124\\
%          \normalsize  D-28759 Bremen\\
%          \normalsize  MS@Dream.HB.North.DE (INTERNET)}
%  \date{\filedate}
%  \maketitle
%
% ^^A -----------------------------
%
%
%  \begin{abstract}
%     This package allows the marking of (preliminary) versions of a
%     document on the output.
%  \end{abstract}
%
% ^^A -----------------------------
%
%  \tableofcontents
%
% ^^A -----------------------------
%
%  \section{Introduction}
%  ^^A
%  This package allows the marking of (preliminary) versions of a
%  document.
%  This is achieved using the command \cs{PrelimText}, whose expansion
%  is added \emph{below the footer} of every page of a document (look
%  at the bottom of this page for an example).
%
% ^^A -----------------------------
%
%  \section{Usage}
%  ^^A
%  Simply using this package via 
%  \mbox{\cs{usepackage\{}\textsf{prelim}\texttt{\}}} does produce a
%  text in the form of ``Preliminary version -- \today{} -- \PrintTime''.
%
%  \DescribeMacro{\PrelimText}
%  The text is produced by the command \cs{PrelimText}, which can be
%  changed via \cs{renewcommand*} or by setting options at the 
%  \cs{usepackage} command (see section~\ref{sec:options}).
%
% ^^A -----------------------------
%
%  \section{Options}
%  \label{sec:options}
%  ^^A
%  The package has the following options:
%  \nopagebreak
%  \begin{description}
%     \item[\normalfont\textsf{draft}]
%        If this option is used a text appears below the normal 
%        pagebody.
%        It is the default.
%     \item[\normalfont\textsf{final}]
%        If this option is used \textsf{prelim2e} does produce no text.
%     \item[\normalfont\textsf{english}]
%        This sets the text to ``Preliminary version''.
%        It is the default.
%     \item[\normalfont\textsf{german}]
%        This sets the text to ``Vorl\"aufige Version''.
%        It does not use the \textsf{german} or \textsf{babel} package.
%  \end{description}
%
% ^^A -----------------------------
%
%  \section{Required packages}
%  ^^A
%  The package requires the following packages:
%  \begin{description}
%     \item[\normalfont\textsf{everyshi}]
%        This package is used to implement the setting of the text below
%        the normal pagebody.
%     \item[\normalfont\textsf{printtim}]
%        This package is used to typeset the current time.
%  \end{description}
%
% ^^A -----------------------------
%
%  \StopEventually{^^A
%     \PrintIndex\PrintChanges
%     ^^A Make sure that the index is not printed twice
%     ^^A (ltxdoc.cfg might have a second \PrintIndex command)
%     \let\PrintChanges\relax
%     \let\PrintIndex\relax
%     }
%
% ^^A -----------------------------
%
%  \section{The implementation}
%
%    \begin{macrocode}
%<*package>
%    \end{macrocode}
%
% ^^A -----------------------------
%
%  \subsection{Initial Code}
%  ^^A
%  \begin{macro}{\if@prelim@draft}
%  \cs{if@prelim@draft} is used to flag the use of the \textsf{draft}
%  or \textsf{final} option.
%    \begin{macrocode}
\newif\if@prelim@draft
%    \end{macrocode}
%  \end{macro}
%
%  \begin{macro}{\PrelimWords}
%  \cs{PrelimWords} holds the language-dependend text used in 
%  \cs{PrelimText}
%    \begin{macrocode}
\newcommand*{\PrelimWords}{}
%    \end{macrocode}
%  \end{macro}
%
% ^^A -----------------------------
%
%  \subsection{Declaration of options}
%
% ^^A -----------------------------
%
%  \subsubsection{\textsf{draft} option}
%  ^^A
%  The \textsf{draft} and \textsf{final} option control the behavior
%  of \textsf{prelim2e}: Only if \textsf{final} is used in 
%  \cs{documentclass} or 
%  \mbox{\cs{usepackage\{}\textsf{prelim}\texttt{\}}}, then text is
%  produced.
%    \begin{macrocode}
\DeclareOption{draft}{\@prelim@drafttrue}
\DeclareOption{final}{\@prelim@draftfalse}
%    \end{macrocode}
%
% ^^A -----------------------------
%
%  \subsubsection{Language options}
%  ^^A
%  \textsf{english} and \textsf{german} control the content of
%  \cs{PrelimWords}.
%    \begin{macrocode}
\DeclareOption{english}{%
   \renewcommand*{\PrelimWords}{Preliminary version}}
\DeclareOption{german}{%
   \renewcommand*{\PrelimWords}{Vorl\"aufige Version}}
%    \end{macrocode}
%
% ^^A -----------------------------
%
%  \subsection{Executing options}
%  ^^A
%  The default options are \textsf{draft} and \textsf{english}.
%    \begin{macrocode}
\ExecuteOptions{draft,english}
\ProcessOptions
%    \end{macrocode}
%
% ^^A -----------------------------
%
%  \subsection{Loading packages}
%  ^^A
%  We need the \textsf{everyshi} and the \textsf{printtim} packages.
%    \begin{macrocode}
\RequirePackage{everyshi}[1995/01/25]
\RequirePackage{printtim}
%    \end{macrocode}
%
%  \subsection{Producing the text}
%  ^^A
%  \begin{macro}{\PrelimText}
%  \cs{PrelimText} produces the text which is put below the page.
%  It can be changed via \cs{renewcommand*}.
%  The style of the text is controlled by \cs{PrelimTextStyle}.
%  We first have to reset the style and size, otherwise the settings in
%  effect at the point of text where \cs{ouput} is called would be used.
%    \begin{macrocode}
\newcommand*{\PrelimText}{%
   \textnormal{%
      \footnotesize%
      \PrelimTextStyle%
      \PrelimWords{} -- \today{} -- \PrintTime%
      }%
   }
%    \end{macrocode}
%  \end{macro}
%
%  \begin{macro}{\PrelimTextStyle}
%  \cs{PrelimTextStyle} controls the style of the text produced by
%  \cs{PrelimText}.
%  It's default is empty.
%    \begin{macrocode}
\newcommand*{\PrelimTextStyle}{}
%    \end{macrocode}
%  \end{macro}
%
% ^^A -----------------------------
%
%  \subsection{Putting the text below the page}
%  ^^A
%  We put the text below the page via \cs{EveryShipout} provided by
%  the \textsf{everyshi} package.
%  This is done by \cs{@Prelim@EveryShipout}.
%
%  \begin{macro}{\@Prelim@EveryShipout}
%  \cs{@Prelim@EveryShipout} put the text produced by \cs{PrelimText}
%  below the page.
%  To do this we modify \cs{box255}: We append a \cs{vbox} with height
%  and depth of 0pt and the width of \cs{box255} which contains a 
%  \cs{hbox} with the width of \cs{box255} in which \cs{PrelimText}
%  is centered.
%    \begin{macrocode}
\newcommand{\@Prelim@EveryShipout}{
   \bgroup
%    \end{macrocode}
%  First we save the dimensions of \cs{box255}: height, width and depth;
%  and calculate the total height of \cs{box255}.
%    \begin{macrocode}
      \dimen\z@=\wd\@cclv
      \dimen\@ne=\ht\@cclv
      \dimen\tw@=\dp\@cclv
      \dimen\thr@@=\dimen1
      \advance\dimen\thr@@ by \dimen\tw@
%    \end{macrocode}
%  Then we set \cs{box255}: 
%  A \cs{vbox} to the total height of \cs{box255}.
%  In this a \cs{hbox} to the width of \cs{box255} is included, in which
%  \cs{box255} is set.
%    \begin{macrocode}
      \global\setbox\@cclv\vbox to \dimen\thr@@{%
         \hbox to \dimen\z@{%
            \box\@cclv%
            \hss%
            }%
%    \end{macrocode}
%  To this we append the text produced by \cs{PrelimText}.
%  It is put in a \cs{vbox} to 0pt in which a \cs{hbox} to the width of 
%  \cs{box255} is included, in which \cs{PrelimText} is set.
%  We have to reset \cs{protect} because it is set to \cs{noexpand} by
%  the output routine.
%    \begin{macrocode}
         \vbox to \z@{%
            \hbox to \dimen\z@{%
               \let\protect\relax
               \hfill\PrelimText\hfill%
               }%
            \vss%
            }%
         \vss%
         }%
%    \end{macrocode}
%  Finally we set the dimensions of \cs{box255} to the values they had
%  before \cs{@Prelim@EveryShipout}.
%    \begin{macrocode}
      \wd\@cclv=\dimen\z@
      \ht\@cclv=\dimen\@ne
      \dp\@cclv=\dimen\tw@
   \egroup
   }
%    \end{macrocode}
%  \end{macro}
%
% ^^A -----------------------------
%
%  \subsection{Tieing \textsf{prelim} into the system}
%  ^^A
%  \cs{@Prelim@EveryShipout} is tied into the system via 
%  \cs{EveryShipout}.
%  But only if the \textsf{draft} option is used.
%    \begin{macrocode}
\if@prelim@draft
   \EveryShipout{\@Prelim@EveryShipout}
\fi
%    \end{macrocode}
%
%    \begin{macrocode}
%</package>
%    \end{macrocode}
%
% ^^A -----------------------------
%
%  \section{Acknowledgements}
%
%  The idea of this package is based on \texttt{prelim.sty} for 
%  \LaTeX2.09 by Robert Tolksdorf.
%  It provides nearly the same functionality as \textsf{prelim2e}, but 
%  in a very dirty way: it uses a modified output routine and does not 
%  work with \LaTeXe.
%
%  As usual Rebecca Stiels improved the quality of this documentation.
%  If you need a translator from English or Fran\c{c}ais to German, send
%  her an e-mail to \texttt{Rebecca@Andurg.HB.North.DE}.
%  And if you need a \TeX{}nician or computer scientist, contact \emph{me}:
%  I'm looking for a job.
%
% ^^A -----------------------------
%
%  \Finale
