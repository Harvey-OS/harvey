% \iffalse
%% File: count1to.dtx Copyright (C) 1994 Martin Schr\"oder
%
%<package>\NeedsTeXFormat{LaTeX2e}
%<package>\ProvidesPackage{count1to}
%<package>         [1994/12/09 v1.01 Count1to9 Package (MS)]
%
%<*driver>
\documentclass{ltxdoc}
\usepackage{count1to}
\GetFileInfo{count1to.sty}
\setcounter{IndexColumns}{2}
\EnableCrossrefs
\CodelineIndex
\RecordChanges
\setcounter{IndexColumns}{2}
\setlength{\IndexMin}{30ex}
\AtBeginDocument{\addtocontents{toc}{\protect\begin{multicols}{2}}}
\AtEndDocument{\addtocontents{toc}{\protect\end{multicols}}}
\begin{document}
\makeatletter
\makeatother
\DocInput{count1to.dtx}
\end{document}
%</driver>
%
% Copyright (C) 1994 by Martin Schr\"oder.  All rights reserved.
%
% IMPORTANT NOTICE:
%
% You are not allowed to change this file.  You may however copy
% this file to a file with a different name and then change the
% copy if you obey the restrictions on file changes described in
% count1to.ins.
%
% You are NOT ALLOWED to distribute this file alone.  You are NOT
% ALLOWED to take money for the distribution or use of this file
% (or a changed version) except for a nominal charge for copying
% etc.
%
% You are allowed to distribute this file under the condition that
% it is distributed together with all files mentioned in
% count1to.ins.
%
% If you receive only some of these files from someone, complain!
%
% However, if these files are distributed by established suppliers
% as part of a complete TeX distribution, and the structure of the
% distribution would make it difficult to distribute the whole set
% of files, *those parties* are allowed to distribute only some of
% the files provided that it is made clear that the user will get
% a complete distribution-set upon request to that supplier (not
% me).  Notice that this permission is not granted to the end
% user.
%
%
% For error reports in case of UNCHANGED versions see count1to.ins
%
% \fi
%
% \CheckSum{103}
%
%% \CharacterTable
%% {Upper-case    \A\B\C\D\E\F\G\H\I\J\K\L\M\N\O\P\Q\R\S\T\U\V\W\X\Y\Z
%%  Lower-case    \a\b\c\d\e\f\g\h\i\j\k\l\m\n\o\p\q\r\s\t\u\v\w\x\y\z
%%  Digits        \0\1\2\3\4\5\6\7\8\9
%%  Exclamation   \!     Double quote  \"     Hash (number) \#
%%  Dollar        \$     Percent       \%     Ampersand     \&
%%  Acute accent  \'     Left paren    \(     Right paren   \)
%%  Asterisk      \*     Plus          \+     Comma         \,
%%  Minus         \-     Point         \.     Solidus       \/
%%  Colon         \:     Semicolon     \;     Less than     \<
%%  Equals        \=     Greater than  \>     Question mark \?
%%  Commercial at \@     Left bracket  \[     Backslash     \\
%%  Right bracket \]     Circumflex    \^     Underscore    \_
%%  Grave accent  \`     Left brace    \{     Vertical bar  \|
%%  Right brace   \}     Tilde         \~}
%%
%% \iffalse meta-comment
%% ===================================================================
%%  @LaTeX-style-file{
%%     author          = {Martin Schr\"oder},
%%     version         = "1.01",
%%     date            = "9 December 1993",
%%     filename        = "count1to.sty",
%%     address         = {Martin Schr\"oder
%%                        Friedrich-Humbert-Stra\ss{}e 124
%%                        D-28759 Bremen
%%     telephone       = "+49-421-628813",
%%     email           = "MS@Dream.HB.North.DE (INTERNET)",
%%     codetable       = "ISO/ASCII",
%%     keywords        = "LaTeX, pages",
%%     supported       = "yes",
%%     docstring       = "LaTeX package which sets count1 to count9,
%%                        which can be used to select certain pages
%%                        with a driver.
%%                        Uses the everyshi package."
%%  }
%% ===================================================================
%% \fi
%
%  \changes{v1.00}{1994/12/07}{New}
%  \changes{v1.01}{1994/12/09}{Documentation improved}
%
%  \newcommand{\Count}[1]{\texttt{\symbol{92}count#1}}
%
%  \IndexPrologue{^^A
%     \section*{Index}^^A
%     \markboth{Index}{Index}^^A
%     Numbers written in {\it italic\/} refer to the page where the
%     corresponding entry is described, the ones
%     \underline{underlined} to the definition, the rest to the places
%     where the entry is used.}
%
% ^^A -----------------------------
%
%  \title{\unskip
%         The \textsf{count1to} package^^A
%           \thanks{^^A
%              The version umber of this file is \fileversion,
%              last revised \filedate.\newline
%              The name \textsf{count1to} is a tribute to the $8+3$ 
%              file-naming convention of certain ``operating 
%              systems''; strictly speaking it should be
%              \textsf{count1to9}.}^^A
%        }
%  \author{Martin Schr\"oder\\[0.5ex]
%          \normalsize  Friedrich-Humbert-Stra\ss{}e 124\\
%          \normalsize  D-28759 Bremen\\
%          \normalsize  MS@Dream.HB.North.DE (INTERNET)}
%  \date{\filedate}
%  \maketitle
%
% ^^A -----------------------------
%
%
%  \begin{abstract}
%     This package sets \Count{1} to \Count{8} with the values
%     of \texttt{page} to \texttt{subparagraph}.
%     \Count{9} is used to flag odd pages.
%     The values of these counters are displayed and written in the
%     \textsf{.dvi} file by \TeX{} and can later be used to select
%     the pages of certain parts of the document for printing if the
%     device driver supports this.
%  \end{abstract}
%
%  \pagestyle{headings}
%
% ^^A -----------------------------
%
%  \tableofcontents
%
% ^^A -----------------------------
%
%  \section{Introduction}
%
%  Most of the time users want to print only certain parts of a 
%  document; but these can only be selected by using the page numbers
%  of these parts with most device drivers.
%  This can be dif\/ficult or impossible if pages in dif\/ferent 
%  parts of the document have the same number---e.\,g. in the 
%  frontmatter and the first text pages (iii vs. 3).
%
%  \TeX{} provides an easy solution to this problem: whenever a page
%  is completed by the output routine and shipped out via 
%  \cs{shipout}, it displayes the values of \Count{0} to \Count{9}
%  on the display (e.\,g. \texttt{[1]}) \emph{and writes them to the
%  \textsf.{dvi} file}.
%
%  \begin{quote}
%     ``The first ten \Count{} registers, \Count{0} to \Count{9},
%     are reserved for a special purpose: \TeX{} displays these ten
%     counts on your terminal whenever ouputting a page, and it 
%     transmits them to the output file as an identification of that
%     page.
%     The counts are separated by decimal points on your terminal, 
%     with trailing `\texttt{.0}' patterns suppressed.
%     Thus, for example, if \Count{0=5} and \Count{2=7} when a
%     page is shipped out to the \texttt{dvi} file, and if the other
%     counters are zero, \TeX{} will type `\texttt{[5.0.7]}'.
%     Plain \TeX{} uses \Count{0} for the page number, and it keeps
%     \Count{1} through \Count{9} equal to zero; that is why you
%     see `\texttt{[1]}' when a page 1 is being output.
%     In more complex applications the page numbers have further 
%     structure; ten counts are shipped out so that there will be
%     plenty of identification.''\cite[p. 119]{KnuthTeXa}
%  \end{quote}
%
%  Surprinsingly, until recently there existed no package for 
%  \LaTeX{} that used these \Count{}ers although some drivers allow 
%  the selection of pages based on other \Count{}s then \Count{0} 
%  (e.\,g. em\TeX).
%
%  This package is the solution: It uses the \textsf{everyshi}
%  package to set \Count{1} to \Count{9} before each \cs{shipout}
%  with these values:
%
%  \begin{center}
%  \DeleteShortVerb{\|}
%  \begin{tabular}{|c|l|}\hline
%  \Count{} & value  \\ \hline
%  0        & relative page number (set by \LaTeX)\\
%  1        & absolute page number\\
%  2        & number of current \cs{part} \\
%  3        & number of current \cs{chapter} (0 with article class)\\
%  4        & number of current \cs{section} \\
%  5        & number of current \cs{subsection} \\
%  6        & number of current \cs{subsubsection} \\
%  7        & number of current \cs{paragraph} \\
%  8        & number of current \cs{subparagraph} \\
%  9        & 1 on odd pages, 0 on even pages\footnotemark\\ \hline
%  \end{tabular}
%  \end{center}
%  \footnotetext{^^A
%     If you have a better application for \Count{9}, let me know.}
%
%
%  \textsf{count1to} also works with classes that do not define some
%  or all of the sectioning commands and their counters, like 
%  \textsf{letter}.
%  Although it is of somewhat little use then \texttt{:-)}.
%
%  A note for users of this package: When you select the pages of
%  some part of your document with a lower structure than 
%  \cs{chapter}, remember that only \cs{part}s and \cs{chapter}s
%  start on a new page; if you want to print a complete
%  \cs{section}, you should also select the first page of the next
%  \cs{section}.
%  Also note that \TeX{} ships out the values of the counters instead
%  of their visual representation (produced with 
%  \cs{the}\emph{counter}), so appendix A sets \Count{4} to 1 in the
%  article class.
%
%  A note for developers of device drivers: Please add support for
%  \Count{1} to \texttt{9} to your programs.
%  It would also be nice if users could easily select the next 
%  page(s) after a certain count (something like 
%  ``\texttt{*.*.*.*.2+1.*}'' should be possible for selecting all
%  pages with \Count{4}~$\mapsto$~section $=$ 2 plus the first 
%  page of section 3).
%
% ^^A -----------------------------
%
%  \section{Options}
%
%  The package has no options.
%
% ^^A -----------------------------
%
%  \section{Required packages}
%
%  The package does require the \textsf{everyshi} package.
%
% ^^A -----------------------------
%
%  \StopEventually{^^A
%     \PrintIndex\PrintChanges
%     ^^A Make sure that the index is not printed twice
%     ^^A (ltxdoc.cfg might have a second \PrintIndex command)
%     \let\PrintChanges\relax
%     \let\PrintIndex\relax
%     }
%
% ^^A -----------------------------
%
%  \section{The implementation}
%
%  \setlength{\parindent}{0pt}
%    \begin{macrocode}
%<*package>
%    \end{macrocode}
%
%  We need the \textsf{everyshi} package.
%    \begin{macrocode}
\RequirePackage{everyshi}[1994/12/09]
%    \end{macrocode}
%
%  We need various \cs{if}s to check if the used counters are
%  defined.
%    \begin{macrocode}
\newif\if@have@part
\newif\if@have@chapter
\newif\if@have@section
\newif\if@have@subsection
\newif\if@have@subsubsection
\newif\if@have@paragraph
\newif\if@have@subparagraph
%    \end{macrocode}
%
%  \begin{macro}{\@countItoIX@ifs}
%  \cs{@countItoIX@ifs} is used to set all these \cs{if}s at
%  \cs{begin\{document\}}.
%    \begin{macrocode}
\newcommand{\@countItoIX@ifs}{
   \@ifundefined{c@part}           {}{\@have@parttrue}
   \@ifundefined{c@chapter}        {}{\@have@chaptertrue}
   \@ifundefined{c@section}        {}{\@have@sectiontrue}
   \@ifundefined{c@subsection}     {}{\@have@subsectiontrue}
   \@ifundefined{c@subsubsection}  {}{\@have@subsubsectiontrue}
   \@ifundefined{c@paragraph}      {}{\@have@paragraphtrue}
   \@ifundefined{c@subparagraph}   {}{\@have@subparagraphtrue}
   }
\AtBeginDocument{\@countItoIX@ifs}
%    \end{macrocode}
%  \end{macro}
%
%  \begin{macro}{\@countItoIX@bugfix}
%  The current release of \LaTeX{} has the ``feature'' that only the 
%  first level of counters associated with a counter via the optional
%  argument of \cs{newcounter} is reset when the counter is stepped;
%  so when you start a new chapter, the number for the subsection is
%  not reset.
%  This is normally no problem, but with this package, it is: If this
%  bug is not fixed or worked-around, then whenever you use
%  \cs{subparagraph} or some other lower sectioning command, and 
%  after that a sectioning command that is at least two levels higher 
%  (e.\,g. \cs{subsubsection}), the value of \texttt{subparagraph} 
%  would not be reset and would still be displayed and shipped out.
%  So we have to associate \emph{all} lower-level sectioning commands
%  with the higher levels.
%  This is done via \cs{@addtoreset} at \cs{begin\{document\}}.
%    \begin{macrocode}
\newcommand{\@countItoIX@bugfix}{
   \if@have@part
      \@addtoreset{section}      {part}
      \@addtoreset{subsection}   {part}
      \@addtoreset{subsubsection}{part}
      \@addtoreset{paragraph}    {part}
      \@addtoreset{subparagraph} {part}
   \fi
   \if@have@chapter
      \@addtoreset{subsection}   {chapter}
      \@addtoreset{subsubsection}{chapter}
      \@addtoreset{paragraph}    {chapter}
      \@addtoreset{subparagraph} {chapter}
   \fi
   \if@have@section
      \@addtoreset{subsubsection}{section}
      \@addtoreset{paragraph}    {section}
      \@addtoreset{subparagraph} {section}
   \fi
   \if@have@subsection
      \@addtoreset{paragraph}    {subsection}
      \@addtoreset{subparagraph} {subsection}
   \fi
   \if@have@subsubsection
      \@addtoreset{subparagraph} {subsubsection}
   \fi
   }
\AtBeginDocument{\@countItoIX@bugfix}
%    \end{macrocode}
%  \end{macro}
%
%  \begin{macro}{\@countItoIX@shipout}
%  \cs{@countItoIX@shipout} is used for setting \Count{1} to 
%  \Count{9} at each \cs{shipout}.
%    \begin{macrocode}
\newcommand{\@countItoIX@shipout}{
%    \end{macrocode}
%  \Count{1} is the absolute page number, which we have to maintain
%  by ourselves.
%    \begin{macrocode}
   \global\advance \count1 by 1
%    \end{macrocode}
%  \Count{2} to \Count{8} are set with the values of
%  \cs{part} to \cs{subparagraph}, if these commands are defined.^^A
%     \footnote{We should probably also check the value of 
%        \texttt{secnumdepth}, but I don't think this is necessary.}
%    \begin{macrocode}
   \if@have@part              \count2\value{part}           \fi
   \if@have@chapter           \count3\value{chapter}        \fi
   \if@have@section           \count4\value{section}        \fi
   \if@have@subsection        \count5\value{subsection}     \fi
   \if@have@subsubsection     \count6\value{subsubsection}  \fi
   \if@have@paragraph         \count7\value{paragraph}      \fi
   \if@have@subparagraph      \count8\value{subparagraph}   \fi
%    \end{macrocode}
%  \Count{9} is set to 1 on odd pages and to 0 on even pages.
%    \begin{macrocode}
   \ifodd\count1\count9=1     \else\count9=0                \fi
   }
\EveryShipout{\@countItoIX@shipout}
%    \end{macrocode}
%  \end{macro}
%
%    \begin{macrocode}
%</package>
%    \end{macrocode}
%
% ^^A -----------------------------
%
%  \section{Acknowledgements}
%
%  As usual Rebecca Stiels improved the quality of this documentation.
%  If you need a translator from English or Fran\c{c}ais to German, 
%  send her an e-mail to \texttt{Rebecca@Andurg.HB.North.DE}.
%  And if you need a \TeX{}nician or computer scientist, send 
%  \emph{me} an e-mail; I'm looking for a job.
%
% ^^A -----------------------------
%
%  \begin{thebibliography}{1}
%     \bibitem{KnuthTeXa}
%        Donald~E.\ Knuth.
%        \newblock \emph{The {\TeX}Book}, volume~A of \emph{Computers
%           and Typesetting}.
%        \newblock Addison-Wes\-ley, 1986.
%        \newblock Eleventh printing, revised, May 1991.
%  \end{thebibliography}
%
% ^^A -----------------------------
%
%  \Finale
