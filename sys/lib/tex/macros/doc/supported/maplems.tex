\documentclass{article}
\usepackage{maplems,palatino}

\title{Export to \LaTeX{} (for \LaTeX{}2e)}
\author{Waterloo Maple Software}
\date{August, 1994}
\begin{document}
\maketitle

\section{Introduction}
Maple V Release 3 introduces the capability of exporting the 
contents of a Maple worksheet into a file suitable for processing 
by the \LaTeX{} document formatting system.

The text regions of the Maple worksheet and the various algebraic
output regions are automatically written to a file along with 
appropriate latex macros so that the contents of the worksheet 
can then be formatted and printed using \LaTeX{}.
Of particular interest is the fact that the math typeset
regions, including the multi-line displays, are reformatted
to the user specified column width using Maple's \LaTeX{} math formatting
capabilities\footnote{
This is an entirely new mechanism based on Maple's high resolution
math formatting technology. It is separate from Maple's library
level {\tt latex(~)} command which can still be used to format
individual expressions.}.  This feature automatically carries out
linebreaking of large mathematical formulae in \LaTeX{}.

The automatically generated \LaTeX{} source document (say ``worksheet.tex'') 
has a structure similar to that shown below.

\begin{verbatim}
\documentclass{article}
\usepackage{maplems}
\begin{document}
First paragraph of text.

Second paragraph of text. This is a test file.
\begin{mapleinput}
(a + b);
\end{mapleinput}
\begin{maplelatex}
\[
{a} + {b}
\]
\end{maplelatex}
We can display mathematics in a variety of forms.
\begin{mapleinput}
Int( 1/(1 + x^4),x);
\end{mapleinput}
\begin{maplelatex}
\[
{\displaystyle \int} {\displaystyle \frac {1}{1 + {x}^{4}}}\,{d}{
x}
\]
\end{maplelatex}
\begin{mapleinput}
value(");
\end{mapleinput}
\begin{maplelatex}
\begin{eqnarray*}
\lefteqn{{\displaystyle \frac {1}{8}}\,\sqrt {2}\,{\rm ln}
 \left( {\vrule height0.89em width0em depth0.89em} \right. \,
{\displaystyle \frac {{x}^{2} + {x}\,\sqrt {2} + 1}{{x}^{2} - {x}
\,\sqrt {2} + 1}}\, \left. {\vrule 
height0.89em width0em depth0.89em} \right)  + 
{\displaystyle \frac {1}{4}}\,\sqrt {2}\,{\rm arctan} \left( 
{\vrule height0.41em width0em depth0.41em} \right. \,{x}\,\sqrt {
2} + 1\, \left. {\vrule height0.41em width0em depth0.41em}
 \right) } \\
 & & \mbox{} + {\displaystyle \frac {1}{4}}\,\sqrt {2}\,{\rm 
arctan} \left( {\vrule height0.41em width0em depth0.41em}
 \right. \,{x}\,\sqrt {2} - 1\, \left. {\vrule 
height0.41em width0em depth0.41em} \right) \mbox{\hspace{88pt}}
\end{eqnarray*}
\end{maplelatex}
And more...   TTY based output is handled as
\begin{mapleinput}
Int((x+a)^3,x=a..b);
\end{mapleinput}
\begin{maplettyout}
                          b
                          /
                         |         3
                         |  (x + a)  dx
                         |
                        /
                        a
\begin{maplettyout}
We can even include a plot.
\mapleplot{worksht.ps1}
\maplesepline
\end{document}
%% End of Maple V Session Output
\end{verbatim}

Maple's text regions from the original worksheet become ordinary
\LaTeX{} paragraphs.  Except for embedded graphics, the other regions
(Maple input, Maple typeset mathematics output, 
and Maple character based mathematics output)
are formatted using an appropriate \LaTeX{} environment. 

If you wish the text regions to be processed as is, with no special
meaning assigned to \LaTeX{} characters such as \verb|_| or \verb|^|, 
use the package option {\em worksheet} as in

\begin{imaplettyout}
\usepackage[worksheet]{maplems}
\end{imaplettyout}

The {\em maplems} style sets up page sizes, headers and footers,
and then uses the style package {\em mapleenv} to 
tell \LaTeX how to format each of these special environments.
Thus, the document can be printed as is.
The document can also be enhanced by editing
in standard \LaTeX{} constructs.
Experienced \LaTeX{} users can redefine these
style macros to achieve specific effects and
Most mathematics publishers have their own style macros
that can be modified to work in conjunction with these
macros.

\section{Printing or Viewing a Maple \LaTeX{} Document}
In order to format and print the document {\tt worksht.tex} 
you need a standard version of \LaTeX{} and the two 
\LaTeX{} style files {\tt maplems.sty} and {\tt mapleenv.sty}
that are provided with this document.  Documents which
are modified by hand to include plots use the {\em epsfig.sty}
\LaTeX{} style package that is part of \LaTeX2e{}. 
It uses the package options {\em dvips} and {\em oztex}
to indicate which postscript driver is being used to handle any
graphics embedded using the {\em mapleplot} macro.

If these style files are not installed in one of the locations
automatically searched by your installation of \LaTeX{} then they 
should appear in the same directory or folder as {\tt worksht.tex}.

To print or view this \LaTeX{} document we first 
produce a ``.dvi'' file.  
On a command based system you would normally type a 
command of the form

{\tt \hspace*{10ex} latex worksht.tex}

On an ``icon'' based system you would start the \LaTeX{} application
and then use its menus to ``open'' the file ``worksht.tex''.

Finally, the resulting file ``worksht.dvi'' is then processed by 
another application such as ``dvips'' to send the document 
to the printer or to preview it on your screen.

\section{How Maple's Export to \LaTeX{} Works}

The content of the original Maple worksheet is 
a sequence of regions.  These are either {\em plain text}
regions, {\em Maple input} regions, {\em Maple tty based output}
(such as error messages) or the 
{\em typeset mathematical output}.

In addition, each platform dependent user interface
permits the embedding of native graphics format files in the worksheet.  
These graphics may be generated by Maple, or by another application.

When a Maple worksheet is ``exported to \LaTeX{}'', the
text based regions, and the typeset mathematics are exported.
All the embedded graphics regions of the worksheet are ignored. 
PostScript graphics can be included directly into your document 
by hand editing the resulting file to include references to plot files
using the \LaTeX{} macro \verb|\mapleplot{}| that is provided
with these styles.

To create a \LaTeX{} version of your Maple worksheet proceed
as follows:

\begin{enumerate}
\item
Open the worksheet from within Maple and ensure that all
the Maple results are up-to-date.  If your worksheet is designed
to execute the Maple commands in sequence top to bottom
you can refresh your worksheet by selecting the {\em Execute
All} option.  (On the Macintosh version of Maple, simply select
the entire worksheet and press enter.)

\item
If your worksheet contains plot commands then re-executing 
the entire worksheet will regenerate those plots in 
separate windows.  You can use the {\tt plotsetup}
command to cause those plot commands to generate plots
to specific files.  Alternatively, in Windows 
or with the Motif interface, they can be saved from
the plot windows in PostScript to files of your choice.

\item
To add these plots to your \LaTeX{} document,
use a plain text editor to insert appropriate 
commands of the form

\begin{verbatim}
\mapleplot{filename.ps}
\end{verbatim}

where {\tt filename.ps} is the user chosen name of the
plot file.  Typically, such commands are added immediately
following the corresponding plot commands.

\item
Use a plain text editor to elaborate on the annotations
that are part of your document.  For example, a new section
heading ``An Example'' can be added directly to the worksheet simply
by inserting a \LaTeX{} command line of the form

\begin{verbatim}
\section{An Example}
\end{verbatim}

directly into the exported document.

\item
Process your \LaTeX{} document with \LaTeX{}.

\end{enumerate}


\section{\LaTeX{} Styles}
The style of your document can be changed simply by choosing
a different \LaTeX{} style.  There are many different available
styles, and most mathematics publishers can 
provide you with their own styles.

For example, to define an alternative document ``style'' 
you simply replace the existing style definition.

\begin{verbatim}
\documentclass{article}
\usepackage{maplems}
\end{verbatim}

\noindent
by (for example)

\begin{verbatim}
\documentclass[fullpage]{article}
\usepackage[mapleenv]
\end{verbatim}

Here, the standard \LaTeX{} ``article'' style has been modified
by the contents of the two style files  {\tt fullpage.sty} and
{\tt mapleenv.sty}.  The style ``mapleenv'' is 
necessary to provide the basic definitions for the 
special Maple environments  but does not affect the
basic page layout.

\subsection{The Maple Style file {\tt maplems.sty}}

The ``Export to \LaTeX'' file is created with the ``article'' style
further modified by the style ``maplems''.  The effect of
using ``maplems'' is to define some special
commands needed by Maple, and to set the default
page layout settings.
They are defined by the contents of the file
\verb|maplems.sty|.  A copy of earlier versions of this file is 
found in the Maple directory   {\tt maple/etc}, (or the equivalent
location on DOS and Macintosh platforms.)
The changes that have been incorporated here are to make the
change to latex2e.

This file has several sections, each setting a different group
of style parameters.  By changing these settings you can
change aspects of the overall style.  
These can be changed by editing a copy of
this file, or they can be reset directly in the ``preamble'' of your
document (i.e., the part between the line 
beginning {\tt \backslash documentclass...}
and the line {\tt \backslash begin\{document\}})
by adding commands such as

\begin{verbatim}
\setlength{\LeftMapleSkip}{0in}
\end{verbatim}

\noindent
(This particular command would cause the indentation of Maple 
commands to be set to 0 inches.)


Portions of these style files are shown below.
The lines beginning with \% are comments to \TeX and \LaTeX.

\subsubsection{The Page Headings}
The page headings are set by the following commands.
Some possible alternative styles are listed as comments.

\begin{verbatim}
% \pagestyle{noheadings}
% \pagestyle{plain}
\pagestyle{myheadings}
\markright{\protect\rule[-5pt]{\linewidth}{1pt}\hspace{-\linewidth}%
{\protect\large Maple V\ \  Release 3}}%
\end{verbatim}

\noindent
For example, to change to the ``plain'' page style add the line
{\tt \backslash pagestyle\{plain\}} to the preamble of 
your document.

\subsubsection{Page Dimensions and spacing}
Next, some of the main page dimensions are reset.

\begin{verbatim}
% main document settings  
%
\topmargin=     -0.2in
\textheight=    8.75in
\textwidth=     6.0in
\headheight=    2.5ex
\headsep=       0.17in
\oddsidemargin= 0.25in
\evensidemargin=\oddsidemargin
\parsep=        2ex		  % space between item paragraphs
\parskip=       1.5ex  		  % space between paragraphs
\end{verbatim}

\noindent
To change these, again use {\tt \backslash setlength} in the
preamble of your document.

\subsubsection{Spacing and Sizes for the Maple Environments}
Finally the \LaTeX{} macros defining the special Maple environments
are read in and their user settable parameters are set to
default values.

\begin{verbatim}
% parameters controlling the special maple environments
%
\input mapleenv.sty
\MaplePrompttrue      % generate a prompt at start of each line?
\MaplePromptString =  {\raise 1pt \hbox{$\scriptstyle>$\space}}
\AboveMapleSkip =     1ex plus 2 pt minus 1 pt
\BelowMapleSkip =     \AboveMapleSkip
\LeftMapleSkip  =     5ex
\AboveMaplePlot =     2\AboveMapleSkip
\BelowMaplePlot =     2\AboveMapleSkip
\MaplePlotHeight =    25ex
\MaplePlotWidth =     1.3\MaplePlotHeight
\edef\MaplePlotAngle{270}
\let\MapleSepLineWidth\linewidth  % \let so that it will be redefined 
				  % properly for narrow environments.
\MapleSepLineHeight = 1pt
\let\MapleFont\tt     % font used for input and ttyout
\let\MapleSize\small  % font size for input and ttyout
\MapleFirstLinefalse  % hides first \cr of \mapleinput
\MapleTab = 8	      % spaces used by the tab character.
\end{verbatim}

\subsection{The Maple Style file {\tt mapleenv.sty}}
Use the style ``mapleenv'' in conjunction with
your chosen style when all that you need is the definitions
of the special regions. In this way, the special Maple environments
are defined, but the basic page style and layout remains
unmodified.

{ \parsep=.5ex
The special Maple environments and macros that are defined are:

\begin{verbatim}
\begin{mapleinput} ... \end{mapleinput}
\end{verbatim}

\begin{verbatim}
\begin{maplettyout} ... \end{maplettyout}
\end{verbatim}

\begin{verbatim}
\begin{maplelatex} ... \end{maplelatex}
\end{verbatim}

\begin{verbatim}
\maplesepline
\end{verbatim}

\noindent
and
\begin{verbatim}
\mapleplot{plotfile.ps}
\end{verbatim}
}

The definitions for these \LaTeX{} environments are parameterized.
By changing the parameters (in the same manner as done 
done in the file {\tt maplems.sty})
you can affect details such as the amount of space above or below
the environments, 
the amount of indentation used, 
the character used to identify prompts., etc.  
This can be done in the preamble of your document.

The two text based environments are special versions of 
\LaTeX{} ``verbatim'' environments while the {\tt maplelatex}
environment is used to permit control of page breaking 
and the spacing above and below the environment.

The {\tt \backslash maplesepline} command is used to create separator lines
such as found in the actual Maple worksheets. It has no arguments.

The {\tt \backslash mapleplot} macro is provided as an example of
how to include graphs.  It has one argument --- the name of the
file containing the PostScript description of the plot.

\section{Summary}
These style files are provided simply as samples of how to define
such \LaTeX{} environments.  You may use them or 
modify them, subject to the conditions outlined in each 
file.  Non-profit redistribution is also permitted.

\end{document}
