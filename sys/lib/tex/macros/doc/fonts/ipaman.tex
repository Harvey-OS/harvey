%%%%%%%%%%%%%%%%%%%%%%%%%%%%%%%%%%%%%%%%%%%%%%%%%%%%%%%%%%%%%%%%%%%%%%%%%%%
%%%%%                International Phonetic Alphabet                  %%%%%
%%%%%                            -- Manual --                         %%%%%
%%%%%%%%%%%%%%%%%%%%%%%%%%%%%%%%%%%%%%%%%%%%%%%%%%%%%%%%%%%%%%%%%%%%%%%%%%%
%%%%%
%%%%% adapted to LaTeX by W. Antweiler, UnivCologneCompCen, 05/28/90
%%%%%
%%%%% adapted by Andrej Brodnik (Andy) for NFSS, University of
%%%%% Waterloo, 11.01.93
%%%%%
%%%%% adapted by Andrej Brodnik and Denis Roegel for NFSS2, CRIN
%%%%% (Centre de Recherche en Informatique de Nancy), F-54506
%%%%% VANDOEUVRE-LES-NANCY, for NFSS2, 16.07.94
%%%%%
\documentstyle[ipa,12pt]{article}
\newcommand{\B}[1]{$\backslash$#1}
\newcommand{\MF}{{\rm\sc{}metafont}}
\newcommand{\wsu}{{\small\rm WSUIPA}}
\newcommand{\ipam}{{\small\rm IPAMACS}}
\newcommand{\ipas}{{\small\rm IPA}}
\newcommand{\PL}{{\em Phonetic Symbol Guide}}

\begin{document}
\title{Using the WSU International Phonetic Alphabet}
\author{Washington State University}
\date{
  January 9, 1990   \\
  January 11, 1993\footnote{Adapted for \LaTeX\ and NFSS} \\
  July 16, 1994\footnote{Adapted for \LaTeX\ and NFSS2}
}
\maketitle

\section{Preface}

This users guide is intended to serve as a help and supplement to the
Washington State University International Phonetic Alphabet (\wsu)
fonts. It is not intended to be a manual on \TeX, \LaTeX, linguistics,
or the correct usage of the characters, accents and diacritics
contained within the font. It is assumed that the user is familiar
with the version of \LaTeX\ he or she is using and with the characters
contained within the \wsu\ font. Because the uses for a phonetic
alphabet font are probably as varied as the many disciplines which use
the characters, this guide is intended to merely show some of the most
common ways of using the \wsu\ font. Surely each person who implements
the font will develop his or her own favorite way of working with and
using the font.

The principal source of information regarding the characters and their
shapes has come from sources within the linguistic field---most
notably, Geoffrey K. Pullum and William A. Ladusaw, whose book, {\em
Phonetic Symbol Guide} was used extensively. The \PL\ not only lists
the International Phonetic Alphabet (IPA) characters and diacritics,
it also gives an enlarged illustration of each character which shows
its baseline, height, depth and x-height. These character
illustrations were invaluable in the design of the \wsu\ \MF\
characters.

Several other people involved in the wonderful world of linguistics
also aided in the design of the \wsu\ font. Their input concerning
everything from character shape, to placement within the font, to
determining which characters should be included, is gratefully
acknowledged. I would especially like to thank Karen Mullen, Associate
Professor of English, University of Louisville, Louisville, Kentucky.
She tested, used the \wsu\ fonts, and took the time to give several
suggestions and recommendations that have improved the font. Also,
Christina A. Thiele, Managing Director, Journal Production Centre,
Carleton University, Ottawa, Ontario, who not only provided a wealth
of phonetic information, but also a wealth of encouragement throughout
the entire font designing process.

Included with the \wsu\ fonts are:
%
\begin{itemize}
  \item The basic \wsu\ font which contains 128 phonetic characters
and/or diacritics in five different point sizes (8, 9, 10, 11 and 12)
and in three typefaces (roman, slanted and bold extended).
  \item Each size and typeface includes a TFM (\TeX\ Font Metric) file
and its related GF, PK or PXL file (after you generate them).
  \item A macro package (\ipam.TEX) for use with the \wsu\ font for
\TeX\ and \LaTeX\ without NFSS users. Note, that this package is not
necessary for NFSS \LaTeX users.
  \item A style file (\ipas.STY) for NFSS \LaTeX\ users, and
  \item three {\em WSUIPA Font User's Guides}. The first one is for
\TeX\ ({\small\rm IPAMAN.TEX} -- the original manual), the second one
for \LaTeX\ without NFSS ({\small\rm IPAMAN.LTX} -- the manual adapted
by W.~Antweiler) and the third one is this one.
\end{itemize}

This guide is divided into three main sections:
%
\begin{enumerate}
  \item an introduction to the \wsu\ font,
  \item a description of the font layout and the characters included
in the font and
  \item how to use \ipas\ more in more sophisticated and in simpler
way.
\end{enumerate}

Although the \wsu\ fonts are written in \MF, the user does not need
experience with or knowledge of \MF\ to use the font.


\section{Introduction to the WSUIPA Font}

The \wsu\ fonts are written in \MF\ and use the same font parameter
values as their Computer Modern (CM) counterparts (cmr and cmbx). In
fact, while designing the character shapes, the computer modern code
was left intact or modified only slightly whenever possible.
Therefore, the \wsu\ fonts are totally compatible with computer modern
fonts---the design and shape of the characters are computer modern
wherever possible. For the characters that are drastically different
from any computer modern characters or symbols, such as the ``Gamma,''
``Baby Gamma'' and the ``Esh,'' Pullum and Ladusaw's \PL\ was used as
the character design standard.

The \wsu\ fonts are obviously not an exhaustive collection of phonetic
or even recognized International Phonetic Alphabet characters. The
characters included in the \wsu\ font were chosen either because they
were listed as a ``major'' entry in Pullum and Ladusaw's \PL, or
because of the recommendations from various people with an interest in
the font.

\PL\ contains what it classifies as ``major'' and ``minor''
characters, based upon whether the symbol is an officially recognized
IPA character or is determined by Pullum and Ladusaw to be a standard
symbol in current American transcriptual practice. \PL\ lists 78
``major'' entries which are not easily attainable in the CM fonts. All
78 of these characters are included in the \wsu\ font, along with 50
``minor'' entries from \PL. The selection of which ``minor'' entries
to include was based on information and requests from various
phoneticians and linguists who contacted WSU during the beginning
stages of the creation of the font.


\section{The Layout of the WSUIPA Font}

The character-grouping pattern followed by Pullum and Ladusaw in their
\PL\ was adopted as the basis for the \wsu\ font layout. The
characters are grouped together according to shape rather than usage.
Therefore, the ``a'' shapes occupy the first positions: '00 through
'04, ``b'' shapes are in positions '05 through '11 and so forth. The
accents and diacritics follow the character shapes and are in the last
positions in the font. This approach was taken because under it was
assumed that under most circumstances, the user would most likely be
accessing the characters with the use of macros rather than by typing
large portions of entirely phonetic text. Aside from keeping like
shapes together, the positioning of the \wsu\ font was fairly
arbitrary.


\section{Character Description}

Within this section each of the \wsu\ characters are shown in a \MF\
``smoke mode'' proof. The octal character position is indicated, along
with the corresponding macro name in \ipam, and the name of the
character given in \PL. Only the roman characters will be illustrated
since the shapes of the slanted and bold extended are the same.

There are several Greek characters included in the \wsu\ font, and
initially it may appear these are merely copies of the corresponding
CM Greek characters. That is true with one very important distinction:
the \wsu\ Greek characters included are not italic. All the lowercase
Greek characters included in the CM fonts are italic which makes them
unsuitable for some situations in phonetics where a non-italic Greek
character is essential.

\newcommand{\D}[1]{#1&{\ipa\char#1}}
\raggedbottom
\begin{center}
\begin{tabular}{|l|c|l|l|}
\hline
\multicolumn{4}{|c|}{\bf WSUIPA Characters}\\
\hline\hline
Char&Char&{\sc IPAMACS}&{\sc Pullum \&\ Ladusaw}\\
Code&    &  Name       &  Name\\
\hline
\D{'00} &\B{inva}  &turned a\\
\D{'01} &\B{scripta} &script a\\
\D{'02} &\B{nialpha} &lowercase non-italic alpha\\
\D{'03} &\B{invscripta} &turned script a\\
\D{'04} &\B{invv} &inverted v\\
\D{'05} &\B{crossb} &crossed b\\
\D{'06} &\B{barb} &barred b\\
\D{'07} &\B{slashb} &slashed b\\
\D{'10} &\B{hookb} &hooktop b\\
\D{'11} &\B{nibeta} &non-italic lowercase beta\\
\D{'12} &\B{slashc} &slashed c\\
\D{'13} &\B{curlyc} &curly-tail c\\
\D{'14} &\B{clickc} &stretched c\\
\D{'15} &\B{crossd} &crossed d\\
\D{'16} &\B{bard} &barred d\\
\D{'17} &\B{slashd} &slashed d\\
\D{'20} &\B{hookd} &hooktop d\\
\D{'21} &\B{taild} &right-tail d\\
\D{'22} &\B{dz} &d-yogh ligature\\
\D{'23} &\B{eth} &eth\\
\D{'24} &\B{scd} &small capital D\\
\D{'25} &\B{schwa} &schwa\\
\D{'26} &\B{er} &right-hook schwa\\
\D{'27} &\B{reve} &reversed e\\
\D{'30} &\B{niepsilon} &non-italic greek epsilon\\
\D{'31} &\B{revepsilon} &reversed non-italic epsilon\\
\D{'32} &\B{hookrevepsilon} &right-hook reversed non-italic epsilon\\
\D{'33} &\B{closedrevepsilon} &closed reversed non-italic epsilon\\
\D{'34} &\B{scriptg} &lowercase variant g\\
\D{'35} &\B{hookg} &hooktop g\\
\D{'36} &\B{scg} &small capital G\\
\D{'37} &\B{nigamma} &non-italic gamma\\
\hline
\end{tabular}
\end{center}
\newpage
\begin{center}
\begin{tabular}{|l|c|l|l|}
\hline
\multicolumn{4}{|c|}{\bf WSUIPA Characters}\\
\hline\hline
Char&Char&{\sc IPAMACS}&{\sc Pullum \&\ Ladusaw}\\
Code&    &  Name       &  Name\\
\hline
\D{'40} &\B{ipagamma} &IPA Gamma\\
\D{'41} &\B{babygamma} &baby gamma\\
\D{'42} &\B{hv} &h-v ligature\\
\D{'43} &\B{crossh} &crossed h\\
\D{'44} &\B{hookg} &hooktop g\\
\D{'45} &\B{hookheng} &hooktop heng\\
\D{'46} &\B{invh} &turned h\\
\D{'47} &\B{bari} &barred i\\
\D{'50} &\B{dlbari} &barred dotless i\\
\D{'51} &\B{niiota} &non-italic greek iota\\
\D{'52} &\B{sci} &small capital I\\
\D{'53} &\B{barsci} &barred small capital I\\
\D{'54} &\B{invf} &barred dotless j\\
\D{'55} &\B{tildel} &l with tilde\\
\D{'56} &\B{barl} &barred l\\
\D{'57} &\B{latfric} &belted l\\
\D{'60} &\B{taill} &l with right tail\\
\D{'61} &\B{lz} &l-yogh ligature\\
\D{'62} &\B{nilambda} &non-italic greek lambda\\
\D{'63} &\B{crossnilambda} &crossed lambda\\
\D{'64} &\B{labdentalnas} &m with leftward tail at right\\
\D{'65} &\B{invm} &turned m\\
\D{'66} &\B{legm} &turned m with long right leg\\
\D{'67} &\B{nj} &n with leftward hook at left\\
\D{'70} &\B{eng} &eng\\
\D{'71} &\B{tailn} &n with right tail\\
\D{'72} &\B{scn} &small capital N\\
\D{'73} &\B{clickb} &bull's eye\\
\D{'74} &\B{baro} &barred o\\
\D{'75} &\B{openo} &open o\\
\D{'76} &\B{niomega} &non-italic lowercase greek omega\\
\D{'77} &\B{closedniomega} &closed omega\\
\hline
\end{tabular}
\end{center}
\newpage
\begin{center}
\begin{tabular}{|l|c|l|l|}
\hline
\multicolumn{4}{|c|}{\bf WSUIPA Characters}\\
\hline\hline
Char&Char&{\sc IPAMACS}&{\sc Pullum \&\ Ladusaw}\\
Code&    &  Name       &  Name\\
\hline
\D{'100} &\B{oo} &double o\\
\D{'101} &\B{barp} &barred p\\
\D{'102} &\B{thorn} &thorn\\
\D{'103} &\B{niphi} &non-italic lowercase greek phi\\
\D{'104} &\B{flapr} &fish hook r\\
\D{'105} &\B{legr} &r with long leg\\
\D{'106} &\B{tailr} &r with right tail\\
\D{'107} &\B{invr} &turned r\\
\D{'110} &\B{tailinvr} &turned r with right tail\\
\D{'111} &\B{invlegr} &turned long-legged r\\
\D{'112} &\B{scr} &small capital R\\
\D{'113} &\B{invscr} &inverted small capital R\\
\D{'114} &\B{tails} &s with right tail\\
\D{'115} &\B{esh} &esh\\
\D{'116} &\B{curlyesh} &curly-tail esh\\
\D{'117} &\B{nisigma} &non-italic lowercase greek sigma\\
\D{'120} &\B{tailt} &t with right tail\\
\D{'121} &\B{tesh} &t-esh ligature\\
\D{'122} &\B{clickt} &turned t\\
\D{'123} &\B{nitheta} &non-italic lowercase greek theta\\
\D{'124} &\B{baru} &barred u\\
\D{'125} &\B{slashu} &slashed u\\
\D{'126} &\B{niupsilon} &non-italic lowercase greek upsilon\\
\D{'127} &\B{scu} &small capital U\\
\D{'130} &\B{barscu} &barred small capital U\\
\D{'131} &\B{scriptv} &script v\\
\D{'132} &\B{invw} &inverted w\\
\D{'133} &\B{nichi} &non-italic lowercase greek chi\\
\D{'134} &\B{invy} &turned y\\
\D{'135} &\B{scy} &small capital Y\\
\D{'136} &\B{curlyz} &curly-tail z\\
\D{'137} &\B{tailz} &z with right tail\\
\hline
\end{tabular}
\end{center}
\newpage
\begin{center}
\begin{tabular}{|l|c|l|l|}
\hline
\multicolumn{4}{|c|}{\bf WSUIPA Characters}\\
\hline\hline
Char&Char&{\sc IPAMACS}&{\sc Pullum \&\ Ladusaw}\\
Code&    &  Name       &  Name\\
\hline
\D{'140} &\B{yogh} &yogh\\
\D{'141} &\B{curlyyogh} &curly-tail yogh\\
\D{'142} &\B{glotstop} &glottal stop\\
\D{'143} &\B{revglotstop} &reversed glottal stop\\
\D{'144} &\B{invglotstop} &inverted glottal stop\\
\D{'145} &\B{ejective} &ejective\\
\D{'146} &\B{reveject} &reversed ejective\\
\D{'147} &\B{dental{\char'043}1} &subscript bridge\\
\D{'150} &\B{stress} &vertical stroke (superior)\\
\D{'151} &\B{secstress} &vertical stroke (inferior)\\
\D{'152} &\B{syllabic} &syllabicity mark\\
\D{'153} &\B{corner} &corner\\
\D{'154} &\B{upt} &IPA pointer\\
\D{'155} &\B{downt} &IPA pointer\\
\D{'156} &\B{leftt} &IPA pointer\\
\D{'157} &\B{rightt} &IPA pointer\\
\D{'160} &\B{halflength} &half-length mark\\
\D{'161} &\B{length} &length mark\\
\D{'162} &\B{underdots} &subscript umlaut\\
\D{'163} &\B{ain} &reversed apostrophe\\
\D{'164} &\B{upp} &pointer\\
\D{'165} &\B{downp} &pointer\\
\D{'166} &\B{leftp} &pointer\\
\D{'167} &\B{rightp} &pointer\\
\D{'170} &\B{overring} &over-ring\\
\D{'171} &\B{underring} &under-ring\\
\D{'172} &\B{open} &subscript left half-ring\\
\D{'173} &\B{midtilde} &superimposed (mid-) tilde\\
\D{'174} &\B{undertilde} &subscript tilde\\
\D{'175} &\B{underwedge} &subscript wedge\\
\D{'176} &\B{polishhook} &polish hook\\
\D{'177} &\B{underarch} &subscript arch\\
\hline
\end{tabular}
\end{center}
\newpage
\flushbottom


\section{Using the {\tt{}ipa.sty} Style File}

In addition to the necessary font-related files, the \wsu\ font
package also includes a file called {\ipas.STY}. This is a macro style
file which supports \LaTeX\ under NFSS (new font selection scheme). It
greatly simplifies the usage of new font as it will be seen in this
section. The file also defines mnemonics for simpler usage of
characters from a new font. They are the same as ones given in tables
in the previous section.

If you are using IPA font all what you need to do is to mention \ipas\
style file in the header of your \LaTeX\ source. For example:
%
\begin{verbatim}
\documentstyle[ipa]{article}
\end{verbatim}

The \ipas\ style file defines a new font family \B{ipa}, but you
don't need to know this, because you will use it only through the
mnemonics.

Because the new font is installed under NFSS, the system will also
always choose the correct shape, series, and size whenever you wish to
use it. For example:
%
\begin{quote}\begin{verbatim}
A shibilant is a term occasionally found for a
fricative corresponding to a ``hushing'' sound,
e.g., IPA [\esh] (more technically, a grooved
laminal fricative with a sign {\large\bf \esh}.).
\end{verbatim}\end{quote}
%
which will print as:
%
\begin{quote}
A shibilant is a term occasionally found for a
fricative corresponding to a ``hushing'' sound,
e.g., IPA [\esh] (more technically, a grooved
laminal fricative with a sign {\large\bf \esh}.).
\end{quote}
%
or even
%
\begin{quote}\begin{verbatim}
\dots which is the case in the aforementioned
instances, however, {\large\bf we get the other
retroflex consonants:
[\nj], [\taill], [\taild], and [\tailr]}. They
are printed {\small\sl also as
[\nj], [\taill], [\taild], and [\tailr]}.
\end{verbatim}\end{quote}
%
will print as,
%
\begin{quote}
\dots which is the case in the aforementioned
instances, however, {\bf\large we get the
other retroflex consonants:
[\nj], [\taill], [\taild], and [\tailr]}. They
are printed {\sl\small also as
[\nj], [\taill], [\taild], and [\tailr]}.
\end{quote}
%
%
This means that you can freely change sizes, series, shapes and fonts
(assuming that there exists a wanted font -- note however, that IPA
font comes only in bold extended and medium series, and in normal and
slanted shape.)

A number of \wsu\ diacritics can be used on top or the bottom of other
characters. To handle this \ipas\ provides two macros: \B{diatop} and
\B{diaunder}. The first one (\B{diatop}) has the syntax:
%
\begin{center}
\verb+\diatop[diacritics|character]+
\end{center}
%
It takes one argument which is delimited by square brackets ([])
rather than curly braces, and has two parts.\footnote{From the \LaTeX\
perspective more correctly would be to use curly braces (\{\}) instead
of square brackets ([]), but we left the square brackets to remain
compatible with the previous versions of the \ipam\ file.}
%
The first part of the argument is delimited, or separated, from the
second part by a vertical bar ($\vert$). The macro \B{diatop} puts the
first part of the argument over the second.

Using \B{diatop} we can put a ring over another character by,
%
\begin{quote}\begin{verbatim}
The overring may be used over letters with descenders
as an alternative to under-ring to indicate devoicing,
e.g. [\diatop[\overring|g]].
\end{verbatim}\end{quote}
%
and the output would be,
%
\begin{quote}
The overring may be used over letters with descenders as an
alternative to under-ring to indicate devoicing, e.g.
[\diatop[\overring|g]].
\end{quote}

The \B{diatop} macro also allows placement more than one character
stacked over another character. For example,
%
\begin{quote}\begin{verbatim}
For a really special \diatop[{\diatop[\'|\overring]}|n]
\end{verbatim}\end{quote}
%
will print as:
%
\begin{quote}
For a really special \diatop[{\diatop[\'|\overring]}|n]
\end{quote}
%
Notice the use of curly braces to group the argument of the first \B{diatop}
when more than one \B{diatop} command is used.

The second macro (\B{diaunder}) has a similar syntax:
%
\begin{center}
\verb+\diaunder[diacritics|character]+
\end{center}
%
As \B{diatop} it also takes one argument which is delimited by square
brackets, and has two parts separated by a vertical bar ($\vert$). It
puts the first part of the argument under the second.

Using \B{diaunder} we can put \underring under r as in:
%
\begin{quote}\begin{verbatim}
A voiceless trilled r \diaunder[\underring|r]
in certain Scottish dialects.
\end{verbatim}\end{quote}
%
which will print as:
%
\begin{quote}
A voiceless trilled r \diaunder[\underring|r]
in certain Scottish dialects.
\end{quote}
%

Furthermore, it is also possible to get one or more accents over a
character and another accent or character under it. For example:
%
\begin{quote}\begin{verbatim}
This is a really, really special
\diatop[\overring|{\diaunder[\underring|r]}]
\end{verbatim}\end{quote}
%
will print as:
%
\begin{quote}
This is a really, really special
\diatop[\overring|{\diaunder[\underring|r]}]
\end{quote}

\section{Historical remarks}

The first version of this manual was written by W.\ Antweiler of the
University of Cologne Computing Centre, Robert-Koch-Str.\ 10, D-5000
K\"{o}ln 41, Germany, e-mail: {\tt a0062@uvax.rrz.uni-koeln.de}. This
manual did not include information about the original author. All
fonts along with the \TeX-macros and the original manual can be
retrieved via anonymous ftp from {\tt ymir.claremont.edu}.

The next version consisted of a changed manual and new \LaTeX\ style
file {\tt ipa.sty}, which supported NFSS. The style file was based on
previous \TeX-macros. It with an accompanying WSUIPA manual was
prepared by Andrej Brodnik (Andy) of the University of Waterloo,
Department of Computer Science, 200 University Avenue West, Waterloo,
Ontario, N2L 3G1, Canada, e-mail: {\tt abrodnik@uwaterloo.ca}.

This version of the manual is almost the same as the previous one,
only style file was upgraded for \LaTeX\ NFSS2. They were prepared by
Denis Roegel of CRIN (Centre de Recherche en Informatique de Nancy),
F-54506 Vandoeuvre-les-Nancy, France, e-mail: {\tt roegel@loria.fr}
and Andrej Brodnik.

\end{document}


