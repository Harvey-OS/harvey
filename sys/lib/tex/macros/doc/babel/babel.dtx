% \iffalse meta-comment
%
% Copyright (C) 1989-1994 by Johannes Braams
% All rights reserved.
% For additional copyright information see further down in this file.
% 
% This file is part of the Babel system, release 3.4 patchlevel 2.
% ----------------------------------------------------------------
% 
% This file is distributed in the hope that it will be useful,
% but WITHOUT ANY WARRANTY; without even the implied warranty of
% MERCHANTABILITY or FITNESS FOR A PARTICULAR PURPOSE.
% 
% 
% IMPORTANT NOTICE:
% 
% For error reports in case of UNCHANGED versions see readme file.
% 
% Please do not request updates from me directly.  Distribution is
% done through Mail-Servers and TeX organizations.
% 
% You are not allowed to change this file.
% 
% You are allowed to distribute this file under the condition that
% it is distributed together with all files mentioned in manifest.txt.
% 
% If you receive only some of these files from someone, complain!
% 
% You are NOT ALLOWED to distribute this file alone.  You are NOT
% ALLOWED to take money for the distribution or use of either this
% file or a changed version, except for a nominal charge for copying
% etc.
% \fi
% \CheckSum{1088}
%%%\iffalse
%%%  @LaTeX-file{
%%%     author          = "Johannes L. Braams",
%%%     version         = "3.4f",
%%%     date            = "03 July 1994",
%%%     time            = "23:33:58 MET",
%%%     filename        = "babel.dtx",
%%%     address         = "PTT Research
%%%                        St. Paulusstraat 4
%%%                        2264 XZ Leidschendam
%%%                        The Netherlands",
%%%     telephone       = "(70) 3325051",
%%%     FAX             = "(70) 3326477",
%%%     checksum        = "60652 2080 9803 80663",
%%%     email           = "J.L.Braams@research.ptt.nl (Internet)",
%%%     codetable       = "ISO/ASCII",
%%%     keywords        = "babel",
%%%     supported       = "yes",
%%%     docstring       = "This file, babel.dtx contains the core of
%%%                        the babel system as it was described in
%%%                        TuGboat Volume 12 no 2 and Volume 14 no 1.
%%%
%%%                        The checksum field above contains a CRC-16
%%%                        checksum as the first value, followed by the
%%%                        equivalent of the standard UNIX wc (word
%%%                        count) utility output of lines, words, and
%%%                        characters.  This is produced by Robert
%%%                        Solovay's checksum utility.",
%%%      }
%%% \fi
%%
% \def\filename{babel.dtx}
% \let\thisfilename\filename
%
%\iffalse
%
% Babel DOCUMENT-STYLE option for LaTeX version 2.09 or plain TeX;
% Babel package for LaTeX2e.
%
%% Copyright (C) 1989-1994 by Johannes Braams,
%%                            PTT Research Neher Laboratories
%%                            all rights reserved.
%
%% Please report errors to: J.L. Braams
%%                          J.L.Braams@research.ptt.nl
%<*filedriver>
\documentclass[english]{ltxdoc}
\usepackage{babel}
\GetFileInfo{babel.sty}
\newcommand{\TeXhax}{\TeX hax}
\newcommand{\babel}{\text\sf{babel}}
\newcommand{\ttbs}{\char'134}
\newcommand{\m}[1]{\mbox{$\langle$\it#1\/$\rangle$}}
\newcommand{\langvar}{\m{lang}}
\newcommand{\note}[1]{}
\newcommand{\bsl}{\protect\bslash}
\newcommand{\Lopt}[1]{\textsf{#1}}
\newcommand{\file}[1]{\texttt{#1}}
\begin{document}
 \DocInput{babel.dtx}
\end{document}
%</filedriver>
%\fi
%
% \changes{babel~2.0a}{2 apr 90}{Added text about german.sty}
% \changes{babel~2.0b}{18 apr 90}{Changed order of code to prevent
%    plain \TeX from seeing all of it}
% \changes{babel~2.1}{24 apr 90}{Modified user interface, {\tt\bsl
%    langTeX} no longer necessary}
% \changes{babel~2.1a}{1 may 90}{Incorporated Nico's comments}
% \changes{babel~2.1b}{1 may 90}{rename {\tt\bsl language} to {\tt\bsl
%    current@language}}
% \changes{babel~2.1c}{22 may 90}{abstract for report fixed, missing
%    {\tt \}}, found by Nicolas Brouard BROUARD@FRINED51.BITNET}
% \changes{babel~2.1d}{4 july 90}{Missing right brace in definition of
%    abstract environment, found by Werenfried Spit}
% \changes{babel~2.1e}{16 july 90}{Incorporated more comments from
%    Nico}
% \changes{babel~2.2}{17 july 90}{Renamed {\tt\bsl newlanguage} to
%    {\tt\bsl addlanguage}}
% \changes{babel~2.2a}{27 august 90}{Modified the documentation
%    somewhat}
% \changes{babel~3.0}{23 april 91}{Moved part of the code to
%    hyphen.doc in preparation for \TeX~3.0}
% \changes{babel~3.0a}{21 may 91}{Updated comments in various places}
% \changes{babel~3.0b}{25 may 91}{Removed some problems in change log}
% \changes{babel~3.0c}{15 july 91}{Renamed babel.sty and latexhax.sty
%    to .com}
% \changes{babel~3.1}{31 october 91}{Added the support for active
%    characters and for extending a macro}
% \changes{babel~3.1}{5 november 91}{Removed the need for {\tt
%    latexhax}}
% \changes{babel~3.2}{10 november 91}{Some Changes by br}
% \changes{babel~3.2a}{15 february 92}{Fixups of the code and
%    documentation}
% \changes{babel~3.3}{6 july 93}{Included driver file, and prepared
%    for dsitribution}
% \changes{babel~3.4}{1994/01/30}{Updated for LaTeX2e}
% \changes{babel~3.4}{1994/02/28}{Added language specific file for
%    bahasa}
% \changes{babel~3.4b}{1994/05/18}{Added a small driver to be able to
%    process just this file}
%
% \title {Babel, a multilingual package for use with \LaTeX's standard
%    document classes\thanks{During the development ideas from Nico
%    Poppelier, Piet van Oostrum and many others have been used.
%    Bernd Raichle has provided many helpful suggestions.}}
%
% \author{Johannes Braams\\
%         PTT Research Neher Laboratories\\
%         P.O. Box 421\\
%         2260 AK Leidschendam\\
%         {\tt J.L.Braams@research.ptt.nl}}
%
% \date{Printed \today}
%
% \maketitle
%
% \begin{abstract}
%    The standard distribution of \LaTeX\ contains a number of
%    document classes that are meant to be used, but also serve as
%    examples for other users to create their own document classes.
%    These document classes have become very popular among \LaTeX\
%    users. But it should be kept in mind that they were designed for
%    American tastes and typography. At one time they contained a
%    number of hard-wired texts. This report describes \babel{}, a
%    package that makes use of the new capabilities of \TeX\ version 3
%    to provide an environment in which multilingual documents can be
%    written.
% \end{abstract}
%
% \section{The user interface}\label{U-I}
%
%    The user interface of this package is quite simple. It basially
%    consists of only two commands. These commands can be used to
%    select another language or to find out what the current language
%    is.
%
% \DescribeMacro\selectlanguage When a user wants to switch from one
%    language to another he can do so using the macro
%    |\selectlanguage|. This macro takes the language, defined
%    previously by a language-specific option, as its argument. It
%    calls several macros that should be defined in the
%    language-specific files to activate the special definitions for
%    the language chosen.
%
% \DescribeMacro\languagename The name of the current language is
%    stored in the control sequence |\languagename|.
%
% \DescribeMacro\iflanguage
%    If more than one language is used it might be necessary to know
%    which language is active at a specific time. This can be checked
%    by a call to |\iflanguage|. This macro takes three arguments.
%    The first argument is the name of a language,
%    the second and third arguments are the actions
%    to take if the result of the test is {\tt true} or {\tt false}
%    respectively.
%
% \section{Changes for \LaTeXe}
%
%    With the advent of \LaTeXe\ the interface to \babel\ in the
%    preamble of the doument has changed. With \LaTeX2.09 one used to
%    call up the \babel\ system with a line such as:
%
%\begin{verbatim}
%\documentstyle[dutch,english]{article}
%\end{verbatim}
%
%    which would tell \LaTeX\ that the document would be written in
%    two languages, dutch and english and that english would be the
%    first language in use.
%
%    The \LaTeX2e\ way of providing the same information is:
%
%\begin{verbatim}
%\documentlass{article}
%\usepackage[dutch,english]{babel}
%\end{verbatim}
%
%    or, making \Lopt{dutch} and \Lopt{english} global options in
%    order to let other packages detect and use them:
%
%\begin{verbatim}
%\documentlass[dutch,english]{article}
%\usepackage{babel}
%\usepackage{varioref}
%\end{verbatim}
%
%    In this last example the package \texttt{varioref} will also see
%    the options and will be able to use them.
%
% \section{The interface between the core of \babel{} and
%          the language-specific options}
%
%    In the core of the \babel{} system two macros are defined that
%    are to be used in language specific files. Their purpose is to
%    make a new language known.
%
% \DescribeMacro\addlanguage
%    The macro |\addlanguage| is a non-outer version of the macro
%    |\newlanguage|, defined in \file{plain.tex} version~3.x. For
%    older versions of \file{plain.tex} and \file{lplain.tex} a
%    substitute definition is used.
%
% \DescribeMacro\adddialect The macro |\adddialect| can be used in the
%    case where two languages can (or have to) use the same
%    hyphenation patterns. This can be useful when a user wants to use
%    a language for which no patterns are preloaded in the format. In
%    such a case the default behaviour of the \babel{} system is to
%    define this language as a `dialect' of the language for which the
%    patterns were loaded as |\language0|.
%
%    The language-specific files have to conform to a number of
%    conventions. The reason for this is that these files have to fill
%    in the gaps left by the common code in \\file{babel.def}, i.\,e.,
%    the definitions of the macros that produce texts.  Also the
%    language-switching possibility which has been built into the
%    \babel{} system has its implications.
%
%    The following assumptions are made:
%   \begin{itemize}
%    \item Some of the language-specific definitions might be used by
%    plain \TeX\ users, so the files have to be coded such that they
%    can be read by \LaTeX\ as well as by plain \TeX. This can be
%    checked by looking at the value of the macro |\fmtname|.
%
%    \item The common part of the \babel{} system redefines a number
%    of macros and environments (defined previously in the document
%    style) to put in the names of macros that replace the previously
%    hard-wired texts.  These macros have to be defined in the
%    language-specific files.
%
%    \item The language-specific files define four macros, used to
%    activate and deactivate the language-specific definitions.  These
%    macros are {\tt\ttbs captions\langvar}, {\tt\ttbs date\langvar},
%    {\tt\ttbs extras\langvar} and {\tt\ttbs noextras\langvar}. These
%    macros and their functions are discussed below.
%
%    \item When a language-specific file is loaded as an option, it
%    can define {\tt\ttbs l@\langvar} to be a dialect of |\language0|
%    when {\tt\ttbs l@\langvar} is undefined.
%
%    \item The language-specific files can be read as an option,
%    but also in the middle of document processing. This means that
%    they have to function independently of the current |\catcode| of
%    the {\tt @}~sign.
%   \end{itemize}
%
%
%    \DescribeMacro\captionslang
%    The macro {\tt\ttbs captions\langvar} defines the macros that
%    hold the texts to replace the original hard-wired texts.
%    \DescribeMacro\datelang
%    The macro {\tt\ttbs date\langvar} defines |\today| and
%    \DescribeMacro\extraslang
%    the macro {\tt\ttbs extras\langvar} contains all the extra
%    definitions needed for a specific language.
%
%    \DescribeMacro\noextraslang
%    Because we want to offer the user the possibility to switch
%    between languages and we do not know in what state \TeX\ might be
%    after the execution of {\tt\ttbs extras\langvar}, a macro that
%    brings \TeX\ into a predefined state is needed. It will be no
%    surprise that the name of this macro is {\tt\ttbs
%    noextras\langvar}.
%
% \subsection{Support for active characters}
%
%    In quite a few language specific files, active characters are
%    introduced. To facilitate this, some support macros are provided.
%
% \DescribeMacro{\babel@add@special}
% \DescribeMacro{\babel@remove@special}
%    ``Plain \TeX\ includes a macro called |\dospecials| that is
%    essentially a set macro, representing the set of all characters
%    that have a special category code.'' \cite[p.~380]{DEK} It is
%    used to set text `verbatim'.  To make this work if more
%    characters get a special category code, you have to add this
%    character to the macro |\dospecial|.  \LaTeX\ adds another macro
%    called |\@sanitize| representing the same character set, but
%    without the curly braces.  The macros
%    |\babel@add@special|\meta{char} and
%    |\babel@remove@special|\meta{char} add and remove the character
%    \meta{char} to these two sets.
%
% \subsection{Support for saving macro definitions}
%
%    Language specific files may want to {\em re\/}define macros that
%    already exist. Therefore a mechanism for saving (and restoring)
%    the original definition of those macros is provided. We provide
%    two macros for this\footnote{This mechanism was introduced by
%    Bernd Raichle.}.
%
% \DescribeMacro{\babel@save} To save the current meaning of any
%    control sequence the macro |\babel@save| is provided. It takes
%    one argument, \meta{csname}, the control sequence for which the
%    meaning has to be saved.
%
% \DescribeMacro{\babel@savevariable} A second macro is provided to
%    save the current value of a variable.  In this context anything
%    that is allowed after the |\the| primitive is considered to be a
%    variable. The macro takes one argument, the \meta{variable}.
%
%    The effect of the aforementioned macros is that a piece of code
%    is appended to the current definition of |\originalTeX|. When
%    |originalTeX| is expanded this code restores the previous
%    definition of the control sequence or the previous value of the
%    variable.
%
% \subsection{Support for extending macros}
%
% \DescribeMacro{\addto}
%    The macro |\addto{|\meta{control sequence}{\tt}{}\meta{\TeX\
%    code}{\tt\}} can be used to extend the definition of a macro. The
%    macro need not be defined. This macro can, for instance, be used
%    in adding instructions to a macro like |\extrasenglish|.
%
% \subsection{Macros common to a number of languages}
%
% \DescribeMacro{\allowhyphens}
%    In a couple of european languages compound words are used. This
%    means that when \TeX\ has to hyphenate such a compound word it
%    only does that at the `{\tt-}' that is used in such words. To
%    allow hyphenation in the rest of such a compound word the macro
%    |\allowhyphens| can be used.
%
% \DescribeMacro{\set@low@box}
%    For some languages quotes need to be lowered to the baseline. For
%    this purpose the macro |\set@low@box| is available. It takes one
%    argument and puts that argument in an |\hbox|, at the
%    baseline. The result is available in |\box0| for further
%    processing.
%
% \DescribeMacro{\save@sf@q}
%    Sometimes it is necessary to preserve the |\spacefactor|.  For
%    this purpose the macro |\save@sf@q| is available. It takes one
%    argument, saves the current spacefactor, executes the argument
%    and restores the spacefactor.
%
% \section{Compatibility with {\tt german.sty}}\label{l-h}
%
%    As has been discussed before, the file \file{german.sty} has been
%    one of the sources of inspiration for the \babel{}
%    system. Because of this I wanted to include \file{german.sty} in
%    the \babel{} system.  To be able to do that I had to allow for
%    one incompatibility: in the definition of the macro
%    |\selectlanguage| in file{german.sty} the argument is used as the
%    {$\langle \it number \rangle$} for an |\ifcase|. So in this case
%    a call to |\selectlanguage| might look like
%    |\selectlanguage{\german}|.
%
%    In the definition of the macro |\selectlanguage| in
%    \file{babel.def} the argument is used as a part of other
%    macronames, so a call to |\selectlanguage| now looks like
%    |\selectlanguage{german}|.  Notice the absence of the escape
%    character.  As of version~3.1a of \babel{} both syntaxes are
%    allowed.
%
%    All other features of the original \file{german.sty} have been
%    copied into a new file, called \file{germanb.sty}\footnote{The
%    `b' is added to the name to distinguish the file from Partls'
%    file.}.
%
%    Although the \babel{} system was developed to be used with
%    \LaTeX, some of the features implemented in language specific
%    files might be needed by plain \TeX\ users. Care has been taken
%    that all files in the system can be processed by plain \TeX.
%
%
%\StopEventually{%
% \clearpage
% \let\filename\thisfilename
% \section{Conclusion}
%
%    A system of document options has been presented that enable the
%    user of \LaTeX\ to adapt the standard document classes of \LaTeX\
%    to the language he or she prefers to use. These options offer the
%    possibility to switch between languages in one document. The
%    basic interface consists of using ones option, which is the same
%    for {\em all\/} standard document classes.
%
%    In some cases the language specific option files provide macros
%    that can be of use to plain \TeX\ users as well as to \LaTeX\
%    users.  The \babel{} system has been implemented in such a way
%    that it can be used by both groups of users.
%
% \section{Acknowledgements}
%
%    I would like to thank all who volunteered as $\beta$-testers for
%    their time. I would like to mention Julio Sanchez who supplied
%    the option file for the Spanish language and Maurizio Codogno who
%    supplied the option file for the Italian language. Werenfried
%    Spit supplied the files for the Russian language. Michel Goossens
%    supplied contributions for most of the other languages.  Nico
%    Poppelier helped polishing the text of the documentation and
%    supplied parts of the macros for the Dutch language.  Paul
%    Wackers and Werenfried Spit helped finding and repairing bugs.
%
%    During the further development of the babel system I received
%    much help from Bernd Raichle, for which I am grateful.
%
% \begin{thebibliography}{9}
%  \bibitem{DEK} Donald E. Knuth,
%    {\em The \TeX book}, Addison-Wesley, 1986.
%  \bibitem{LLbook} Leslie Lamport,
%    {\em \LaTeX, A document preparation System}, Addison-Wesley, 1986.
%   \bibitem{treebus} K.F. Treebus.
%   {\em Tekstwijzer, een gids voor het grafisch verwerken van tekst.}
%   SDU Uitgeverij ('s-Gravenhage, 1988). A Dutch book on layout
%   design and typography.
%  \bibitem{HP} Hubert Partl,
%    {\em German \TeX}, {\em TUGboat\/} 9 (1988) \#1, p.~70--72.
%  \bibitem{LLth} Leslie Lamport,
%    in: \TeXhax\ Digest, Volume 89, \#13, 17 februari 1989.
%  \bibitem{BEP} Johannes Braams, Victor Eijkhout and Nico Poppelier,
%   {\em The development of national \LaTeX\ styles\/},
%   {\em TUGboat\/} 10 (1989) \#3, p.~401--406.
%  \bibitem{ilatex} Joachim Schrod,
%   {\em International \LaTeX\ is ready to use\/},
%   {\em TUGboat\/} 11 (1990) \#1, p.~87--90.
% \end{thebibliography}
% }
%
% \section{Identification}
%
%    The file \texttt{babel.sty}\footnote{The file described in this
%    section is called {\tt \filename}, has version
%    number~\fileversion\ and was last revised on~\filedate.} is meant
%    for \LaTeXe, therefore we make sure that the format file used is
%    the right one.
%
%    \begin{macrocode}
%<+package>\NeedsTeXFormat{LaTeX2e}
%    \end{macrocode}
%
% \changes{babel~3.4c}{1994/06/05}{lhyphen.cfg has become
%    lthyphen.cfg}
%    The identification code for each file is something that was
%    introduced in \LaTeXe. When the command |\ProvidesFile| does not
%    exist, a dummy definition is provided.
% \changes{babel~3.4e}{1994/06/24}{Redid the identification code,
%    provided dummy definition of \cs{ProvidesFile} for plain \TeX}
%    \begin{macrocode}
%<*!package>
\ifx\ProvidesFile\undefined
  \def\ProvidesFile#1[#2 #3 #4]{\wlog{#4 #3 <#2>}}\fi
%</!package>
%    \end{macrocode}
%    Identify each file that is produced from this source file.
%    \begin{macrocode}
%<+package>\ProvidesPackage{babel}
%<+def>\ProvidesFile{babel.def}
%<+kernel|patterns>\ProvidesFile{lthyphen.cfg}
%<+kernel&!patterns>\ProvidesFile{switch.def}
%<+driver&!user>\ProvidesFile{babel.drv}
%<+driver&user>\ProvidesFile{user.drv}
                [1994/07/03 v3.4f
%<+package>     The Babel package]
%<+def>         Babel common definitions]
%<+kernel>      Babel language switching mechanism]
%<+driver>]
%    \end{macrocode}
%
% \section{The Package File}
%
%    In order to make use of the new features of \LaTeXe, a new file
%    is introdued to the \babel\ system, \file{babel.sty}. This file
%    is loaded by the |\usepackage| command and defines all the
%    language options known in the \babel system.
%
%    For all the languages supported we need to declare an option.
%
%    `American' is a version of `English' which can have its own
%    hyphenation patterns. The default english patterns are in fact
%    for  american english. We allow or the patterns to be loaded as
%    `english' `american' or `USenglish'.
%    \begin{macrocode}
%<*package>
\DeclareOption{american}{%
  \ifx\l@USenglish\undefined
  \else
    \let\l@american\l@USenglish
  \fi
  %% 
%% This is file `english.sty', generated 
%% on <1995/2/23> with the docstrip utility (2.2i).
%% 
%% The original source files were:
%% 
%% english.dtx 
%% 
%% Copyright (C) 1989-1994 by Johannes Braams 
%% All rights reserved. 
%% For additional copyright information see further down in this file. 
%% 
%% This file is part of the Babel system, release 3.4 patchlevel 2. 
%% ---------------------------------------------------------------- 
%% 
%% This file is distributed in the hope that it will be useful, 
%% but WITHOUT ANY WARRANTY; without even the implied warranty of 
%% MERCHANTABILITY or FITNESS FOR A PARTICULAR PURPOSE. 
%% 
%% 
%% IMPORTANT NOTICE: 
%% 
%% For error reports in case of UNCHANGED versions see readme file. 
%% 
%% Please do not request updates from me directly. Distribution is 
%% done through Mail-Servers and TeX organizations. 
%% 
%% You are not allowed to change this file. 
%% 
%% You are allowed to distribute this file under the condition that 
%% it is distributed together with all files mentioned in manifest.txt. 
%% 
%% If you receive only some of these files from someone, complain! 
%% 
%% You are NOT ALLOWED to distribute this file alone. You are NOT 
%% ALLOWED to take money for the distribution or use of either this 
%% file or a changed version, except for a nominal charge for copying 
%% etc. 
%%% \iffalse ============================================================
%%%  @LaTeX-style-file{
%%%     author          = "Braams J.L.",
%%%     version         = "3.3c",
%%%     date            = "26 June 1994",
%%%     time            = "00:40:45 MET",
%%%     filename        = "english.dtx",
%%%     address         = "PTT Research
%%%                        St. Paulusstraat 4
%%%                        2264 XZ Leidschendam
%%%                        The Netherlands",
%%%     telephone       = "(70) 3325051",
%%%     FAX             = "(70) 3326477",
%%%     checksum        = "46572 339 1547 13293",
%%%     email           = "J.L.Braams@research.ptt.nl (Internet)",
%%%     codetable       = "ISO/ASCII",
%%%     keywords        = "babel, english",
%%%     supported       = "yes",
%%%     abstract        = "",
%%%     docstring       = "This file contains the english language specific
%%%                        definitions for the babel system.",
%%%  }
%%%
%%%  ====================================================================
%%% \fi
\ifx\undefined\captionsenglish
\else
  \selectlanguage{english}
  \expandafter\endinput
\fi
\chardef\atcatcode=\catcode`\@
\catcode`\@=11\relax
\ifx\undefined\babel@core@loaded\input babel.def\relax\fi
\ProvidesFile{english.sty}[1994/06/26 v3.3c
         English support from the babel system]
\ifx\undefined\originalTeX \let\originalTeX\empty\fi
\originalTeX
\ifx\undefined\l@english
  \@nopatterns{English}
  \adddialect\l@english0\fi
\adddialect\l@american\l@english
\addto\captionsenglish{%
  \def\prefacename{Preface}%
  \def\refname{References}%
  \def\abstractname{Abstract}%
  \def\bibname{Bibliography}%
  \def\chaptername{Chapter}%
  \def\appendixname{Appendix}%
  \def\contentsname{Contents}%
  \def\listfigurename{List of Figures}%
  \def\listtablename{List of Tables}%
  \def\indexname{Index}%
  \def\figurename{Figure}%
  \def\tablename{Table}%
  \def\partname{Part}%
  \def\enclname{encl}%
  \def\ccname{cc}%
  \def\headtoname{To}%
  \def\pagename{Page}%
  \def\seename{see}%
  \def\alsoname{see also}}
\let\captionsamerican\captionsenglish
\def\dateenglish{%
\def\today{\ifcase\day\or
  1st\or 2nd\or 3rd\or 4th\or 5th\or
  6th\or 7th\or 8th\or 9th\or 10th\or
  11th\or 12th\or 13th\or 14th\or 15th\or
  16th\or 17th\or 18th\or 19th\or 20th\or
  21st\or 22nd\or 23rd\or 24th\or 25th\or
  26th\or 27th\or 28th\or 29th\or 30th\or
  31st\fi~\ifcase\month\or
  January\or February\or March\or April\or May\or June\or
  July\or August\or September\or October\or November\or December\fi
  \space \number\year}}
\def\dateamerican{%
\def\today{\ifcase\month\or
  January\or February\or March\or April\or May\or June\or
  July\or August\or September\or October\or November\or December\fi
  \space\number\day, \number\year}}
\addto\extrasenglish{}
\addto\noextrasenglish{}
\let\extrasamerican\extrasenglish
\let\noextrasamerican\noextrasenglish
\selectlanguage{english}
\catcode`\@=\atcatcode \let\atcatcode\relax
%%
%% \CharacterTable
%%  {Upper-case    \A\B\C\D\E\F\G\H\I\J\K\L\M\N\O\P\Q\R\S\T\U\V\W\X\Y\Z
%%   Lower-case    \a\b\c\d\e\f\g\h\i\j\k\l\m\n\o\p\q\r\s\t\u\v\w\x\y\z
%%   Digits        \0\1\2\3\4\5\6\7\8\9
%%   Exclamation   \!     Double quote  \"     Hash (number) \#
%%   Dollar        \$     Percent       \%     Ampersand     \&
%%   Acute accent  \'     Left paren    \(     Right paren   \)
%%   Asterisk      \*     Plus          \+     Comma         \,
%%   Minus         \-     Point         \.     Solidus       \/
%%   Colon         \:     Semicolon     \;     Less than     \<
%%   Equals        \=     Greater than  \>     Question mark \?
%%   Commercial at \@     Left bracket  \[     Backslash     \\
%%   Right bracket \]     Circumflex    \^     Underscore    \_
%%   Grave accent  \`     Left brace    \{     Vertical bar  \|
%%   Right brace   \}     Tilde         \~}
%%
\endinput
%% 
%% End of file `english.sty'.
%
  \selectlanguage{american}}
%    \end{macrocode}
%    Austrian is really a dialect o German.
%    \begin{macrocode}
\DeclareOption{austrian}{%% 
%% This is file `germanb.sty', generated 
%% on <1995/2/23> with the docstrip utility (2.2i).
%% 
%% The original source files were:
%% 
%% germanb.dtx 
%% 
%% Copyright (C) 1989-1994 by Johannes Braams 
%% All rights reserved. 
%% For additional copyright information see further down in this file. 
%% 
%% This file is part of the Babel system, release 3.4 patchlevel 2. 
%% ---------------------------------------------------------------- 
%% 
%% This file is distributed in the hope that it will be useful, 
%% but WITHOUT ANY WARRANTY; without even the implied warranty of 
%% MERCHANTABILITY or FITNESS FOR A PARTICULAR PURPOSE. 
%% 
%% 
%% IMPORTANT NOTICE: 
%% 
%% For error reports in case of UNCHANGED versions see readme file. 
%% 
%% Please do not request updates from me directly. Distribution is 
%% done through Mail-Servers and TeX organizations. 
%% 
%% You are not allowed to change this file. 
%% 
%% You are allowed to distribute this file under the condition that 
%% it is distributed together with all files mentioned in manifest.txt. 
%% 
%% If you receive only some of these files from someone, complain! 
%% 
%% You are NOT ALLOWED to distribute this file alone. You are NOT 
%% ALLOWED to take money for the distribution or use of either this 
%% file or a changed version, except for a nominal charge for copying 
%% etc. 
%%% \iffalse ============================================================
%%%  @LaTeX-style-file{
%%%     author-1        = "Bernd Raichle",
%%%     author-2        = "Braams J.L.",
%%%     version         = "2.5c",
%%%     date            = "26 June 1994",
%%%     time            = "00:38:56 MET",
%%%     filename        = "germanb.dtx",
%%%     address         = "PTT Research
%%%                        St. Paulusstraat 4
%%%                        2264 XZ Leidschendam
%%%                        The Netherlands",
%%%     telephone-2     = "(70) 3325051",
%%%     FAX-2           = "(70) 3326477",
%%%     checksum        = "29603 839 3842 33542",
%%%     email-1         = "raichle@azu.Informatik.Uni-Stuttgart.de",
%%%     email-2         = "J.L.Braams@research.ptt.nl (Internet)",
%%%     codetable       = "ISO/ASCII",
%%%     keywords        = "babel, german",
%%%     supported       = "yes",
%%%     abstract        = "",
%%%     docstring       = "This file contains the german language specific
%%%                        definitions for the babel system.
%%%                        It is based on german.tex 2.3e.",
%%%  }
%%%
%%%  ====================================================================
%%% \fi
\ifx\undefined\captionsgerman
\else
  \selectlanguage{german}
  \expandafter\endinput
\fi
\chardef\atcatcode=\catcode`\@
\catcode`\@=11\relax
\ifx\undefined\babel@core@loaded\input babel.def\relax\fi
\ProvidesFile{germanb.sty}[1994/06/26 v2.5c
         German support from the babel system]
\ifx\undefined\originalTeX \let\originalTeX\empty\fi
\originalTeX
\ifx\undefined\l@german
  \@nopatterns{German}
  \adddialect\l@german0
\fi
\adddialect\l@austrian\l@german
\addto\captionsgerman{%
  \def\prefacename{Vorwort}%
  \def\refname{Literatur}%
  \def\abstractname{Zusammenfassung}%
  \def\bibname{Literaturverzeichnis}%
  \def\chaptername{Kapitel}%
  \def\appendixname{Anhang}%
  \def\contentsname{Inhaltsverzeichnis}%    % oder nur: Inhalt
  \def\listfigurename{Abbildungsverzeichnis}%
  \def\listtablename{Tabellenverzeichnis}%
  \def\indexname{Index}%
  \def\figurename{Abbildung}%
  \def\tablename{Tabelle}%                  % oder: Tafel
  \def\partname{Teil}%
  \def\enclname{Anlage(n)}%                 % oder: Beilage(n)
  \def\ccname{Verteiler}%                   % oder: Kopien an
  \def\headtoname{An}%
  \def\pagename{Seite}%
  \def\seename{siehe}%
  \def\alsoname{siehe auch}}
\let\captionsaustrian\captionsgerman
\def\month@german{\ifcase\month\or
  Januar\or Februar\or M\"arz\or April\or Mai\or Juni\or
  Juli\or August\or September\or Oktober\or November\or Dezember\fi}
\def\dategerman{\def\today{\number\day.~\month@german
  \space\number\year}}
\def\dateaustrian{\def\today{\number\day.~\ifnum1=\month
  J\"anner\else \month@german\fi \space\number\year}}
\addto\extrasgerman{\babel@add@special\"}
\addto\noextrasgerman{\babel@remove@special\"}
\addto\extrasgerman{\babel@savevariable{\catcode`\"}\babel@save\active@dq
  \catcode`\"\active \let\active@dq\german@active@dq}
\begingroup \catcode`\"=\active
\def\x{\endgroup
  \addto\extrasgerman{\babel@save"\let"\german@@active@dq}}
\x
\addto\extrasgerman{\babel@savevariable{\lccode`\^^Y}%
  \lccode`\^^Y`\^^Y}
\addto\extrasgerman{\babel@save\3\let\3\ss}
\addto\extrasgerman{\babel@save\"\umlautlow}
\addto\extrasgerman{\babel@savevariable\lefthyphenmin
  \babel@savevariable\righthyphenmin
  \lefthyphenmin\tw@\righthyphenmin\tw@}
\let\extrasaustrian\extrasgerman
\let\noextrasaustrian\noextrasgerman
\def\umlauthigh{\let\"\dieresis}
\def\umlautlow{\def\"{\protect\newumlaut}}
\def\highumlaut#1{\leavevmode\allowhyphens
  \dieresis{#1}\allowhyphens}
\expandafter\ifx\csname U@D\endcsname\relax
  \csname newdimen\endcsname\U@D
\fi
\def\newumlaut#1{\leavevmode\allowhyphens
  {\U@D 1ex%
  {\setbox\z@\hbox{\char127}\dimen@ -.45ex\advance\dimen@\ht\z@
  \ifdim 1ex<\dimen@ \fontdimen5\font\dimen@ \fi}%
  \accent127\fontdimen5\font\U@D #1}\allowhyphens}
\def\glqq{\protect\@glqq}
\def\@glqq{\save@sf@q{\set@low@box{''}\box\z@\kern-.04em\allowhyphens}}
\def\grqq{\protect\@grqq}
\def\@grqq{\save@sf@q{\kern-.07em``\kern.07em}}
\def\glq{\protect\@glq}
\def\@glq{\save@sf@q{\set@low@box{'}\box\z@\kern-.04em\allowhyphens}}
\def\grq{\protect\@grq}
\def\@grq{\save@sf@q{\kern-.07em`\kern.07em}}
\def\flqq{\protect\@flqq}
\def\@flqq{\relax \ifmmode \ll \else
  \save@sf@q{\raise .2ex\hbox{$\scriptscriptstyle \ll $}}\fi}
\def\frqq{\protect\@frqq}
\def\@frqq{\relax \ifmmode \gg \else
  \save@sf@q{\raise .2ex\hbox{$\scriptscriptstyle \gg $}}\fi}
\def\flq{\protect\@flq}
\def\@flq{\relax \ifmmode <\else
  \save@sf@q{\raise .2ex\hbox{$\scriptscriptstyle <$}}\fi}
\def\frq{\protect\@frq}
\def\@frq{\relax \ifmmode >\else
  \save@sf@q{\raise .2ex\hbox{$\scriptscriptstyle >$}}\fi}
\begingroup \catcode`\"12
\def\x{\endgroup
  \def\@UMLAUT{\"}
  \def\@MATHUMLAUT{\mathaccent"707F }
  \def\@SS{\mathchar"7019 }
  \def\dq{"}}
\x
\begingroup \catcode`\"=\active
\def\x{\endgroup
  \ifx"\undefined \let"\dq \fi
  \let\active@dq\dq}
\x
\let\dieresis=\"
\def\german@dqmacro#1{\csname g@dq@\string #1@dq@\endcsname}
\def\german@@active@dq#1{\expandafter\expandafter\expandafter
  \ifx\german@dqmacro{#1}\relax \expandafter\normal@dq
  \else \expandafter\german@@@active@dq \fi {#1}}
\def\normal@dq#1{\dq #1}
\def\german@@@active@dq#1{%
  \ifx\protect\relax \else \expandafter\protect \fi
  \active@dq{#1\empty}}
\def\german@active@dq#1{%
  \csname german@dq@\ifmmode second\else first\fi
    \expandafter\expandafter\expandafter\expandafter
    \expandafter\expandafter\expandafter
  \endcsname
  \german@dqmacro{#1}}
\def\german@dq@first#1#2{#1}
\def\german@dq@second#1#2{#2}
\def\def@german@dqmacro#1#2#3{\expandafter\expandafter\expandafter
  \def\german@dqmacro{#1}{{#2}{#3}}}
\def\let@german@dqmacro#1#2{\begingroup
  \edef\x{\endgroup \let
    \expandafter\expandafter\expandafter\noexpand\german@dqmacro{#1}%
    \expandafter\expandafter\expandafter\noexpand\german@dqmacro{#2}}%
  \x}
\def\german@dq@disc#1#2{%
  \allowhyphens\discretionary{#2-}{}{#1}\allowhyphens}
\def@german@dqmacro{a}{\@UMLAUT a}{\@MATHUMLAUT a}
\def@german@dqmacro{o}{\@UMLAUT o}{\@MATHUMLAUT o}
\def@german@dqmacro{u}{\@UMLAUT u}{\@MATHUMLAUT u}
\def@german@dqmacro{A}{\@UMLAUT A}{\@MATHUMLAUT A}
\def@german@dqmacro{O}{\@UMLAUT O}{\@MATHUMLAUT O}
\def@german@dqmacro{U}{\@UMLAUT U}{\@MATHUMLAUT U}
\def@german@dqmacro{e}{\highumlaut e}{\@MATHUMLAUT a}
\def@german@dqmacro{E}{\highumlaut E}{\@MATHUMLAUT E}
\def@german@dqmacro{i}{\highumlaut\i{}}{\@MATHUMLAUT\imath}
\def@german@dqmacro{I}{\highumlaut I}{\@MATHUMLAUT I}
\def@german@dqmacro{s}{\ss{}}{\@SS{}}
\def@german@dqmacro{S}{SS}{SS}
\def@german@dqmacro{`}{\glqq{}}{\glqq{}}
\def@german@dqmacro{'}{\grqq{}}{\grqq{}}
\def@german@dqmacro{<}{\flqq{}}{\flqq{}}
\def@german@dqmacro{>}{\frqq{}}{\frqq{}}
\def@german@dqmacro{c}{\german@dq@disc ck}{c}
\def@german@dqmacro{C}{\german@dq@disc CK}{C}
\def@german@dqmacro{f}{\german@dq@disc f{ff}}{f}
\def@german@dqmacro{F}{\german@dq@disc F{FF}}{F}
\def@german@dqmacro{l}{\german@dq@disc l{ll}}{l}
\def@german@dqmacro{L}{\german@dq@disc L{LL}}{L}
\def@german@dqmacro{m}{\german@dq@disc m{mm}}{m}
\def@german@dqmacro{M}{\german@dq@disc M{MM}}{M}
\def@german@dqmacro{n}{\german@dq@disc n{nn}}{n}
\def@german@dqmacro{N}{\german@dq@disc N{NN}}{N}
\def@german@dqmacro{p}{\german@dq@disc p{pp}}{p}
\def@german@dqmacro{P}{\german@dq@disc P{PP}}{P}
\def@german@dqmacro{t}{\german@dq@disc t{tt}}{t}
\def@german@dqmacro{T}{\german@dq@disc T{TT}}{T}
\def@german@dqmacro{-}{\allowhyphens\-\allowhyphens}%
               {\allowhyphens\-\allowhyphens}
\def@german@dqmacro{|}{\discretionary{-}{}{\kern.03em}}{}
\def@german@dqmacro{"}{\hskip\z@skip}{\hskip\z@skip}
\def@german@dqmacro{\empty}{{}}{{}}
\let@german@dqmacro{\active@dq{}}{"}
\def\mdqon{\catcode`\"\active}
\def\mdqoff{\catcode`\"12 }
\def\ck{\allowhyphens\discretionary{k-}{k}{ck}\allowhyphens}
\selectlanguage{german}
\catcode`\@=\atcatcode
%%
%% \CharacterTable
%%  {Upper-case    \A\B\C\D\E\F\G\H\I\J\K\L\M\N\O\P\Q\R\S\T\U\V\W\X\Y\Z
%%   Lower-case    \a\b\c\d\e\f\g\h\i\j\k\l\m\n\o\p\q\r\s\t\u\v\w\x\y\z
%%   Digits        \0\1\2\3\4\5\6\7\8\9
%%   Exclamation   \!     Double quote  \"     Hash (number) \#
%%   Dollar        \$     Percent       \%     Ampersand     \&
%%   Acute accent  \'     Left paren    \(     Right paren   \)
%%   Asterisk      \*     Plus          \+     Comma         \,
%%   Minus         \-     Point         \.     Solidus       \/
%%   Colon         \:     Semicolon     \;     Less than     \<
%%   Equals        \=     Greater than  \>     Question mark \?
%%   Commercial at \@     Left bracket  \[     Backslash     \\
%%   Right bracket \]     Circumflex    \^     Underscore    \_
%%   Grave accent  \`     Left brace    \{     Vertical bar  \|
%%   Right brace   \}     Tilde         \~}
%%
\endinput
%% 
%% End of file `germanb.sty'.
%
                         \selectlanguage{austrian}}
\DeclareOption{bahasa}{%% 
%% This is file `bahasa.sty', generated 
%% on <1995/2/23> with the docstrip utility (2.2i).
%% 
%% The original source files were:
%% 
%% bahasa.dtx 
%% 
%% Copyright (C) 1989-1994 by Johannes Braams 
%% All rights reserved. 
%% For additional copyright information see further down in this file. 
%% 
%% This file is part of the Babel system, release 3.4 patchlevel 2. 
%% ---------------------------------------------------------------- 
%% 
%% This file is distributed in the hope that it will be useful, 
%% but WITHOUT ANY WARRANTY; without even the implied warranty of 
%% MERCHANTABILITY or FITNESS FOR A PARTICULAR PURPOSE. 
%% 
%% 
%% IMPORTANT NOTICE: 
%% 
%% For error reports in case of UNCHANGED versions see readme file. 
%% 
%% Please do not request updates from me directly. Distribution is 
%% done through Mail-Servers and TeX organizations. 
%% 
%% You are not allowed to change this file. 
%% 
%% You are allowed to distribute this file under the condition that 
%% it is distributed together with all files mentioned in manifest.txt. 
%% 
%% If you receive only some of these files from someone, complain! 
%% 
%% You are NOT ALLOWED to distribute this file alone. You are NOT 
%% ALLOWED to take money for the distribution or use of either this 
%% file or a changed version, except for a nominal charge for copying 
%% etc. 
%%% \iffalse ============================================================
%%% @LaTeX-style-file{
%%%    author_1            = "Terry Mart",
%%%    author_2            = "J"org Knappen",
%%%    version             = "0.9d",
%%%    date                = "26 June 1994",
%%%    time                = "02:15:34 MET",
%%%    filename            = "bahasa.dtx",
%%%    address             = "Institut f\"ur Kernphysik
%%%                           Johannes Gutenberg-Universit\"at Mainz
%%%                           D-55099 Mainz
%%%                           Germany",
%%%    telephone           = "",
%%%    FAX                 = "",
%%%    checksum            = "21315 246 1122 9276",
%%%    email_1             = "mart@vkpmzd.kph.uni-mainz.de (Internet)",
%%%    email_2             = "knappen@vkpmzd.kph.uni-mainz.de (Internet)",
%%%    codetable           = "ISO/ASCII",
%%%    keywords            = "",
%%%    supported           = "yes",
%%%    abstract            = "",
%%%    docstring           = "The checksum field above contains a CRC-16
%%%                           checksum as the first value, followed by the
%%%                           equivalent of the standard UNIX wc (word
%%%                           count) utility output of lines, words, and
%%%                           characters.  This is produced by Robert
%%%                           Solovay's checksum utility.",
%%% }
%%%
%%% ====================================================================
%%% \fi
\ifx\undefined\captionsbahasa
\else
  \selectlanguage{bahasa}
  \expandafter\endinput
\fi
\chardef\atcatcode=\catcode`\@
\catcode`\@=11\relax
\ifx\undefined\babel@core@loaded\input babel.def\relax\fi
\ProvidesFile{bahasa.sty}[1994/06/26 v0.9c
         Bahasa support from the babel system]
\ifx\undefined\originalTeX \let\originalTeX\empty\else\originalTeX\fi
\ifx\undefined\l@bahasa
  \@nopatterns{Bahasa}
  \adddialect\l@bahasa0\fi
\addto\captionsbahasa{%
  \def\prefacename{Pendahuluan}%
  \def\refname{Pustaka}%
  \def\abstractname{Ringkasan}% (sometime it's called 'intisari'
                              %  or 'ikhtisar')
  \def\bibname{Bibliografi}%
  \def\chaptername{Bab}%
  \def\appendixname{Lampiran}%
  \def\contentsname{Daftar Isi}%
  \def\listfigurename{Daftar Gambar}%
  \def\listtablename{Daftar Tabel}%
  \def\indexname{Indeks}%
  \def\figurename{Gambar}%
  \def\tablename{Tabel}%
  \def\partname{Bagian}%
  \def\enclname{Lampiran}%
  \def\ccname{cc}%
  \def\headtoname{Kepada}%
  \def\pagename{Halaman}%
  \def\seename{lihat}%
  \def\alsoname{lihat juga}}
\def\datebahasa{%
\def\today{\number\day~\ifcase\month\or
  Januari\or Februari\or Maret\or April\or Mei\or Juni\or
  Juli\or Agustus\or September\or Oktober\or Nopember\or Desember\fi
  \space \number\year}}
\addto\extrasbahasa{}
\addto\noextrasbahasa{}
\selectlanguage{bahasa}
\catcode`\@=\atcatcode \let\atcatcode\relax
%%
%% \CharacterTable
%%  {Upper-case    \A\B\C\D\E\F\G\H\I\J\K\L\M\N\O\P\Q\R\S\T\U\V\W\X\Y\Z
%%   Lower-case    \a\b\c\d\e\f\g\h\i\j\k\l\m\n\o\p\q\r\s\t\u\v\w\x\y\z
%%   Digits        \0\1\2\3\4\5\6\7\8\9
%%   Exclamation   \!     Double quote  \"     Hash (number) \#
%%   Dollar        \$     Percent       \%     Ampersand     \&
%%   Acute accent  \'     Left paren    \(     Right paren   \)
%%   Asterisk      \*     Plus          \+     Comma         \,
%%   Minus         \-     Point         \.     Solidus       \/
%%   Colon         \:     Semicolon     \;     Less than     \<
%%   Equals        \=     Greater than  \>     Question mark \?
%%   Commercial at \@     Left bracket  \[     Backslash     \\
%%   Right bracket \]     Circumflex    \^     Underscore    \_
%%   Grave accent  \`     Left brace    \{     Vertical bar  \|
%%   Right brace   \}     Tilde         \~}
%%
\endinput
%% 
%% End of file `bahasa.sty'.
}
\DeclareOption{brazil}{%% 
%% This is file `portuges.sty', generated 
%% on <1995/2/23> with the docstrip utility (2.2i).
%% 
%% The original source files were:
%% 
%% portuges.dtx 
%% 
%% Copyright (C) 1989-1994 by Johannes Braams 
%% All rights reserved. 
%% For additional copyright information see further down in this file. 
%% 
%% This file is part of the Babel system, release 3.4 patchlevel 2. 
%% ---------------------------------------------------------------- 
%% 
%% This file is distributed in the hope that it will be useful, 
%% but WITHOUT ANY WARRANTY; without even the implied warranty of 
%% MERCHANTABILITY or FITNESS FOR A PARTICULAR PURPOSE. 
%% 
%% 
%% IMPORTANT NOTICE: 
%% 
%% For error reports in case of UNCHANGED versions see readme file. 
%% 
%% Please do not request updates from me directly. Distribution is 
%% done through Mail-Servers and TeX organizations. 
%% 
%% You are not allowed to change this file. 
%% 
%% You are allowed to distribute this file under the condition that 
%% it is distributed together with all files mentioned in manifest.txt. 
%% 
%% If you receive only some of these files from someone, complain! 
%% 
%% You are NOT ALLOWED to distribute this file alone. You are NOT 
%% ALLOWED to take money for the distribution or use of either this 
%% file or a changed version, except for a nominal charge for copying 
%% etc. 
%%% \iffalse ============================================================
%%%  @LaTeX-style-file{
%%%     author          = "Braams J.L.",
%%%     version         = "1.2d",
%%%     date            = "26 June 1994",
%%%     time            = "00:54:18 MET",
%%%     filename        = "portuges.doc",
%%%     address         = "PTT Research
%%%                        St. Paulusstraat 4
%%%                        2264 XZ Leidschendam
%%%                        The Netherlands",
%%%     telephone       = "(70) 3325051",
%%%     FAX             = "(70) 3326477",
%%%     checksum        = "22850 303 1380 11594",
%%%     email           = "J.L.Braams@research.ptt.nl (Internet)",
%%%     codetable       = "ISO/ASCII",
%%%     keywords        = "babel, portuguese",
%%%     supported       = "yes",
%%%     abstract        = "",
%%%     docstring       = "This file contains the portuguese language
%%%                        specific definitions for the babel system.",
%%%  }
%%%
%%%  ====================================================================
%%% \fi

\ifx\undefined\captionsportuges
\else
  \selectlanguage{portuges}
  \expandafter\endinput
\fi
\chardef\atcatcode=\catcode`\@
\catcode`\@=11\relax
\ifx\undefined\babel@core@loaded\input babel.def\relax\fi
\ProvidesFile{portuges.sty}[1994/06/26 v1.2d
         Portuguese support from the babel system]
\ifx\undefined\originalTeX \let\originalTeX\empty \else\originalTeX\fi
\ifx\undefined\l@portuges
    \@nopatterns{Portuges}
    \adddialect\l@portuges0\fi
\adddialect\l@brazil\l@portuges
\addto\captionsportuges{%
  \def\prefacename{Preface}%  this needs translation!
  \def\refname{Refer\^encias}%
  \def\abstractname{Resumo}%
  \def\bibname{Bibliografia}%
  \def\chaptername{Cap\'{\i}tulo}%
  \def\appendixname{Ap\^endice}%
  \def\contentsname{\'Indice}%
  \def\listfigurename{Lista de Figuras}%
  \def\listtablename{Lista de Tabelas}%
  \def\indexname{\'Indice Remissivo}%
  \def\figurename{Figura}%
  \def\tablename{Tabela}%
  \def\partname{Parte}%
  \def\enclname{Anexos}%
  \def\ccname{C\'opia a}%
  \def\headtoname{Para}%
  \def\pagename{P\'agina}%
  \def\seename{See}%  this needs translation!
  \def\alsoname{See also}}%  this needs translation!
\let\captionsbrazil\captionsportuges
\def\dateportuges{%
\def\today{\number\day\space de\space\ifcase\month\or
  Janeiro\or Fevereiro\or Mar\c{c}o\or Abril\or Maio\or Junho\or
  Julho\or Agosto\or Setembro\or Outubro\or Novembro\or Dezembro\fi
  \space de\space\number\year}}
\def\datebrazil{%
\def\today{\number\day\space de\space\ifcase\month\or
  janeiro\or fevereiro\or mar\c{c}o\or abril\or maio\or junho\or
  julho\or agosto\or setembro\or outubro\or novembro\or dezembro\fi
  \space de\space\number\year}}
\addto\extrasportuges{}
\addto\noextrasportuges{}
\let\extrasbrazil\extrasportuges
\let\noextrasbrazil\noextrasportuges
\selectlanguage{portuges}
\catcode`\@=\atcatcode \let\atcatcode\relax
%%
%% \CharacterTable
%%  {Upper-case    \A\B\C\D\E\F\G\H\I\J\K\L\M\N\O\P\Q\R\S\T\U\V\W\X\Y\Z
%%   Lower-case    \a\b\c\d\e\f\g\h\i\j\k\l\m\n\o\p\q\r\s\t\u\v\w\x\y\z
%%   Digits        \0\1\2\3\4\5\6\7\8\9
%%   Exclamation   \!     Double quote  \"     Hash (number) \#
%%   Dollar        \$     Percent       \%     Ampersand     \&
%%   Acute accent  \'     Left paren    \(     Right paren   \)
%%   Asterisk      \*     Plus          \+     Comma         \,
%%   Minus         \-     Point         \.     Solidus       \/
%%   Colon         \:     Semicolon     \;     Less than     \<
%%   Equals        \=     Greater than  \>     Question mark \?
%%   Commercial at \@     Left bracket  \[     Backslash     \\
%%   Right bracket \]     Circumflex    \^     Underscore    \_
%%   Grave accent  \`     Left brace    \{     Vertical bar  \|
%%   Right brace   \}     Tilde         \~}
%%
\endinput
%% 
%% End of file `portuges.sty'.
\selectlanguage{brazil}}
\DeclareOption{catalan}{%% 
%% This is file `catalan.sty', generated 
%% on <1995/2/23> with the docstrip utility (2.2i).
%% 
%% The original source files were:
%% 
%% catalan.dtx 
%% 
%% Copyright (C) 1989-1994 by Johannes Braams 
%% All rights reserved. 
%% For additional copyright information see further down in this file. 
%% 
%% This file is part of the Babel system, release 3.4 patchlevel 2. 
%% ---------------------------------------------------------------- 
%% 
%% This file is distributed in the hope that it will be useful, 
%% but WITHOUT ANY WARRANTY; without even the implied warranty of 
%% MERCHANTABILITY or FITNESS FOR A PARTICULAR PURPOSE. 
%% 
%% 
%% IMPORTANT NOTICE: 
%% 
%% For error reports in case of UNCHANGED versions see readme file. 
%% 
%% Please do not request updates from me directly. Distribution is 
%% done through Mail-Servers and TeX organizations. 
%% 
%% You are not allowed to change this file. 
%% 
%% You are allowed to distribute this file under the condition that 
%% it is distributed together with all files mentioned in manifest.txt. 
%% 
%% If you receive only some of these files from someone, complain! 
%% 
%% You are NOT ALLOWED to distribute this file alone. You are NOT 
%% ALLOWED to take money for the distribution or use of either this 
%% file or a changed version, except for a nominal charge for copying 
%% etc. 
%%% \iffalse ================================================================
%%%  @LaTeX-style-file{
%%%     author_1        = "Goncal Badenes",
%%%     author_2        = "Johannes L. Braams",
%%%     version         = "2.1c",
%%%     date            = "26 June 1994",
%%%     time            = "01:06:07 MET",
%%%     address         = "PTT Research
%%%                        St. Paulusstraat 4
%%%                        2264 XZ Leidschendam
%%%                        The Netherlands",
%%%     telephone       = "(70) 3325051",
%%%     FAX             = "(70) 3326477",
%%%     checksum        = "11572 696 3394 27145",
%%%     email_1         = "GONCAL@cnmvax.uab.es",
%%%     email_2         = "J.L.Braams@research.ptt.nl (Internet)",
%%%     codetable       = "ISO/ASCII",
%%%     keywords        = "babel, catalan",
%%%     supported       = "yes",
%%%     abstract        = "",
%%%     docstring       = "This file contains the catalan language specific
%%%                        definitions for the babel system.
%%%
%%%                        The checksum field above contains a CRC-16
%%%                        checksum as the first value, followed by the
%%%                        equivalent of the standard UNIX wc (word
%%%                        count) utility output of lines, words, and
%%%                        characters.  This is produced by Robert
%%%                        Solovay's checksum utility.",
%%%  }
%%%
%%%  ====================================================================
%%% \fi
\ifx\undefined\captionscatalan
\else
  \selectlanguage{catalan}
  \expandafter\endinput
\fi
\chardef\atcatcode=\catcode`\@
\catcode`\@=11\relax
\ifx\undefined\babel@core@loaded\input babel.def\fi
\ProvidesFile{catalan.sty}[1994/06/26 v2.1d
         Catalan support from the babel system]
\ifx\undefined\originalTeX
  \let\originalTeX\empty
\fi
\originalTeX
\ifx\undefined\l@catalan
  \@nopatterns{Catalan}
  \adddialect\l@catalan0
\fi
\addto\captionscatalan{%
  \def\prefacename{Preface}%   <- This needs translation!!
  \def\refname{Refer\`encies}%
  \def\abstractname{Resum}%
  \def\bibname{Bibliografia}%
  \def\chaptername{Cap\'{\i}tol}%
  \def\appendixname{Ap\`endix}%
  \def\contentsname{\'Index}%
  \def\listfigurename{\'Index de figures}%
  \def\listtablename{\'Index de taules}%
  \def\indexname{\'Index de mat\`eries}%
  \def\figurename{Figura}%
  \def\tablename{Taula}%
  \def\partname{Part}%
  \def\enclname{adjunt}%
  \def\ccname{c\`opies a}%
  \def\headtoname{}%
  \def\pagename{p\`agina}%
  \def\seename{See}%           <- This needs translation!!
  \def\alsoname{See also}%     <- This needs translation!!
}
\def\datecatalan{%
  \def\today{\number\day~\ifcase\month\or
    de Gener\or de Febrer\or de Mar\c{c}\or d'Abril\or de Maig\or
    de Juny\or de Juliol\or d'Agost\or de Setembre\or d'Octubre\or
    de Novembre\or de Desembre\fi
    \space de~\number\year}}
\addto\extrascatalan{%
  \babel@add@special\"%
  \babel@add@special\'%
  \babel@add@special\~}
\addto\noextrascatalan{%
  \babel@remove@special\"%
  \babel@remove@special\'%
  \babel@remove@special\~}
\addto\extrascatalan{%
  \babel@savevariable{\catcode`\"}\babel@save\active@dq
  \babel@savevariable{\catcode`\'}\babel@save\active@ac
  \babel@savevariable{\catcode`\~}\babel@save\active@til
  \babel@save~}
\addto\extrascatalan{\catcode`\"\active\catcode`\'\active%
                     \catcode`\~\active}
\begingroup \catcode`\"\active \catcode`\~\active \catcode`\'\active
\def\x{\endgroup
  \addto\extrascatalan{%
    \def'{\protect\active@ac}\let\active@ac\catalan@active@ac
    \def~{\protect\active@til}\let\active@til\catalan@active@til
    \def"{\protect\active@dq}\let\active@dq\catalan@active@dq}}
\x
\addto\extrascatalan{\babel@save\"\babel@save\'%
                     \babel@save\~\babel@save\a}
\addto\extrascatalan{\let\a\catalan@a}
\addto\extrascatalan{\def\"{\protect\@umlaut}%
                     \def\'{\protect\@acute}%
                     \def\~{\protect\@tilde}}
\def\catalan@a#1{{\if\string#1'\aftergroup\@acute
            \else \expandafter\aftergroup\csname a#1\endcsname\fi}}
\begingroup \catcode`\"12
\edef\x{\endgroup
  \def\noexpand\dq{"}
  \def\noexpand\til{~}}
\x
\let\dieresis\"
\let\textacute\'
\let\texttilde\~
\def\@umlaut#1{\expandafter\ifx\csname @um@\ifx #1\i \dotlessi%
                                           \else #1\fi\endcsname\relax%
        \allowhyphens\dieresis{#1}\allowhyphens
        \else\csname @um@#1\endcsname\fi}
\def\@acute#1{\expandafter\ifx\csname @ac@\ifx #1\i \dotlessi%
                                          \else #1\fi\endcsname\relax%
        \allowhyphens\textacute{#1}\allowhyphens%
        \else\csname @ac@#1\endcsname\fi}
\def\@tilde#1{\expandafter\ifx\csname @til@\ifx #1\i \dotlessi%
                                           \else #1\fi\endcsname\relax%
        \allowhyphens\texttilde{#1}\allowhyphens%
        \else\csname @til@#1\endcsname\fi}
\def\catalan@active@dq{\futurelet\next\first@catalan@active@dq}
\def\first@catalan@active@dq{%
  \begingroup
  \ifx \next\bgroup \aftergroup\dq
    \else\ifx \next\egroup \aftergroup\dq
    \else\expandafter\ifx \space\next \aftergroup\dq
    \else\aftergroup\second@catalan@active@dq
    \fi\fi\fi
  \endgroup}
\def\second@catalan@active@dq#1{%
  \ifx #1-\allowhyphens-\allowhyphens
  \else\ifx #1|\discretionary{-}{}{\kern.03em}%
  \else\ifx #1u\@umlaut{u}%
  \else\ifx #1U\@umlaut{U}%
  \else\ifx #1o{\raise1ex\hbox{\underbar{\scriptsize o}}}%
  \else\ifx #1a{\raise1ex\hbox{\underbar{\scriptsize a}}}%
  \else \dq #1\fi\fi\fi\fi\fi\fi}
\def\catalan@active@ac{{%
    \ifmmode\aftergroup\catalan@math@ac
    \else\aftergroup\first@catalan@active@ac\fi}}
\def\catalan@math@ac{^\bgroup\catcode`\'=12\prim@s}
\def\first@catalan@active@ac{\futurelet\next\second@catalan@active@ac}
\def\second@catalan@active@ac{%
  \begingroup
    \ifx \next\bgroup \aftergroup\rq
    \else\ifx \next\egroup \aftergroup\rq
    \else\expandafter\ifx \space\next \aftergroup\rq
    \else\aftergroup\third@catalan@active@ac
    \fi\fi\fi \endgroup}
\def\third@catalan@active@ac#1{%
     \ifx #1a\@acute{a}%
\else\ifx #1e\@acute{e}%
\else\ifx #1i\@acute{\dotlessi}%
\else\ifx #1o\@acute{o}%
\else\ifx #1u\@acute{u}%
\else\ifx #1A\@acute{A}%
\else\ifx #1E\@acute{E}%
\else\ifx #1I\@acute{I}%
\else\ifx #1O\@acute{O}%
\else\ifx #1U\@acute{U}%
\else\ifx #1n\@tilde{n}%  FTC compatibility
\else\ifx #1N\@tilde{N}%  FTC compatibility
\else\if\string#1'\rq\rq% Funny, this one gets the ligature right
\else\rq #1\fi\fi\fi\fi\fi\fi\fi\fi\fi\fi\fi\fi\fi}
\def\catalan@active@til{\futurelet\next\first@catalan@active@til}
\def\first@catalan@active@til{%
  \begingroup
    \ifx \next\bgroup \aftergroup\til
    \else\ifx \next\egroup \aftergroup\til
    \else\expandafter\ifx \space\next \aftergroup\til
    \else\ifx n\next \@tilde{n}\aftergroup\@gobble
    \else\ifx N\next \@tilde{N}\aftergroup\@gobble
    \else\aftergroup\til
    \fi\fi\fi\fi\fi \endgroup}
\def\thechar#1{{\uccode`!=#1\relax\uppercase{\gdef\next{!}}}}
\thechar\i\global\let\dotlessi\next
\def\-{\allowhyphens\discretionary{-}{}{}\allowhyphens}
\selectlanguage{catalan}
\catcode`\@=\atcatcode \let\atcatcode\relax
%% \CharacterTable
%%  {Upper-case    \A\B\C\D\E\F\G\H\I\J\K\L\M\N\O\P\Q\R\S\T\U\V\W\X\Y\Z
%%   Lower-case    \a\b\c\d\e\f\g\h\i\j\k\l\m\n\o\p\q\r\s\t\u\v\w\x\y\z
%%   Digits        \0\1\2\3\4\5\6\7\8\9
%%   Exclamation   \!     Double quote  \"     Hash (number) \#
%%   Dollar        \$     Percent       \%     Ampersand     \&
%%   Acute accent  \'     Left paren    \(     Right paren   \)
%%   Asterisk      \*     Plus          \+     Comma         \,
%%   Minus         \-     Point         \.     Solidus       \/
%%   Colon         \:     Semicolon     \;     Less than     \<
%%   Equals        \=     Greater than  \>     Question mark \?
%%   Commercial at \@     Left bracket  \[     Backslash     \\
%%   Right bracket \]     Circumflex    \^     Underscore    \_
%%   Grave accent  \`     Left brace    \{     Vertical bar  \|
%%   Right brace   \}     Tilde         \~}
%%
\endinput
%% 
%% End of file `catalan.sty'.
}
\DeclareOption{croatian}{% \iffalse meta-comment
%
% Copyright (C) 1989-1994 by Johannes Braams
% All rights reserved.
% For additional copyright information see further down in this file.
% 
% This file is part of the Babel system, release 3.4 patchlevel 2.
% ----------------------------------------------------------------
% 
% This file is distributed in the hope that it will be useful,
% but WITHOUT ANY WARRANTY; without even the implied warranty of
% MERCHANTABILITY or FITNESS FOR A PARTICULAR PURPOSE.
% 
% 
% IMPORTANT NOTICE:
% 
% For error reports in case of UNCHANGED versions see readme file.
% 
% Please do not request updates from me directly.  Distribution is
% done through Mail-Servers and TeX organizations.
% 
% You are not allowed to change this file.
% 
% You are allowed to distribute this file under the condition that
% it is distributed together with all files mentioned in manifest.txt.
% 
% If you receive only some of these files from someone, complain!
% 
% You are NOT ALLOWED to distribute this file alone.  You are NOT
% ALLOWED to take money for the distribution or use of either this
% file or a changed version, except for a nominal charge for copying
% etc.
% \fi
% \CheckSum{127}
%%% \iffalse ============================================================
%%%  @LaTeX-style-file{
%%%     author          = "Braams J.L.",
%%%     version         = "1.3c",
%%%     date            = "04 June 1994",
%%%     time            = "23:45:48 MET",
%%%     filename        = "croatian.doc",
%%%     address         = "PTT Research
%%%                        St. Paulusstraat 4
%%%                        2264 XZ Leidschendam
%%%                        The Netherlands",
%%%     telephone       = "(70) 3325051",
%%%     FAX             = "(70) 3326477",
%%%     checksum        = "58159 256 1199 10030",
%%%     email           = "J.L.Braams@research.ptt.nl (Internet)",
%%%     codetable       = "ISO/ASCII",
%%%     keywords        = "",
%%%     supported       = "yes",
%%%     abstract        = "",
%%%     docstring       = "This file contains the croatian language specific
%%%                        definitions for the babel system.",
%%%  }
%%% ================================================================= \fi
% \def\filename{croatian.dtx}
\def\fileversion{v1.3c}
\def\filedate{1994/06/04}
%
% \iffalse
% Babel DOCUMENT-STYLE option for LaTeX version 2e
% Copyright (C) 1989 - 1994
%           by Johannes Braams, PTT Research Neher Laboratories
%
% Please report errors to: J.L. Braams
%                          J.L.Braams@research.ptt.nl
%
%    This file is part of the babel system, it provides the source code for
%    the Croatian language-specific file.
%    A contribution was made by Alan Pai\'{c}  (paica@cernvm.cern.ch)
%<*filedriver>
\documentclass{ltxdoc}
\newcommand\TeXhax{\TeX hax}
\newcommand\babel{{\sf babel}}
\newcommand\ttbs{\char'134}
\newcommand\langvar{$\langle \it lang \rangle$}
\newcommand\note[1]{}
\newcommand\bsl{\protect\bslash}
\newcommand\Lopt[1]{{\sf #1}}
\newcommand\file[1]{{\tt #1}}
\begin{document}
 \DocInput{croatian.dtx}
\end{document}
%</filedriver>
%\fi
% \changes{croatian-1.0a}{15 july 91}{Renamed babel.sty in babel.com}
% \changes{croatian-1.0c}{25 jan 92}{Removed some typos}
% \changes{croatian-1.1}{15 feb 92}{Brought up-to-date with babel 3.2a}
% \changes{croatian-1.3}{1994/02/27}{Update for LaTeX2e}
%
%  \section{The Croatian language}
%
%    The file \file{\filename}\footnote{The file described in this
%    section has version number \fileversion\ and was last revised on
%    \filedate.  A contribution was made by Alan Pai\'{c} ({\tt
%    paica@cernvm.cern.ch}).}  defines all the language-specific
%    macros for the Croatian language.
%
%    For this language currently no special definitions are needed or
%    available.
%
% \StopEventually{}
%
%    As this file needs to be read only once, we check whether it was
%    read before. If it was, the command |\captionscroatian| is already
%    defined, so we can stop processing. If this command is undefined
%    we proceed with the various definitions and first show the
%    current version of this file.
%
% \changes{croatian-1.0a}{15 july 91}{Added reset of catcode of @
%    before {\tt\bsl endinput}.}
% \changes{croatian-1.0b}{7 okt 91}{Removed use of {\tt\bsl
%    @ifundefined}}
%    \begin{macrocode}
\ifx\undefined\captionscroatian
\else
  \selectlanguage{croatian}
  \expandafter\endinput
\fi
%    \end{macrocode}
%
%    Tell the \LaTeX\ system who we are and write an entry on the
%    transcript.
%    \begin{macrocode}
%<-compat>\ProvidesFile{croatian.sty}[\filedate\space
%<-compat>         Croatian support from the babel system]
%<+compat>\wlog{Babel option `Croatian' version \fileversion\space<\filedate>}
%    \end{macrocode}
%
% \changes{croatian-1.0b}{7 okt 91}{Removed code to load {\tt
%    latexhax.com}}
%
% \begin{macro}{\atcatcode}
%    This file, \file{croatian.sty}, may have been read while \TeX\ is
%    in the middle of processing a document, so we have to make sure the
%    category code of {\tt @} is `letter' while this file is being
%    read.  We save the category code of the @-sign in |\atcatcode|
%    and make it `letter'. Later the category code can be restored to
%    whatever it was before.
%
% \changes{croatian-1.0a}{15 july 91}{Modified handling of catcode of
%    @ again.}
% \changes{croatian-1.0b}{7 okt 91}{Removed use of {\tt\bsl
%    makeatletter} and hence the need to load {\tt latexhax.com}}
%    \begin{macrocode}
\chardef\atcatcode=\catcode`\@
\catcode`\@=11\relax
%    \end{macrocode}
% \end{macro}
%
%    Now we determine whether the the common macros from the file
%    \file{babel.def} need to be read. We can be in one of two
%    situations: either another language option has been read earlier
%    on, in which case that other option has already read
%    \file{babel.def}, or {\tt croatian} is the first language option
%    to be processed. In that case we need to read \file{babel.def}
%    right here before we continue.
%
% \changes{croatian-1.1}{15 feb 92}{Added {\tt\bsl relax} after the
%    argument of {\tt\bsl input}}
%    \begin{macrocode}
\ifx\undefined\babel@core@loaded\input babel.def\relax\fi
%    \end{macrocode}
%
%    Another check that has to be made, is if another language
%    specific file has been read already. In that case its definitions
%    have been activated. This might interfere with definitions this
%    file tries to make. Therefore we make sure that we cancel any
%    special definitions. This can be done by checking the existence
%    of the macro |\originalTeX|. If it exists we simply execute it,
%    otherwise it is |\let| to |\empty|.
% \changes{croatian-1.0a}{15 july 91}{Added {\tt\bsl let\bsl
%    originalTeX\bsl relax} to test for existence}
% \changes{croatian-1.1}{15 feb 92}{{\tt\bsl originalTeX} should be
%    expandable, {\tt\bsl let} it to {\tt\bsl empty}}
%    \begin{macrocode}
\ifx\undefined\originalTeX \let\originalTeX\empty \else\originalTeX\fi
%    \end{macrocode}
%
%    When this file is read as an option, i.e. by the |\usepackage|
%    command, {\tt croatian} will be an `unknown' language in which
%    case we have to make it known. So we check for the existence of
%    |\l@croatian| to see whether we have to do something here.
%
% \changes{croatian-1.0b}{7 okt 91}{Removed use of {\tt\bsl
%    @ifundefined}}
% \changes{croatian-1.1}{15 feb 92}{Added a warning when no
%    hyphenation patterns were loaded.}
%    \begin{macrocode}
\ifx\undefined\l@croatian
    \message{!! The language `Croatian' doesn't have hyphenation patterns.}
    \message{!! I will use the patterns loaded for \string\language=0 instead.}
    \adddialect\l@croatian0\fi
%    \end{macrocode}
%
%    The next step consists of defining commands to switch to (and
%    from) the Croatian language.
%
% \begin{macro}{\captionscroatian}
%    The macro |\captionscroatian| defines all strings used
%    in the four standard documentclasses provided with \LaTeX.
% \changes{croatian-1.1}{15 feb 92}{Added {\tt\bsl seename}, {\tt\bsl
%    alsoname} and {\tt\bsl prefacename}}
% \changes{croatian-1.2}{11 jul 93}{`headpagename should be `pagename}
%    \begin{macrocode}
\addto\captionscroatian{%
  \def\prefacename{Preface}%  this needs translation!
  \def\refname{Literatura}%
  \def\abstractname{Sa\v{z}etak}%
  \def\bibname{Bibliografija}%
  \def\chaptername{Glava}%
  \def\appendixname{Dodatak}%
  \def\contentsname{Sadr\v{z}aj}%
  \def\listfigurename{Slike}%
  \def\listtablename{Tablice}%
  \def\indexname{Indeks}%
  \def\figurename{Slika}%
  \def\tablename{Tablica}%
  \def\partname{Dio}%
  \def\enclname{Prilozi}%
  \def\ccname{Kopije}%
  \def\headtoname{Prima}%
  \def\pagename{Strana}%
  \def\seename{See}%  this needs translation!
  \def\alsoname{See also}}%  this needs translation!
%    \end{macrocode}
% \end{macro}
%
% \begin{macro}{\datecroatian}
%    The macro |\datecroatian| redefines the command |\today| to
%    produce Croatian dates.
%    \begin{macrocode}
\def\datecroatian{%
\def\today{\number\day .~\ifcase\month\or
  sije\v{c}anj\or velja\v{c}a\or o\v{z}ujak\or travanj\or svibanj\or
  lipanj\or srpanj\or kolovoz\or rujan\or listopad\or studeni\or
  prosinac\fi
  \space \number\year}}
%    \end{macrocode}
% \end{macro}
%
% \begin{macro}{\extrascroatian}
% \begin{macro}{\noextrascroatian}
%    The macro |\extrascroatian| will perform all the extra
%    definitions needed for the Croatian language. The macro
%    |\noextrascroatian| is used to cancel the actions of
%    |\extrascroatian|.  For the moment these macros are empty but
%    they are defined for compatibility with the other
%    language-specific files.
%
%    \begin{macrocode}
\addto\extrascroatian{}
\addto\noextrascroatian{}
%    \end{macrocode}
% \end{macro}
% \end{macro}
%
%    Our last action is to activate the commands we have just defined,
%    by calling the macro |\selectlanguage|.
%    \begin{macrocode}
\selectlanguage{croatian}
%    \end{macrocode}
%    Finally, the category code of {\tt @} is reset to its original
%    value. The macrospace used by |\atcatcode| is freed.
% \changes{croatian-1.0a}{15 july 91}{Modified handling of catcode of @-sign.}
%    \begin{macrocode}
\catcode`\@=\atcatcode \let\atcatcode\relax
%    \end{macrocode}
%
% \Finale
%% \CharacterTable
%%  {Upper-case    \A\B\C\D\E\F\G\H\I\J\K\L\M\N\O\P\Q\R\S\T\U\V\W\X\Y\Z
%%   Lower-case    \a\b\c\d\e\f\g\h\i\j\k\l\m\n\o\p\q\r\s\t\u\v\w\x\y\z
%%   Digits        \0\1\2\3\4\5\6\7\8\9
%%   Exclamation   \!     Double quote  \"     Hash (number) \#
%%   Dollar        \$     Percent       \%     Ampersand     \&
%%   Acute accent  \'     Left paren    \(     Right paren   \)
%%   Asterisk      \*     Plus          \+     Comma         \,
%%   Minus         \-     Point         \.     Solidus       \/
%%   Colon         \:     Semicolon     \;     Less than     \<
%%   Equals        \=     Greater than  \>     Question mark \?
%%   Commercial at \@     Left bracket  \[     Backslash     \\
%%   Right bracket \]     Circumflex    \^     Underscore    \_
%%   Grave accent  \`     Left brace    \{     Vertical bar  \|
%%   Right brace   \}     Tilde         \~}
%%
\endinput
}
\DeclareOption{czech}{%% 
%% This is file `czech.sty', generated 
%% on <1995/2/23> with the docstrip utility (2.2i).
%% 
%% The original source files were:
%% 
%% czech.dtx 
%% 
%% Copyright (C) 1989-1994 by Johannes Braams 
%% All rights reserved. 
%% For additional copyright information see further down in this file. 
%% 
%% This file is part of the Babel system, release 3.4 patchlevel 2. 
%% ---------------------------------------------------------------- 
%% 
%% This file is distributed in the hope that it will be useful, 
%% but WITHOUT ANY WARRANTY; without even the implied warranty of 
%% MERCHANTABILITY or FITNESS FOR A PARTICULAR PURPOSE. 
%% 
%% 
%% IMPORTANT NOTICE: 
%% 
%% For error reports in case of UNCHANGED versions see readme file. 
%% 
%% Please do not request updates from me directly. Distribution is 
%% done through Mail-Servers and TeX organizations. 
%% 
%% You are not allowed to change this file. 
%% 
%% You are allowed to distribute this file under the condition that 
%% it is distributed together with all files mentioned in manifest.txt. 
%% 
%% If you receive only some of these files from someone, complain! 
%% 
%% You are NOT ALLOWED to distribute this file alone. You are NOT 
%% ALLOWED to take money for the distribution or use of either this 
%% file or a changed version, except for a nominal charge for copying 
%% etc. 
%%% \iffalse ============================================================
%%%  @LaTeX-style-file{
%%%     author          = "Braams J.L.",
%%%     version         = "1.3c",
%%%     date            = "26 June 1994",
%%%     time            = "01:44:32 MET",
%%%     filename        = "czech.doc",
%%%     address         = "PTT Research
%%%                        St. Paulusstraat 4
%%%                        2264 XZ Leidschendam
%%%                        The Netherlands",
%%%     telephone       = "(70) 3325051",
%%%     FAX             = "(70) 3326477",
%%%     checksum        = "36325 295 1403 11239",
%%%     email           = "J.L.Braams@research.ptt.nl (Internet)",
%%%     codetable       = "ISO/ASCII",
%%%     keywords        = "babel, czech",
%%%     supported       = "yes",
%%%     abstract        = "",
%%%     docstring       = "This file contains the czech language specific
%%%                        definitions for the babel system.",
%%%  }
%%%
%%%  ====================================================================
%%% \fi
\ifx\undefined\captionsczech
\else
  \selectlanguage{czech}
  \expandafter\endinput
\fi
\chardef\atcatcode=\catcode`\@
\catcode`\@=11\relax
\ifx\undefined\babel@core@loaded\input babel.def\relax\fi
\ProvidesFile{czech.sty}[1994/06/26 v1.3d
         Czech support from the babel system]
\ifx\undefined\originalTeX \let\originalTeX\empty \else\originalTeX\fi
\ifx\undefined\l@czech
    \@nopatterns{Czech}
    \adddialect\l@czech0\fi
\addto\captionsczech{%
  \def\prefacename{P\v redmluva}%
  \def\refname{Reference}%
  \def\abstractname{Abstrakt}%
  \def\bibname{Literatura}%
  \def\chaptername{Kapitola}%
  \def\appendixname{Dodatek}%
  \def\contentsname{Obsah}%
  \def\listfigurename{Seznam obr\'azk\w{u}}% \def\w#1{\accent'27 #1}
  \def\listtablename{Seznam tabulek}%
  \def\indexname{Index}%
  \def\figurename{Obr\'azek}%
  \def\tablename{Tabulka}%
  \def\partname{\v{C}\'ast}%
  \def\enclname{P\v{r}\'{\i}loha}%
  \def\ccname{cc}%
  \def\headtoname{Komu}%
  \def\pagename{Strana}%
  \def\seename{viz}%
  \def\alsoname{viz tek\'e}}%
\def\dateczech{%
\def\today{\number\day.~\ifcase\month\or
  ledna\or \'unora\or b\v{r}ezna\or dubna\or kv\v{e}tna\or \v{c}ervna\or
  \v{c}ervence\or srpna\or z\'a\v{r}\'{\i}\or \v{r}\'{\i}jna\or
  listopadu\or prosince\fi
  \space \number\year}}
\addto\extrasczech{\babel@save\q\let\q\czech@q}
\addto\extrasczech{\babel@save\w\let\w\czech@w}
\addto\extrasczech{%
  \ifnum\the\sfcode`\.=\@m
    \frenchspacing
    \addto\noextrasczech{\nonfrenchspacing}%
  \fi}
\def\czech@q#1{#1\kern-2pt\char'47}%
\def\czech@w#1{\accent'23 #1}%
\selectlanguage{czech}
\catcode`\@=\atcatcode \let\atcatcode\relax
%%
%% \CharacterTable
%%  {Upper-case    \A\B\C\D\E\F\G\H\I\J\K\L\M\N\O\P\Q\R\S\T\U\V\W\X\Y\Z
%%   Lower-case    \a\b\c\d\e\f\g\h\i\j\k\l\m\n\o\p\q\r\s\t\u\v\w\x\y\z
%%   Digits        \0\1\2\3\4\5\6\7\8\9
%%   Exclamation   \!     Double quote  \"     Hash (number) \#
%%   Dollar        \$     Percent       \%     Ampersand     \&
%%   Acute accent  \'     Left paren    \(     Right paren   \)
%%   Asterisk      \*     Plus          \+     Comma         \,
%%   Minus         \-     Point         \.     Solidus       \/
%%   Colon         \:     Semicolon     \;     Less than     \<
%%   Equals        \=     Greater than  \>     Question mark \?
%%   Commercial at \@     Left bracket  \[     Backslash     \\
%%   Right bracket \]     Circumflex    \^     Underscore    \_
%%   Grave accent  \`     Left brace    \{     Vertical bar  \|
%%   Right brace   \}     Tilde         \~}
%%
\endinput
%% 
%% End of file `czech.sty'.
}
\DeclareOption{danish}{%% 
%% This is file `danish.sty', generated 
%% on <1995/2/23> with the docstrip utility (2.2i).
%% 
%% The original source files were:
%% 
%% danish.dtx 
%% 
%% Copyright (C) 1989-1994 by Johannes Braams 
%% All rights reserved. 
%% For additional copyright information see further down in this file. 
%% 
%% This file is part of the Babel system, release 3.4 patchlevel 2. 
%% ---------------------------------------------------------------- 
%% 
%% This file is distributed in the hope that it will be useful, 
%% but WITHOUT ANY WARRANTY; without even the implied warranty of 
%% MERCHANTABILITY or FITNESS FOR A PARTICULAR PURPOSE. 
%% 
%% 
%% IMPORTANT NOTICE: 
%% 
%% For error reports in case of UNCHANGED versions see readme file. 
%% 
%% Please do not request updates from me directly. Distribution is 
%% done through Mail-Servers and TeX organizations. 
%% 
%% You are not allowed to change this file. 
%% 
%% You are allowed to distribute this file under the condition that 
%% it is distributed together with all files mentioned in manifest.txt. 
%% 
%% If you receive only some of these files from someone, complain! 
%% 
%% You are NOT ALLOWED to distribute this file alone. You are NOT 
%% ALLOWED to take money for the distribution or use of either this 
%% file or a changed version, except for a nominal charge for copying 
%% etc. 
%%% \iffalse ===========================================================
%%%  @LaTeX-style-file{
%%%     author          = "Braams J.L.",
%%%     version         = "1.3e",
%%%     date            = "26 June 1994",
%%%     time            = "01:24:05 MET",
%%%     filename        = "danish.doc",
%%%     address         = "PTT Research
%%%                        St. Paulusstraat 4
%%%                        2264 XZ Leidschendam
%%%                        The Netherlands",
%%%     telephone       = "(70) 3325051",
%%%     FAX             = "(70) 3326477",
%%%     checksum        = "14659 261 1219 10065",
%%%     email           = "J.L.Braams@research.ptt.nl (Internet)",
%%%     codetable       = "ISO/ASCII",
%%%     keywords        = "babel, danish",
%%%     supported       = "yes",
%%%     abstract        = "",
%%%     docstring       = "This file contains the danish language specific
%%%                        definitions for the babel system.",
%%%  }
%%%
%%%  ====================================================================
%%% \fi
\ifx\undefined\captionsdanish
\else
  \selectlanguage{danish}
  \expandafter\endinput
\fi
\chardef\atcatcode=\catcode`\@
\catcode`\@=11\relax
\ifx\undefined\babel@core@loaded\input babel.def\relax\fi
\ProvidesFile{danish.sty}[1994/06/26 v1.3f
         Danish support from the babel system]
\ifx\undefined\originalTeX \let\originalTeX\empty \else\originalTeX\fi
\ifx\undefined\l@danish
    \@nopatterns{Danish}
    \adddialect\l@danish0\fi
\addto\captionsdanish{%
  \def\prefacename{Forord}%
  \def\refname{Litteratur}%
  \def\abstractname{Resum\'e}%
  \def\bibname{Litteratur}%
  \def\chaptername{Kapitel}%
  \def\appendixname{Bilag}%
  \def\contentsname{Indhold}%
  \def\listfigurename{Figurer}%
  \def\listtablename{Tabeller}%
  \def\indexname{Indeks}%
  \def\figurename{Figur}%
  \def\tablename{Tabel}%
  \def\partname{Del}%
  \def\enclname{Vedlagt}%
  \def\ccname{Kopi til}%   or    Kopi sendt til
  \def\headtoname{Til}% in letter
  \def\pagename{Side}%
  \def\seename{Se}%
  \def\alsoname{Se ogs{\aa}}}%
\def\datedanish{%
\def\today{\number\day.~\ifcase\month\or
  januar\or februar\or marts\or april\or maj\or juni\or
  juli\or august\or september\or oktober\or november\or december\fi
  \space\number\year}}
\addto\extrasdanish{}
\addto\noextrasdanish{}
\selectlanguage{danish}
\catcode`\@=\atcatcode \let\atcatcode\relax
%%
%% \CharacterTable
%%  {Upper-case    \A\B\C\D\E\F\G\H\I\J\K\L\M\N\O\P\Q\R\S\T\U\V\W\X\Y\Z
%%   Lower-case    \a\b\c\d\e\f\g\h\i\j\k\l\m\n\o\p\q\r\s\t\u\v\w\x\y\z
%%   Digits        \0\1\2\3\4\5\6\7\8\9
%%   Exclamation   \!     Double quote  \"     Hash (number) \#
%%   Dollar        \$     Percent       \%     Ampersand     \&
%%   Acute accent  \'     Left paren    \(     Right paren   \)
%%   Asterisk      \*     Plus          \+     Comma         \,
%%   Minus         \-     Point         \.     Solidus       \/
%%   Colon         \:     Semicolon     \;     Less than     \<
%%   Equals        \=     Greater than  \>     Question mark \?
%%   Commercial at \@     Left bracket  \[     Backslash     \\
%%   Right bracket \]     Circumflex    \^     Underscore    \_
%%   Grave accent  \`     Left brace    \{     Vertical bar  \|
%%   Right brace   \}     Tilde         \~}
%%
\endinput
%% 
%% End of file `danish.sty'.
}
\DeclareOption{dutch}{%% 
%% This is file `dutch.sty', generated 
%% on <1995/2/23> with the docstrip utility (2.2i).
%% 
%% The original source files were:
%% 
%% dutch.dtx 
%% 
%% Copyright (C) 1989-1994 by Johannes Braams 
%% All rights reserved. 
%% For additional copyright information see further down in this file. 
%% 
%% This file is part of the Babel system, release 3.4 patchlevel 2. 
%% ---------------------------------------------------------------- 
%% 
%% This file is distributed in the hope that it will be useful, 
%% but WITHOUT ANY WARRANTY; without even the implied warranty of 
%% MERCHANTABILITY or FITNESS FOR A PARTICULAR PURPOSE. 
%% 
%% 
%% IMPORTANT NOTICE: 
%% 
%% For error reports in case of UNCHANGED versions see readme file. 
%% 
%% Please do not request updates from me directly. Distribution is 
%% done through Mail-Servers and TeX organizations. 
%% 
%% You are not allowed to change this file. 
%% 
%% You are allowed to distribute this file under the condition that 
%% it is distributed together with all files mentioned in manifest.txt. 
%% 
%% If you receive only some of these files from someone, complain! 
%% 
%% You are NOT ALLOWED to distribute this file alone. You are NOT 
%% ALLOWED to take money for the distribution or use of either this 
%% file or a changed version, except for a nominal charge for copying 
%% etc. 
%%% \iffalse ============================================================
%%%  @LaTeX-style-file{
%%%     author          = "Braams J.L.",
%%%     version         = "3.6c",
%%%     date            = "26 June 1994",
%%%     time            = "02:19:03 MET",
%%%     filename        = "dutch.dtx",
%%%     address         = "PTT Research
%%%                        St. Paulusstraat 4
%%%                        2264 XZ Leidschendam
%%%                        The Netherlands",
%%%     telephone       = "(70) 3325051",
%%%     FAX             = "(70) 3326477",
%%%     checksum        = "18631 699 3432 28815",
%%%     email           = "J.L.Braams@research.ptt.nl (Internet)",
%%%     codetable       = "ISO/ASCII",
%%%     keywords        = "babel, dutch",
%%%     supported       = "yes",
%%%     abstract        = "",
%%%     docstring       = "This file contains the dutch language specific
%%%                        definitions for the babel system.",
%%%  }
%%% ====================================================================
%%% \fi
\ifx\undefined\captionsdutch
\else
  \selectlanguage{dutch}
  \expandafter\endinput
\fi
\chardef\atcatcode=\catcode`\@
\catcode`\@=11\relax
\ifx\undefined\babel@core@loaded\input babel.def\relax\fi
\ProvidesFile{dutch.sty}[1994/06/26 v3.6c
         Dutch support from the babel system]
\ifx\undefined\originalTeX \let\originalTeX\empty \fi
\originalTeX
\ifx\undefined\l@dutch
  \@nopatterns{Dutch}
  \adddialect\l@dutch0
\fi
\addto\captionsdutch{%
  \def\prefacename{Voorwoord}%
  \def\refname{Referenties}%
  \def\abstractname{Samenvatting}%
  \def\bibname{Bibliografie}%
  \def\chaptername{Hoofdstuk}%
  \def\appendixname{Bijlage}%
  \def\contentsname{Inhoudsopgave}%
  \def\listfigurename{Lijst van figuren}%
  \def\listtablename{Lijst van tabellen}%
  \def\indexname{Index}%
  \def\figurename{Figuur}%
  \def\tablename{Tabel}%
  \def\partname{Deel}%
  \def\enclname{Bijlage(n)}%
  \def\ccname{cc}%
  \def\headtoname{Aan}%
  \def\pagename{Pagina}%
  \def\seename{zie}%
  \def\alsoname{zie ook}}
\def\datedutch{%
\def\today{\number\day~\ifcase\month\or
  januari\or februari\or maart\or april\or mei\or juni\or juli\or
  augustus\or september\or oktober\or november\or december\fi
  \space \number\year}}
\addto\extrasdutch{\babel@add@special\"}
\addto\noextrasdutch{\babel@remove@special\"}
\addto\extrasdutch{%
  \babel@savevariable{\catcode`\"}%
  \babel@save\active@dq
  \catcode`\"\active \let\active@dq\dutch@active@dq}
\begingroup \catcode`\"=\active
\def\x{\endgroup
  \addto\extrasdutch{\babel@save"\let"\dutch@@active@dq}}
\x
\addto\extrasdutch{\babel@save\"}
\addto\extrasdutch{\def\"{\protect\dutch@umlaut}}
\addto\extrasdutch{\babel@savevariable\lefthyphenmin
  \babel@savevariable\righthyphenmin
  \lefthyphenmin\tw@\righthyphenmin\thr@@}
\expandafter\ifx\csname U@D\endcsname\relax
  \csname newdimen\endcsname\U@D
\fi
\def\dutch@umlaut#1{\@umlaut#1\allowhyphens}
\def\@umlaut#1{%
  {\U@D 1ex%
  {\setbox\z@\hbox{\char127}\dimen@ -.45ex\advance\dimen@\ht\z@
  \ifdim 1ex<\dimen@ \fontdimen5\font\dimen@ \fi}%
  \accent127\fontdimen5\font\U@D #1}}
\def\dlqq{\protect\@dlqq}
\def\@dlqq{\save@sf@q{\set@low@box{''}\box\z@\kern-.04em\allowhyphens}}
\def\drqq{\protect\@drqq}
\def\@drqq{\save@sf@q{''}}
\begingroup \catcode`\"12
  \def\x{\endgroup
  \def\@MATHUMLAUT{\mathaccent"707F }
  \def\dq{"}}
\x
\begingroup \catcode`\"=\active
\def\x{\endgroup
  \let"\dq
  \let\active@dq\dq}
\x
\let\dieresis\"
\def\@trema#1{%
  \allowhyphens\discretionary{-}{#1}{\@umlaut #1}\allowhyphens}
\def\dutch@dqmacro#1{\csname d@dq@\string #1@dq@\endcsname}
\def\dutch@@active@dq#1{\expandafter\expandafter\expandafter
  \ifx\dutch@dqmacro{#1}\relax \expandafter\normal@dq
  \else \expandafter\dutch@@@active@dq \fi {#1}}
\def\normal@dq#1{\dq #1}
\def\dutch@@@active@dq#1{%
  \ifx\protect\relax \else \expandafter\protect \fi
  \active@dq{#1\empty}}
\def\dutch@active@dq#1{%
  \csname dutch@dq@\ifmmode second\else first\fi
    \expandafter\expandafter\expandafter\expandafter
    \expandafter\expandafter\expandafter
  \endcsname
  \dutch@dqmacro{#1}}
\def\dutch@dq@first#1#2{#1}
\def\dutch@dq@second#1#2{#2}
\def\def@dutch@dqmacro#1#2#3{\expandafter\expandafter\expandafter
  \def\dutch@dqmacro{#1}{{#2}{#3}}}
\def\let@dutch@dqmacro#1#2{\begingroup
  \edef\x{\endgroup \let
    \expandafter\expandafter\expandafter\noexpand\dutch@dqmacro{#1}%
    \expandafter\expandafter\expandafter\noexpand\dutch@dqmacro{#2}}%
  \x}
\def@dutch@dqmacro{a}{\@trema a}{\@MATHUMLAUT a}
\def@dutch@dqmacro{e}{\@trema e}{\@MATHUMLAUT e}
\def@dutch@dqmacro{i}%
        {\allowhyphens\discretionary{-}{i}{\dieresis\i}\allowhyphens}%
        {\@MATHUMLAUT \imath}
\def@dutch@dqmacro{o}{\@trema o}{\@MATHUMLAUT o}
\def@dutch@dqmacro{u}{\@trema u}{\@MATHUMLAUT u}
\def@dutch@dqmacro{`}{\dlqq{}}{\dlqq{}}
\def@dutch@dqmacro{'}{\drqq{}}{\drqq{}}
\def@dutch@dqmacro{-}{\allowhyphens-\allowhyphens}%
                     {\allowhyphens-\allowhyphens}
\def@dutch@dqmacro{|}{\discretionary{-}{}{\kern.03em}}{}
\def@dutch@dqmacro{y}{\allowhyphens i\kern-0.06\p@ j\allowhyphens}%
                     {\@MATHUMLAUT y}
\def@dutch@dqmacro{Y}{\allowhyphens I\kern-0.06\p@ J\allowhyphens}%
                     {\@MATHUMLAUT Y}
\addto\extrasdutch{\babel@save\-}
\addto\extrasdutch{\def\-{\allowhyphens
                          \discretionary{-}{}{}\allowhyphens}}
\selectlanguage{dutch}
\catcode`\@=\atcatcode \let\atcatcode\relax
%%
%% \CharacterTable
%%  {Upper-case    \A\B\C\D\E\F\G\H\I\J\K\L\M\N\O\P\Q\R\S\T\U\V\W\X\Y\Z
%%   Lower-case    \a\b\c\d\e\f\g\h\i\j\k\l\m\n\o\p\q\r\s\t\u\v\w\x\y\z
%%   Digits        \0\1\2\3\4\5\6\7\8\9
%%   Exclamation   \!     Double quote  \"     Hash (number) \#
%%   Dollar        \$     Percent       \%     Ampersand     \&
%%   Acute accent  \'     Left paren    \(     Right paren   \)
%%   Asterisk      \*     Plus          \+     Comma         \,
%%   Minus         \-     Point         \.     Solidus       \/
%%   Colon         \:     Semicolon     \;     Less than     \<
%%   Equals        \=     Greater than  \>     Question mark \?
%%   Commercial at \@     Left bracket  \[     Backslash     \\
%%   Right bracket \]     Circumflex    \^     Underscore    \_
%%   Grave accent  \`     Left brace    \{     Vertical bar  \|
%%   Right brace   \}     Tilde         \~}
%%
\endinput
%% 
%% End of file `dutch.sty'.
}
%    \end{macrocode}
%    We allow or the british english patterns to be loaded as either
%    `english' or `UKenglish'
%    \begin{macrocode}
\DeclareOption{english}{%
  \ifx\l@UKenglish\undefined
  \else
    \let\l@english\l@UKenglish
  \fi
  %% 
%% This is file `english.sty', generated 
%% on <1995/2/23> with the docstrip utility (2.2i).
%% 
%% The original source files were:
%% 
%% english.dtx 
%% 
%% Copyright (C) 1989-1994 by Johannes Braams 
%% All rights reserved. 
%% For additional copyright information see further down in this file. 
%% 
%% This file is part of the Babel system, release 3.4 patchlevel 2. 
%% ---------------------------------------------------------------- 
%% 
%% This file is distributed in the hope that it will be useful, 
%% but WITHOUT ANY WARRANTY; without even the implied warranty of 
%% MERCHANTABILITY or FITNESS FOR A PARTICULAR PURPOSE. 
%% 
%% 
%% IMPORTANT NOTICE: 
%% 
%% For error reports in case of UNCHANGED versions see readme file. 
%% 
%% Please do not request updates from me directly. Distribution is 
%% done through Mail-Servers and TeX organizations. 
%% 
%% You are not allowed to change this file. 
%% 
%% You are allowed to distribute this file under the condition that 
%% it is distributed together with all files mentioned in manifest.txt. 
%% 
%% If you receive only some of these files from someone, complain! 
%% 
%% You are NOT ALLOWED to distribute this file alone. You are NOT 
%% ALLOWED to take money for the distribution or use of either this 
%% file or a changed version, except for a nominal charge for copying 
%% etc. 
%%% \iffalse ============================================================
%%%  @LaTeX-style-file{
%%%     author          = "Braams J.L.",
%%%     version         = "3.3c",
%%%     date            = "26 June 1994",
%%%     time            = "00:40:45 MET",
%%%     filename        = "english.dtx",
%%%     address         = "PTT Research
%%%                        St. Paulusstraat 4
%%%                        2264 XZ Leidschendam
%%%                        The Netherlands",
%%%     telephone       = "(70) 3325051",
%%%     FAX             = "(70) 3326477",
%%%     checksum        = "46572 339 1547 13293",
%%%     email           = "J.L.Braams@research.ptt.nl (Internet)",
%%%     codetable       = "ISO/ASCII",
%%%     keywords        = "babel, english",
%%%     supported       = "yes",
%%%     abstract        = "",
%%%     docstring       = "This file contains the english language specific
%%%                        definitions for the babel system.",
%%%  }
%%%
%%%  ====================================================================
%%% \fi
\ifx\undefined\captionsenglish
\else
  \selectlanguage{english}
  \expandafter\endinput
\fi
\chardef\atcatcode=\catcode`\@
\catcode`\@=11\relax
\ifx\undefined\babel@core@loaded\input babel.def\relax\fi
\ProvidesFile{english.sty}[1994/06/26 v3.3c
         English support from the babel system]
\ifx\undefined\originalTeX \let\originalTeX\empty\fi
\originalTeX
\ifx\undefined\l@english
  \@nopatterns{English}
  \adddialect\l@english0\fi
\adddialect\l@american\l@english
\addto\captionsenglish{%
  \def\prefacename{Preface}%
  \def\refname{References}%
  \def\abstractname{Abstract}%
  \def\bibname{Bibliography}%
  \def\chaptername{Chapter}%
  \def\appendixname{Appendix}%
  \def\contentsname{Contents}%
  \def\listfigurename{List of Figures}%
  \def\listtablename{List of Tables}%
  \def\indexname{Index}%
  \def\figurename{Figure}%
  \def\tablename{Table}%
  \def\partname{Part}%
  \def\enclname{encl}%
  \def\ccname{cc}%
  \def\headtoname{To}%
  \def\pagename{Page}%
  \def\seename{see}%
  \def\alsoname{see also}}
\let\captionsamerican\captionsenglish
\def\dateenglish{%
\def\today{\ifcase\day\or
  1st\or 2nd\or 3rd\or 4th\or 5th\or
  6th\or 7th\or 8th\or 9th\or 10th\or
  11th\or 12th\or 13th\or 14th\or 15th\or
  16th\or 17th\or 18th\or 19th\or 20th\or
  21st\or 22nd\or 23rd\or 24th\or 25th\or
  26th\or 27th\or 28th\or 29th\or 30th\or
  31st\fi~\ifcase\month\or
  January\or February\or March\or April\or May\or June\or
  July\or August\or September\or October\or November\or December\fi
  \space \number\year}}
\def\dateamerican{%
\def\today{\ifcase\month\or
  January\or February\or March\or April\or May\or June\or
  July\or August\or September\or October\or November\or December\fi
  \space\number\day, \number\year}}
\addto\extrasenglish{}
\addto\noextrasenglish{}
\let\extrasamerican\extrasenglish
\let\noextrasamerican\noextrasenglish
\selectlanguage{english}
\catcode`\@=\atcatcode \let\atcatcode\relax
%%
%% \CharacterTable
%%  {Upper-case    \A\B\C\D\E\F\G\H\I\J\K\L\M\N\O\P\Q\R\S\T\U\V\W\X\Y\Z
%%   Lower-case    \a\b\c\d\e\f\g\h\i\j\k\l\m\n\o\p\q\r\s\t\u\v\w\x\y\z
%%   Digits        \0\1\2\3\4\5\6\7\8\9
%%   Exclamation   \!     Double quote  \"     Hash (number) \#
%%   Dollar        \$     Percent       \%     Ampersand     \&
%%   Acute accent  \'     Left paren    \(     Right paren   \)
%%   Asterisk      \*     Plus          \+     Comma         \,
%%   Minus         \-     Point         \.     Solidus       \/
%%   Colon         \:     Semicolon     \;     Less than     \<
%%   Equals        \=     Greater than  \>     Question mark \?
%%   Commercial at \@     Left bracket  \[     Backslash     \\
%%   Right bracket \]     Circumflex    \^     Underscore    \_
%%   Grave accent  \`     Left brace    \{     Vertical bar  \|
%%   Right brace   \}     Tilde         \~}
%%
\endinput
%% 
%% End of file `english.sty'.
}
\DeclareOption{esperanto}{%% 
%% This is file `esperant.sty', generated 
%% on <1995/2/23> with the docstrip utility (2.2i).
%% 
%% The original source files were:
%% 
%% esperant.dtx 
%% 
%% Copyright (C) 1989-1994 by Johannes Braams 
%% All rights reserved. 
%% For additional copyright information see further down in this file. 
%% 
%% This file is part of the Babel system, release 3.4 patchlevel 2. 
%% ---------------------------------------------------------------- 
%% 
%% This file is distributed in the hope that it will be useful, 
%% but WITHOUT ANY WARRANTY; without even the implied warranty of 
%% MERCHANTABILITY or FITNESS FOR A PARTICULAR PURPOSE. 
%% 
%% 
%% IMPORTANT NOTICE: 
%% 
%% For error reports in case of UNCHANGED versions see readme file. 
%% 
%% Please do not request updates from me directly. Distribution is 
%% done through Mail-Servers and TeX organizations. 
%% 
%% You are not allowed to change this file. 
%% 
%% You are allowed to distribute this file under the condition that 
%% it is distributed together with all files mentioned in manifest.txt. 
%% 
%% If you receive only some of these files from someone, complain! 
%% 
%% You are NOT ALLOWED to distribute this file alone. You are NOT 
%% ALLOWED to take money for the distribution or use of either this 
%% file or a changed version, except for a nominal charge for copying 
%% etc. 
%%% \iffalse ============================================================
%%%  @LaTeX-style-file{
%%%     author          = "Johannes L. Braams.",
%%%     version         = "1.4d",
%%%     date            = "26 June 1994",
%%%     time            = "00:24:15 MET",
%%%     filename        = "esperant.dtx",
%%%     address         = "PTT Research
%%%                        St. Paulusstraat 4
%%%                        2264 XZ Leidschendam
%%%                        The Netherlands",
%%%     telephone       = "(70) 3325051",
%%%     FAX             = "(70) 3326477",
%%%     checksum        = "01262 385 1778 14938",
%%%     email           = "J.L.Braams@research.ptt.nl (Internet)",
%%%     codetable       = "ISO/ASCII",
%%%     keywords        = "",
%%%     supported       = "yes",
%%%     abstract        = "",
%%%     docstring       = "This file contains the esperanto language
%%%                        specific definitions for the babel system.",
%%%  }
%%%
%%%  ====================================================================
%%% \fi
\ifx\undefined\captionsesperanto
\else
  \selectlanguage{esperanto}
  \expandafter\endinput
\fi
\chardef\atcatcode=\catcode`\@
\catcode`\@=11\relax
\ifx\undefined\babel@core@loaded\input babel.def\relax\fi
\ProvidesFile{esperant.sty}[1994/06/26 v1.4d
         Esperanto support from the babel system]
\ifx\undefined\originalTeX \let\originalTeX\empty \else\originalTeX\fi
\ifx\undefined\l@esperanto
  \@nopatterns{Esperanto}
  \adddialect\l@esperanto0\fi
\addto\captionsesperanto{%
  \def\prefacename{Anta\u{u}parolo}%
  \def\refname{Cita\^\j{}oj}%
  \def\abstractname{Resumo}%
  \def\bibname{Bibliografio}%
  \def\chaptername{{\^C}apitro}%
  \def\appendixname{Apendico}%
  \def\contentsname{Enhavo}%
  \def\listfigurename{Listo de figuroj}%
  \def\listtablename{Listo de tabeloj}%
  \def\indexname{Indekso}%
  \def\figurename{Figuro}%
  \def\tablename{Tabelo}%
  \def\partname{Parto}%
  \def\enclname{Aldono(j)}%
  \def\ccname{Kopie al}%
  \def\headtoname{Al}%
  \def\pagename{Pa\^go}%
  \def\subjectname{Temo}%
  \def\seename{vidu}%   a^u: vd.
  \def\alsoname{vidu anka\u{u}}}% a^u vd. anka\u{u}
\def\dateesperanto{%
\def\today{\number\day{--a}~de~\ifcase\month\or
  januaro\or februaro\or marto\or aprilo\or majo\or junio\or
  julio\or a\u{u}gusto\or septembro\or oktobro\or novembro\or
  decembro\fi,\space \number\year}}
\addto\extrasesperanto{\babel@add@special\^}
\addto\noextrasesperanto{\babel@remove@special\^}
\addto\extrasesperanto{%
  \babel@savevariable{\catcode`\^}%
  \babel@save\active@hat
  \catcode`\^\active \let\active@hat\esp@active@hat}
\begingroup \catcode`\^\active
\def\x{\endgroup
  \addto\extrasesperanto{\babel@save^\let^\active@hat}}
\x
\begingroup \catcode`\^\active
  \def\x{\endgroup
    \let^\sp
    \let\active@hat\sp}
\x
\def\esp@active@hat#1{%
    \ifmmode\sp{#1}%
    \else\ifx#1u\u u\allowhyphens
    \else\ifx#1U\u U\allowhyphens
    \else\ifx#1h{h\llap{\^{}}}\allowhyphens
    \else\ifx#1j\^\j\allowhyphens
    \else\ifx#1|\discretionary{-}{}{}\allowhyphens
    \else\^{#1}\allowhyphens
    \fi\fi\fi\fi\fi\fi}
\def\esper#1{\@esper{\@nameuse{c@#1}}}
\def\Esper#1{\@Esper{\@nameuse{c@#1}}}
\def\@esper#1{\ifcase#1\or a\or b\or c\or \^c\or d\else\@iesper{#1}\fi}
\def\@Esper#1{\ifcase#1\or A\or B\or C\or \^C\or D\else\@Iesper{#1}\fi}
\def\@iesper#1{\ifcase#1\or \or \or \or \or \or e\or f\or g\or \^g\or
    h\or h\llap{\^{}}\or i\or j\or \^\j\or k\orl\or m\or n\or o\or
    p\or s\or \^s\or t\or u\or \u{u}\or v\or z\else\@ctrerr\fi}
\def\@Iesper#1{\ifcase#1\or \or \or \or \or \or E\or F\or G\or \^G\or
    H\or \^H\or I\or J\or \^\J\or K\or L\or M\or N\or O\or
    P\or S\or \^S\or T\or U\or \u{U}\or V\or Z\else\@ctrerr\fi}
\addto\dateesperanto{\def\hodiau{la \today}}
\def\hodiaun{la \number\day --an~de~\ifcase\month\or
  januaro\or februaro\or marto\or aprilo\or majo\or junio\or
  julio\or a\u{u}gusto\or septembro\or oktobro\or novembro\or
  decembro\fi, \space \number\year}
\selectlanguage{esperanto}
\catcode`\@=\atcatcode \let\atcatcode\relax
%%
%% \CharacterTable
%%  {Upper-case    \A\B\C\D\E\F\G\H\I\J\K\L\M\N\O\P\Q\R\S\T\U\V\W\X\Y\Z
%%   Lower-case    \a\b\c\d\e\f\g\h\i\j\k\l\m\n\o\p\q\r\s\t\u\v\w\x\y\z
%%   Digits        \0\1\2\3\4\5\6\7\8\9
%%   Exclamation   \!     Double quote  \"     Hash (number) \#
%%   Dollar        \$     Percent       \%     Ampersand     \&
%%   Acute accent  \'     Left paren    \(     Right paren   \)
%%   Asterisk      \*     Plus          \+     Comma         \,
%%   Minus         \-     Point         \.     Solidus       \/
%%   Colon         \:     Semicolon     \;     Less than     \<
%%   Equals        \=     Greater than  \>     Question mark \?
%%   Commercial at \@     Left bracket  \[     Backslash     \\
%%   Right bracket \]     Circumflex    \^     Underscore    \_
%%   Grave accent  \`     Left brace    \{     Vertical bar  \|
%%   Right brace   \}     Tilde         \~}
%%
\endinput
%% 
%% End of file `esperant.sty'.
}
\DeclareOption{finnish}{%% 
%% This is file `finnish.sty', generated 
%% on <1995/2/23> with the docstrip utility (2.2i).
%% 
%% The original source files were:
%% 
%% finnish.dtx 
%% 
%% Copyright (C) 1989-1994 by Johannes Braams 
%% All rights reserved. 
%% For additional copyright information see further down in this file. 
%% 
%% This file is part of the Babel system, release 3.4 patchlevel 2. 
%% ---------------------------------------------------------------- 
%% 
%% This file is distributed in the hope that it will be useful, 
%% but WITHOUT ANY WARRANTY; without even the implied warranty of 
%% MERCHANTABILITY or FITNESS FOR A PARTICULAR PURPOSE. 
%% 
%% 
%% IMPORTANT NOTICE: 
%% 
%% For error reports in case of UNCHANGED versions see readme file. 
%% 
%% Please do not request updates from me directly. Distribution is 
%% done through Mail-Servers and TeX organizations. 
%% 
%% You are not allowed to change this file. 
%% 
%% You are allowed to distribute this file under the condition that 
%% it is distributed together with all files mentioned in manifest.txt. 
%% 
%% If you receive only some of these files from someone, complain! 
%% 
%% You are NOT ALLOWED to distribute this file alone. You are NOT 
%% ALLOWED to take money for the distribution or use of either this 
%% file or a changed version, except for a nominal charge for copying 
%% etc. 
%%% \iffalse ============================================================
%%%  @LaTeX-style-file{
%%%     author          = "Braams J.L.",
%%%     version         = "1.3d",
%%%     date            = "30 June 1994",
%%%     time            = "14:07:22 MET",
%%%     Filename        = "finnish.doc",
%%%     address         = "PTT Research
%%%                        St. Paulusstraat 4
%%%                        2264 XZ Leidschendam
%%%                        The Netherlands",
%%%     telephone       = "(70) 3325051",
%%%     FAX             = "(70) 3326477",
%%%     checksum        = "21000 264 1229 10160",
%%%     email           = "J.L.Braams@research.ptt.nl (Internet)",
%%%     codetable       = "ISO/ASCII",
%%%     keywords        = "babel, finnish",
%%%     supported       = "yes",
%%%     abstract        = "",
%%%     docstring       = "This file contains the finnish language
%%%                       specific  definitions for the babel system.",
%%%  }
%%%
%%%  ====================================================================
%%% \fi
\ifx\undefined\captionsfinnish
\else
  \selectlanguage{finnish}
  \expandafter\endinput
\fi
\chardef\atcatcode=\catcode`\@
\catcode`\@=11\relax
\ifx\undefined\babel@core@loaded\input babel.def\relax\fi
\ProvidesFile{finnish.sty}[1994/06/30 v1.3d
         Finnish support from the babel system]
\ifx\undefined\originalTeX \let\originalTeX\empty \else\originalTeX\fi
\ifx\undefined\l@finnish
    \@nopatterns{Finnish}
    \adddialect\l@finnish0\fi
\addto\captionsfinnish{%
  \def\prefacename{Esipuhe}%
  \def\refname{Viitteet}%
  \def\abstractname{Tiivistelm\"a}
  \def\bibname{Kirjallisuutta}%
  \def\chaptername{Luku}%
  \def\appendixname{Liite}%
  \def\contentsname{Sis\"alt\"o}%   /* Could be "Sis\"allys" as well */
  \def\listfigurename{Kuvat}%
  \def\listtablename{Taulukot}%
  \def\indexname{Hakemisto}%
  \def\figurename{Kuva}%
  \def\tablename{Taulukko}%
  \def\partname{Osa}%
  \def\enclname{Liitteet}%
  \def\ccname{Jakelu}%
  \def\headtoname{Vastaanottaja}%
  \def\pagename{Sivu}%
  \def\seename{katso}%
  \def\alsoname{katso my\"os}}%
\def\datefinnish{%
\def\today{\number\day~\ifcase\month\or
  tammikuuta\or helmikuuta\or maaliskuuta\or huhtikuuta\or
  toukokuuta\or kes\"akuuta\or hein\"akuuta\or elokuuta\or
  syyskuuta\or lokakuuta\or marraskuuta\or joulukuuta\fi
  \space\number\year}}
\addto\extrasfinnish{}
\addto\noextrasfinnish{}
\selectlanguage{finnish}
\catcode`\@=\atcatcode \let\atcatcode\relax
%%
%% \CharacterTable
%%  {Upper-case    \A\B\C\D\E\F\G\H\I\J\K\L\M\N\O\P\Q\R\S\T\U\V\W\X\Y\Z
%%   Lower-case    \a\b\c\d\e\f\g\h\i\j\k\l\m\n\o\p\q\r\s\t\u\v\w\x\y\z
%%   Digits        \0\1\2\3\4\5\6\7\8\9
%%   Exclamation   \!     Double quote  \"     Hash (number) \#
%%   Dollar        \$     Percent       \%     Ampersand     \&
%%   Acute accent  \'     Left paren    \(     Right paren   \)
%%   Asterisk      \*     Plus          \+     Comma         \,
%%   Minus         \-     Point         \.     Solidus       \/
%%   Colon         \:     Semicolon     \;     Less than     \<
%%   Equals        \=     Greater than  \>     Question mark \?
%%   Commercial at \@     Left bracket  \[     Backslash     \\
%%   Right bracket \]     Circumflex    \^     Underscore    \_
%%   Grave accent  \`     Left brace    \{     Vertical bar  \|
%%   Right brace   \}     Tilde         \~}
%%
\endinput
%% 
%% End of file `finnish.sty'.
}
%    \end{macrocode}
%    The \babel\ support or French is stored in \file{francais.sty};
%    thereore the \LaTeX2.09 option used to be \Lopt{francais}.
%    The hyphenation patterns may be loaded as either `french' or as
%    `francais'.
%    \begin{macrocode}
\DeclareOption{francais}{%
  \ifx\l@francais\undefined
    \let\l@francais\l@french
  \fi
  %% 
%% This is file `francais.sty', generated 
%% on <1995/2/23> with the docstrip utility (2.2i).
%% 
%% The original source files were:
%% 
%% francais.dtx 
%% 
%% Copyright (C) 1989-1994 by Johannes Braams 
%% All rights reserved. 
%% For additional copyright information see further down in this file. 
%% 
%% This file is part of the Babel system, release 3.4 patchlevel 2. 
%% ---------------------------------------------------------------- 
%% 
%% This file is distributed in the hope that it will be useful, 
%% but WITHOUT ANY WARRANTY; without even the implied warranty of 
%% MERCHANTABILITY or FITNESS FOR A PARTICULAR PURPOSE. 
%% 
%% 
%% IMPORTANT NOTICE: 
%% 
%% For error reports in case of UNCHANGED versions see readme file. 
%% 
%% Please do not request updates from me directly. Distribution is 
%% done through Mail-Servers and TeX organizations. 
%% 
%% You are not allowed to change this file. 
%% 
%% You are allowed to distribute this file under the condition that 
%% it is distributed together with all files mentioned in manifest.txt. 
%% 
%% If you receive only some of these files from someone, complain! 
%% 
%% You are NOT ALLOWED to distribute this file alone. You are NOT 
%% ALLOWED to take money for the distribution or use of either this 
%% file or a changed version, except for a nominal charge for copying 
%% etc. 
%%% \iffalse ============================================================
%%%  @LaTeX-style-file{
%%%     author          = "Braams J.L.",
%%%     version         = "4.5c",
%%%     date            = "27 June 1994",
%%%     time            = "23:49:20 MET",
%%%     filename        = "francais.dtx",
%%%     address         = "PTT Research
%%%                        St. Paulusstraat 4
%%%                        2264 XZ Leidschendam
%%%                        The Netherlands",
%%%     telephone       = "(70) 3325051",
%%%     FAX             = "(70) 3326477",
%%%     checksum        = "16184 485 2054 18844",
%%%     email           = "J.L.Braams@research.ptt.nl (Internet)",
%%%     codetable       = "ISO/ASCII",
%%%     keywords        = "babel, french",
%%%     supported       = "yes",
%%%     abstract        = "",
%%%     docstring       = "This file contains the french language
%%%                        specific definitions for the babel system.",
%%%  }
%%%
%%%  ====================================================================
%%% \fi
\ifx\undefined\captionsfrancais
\else
  \selectlanguage{francais}
  \expandafter\endinput
\fi
\chardef\atcatcode=\catcode`\@
\catcode`\@=11\relax
\ifx\undefined\babel@core@loaded\input babel.def\relax\fi
\ProvidesFile{francais.sty}[1994/06/26 v4.5c
         French support from the babel system]
\ifx\undefined\originalTeX \let\originalTeX\empty \fi
\originalTeX
\ifx\undefined\l@francais
    \@nopatterns{Francais}
    \adddialect\l@francais0\fi
\addto\captionsfrancais{%
  \def\prefacename{Pr\'eface}%
  \def\refname{R\'ef\'erences}%
  \def\abstractname{R\'esum\'e}%
  \def\bibname{Bibliographie}%
  \def\chaptername{Chapitre}%
  \def\appendixname{Annexe}%
  \def\contentsname{Table des mati\`eres}%
  \def\listfigurename{Liste des figures}%
  \def\listtablename{Liste des tableaux}%
  \def\indexname{Index}%
  \def\figurename{Figure}%
  \def\tablename{Tableau}%
  \def\partname{Partie}%
  \def\enclname{P.~J.}%
  \def\ccname{Copie \`a}%
  \def\headtoname{A}
  \def\pagename{Page}%
  \def\seename{voir}%
  \def\alsoname{voir aussi}}
\def\datefrancais{%
\def\today{\ifnum\day=1\relax 1\/$^{\rm er}$\else
  \number\day\fi \space\ifcase\month\or
  janvier\or f\'evrier\or mars\or avril\or mai\or juin\or
  juillet\or ao\^ut\or septembre\or octobre\or novembre\or
  d\'ecembre\fi
  \space\number\year}}
\addto\extrasfrancais{%
  \babel@add@special\;\babel@add@special\:
  \babel@add@special\!\babel@add@special\?}
\addto\noextrasfrancais{\babel@remove@special\;\babel@remove@special\:%
                        \babel@remove@special\!\babel@remove@special\?}
\addto\extrasfrancais{%
  \babel@savevariable{\catcode`\;}\babel@save\active@sc
  \babel@savevariable{\catcode`\:}\babel@save\active@co
  \babel@savevariable{\catcode`\!}\babel@save\active@em
  \babel@savevariable{\catcode`\?}\babel@save\active@qm}
\addto\extrasfrancais{%
  \catcode`\;\active\catcode`\:\active
  \catcode`\!\active\catcode`\?\active}
\begingroup
\catcode`\;\active\catcode`\:\active
\catcode`\!\active\catcode`\?\active
\def\x{\endgroup
  \addto\extrasfrancais{%
    \babel@save;\def;{\protect\active@sc}\let\active@sc\french@active@sc
    \babel@save:\def:{\protect\active@co}\let\active@co\french@active@co
    \babel@save!\def!{\protect\active@em}\let\active@em\french@active@em
    \babel@save?\def?{\protect\active@qm}\let\active@qm\french@active@qm}}
\x
\addto\extrasfrancais{%
  \babel@save\"
  \def\"{\protect\french@dieresis}%
  \babel@save\^
  \def\^{\protect\french@circumflex}}
\addto\extrasfrancais{%
    \ifnum\the\sfcode`\.=\@m
      \frenchspacing
      \addto\noextrasfrancais{\nonfrenchspacing}%
    \fi}
\def\french@active@sc{%
    \ifmmode
      \string;%
    \else\relax
      \ifhmode
        \ifdim\lastskip>\z@
          \unskip\penalty\@M\thinspace
        \fi
      \fi
      \string;%
    \fi}%
\def\french@active@co{\ifmmode\string:\else\relax
    \ifhmode\ifdim\lastskip>\z@
    \unskip\penalty\@M\thinspace\fi\fi\string:\fi}
\def\french@active@em{\ifmmode\string!\else\relax
    \ifhmode\ifdim\lastskip>\z@
    \unskip\penalty\@M\thinspace\fi\fi\string!\fi}
\def\french@active@qm{%
    \ifmmode\string?
    \else\relax%
      \ifhmode\ifdim\lastskip>\z@
        \unskip\kern\fontdimen2\font
        \kern-1.4\fontdimen3\font
    \fi\fi
    \string?\fi}
\def\at{{\char '100}}
\def\boi{{\tt\char '134}}
\def\circonflexe{{\char '136}}
\def\tild{{\char '176}}
\def\degre{{\char '27}}
\def\french@circumflex#1{{\accent"5E \if#1i\i\else#1\fi}}
\def\french@dieresis#1{{\accent"7F \if#1i\i\else#1\fi}}
\def\up#1{\raise 1ex\hbox{\small#1}}
\def\No{\kern-.25em\lower.2ex\hbox{\degre}}
\def\ieme{$^{\rm e }$\kern+.17em}
\def\FrenchEnumerate#1{$#1^{\rm o}$\kern+.29em}
\def\FrenchPopularEnumerate#1{#1\No\kern-.25em)\kern+.3em}
\def\primo{\FrenchEnumerate1}
\def\secundo{\FrenchEnumerate2}
\def\tertio{\FrenchEnumerate3}
\def\quatro{\FrenchEnumerate4}
\def\fprimo){\FrenchPopularEnumerate1}
\def\fsecundo){\FrenchPopularEnumerate2}
\def\ftertio){\FrenchPopularEnumerate3}
\def\fquatro){\FrenchPopularEnumerate4}
\selectlanguage{francais}
\catcode`\@=\atcatcode \let\atcatcode\relax
%%
%% \CharacterTable
%%  {Upper-case    \A\B\C\D\E\F\G\H\I\J\K\L\M\N\O\P\Q\R\S\T\U\V\W\X\Y\Z
%%   Lower-case    \a\b\c\d\e\f\g\h\i\j\k\l\m\n\o\p\q\r\s\t\u\v\w\x\y\z
%%   Digits        \0\1\2\3\4\5\6\7\8\9
%%   Exclamation   \!     Double quote  \"     Hash (number) \#
%%   Dollar        \$     Percent       \%     Ampersand     \&
%%   Acute accent  \'     Left paren    \(     Right paren   \)
%%   Asterisk      \*     Plus          \+     Comma         \,
%%   Minus         \-     Point         \.     Solidus       \/
%%   Colon         \:     Semicolon     \;     Less than     \<
%%   Equals        \=     Greater than  \>     Question mark \?
%%   Commercial at \@     Left bracket  \[     Backslash     \\
%%   Right bracket \]     Circumflex    \^     Underscore    \_
%%   Grave accent  \`     Left brace    \{     Vertical bar  \|
%%   Right brace   \}     Tilde         \~}
%%
\endinput
%% 
%% End of file `francais.sty'.
}
%    \end{macrocode}
%    With \LaTeXe\ we can now also use the option \Lopt{french} and
%    still call the file \file{francais.sty}.
%    \begin{macrocode}
\DeclareOption{french}{%
  \ifx\l@french\undefined
    \let\l@french\l@francais
  \fi
  %% 
%% This is file `francais.sty', generated 
%% on <1995/2/23> with the docstrip utility (2.2i).
%% 
%% The original source files were:
%% 
%% francais.dtx 
%% 
%% Copyright (C) 1989-1994 by Johannes Braams 
%% All rights reserved. 
%% For additional copyright information see further down in this file. 
%% 
%% This file is part of the Babel system, release 3.4 patchlevel 2. 
%% ---------------------------------------------------------------- 
%% 
%% This file is distributed in the hope that it will be useful, 
%% but WITHOUT ANY WARRANTY; without even the implied warranty of 
%% MERCHANTABILITY or FITNESS FOR A PARTICULAR PURPOSE. 
%% 
%% 
%% IMPORTANT NOTICE: 
%% 
%% For error reports in case of UNCHANGED versions see readme file. 
%% 
%% Please do not request updates from me directly. Distribution is 
%% done through Mail-Servers and TeX organizations. 
%% 
%% You are not allowed to change this file. 
%% 
%% You are allowed to distribute this file under the condition that 
%% it is distributed together with all files mentioned in manifest.txt. 
%% 
%% If you receive only some of these files from someone, complain! 
%% 
%% You are NOT ALLOWED to distribute this file alone. You are NOT 
%% ALLOWED to take money for the distribution or use of either this 
%% file or a changed version, except for a nominal charge for copying 
%% etc. 
%%% \iffalse ============================================================
%%%  @LaTeX-style-file{
%%%     author          = "Braams J.L.",
%%%     version         = "4.5c",
%%%     date            = "27 June 1994",
%%%     time            = "23:49:20 MET",
%%%     filename        = "francais.dtx",
%%%     address         = "PTT Research
%%%                        St. Paulusstraat 4
%%%                        2264 XZ Leidschendam
%%%                        The Netherlands",
%%%     telephone       = "(70) 3325051",
%%%     FAX             = "(70) 3326477",
%%%     checksum        = "16184 485 2054 18844",
%%%     email           = "J.L.Braams@research.ptt.nl (Internet)",
%%%     codetable       = "ISO/ASCII",
%%%     keywords        = "babel, french",
%%%     supported       = "yes",
%%%     abstract        = "",
%%%     docstring       = "This file contains the french language
%%%                        specific definitions for the babel system.",
%%%  }
%%%
%%%  ====================================================================
%%% \fi
\ifx\undefined\captionsfrancais
\else
  \selectlanguage{francais}
  \expandafter\endinput
\fi
\chardef\atcatcode=\catcode`\@
\catcode`\@=11\relax
\ifx\undefined\babel@core@loaded\input babel.def\relax\fi
\ProvidesFile{francais.sty}[1994/06/26 v4.5c
         French support from the babel system]
\ifx\undefined\originalTeX \let\originalTeX\empty \fi
\originalTeX
\ifx\undefined\l@francais
    \@nopatterns{Francais}
    \adddialect\l@francais0\fi
\addto\captionsfrancais{%
  \def\prefacename{Pr\'eface}%
  \def\refname{R\'ef\'erences}%
  \def\abstractname{R\'esum\'e}%
  \def\bibname{Bibliographie}%
  \def\chaptername{Chapitre}%
  \def\appendixname{Annexe}%
  \def\contentsname{Table des mati\`eres}%
  \def\listfigurename{Liste des figures}%
  \def\listtablename{Liste des tableaux}%
  \def\indexname{Index}%
  \def\figurename{Figure}%
  \def\tablename{Tableau}%
  \def\partname{Partie}%
  \def\enclname{P.~J.}%
  \def\ccname{Copie \`a}%
  \def\headtoname{A}
  \def\pagename{Page}%
  \def\seename{voir}%
  \def\alsoname{voir aussi}}
\def\datefrancais{%
\def\today{\ifnum\day=1\relax 1\/$^{\rm er}$\else
  \number\day\fi \space\ifcase\month\or
  janvier\or f\'evrier\or mars\or avril\or mai\or juin\or
  juillet\or ao\^ut\or septembre\or octobre\or novembre\or
  d\'ecembre\fi
  \space\number\year}}
\addto\extrasfrancais{%
  \babel@add@special\;\babel@add@special\:
  \babel@add@special\!\babel@add@special\?}
\addto\noextrasfrancais{\babel@remove@special\;\babel@remove@special\:%
                        \babel@remove@special\!\babel@remove@special\?}
\addto\extrasfrancais{%
  \babel@savevariable{\catcode`\;}\babel@save\active@sc
  \babel@savevariable{\catcode`\:}\babel@save\active@co
  \babel@savevariable{\catcode`\!}\babel@save\active@em
  \babel@savevariable{\catcode`\?}\babel@save\active@qm}
\addto\extrasfrancais{%
  \catcode`\;\active\catcode`\:\active
  \catcode`\!\active\catcode`\?\active}
\begingroup
\catcode`\;\active\catcode`\:\active
\catcode`\!\active\catcode`\?\active
\def\x{\endgroup
  \addto\extrasfrancais{%
    \babel@save;\def;{\protect\active@sc}\let\active@sc\french@active@sc
    \babel@save:\def:{\protect\active@co}\let\active@co\french@active@co
    \babel@save!\def!{\protect\active@em}\let\active@em\french@active@em
    \babel@save?\def?{\protect\active@qm}\let\active@qm\french@active@qm}}
\x
\addto\extrasfrancais{%
  \babel@save\"
  \def\"{\protect\french@dieresis}%
  \babel@save\^
  \def\^{\protect\french@circumflex}}
\addto\extrasfrancais{%
    \ifnum\the\sfcode`\.=\@m
      \frenchspacing
      \addto\noextrasfrancais{\nonfrenchspacing}%
    \fi}
\def\french@active@sc{%
    \ifmmode
      \string;%
    \else\relax
      \ifhmode
        \ifdim\lastskip>\z@
          \unskip\penalty\@M\thinspace
        \fi
      \fi
      \string;%
    \fi}%
\def\french@active@co{\ifmmode\string:\else\relax
    \ifhmode\ifdim\lastskip>\z@
    \unskip\penalty\@M\thinspace\fi\fi\string:\fi}
\def\french@active@em{\ifmmode\string!\else\relax
    \ifhmode\ifdim\lastskip>\z@
    \unskip\penalty\@M\thinspace\fi\fi\string!\fi}
\def\french@active@qm{%
    \ifmmode\string?
    \else\relax%
      \ifhmode\ifdim\lastskip>\z@
        \unskip\kern\fontdimen2\font
        \kern-1.4\fontdimen3\font
    \fi\fi
    \string?\fi}
\def\at{{\char '100}}
\def\boi{{\tt\char '134}}
\def\circonflexe{{\char '136}}
\def\tild{{\char '176}}
\def\degre{{\char '27}}
\def\french@circumflex#1{{\accent"5E \if#1i\i\else#1\fi}}
\def\french@dieresis#1{{\accent"7F \if#1i\i\else#1\fi}}
\def\up#1{\raise 1ex\hbox{\small#1}}
\def\No{\kern-.25em\lower.2ex\hbox{\degre}}
\def\ieme{$^{\rm e }$\kern+.17em}
\def\FrenchEnumerate#1{$#1^{\rm o}$\kern+.29em}
\def\FrenchPopularEnumerate#1{#1\No\kern-.25em)\kern+.3em}
\def\primo{\FrenchEnumerate1}
\def\secundo{\FrenchEnumerate2}
\def\tertio{\FrenchEnumerate3}
\def\quatro{\FrenchEnumerate4}
\def\fprimo){\FrenchPopularEnumerate1}
\def\fsecundo){\FrenchPopularEnumerate2}
\def\ftertio){\FrenchPopularEnumerate3}
\def\fquatro){\FrenchPopularEnumerate4}
\selectlanguage{francais}
\catcode`\@=\atcatcode \let\atcatcode\relax
%%
%% \CharacterTable
%%  {Upper-case    \A\B\C\D\E\F\G\H\I\J\K\L\M\N\O\P\Q\R\S\T\U\V\W\X\Y\Z
%%   Lower-case    \a\b\c\d\e\f\g\h\i\j\k\l\m\n\o\p\q\r\s\t\u\v\w\x\y\z
%%   Digits        \0\1\2\3\4\5\6\7\8\9
%%   Exclamation   \!     Double quote  \"     Hash (number) \#
%%   Dollar        \$     Percent       \%     Ampersand     \&
%%   Acute accent  \'     Left paren    \(     Right paren   \)
%%   Asterisk      \*     Plus          \+     Comma         \,
%%   Minus         \-     Point         \.     Solidus       \/
%%   Colon         \:     Semicolon     \;     Less than     \<
%%   Equals        \=     Greater than  \>     Question mark \?
%%   Commercial at \@     Left bracket  \[     Backslash     \\
%%   Right bracket \]     Circumflex    \^     Underscore    \_
%%   Grave accent  \`     Left brace    \{     Vertical bar  \|
%%   Right brace   \}     Tilde         \~}
%%
\endinput
%% 
%% End of file `francais.sty'.
%
  \let\captionsfrench\captionsfrancais
  \let\datefrench\datefrancais
  \let\extrasfrench\extrasfrancais
  \let\noextrasfrench\noextrasfrancais
  }
\DeclareOption{galician}{%% 
%% This is file `galician.sty', generated 
%% on <1995/2/23> with the docstrip utility (2.2i).
%% 
%% The original source files were:
%% 
%% galician.dtx 
%% 
%% Copyright (C) 1989-1994 by Johannes Braams 
%% All rights reserved. 
%% For additional copyright information see further down in this file. 
%% 
%% This file is part of the Babel system, release 3.4 patchlevel 2. 
%% ---------------------------------------------------------------- 
%% 
%% This file is distributed in the hope that it will be useful, 
%% but WITHOUT ANY WARRANTY; without even the implied warranty of 
%% MERCHANTABILITY or FITNESS FOR A PARTICULAR PURPOSE. 
%% 
%% 
%% IMPORTANT NOTICE: 
%% 
%% For error reports in case of UNCHANGED versions see readme file. 
%% 
%% Please do not request updates from me directly. Distribution is 
%% done through Mail-Servers and TeX organizations. 
%% 
%% You are not allowed to change this file. 
%% 
%% You are allowed to distribute this file under the condition that 
%% it is distributed together with all files mentioned in manifest.txt. 
%% 
%% If you receive only some of these files from someone, complain! 
%% 
%% You are NOT ALLOWED to distribute this file alone. You are NOT 
%% ALLOWED to take money for the distribution or use of either this 
%% file or a changed version, except for a nominal charge for copying 
%% etc. 
%%% \iffalse ===========================================================
%%%  @LaTeX-style-file{
%%%     author_1        = "Manuel Carriba",
%%%     author_2        = "Johannes L. Braams",
%%%     version         = "1.1c
%%%     date            = "26 June 1994",
%%%     time            = "01:16:07 MET",
%%%     filename        = "galician.doc",
%%%     address         = "PTT Research
%%%                        St. Paulusstraat 4
%%%                        2264 XZ Leidschendam
%%%                        The Netherlands",
%%%     telephone       = "(70) 3325051",
%%%     FAX             = "(70) 3326477",
%%%     checksum        = "28930 712 3452 27531",
%%%     email_1         = "M.Carriba@dcs.shef.ac.uk (Internet)",
%%%     email_2         = "J.L.Braams@research.ptt.nl (Internet)",
%%%     codetable       = "ISO/ASCII",
%%%     keywords        = "babel, galician",
%%%     supported       = "yes",
%%%     abstract        = "",
%%%     docstring       = "This file contains the galician language
%%%                        specific definitions for the babel system.
%%%
%%%                        The checksum field above contains a CRC-16
%%%                        checksum as the first value, followed by the
%%%                        equivalent of the standard UNIX wc (word
%%%                        count) utility output of lines, words, and
%%%                        characters.  This is produced by Robert
%%%                        Solovay's checksum utility.",
%%%  }
%%%
%%%  ====================================================================
%%% \fi
\ifx\undefined\captionsgalician
\else
  \selectlanguage{galician}
  \expandafter\endinput
\fi
\ProvidesFile{galician.sty}[1994/06/26 v1.1c
         Galician support from the babel system]
\chardef\atcatcode=\catcode`\@
\catcode`\@=11\relax
\ifx\undefined\babel@core@loaded\input babel.def\relax\fi
\ifx\undefined\originalTeX
  \let\originalTeX\empty
\fi
\originalTeX
\ifx\undefined\l@galician
  \@nopatterns{Galician}
  \adddialect\l@galician0\fi
\addto\captionsgalician{%
  \def\prefacename{Preface}%   <- This needs translation!!
  \def\refname{Referencias}%
  \def\abstractname{Resume}%
  \def\bibname{Bibliograf\'{\i}a}%
  \def\chaptername{Cap\'{\i}tulo}%
  \def\appendixname{Ap\'endice}%
  \def\contentsname{\'Indice Xeral}%
  \def\listfigurename{\'Indice de Figuras}%
  \def\listtablename{\'Indice de T\'aboas}%
  \def\indexname{\'Indice de Materias}%
  \def\figurename{Figura}%
  \def\tablename{T\'aboa}%
  \def\partname{Parte}%
  \def\enclname{Adxunto}%
  \def\ccname{Copia a}%
  \def\headtoname{A}% Is this OK??
  \def\pagename{P\'axina}%
  \def\seename{See}%           <- This needs translation!!
  \def\alsoname{See also}%     <- This needs translation!!
}
\def\dategalician{%
  \def\today{\number\day~de\space\ifcase\month\or
    xaneiro\or febreiro\or marzal\or abril\or maio\or xu\~no\or
    xullo\or agosto\or setembro\or outubro\or novembro\or decembro\fi
    \space de~\number\year}}
\addto\extrasgalician{%
  \babel@add@special\"%
  \babel@add@special\'%
  \babel@add@special\~}
\addto\noextrasgalician{%
  \babel@remove@special\"%
  \babel@remove@special\'%
  \babel@remove@special\~}
\addto\extrasgalician{%
  \babel@savevariable{\catcode`\"}\babel@save\active@dq
  \babel@savevariable{\catcode`\'}\babel@save\active@ac
  \babel@savevariable{\catcode`\~}\babel@save\active@til
  \babel@save~}
\addto\extrasgalician{\catcode`\"\active\catcode`\'\active%
                     \catcode`\~\active}
\begingroup \catcode`\"\active \catcode`\~\active \catcode`\'\active
\def\x{\endgroup
  \addto\extrasgalician{%
    \def'{\protect\active@ac}\let\active@ac\galician@active@ac
    \def~{\protect\active@til}\let\active@til\galician@active@til
    \def"{\protect\active@dq}\let\active@dq\galician@active@dq}}
\x
\addto\extrasgalician{\babel@save\"\babel@save\'%
                      \babel@save\~\babel@save\a}
\addto\extrasgalician{\let\a\galician@a}
\addto\extrasgalician{\def\"{\protect\@umlaut}%
                      \def\'{\protect\@acute}%
                      \def\~{\protect\@tilde}}
\def\galician@a#1{{\if\string#1'\aftergroup\@acute
            \else \expandafter\aftergroup\csname a#1\endcsname\fi}}
\begingroup \catcode`\"12
\edef\x{\endgroup
  \def\noexpand\dq{"}
  \def\noexpand\til{~}}
\x
\let\dieresis\"
\let\textacute\'
\let\texttilde\~
\def\@umlaut#1{\expandafter\ifx\csname @um@\ifx #1\i \dotlessi%
                                           \else #1\fi\endcsname\relax%
        \allowhyphens\dieresis{#1}\allowhyphens
        \else\csname @um@#1\endcsname\fi}
\def\@acute#1{\expandafter\ifx\csname @ac@\ifx #1\i \dotlessi%
                                          \else #1\fi\endcsname\relax%
        \allowhyphens\textacute{#1}\allowhyphens%
        \else\csname @ac@#1\endcsname\fi}
\def\@tilde#1{\expandafter\ifx\csname @til@\ifx #1\i \dotlessi%
                                           \else #1\fi\endcsname\relax%
        \allowhyphens\texttilde{#1}\allowhyphens%
        \else\csname @til@#1\endcsname\fi}
\def\galician@active@dq{\futurelet\next\first@galician@active@dq}
\def\first@galician@active@dq{%
  \begingroup
  \ifx \next\bgroup \aftergroup\dq
    \else\ifx \next\egroup \aftergroup\dq
    \else\expandafter\ifx \space\next \aftergroup\dq
    \else\aftergroup\second@galician@active@dq
    \fi\fi\fi
  \endgroup}
\def\second@galician@active@dq#1{%
  \ifx #1-\allowhyphens-\allowhyphens
  \else\ifx #1|\discretionary{-}{}{\kern.03em}%
  \else\ifx #1u\@umlaut{u}%
  \else\ifx #1U\@umlaut{U}%
  \else\ifx #1o{\raise1ex\hbox{\underbar{\scriptsize o}}}%
  \else\ifx #1a{\raise1ex\hbox{\underbar{\scriptsize a}}}%
  \else \dq #1\fi\fi\fi\fi\fi\fi}
\def\galician@active@ac{{%
    \ifmmode\aftergroup\galician@math@ac
    \else\aftergroup\first@galician@active@ac\fi}}
\def\galician@math@ac{^\bgroup\catcode`\'=12\prim@s}
\def\first@galician@active@ac{\futurelet\next\second@galician@active@ac}
\def\second@galician@active@ac{%
  \begingroup
    \ifx \next\bgroup \aftergroup\rq
    \else\ifx \next\egroup \aftergroup\rq
    \else\expandafter\ifx \space\next \aftergroup\rq
    \else\aftergroup\third@galician@active@ac
    \fi\fi\fi \endgroup}
\def\third@galician@active@ac#1{%
     \ifx #1a\@acute{a}%
\else\ifx #1e\@acute{e}%
\else\ifx #1i\@acute{\dotlessi}%
\else\ifx #1o\@acute{o}%
\else\ifx #1u\@acute{u}%
\else\ifx #1A\@acute{A}%
\else\ifx #1E\@acute{E}%
\else\ifx #1I\@acute{I}%
\else\ifx #1O\@acute{O}%
\else\ifx #1U\@acute{U}%
\else\ifx #1n\@tilde{n}%  FTC compatibility
\else\ifx #1N\@tilde{N}%  FTC compatibility
\else\if\string#1'\rq\rq% Funny, this one gets the ligature right
\else\rq #1\fi\fi\fi\fi\fi\fi\fi\fi\fi\fi\fi\fi\fi}
\def\galician@active@til{\futurelet\next\first@galician@active@til}
\def\first@galician@active@til{%
  \begingroup
    \ifx \next\bgroup \aftergroup\til
    \else\ifx \next\egroup \aftergroup\til
    \else\expandafter\ifx \space\next \aftergroup\til
    \else\ifx n\next \@tilde{n}\aftergroup\@gobble
    \else\ifx N\next \@tilde{N}\aftergroup\@gobble
    \else\aftergroup\til
    \fi\fi\fi\fi\fi \endgroup}
    \def\thechar#1{{\uccode`!=#1\relax\uppercase{\gdef\next{!}}}}
    \thechar\i\global\let\dotlessi\next
\def\thechar#1{{\uccode`!=#1\relax\uppercase{\gdef\next{!}}}}
\thechar\i\global\let\dotlessi\next
\def\-{\allowhyphens\discretionary{-}{}{}\allowhyphens}
\selectlanguage{galician}
\catcode`\@|\atcatcode \let\atcatcode\relax
%% \CharacterTable
%%  {Upper-case    \A\B\C\D\E\F\G\H\I\J\K\L\M\N\O\P\Q\R\S\T\U\V\W\X\Y\Z
%%   Lower-case    \a\b\c\d\e\f\g\h\i\j\k\l\m\n\o\p\q\r\s\t\u\v\w\x\y\z
%%   Digits        \0\1\2\3\4\5\6\7\8\9
%%   Exclamation   \!     Double quote  \"     Hash (number) \#
%%   Dollar        \$     Percent       \%     Ampersand     \&
%%   Acute accent  \'     Left paren    \(     Right paren   \)
%%   Asterisk      \*     Plus          \+     Comma         \,
%%   Minus         \-     Point         \.     Solidus       \/
%%   Colon         \:     Semicolon     \;     Less than     \<
%%   Equals        \=     Greater than  \>     Question mark \?
%%   Commercial at \@     Left bracket  \[     Backslash     \\
%%   Right bracket \]     Circumflex    \^     Underscore    \_
%%   Grave accent  \`     Left brace    \{     Vertical bar  \|
%%   Right brace   \}     Tilde         \~}
%%
\endinput
%% 
%% End of file `galician.sty'.
}
\DeclareOption{german}{%% 
%% This is file `germanb.sty', generated 
%% on <1995/2/23> with the docstrip utility (2.2i).
%% 
%% The original source files were:
%% 
%% germanb.dtx 
%% 
%% Copyright (C) 1989-1994 by Johannes Braams 
%% All rights reserved. 
%% For additional copyright information see further down in this file. 
%% 
%% This file is part of the Babel system, release 3.4 patchlevel 2. 
%% ---------------------------------------------------------------- 
%% 
%% This file is distributed in the hope that it will be useful, 
%% but WITHOUT ANY WARRANTY; without even the implied warranty of 
%% MERCHANTABILITY or FITNESS FOR A PARTICULAR PURPOSE. 
%% 
%% 
%% IMPORTANT NOTICE: 
%% 
%% For error reports in case of UNCHANGED versions see readme file. 
%% 
%% Please do not request updates from me directly. Distribution is 
%% done through Mail-Servers and TeX organizations. 
%% 
%% You are not allowed to change this file. 
%% 
%% You are allowed to distribute this file under the condition that 
%% it is distributed together with all files mentioned in manifest.txt. 
%% 
%% If you receive only some of these files from someone, complain! 
%% 
%% You are NOT ALLOWED to distribute this file alone. You are NOT 
%% ALLOWED to take money for the distribution or use of either this 
%% file or a changed version, except for a nominal charge for copying 
%% etc. 
%%% \iffalse ============================================================
%%%  @LaTeX-style-file{
%%%     author-1        = "Bernd Raichle",
%%%     author-2        = "Braams J.L.",
%%%     version         = "2.5c",
%%%     date            = "26 June 1994",
%%%     time            = "00:38:56 MET",
%%%     filename        = "germanb.dtx",
%%%     address         = "PTT Research
%%%                        St. Paulusstraat 4
%%%                        2264 XZ Leidschendam
%%%                        The Netherlands",
%%%     telephone-2     = "(70) 3325051",
%%%     FAX-2           = "(70) 3326477",
%%%     checksum        = "29603 839 3842 33542",
%%%     email-1         = "raichle@azu.Informatik.Uni-Stuttgart.de",
%%%     email-2         = "J.L.Braams@research.ptt.nl (Internet)",
%%%     codetable       = "ISO/ASCII",
%%%     keywords        = "babel, german",
%%%     supported       = "yes",
%%%     abstract        = "",
%%%     docstring       = "This file contains the german language specific
%%%                        definitions for the babel system.
%%%                        It is based on german.tex 2.3e.",
%%%  }
%%%
%%%  ====================================================================
%%% \fi
\ifx\undefined\captionsgerman
\else
  \selectlanguage{german}
  \expandafter\endinput
\fi
\chardef\atcatcode=\catcode`\@
\catcode`\@=11\relax
\ifx\undefined\babel@core@loaded\input babel.def\relax\fi
\ProvidesFile{germanb.sty}[1994/06/26 v2.5c
         German support from the babel system]
\ifx\undefined\originalTeX \let\originalTeX\empty\fi
\originalTeX
\ifx\undefined\l@german
  \@nopatterns{German}
  \adddialect\l@german0
\fi
\adddialect\l@austrian\l@german
\addto\captionsgerman{%
  \def\prefacename{Vorwort}%
  \def\refname{Literatur}%
  \def\abstractname{Zusammenfassung}%
  \def\bibname{Literaturverzeichnis}%
  \def\chaptername{Kapitel}%
  \def\appendixname{Anhang}%
  \def\contentsname{Inhaltsverzeichnis}%    % oder nur: Inhalt
  \def\listfigurename{Abbildungsverzeichnis}%
  \def\listtablename{Tabellenverzeichnis}%
  \def\indexname{Index}%
  \def\figurename{Abbildung}%
  \def\tablename{Tabelle}%                  % oder: Tafel
  \def\partname{Teil}%
  \def\enclname{Anlage(n)}%                 % oder: Beilage(n)
  \def\ccname{Verteiler}%                   % oder: Kopien an
  \def\headtoname{An}%
  \def\pagename{Seite}%
  \def\seename{siehe}%
  \def\alsoname{siehe auch}}
\let\captionsaustrian\captionsgerman
\def\month@german{\ifcase\month\or
  Januar\or Februar\or M\"arz\or April\or Mai\or Juni\or
  Juli\or August\or September\or Oktober\or November\or Dezember\fi}
\def\dategerman{\def\today{\number\day.~\month@german
  \space\number\year}}
\def\dateaustrian{\def\today{\number\day.~\ifnum1=\month
  J\"anner\else \month@german\fi \space\number\year}}
\addto\extrasgerman{\babel@add@special\"}
\addto\noextrasgerman{\babel@remove@special\"}
\addto\extrasgerman{\babel@savevariable{\catcode`\"}\babel@save\active@dq
  \catcode`\"\active \let\active@dq\german@active@dq}
\begingroup \catcode`\"=\active
\def\x{\endgroup
  \addto\extrasgerman{\babel@save"\let"\german@@active@dq}}
\x
\addto\extrasgerman{\babel@savevariable{\lccode`\^^Y}%
  \lccode`\^^Y`\^^Y}
\addto\extrasgerman{\babel@save\3\let\3\ss}
\addto\extrasgerman{\babel@save\"\umlautlow}
\addto\extrasgerman{\babel@savevariable\lefthyphenmin
  \babel@savevariable\righthyphenmin
  \lefthyphenmin\tw@\righthyphenmin\tw@}
\let\extrasaustrian\extrasgerman
\let\noextrasaustrian\noextrasgerman
\def\umlauthigh{\let\"\dieresis}
\def\umlautlow{\def\"{\protect\newumlaut}}
\def\highumlaut#1{\leavevmode\allowhyphens
  \dieresis{#1}\allowhyphens}
\expandafter\ifx\csname U@D\endcsname\relax
  \csname newdimen\endcsname\U@D
\fi
\def\newumlaut#1{\leavevmode\allowhyphens
  {\U@D 1ex%
  {\setbox\z@\hbox{\char127}\dimen@ -.45ex\advance\dimen@\ht\z@
  \ifdim 1ex<\dimen@ \fontdimen5\font\dimen@ \fi}%
  \accent127\fontdimen5\font\U@D #1}\allowhyphens}
\def\glqq{\protect\@glqq}
\def\@glqq{\save@sf@q{\set@low@box{''}\box\z@\kern-.04em\allowhyphens}}
\def\grqq{\protect\@grqq}
\def\@grqq{\save@sf@q{\kern-.07em``\kern.07em}}
\def\glq{\protect\@glq}
\def\@glq{\save@sf@q{\set@low@box{'}\box\z@\kern-.04em\allowhyphens}}
\def\grq{\protect\@grq}
\def\@grq{\save@sf@q{\kern-.07em`\kern.07em}}
\def\flqq{\protect\@flqq}
\def\@flqq{\relax \ifmmode \ll \else
  \save@sf@q{\raise .2ex\hbox{$\scriptscriptstyle \ll $}}\fi}
\def\frqq{\protect\@frqq}
\def\@frqq{\relax \ifmmode \gg \else
  \save@sf@q{\raise .2ex\hbox{$\scriptscriptstyle \gg $}}\fi}
\def\flq{\protect\@flq}
\def\@flq{\relax \ifmmode <\else
  \save@sf@q{\raise .2ex\hbox{$\scriptscriptstyle <$}}\fi}
\def\frq{\protect\@frq}
\def\@frq{\relax \ifmmode >\else
  \save@sf@q{\raise .2ex\hbox{$\scriptscriptstyle >$}}\fi}
\begingroup \catcode`\"12
\def\x{\endgroup
  \def\@UMLAUT{\"}
  \def\@MATHUMLAUT{\mathaccent"707F }
  \def\@SS{\mathchar"7019 }
  \def\dq{"}}
\x
\begingroup \catcode`\"=\active
\def\x{\endgroup
  \ifx"\undefined \let"\dq \fi
  \let\active@dq\dq}
\x
\let\dieresis=\"
\def\german@dqmacro#1{\csname g@dq@\string #1@dq@\endcsname}
\def\german@@active@dq#1{\expandafter\expandafter\expandafter
  \ifx\german@dqmacro{#1}\relax \expandafter\normal@dq
  \else \expandafter\german@@@active@dq \fi {#1}}
\def\normal@dq#1{\dq #1}
\def\german@@@active@dq#1{%
  \ifx\protect\relax \else \expandafter\protect \fi
  \active@dq{#1\empty}}
\def\german@active@dq#1{%
  \csname german@dq@\ifmmode second\else first\fi
    \expandafter\expandafter\expandafter\expandafter
    \expandafter\expandafter\expandafter
  \endcsname
  \german@dqmacro{#1}}
\def\german@dq@first#1#2{#1}
\def\german@dq@second#1#2{#2}
\def\def@german@dqmacro#1#2#3{\expandafter\expandafter\expandafter
  \def\german@dqmacro{#1}{{#2}{#3}}}
\def\let@german@dqmacro#1#2{\begingroup
  \edef\x{\endgroup \let
    \expandafter\expandafter\expandafter\noexpand\german@dqmacro{#1}%
    \expandafter\expandafter\expandafter\noexpand\german@dqmacro{#2}}%
  \x}
\def\german@dq@disc#1#2{%
  \allowhyphens\discretionary{#2-}{}{#1}\allowhyphens}
\def@german@dqmacro{a}{\@UMLAUT a}{\@MATHUMLAUT a}
\def@german@dqmacro{o}{\@UMLAUT o}{\@MATHUMLAUT o}
\def@german@dqmacro{u}{\@UMLAUT u}{\@MATHUMLAUT u}
\def@german@dqmacro{A}{\@UMLAUT A}{\@MATHUMLAUT A}
\def@german@dqmacro{O}{\@UMLAUT O}{\@MATHUMLAUT O}
\def@german@dqmacro{U}{\@UMLAUT U}{\@MATHUMLAUT U}
\def@german@dqmacro{e}{\highumlaut e}{\@MATHUMLAUT a}
\def@german@dqmacro{E}{\highumlaut E}{\@MATHUMLAUT E}
\def@german@dqmacro{i}{\highumlaut\i{}}{\@MATHUMLAUT\imath}
\def@german@dqmacro{I}{\highumlaut I}{\@MATHUMLAUT I}
\def@german@dqmacro{s}{\ss{}}{\@SS{}}
\def@german@dqmacro{S}{SS}{SS}
\def@german@dqmacro{`}{\glqq{}}{\glqq{}}
\def@german@dqmacro{'}{\grqq{}}{\grqq{}}
\def@german@dqmacro{<}{\flqq{}}{\flqq{}}
\def@german@dqmacro{>}{\frqq{}}{\frqq{}}
\def@german@dqmacro{c}{\german@dq@disc ck}{c}
\def@german@dqmacro{C}{\german@dq@disc CK}{C}
\def@german@dqmacro{f}{\german@dq@disc f{ff}}{f}
\def@german@dqmacro{F}{\german@dq@disc F{FF}}{F}
\def@german@dqmacro{l}{\german@dq@disc l{ll}}{l}
\def@german@dqmacro{L}{\german@dq@disc L{LL}}{L}
\def@german@dqmacro{m}{\german@dq@disc m{mm}}{m}
\def@german@dqmacro{M}{\german@dq@disc M{MM}}{M}
\def@german@dqmacro{n}{\german@dq@disc n{nn}}{n}
\def@german@dqmacro{N}{\german@dq@disc N{NN}}{N}
\def@german@dqmacro{p}{\german@dq@disc p{pp}}{p}
\def@german@dqmacro{P}{\german@dq@disc P{PP}}{P}
\def@german@dqmacro{t}{\german@dq@disc t{tt}}{t}
\def@german@dqmacro{T}{\german@dq@disc T{TT}}{T}
\def@german@dqmacro{-}{\allowhyphens\-\allowhyphens}%
               {\allowhyphens\-\allowhyphens}
\def@german@dqmacro{|}{\discretionary{-}{}{\kern.03em}}{}
\def@german@dqmacro{"}{\hskip\z@skip}{\hskip\z@skip}
\def@german@dqmacro{\empty}{{}}{{}}
\let@german@dqmacro{\active@dq{}}{"}
\def\mdqon{\catcode`\"\active}
\def\mdqoff{\catcode`\"12 }
\def\ck{\allowhyphens\discretionary{k-}{k}{ck}\allowhyphens}
\selectlanguage{german}
\catcode`\@=\atcatcode
%%
%% \CharacterTable
%%  {Upper-case    \A\B\C\D\E\F\G\H\I\J\K\L\M\N\O\P\Q\R\S\T\U\V\W\X\Y\Z
%%   Lower-case    \a\b\c\d\e\f\g\h\i\j\k\l\m\n\o\p\q\r\s\t\u\v\w\x\y\z
%%   Digits        \0\1\2\3\4\5\6\7\8\9
%%   Exclamation   \!     Double quote  \"     Hash (number) \#
%%   Dollar        \$     Percent       \%     Ampersand     \&
%%   Acute accent  \'     Left paren    \(     Right paren   \)
%%   Asterisk      \*     Plus          \+     Comma         \,
%%   Minus         \-     Point         \.     Solidus       \/
%%   Colon         \:     Semicolon     \;     Less than     \<
%%   Equals        \=     Greater than  \>     Question mark \?
%%   Commercial at \@     Left bracket  \[     Backslash     \\
%%   Right bracket \]     Circumflex    \^     Underscore    \_
%%   Grave accent  \`     Left brace    \{     Vertical bar  \|
%%   Right brace   \}     Tilde         \~}
%%
\endinput
%% 
%% End of file `germanb.sty'.
}
\DeclareOption{germanb}{%% 
%% This is file `germanb.sty', generated 
%% on <1995/2/23> with the docstrip utility (2.2i).
%% 
%% The original source files were:
%% 
%% germanb.dtx 
%% 
%% Copyright (C) 1989-1994 by Johannes Braams 
%% All rights reserved. 
%% For additional copyright information see further down in this file. 
%% 
%% This file is part of the Babel system, release 3.4 patchlevel 2. 
%% ---------------------------------------------------------------- 
%% 
%% This file is distributed in the hope that it will be useful, 
%% but WITHOUT ANY WARRANTY; without even the implied warranty of 
%% MERCHANTABILITY or FITNESS FOR A PARTICULAR PURPOSE. 
%% 
%% 
%% IMPORTANT NOTICE: 
%% 
%% For error reports in case of UNCHANGED versions see readme file. 
%% 
%% Please do not request updates from me directly. Distribution is 
%% done through Mail-Servers and TeX organizations. 
%% 
%% You are not allowed to change this file. 
%% 
%% You are allowed to distribute this file under the condition that 
%% it is distributed together with all files mentioned in manifest.txt. 
%% 
%% If you receive only some of these files from someone, complain! 
%% 
%% You are NOT ALLOWED to distribute this file alone. You are NOT 
%% ALLOWED to take money for the distribution or use of either this 
%% file or a changed version, except for a nominal charge for copying 
%% etc. 
%%% \iffalse ============================================================
%%%  @LaTeX-style-file{
%%%     author-1        = "Bernd Raichle",
%%%     author-2        = "Braams J.L.",
%%%     version         = "2.5c",
%%%     date            = "26 June 1994",
%%%     time            = "00:38:56 MET",
%%%     filename        = "germanb.dtx",
%%%     address         = "PTT Research
%%%                        St. Paulusstraat 4
%%%                        2264 XZ Leidschendam
%%%                        The Netherlands",
%%%     telephone-2     = "(70) 3325051",
%%%     FAX-2           = "(70) 3326477",
%%%     checksum        = "29603 839 3842 33542",
%%%     email-1         = "raichle@azu.Informatik.Uni-Stuttgart.de",
%%%     email-2         = "J.L.Braams@research.ptt.nl (Internet)",
%%%     codetable       = "ISO/ASCII",
%%%     keywords        = "babel, german",
%%%     supported       = "yes",
%%%     abstract        = "",
%%%     docstring       = "This file contains the german language specific
%%%                        definitions for the babel system.
%%%                        It is based on german.tex 2.3e.",
%%%  }
%%%
%%%  ====================================================================
%%% \fi
\ifx\undefined\captionsgerman
\else
  \selectlanguage{german}
  \expandafter\endinput
\fi
\chardef\atcatcode=\catcode`\@
\catcode`\@=11\relax
\ifx\undefined\babel@core@loaded\input babel.def\relax\fi
\ProvidesFile{germanb.sty}[1994/06/26 v2.5c
         German support from the babel system]
\ifx\undefined\originalTeX \let\originalTeX\empty\fi
\originalTeX
\ifx\undefined\l@german
  \@nopatterns{German}
  \adddialect\l@german0
\fi
\adddialect\l@austrian\l@german
\addto\captionsgerman{%
  \def\prefacename{Vorwort}%
  \def\refname{Literatur}%
  \def\abstractname{Zusammenfassung}%
  \def\bibname{Literaturverzeichnis}%
  \def\chaptername{Kapitel}%
  \def\appendixname{Anhang}%
  \def\contentsname{Inhaltsverzeichnis}%    % oder nur: Inhalt
  \def\listfigurename{Abbildungsverzeichnis}%
  \def\listtablename{Tabellenverzeichnis}%
  \def\indexname{Index}%
  \def\figurename{Abbildung}%
  \def\tablename{Tabelle}%                  % oder: Tafel
  \def\partname{Teil}%
  \def\enclname{Anlage(n)}%                 % oder: Beilage(n)
  \def\ccname{Verteiler}%                   % oder: Kopien an
  \def\headtoname{An}%
  \def\pagename{Seite}%
  \def\seename{siehe}%
  \def\alsoname{siehe auch}}
\let\captionsaustrian\captionsgerman
\def\month@german{\ifcase\month\or
  Januar\or Februar\or M\"arz\or April\or Mai\or Juni\or
  Juli\or August\or September\or Oktober\or November\or Dezember\fi}
\def\dategerman{\def\today{\number\day.~\month@german
  \space\number\year}}
\def\dateaustrian{\def\today{\number\day.~\ifnum1=\month
  J\"anner\else \month@german\fi \space\number\year}}
\addto\extrasgerman{\babel@add@special\"}
\addto\noextrasgerman{\babel@remove@special\"}
\addto\extrasgerman{\babel@savevariable{\catcode`\"}\babel@save\active@dq
  \catcode`\"\active \let\active@dq\german@active@dq}
\begingroup \catcode`\"=\active
\def\x{\endgroup
  \addto\extrasgerman{\babel@save"\let"\german@@active@dq}}
\x
\addto\extrasgerman{\babel@savevariable{\lccode`\^^Y}%
  \lccode`\^^Y`\^^Y}
\addto\extrasgerman{\babel@save\3\let\3\ss}
\addto\extrasgerman{\babel@save\"\umlautlow}
\addto\extrasgerman{\babel@savevariable\lefthyphenmin
  \babel@savevariable\righthyphenmin
  \lefthyphenmin\tw@\righthyphenmin\tw@}
\let\extrasaustrian\extrasgerman
\let\noextrasaustrian\noextrasgerman
\def\umlauthigh{\let\"\dieresis}
\def\umlautlow{\def\"{\protect\newumlaut}}
\def\highumlaut#1{\leavevmode\allowhyphens
  \dieresis{#1}\allowhyphens}
\expandafter\ifx\csname U@D\endcsname\relax
  \csname newdimen\endcsname\U@D
\fi
\def\newumlaut#1{\leavevmode\allowhyphens
  {\U@D 1ex%
  {\setbox\z@\hbox{\char127}\dimen@ -.45ex\advance\dimen@\ht\z@
  \ifdim 1ex<\dimen@ \fontdimen5\font\dimen@ \fi}%
  \accent127\fontdimen5\font\U@D #1}\allowhyphens}
\def\glqq{\protect\@glqq}
\def\@glqq{\save@sf@q{\set@low@box{''}\box\z@\kern-.04em\allowhyphens}}
\def\grqq{\protect\@grqq}
\def\@grqq{\save@sf@q{\kern-.07em``\kern.07em}}
\def\glq{\protect\@glq}
\def\@glq{\save@sf@q{\set@low@box{'}\box\z@\kern-.04em\allowhyphens}}
\def\grq{\protect\@grq}
\def\@grq{\save@sf@q{\kern-.07em`\kern.07em}}
\def\flqq{\protect\@flqq}
\def\@flqq{\relax \ifmmode \ll \else
  \save@sf@q{\raise .2ex\hbox{$\scriptscriptstyle \ll $}}\fi}
\def\frqq{\protect\@frqq}
\def\@frqq{\relax \ifmmode \gg \else
  \save@sf@q{\raise .2ex\hbox{$\scriptscriptstyle \gg $}}\fi}
\def\flq{\protect\@flq}
\def\@flq{\relax \ifmmode <\else
  \save@sf@q{\raise .2ex\hbox{$\scriptscriptstyle <$}}\fi}
\def\frq{\protect\@frq}
\def\@frq{\relax \ifmmode >\else
  \save@sf@q{\raise .2ex\hbox{$\scriptscriptstyle >$}}\fi}
\begingroup \catcode`\"12
\def\x{\endgroup
  \def\@UMLAUT{\"}
  \def\@MATHUMLAUT{\mathaccent"707F }
  \def\@SS{\mathchar"7019 }
  \def\dq{"}}
\x
\begingroup \catcode`\"=\active
\def\x{\endgroup
  \ifx"\undefined \let"\dq \fi
  \let\active@dq\dq}
\x
\let\dieresis=\"
\def\german@dqmacro#1{\csname g@dq@\string #1@dq@\endcsname}
\def\german@@active@dq#1{\expandafter\expandafter\expandafter
  \ifx\german@dqmacro{#1}\relax \expandafter\normal@dq
  \else \expandafter\german@@@active@dq \fi {#1}}
\def\normal@dq#1{\dq #1}
\def\german@@@active@dq#1{%
  \ifx\protect\relax \else \expandafter\protect \fi
  \active@dq{#1\empty}}
\def\german@active@dq#1{%
  \csname german@dq@\ifmmode second\else first\fi
    \expandafter\expandafter\expandafter\expandafter
    \expandafter\expandafter\expandafter
  \endcsname
  \german@dqmacro{#1}}
\def\german@dq@first#1#2{#1}
\def\german@dq@second#1#2{#2}
\def\def@german@dqmacro#1#2#3{\expandafter\expandafter\expandafter
  \def\german@dqmacro{#1}{{#2}{#3}}}
\def\let@german@dqmacro#1#2{\begingroup
  \edef\x{\endgroup \let
    \expandafter\expandafter\expandafter\noexpand\german@dqmacro{#1}%
    \expandafter\expandafter\expandafter\noexpand\german@dqmacro{#2}}%
  \x}
\def\german@dq@disc#1#2{%
  \allowhyphens\discretionary{#2-}{}{#1}\allowhyphens}
\def@german@dqmacro{a}{\@UMLAUT a}{\@MATHUMLAUT a}
\def@german@dqmacro{o}{\@UMLAUT o}{\@MATHUMLAUT o}
\def@german@dqmacro{u}{\@UMLAUT u}{\@MATHUMLAUT u}
\def@german@dqmacro{A}{\@UMLAUT A}{\@MATHUMLAUT A}
\def@german@dqmacro{O}{\@UMLAUT O}{\@MATHUMLAUT O}
\def@german@dqmacro{U}{\@UMLAUT U}{\@MATHUMLAUT U}
\def@german@dqmacro{e}{\highumlaut e}{\@MATHUMLAUT a}
\def@german@dqmacro{E}{\highumlaut E}{\@MATHUMLAUT E}
\def@german@dqmacro{i}{\highumlaut\i{}}{\@MATHUMLAUT\imath}
\def@german@dqmacro{I}{\highumlaut I}{\@MATHUMLAUT I}
\def@german@dqmacro{s}{\ss{}}{\@SS{}}
\def@german@dqmacro{S}{SS}{SS}
\def@german@dqmacro{`}{\glqq{}}{\glqq{}}
\def@german@dqmacro{'}{\grqq{}}{\grqq{}}
\def@german@dqmacro{<}{\flqq{}}{\flqq{}}
\def@german@dqmacro{>}{\frqq{}}{\frqq{}}
\def@german@dqmacro{c}{\german@dq@disc ck}{c}
\def@german@dqmacro{C}{\german@dq@disc CK}{C}
\def@german@dqmacro{f}{\german@dq@disc f{ff}}{f}
\def@german@dqmacro{F}{\german@dq@disc F{FF}}{F}
\def@german@dqmacro{l}{\german@dq@disc l{ll}}{l}
\def@german@dqmacro{L}{\german@dq@disc L{LL}}{L}
\def@german@dqmacro{m}{\german@dq@disc m{mm}}{m}
\def@german@dqmacro{M}{\german@dq@disc M{MM}}{M}
\def@german@dqmacro{n}{\german@dq@disc n{nn}}{n}
\def@german@dqmacro{N}{\german@dq@disc N{NN}}{N}
\def@german@dqmacro{p}{\german@dq@disc p{pp}}{p}
\def@german@dqmacro{P}{\german@dq@disc P{PP}}{P}
\def@german@dqmacro{t}{\german@dq@disc t{tt}}{t}
\def@german@dqmacro{T}{\german@dq@disc T{TT}}{T}
\def@german@dqmacro{-}{\allowhyphens\-\allowhyphens}%
               {\allowhyphens\-\allowhyphens}
\def@german@dqmacro{|}{\discretionary{-}{}{\kern.03em}}{}
\def@german@dqmacro{"}{\hskip\z@skip}{\hskip\z@skip}
\def@german@dqmacro{\empty}{{}}{{}}
\let@german@dqmacro{\active@dq{}}{"}
\def\mdqon{\catcode`\"\active}
\def\mdqoff{\catcode`\"12 }
\def\ck{\allowhyphens\discretionary{k-}{k}{ck}\allowhyphens}
\selectlanguage{german}
\catcode`\@=\atcatcode
%%
%% \CharacterTable
%%  {Upper-case    \A\B\C\D\E\F\G\H\I\J\K\L\M\N\O\P\Q\R\S\T\U\V\W\X\Y\Z
%%   Lower-case    \a\b\c\d\e\f\g\h\i\j\k\l\m\n\o\p\q\r\s\t\u\v\w\x\y\z
%%   Digits        \0\1\2\3\4\5\6\7\8\9
%%   Exclamation   \!     Double quote  \"     Hash (number) \#
%%   Dollar        \$     Percent       \%     Ampersand     \&
%%   Acute accent  \'     Left paren    \(     Right paren   \)
%%   Asterisk      \*     Plus          \+     Comma         \,
%%   Minus         \-     Point         \.     Solidus       \/
%%   Colon         \:     Semicolon     \;     Less than     \<
%%   Equals        \=     Greater than  \>     Question mark \?
%%   Commercial at \@     Left bracket  \[     Backslash     \\
%%   Right bracket \]     Circumflex    \^     Underscore    \_
%%   Grave accent  \`     Left brace    \{     Vertical bar  \|
%%   Right brace   \}     Tilde         \~}
%%
\endinput
%% 
%% End of file `germanb.sty'.
}
%    \end{macrocode}
%    \Lopt{hungarian} is just a synonym for \Lopt{magyar}
%    \begin{macrocode}
\DeclareOption{hungarian}{%% 
%% This is file `magyar.sty', generated 
%% on <1995/2/23> with the docstrip utility (2.2i).
%% 
%% The original source files were:
%% 
%% magyar.dtx 
%% 
%% Copyright (C) 1989-1994 by Johannes Braams 
%% All rights reserved. 
%% For additional copyright information see further down in this file. 
%% 
%% This file is part of the Babel system, release 3.4 patchlevel 2. 
%% ---------------------------------------------------------------- 
%% 
%% This file is distributed in the hope that it will be useful, 
%% but WITHOUT ANY WARRANTY; without even the implied warranty of 
%% MERCHANTABILITY or FITNESS FOR A PARTICULAR PURPOSE. 
%% 
%% 
%% IMPORTANT NOTICE: 
%% 
%% For error reports in case of UNCHANGED versions see readme file. 
%% 
%% Please do not request updates from me directly. Distribution is 
%% done through Mail-Servers and TeX organizations. 
%% 
%% You are not allowed to change this file. 
%% 
%% You are allowed to distribute this file under the condition that 
%% it is distributed together with all files mentioned in manifest.txt. 
%% 
%% If you receive only some of these files from someone, complain! 
%% 
%% You are NOT ALLOWED to distribute this file alone. You are NOT 
%% ALLOWED to take money for the distribution or use of either this 
%% file or a changed version, except for a nominal charge for copying 
%% etc. 
%%% \iffalse ============================================================
%%%  @LaTeX-style-file{
%%%     author_1        = "Braams, J.L.",
%%%     author_2        = "Biro, A. ",
%%%     version         = "1.3b",
%%%     date            = "26 June 1994",
%%%     time            = "01:41:33 MET",
%%%     filename        = "magyar.doc",
%%%     address_1       = "PTT Research
%%%                        St. Paulusstraat 4
%%%                        2264 XZ Leidschendam
%%%                        The Netherlands",
%%%     telephone_1     = "(70) 3325051",
%%%     FAX_1           = "(70) 3326477",
%%%     checksum        = "39098 318 1509 12682",
%%%     email_1         = "J.L.Braams@research.ptt.nl (Internet)",
%%%     email_2         = "",
%%%     codetable       = "ISO/ASCII",
%%%     keywords        = "babel, hungarian",
%%%     supported       = "yes",
%%%     abstract        = "",
%%%     docstring       = "This file contains the hungarian language specific
%%%                        definitions for the babel system.",
%%%  }
%%%
%%%  ====================================================================
%%% \fi
\ifx\undefined\captionsmagyar
\else
  \selectlanguage{magyar}
  \expandafter\endinput
\fi
\chardef\atcatcode=\catcode`\@
\catcode`\@=11\relax
\ifx\undefined\babel@core@loaded\input babel.def\relax\fi
\ProvidesFile{magyar.sty}[1994/06/26 v1.3c
         Magyar support from the babel system]
\ifx\undefined\originalTeX \let\originalTeX\empty \else\originalTeX\fi
\ifx\undefined\l@magyar
    \@nopatterns{Magyar}
    \adddialect\l@magyar0\fi
\addto\captionsmagyar{%
  \def\prefacename{El\H osz\'o}%
  \def\refname{Referenci\'ak}%
  \def\abstractname{Kivonat}%
  \def\bibname{Bibliogr\'afia}%
  \def\chaptername{fejezet}%
  \def\appendixname{f\"uggel\'ek}%
  \def\contentsname{Tartalom}%
  \def\listfigurename{\'Abr\'ak jegyz\'eke}%
  \def\listtablename{T\'abl\'azatok jegyz\'eke}%
  \def\indexname{T\'argymutat\'o}%
  \def\figurename{\'abra}%
  \def\tablename{t\'abl\'azat}%
  \def\partname{r\'esz}%
  \def\enclname{Mell\'eklet}%
  \def\ccname{K\"orlev\'el--c\'\i mzettek}%
  \def\headtoname{C\'\i mzett}%
  \def\pagename{oldal}%
  \def\seename{L\'asd}%
  \def\alsoname{L\'asd m\'eg}}%
\def\datemagyar{%
  \def\today{\number\year.~\ifcase\month\or
  janu\'ar\or febru\'ar\or m\'arcius\or
  \'aprilis\or m\'ajus\or j\'unius\or
  j\'ulius\or augusztus\or szeptember\or
  okt\'ober\or november\or december\fi
    \space\ifcase\day\or
    1.\or  2.\or  3.\or  4.\or  5.\or
    6.\or  7.\or  8.\or  9.\or 10.\or
   11.\or 12.\or 13.\or 14.\or 15.\or
   16.\or 17.\or 18.\or 19.\or 20.\or
   21.\or 22.\or 23.\or 24.\or 25.\or
   26.\or 27.\or 28.\or 29.\or 30.\or
   31.\fi}}
\def\ondatemagyar{%
  \number\year.~\ifcase\month\or
  janu\'ar\or febru\'ar\or m\'arcius\or
  \'aprilis\or m\'ajus\or j\'unius\or
  j\'ulius\or augusztus\or szeptember\or
  okt\'ober\or november\or december\fi
    \space\ifcase\day\or
    1-j\'en\or  2-\'an\or  3-\'an\or  4-\'en\or  5-\'en\or
    6-\'an\or  7-\'en\or  8-\'an\or  9-\'en\or 10-\'en\or
   11-\'en\or 12-\'en\or 13-\'an\or 14-\'en\or 15-\'en\or
   16-\'an\or 17-\'en\or 18-\'an\or 19-\'en\or 20-\'an\or
   21-\'en\or 22-\'en\or 23-\'an\or 24-\'en\or 25-\'en\or
   26-\'an\or 27-\'en\or 28-\'an\or 29-\'en\or 30-\'an\or
   31-\'en\fi}
\addto\extrasmagyar{\let\ontoday\ondatemagyar}
\addto\noextrasmagyar{\let\ontoday\undefined}
\selectlanguage{magyar}
\catcode`\@=\atcatcode \let\atcatcode\relax
%%
%% \CharacterTable
%%  {Upper-case    \A\B\C\D\E\F\G\H\I\J\K\L\M\N\O\P\Q\R\S\T\U\V\W\X\Y\Z
%%   Lower-case    \a\b\c\d\e\f\g\h\i\j\k\l\m\n\o\p\q\r\s\t\u\v\w\x\y\z
%%   Digits        \0\1\2\3\4\5\6\7\8\9
%%   Exclamation   \!     Double quote  \"     Hash (number) \#
%%   Dollar        \$     Percent       \%     Ampersand     \&
%%   Acute accent  \'     Left paren    \(     Right paren   \)
%%   Asterisk      \*     Plus          \+     Comma         \,
%%   Minus         \-     Point         \.     Solidus       \/
%%   Colon         \:     Semicolon     \;     Less than     \<
%%   Equals        \=     Greater than  \>     Question mark \?
%%   Commercial at \@     Left bracket  \[     Backslash     \\
%%   Right bracket \]     Circumflex    \^     Underscore    \_
%%   Grave accent  \`     Left brace    \{     Vertical bar  \|
%%   Right brace   \}     Tilde         \~}
%%
\endinput
%% 
%% End of file `magyar.sty'.
}
\DeclareOption{italian}{%% 
%% This is file `italian.sty', generated 
%% on <1995/2/23> with the docstrip utility (2.2i).
%% 
%% The original source files were:
%% 
%% italian.dtx 
%% 
%% Copyright (C) 1989-1994 by Johannes Braams 
%% All rights reserved. 
%% For additional copyright information see further down in this file. 
%% 
%% This file is part of the Babel system, release 3.4 patchlevel 2. 
%% ---------------------------------------------------------------- 
%% 
%% This file is distributed in the hope that it will be useful, 
%% but WITHOUT ANY WARRANTY; without even the implied warranty of 
%% MERCHANTABILITY or FITNESS FOR A PARTICULAR PURPOSE. 
%% 
%% 
%% IMPORTANT NOTICE: 
%% 
%% For error reports in case of UNCHANGED versions see readme file. 
%% 
%% Please do not request updates from me directly. Distribution is 
%% done through Mail-Servers and TeX organizations. 
%% 
%% You are not allowed to change this file. 
%% 
%% You are allowed to distribute this file under the condition that 
%% it is distributed together with all files mentioned in manifest.txt. 
%% 
%% If you receive only some of these files from someone, complain! 
%% 
%% You are NOT ALLOWED to distribute this file alone. You are NOT 
%% ALLOWED to take money for the distribution or use of either this 
%% file or a changed version, except for a nominal charge for copying 
%% etc. 
%%% \iffalse ============================================================
%%%  @LaTeX-style-file{
%%%     author          = "Braams J.L.",
%%%     version         = "1.2e",
%%%     date            = "26 June 1994",
%%%     time            = "00:49:53 MET",
%%%     filename        = "italian.dtx",
%%%     address         = "PTT Research
%%%                        St. Paulusstraat 4
%%%                        2264 XZ Leidschendam
%%%                        The Netherlands",
%%%     telephone       = "(70) 3325051",
%%%     FAX             = "(70) 3326477",
%%%     checksum        = "28615 312 1394 12215",
%%%     email           = "J.L.Braams@research.ptt.nl (Internet)",
%%%     codetable       = "ISO/ASCII",
%%%     keywords        = "babel, italian",
%%%     supported       = "yes",
%%%     abstract        = "",
%%%     docstring       = "This file contains the italian language specific
%%%                        definitions for the babel system.",
%%%  }
%%%
%%%  ====================================================================
%%% \fi
\ifx\undefined\captionsitalian
\else
  \selectlanguage{italian}
  \expandafter\endinput
\fi
\chardef\atcatcode=\catcode`\@
\catcode`\@=11\relax
\ifx\undefined\babel@core@loaded\input babel.def\relax\fi
\ProvidesFile{italian.sty}[1994/06/26 v1.2e
         Italian support from the babel system]
\ifx\undefined\originalTeX \let\originalTeX\empty \fi
\originalTeX
\ifx\undefined\l@italian
    \@nopatterns{Italian}
    \adddialect\l@italian0\fi
\addto\captionsitalian{%
  \def\prefacename{Prefazione}%
  \def\refname{Riferimenti bibliografici}%
  \def\abstractname{Sommario}%
  \def\bibname{Bibliografia}%
  \def\chaptername{Capitolo}%
  \def\appendixname{Appendice}%
  \def\contentsname{Indice}%
  \def\listfigurename{Elenco delle figure}%
  \def\listtablename{Elenco delle tabelle}%
  \def\indexname{Indice analitico}%
  \def\figurename{Figura}%
  \def\tablename{Tabella}%
  \def\partname{Parte}%
  \def\enclname{Allegati}%
  \def\ccname{e~p.~c.}%
  \def\headtoname{Per}%
  \def\pagename{Pag.}%    % in Italian abbreviation is preferred
  \def\seename{vedi}%
  \def\alsoname{vedi anche}}
\def\dateitalian{%
\def\today{\number\day~\ifcase\month\or
  gennaio\or febbraio\or marzo\or aprile\or maggio\or giugno\or
  luglio\or agosto\or settembre\or ottobre\or novembre\or dicembre\fi
  \space \number\year}}
\addto\extrasitalian{%
  \babel@savevariable\lefthyphenmin
  \babel@savevariable\righthyphenmin
  \lefthyphenmin\tw@\righthyphenmin\tw@}
\addto\noextrasitalian{}
\addto\extrasitalian{%
  \babel@savevariable\clubpenalty
  \babel@savevariable\widowpenalty
  \clubpenalty3000\widowpenalty3000}
\addto\extrasitalian{%
  \babel@savevariable\finalhyphendemerits
  \finalhyphendemerits50000000}
\selectlanguage{italian}
\catcode`\@=\atcatcode \let\atcatcode\relax
%%
%% \CharacterTable
%%  {Upper-case    \A\B\C\D\E\F\G\H\I\J\K\L\M\N\O\P\Q\R\S\T\U\V\W\X\Y\Z
%%   Lower-case    \a\b\c\d\e\f\g\h\i\j\k\l\m\n\o\p\q\r\s\t\u\v\w\x\y\z
%%   Digits        \0\1\2\3\4\5\6\7\8\9
%%   Exclamation   \!     Double quote  \"     Hash (number) \#
%%   Dollar        \$     Percent       \%     Ampersand     \&
%%   Acute accent  \'     Left paren    \(     Right paren   \)
%%   Asterisk      \*     Plus          \+     Comma         \,
%%   Minus         \-     Point         \.     Solidus       \/
%%   Colon         \:     Semicolon     \;     Less than     \<
%%   Equals        \=     Greater than  \>     Question mark \?
%%   Commercial at \@     Left bracket  \[     Backslash     \\
%%   Right bracket \]     Circumflex    \^     Underscore    \_
%%   Grave accent  \`     Left brace    \{     Vertical bar  \|
%%   Right brace   \}     Tilde         \~}
%%
\endinput
%% 
%% End of file `italian.sty'.
}
\DeclareOption{magyar}{%% 
%% This is file `magyar.sty', generated 
%% on <1995/2/23> with the docstrip utility (2.2i).
%% 
%% The original source files were:
%% 
%% magyar.dtx 
%% 
%% Copyright (C) 1989-1994 by Johannes Braams 
%% All rights reserved. 
%% For additional copyright information see further down in this file. 
%% 
%% This file is part of the Babel system, release 3.4 patchlevel 2. 
%% ---------------------------------------------------------------- 
%% 
%% This file is distributed in the hope that it will be useful, 
%% but WITHOUT ANY WARRANTY; without even the implied warranty of 
%% MERCHANTABILITY or FITNESS FOR A PARTICULAR PURPOSE. 
%% 
%% 
%% IMPORTANT NOTICE: 
%% 
%% For error reports in case of UNCHANGED versions see readme file. 
%% 
%% Please do not request updates from me directly. Distribution is 
%% done through Mail-Servers and TeX organizations. 
%% 
%% You are not allowed to change this file. 
%% 
%% You are allowed to distribute this file under the condition that 
%% it is distributed together with all files mentioned in manifest.txt. 
%% 
%% If you receive only some of these files from someone, complain! 
%% 
%% You are NOT ALLOWED to distribute this file alone. You are NOT 
%% ALLOWED to take money for the distribution or use of either this 
%% file or a changed version, except for a nominal charge for copying 
%% etc. 
%%% \iffalse ============================================================
%%%  @LaTeX-style-file{
%%%     author_1        = "Braams, J.L.",
%%%     author_2        = "Biro, A. ",
%%%     version         = "1.3b",
%%%     date            = "26 June 1994",
%%%     time            = "01:41:33 MET",
%%%     filename        = "magyar.doc",
%%%     address_1       = "PTT Research
%%%                        St. Paulusstraat 4
%%%                        2264 XZ Leidschendam
%%%                        The Netherlands",
%%%     telephone_1     = "(70) 3325051",
%%%     FAX_1           = "(70) 3326477",
%%%     checksum        = "39098 318 1509 12682",
%%%     email_1         = "J.L.Braams@research.ptt.nl (Internet)",
%%%     email_2         = "",
%%%     codetable       = "ISO/ASCII",
%%%     keywords        = "babel, hungarian",
%%%     supported       = "yes",
%%%     abstract        = "",
%%%     docstring       = "This file contains the hungarian language specific
%%%                        definitions for the babel system.",
%%%  }
%%%
%%%  ====================================================================
%%% \fi
\ifx\undefined\captionsmagyar
\else
  \selectlanguage{magyar}
  \expandafter\endinput
\fi
\chardef\atcatcode=\catcode`\@
\catcode`\@=11\relax
\ifx\undefined\babel@core@loaded\input babel.def\relax\fi
\ProvidesFile{magyar.sty}[1994/06/26 v1.3c
         Magyar support from the babel system]
\ifx\undefined\originalTeX \let\originalTeX\empty \else\originalTeX\fi
\ifx\undefined\l@magyar
    \@nopatterns{Magyar}
    \adddialect\l@magyar0\fi
\addto\captionsmagyar{%
  \def\prefacename{El\H osz\'o}%
  \def\refname{Referenci\'ak}%
  \def\abstractname{Kivonat}%
  \def\bibname{Bibliogr\'afia}%
  \def\chaptername{fejezet}%
  \def\appendixname{f\"uggel\'ek}%
  \def\contentsname{Tartalom}%
  \def\listfigurename{\'Abr\'ak jegyz\'eke}%
  \def\listtablename{T\'abl\'azatok jegyz\'eke}%
  \def\indexname{T\'argymutat\'o}%
  \def\figurename{\'abra}%
  \def\tablename{t\'abl\'azat}%
  \def\partname{r\'esz}%
  \def\enclname{Mell\'eklet}%
  \def\ccname{K\"orlev\'el--c\'\i mzettek}%
  \def\headtoname{C\'\i mzett}%
  \def\pagename{oldal}%
  \def\seename{L\'asd}%
  \def\alsoname{L\'asd m\'eg}}%
\def\datemagyar{%
  \def\today{\number\year.~\ifcase\month\or
  janu\'ar\or febru\'ar\or m\'arcius\or
  \'aprilis\or m\'ajus\or j\'unius\or
  j\'ulius\or augusztus\or szeptember\or
  okt\'ober\or november\or december\fi
    \space\ifcase\day\or
    1.\or  2.\or  3.\or  4.\or  5.\or
    6.\or  7.\or  8.\or  9.\or 10.\or
   11.\or 12.\or 13.\or 14.\or 15.\or
   16.\or 17.\or 18.\or 19.\or 20.\or
   21.\or 22.\or 23.\or 24.\or 25.\or
   26.\or 27.\or 28.\or 29.\or 30.\or
   31.\fi}}
\def\ondatemagyar{%
  \number\year.~\ifcase\month\or
  janu\'ar\or febru\'ar\or m\'arcius\or
  \'aprilis\or m\'ajus\or j\'unius\or
  j\'ulius\or augusztus\or szeptember\or
  okt\'ober\or november\or december\fi
    \space\ifcase\day\or
    1-j\'en\or  2-\'an\or  3-\'an\or  4-\'en\or  5-\'en\or
    6-\'an\or  7-\'en\or  8-\'an\or  9-\'en\or 10-\'en\or
   11-\'en\or 12-\'en\or 13-\'an\or 14-\'en\or 15-\'en\or
   16-\'an\or 17-\'en\or 18-\'an\or 19-\'en\or 20-\'an\or
   21-\'en\or 22-\'en\or 23-\'an\or 24-\'en\or 25-\'en\or
   26-\'an\or 27-\'en\or 28-\'an\or 29-\'en\or 30-\'an\or
   31-\'en\fi}
\addto\extrasmagyar{\let\ontoday\ondatemagyar}
\addto\noextrasmagyar{\let\ontoday\undefined}
\selectlanguage{magyar}
\catcode`\@=\atcatcode \let\atcatcode\relax
%%
%% \CharacterTable
%%  {Upper-case    \A\B\C\D\E\F\G\H\I\J\K\L\M\N\O\P\Q\R\S\T\U\V\W\X\Y\Z
%%   Lower-case    \a\b\c\d\e\f\g\h\i\j\k\l\m\n\o\p\q\r\s\t\u\v\w\x\y\z
%%   Digits        \0\1\2\3\4\5\6\7\8\9
%%   Exclamation   \!     Double quote  \"     Hash (number) \#
%%   Dollar        \$     Percent       \%     Ampersand     \&
%%   Acute accent  \'     Left paren    \(     Right paren   \)
%%   Asterisk      \*     Plus          \+     Comma         \,
%%   Minus         \-     Point         \.     Solidus       \/
%%   Colon         \:     Semicolon     \;     Less than     \<
%%   Equals        \=     Greater than  \>     Question mark \?
%%   Commercial at \@     Left bracket  \[     Backslash     \\
%%   Right bracket \]     Circumflex    \^     Underscore    \_
%%   Grave accent  \`     Left brace    \{     Vertical bar  \|
%%   Right brace   \}     Tilde         \~}
%%
\endinput
%% 
%% End of file `magyar.sty'.
}
\DeclareOption{norsk}{%% 
%% This is file `norsk.sty', generated 
%% on <1995/2/23> with the docstrip utility (2.2i).
%% 
%% The original source files were:
%% 
%% norsk.dtx 
%% 
%% Copyright (C) 1989-1994 by Johannes Braams 
%% All rights reserved. 
%% For additional copyright information see further down in this file. 
%% 
%% This file is part of the Babel system, release 3.4 patchlevel 2. 
%% ---------------------------------------------------------------- 
%% 
%% This file is distributed in the hope that it will be useful, 
%% but WITHOUT ANY WARRANTY; without even the implied warranty of 
%% MERCHANTABILITY or FITNESS FOR A PARTICULAR PURPOSE. 
%% 
%% 
%% IMPORTANT NOTICE: 
%% 
%% For error reports in case of UNCHANGED versions see readme file. 
%% 
%% Please do not request updates from me directly. Distribution is 
%% done through Mail-Servers and TeX organizations. 
%% 
%% You are not allowed to change this file. 
%% 
%% You are allowed to distribute this file under the condition that 
%% it is distributed together with all files mentioned in manifest.txt. 
%% 
%% If you receive only some of these files from someone, complain! 
%% 
%% You are NOT ALLOWED to distribute this file alone. You are NOT 
%% ALLOWED to take money for the distribution or use of either this 
%% file or a changed version, except for a nominal charge for copying 
%% etc. 
%%% \iffalse ============================================================
%%%  @LaTeX-style-file{
%%%     author          = "Braams J.L.",
%%%     version         = "1.2d",
%%%     date            = "26 June 1994",
%%%     time            = "01:27:55 MET",
%%%     filename        = "norsk.doc",
%%%     address         = "PTT Research
%%%                        St. Paulusstraat 4
%%%                        2264 XZ Leidschendam
%%%                        The Netherlands",
%%%     telephone       = "(70) 3325051",
%%%     FAX             = "(70) 3326477",
%%%     checksum        = "24272 330 1485 12555",
%%%     email           = "J.L.Braams@research.ptt.nl (Internet)",
%%%     codetable       = "ISO/ASCII",
%%%     keywords        = "babel, norwegian",
%%%     supported       = "yes",
%%%     abstract        = "",
%%%     docstring       = "This file contains the norwegian language specific
%%%                        definitions for the babel system.",
%%%  }
%%%
%%%  ====================================================================
%%% \fi
\ifx\undefined\captionsnorsk
\else
  \selectlanguage{norsk}
  \expandafter\endinput
\fi
\chardef\atcatcode=\catcode`\@
\catcode`\@=11\relax
\ifx\undefined\babel@core@loaded\input babel.def\relax\fi
\ProvidesFile{norsk.sty}[1994/06/26 v1.2d
         Norsk support from the babel system]
\ifx\undefined\originalTeX \let\originalTeX\empty \else\originalTeX\fi
\ifx\undefined\l@norsk
    \@nopatterns{Norsk}
    \adddialect\l@norsk0\fi
\adddialect\l@nynorsk\l@norsk
\addto\captionsnorsk{%
  \def\prefacename{Forord}%
  \def\refname{Referanser}%
  \def\abstractname{Sammendrag}%
  \def\bibname{Litteratur}%      or Litteraturoversikt
  \def\chaptername{Kapittel}%
  \def\appendixname{Tillegg}%    or Appendiks
  \def\contentsname{Innhold}%
  \def\listfigurename{Figurer}%  or Figurliste
  \def\listtablename{Tabeller}%  or Tabelliste
  \def\indexname{Register}%
  \def\figurename{Figur}%
  \def\tablename{Tabell}%
  \def\partname{Del}%
  \def\enclname{Vedlegg}%
  \def\ccname{Kopi sendt}%
  \def\headtoname{Til}% in letter
  \def\pagename{Side}%
  \def\seename{Se}%
  \def\alsoname{Se ogs\aa{}}%
  }
\addto\captionsnynorsk{%
  \def\prefacename{Forord}%
  \def\refname{Referansar}%
  \def\abstractname{Samandrag}%
  \def\bibname{Litteratur}%     or Litteraturoversyn
                         %    or Referansar
  \def\chaptername{Kapittel}%
  \def\appendixname{Tillegg}%   or Appendiks
  \def\contentsname{Innhald}%
  \def\listfigurename{Figurar}% or Figurliste
  \def\listtablename{Tabellar}% or Tabelliste
  \def\indexname{Register}%
  \def\figurename{Figur}%
  \def\tablename{Tabell}%
  \def\partname{Del}%
  \def\enclname{Vedlegg}%
  \def\ccname{Kopi sendt}%
  \def\headtoname{Til}% in letter
  \def\pagename{Side}%
  \def\seename{Sj\aa{}}%
  \def\alsoname{Sj\aa{} ogs\aa{}}%
  }
\def\datenorsk{%
\def\today{\number\day.~\ifcase\month\or
  januar\or februar\or mars\or april\or mai\or juni\or
  juli\or august\or september\or oktober\or november\or desember\fi
  \space\number\year}}
\let\datenynorsk\datenorsk
\addto\extrasnorsk{}
\addto\noextrasnorsk{}
\let\extrasnynorsk\extrasnorsk
\let\noextrasnynorsk\noextrasnorsk
\selectlanguage{norsk}
\catcode`\@=\atcatcode \let\atcatcode\relax
%%
%% \CharacterTable
%%  {Upper-case    \A\B\C\D\E\F\G\H\I\J\K\L\M\N\O\P\Q\R\S\T\U\V\W\X\Y\Z
%%   Lower-case    \a\b\c\d\e\f\g\h\i\j\k\l\m\n\o\p\q\r\s\t\u\v\w\x\y\z
%%   Digits        \0\1\2\3\4\5\6\7\8\9
%%   Exclamation   \!     Double quote  \"     Hash (number) \#
%%   Dollar        \$     Percent       \%     Ampersand     \&
%%   Acute accent  \'     Left paren    \(     Right paren   \)
%%   Asterisk      \*     Plus          \+     Comma         \,
%%   Minus         \-     Point         \.     Solidus       \/
%%   Colon         \:     Semicolon     \;     Less than     \<
%%   Equals        \=     Greater than  \>     Question mark \?
%%   Commercial at \@     Left bracket  \[     Backslash     \\
%%   Right bracket \]     Circumflex    \^     Underscore    \_
%%   Grave accent  \`     Left brace    \{     Vertical bar  \|
%%   Right brace   \}     Tilde         \~}
%%
\endinput
%% 
%% End of file `norsk.sty'.
}
%    \end{macrocode}
%    For Norwegian two spelling variants are provided.
%    \begin{macrocode}
\DeclareOption{nynorsk}{%
  %% 
%% This is file `norsk.sty', generated 
%% on <1995/2/23> with the docstrip utility (2.2i).
%% 
%% The original source files were:
%% 
%% norsk.dtx 
%% 
%% Copyright (C) 1989-1994 by Johannes Braams 
%% All rights reserved. 
%% For additional copyright information see further down in this file. 
%% 
%% This file is part of the Babel system, release 3.4 patchlevel 2. 
%% ---------------------------------------------------------------- 
%% 
%% This file is distributed in the hope that it will be useful, 
%% but WITHOUT ANY WARRANTY; without even the implied warranty of 
%% MERCHANTABILITY or FITNESS FOR A PARTICULAR PURPOSE. 
%% 
%% 
%% IMPORTANT NOTICE: 
%% 
%% For error reports in case of UNCHANGED versions see readme file. 
%% 
%% Please do not request updates from me directly. Distribution is 
%% done through Mail-Servers and TeX organizations. 
%% 
%% You are not allowed to change this file. 
%% 
%% You are allowed to distribute this file under the condition that 
%% it is distributed together with all files mentioned in manifest.txt. 
%% 
%% If you receive only some of these files from someone, complain! 
%% 
%% You are NOT ALLOWED to distribute this file alone. You are NOT 
%% ALLOWED to take money for the distribution or use of either this 
%% file or a changed version, except for a nominal charge for copying 
%% etc. 
%%% \iffalse ============================================================
%%%  @LaTeX-style-file{
%%%     author          = "Braams J.L.",
%%%     version         = "1.2d",
%%%     date            = "26 June 1994",
%%%     time            = "01:27:55 MET",
%%%     filename        = "norsk.doc",
%%%     address         = "PTT Research
%%%                        St. Paulusstraat 4
%%%                        2264 XZ Leidschendam
%%%                        The Netherlands",
%%%     telephone       = "(70) 3325051",
%%%     FAX             = "(70) 3326477",
%%%     checksum        = "24272 330 1485 12555",
%%%     email           = "J.L.Braams@research.ptt.nl (Internet)",
%%%     codetable       = "ISO/ASCII",
%%%     keywords        = "babel, norwegian",
%%%     supported       = "yes",
%%%     abstract        = "",
%%%     docstring       = "This file contains the norwegian language specific
%%%                        definitions for the babel system.",
%%%  }
%%%
%%%  ====================================================================
%%% \fi
\ifx\undefined\captionsnorsk
\else
  \selectlanguage{norsk}
  \expandafter\endinput
\fi
\chardef\atcatcode=\catcode`\@
\catcode`\@=11\relax
\ifx\undefined\babel@core@loaded\input babel.def\relax\fi
\ProvidesFile{norsk.sty}[1994/06/26 v1.2d
         Norsk support from the babel system]
\ifx\undefined\originalTeX \let\originalTeX\empty \else\originalTeX\fi
\ifx\undefined\l@norsk
    \@nopatterns{Norsk}
    \adddialect\l@norsk0\fi
\adddialect\l@nynorsk\l@norsk
\addto\captionsnorsk{%
  \def\prefacename{Forord}%
  \def\refname{Referanser}%
  \def\abstractname{Sammendrag}%
  \def\bibname{Litteratur}%      or Litteraturoversikt
  \def\chaptername{Kapittel}%
  \def\appendixname{Tillegg}%    or Appendiks
  \def\contentsname{Innhold}%
  \def\listfigurename{Figurer}%  or Figurliste
  \def\listtablename{Tabeller}%  or Tabelliste
  \def\indexname{Register}%
  \def\figurename{Figur}%
  \def\tablename{Tabell}%
  \def\partname{Del}%
  \def\enclname{Vedlegg}%
  \def\ccname{Kopi sendt}%
  \def\headtoname{Til}% in letter
  \def\pagename{Side}%
  \def\seename{Se}%
  \def\alsoname{Se ogs\aa{}}%
  }
\addto\captionsnynorsk{%
  \def\prefacename{Forord}%
  \def\refname{Referansar}%
  \def\abstractname{Samandrag}%
  \def\bibname{Litteratur}%     or Litteraturoversyn
                         %    or Referansar
  \def\chaptername{Kapittel}%
  \def\appendixname{Tillegg}%   or Appendiks
  \def\contentsname{Innhald}%
  \def\listfigurename{Figurar}% or Figurliste
  \def\listtablename{Tabellar}% or Tabelliste
  \def\indexname{Register}%
  \def\figurename{Figur}%
  \def\tablename{Tabell}%
  \def\partname{Del}%
  \def\enclname{Vedlegg}%
  \def\ccname{Kopi sendt}%
  \def\headtoname{Til}% in letter
  \def\pagename{Side}%
  \def\seename{Sj\aa{}}%
  \def\alsoname{Sj\aa{} ogs\aa{}}%
  }
\def\datenorsk{%
\def\today{\number\day.~\ifcase\month\or
  januar\or februar\or mars\or april\or mai\or juni\or
  juli\or august\or september\or oktober\or november\or desember\fi
  \space\number\year}}
\let\datenynorsk\datenorsk
\addto\extrasnorsk{}
\addto\noextrasnorsk{}
\let\extrasnynorsk\extrasnorsk
\let\noextrasnynorsk\noextrasnorsk
\selectlanguage{norsk}
\catcode`\@=\atcatcode \let\atcatcode\relax
%%
%% \CharacterTable
%%  {Upper-case    \A\B\C\D\E\F\G\H\I\J\K\L\M\N\O\P\Q\R\S\T\U\V\W\X\Y\Z
%%   Lower-case    \a\b\c\d\e\f\g\h\i\j\k\l\m\n\o\p\q\r\s\t\u\v\w\x\y\z
%%   Digits        \0\1\2\3\4\5\6\7\8\9
%%   Exclamation   \!     Double quote  \"     Hash (number) \#
%%   Dollar        \$     Percent       \%     Ampersand     \&
%%   Acute accent  \'     Left paren    \(     Right paren   \)
%%   Asterisk      \*     Plus          \+     Comma         \,
%%   Minus         \-     Point         \.     Solidus       \/
%%   Colon         \:     Semicolon     \;     Less than     \<
%%   Equals        \=     Greater than  \>     Question mark \?
%%   Commercial at \@     Left bracket  \[     Backslash     \\
%%   Right bracket \]     Circumflex    \^     Underscore    \_
%%   Grave accent  \`     Left brace    \{     Vertical bar  \|
%%   Right brace   \}     Tilde         \~}
%%
\endinput
%% 
%% End of file `norsk.sty'.
%
  \selectlanguage{nynorsk}}
\DeclareOption{polish}{%% 
%% This is file `polish.sty', generated 
%% on <1995/2/23> with the docstrip utility (2.2i).
%% 
%% The original source files were:
%% 
%% polish.dtx 
%% 
%% Copyright (C) 1989-1994 by Johannes Braams 
%% All rights reserved. 
%% For additional copyright information see further down in this file. 
%% 
%% This file is part of the Babel system, release 3.4 patchlevel 2. 
%% ---------------------------------------------------------------- 
%% 
%% This file is distributed in the hope that it will be useful, 
%% but WITHOUT ANY WARRANTY; without even the implied warranty of 
%% MERCHANTABILITY or FITNESS FOR A PARTICULAR PURPOSE. 
%% 
%% 
%% IMPORTANT NOTICE: 
%% 
%% For error reports in case of UNCHANGED versions see readme file. 
%% 
%% Please do not request updates from me directly. Distribution is 
%% done through Mail-Servers and TeX organizations. 
%% 
%% You are not allowed to change this file. 
%% 
%% You are allowed to distribute this file under the condition that 
%% it is distributed together with all files mentioned in manifest.txt. 
%% 
%% If you receive only some of these files from someone, complain! 
%% 
%% You are NOT ALLOWED to distribute this file alone. You are NOT 
%% ALLOWED to take money for the distribution or use of either this 
%% file or a changed version, except for a nominal charge for copying 
%% etc. 
%%% \iffalse ===========================================================
%%%  @LaTeX-style-file{
%%%     author_1        = "Elmar Schalueck",
%%%     author_2        = "Michael Janich",
%%%     author_3        = "Johannes L. Braams",
%%%     version         = "1.1c",
%%%     date            = "26 June 1994",
%%%     time            = "01:59:23 MET",
%%%     filename        = "polish.doc",
%%%     address_1/2     = "Universitaet-Gesamthochschule Paderborn
%%%                        Warburger Strasse 100
%%%                        4790 Paderborn
%%%                        Germany",
%%%     address_3       = "PTT Research
%%%                        St. Paulusstraat 4
%%%                        2264 XZ Leidschendam
%%%                        The Netherlands",
%%%     telephone       = "(70) 3325051",
%%%     FAX             = "(70) 3326477",
%%%     checksum        = "63394 715 2911 25329",
%%%     email_1         = "elmar@uni-paderborn.de",
%%%     email_2         = "massa@uni-paderborn.de",
%%%     email_3         = "J.L.Braams@research.ptt.nl (Internet)",
%%%     codetable       = "ISO/ASCII",
%%%     keywords        = "",
%%%     supported       = "yes",
%%%     abstract        = "",
%%%     docstring       = "This file contains the polish language specific
%%%                        definitions for the babel system.
%%%
%%%                        The checksum field above contains a CRC-16
%%%                        checksum as the first value, followed by the
%%%                        equivalent of the standard UNIX wc (word
%%%                        count) utility output of lines, words, and
%%%                        characters.  This is produced by Robert
%%%                        Solovay's checksum utility.",
%%%  }
%%%
%%%  ====================================================================
%%% \fi
\ifx\undefined\captionspolish
\else
  \selectlanguage{polish}
  \expandafter\endinput
\fi
\chardef\atcatcode=\catcode`\@
\catcode`\@=11\relax
\ifx\undefined\babel@core@loaded\input babel.def\relax\fi
\ProvidesFile{polish.sty}[1994/06/26 v1.1c
         Polish support from the babel system]
\ifx\undefined\originalTeX
  \let\originalTeX\empty
\fi
\originalTeX
\ifx\undefined\l@polish
  \@nopatterns{Polish}
  \adddialect\l@polish0\fi
\addto\captionspolish{%
  \def\prefacename{Przedmowa}%
  \def\refname{Bibliografia}%
  \def\abstractname{Streszczenie}%
  \def\bibname{Literatura}%
  \def\chaptername{Rozdzia\l}%
  \def\appendixname{Dodatek}%
  \def\contentsname{Spis rzeczy}%
  \def\listfigurename{Spis rysunk\'ow}%
  \def\listtablename{Spis tablic}%
  \def\indexname{Indeks}%
  \def\figurename{Rysunek}%
  \def\tablename{Tablica}%
  \def\partname{Cz\eob{}\'s\'c}%
  \def\enclname{Za\l\aob{}cznik}%
  \def\ccname{Kopie:}%
  \def\headtoname{Do}%
  \def\pagename{Strona}%
  \def\seename{Por\'ownaj}%
  \def\alsoname{Por\'ownaj tak\.ze}%
}
\def\datepolish{%
  \def\today{\number\day~\ifcase\month\or
  stycznia\or lutego\or marca\or kwietnia\or maja\or czerwca\or lipca\or
  sierpnia\or wrze\'snia\or pa\'zdziernika\or listopada\or grudnia\fi
  \space\number\year}
}
\addto\extraspolish{\babel@add@special\"}
\addto\noextraspolish{\babel@remove@special\"}
\addto\extraspolish{\babel@savevariable{\catcode`\"}\babel@save\active@dq
  \catcode`\"\active \let\active@dq\polish@active@dq}
\begingroup \catcode`\"=\active
\def\x{\endgroup
  \addto\extraspolish{\babel@save"\let"\polish@@active@dq}}
\x
\def\glqq{\protect\@glqq}
\def\@glqq{\save@sf@q{\set@low@box{''}\box\z@\kern-.04em\allowhyphens}}
\def\grqq{\protect\@grqq}
\def\@grqq{\save@sf@q{\kern-.07em``\kern.07em}}
\def\glq{\protect\@glq}
\def\@glq{\save@sf@q{\set@low@box{'}\box\z@\kern-.04em\allowhyphens}}
\def\grq{\protect\@grq}
\def\@grq{\save@sf@q{\kern-.07em`\kern.07em}}
\def\flqq{\protect\@flqq}
\def\@flqq{\relax \ifmmode \ll \else
  \save@sf@q{\raise .2ex\hbox{$\scriptscriptstyle \ll $}}\fi}
\def\frqq{\protect\@frqq}
\def\@frqq{\relax \ifmmode \gg \else
  \save@sf@q{\raise .2ex\hbox{$\scriptscriptstyle \gg $}}\fi}
\def\flq{\protect\@flq}
\def\@flq{\relax \ifmmode <\else
  \save@sf@q{\raise .2ex\hbox{$\scriptscriptstyle <$}}\fi}
\def\frq{\protect\@frq}
\def\@frq{\relax \ifmmode >\else
  \save@sf@q{\raise .2ex\hbox{$\scriptscriptstyle >$}}\fi}
\newdimen\pl@left
\newdimen\pl@down
\newdimen\pl@right
\newdimen\pl@temp
\def\sob#1#2#3#4#5{%parameters: letter and fractions hl,ho,vl,vo
  \setbox0\hbox{#1}\setbox1\hbox{$_\mathchar'454$}\setbox2\hbox{p}%
  \pl@right=#2\wd0 \advance\pl@right by-#3\wd1
  \pl@down=#5\ht1 \advance\pl@down by-#4\ht0
  \pl@left=\pl@right \advance\pl@left by\wd1
  \pl@temp=-\pl@down \advance\pl@temp by\dp2 \dp1=\pl@temp
  \kern\pl@right\lower\pl@down\box1\kern-\pl@left #1}
\def\aob{\sob a{.66}{.20}{0}{.90}}
\def\Aob{\sob A{.80}{.50}{0}{.90}}
\def\eob{\sob e{.50}{.35}{0}{.93}}
\def\Eob{\sob E{.60}{.35}{0}{.90}}
\def\spb#1#2#3#4#5{%
  \setbox0\hbox{#1}\setbox1\hbox{\char'023}%
  \pl@right=#2\wd0 \advance\pl@right by-#3\wd1
  \pl@down=#5\ht1 \advance\pl@down by-#4\ht0
  \pl@left=\pl@right \advance\pl@left by\wd1
  \ht1=\pl@down \dp1=-\pl@down
  \kern\pl@right\lower\pl@down\box1\kern-\pl@left #1}
\def\skb#1#2#3#4#5{%
  \setbox0\hbox{#1}\setbox1\hbox{\char'056}%
  \pl@right=#2\wd0 \advance\pl@right by-#3\wd1
  \pl@down=#5\ht1 \advance\pl@down by-#4\ht0
  \pl@left=\pl@right \advance\pl@left by\wd1
  \kern\pl@right\lower\pl@down\box1\kern-\pl@left #1}
\def\textpl{%
  \def\lpb{\plll}%
  \def\Lpb{\pLLL}%
  \def\zkb{\.z}%
  \def\Zkb{\.Z}}
\textpl
\let\lll=\l \let\LLL=\L
\def\plll{\lll}
\def\pLLL{\LLL}
\def\telepl{%
  \def\lpb{\spb l{.45}{.5}{.4}{.8}}%
  \def\Lpb{\spb L{.23}{.5}{.4}{.8}}%
  \def\zkb{\skb z{.5}{.5}{1.2}{0}}%
  \def\Zkb{\skb Z{.5}{.5}{1.1}{0}}}
\ifx\prm\undefined
  \let\@rm=\rm \def\rm{\@rm\textpl}
\else
  \def\rm{\protect\pprm}\def\pprm{\prm\textpl}
\fi
\ifx\pit\undefined
  \let\@it\it \def\it{\@it\textpl}
\else
  \def\it{\protect\ppit}\def\ppit{\pit\textpl}
\fi
\ifx\pbf\undefined
  \let\@bf\bf \def\bf{\@bf\textpl}
\else
  \def\bf{\protect\ppbf}\def\ppbf{\pbf\textpl}
\fi
\ifx\psl\undefined
  \let\@sl\sl \def\sl{\@sl\textpl}
\else
  \def\sl{\protect\ppsl}\gdef\ppsl{\psl\textpl}
\fi
\ifx\psf\undefined
\else
  \def\sf{\protect\ppsf}\def\ppsf{\psf\textpl}
\fi
\ifx\psc\undefined
\else
  \def\sc{\protect\ppsc}\gdef\ppsc{\psc\textpl}
\fi
\ifx\ptt\undefined
  \let\@tt\tt \def\tt{\@tt\telepl}
\else
  \def\tt{\protect\pptt}\def\pptt{\ptt\telepl}
\fi
\begingroup \catcode`\"12
\def\x{\endgroup
  \def\@UMLAUT{\"}
  \def\@MATHUMLAUT{\mathaccent"707F }
  \def\dq{"}}
\x
\begingroup \catcode`\"=\active
\def\x{\endgroup
  \ifx"\undefined \let"\dq \fi
  \let\active@dq\dq}
\x
\let\dieresis=\"
\def\polish@dqmacro#1{\csname p@dq@\string #1@dq@\endcsname}
\def\polish@@active@dq#1{\expandafter\expandafter\expandafter
  \ifx\polish@dqmacro{#1}\relax \expandafter\normal@dq
  \else \expandafter\polish@@@active@dq \fi {#1}}
\def\normal@dq#1{\dq #1}
\def\polish@@@active@dq#1{%
  \ifx\protect\relax \else \expandafter\protect \fi
  \active@dq{#1\empty}}
\def\polish@active@dq#1{%
  \csname polish@dq@\ifmmode second\else first\fi
    \expandafter\expandafter\expandafter\expandafter
    \expandafter\expandafter\expandafter
  \endcsname
  \polish@dqmacro{#1}}
\def\polish@dq@first#1#2{#1}
\def\polish@dq@second#1#2{#2}
\def\def@polish@dqmacro#1#2#3{\expandafter\expandafter\expandafter
  \def\polish@dqmacro{#1}{{#2}{#3}}}
\def\let@polish@dqmacro#1#2{\begingroup
  \edef\x{\endgroup \let
    \expandafter\expandafter\expandafter\noexpand\polish@dqmacro{#1}%
    \expandafter\expandafter\expandafter\noexpand\polish@dqmacro{#2}}%
  \x}
\def@polish@dqmacro{a}{\aob}{\@MATHUMLAUT a}
\def@polish@dqmacro{A}{\Aob}{\@MATHUMLAUT A}
\def@polish@dqmacro{c}{\'c}{\@MATHUMLAUT c}
\def@polish@dqmacro{C}{\'c}{\@MATHUMLAUT C}
\def@polish@dqmacro{e}{\eob}{\@MATHUMLAUT e}
\def@polish@dqmacro{E}{\Eob}{\@MATHUMLAUT E}
\def@polish@dqmacro{l}{\lpb}{\@MATHUMLAUT l}
\def@polish@dqmacro{L}{\Lpb}{\@MATHUMLAUT L}
\def@polish@dqmacro{n}{\'n}{\@MATHUMLAUT n}
\def@polish@dqmacro{N}{\'N}{\@MATHUMLAUT N}
\def@polish@dqmacro{o}{\'o}{\@MATHUMLAUT o}
\def@polish@dqmacro{O}{\'O}{\@MATHUMLAUT O}
\def@polish@dqmacro{r}{\zkb}{\@MATHUMLAUT r}
\def@polish@dqmacro{R}{\Zkb}{\@MATHUMLAUT R}
\def@polish@dqmacro{s}{\'s}{\@MATHUMLAUT s}
\def@polish@dqmacro{S}{\'S}{\@MATHUMLAUT S}
\def@polish@dqmacro{z}{\'z}{\@MATHUMLAUT z}
\def@polish@dqmacro{Z}{\'Z}{\@MATHUMLAUT Z}
\def@polish@dqmacro{`}{\glqq{}}{\glqq{}}
\def@polish@dqmacro{'}{\grqq{}}{\grqq{}}
\def@polish@dqmacro{<}{\flqq{}}{\flqq{}}
\def@polish@dqmacro{>}{\frqq{}}{\frqq{}}
\def@polish@dqmacro{-}{\allowhyphens\-\allowhyphens}%
               {\allowhyphens\-\allowhyphens}
\def@polish@dqmacro{|}{\discretionary{-}{}{\kern.03em}}{}
\def@polish@dqmacro{"}{\hskip\z@skip}{\hskip\z@skip}
\def@polish@dqmacro{\empty}{{}}{{}}
\let@polish@dqmacro{\active@dq{}}{"}
\def\mdqon{\catcode`\"\active}
\def\mdqoff{\catcode`\"12 }
\selectlanguage{polish}
\catcode`\@=\atcatcode \let\atcatcode\relax
%% \CharacterTable
%%  {Upper-case    \A\B\C\D\E\F\G\H\I\J\K\L\M\N\O\P\Q\R\S\T\U\V\W\X\Y\Z
%%   Lower-case    \a\b\c\d\e\f\g\h\i\j\k\l\m\n\o\p\q\r\s\t\u\v\w\x\y\z
%%   Digits        \0\1\2\3\4\5\6\7\8\9
%%   Exclamation   \!     Double quote  \"     Hash (number) \#
%%   Dollar        \$     Percent       \%     Ampersand     \&
%%   Acute accent  \'     Left paren    \(     Right paren   \)
%%   Asterisk      \*     Plus          \+     Comma         \,
%%   Minus         \-     Point         \.     Solidus       \/
%%   Colon         \:     Semicolon     \;     Less than     \<
%%   Equals        \=     Greater than  \>     Question mark \?
%%   Commercial at \@     Left bracket  \[     Backslash     \\
%%   Right bracket \]     Circumflex    \^     Underscore    \_
%%   Grave accent  \`     Left brace    \{     Vertical bar  \|
%%   Right brace   \}     Tilde         \~}
%%
\endinput
%% 
%% End of file `polish.sty'.
}
\DeclareOption{portuges}{%% 
%% This is file `portuges.sty', generated 
%% on <1995/2/23> with the docstrip utility (2.2i).
%% 
%% The original source files were:
%% 
%% portuges.dtx 
%% 
%% Copyright (C) 1989-1994 by Johannes Braams 
%% All rights reserved. 
%% For additional copyright information see further down in this file. 
%% 
%% This file is part of the Babel system, release 3.4 patchlevel 2. 
%% ---------------------------------------------------------------- 
%% 
%% This file is distributed in the hope that it will be useful, 
%% but WITHOUT ANY WARRANTY; without even the implied warranty of 
%% MERCHANTABILITY or FITNESS FOR A PARTICULAR PURPOSE. 
%% 
%% 
%% IMPORTANT NOTICE: 
%% 
%% For error reports in case of UNCHANGED versions see readme file. 
%% 
%% Please do not request updates from me directly. Distribution is 
%% done through Mail-Servers and TeX organizations. 
%% 
%% You are not allowed to change this file. 
%% 
%% You are allowed to distribute this file under the condition that 
%% it is distributed together with all files mentioned in manifest.txt. 
%% 
%% If you receive only some of these files from someone, complain! 
%% 
%% You are NOT ALLOWED to distribute this file alone. You are NOT 
%% ALLOWED to take money for the distribution or use of either this 
%% file or a changed version, except for a nominal charge for copying 
%% etc. 
%%% \iffalse ============================================================
%%%  @LaTeX-style-file{
%%%     author          = "Braams J.L.",
%%%     version         = "1.2d",
%%%     date            = "26 June 1994",
%%%     time            = "00:54:18 MET",
%%%     filename        = "portuges.doc",
%%%     address         = "PTT Research
%%%                        St. Paulusstraat 4
%%%                        2264 XZ Leidschendam
%%%                        The Netherlands",
%%%     telephone       = "(70) 3325051",
%%%     FAX             = "(70) 3326477",
%%%     checksum        = "22850 303 1380 11594",
%%%     email           = "J.L.Braams@research.ptt.nl (Internet)",
%%%     codetable       = "ISO/ASCII",
%%%     keywords        = "babel, portuguese",
%%%     supported       = "yes",
%%%     abstract        = "",
%%%     docstring       = "This file contains the portuguese language
%%%                        specific definitions for the babel system.",
%%%  }
%%%
%%%  ====================================================================
%%% \fi

\ifx\undefined\captionsportuges
\else
  \selectlanguage{portuges}
  \expandafter\endinput
\fi
\chardef\atcatcode=\catcode`\@
\catcode`\@=11\relax
\ifx\undefined\babel@core@loaded\input babel.def\relax\fi
\ProvidesFile{portuges.sty}[1994/06/26 v1.2d
         Portuguese support from the babel system]
\ifx\undefined\originalTeX \let\originalTeX\empty \else\originalTeX\fi
\ifx\undefined\l@portuges
    \@nopatterns{Portuges}
    \adddialect\l@portuges0\fi
\adddialect\l@brazil\l@portuges
\addto\captionsportuges{%
  \def\prefacename{Preface}%  this needs translation!
  \def\refname{Refer\^encias}%
  \def\abstractname{Resumo}%
  \def\bibname{Bibliografia}%
  \def\chaptername{Cap\'{\i}tulo}%
  \def\appendixname{Ap\^endice}%
  \def\contentsname{\'Indice}%
  \def\listfigurename{Lista de Figuras}%
  \def\listtablename{Lista de Tabelas}%
  \def\indexname{\'Indice Remissivo}%
  \def\figurename{Figura}%
  \def\tablename{Tabela}%
  \def\partname{Parte}%
  \def\enclname{Anexos}%
  \def\ccname{C\'opia a}%
  \def\headtoname{Para}%
  \def\pagename{P\'agina}%
  \def\seename{See}%  this needs translation!
  \def\alsoname{See also}}%  this needs translation!
\let\captionsbrazil\captionsportuges
\def\dateportuges{%
\def\today{\number\day\space de\space\ifcase\month\or
  Janeiro\or Fevereiro\or Mar\c{c}o\or Abril\or Maio\or Junho\or
  Julho\or Agosto\or Setembro\or Outubro\or Novembro\or Dezembro\fi
  \space de\space\number\year}}
\def\datebrazil{%
\def\today{\number\day\space de\space\ifcase\month\or
  janeiro\or fevereiro\or mar\c{c}o\or abril\or maio\or junho\or
  julho\or agosto\or setembro\or outubro\or novembro\or dezembro\fi
  \space de\space\number\year}}
\addto\extrasportuges{}
\addto\noextrasportuges{}
\let\extrasbrazil\extrasportuges
\let\noextrasbrazil\noextrasportuges
\selectlanguage{portuges}
\catcode`\@=\atcatcode \let\atcatcode\relax
%%
%% \CharacterTable
%%  {Upper-case    \A\B\C\D\E\F\G\H\I\J\K\L\M\N\O\P\Q\R\S\T\U\V\W\X\Y\Z
%%   Lower-case    \a\b\c\d\e\f\g\h\i\j\k\l\m\n\o\p\q\r\s\t\u\v\w\x\y\z
%%   Digits        \0\1\2\3\4\5\6\7\8\9
%%   Exclamation   \!     Double quote  \"     Hash (number) \#
%%   Dollar        \$     Percent       \%     Ampersand     \&
%%   Acute accent  \'     Left paren    \(     Right paren   \)
%%   Asterisk      \*     Plus          \+     Comma         \,
%%   Minus         \-     Point         \.     Solidus       \/
%%   Colon         \:     Semicolon     \;     Less than     \<
%%   Equals        \=     Greater than  \>     Question mark \?
%%   Commercial at \@     Left bracket  \[     Backslash     \\
%%   Right bracket \]     Circumflex    \^     Underscore    \_
%%   Grave accent  \`     Left brace    \{     Vertical bar  \|
%%   Right brace   \}     Tilde         \~}
%%
\endinput
%% 
%% End of file `portuges.sty'.
}
\DeclareOption{romanian}{%% 
%% This is file `romanian.sty', generated 
%% on <1995/2/23> with the docstrip utility (2.2i).
%% 
%% The original source files were:
%% 
%% romanian.dtx 
%% 
%% Copyright (C) 1989-1994 by Johannes Braams 
%% All rights reserved. 
%% For additional copyright information see further down in this file. 
%% 
%% This file is part of the Babel system, release 3.4 patchlevel 2. 
%% ---------------------------------------------------------------- 
%% 
%% This file is distributed in the hope that it will be useful, 
%% but WITHOUT ANY WARRANTY; without even the implied warranty of 
%% MERCHANTABILITY or FITNESS FOR A PARTICULAR PURPOSE. 
%% 
%% 
%% IMPORTANT NOTICE: 
%% 
%% For error reports in case of UNCHANGED versions see readme file. 
%% 
%% Please do not request updates from me directly. Distribution is 
%% done through Mail-Servers and TeX organizations. 
%% 
%% You are not allowed to change this file. 
%% 
%% You are allowed to distribute this file under the condition that 
%% it is distributed together with all files mentioned in manifest.txt. 
%% 
%% If you receive only some of these files from someone, complain! 
%% 
%% You are NOT ALLOWED to distribute this file alone. You are NOT 
%% ALLOWED to take money for the distribution or use of either this 
%% file or a changed version, except for a nominal charge for copying 
%% etc. 
%%% \iffalse ============================================================
%%%  @LaTeX-style-file{
%%%     author          = "Braams J.L.",
%%%     version         = "1.2d",
%%%     date            = "26 June 1994",
%%%     time            = "02:04:49 MET",
%%%     filename        = "romanian.doc",
%%%     address         = "PTT Research
%%%                        St. Paulusstraat 4
%%%                        2264 XZ Leidschendam
%%%                        The Netherlands",
%%%     telephone       = "(70) 3325051",
%%%     FAX             = "(70) 3326477",
%%%     checksum        = "03081 266 1236 10289",
%%%     email           = "J.L.Braams@research.ptt.nl (Internet)",
%%%     codetable       = "ISO/ASCII",
%%%     keywords        = "babel, romanian",
%%%     supported       = "yes",
%%%     abstract        = "",
%%%     docstring       = "This file contains the romanian language specific
%%%                        definitions for the babel system.",
%%%  }
%%%
%%%  ====================================================================
%%% \fi
\ifx\undefined\captionsromanian
\else
  \selectlanguage{romanian}
  \expandafter\endinput
\fi
\chardef\atcatcode=\catcode`\@
\catcode`\@=11\relax
\ifx\undefined\babel@core@loaded\input babel.def\relax\fi
\ProvidesFile{romanian.sty}[1994/06/26 v1.2d
         Romanian support from the babel system]
\ifx\undefined\originalTeX \let\originalTeX\empty \else\originalTeX\fi
\ifx\undefined\l@romanian
    \@nopatterns{Romanian}
    \adddialect\l@romanian0\fi
\addto\captionsromanian{%
  \def\prefacename{Prefa\c{t}\u{a}}%
  \def\refname{Bibliografie}%
  \def\abstractname{Rezumat}%
  \def\bibname{Bibliografie}%
  \def\chaptername{Capitolul}%
  \def\appendixname{Anexa}%
  \def\contentsname{Cuprins}%
  \def\listfigurename{List\u{a} de figuri}%
  \def\listtablename{List\u{a} de tabele}%
  \def\indexname{Glosar}%
  \def\figurename{Figura}%    % sau Plan\c{s}a
  \def\tablename{Tabela}%
  \def\partname{Partea}%
  \def\enclname{Anex\u{a}}%   % sau Anexe
  \def\ccname{Copie}%
  \def\headtoname{Pentru}%
  \def\pagename{Pagina}%
  \def\seename{Vezi}%
  \def\alsoname{Vezi de asemenea}}%
\def\dateromanian{%
\def\today{\number\day~\ifcase\month\or
  ianuarie\or februarie\or martie\or aprilie\or mai\or iunie\or
  iulie\or august\or septembrie\or octombrie\or noiembrie\or decembrie\fi
  \space \number\year}}
\addto\extrasromanian{}
\addto\noextrasromanian{}
\selectlanguage{romanian}
\catcode`\@=\atcatcode \let\atcatcode\relax
%%
%% \CharacterTable
%%  {Upper-case    \A\B\C\D\E\F\G\H\I\J\K\L\M\N\O\P\Q\R\S\T\U\V\W\X\Y\Z
%%   Lower-case    \a\b\c\d\e\f\g\h\i\j\k\l\m\n\o\p\q\r\s\t\u\v\w\x\y\z
%%   Digits        \0\1\2\3\4\5\6\7\8\9
%%   Exclamation   \!     Double quote  \"     Hash (number) \#
%%   Dollar        \$     Percent       \%     Ampersand     \&
%%   Acute accent  \'     Left paren    \(     Right paren   \)
%%   Asterisk      \*     Plus          \+     Comma         \,
%%   Minus         \-     Point         \.     Solidus       \/
%%   Colon         \:     Semicolon     \;     Less than     \<
%%   Equals        \=     Greater than  \>     Question mark \?
%%   Commercial at \@     Left bracket  \[     Backslash     \\
%%   Right bracket \]     Circumflex    \^     Underscore    \_
%%   Grave accent  \`     Left brace    \{     Vertical bar  \|
%%   Right brace   \}     Tilde         \~}
%%
\endinput
%% 
%% End of file `romanian.sty'.
}
%\DeclareOption{russian}{\input{russian.sty}}
\DeclareOption{spanish}{%% 
%% This is file `spanish.sty', generated 
%% on <1995/2/23> with the docstrip utility (2.2i).
%% 
%% The original source files were:
%% 
%% spanish.dtx 
%% 
%% Copyright (C) 1989-1994 by Johannes Braams 
%% All rights reserved. 
%% For additional copyright information see further down in this file. 
%% 
%% This file is part of the Babel system, release 3.4 patchlevel 2. 
%% ---------------------------------------------------------------- 
%% 
%% This file is distributed in the hope that it will be useful, 
%% but WITHOUT ANY WARRANTY; without even the implied warranty of 
%% MERCHANTABILITY or FITNESS FOR A PARTICULAR PURPOSE. 
%% 
%% 
%% IMPORTANT NOTICE: 
%% 
%% For error reports in case of UNCHANGED versions see readme file. 
%% 
%% Please do not request updates from me directly. Distribution is 
%% done through Mail-Servers and TeX organizations. 
%% 
%% You are not allowed to change this file. 
%% 
%% You are allowed to distribute this file under the condition that 
%% it is distributed together with all files mentioned in manifest.txt. 
%% 
%% If you receive only some of these files from someone, complain! 
%% 
%% You are NOT ALLOWED to distribute this file alone. You are NOT 
%% ALLOWED to take money for the distribution or use of either this 
%% file or a changed version, except for a nominal charge for copying 
%% etc. 
%%% \iffalse ============================================================
%%%  @LaTeX-style-file{
%%%     author-1        = "Julio Sanchez"
%%%     author-2        = "Braams J.L.",
%%%     version         = "3.3d",
%%%     date            = "26 June 1994",
%%%     time            = "00:58:23 MET",
%%%     filename        = "spanish.doc",
%%%     address-1       = "GMV, SA
%%%                        c/ Isaac Newton 11
%%%                        PTM - Tres Cantos
%%%                        E-28760 Madrid
%%%                        Spain",
%%%     address-2       = "PTT Research
%%%                        St. Paulusstraat 4
%%%                        2264 XZ Leidschendam
%%%                        The Netherlands",
%%%     telephone-1     = "+34 1 807 21 85",
%%%     telephone-2     = "(70) 3325051",
%%%     FAX-1           = "+34 1 807 21 99",
%%%     FAX-2           = "(70) 3326477",
%%%     checksum        = "37806 861 4587 37084",
%%%     email-1         = "jsanchez@gmv.es (Internet)",
%%%     email-2         = "J.L.Braams@research.ptt.nl (Internet)",
%%%     codetable       = "ISO/ASCII",
%%%     keywords        = "babel, spanish",
%%%     supported       = "yes",
%%%     abstract        = "",
%%%     docstring       = "This file contains the spanish language
%%%                        specific definitions for the babel system.",
%%%  }
%%%
%%%  ====================================================================
%%% \fi
\ifx\undefined\captionsspanish
\else
  \selectlanguage{spanish}
  \expandafter\endinput
\fi
\chardef\atcatcode=\catcode`\@
\catcode`\@=11\relax
\ifx\undefined\babel@core@loaded\input babel.def\relax\fi
\ProvidesFile{spanish.sty}[1994/06/26 v3.3d
         Spanish support from the babel system]
\ifx\undefined\originalTeX \let\originalTeX\empty \else\originalTeX\fi
\ifx\undefined\l@spanish
  \@nopatterns{Spanish}
  \adddialect\l@spanish0
\fi
\addto\captionsspanish{%
  \def\prefacename{Prefacio}%
  \def\refname{Referencias}%
  \def\abstractname{Resumen}%
  \def\bibname{Bibliograf\'{\i}a}%
  \def\chaptername{Cap\'{\i}tulo}%
  \def\appendixname{Ap\'endice}%
  \def\contentsname{\'Indice General}%
  \def\listfigurename{\'Indice de Figuras}%
  \def\listtablename{\'Indice de Tablas}%
  \def\indexname{\'Indice de Materias}%
  \def\figurename{Figura}%
  \def\tablename{Tabla}%
  \def\partname{Parte}%
  \def\enclname{Adjunto}%
  \def\ccname{Copia a}%
  \def\headtoname{A}%
  \def\pagename{P\'agina}%
  \def\seename{v\'ease}%
  \def\alsoname{v\'ease tambi\'en}}%
\def\datespanish{%
\def\today{\number\day~de\space\ifcase\month\or
  enero\or febrero\or marzo\or abril\or mayo\or junio\or
  julio\or agosto\or septiembre\or octubre\or noviembre\or diciembre\fi
  \space de~\number\year}}
\addto\extrasspanish{%
  \babel@add@special\"%
  \babel@add@special\'%
  \babel@add@special\~}
\addto\noextrasspanish{%
  \babel@remove@special\"%
  \babel@remove@special\'%
  \babel@remove@special\~}
\addto\extrasspanish{%
  \babel@savevariable{\catcode`\"}\babel@save\active@dq
  \babel@savevariable{\catcode`\'}\babel@save\active@ac
  \babel@savevariable{\catcode`\~}\babel@save\active@til
  \babel@save~}
\addto\extrasspanish{\catcode`\"\active\catcode`\'\active%
                     \catcode`\~\active}
\begingroup \catcode`\"\active \catcode`\~\active \catcode`\'\active
\def\x{\endgroup
  \addto\extrasspanish{%
    \def'{\spanish@@active{ac}}%
    \def\active@ac{\spanish@active{ac}}%
    \def~{\spanish@@active{til}}%
    \def\active@til{\spanish@active{til}}%
    \def"{\spanish@@active{dq}}%
    \def\active@dq{\spanish@active{dq}}}}
\x
\addto\extrasspanish{%
    \babel@savevariable\lefthyphenmin
    \babel@savevariable\righthyphenmin
    \lefthyphenmin\tw@
    \righthyphenmin\tw@}
\addto\extrasspanish{\babel@save\"\babel@save\'%
                     \babel@save\~\babel@save\a}
\addto\extrasspanish{\let\a\spanish@a}
\addto\extrasspanish{\let\"\@umlaut
                     \let\'\@acute
                     \let\~\@tilde}
\def\flqq{\protect\@flqq}
\def\@flqq{\relax \ifmmode \ll \else
  \save@sf@q{\raise .2ex\hbox{$\scriptscriptstyle \ll $}}\fi}
\def\frqq{\protect\@frqq}
\def\@frqq{\relax \ifmmode \gg \else
  \save@sf@q{\raise .2ex\hbox{$\scriptscriptstyle \gg $}}\fi}
\def\flq{\protect\@flq}
\def\@flq{\relax \ifmmode <\else
  \save@sf@q{\raise .2ex\hbox{$\scriptscriptstyle <$}}\fi}
\def\frq{\protect\@frq}
\def\@frq{\relax \ifmmode >\else
  \save@sf@q{\raise .2ex\hbox{$\scriptscriptstyle >$}}\fi}
\def\spanish@a#1{{\if\string#1'\aftergroup\@acute
            \else \expandafter\aftergroup\csname a#1\endcsname\fi}}
\begingroup \catcode`\"12 \catcode`\'12
\edef\x{\endgroup
  \def\noexpand\dq{"}
  \def\noexpand\til{~}
  \def\noexpand\ac{'}}
\x
\let\dieresis\"
\let\textacute\'
\let\texttilde\~
\ifx\undefined\DeclareFontShape
    \wlog{Warning: You are using an old LaTeX}
    \wlog{Some word breaks will not be found.}
    \def\@umlaut#1{\allowhyphens\dieresis{#1}\allowhyphens}
    \def\@acute#1{\allowhyphens\textacute{#1}\allowhyphens}
    \def\@tilde#1{\allowhyphens\texttilde{#1}\allowhyphens}
\else
    \edef\next{T1}
    \ifx\f@encoding\next
        \let\@umlaut\dieresis
        \let\@acute\textacute
        \let\@tilde\texttilde
    \else
        \wlog{Warning: You are using encoding \f@encoding\space instead of T1.}
        \wlog{Some word breaks will not be found.}
        \def\@umlaut#1{\allowhyphens\dieresis{#1}\allowhyphens}
        \def\@acute#1{\allowhyphens\textacute{#1}\allowhyphens}
        \def\@tilde#1{\allowhyphens\texttilde{#1}\allowhyphens}
    \fi
\fi
\def\spanish@shorthand#1#2{\csname \languagename @#1@\string #2@#1@\endcsname}
\def\spanish@@active#1#2{\spanish@@active@@{#1}{#2}}
\def\spanish@@active@@#1#2{\expandafter\expandafter\expandafter
     \ifx\spanish@shorthand{#1}{#2}\relax \expandafter\spanish@normal
     \else \expandafter\spanish@@@active \fi {#1}{#2}}
\def\spanish@normal#1#2{\csname #1\endcsname #2}
\def\spanish@@@active#1#2{%
   \ifx\protect\relax \else \expandafter\protect \fi
   \csname active@#1\endcsname{#2\empty}}
\def\spanish@active#1#2{%
  \csname spanish@choose@\ifmmode second\else first\fi
    \expandafter\expandafter\expandafter\expandafter
    \expandafter\expandafter\expandafter
  \endcsname
  \spanish@shorthand{#1}{#2}}
\def\spanish@choose@first#1#2{#1}
\def\spanish@choose@second#1#2{#2}
\def\languagename{spanish}
\def\def@spanish@shorthand#1#2#3#4{\expandafter\expandafter\expandafter
  \def\spanish@shorthand{#1}{#2}{{#3}{#4}}}
\def\let@spanish@shorthand#1#2#3{\begingroup
  \edef\x{\endgroup \let
    \expandafter\expandafter\expandafter\noexpand\spanish@shorthand{#1}{#2}%
    \expandafter\expandafter\expandafter\noexpand\spanish@shorthand{#1}{#3}}%
  \x}
\def\spanish@disc#1#2{\allowhyphens\discretionary{#2-}{}{#1}\allowhyphens}
\def@spanish@shorthand{dq}{u}{\@umlaut u}{\@umlaut u}
\def@spanish@shorthand{dq}{U}{\@umlaut U}{\@umlaut U}
\def@spanish@shorthand{dq}{<}{\flqq{}}{\flqq{}}
\def@spanish@shorthand{dq}{>}{\frqq{}}{\frqq{}}
\def@spanish@shorthand{dq}{o}{\raise1ex\hbox{\underbar{\scriptsize o}}}%
        {\raise1ex\hbox{\underbar{\scriptsize o}}}
\def@spanish@shorthand{dq}{a}{\raise1ex\hbox{\underbar{\scriptsize a}}}%
        {\raise1ex\hbox{\underbar{\scriptsize a}}}
\def@spanish@shorthand{ac}{a}{\@acute a}{\@acute a}
\def@spanish@shorthand{ac}{e}{\@acute e}{\@acute e}
\def@spanish@shorthand{ac}{i}{\@acute \i{}}{\@acute \imath}
\def@spanish@shorthand{ac}{o}{\@acute o}{\@acute o}
\def@spanish@shorthand{ac}{u}{\@acute u}{\@acute u}
\def@spanish@shorthand{ac}{A}{\@acute A}{\@acute A}
\def@spanish@shorthand{ac}{E}{\@acute E}{\@acute E}
\def@spanish@shorthand{ac}{I}{\@acute I}{\@acute I}
\def@spanish@shorthand{ac}{O}{\@acute O}{\@acute O}
\def@spanish@shorthand{ac}{U}{\@acute U}{\@acute U}
\def@spanish@shorthand{til}{n}{\@tilde n}{\@tilde n}
\def@spanish@shorthand{til}{N}{\@tilde N}{\@tilde N}
\def@spanish@shorthand{dq}{-}{\allowhyphens\-\allowhyphens}%
               {\allowhyphens\-\allowhyphens}
\def@spanish@shorthand{dq}{|}{\discretionary{-}{}{\kern.03em}}{}
\def@spanish@shorthand{dq}{"}{\hskip\z@skip}{\hskip\z@skip}
\def@spanish@shorthand{ac}{'}{\rq\rq}
        {^\bgroup\catcode`\'=12\prime\prime\futurelet\next\pr@m@s}
\def@spanish@shorthand{dq}{\empty}{{}}{{}}
\let@spanish@shorthand{dq}{\active@dq{}}{"}
\let@spanish@shorthand{dq}{\active@ac{}}{"}
\def@spanish@shorthand{ac}{\empty}{{}}{{}}
\let@spanish@shorthand{ac}{\active@ac{}}{'}
\def@spanish@shorthand{ac}{$}{\rq $}{^\bgroup\prim@s $}
\catcode`\{=12 \catcode`\}=12
\catcode`\[=1  \catcode`\]=2
\def@spanish@shorthand[ac][{][\rq \bgroup][^\bgroup\prim@s \bgroup]
\def@spanish@shorthand[ac][}][\rq \egroup][^\bgroup\prim@s \egroup]
\def@spanish@shorthand[til][{][\til \bgroup][\til \bgroup]
\def@spanish@shorthand[til][}][\til \egroup][\til \egroup]
\catcode`\{=1  \catcode`\}=2
\catcode`\[=12 \catcode`\]=12
\selectlanguage{spanish}
\catcode`\@=\atcatcode \let\atcatcode\relax
%%
%% \CharacterTable
%%  {Upper-case    \A\B\C\D\E\F\G\H\I\J\K\L\M\N\O\P\Q\R\S\T\U\V\W\X\Y\Z
%%   Lower-case    \a\b\c\d\e\f\g\h\i\j\k\l\m\n\o\p\q\r\s\t\u\v\w\x\y\z
%%   Digits        \0\1\2\3\4\5\6\7\8\9
%%   Exclamation   \!     Double quote  \"     Hash (number) \#
%%   Dollar        \$     Percent       \%     Ampersand     \&
%%   Acute accent  \'     Left paren    \(     Right paren   \)
%%   Asterisk      \*     Plus          \+     Comma         \,
%%   Minus         \-     Point         \.     Solidus       \/
%%   Colon         \:     Semicolon     \;     Less than     \<
%%   Equals        \=     Greater than  \>     Question mark \?
%%   Commercial at \@     Left bracket  \[     Backslash     \\
%%   Right bracket \]     Circumflex    \^     Underscore    \_
%%   Grave accent  \`     Left brace    \{     Vertical bar  \|
%%   Right brace   \}     Tilde         \~}
%%
\endinput
%% 
%% End of file `spanish.sty'.
}
\DeclareOption{slovak}{%% 
%% This is file `slovak.sty', generated 
%% on <1995/2/23> with the docstrip utility (2.2i).
%% 
%% The original source files were:
%% 
%% slovak.dtx 
%% 
%% Copyright (C) 1989-1994 by Johannes Braams 
%% All rights reserved. 
%% For additional copyright information see further down in this file. 
%% 
%% This file is part of the Babel system, release 3.4 patchlevel 2. 
%% ---------------------------------------------------------------- 
%% 
%% This file is distributed in the hope that it will be useful, 
%% but WITHOUT ANY WARRANTY; without even the implied warranty of 
%% MERCHANTABILITY or FITNESS FOR A PARTICULAR PURPOSE. 
%% 
%% 
%% IMPORTANT NOTICE: 
%% 
%% For error reports in case of UNCHANGED versions see readme file. 
%% 
%% Please do not request updates from me directly. Distribution is 
%% done through Mail-Servers and TeX organizations. 
%% 
%% You are not allowed to change this file. 
%% 
%% You are allowed to distribute this file under the condition that 
%% it is distributed together with all files mentioned in manifest.txt. 
%% 
%% If you receive only some of these files from someone, complain! 
%% 
%% You are NOT ALLOWED to distribute this file alone. You are NOT 
%% ALLOWED to take money for the distribution or use of either this 
%% file or a changed version, except for a nominal charge for copying 
%% etc. 
%%% \iffalse ============================================================
%%%  @LaTeX-style-file{
%%%     author-1        = "Chlebikova Jana",
%%%     author-2        = "Braams J.L.",
%%%     version         = "1.2d",
%%%     date            = "30 June 1994",
%%%     time            = "14:10:13 MET",
%%%     filename        = "slovak.doc",
%%%     address-1       = "Department of Artificial Intelligence
%%%                        Faculty of Mathematics and Physics
%%%                        Mlynska dolina
%%%                        84215 Bratislava
%%%                        Slovakia",
%%%     address-2       = "PTT Research
%%%                        St. Paulusstraat 4
%%%                        2264 XZ Leidschendam
%%%                        The Netherlands",
%%%     telephone-1     = "(42)(7) 720003 l. 835",
%%%     telephone-2     = "(70) 3325051",
%%%     FAX-1           = "(42)(7) 725882",
%%%     FAX-2           = "(70) 3326477",
%%%     checksum        = "51098 265 1214 10014",
%%%     email-1         = "chlebikj@mff.uniba.cs (Internet)",
%%%     email-2         = "J.L.Braams@research.ptt.nl (Internet)",
%%%     codetable       = "ISO/ASCII",
%%%     keywords        = "babel, slovak",
%%%     supported       = "yes",
%%%     abstract        = "",
%%%     docstring       = "This file contains the slovak language specific
%%%                        definitions for the babel system.",
%%%  }
%%%
%%%  ====================================================================
%%% \fi
\ifx\undefined\captionsslovak
\else
  \selectlanguage{slovak}
  \expandafter\endinput
\fi
\chardef\atcatcode=\catcode`\@
\catcode`\@=11\relax
\ifx\undefined\babel@core@loaded\input babel.def\relax\fi
\ProvidesFile{slovak.sty}[1994/06/30 v1.2d
         Slovak support from the babel system]
\ifx\undefined\originalTeX \let\originalTeX\empty \else\originalTeX\fi
\ifx\undefined\l@slovak
    \@nopatterns{Slovak}
    \adddialect\l@slovak0\fi
\addto\captionsslovak{%
  \def\prefacename{\'Uvod}%
  \def\refname{Referencia}%
  \def\abstractname{Abstrakt}%
  \def\bibname{Literat\'ura}%
  \def\chaptername{Kapitola}%
  \def\appendixname{Dodatok}%
  \def\contentsname{Obsah}%
  \def\listfigurename{Zoznam obr\'azkov}%
  \def\listtablename{Zoznam tabuliek}%
  \def\indexname{Index}%
  \def\figurename{Obr\'azok}%
  \def\tablename{Tabu\q lka}%%% special letter l with hook
  \def\partname{\v{C}as\q t}%%% special letter t with hook
  \def\enclname{Pr\'{\i}loha}%
  \def\ccname{CC}%
  \def\headtoname{Komu}%
  \def\pagename{Strana}%
  \def\seename{vi\q d}%%%  Special letter d with hook
  \def\alsoname{vi\q d tie\v z}}%%%  Special letter d with hook
\def\dateslovak{%
\def\today{\number\day.~\ifcase\month\or
janu\'ara\or febru\'ara\or marca\or apr\'{\i}la\or m\'aja\or j\'una\or
  j\'ula\or augusat\or septembra\or okt\'obra\or
  novembra\or decembra\fi
    \space \number\year}}
\addto\extrasslovak{\babel@save\q\let\q\slovak@q}
\addto\extrasdutch{\babel@savevariable\lefthyphenmin
  \babel@savevariable\righthyphenmin
  \lefthyphenmin\tw@\righthyphenmin\tw@}
\def\slovak@q#1{#1\kern-2pt\char'47}%
\selectlanguage{slovak}
\catcode`\@=\atcatcode \let\atcatcode\relax
%%
%% \CharacterTable
%%  {Upper-case    \A\B\C\D\E\F\G\H\I\J\K\L\M\N\O\P\Q\R\S\T\U\V\W\X\Y\Z
%%   Lower-case    \a\b\c\d\e\f\g\h\i\j\k\l\m\n\o\p\q\r\s\t\u\v\w\x\y\z
%%   Digits        \0\1\2\3\4\5\6\7\8\9
%%   Exclamation   \!     Double quote  \"     Hash (number) \#
%%   Dollar        \$     Percent       \%     Ampersand     \&
%%   Acute accent  \'     Left paren    \(     Right paren   \)
%%   Asterisk      \*     Plus          \+     Comma         \,
%%   Minus         \-     Point         \.     Solidus       \/
%%   Colon         \:     Semicolon     \;     Less than     \<
%%   Equals        \=     Greater than  \>     Question mark \?
%%   Commercial at \@     Left bracket  \[     Backslash     \\
%%   Right bracket \]     Circumflex    \^     Underscore    \_
%%   Grave accent  \`     Left brace    \{     Vertical bar  \|
%%   Right brace   \}     Tilde         \~}
%%
\endinput
%% 
%% End of file `slovak.sty'.
}
\DeclareOption{slovene}{%% 
%% This is file `slovene.sty', generated 
%% on <1995/2/23> with the docstrip utility (2.2i).
%% 
%% The original source files were:
%% 
%% slovene.dtx 
%% 
%% Copyright (C) 1989-1994 by Johannes Braams 
%% All rights reserved. 
%% For additional copyright information see further down in this file. 
%% 
%% This file is part of the Babel system, release 3.4 patchlevel 2. 
%% ---------------------------------------------------------------- 
%% 
%% This file is distributed in the hope that it will be useful, 
%% but WITHOUT ANY WARRANTY; without even the implied warranty of 
%% MERCHANTABILITY or FITNESS FOR A PARTICULAR PURPOSE. 
%% 
%% 
%% IMPORTANT NOTICE: 
%% 
%% For error reports in case of UNCHANGED versions see readme file. 
%% 
%% Please do not request updates from me directly. Distribution is 
%% done through Mail-Servers and TeX organizations. 
%% 
%% You are not allowed to change this file. 
%% 
%% You are allowed to distribute this file under the condition that 
%% it is distributed together with all files mentioned in manifest.txt. 
%% 
%% If you receive only some of these files from someone, complain! 
%% 
%% You are NOT ALLOWED to distribute this file alone. You are NOT 
%% ALLOWED to take money for the distribution or use of either this 
%% file or a changed version, except for a nominal charge for copying 
%% etc. 
%%% \iffalse ============================================================
%%%  @LaTeX-style-file{
%%%     author          = "Braams J.L.",
%%%     version         = "1.2d",
%%%     date            = "26 June 1994",
%%%     time            = "02:08:28 MET",
%%%     filename        = "slovene.doc",
%%%     address         = "PTT Research
%%%                        St. Paulusstraat 4
%%%                        2264 XZ Leidschendam
%%%                        The Netherlands",
%%%     telephone       = "(70) 3325051",
%%%     FAX             = "(70) 3326477",
%%%     checksum        = "20695 263 1222 10089",
%%%     email           = "J.L.Braams@research.ptt.nl (Internet)",
%%%     codetable       = "ISO/ASCII",
%%%     keywords        = "babel, slovene",
%%%     supported       = "yes",
%%%     abstract        = "",
%%%     docstring       = "This file contains the slovanian language specific
%%%                        definitions for the babel system.",
%%%  }
%%%
%%%  ====================================================================
%%% \fi
\ifx\undefined\captionsslovene
\else
  \selectlanguage{slovene}
  \expandafter\endinput
\fi
\chardef\atcatcode=\catcode`\@
\catcode`\@=11\relax
\ifx\undefined\babel@core@loaded\input babel.def\relax\fi
\ProvidesFile{slovene.sty}[1994/06/26 v1.2d
         Slovene support from the babel system]
\ifx\undefined\originalTeX \let\originalTeX\empty \else\originalTeX\fi
\ifx\undefined\l@slovene
    \@nopatterns{Slovene}
    \adddialect\l@slovene0\fi
\addto\captionsslovene{%
  \def\prefacename{Predgovor}%
  \def\refname{Literatura}%
  \def\abstractname{Povzetek}%
  \def\bibname{Literatura}%
  \def\chaptername{Poglavje}%
  \def\appendixname{Dodatek}%
  \def\contentsname{Kazalo}%
  \def\listfigurename{Slike}%
  \def\listtablename{Tabele}%
  \def\indexname{Indeks}%
  \def\figurename{Slika}%
  \def\tablename{Tabela}%
  \def\partname{Del}%
  \def\enclname{Priloge}%
  \def\ccname{Kopije}%
  \def\headtoname{Prejme}%
  \def\pagename{Stran}%
  \def\seename{glej}%
  \def\alsoname{glej tudi}}%
\def\dateslovene{%
\def\today{\number\day.~\ifcase\month\or
  januar\or februar\or marec\or april\or maj\or junij\or
  julij\or avgust\or september\or oktober\or november\or december\fi
  \space \number\year}}
\addto\extrasslovene{}
\addto\noextrasslovene{}
\selectlanguage{slovene}
\catcode`\@=\atcatcode \let\atcatcode\relax
%%
%% \CharacterTable
%%  {Upper-case    \A\B\C\D\E\F\G\H\I\J\K\L\M\N\O\P\Q\R\S\T\U\V\W\X\Y\Z
%%   Lower-case    \a\b\c\d\e\f\g\h\i\j\k\l\m\n\o\p\q\r\s\t\u\v\w\x\y\z
%%   Digits        \0\1\2\3\4\5\6\7\8\9
%%   Exclamation   \!     Double quote  \"     Hash (number) \#
%%   Dollar        \$     Percent       \%     Ampersand     \&
%%   Acute accent  \'     Left paren    \(     Right paren   \)
%%   Asterisk      \*     Plus          \+     Comma         \,
%%   Minus         \-     Point         \.     Solidus       \/
%%   Colon         \:     Semicolon     \;     Less than     \<
%%   Equals        \=     Greater than  \>     Question mark \?
%%   Commercial at \@     Left bracket  \[     Backslash     \\
%%   Right bracket \]     Circumflex    \^     Underscore    \_
%%   Grave accent  \`     Left brace    \{     Vertical bar  \|
%%   Right brace   \}     Tilde         \~}
%%
\endinput
%% 
%% End of file `slovene.sty'.
}
\DeclareOption{swedish}{%% 
%% This is file `swedish.sty', generated 
%% on <1995/2/23> with the docstrip utility (2.2i).
%% 
%% The original source files were:
%% 
%% swedish.dtx 
%% 
%% Copyright (C) 1989-1994 by Johannes Braams 
%% All rights reserved. 
%% For additional copyright information see further down in this file. 
%% 
%% This file is part of the Babel system, release 3.4 patchlevel 2. 
%% ---------------------------------------------------------------- 
%% 
%% This file is distributed in the hope that it will be useful, 
%% but WITHOUT ANY WARRANTY; without even the implied warranty of 
%% MERCHANTABILITY or FITNESS FOR A PARTICULAR PURPOSE. 
%% 
%% 
%% IMPORTANT NOTICE: 
%% 
%% For error reports in case of UNCHANGED versions see readme file. 
%% 
%% Please do not request updates from me directly. Distribution is 
%% done through Mail-Servers and TeX organizations. 
%% 
%% You are not allowed to change this file. 
%% 
%% You are allowed to distribute this file under the condition that 
%% it is distributed together with all files mentioned in manifest.txt. 
%% 
%% If you receive only some of these files from someone, complain! 
%% 
%% You are NOT ALLOWED to distribute this file alone. You are NOT 
%% ALLOWED to take money for the distribution or use of either this 
%% file or a changed version, except for a nominal charge for copying 
%% etc. 
%%% \iffalse ============================================================
%%%  @LaTeX-style-file{
%%%     author          = "Braams J.L.",
%%%     version         = "1.3c",
%%%     date            = "26 June 1994",
%%%     time            = "01:30:49 MET",
%%%     filename        = "swedish.doc",
%%%     address         = "PTT Research
%%%                        St. Paulusstraat 4
%%%                        2264 XZ Leidschendam
%%%                        The Netherlands",
%%%     telephone       = "(70) 3325051",
%%%     FAX             = "(70) 3326477",
%%%     checksum        = "07650 265 1238 10283",
%%%     email           = "J.L.Braams@research.ptt.nl (Internet)",
%%%     codetable       = "ISO/ASCII",
%%%     keywords        = "babel, swedish",
%%%     supported       = "yes",
%%%     abstract        = "",
%%%     docstring       = "This file contains the swedish language
%%%                        specific definitions for the babel system.",
%%%  }
%%%
%%%  ====================================================================
%%% \fi
\ifx\undefined\captionsswedish
\else
  \selectlanguage{swedish}
  \expandafter\endinput
\fi
\chardef\atcatcode=\catcode`\@
\catcode`\@=11\relax
\ifx\undefined\babel@core@loaded\input babel.def\relax\fi
\ProvidesFile{swedish.sty}[1994/06/26/ v1.3d
         Swedish support from the babel system]
\ifx\undefined\originalTeX \let\originalTeX\empty \else\originalTeX\fi
\ifx\undefined\l@swedish
    \@nopatterns{Swedish}
    \adddialect\l@swedish0\fi
\addto\captionsswedish{%
  \def\prefacename{F\"orord}%
  \def\refname{Referenser}%
  \def\abstractname{Sammanfattning}%
  \def\bibname{Litteraturf\"orteckning}%
  \def\chaptername{Kapitel}%
  \def\appendixname{Bilaga}%
  \def\contentsname{Inneh\aa ll}%
  \def\listfigurename{Figurer}%
  \def\listtablename{Tabeller}%
  \def\indexname{Sakregister}%
  \def\figurename{Figur}%
  \def\tablename{Tabell}%
  \def\partname{Del}%
  \def\enclname{Bil}%
  \def\ccname{Kopia f\"or k\"annedom}%
  \def\headtoname{Till}% in letter
  \def\pagename{Sida}%
  \def\seename{se}%
  \def\alsoname{se ocks\aa\ }}%
\def\dateswedish{%
\def\today{\number\day~\ifcase\month\or
  januari\or februari\or mars\or april\or maj\or juni\or
  juli\or augusti\or september\or oktober\or november\or december\fi
  \space\number\year}}
\addto\extrasswedish{}
\addto\noextrasswedish{}
\selectlanguage{swedish}
\catcode`\@=\atcatcode \let\atcatcode\relax
%%
%% \CharacterTable
%%  {Upper-case    \A\B\C\D\E\F\G\H\I\J\K\L\M\N\O\P\Q\R\S\T\U\V\W\X\Y\Z
%%   Lower-case    \a\b\c\d\e\f\g\h\i\j\k\l\m\n\o\p\q\r\s\t\u\v\w\x\y\z
%%   Digits        \0\1\2\3\4\5\6\7\8\9
%%   Exclamation   \!     Double quote  \"     Hash (number) \#
%%   Dollar        \$     Percent       \%     Ampersand     \&
%%   Acute accent  \'     Left paren    \(     Right paren   \)
%%   Asterisk      \*     Plus          \+     Comma         \,
%%   Minus         \-     Point         \.     Solidus       \/
%%   Colon         \:     Semicolon     \;     Less than     \<
%%   Equals        \=     Greater than  \>     Question mark \?
%%   Commercial at \@     Left bracket  \[     Backslash     \\
%%   Right bracket \]     Circumflex    \^     Underscore    \_
%%   Grave accent  \`     Left brace    \{     Vertical bar  \|
%%   Right brace   \}     Tilde         \~}
%%
\endinput
%% 
%% End of file `swedish.sty'.
}
\DeclareOption{turkish}{%% 
%% This is file `turkish.sty', generated 
%% on <1995/2/23> with the docstrip utility (2.2i).
%% 
%% The original source files were:
%% 
%% turkish.dtx 
%% 
%% Copyright (C) 1989-1994 by Johannes Braams 
%% All rights reserved. 
%% For additional copyright information see further down in this file. 
%% 
%% This file is part of the Babel system, release 3.4 patchlevel 2. 
%% ---------------------------------------------------------------- 
%% 
%% This file is distributed in the hope that it will be useful, 
%% but WITHOUT ANY WARRANTY; without even the implied warranty of 
%% MERCHANTABILITY or FITNESS FOR A PARTICULAR PURPOSE. 
%% 
%% 
%% IMPORTANT NOTICE: 
%% 
%% For error reports in case of UNCHANGED versions see readme file. 
%% 
%% Please do not request updates from me directly. Distribution is 
%% done through Mail-Servers and TeX organizations. 
%% 
%% You are not allowed to change this file. 
%% 
%% You are allowed to distribute this file under the condition that 
%% it is distributed together with all files mentioned in manifest.txt. 
%% 
%% If you receive only some of these files from someone, complain! 
%% 
%% You are NOT ALLOWED to distribute this file alone. You are NOT 
%% ALLOWED to take money for the distribution or use of either this 
%% file or a changed version, except for a nominal charge for copying 
%% etc. 
%%% \iffalse ===========================================================
%%%  @LaTeX-style-file{
%%%     author-1        = "Mustafa Burc",
%%%     author-2        = "Braams J.L.",
%%%     version         = "1.2b",
%%%     date            = "26 June 1994",
%%%     time            = "02:11:18 MET",
%%%     filename        = "turkish.doc",
%%%     address-1       = "Hamburg
%%%                        Germany",
%%%     address-2       = "PTT Research
%%%                        St. Paulusstraat 4
%%%                        2264 XZ Leidschendam
%%%                        The Netherlands",
%%%     telephone-1     = "(40) 2503476",
%%%     telephone-2     = "(70) 3325051",
%%%     FAX-2           = "(70) 3326477",
%%%     checksum        = "54850 334 1418 12575",
%%%     email-1         = "rz6001@rziris01.rrz.uni-hamburg.de",
%%%     email-2         = "J.L.Braams@research.ptt.nl (Internet)",
%%%     codetable       = "ISO/ASCII",
%%%     keywords        = "babel, turkish",
%%%     supported       = "yes",
%%%     abstract        = "",
%%%     docstring       = "This file contains the turkish language
%%%                        specific definitions for the babel system.
%%%
%%%                        The checksum field above contains a CRC-16
%%%                        checksum as the first value, followed by the
%%%                        equivalent of the standard UNIX wc (word
%%%                        count) utility output of lines, words, and
%%%                        characters.  This is produced by Robert
%%%                        Solovay's checksum utility.",
%%%  }
%%%
%%%  ====================================================================
%%% \fi
\ifx\undefined\captionsturkish
\else
  \selectlanguage{turkish}
  \expandafter\endinput
\fi
\chardef\atcatcode=\catcode`\@
\catcode`\@=11\relax
\ifx\undefined\babel@core@loaded\input babel.def\relax\fi
\ProvidesFile{turkish.sty}[1994/06/26 v1.2c
         Turkish support from the babel system]
\ifx\undefined\originalTeX \let\originalTeX\empty\fi
\originalTeX
\ifx\undefined\l@turkish
  \@nopatterns{Turkish}
  \adddialect\l@turkish0\fi
\addto\captionsturkish{%
  \def\prefacename{Preface}% <-- This needs translation!!
  \def\refname{Ba\c svurulan Kitaplar}%
  \def\abstractname{Konu}%
  \def\bibname{Bibliografi}%
  \def\chaptername{Anab\"ol\"um}%
  \def\appendixname{Appendix}%
  \def\contentsname{\.I\c cindekiler}%
  \def\listfigurename{\c Sekiller Listesi}%
  \def\listtablename{Tablolar\i{}n Listesi}%
  \def\indexname{\.Index}%
  \def\figurename{\c Sekiller}%
  \def\tablename{Tablo}%
  \def\partname{B\"ol\"um}%
  \def\enclname{Ekler}%
  \def\ccname{G\"onderen}%
  \def\headtoname{Al\i{}c\i}%
  \def\pagename{Sayfa}%
  \def\subjectname{To}% <-- This needs translation!!
  \def\seename{see}% <-- This needs translation!!
  \def\alsoname{see also}% <-- This needs translation!!
}%
\def\dateturkish{%
  \def\today{\number\day.~\ifcase\mont\or
    Ocak\or \c Subat\or Mart\or Nisan\or May\i{}s\or Haziran\or
    Temmuz\or A\u gustos\or Eyl\"ul\or Ekim\or Kas\i{}m\or
    Aral\i{}k\fi
    \space\number\year}}
}
\addto\extrasturkish{%
  \babel@add@special\:%
  \babel@add@special\!%
  \babel@add@special\=}
\addto\noextrasturkish{%
  \babel@remove@special\:%
  \babel@remove@special\!%
  \babel@remove@special\=}
\addto\extrasturkish{%
  \babel@savevariable{\catcode`\:}\babel@save\active@co
  \babel@savevariable{\catcode`\!}\babel@save\active@em
  \babel@savevariable{\catcode`\=}\babel@save\active@es
}
\addto\extrasturkish{\catcode`\:\active\catcode`\!\active
                     \catcode`\=\active}
\begingroup
  \catcode`\:\active
  \catcode`\!\active
  \catcode`\=\active
  \def\x{\endgroup
    \addto\extrasturkish{%
      \babel@save:\babel@save!\babel@save=%
      \def:{\protect\active@co}\let\active@co\turkish@active@co
      \def!{\protect\active@em}\let\active@em\turkish@active@em
      \def={\protect\active@es}\let\active@es\turkish@active@es}}
\x
\addto\extrasturkish{%
  \ifnum\the\sfcode`\.=\@m
    \frenchspacing
    \addto\noextrasturkish{\nonfrenchspacing}%
  \fi}
\turkish@active@co{\ifmmode\string:\else\relax
    \ifhmode\ifdim\lastskip>\z@
    \unskip\penalty\@M\thinspace\fi\fi\string:\fi}
\def\turkish@active@em{\ifmmode\string!\else\relax
     \ifhmode\ifdim\lastskip>\z@
    \unskip\penalty\@M\thinspace\fi\fi\string!\fi}
\def\turkish@active@es{\ifmmode\string=\else\relax
    \ifhmode\ifdim\lastskip>\z@
    \unskip\kern\fontdimen2\font
     \kern-1.4\fontdimen3\font
    \fi\fi
    \string=\fi}
\selectlanguage{turkish}
\catcode`\@\atcatcode \let\atcatcode\relax
%%
%% \CharacterTable
%%  {Upper-case    \A\B\C\D\E\F\G\H\I\J\K\L\M\N\O\P\Q\R\S\T\U\V\W\X\Y\Z
%%   Lower-case    \a\b\c\d\e\f\g\h\i\j\k\l\m\n\o\p\q\r\s\t\u\v\w\x\y\z
%%   Digits        \0\1\2\3\4\5\6\7\8\9
%%   Exclamation   \!     Double quote  \"     Hash (number) \#
%%   Dollar        \$     Percent       \%     Ampersand     \&
%%   Acute accent  \'     Left paren    \(     Right paren   \)
%%   Asterisk      \*     Plus          \+     Comma         \,
%%   Minus         \-     Point         \.     Solidus       \/
%%   Colon         \:     Semicolon     \;     Less than     \<
%%   Equals        \=     Greater than  \>     Question mark \?
%%   Commercial at \@     Left bracket  \[     Backslash     \\
%%   Right bracket \]     Circumflex    \^     Underscore    \_
%%   Grave accent  \`     Left brace    \{     Vertical bar  \|
%%   Right brace   \}     Tilde         \~}
%%
\endinput
%% 
%% End of file `turkish.sty'.
}
%    \end{macrocode}
%
%    The options have to be processed in the order in which the user
%    specified them:
%    \begin{macrocode}
\ProcessOptions*
%</package>
%    \end{macrocode}
%
% \section{The Kernel of Babel}
%
%    The kernel of the \babel\ system is stored in either
%    \file{lthyphen.cfg} or \file{switch.def} and \file{babel.def}. The
%    file \file{lthyphen.cfg} is a file that can be loaded into the
%    format, which is necessary when you want to be able to switch
%    hyphenation patterns. The file \file{babel.def} contains some
%    \TeX\ code that can be read in at run time. When \file{babel.def}
%    is loaded it checks if \file{lthyphen.cfg} is in the format; if
%    not the file \file{switch.def} is loaded.
%
%    Because plain \TeX\ users might want to use some of the features
%    of the \babel{} system too, care has to be taken that plain \TeX\
%    can process the files. For this reason the current format will
%    have to be checked in a number of places. Some of the code below
%    is common to plain \TeX\ and \LaTeX, some of it is for the
%    \LaTeX\ case only.
%
%    Check the presence of the command |\iflanguage|, if it is
%    undefined read the file \file{switch.def}.
% \changes{babel~3.0d}{29 oct 91}{Removed use of {\tt\bsl @ifundefined}}
%    \begin{macrocode}
%<*def>
\ifx\undefined\iflanguage
  \input switch.def\relax
\fi
%    \end{macrocode}
%
%    To communicate to the language specific files that the core of
%    the \babel{} system has been loaded, the following control
%    sequence is just |\let| equal to |\relax|.
%    \begin{macrocode}
\let\babel@core@loaded\relax
%</def>
%    \end{macrocode}
%
% \subsection{Multiple languages}
%
%    With \TeX\ version~3.0 it has become possible to load hyphenation
%    patterns for more than one language. This means that some extra
%    administration has to be taken care of.  The user has to know for
%    which languages patterns have been loaded, and what values of
%    |\language| have been used.
%
%    Some discussion has been going on in the \TeX\ world about how to
%    use |\language|. Some have suggested to set a fixed standard,
%    i.\,e., patterns for each language should {\em always} be loaded
%    in the same location. It has also been suggested to use the {\sc
%    iso} list for this purpose. Others have pointed out that the {\sc
%    iso} list contains more than 256~languages, which have {\em
%    not\/} been numbered consecutively.
%
%    I think the best way to use |\language|, is to use it
%    dynamically.  This code implements an algorithm to do so. It uses
%    an external file in which the person who maintains a \TeX\
%    environment has to record for which languages he has hyphenation
%    patterns {\em and\/} in which files these are
%    stored\footnote{This is because different operating systems
%    sometimes use {\em very} different filenaming conventions.}. This
%    ``configuration file'' can contain empty lines and comments, as
%    can be seen from this example:
% \begin{verbatim}
%    % File    : language.dat
%    % Purpose : tell iniTeX what files with patterns to load.
%    english    english.hyphenations
%
%    dutch      hyphen.dutch % Nederlands
%    german hyphen.ger
% \end{verbatim}
%
% \changes{hyphen-1.1}{31 oct 91}{Removed two of the {\tt docstrip}
%    options}
%
%    As the file {\tt switch.def} needs to be read only once, we check
%    whether it was read before.  If it was, the command |\iflanguage|
%    is already defined, so we can stop processing.
%    \begin{macrocode}
%<*kernel>
%<*!patterns>
\expandafter\ifx\csname iflanguage\endcsname\relax \else
\expandafter\endinput
\fi
%</!patterns>
%    \end{macrocode}
%
% \begin{macro}{\language}
%    Plain \TeX\ version~3.0 provides the primitive |\language| that
%    is used to store the current language. When used with a pre-3.0
%    version this function has to be implemented by allocating a
%    counter.
%
%    \begin{macrocode}
\ifx\language\undefined
  \csname newcount\endcsname\language
\fi
%    \end{macrocode}
% \end{macro}
%
% \begin{macro}{\last@language}
%    Another counter is used to store the last language defined.  For
%    pre-3.0 formats an extra counter has to be allocated,
%    \begin{macrocode}
\ifx\newlanguage\undefined
  \csname newcount\endcsname\last@language
%    \end{macrocode}
%    plain \TeX\ version 3.0 uses |\count 19| for this purpose.
%    \begin{macrocode}
\else
  \countdef\last@language=19
\fi
%    \end{macrocode}
% \end{macro}
%
% \begin{macro}{\addlanguage}
%
%    To add languages to \TeX's memory plain \TeX\ version~3.0
%    supplies |\newlanguage|, in a pre-3.0 environment a similar macro
%    has to be provided. For both cases a new macro is defined here,
%    because the original |\newlanguage| was defined to be |\outer|.
%
%    For a format based on plain version~2.x, the definition of
%    |\newlanguage| can not be copied because |\count 19| is used for
%    other purposes in these formats. Therefore |\addlanguage| is
%    defined using a definition based on the macros used to define
%    |\newlanguage| in plain \TeX\ version~3.0.
% \changes{hyphen-1.1a}{11 nov 91}{Added a {\tt\%}, removed {\tt by}}
%    \begin{macrocode}
\ifx\newlanguage\undefined
  \def\addlanguage#1{%
    \global\advance\last@language \@ne
    \ifnum\last@language<\@cclvi
    \else
        \errmessage{No room for a new \string\language!}%
    \fi
    \global\chardef#1\last@language
    \wlog{\string#1 = \string\language\the\last@language}}
%    \end{macrocode}
%
%    For formats based on plain version~3.0 the definition of
%    |\newlanguage| can be simply copied, removing |\outer|.
%
%    \begin{macrocode}
\else
  \def\addlanguage{\alloc@9\language\chardef\@cclvi}
\fi
%    \end{macrocode}
% \end{macro}
%
% \begin{macro}{\adddialect}
%    The macro |\adddialect| can be used to add the name of a dialect
%    or variant language, for which an already defined hyphenation
%    table can be used.
% \changes{hyphen-1.1a}{11 nov 91}{Added {\tt\bsl relax}}
%    \begin{macrocode}
\def\adddialect#1#2{%
    \global\chardef#1#2\relax
    \wlog{\string#1 = a dialect from \string\language#2}}
%    \end{macrocode}
% \end{macro}
%
% \begin{macro}{\iflanguage}
%    Users might want to test (in a private package for instance)
%    which language is currently active. For this we provide a test
%    macro, |\iflanguage|, that has three arguments.  It checks
%    whether the first argument is a known language. If so, it
%    compares the first argument with the value of |\language|. Then,
%    depending on the result of the comparison, it executes either the
%    second or the third argument.
% \changes{hyphen-1.0b}{29 may 91}{Added {\tt\bsl @bsphack} and
%    {\tt\bsl @esphack}}
% \changes{hyphen-1.0d}{21 july 91}{Added comment character after
%    {\tt\#2}}
% \changes{hyphen-1.0e}{8 aug 91}{Removed superfluous {\tt\bsl
%    expandafter}}
% \changes{hyphen-1.0h}{7 okt 91}{Removed space hacks and use of
%    {\tt\bsl @ifundefined}}
% \changes{hyphen-1.1a}{11 nov 91}{Refrased {\tt\bsl ifnum} test}
%    \begin{macrocode}
\def\iflanguage#1#2#3{%
  \expandafter\ifx\csname l@#1\endcsname\relax
    \@nolanerr{#1}%
  \else
    \ifnum\csname l@#1\endcsname=\language #2%
    \else#3\fi
  \fi}
%    \end{macrocode}
% \end{macro}
%
% \begin{macro}{\selectlanguage}
%    The macro |\selectlanguage| checks whether the language is
%    already defined before it performs its actual task, which is to
%    update |\language| and activate language-specific definitions.
%
%    To allow the call of |\selectlanguage| either with a control
%    sequence name or with a simple string, we have to use a trick to
%    delete an existing escape character.
%
%    To convert a control sequence to a string, we use the |\string|
%    primitive.  Next we have to look at the first character of this
%    string and compare it with the escape character.  Because this
%    escape character can be changed by setting the internal integer
%    |\escapechar| to a character number, we have to compare this
%    number with the character of the string.  To do this we have to
%    use \TeX's backquote notation to specify the character as a
%    number.
%
%    If the first character of the |\string|'ed argument is the
%    current escape character, the comparison has stripped this
%    character and the rest in the `then' part consists of the rest of
%    the control sequence name.  Otherwise we know that either the
%    argument is not a control sequence or |\escapechar| is set to a
%    value outside of the character range~$0$--$255$.
%
%    If the user gives an empty argument, we provide a default
%    argument for |\string|.  This argument should expand to nothing.
%
% \changes{hyphen-1.0c}{06/06/91}{Made {\tt\bsl selectlanguage}
%    robust}
% \changes{hyphen-1.1a}{11 nov 91}{Modified to allow arguments that
%    start with an escape character}
% \changes{hyphen-1.1b}{17 nov 91}{Simplified the modification to
%    allow the use in a {\tt\bsl write} command}
%    \begin{macrocode}
\def\selectlanguage#1{\protect\p@selectlanguage{%
  \ifnum\escapechar=\expandafter`\string#1\empty
     \else \string#1\empty\fi}}
%    \end{macrocode}
%    Because the command |\selectlanguage| could be used in a moving
%    argument it expands to |\protect\p@selectlanguage|.  Therefore,
%    we have to make sure that a macro |\protect| exists.  If it
%    doesn't it is |\let| to |\relax|.
%    \begin{macrocode}
\ifx\undefined\protect\let\protect\relax\fi
%    \end{macrocode}
%
%    {\small Remark: If the |\selectlanguage| command is written to a
%    file, a possible control sequence argument gets totally expanded
%    to the string without the leading escape character.  In the
%    normal case we have to deal with the fact that the argument of
%    |\p@selectlanguage| is totally unexpanded at the moment.  There
%    is only one disadvantage in the current implementation:
%    |\originalTeX| contains this unexpanded argument and therefore
%    needs more memory for its macro definition.  \par}
%
%    First, check if the user asks for a known language. If so,
%    update the value of |\language| and call |\originalTeX|
%    to bring \TeX\ in a certain pre-defined state.
% \changes{hyphen-1.0b}{29 may 91}{Added {\tt\bsl @bsphack} and
%    {\tt\bsl @esphack}}
% \changes{hyphen-1.0e}{8 aug 91}{Removed superfluous {\tt\bsl
%    expandafter}}
% \changes{hyphen-1.0h}{7 okt 91}{Removed space hacks and use of
%                                    {\tt\bsl @ifundefined}}
% \changes{hyphen-1.1b}{17 nov 91}{Added {\tt\bsl relax} as first
%    command to stop an expansion if {\tt\bsl protect} is empty}
%    \begin{macrocode}
\def\p@selectlanguage#1{\relax
  \expandafter\ifx\csname l@#1\endcsname\relax
    \@nolanerr{#1}%
  \else
    \language=\csname l@#1\endcsname\relax
    \originalTeX
%    \end{macrocode}
%    The name of the language is stored in the control sequence
%    |\languagename|. The contents of this control sequence could be
%    tested in the following way:
% \begin{verbatim}
%    \edef\tmp{\string english}
%    \ifx\languagename\tmp
%        ...
%    \else
%        ...
%    \fi
% \end{verbatim}
%    The construction with |\string| is necessary because
%    |\languagename| returns the name with characters of category code
%    {\tt 12} (other).
% \changes{hyphen-1.1f}{17 feb 92}{Store the name of the language in
%    {\tt\bsl languagename}}
%    \begin{macrocode}
    \edef\languagename{#1}%
%    \end{macrocode}
% \changes{hyphen-1.1a}{11 nov 91}{Added initialisation of
%    {\tt\bsl...@xcodes} macros}
% \changes{hyphen-1.1c}{20 nov 91}{Deleted initialisation of
%    {\tt\bsl...@xcodes} macros {\tt :-)}}
%    Then we have to {\em re\/}define |\originalTeX| to compensate for
%    the things that have been activated.  To save memory space for
%    the macro definition of |\originalTeX|, we construct the control
%    sequence name for the {\tt\ttbs noextras\langvar} command at
%    definition time by expanding the |\csname| primitive.
% \changes{hyphen-1.0c}{06 jun 91}{Replaced {\tt\bsl gdef} with
%    {\tt\bsl def}}
% \changes{hyphen-1.1}{31 oct 91}{{\tt\bsl originalTeX} should only be
%    executed once}
% \changes{hyphen-1.1b}{17 nov 91}{Added three {\tt\bsl expandafter}
%    to save macro space for {\tt\bsl originalTeX}}
% \changes{hyphen-1.1c}{20 nov 91}{Moved definition of {\tt\bsl
%    originalTeX} before {\tt\bsl extras\langvar}}
% \changes{hyphen-1.1d}{24 nov 91}{Set {\tt\bsl originalTeX} to
%    {\tt\bsl empty}, because it should be expandable.}
%    \begin{macrocode}
    \expandafter\def\expandafter\originalTeX
        \expandafter{\csname noextras#1\endcsname
                     \let\originalTeX\empty}%
    \babel@beginsave
%    \end{macrocode}
%    Now activate the language-specific definitions. This is done by
%    constructing the names of three macros by concatenating three
%    words with the argument of |\selectlanguage|, and calling these
%    macros.
%    \begin{macrocode}
    \csname captions#1\endcsname
    \csname date#1\endcsname
    \csname extras#1\endcsname\relax
  \fi}%
%    \end{macrocode}
% \end{macro}
%
%    The macro|\originalTeX| should be known to \TeX\ at this moment.
%    As it has to be expandable we |\let| it to |\empty| instead of
%    |\relax|.
% \changes{hyphen-1.1d}{24 nov 91}{Set {\tt\bsl originalTeX} to
%    {\tt\bsl empty}, because it should be expandable.}
%    \begin{macrocode}
\ifx\undefined\originalTeX\let\originalTeX\empty\fi
%    \end{macrocode}
%    Because this part of the code can be included in a format, we
%    make sure that the macro which initialises the save mechanism,
%    |\babel@beginsave|, is not considered to be undefined.
%    \begin{macrocode}
\ifx\undefined\babel@beginsave\let\babel@beginsave\relax\fi
%    \end{macrocode}
%
% \begin{macro}{\@nolanerr}
% \changes{babel~3.4e}{1994/06/25}{Use \cs{PackageError} in \LaTeXe\
%    mode}
% \begin{macro}{\@nopatterns}
% \changes{babel~3.4e}{1994/06/25}{Macro added}
%    The \babel\ package will signal an error when a documents tries
%    to select a language that hasn't been defined earlier. When a
%    user selects a language for which no hyphenation patterns were
%    loaded into the format he will be given a warning about that
%    fact. We revert to the patterns for |\language|=0 in that case.
%    In most formats that will be (us)english, but it might also be
%    empty.
%
%    When the format knows about |\PackageError| it must be \LaTeXe,
%    so we can safely use its error handling interface. Otherwise
%    we'll have to `keep it simple'.
% \changes{hyphen-1.0h}{7 okt 91}{Added a percent sign to remove
%    unwanted white space}
%    \begin{macrocode}
\ifx\PackageError\undefined
  \def\@nolanerr#1{%
    \errhelp{Your command will be ignored, type <return> to proceed}%
    \errmessage{You haven't defined the language #1\space yet}}
  \def\@nopatterns#1{%
    \message{No hyphenation patterns were loaded for}
    \message{the language `#1'}
    \message{I will use the patterns loaded for \string\language=0
          instead}}
\else
  \def\@nolanerr#1{%
    \PackageError{babel}%
                 {You haven't defined the language #1\space yet}%
        {Your command will be ignored, type <return> to proceed}}
  \def\@nopatterns#1{%
    \PackageWarningNoLine{babel}%
        {No hyphenation patterns were loaded for\MessageBreak
          the language `#1'\MessageBreak
          I will use the patterns loaded for \string\language=0
          instead}}
\fi
%    \end{macrocode}
% \end{macro}
% \end{macro}
%
%    The following code is meant to be read by ini\TeX\ because it
%    should instruct \TeX\ to read hyphenation patterns. To this end
%    the {\tt docstrip} option {\tt patterns} can be used to include
%    this code in the file \file{lthyphen.cfg}.
%    \begin{macrocode}
%<*patterns>
%    \end{macrocode}
%
% \begin{macro}{\patterns@loaded}
% \changes{hyphen-1.0i}{27 okt 91}{Added a token register for
%    collecting the names of patterns that are loaded by ini\TeX.}
%    It has been suggested to add a remark to \LaTeX's |\everyjob|
%    message, stating which hyphenation patterns have been loaded.
%    This can be done by first collecting (in a token register) the
%    names when processing the file {\tt language.dat} and afterwards
%    adding a string to the |\everyjob| message. The token register is
%    initially empty.
%
%    \begin{macrocode}
\newtoks\patterns@loaded \global\patterns@loaded={}
%    \end{macrocode}
% \end{macro}
%
% \begin{macro}{\process@language}
%    The macro |\process@language| is used to process a non-empty line
%    from the `configuration file'. It has two arguments, delimited by
%    white space. The first argument is the `name' of a language, the
%    second is the name of the file that contains the patterns.
%
%    The first thing to do is call |\addlanguage| to allocate a
%    pattern register and to make that register `active'
% \changes{hyphen-1.0e}{8 aug 91}{Removed superfluous {\tt\bsl
%    expandafter}}
% \changes{hyphen-1.0f}{21 aug 91}{Reinserted {\tt\bsl expandafter}}
% \changes{hyphen-1.0i}{27 okt 1991}{Added the collection of pattern
%    names.}
%    \begin{macrocode}
\def\process@language#1 #2 {%
    \expandafter\addlanguage\csname l@#1\endcsname
    \expandafter\language\csname l@#1\endcsname
%    \end{macrocode}
%    Then the `name' of the language that will be loaded now is
%    added to the token register |\patterns@loaded|. and finally
%    the pattern file is read.
%    \begin{macrocode}
    \global\patterns@loaded\expandafter{\the\patterns@loaded#1, }%
%    \end{macrocode}
%
% \changes{v3.4e}{1994/06/24}{Added code to detect assignments to
%    left- and righthyphenmin in the patternfile.}
%
%    Some pattern files contain assignments to |\lefthyphenmin| and
%    |\righthyphenmin|. \TeX\ does not keep track of these
%    assignments. Therefore we try to detect such assignments and add
%    them to the |\extras|\langvar\ command. When no assignments
%    were made we provide a default setting.
%    \begin{macrocode}
    \lefthyphenmin\m@ne
    \input #2\relax
    \ifnum\lefthyphenmin=\m@ne
      \lefthyphenmin\tw@
      \righthyphenmin\thr@@
    \fi
    \expandafter\edef\csname extras#1\endcsname{%
      \lefthyphenmin\the\lefthyphenmin\relax
      \righthyphenmin\the\righthyphenmin\relax}}
%    \end{macrocode}
% \end{macro}
%
%    The configuration file can now be opened for reading.
%    \begin{macrocode}
\openin1 = language.dat
%    \end{macrocode}
%
%    See if the file exists, if not, use the default hyphenation file
%    \file{hyphen.tex}. The user will be informed about this.
%
%    \begin{macrocode}
\ifeof1
    \message{I couldn't find the file language.dat,\space
             I will try the file hyphen.tex}
    \input hyphen.tex\relax
\else
%    \end{macrocode}
%
%    Pattern registers are allocated using count register
%    |\last@language|. Its initial value is~0. The definition of the
%    macro |\newlanguage| is such that it first increments the count
%    register and then defines the language. In order to have the
%    first patterns loaded in pattern register number~0 we initialize
%    |\last@language| with the value~$-1$.
%
% \changes{hyphen-1.1}{21 may 91}{Removed use of {\tt\bsl toks0}}
%    \begin{macrocode}
\last@language=-1
%    \end{macrocode}
%
%    We now read lines from the file until the end is found
%
%    \begin{macrocode}
\loop
%    \end{macrocode}
%
%    While reading from the input it is useful to switch off
%    recognition of the end-of-line character. This saves us stripping
%    off spaces from the contents of the controlsequence.
%
%    \begin{macrocode}
  \endlinechar=-1
  \read1 to \@line
  \endlinechar`\^^M
%    \end{macrocode}
%
%    Empty lines are skipped.
%
%    \begin{macrocode}
  \ifx\@line\empty
  \else
%    \end{macrocode}
%
%    All the |\expandafter|s are needed because the definition of
%    |\process@language| requires a space after the second argument.
%    The space might be there, but to make sure it is {\em always\/}
%    present an extra space is added.
%
% \changes{hyphen-1.1}{21 may 91}{Removed use of {\tt\bsl toks0}}
% \changes{hyphen-1.1}{31 oct 91}{Different way of adding a space
%    token}
%    \begin{macrocode}
    \expandafter\expandafter\expandafter
    \process@language\expandafter\@line\space
  \fi
%    \end{macrocode}
%
%    Check for the end of the file.  To avoid a new {\tt if} control
%    sequence we create the necessary |\iftrue| or |\iffalse| with the
%    help of |\csname|.  But there is one complication with this
%    approach: when skipping the {\tt loop...repeat} \TeX\ has to read
%    |\if|/|\fi| pairs.  So we have to insert a `dummy' |\iftrue|.
% \changes{hyphen-1.1}{31 oct 91}{Removed the extra {\tt if} control
%    sequence}
%    \begin{macrocode}
    \iftrue \csname fi\endcsname
    \csname if\ifeof1 false\else true\fi\endcsname
  \repeat
%    \end{macrocode}
%
%    Reactivate the default patterns,
%    \begin{macrocode}
  \language=0
\fi
%    \end{macrocode}
%    and close the configuration file.
% \changes{hyphen-1.1c}{20 nov 91}{Free macro space for
%                       {\tt\bsl process@language}}
%    \begin{macrocode}
\closein1
\let\process@language=\undefined
%    \end{macrocode}
%
% \changes{hyphen-1.1}{31 oct 91}{Added redefinition of {\tt\bsl dump}
%    to add a message to {\tt\bsl everyjob}}
%    We want to add a message to the message \LaTeX\ puts in the
%    |\everyjob| register. This could be done by the following code:
%    \begin{verbatim}
%    \let\orgeveryjob\everyjob
%    \def\everyjob#1{%
%      \orgeveryjob{#1}%
%      \orgeveryjob\expandafter{\the\orgeveryjob\immediate\write16{%
%          hyphenation patterns for \the\loaded@patterns loaded.}}%
%      \let\everyjob\orgeveryjob\let\orgeveryjob\undefined}
%    \end{verbatim}
%    The code above redefines the control sequence {\tt\bsl everyjob}
%    in order to be able to add something to the current contents of
%    the register. This is necessary because the processing of
%    hyphenation patterns happens long before \LaTeX\ fills the
%    register.\\
%    There are some problems with this approach though.
%  \begin{itemize}
%    \item When someone wants to use several hyphenation patterns with
%    \SliTeX\ the above scheme won't work. The reason is that \SliTeX\
%    overwrites the contents of the |\everyjob| register with its own
%    message.
%    \item Plain \TeX\ does not use the |\everyjob| register so the
%    message would not be displayed.
%  \end{itemize}
%    To circumvent this a `dirty trick' can be used. As this code is
%    only processed when creating a new format file there is one
%    command that is sure to be used, |\dump|. Therefore the orginal
%    |\dump| is saved in |\org@dump| and a new definition is supplied.
%    \begin{macrocode}
\let\orig@dump=\dump
\def\dump{%
%    \end{macrocode}
%    This new definition starts by adding an instruction to write a
%    message on the terminal and in the transcript file to inform the
%    user of the preloaded hyphenation patterns.
%    \begin{macrocode}
  \everyjob\expandafter{\the\everyjob%
    \immediate\write16{Hyphenation patterns for \the\patterns@loaded
      loaded.}}%
%    \end{macrocode}
%    Then everything is restored to the old situation and the format
%    is dumped.
%    \begin{macrocode}
  \let\dump\orig@dump\let\orig@dump\undefined\dump}
%    \end{macrocode}
%
%    Here the code for ini\TeX\ ends.
%    \begin{macrocode}
%</patterns>
%</kernel>
%    \end{macrocode}
%
% \subsection{Support for active characters}
%
% \begin{macro}{\babel@add@special}
% \changes{babel~3.2}{10 nov 91}{Added macro}
%    The macro |\babel@add@special| is used to add a new character (or
%    single character control sequence) to the macro |\dospecials|
%    (and |\@sanitize| if \LaTeX\ is used).
%
%    To keep all changes local, we begin a new group.  Then we
%    redefine the macros |\do| and |\@makeother| to add themselves and
%    the given character without expansion.
%    \begin{macrocode}
%<*def>
\def\babel@add@special#1{\begingroup
    \def\do{\noexpand\do\noexpand}%
    \def\@makeother{\noexpand\@makeother\noexpand}%
%    \end{macrocode}
%    To add the character to the macros, we expand the original macros
%    with the additional character inside the redefinition of the
%    macros.  Because |\@sanitize| can be undefined, we put the
%    definition inside a conditional.
%    \begin{macrocode}
    \edef\x{\endgroup
      \def\noexpand\dospecials{\dospecials\do#1}%
      \expandafter\ifx\csname @sanitize\endcsname\relax \else
        \def\noexpand\@sanitize{\@sanitize\@makeother#1}%
      \fi}%
%    \end{macrocode}
%    The macro |\x| contains at this moment the following:\\
%    |\endgroup\def\dospecials{|{\it old contents}%
%    |\do|\meta{char}|}|.\\
%    If |\@sanitize| is defined, it contains an additional definition
%    of this macro.  The last thing we have to do, is the expansion of
%    |\x|.  Then |\endgroup| is executed, which restores the old
%    meaning of |\x|, |\do| and |\@makeother|.  After the group is
%    closed, the new definition of |\dospecials| (and |\@sanitize|) is
%    assigned.
%    \begin{macrocode}
  \x}
%    \end{macrocode}
% \end{macro}
%
% \begin{macro}{\babel@remove@special}
% \changes{babel~3.2}{10 nov 91}{Added macro}
%    The companion of the former macro is |\babel@remove@special|.  It
%    is used to remove a character from the set macros |\dospecials|
%    and |\@sanitize|.
%
%    To keep all changes local, we begin a new group.  Then we define
%    a help macro |\x|, which expands to empty if the characters
%    match, otherwise it expands to its nonexpandable input.  Because
%    \TeX\ inserts a |\relax|, if the corresponding |\else| or |\fi|
%    is scanned before the comparison is evaluated, we provide a `stop
%    sign' which should expand to nothing.
%    \begin{macrocode}
\def\babel@remove@special#1{\begingroup
    \def\x##1##2{\ifnum`#1=`##2\noexpand\empty
                 \else\noexpand##1\noexpand##2\fi}%
%    \end{macrocode}
%    With the help of this macro we define |\do| and |\make@other|.
%    \begin{macrocode}
    \def\do{\x\do}%
    \def\@makeother{\x\@makeother}%
%    \end{macrocode}
%    The rest of the work is similar to |\babel@add@special|.
%    \begin{macrocode}
    \edef\x{\endgroup
      \def\noexpand\dospecials{\dospecials}%
      \expandafter\ifx\csname @sanitize\endcsname\relax \else
        \def\noexpand\@sanitize{\@sanitize}%
      \fi}%
  \x}
%    \end{macrocode}
% \end{macro}
%
% \subsection{Support for saving macro definitions}
%
%    To save the meaning of control sequences using |\babel@save|, we
%    use temporary control sequences.  To save hash table entries for
%    these control sequences, we don't use the name of the control
%    sequence to be saved to construct the temporary name.  Instead we
%    simply use the value of a counter, which is reset to zero each
%    time we begin to save new values.  This works well because we
%    release the saved meanings before we begin to save a new set of
%    control sequence meanings (see |\selectlanguage| and
%    |\originalTeX|).
%
% \begin{macro}{\babel@savecnt}
% \changes{babel~3.2}{10 nov 91}{Added macro}
% \begin{macro}{\babel@beginsave}
% \changes{babel~3.2}{10 nov 91}{Added macro}
%    The initialization of a new save cycle: reset the counter to
%    zero.
%    \begin{macrocode}
\def\babel@beginsave{\babel@savecnt\z@}
%    \end{macrocode}
%    Before it's forgotten, allocate the counter and initialize all.
%    \begin{macrocode}
\newcount\babel@savecnt
\babel@beginsave
%    \end{macrocode}
% \end{macro}
% \end{macro}
%
% \begin{macro}{\babel@save}
% \changes{babel~3.2}{10 nov 91}{Added macro}
%    The macro |\babel@save|\meta{csname} saves the current meaning of
%    the control sequence \meta{csname} to
%    |\originalTeX|\footnote{{\tt\bsl originalTeX} has to be
%    expandable, i.\,e.\ you shouldn't let it to {\tt\bsl relax}.}.
%    To do this, we let the current meaning to a temporary control
%    sequence, the restore commands are appended to |\originalTeX| and
%    the counter is incremented.
% \changes{babel~3.2c}{17/3/92}{missing {\tt\bsl} led to errors when
%    executing {\tt\bsl originalTeX}}
% \changes{babel~3.2d}{02/07/92}{saving in `babel@i and restoring
%    from `@babel@i doesn't work very well...}
%    \begin{macrocode}
\def\babel@save#1{%
  \expandafter\let\csname babel@\number\babel@savecnt\endcsname #1\relax
  \begingroup
    \toks@\expandafter{\originalTeX \let#1=}%
    \edef\x{\endgroup
      \def\noexpand\originalTeX{\the\toks@ \expandafter\noexpand
         \csname babel@\number\babel@savecnt\endcsname\relax}}%
  \x
  \advance\babel@savecnt\@ne}
%    \end{macrocode}
% \end{macro}
%
% \begin{macro}{\babel@savevariable}
% \changes{babel~3.2}{10 nov 91}{Added macro}
%    The macro |\babel@savevariable|\meta{variable} saves the value of
%    the variable.  \meta{variable} can be anything allowed after the
%    |\the| primitive.
%    \begin{macrocode}
\def\babel@savevariable#1{\begingroup
    \toks@\expandafter{\originalTeX #1=}%
    \edef\x{\endgroup
      \def\noexpand\originalTeX{\the\toks@ \the#1\relax}}%
  \x}
%    \end{macrocode}
% \end{macro}
%
% \subsection{Support for extending macros}
%
% \begin{macro}{\addto}
%    For each language four control sequences have to be defined that
%    control the language specific definitions. To be able to add
%    something to these macro once they have been defined the macro
%    |\addto| is introduced. It takes two arguments, a \meta{control
%    sequence} and \TeX-code to be added to the \meta{control
%    sequence}.
%
%    If the \meta{control sequence} has not been defined before it is
%    defined now.
% \changes{babel~3.1}{5 nov 91}{Added macro}
% \changes{babel~3.4}{4 feb 94}{Changed to use toks register}
%    \begin{macrocode}
\def\addto#1#2{%
  \ifx#1\undefined
    \def#1{#2}
  \else
%    \end{macrocode}
%    Otherwise the replacement text for the \meta{control sequence} is
%    expanded and stored in a token register, together with the
%    \TeX-code to be added.  Finally the \meta{control sequence} is
%    {\em re\/}defined, using the contents of the token register.
%    \begin{macrocode}
  {\toks@\expandafter{#1#2}%
   \xdef#1{\the\toks@}}%
  \fi
}
%    \end{macrocode}
% \end{macro}
%
% \subsection{Macros common to a number of languages}
%
%  \begin{macro}{\allowhyphens}
% \changes{babel~3.2b}{16 feb 92}{Moved macro from language-specific
%    files}
%    This macro makes hyphenation possible. Basically its definition
%    is nothing more than |\nobreak| |\hskip| \texttt{0pt plus
%    0pt}\footnote{\TeX\ begins and ends a word for hyphenation at a
%    glue node. The penalty prevents a linebreak at this glue node.}.
%    \begin{macrocode}
\def\allowhyphens{\penalty\@M \hskip\z@skip}
%    \end{macrocode}
%  \end{macro}
%
%  \begin{macro}{\set@low@box}
% \changes{babel~3.2b}{16 feb 92}{Moved macro from language-specific
%    files}
%    The following macro is used to lower quotes to the same level as
%    the comma.  It prepares its argument in box register~0.
%    \begin{macrocode}
\def\set@low@box#1{\setbox\tw@\hbox{,}\setbox\z@\hbox{#1}%
    \dimen\z@\ht\z@ \advance\dimen\z@ -\ht\tw@%
    \setbox\z@\hbox{\lower\dimen\z@ \box\z@}\ht\z@\ht\tw@ \dp\z@\dp\tw@}
%    \end{macrocode}
%  \end{macro}
%
%  \begin{macro}{\save@sf@q}
% \changes{babel~3.2b}{16 feb 92}{Moved macro from language-specific
%    files}
%    The macro |\save@sf@q| is used to save and reset the current
%    space factor.
%    \begin{macrocode}
\def\save@sf@q#1{{\ifhmode
    \edef\@SF{\spacefactor\the\spacefactor}\else
    \let\@SF\empty \fi \leavevmode #1\@SF}}
%    \end{macrocode}
%  \end{macro}
%
% \subsection{The redefinition of the style commands}
%
%    The rest of the code in this file can only be processed by
%    \LaTeX, so we check the current format. If it is plain \TeX,
%    processing should stop here. But, because of the need to limit
%    the scope of the definition of |\format|, a macro that is used
%    locally in the following |\if|~statement, this comparison is done
%    inside a group. To prevent \TeX\ from complaining about an
%    unclosed group, the processing of the command |\endinput| is
%    deferred until after the group is closed. This is accomplished by
%    the command |\aftergroup|.
%    \begin{macrocode}
{\def\format{lplain}
\ifx\fmtname\format
\else
  \def\format{LaTeX2e}
  \ifx\fmtname\format
  \else
    \aftergroup\endinput
  \fi
\fi}
%    \end{macrocode}
%
%    Now that we're sure that the code is seen by \LaTeX\ only, we
%    have to find out what the main (primary) document style is
%    because we want to redefine some macros.  This is only necessary
%    for releases of \LaTeX\ dated before december~1991. Therefore
%    this part of the code can be optionally included in
%    \file{babel.def} by specifying the {\tt docstrip} option {\tt
%    names}.
%    \begin{macrocode}
%<*names>
%    \end{macrocode}
%
%    The standard styles can be distinguished by checking whether some
%    macros are defined. In table~\ref{styles} an overview is given of
%    the macros that can be used for this purpose.
% \begin{table}[htb]
% \begin{center}
% \DeleteShortVerb{\|}
% \begin{tabular}{|lcp{8cm}|}
%   \hline
%   article         & : & both the \verb+\chapter+ and \verb+\opening+
%                         macros are undefined\\
%   report and book & : & the \verb+\chapter+ macro is defined and
%                         the \verb+\opening+ is undefined\\
%   letter          & : & the \verb+\chapter+ macro is undefined and
%                         the \verb+\opening+ is defined\\
%   \hline
% \end{tabular}
% \caption{How to determine the main document style}\label{styles}
% \MakeShortVerb{\|}
% \end{center}
% \end{table}
%
%    \noindent The macros that have to be redefined for the {\tt
%    report} and {\tt book} document styles happen to be the same, so
%    there is no need to distinguish between those two styles.
%
% \begin{macro}{\doc@style}
%    First a parameter |\doc@style| is defined to identify the current
%    document style. This parameter might have been defined by a
%    document style that already uses macros instead of hard-wired
%    texts, such as {\tt artikel1.sty}~\cite{BEP}, so the existence of
%    |\doc@style| is checked. If this macro is undefined, i.\,e., if
%    the document style is unknown and could therefore contain
%    hard-wired texts, |\doc@style| is defined to the default
%    value~`0'.
% \changes{babel~3.0d}{29 oct 91}{Removed use of {\tt\bsl
%    @ifundefined}}
%    \begin{macrocode}
\ifx\undefined\doc@style
  \def\doc@style{0}%
%    \end{macrocode}
%    This parameter is defined in the following {\tt if} construction
%    (see table~\ref{styles}):
%
%    \begin{macrocode}
  \ifx\undefined\opening
    \ifx\undefined\chapter
      \def\doc@style{1}%
    \else
      \def\doc@style{2}%
    \fi
  \else
    \def\doc@style{3}%
  \fi%
\fi%
%    \end{macrocode}
% \end{macro}
%
% \changes{babel~3.1}{5 nov 91}{Removed definition of {\tt\bsl
%    if@restonecol}}
%
%    \subsubsection{Redefinition of macros}
%
%    Now here comes the real work: we start to redefine things and
%    replace hard-wired texts by macros. These redefinitions should be
%    carried out conditionally, in case it has already been done.
%
%    For the {\tt figure} and {\tt table} environments we have in all
%    styles:
%    \begin{macrocode}
\@ifundefined{figurename}{\def\fnum@figure{\figurename{} \thefigure}}{}
\@ifundefined{tablename}{\def\fnum@table{\tablename{} \thetable}}{}
%    \end{macrocode}
%
%    The rest of the macros have to be treated differently for each
%    style.  When |\doc@style| still has its default value nothing
%    needs to be done.
%    \begin{macrocode}
\ifcase \doc@style\relax
\or
%    \end{macrocode}
%
%    This means that \file{\tt babel.def} is read after the {\tt
%    article} style, where no |\chapter| and |\opening| commands are
%    defined\footnote{A fact that was pointed out to me by Nico
%    Poppelier and was already used in Piet van Oostrum's document
%    style option~{\tt nl}.}.
%
%    First we have the |\tableofcontents|,
%    |\listoffigures| and |\listoftables|:
%    \begin{macrocode}
\@ifundefined{contentsname}%
    {\def\tableofcontents{\section*{\contentsname\@mkboth
          {\uppercase{\contentsname}}{\uppercase{\contentsname}}}%
      \@starttoc{toc}}}{}

\@ifundefined{listfigurename}%
    {\def\listoffigures{\section*{\listfigurename\@mkboth
          {\uppercase{\listfigurename}}{\uppercase{\listfigurename}}}
     \@starttoc{lof}}}{}

\@ifundefined{listtablename}%
    {\def\listoftables{\section*{\listtablename\@mkboth
          {\uppercase{\listtablename}}{\uppercase{\listtablename}}}
      \@starttoc{lot}}}{}
%    \end{macrocode}
%
% Then the |\thebibliography| and |\theindex| environments.
%
%    \begin{macrocode}
\@ifundefined{refname}%
    {\def\thebibliography#1{\section*{\refname
      \@mkboth{\uppercase{\refname}}{\uppercase{\refname}}}%
      \list{[\arabic{enumi}]}{\settowidth\labelwidth{[#1]}%
        \leftmargin\labelwidth
        \advance\leftmargin\labelsep
        \usecounter{enumi}}%
        \def\newblock{\hskip.11em plus.33em minus.07em}%
        \sloppy\clubpenalty4000\widowpenalty\clubpenalty
        \sfcode`\.=1000\relax}}{}

\@ifundefined{indexname}%
    {\def\theindex{\@restonecoltrue\if@twocolumn\@restonecolfalse\fi
     \columnseprule \z@
     \columnsep 35pt\twocolumn[\section*{\indexname}]%
       \@mkboth{\uppercase{\indexname}}{\uppercase{\indexname}}%
       \thispagestyle{plain}%
       \parskip\z@ plus.3pt\parindent\z@\let\item\@idxitem}}{}
%    \end{macrocode}
%
% The |abstract| environment:
%
%    \begin{macrocode}
\@ifundefined{abstractname}%
    {\def\abstract{\if@twocolumn
    \section*{\abstractname}%
    \else \small
    \begin{center}%
    {\bf \abstractname\vspace{-.5em}\vspace{\z@}}%
    \end{center}%
    \quotation
    \fi}}{}
%    \end{macrocode}
%
% And last but not least, the macro |\part|:
%
%    \begin{macrocode}
\@ifundefined{partname}%
{\def\@part[#1]#2{\ifnum \c@secnumdepth >\m@ne % IF secnumdepth > -1
        \refstepcounter{part}%                 %  THEN step part counter
        \addcontentsline{toc}{part}{\thepart   %       add toc line
        \hspace{1em}#1}\else                   %  ELSE add unnumbered line
      \addcontentsline{toc}{part}{#1}\fi       % FI
   {\parindent\z@ \raggedright
    \ifnum \c@secnumdepth >\m@ne      % IF secnumdepth > -1
      \Large \bf \partname{} \thepart %  THEN Print 'Part' and
      \par \nobreak        %                  number in \Large boldface.
    \fi                    %            FI
    \huge \bf              % Select \huge boldface.
    #2\markboth{}{}\par}%  % Print title and set heading marks null.
    \nobreak               % TeX penalty to prevent page break.
    \vskip 3ex             % Space between title and text.
   \@afterheading          % Routine called after part and
    }                      %     section heading.
}{}
%    \end{macrocode}
%
%    This is all that needs to be done for the {\tt article} style.
%
%    \begin{macrocode}
\or
%    \end{macrocode}
%
%    The next case is formed by the two styles {\tt book} and {\tt report}.
%    Basically we have to do the same as for the {\tt article} style,
%    except now we must also change the |\chapter| command.
%
%    The tables of contents, figures and tables:
%    \begin{macrocode}
\@ifundefined{contentsname}%
    {\def\tableofcontents{\@restonecolfalse
      \if@twocolumn\@restonecoltrue\onecolumn
      \fi\chapter*{\contentsname\@mkboth
          {\uppercase{\contentsname}}{\uppercase{\contentsname}}}%
      \@starttoc{toc}%
      \csname if@restonecol\endcsname\twocolumn
      \csname fi\endcsname}}{}

\@ifundefined{listfigurename}
    {\def\listoffigures{\@restonecolfalse
      \if@twocolumn\@restonecoltrue\onecolumn
      \fi\chapter*{\listfigurename\@mkboth
          {\uppercase{\listfigurename}}{\uppercase{\listfigurename}}}%
      \@starttoc{lof}%
      \csname if@restonecol\endcsname\twocolumn
      \csname fi\endcsname}}{}

\@ifundefined{listtablename}
    {\def\listoftables{\@restonecolfalse
      \if@twocolumn\@restonecoltrue\onecolumn
      \fi\chapter*{\listtablename\@mkboth
          {\uppercase{\listtablename}}{\uppercase{\listtablename}}}%
      \@starttoc{lot}%
      \csname if@restonecol\endcsname\twocolumn
      \csname fi\endcsname}}{}
%    \end{macrocode}
%
%    Again, the |bibliography| and |index| environments;
%    notice that in this case we use |\bibname| instead of
%    |\refname| as in the definitions for the {\tt article} style.
%    The reason for this is that in the {\tt article} document style
%    the term `References' is used in the definition of
%    |\thebibliography|. In the {\tt report} and {\tt book}
%    document styles the term `Bibliography' is used.
%    \begin{macrocode}
\@ifundefined{bibname}
    {\def\thebibliography#1{\chapter*{\bibname
     \@mkboth{\uppercase{\bibname}}{\uppercase{\bibname}}}%
     \list{[\arabic{enumi}]}{\settowidth\labelwidth{[#1]}%
     \leftmargin\labelwidth \advance\leftmargin\labelsep
     \usecounter{enumi}}%
     \def\newblock{\hskip.11em plus.33em minus.07em}%
     \sloppy\clubpenalty4000\widowpenalty\clubpenalty
     \sfcode`\.=1000\relax}}{}

\@ifundefined{indexname}
    {\def\theindex{\@restonecoltrue\if@twocolumn\@restonecolfalse\fi
    \columnseprule \z@
    \columnsep 35pt\twocolumn[\@makeschapterhead{\indexname}]%
      \@mkboth{\uppercase{\indexname}}{\uppercase{\indexname}}%
    \thispagestyle{plain}%
    \parskip\z@ plus.3pt\parindent\z@ \let\item\@idxitem}}{}
%    \end{macrocode}
%
% Here is the |abstract| environment:
%    \begin{macrocode}
\@ifundefined{abstractname}
    {\def\abstract{\titlepage
    \null\vfil
    \begin{center}%
    {\bf \abstractname}%
    \end{center}}}{}
%    \end{macrocode}
%
%     And last but not least the |\chapter|, |\appendix| and
%    |\part| macros.
%    \begin{macrocode}
\@ifundefined{chaptername}{\def\@chapapp{\chaptername}}{}

\@ifundefined{appendixname}
    {\def\appendix{\par
      \setcounter{chapter}{0}%
      \setcounter{section}{0}%
      \def\@chapapp{\appendixname}%
      \def\thechapter{\Alph{chapter}}}}{}

\@ifundefined{partname}
    {\def\@part[#1]#2{\ifnum \c@secnumdepth >-2\relax% IF secnumdepth > -2
            \refstepcounter{part}%              %  THEN step part counter
            \addcontentsline{toc}{part}{\thepart%       add toc line
            \hspace{1em}#1}\else                %  ELSE add unnumbered line
            \addcontentsline{toc}{part}{#1}\fi  % FI
       \markboth{}{}%
       {\centering                      % %% added 19 Jan 88
        \ifnum \c@secnumdepth >-2\relax % IF secnumdepth > -2
          \huge\bf \partname{} \thepart %  THEN Print 'Part' and number
        \par                            %        in \huge bold.
        \vskip 20pt \fi                 %       Add space before title.
        \Huge \bf                       % FI
        #1\par}\@endpart}}{}            % Print Title in \Huge bold.
%    \end{macrocode}
%
%    \begin{macrocode}
\or
%    \end{macrocode}
%
%    Now we address the case where \file{babel.def} is read after the
%    {\tt letter} style.  The {\tt letter} document style defines the
%    macro |\opening| and some other macros that are specific to {\tt
%    letter}. This means that we have to redefine other macros,
%    compared to the previous two cases.
%
%    First two macros for the material at the end of a letter, the
%    |\cc| and |\encl| macros.
%    \begin{macrocode}
\@ifundefined{ccname}%
    {\def\cc#1{\par\noindent
     \parbox[t]{\textwidth}%
     {\@hangfrom{\rm \ccname : }\ignorespaces #1\strut}\par}}{}

\@ifundefined{enclname}%
    {\def\encl#1{\par\noindent
     \parbox[t]{\textwidth}%
     {\@hangfrom{\rm \enclname : }\ignorespaces #1\strut}\par}}{}
%    \end{macrocode}
%
%    The last thing we have to do here is to redefine the {\tt
%    headings} pagestyle:
% \changes{babel~3.3}{11 jul 93}{`headpagename should be `pagename}
%    \begin{macrocode}
\@ifundefined{headtoname}
    {\def\ps@headings{%
        \def\@oddhead{\sl \headtoname{} \ignorespaces\toname \hfil
                      \@date \hfil \pagename{} \thepage}%
        \def\@oddfoot{}}}{}
%    \end{macrocode}
%
%    This was the last of the four standard document styles, so if
%    |\doc@style| has another value we do nothing and just
%    close the {\tt if} construction.
%    \begin{macrocode}
\fi
%    \end{macrocode}
%    Here ends the code that can be optionally included when a version
%    of \LaTeX\ is in use that is dated {\em before\/} december~1991.
%    \begin{macrocode}
%</names>
%    \end{macrocode}
%
% \subsection{Cross referencing macros}
%
%    The \LaTeX\ book states:
% \begin{quote}
%    The {\em key\/} argument is any sequence of letters, digits, and
%    punctuation symbols; upper- and lowercase letters are regarded as
%    different.
% \end{quote}
%    When the above quote should still be true when a document is
%    typeset in a language that has active characters, special care
%    has to be taken of the category codes of these characters when
%    they appear in an argument of the cross referencing macros.
%
%    When a cross referencing command processes its argument, all
%    tokens in this argument should be character tokens with category
%    `letter' or `other'.
%
%    The only way to accomplish this in most cases is to use the trick
%    described in the \TeX book~\cite{DEK} (Appendix~D, page~382).
%    The primitive |\meaning| applied to a token expands to the
%    current meaning of this token.  For example, `|\meaning\A|' with
%    |\A| defined as `|\def\A#1{\B}|' expands to the characters
%    `|macro:#1->\B|' with all category codes set to `other' or
%    `space'.
%
% \begin{macro}{\babel@sanitize@arg}
%    To call a macro with a `sanitized' argument, instead of |\A{\B}|
%    one would write |\babel@sanitize@arg{\A}{\B}|.  (But be careful,
%    this macro is not fully expandable!)
%    \begin{macrocode}
\long\def\babel@sanitize@arg#1#2{\bgroup\def\@tempa{#2}%
  \expandafter\babel@strip@meaning\meaning\@tempa\relax{#1}}
\def\babel@strip@meaning#1->#2\relax#3{\egroup #3{#2}}
%    \end{macrocode}
% \end{macro}
%
%    To redefine a command, we save the old meaning of the macro.
%    Then we redefine it to call the original macro with the
%    `sanitized' argument.  The reason why we do it this way is that
%    we don't want to redefine the \LaTeX\ macros completely incase
%    their definitions change (they have changed in the past).
%
% \begin{macro}{\label}
%    The |\label| macro is one of the cross referencing macros
%    affected. First we save its original definition.
%    \begin{macrocode}
\let\LTX@label=\label
%    \end{macrocode}
%    Then the macro |label| is redefined.
%    \begin{macrocode}
\def\label#1{\babel@sanitize@arg\LTX@label{#1}}
%    \end{macrocode}
% \end{macro}
%
% \begin{macro}{\newlabel}
%    The macro |\label| writes a line with a |\newlabel| command
%    into the |.aux| file to define labels.
%    \begin{macrocode}
\let\LTX@newlabel=\newlabel
\def\newlabel#1#2{\babel@sanitize@arg\LTX@newlabel{#1}{#2}}
%    \end{macrocode}
% \end{macro}
%
% \begin{macro}{\ref}
% \begin{macro}{\pageref}
%    The same holds for the macro |\ref| that references a label
%    and |\pageref| to reference a page.
%    \begin{macrocode}
\let\LTX@ref=\ref
\def\ref#1{\babel@sanitize@arg\LTX@ref{#1}}
%    \end{macrocode}
%    \begin{macrocode}
\let\LTX@pageref=\pageref
\def\pageref#1{\babel@sanitize@arg\LTX@pageref{#1}}
%    \end{macrocode}
% \end{macro}
% \end{macro}
%
% \begin{macro}{\@citex}
%    The macro used to cite from a bibliography, |\cite| uses an
%    internal macro, |\@citex|.
%    It is this internal macro that picks up the argument,
%    so we redefine this internal macro and leave |\cite| alone.
%    \begin{macrocode}
\let\LTX@@citex\@citex
\def\@citex[#1]#2{\babel@sanitize@arg{\LTX@@citex[#1]}{#2}}
%    \end{macrocode}
% \end{macro}
%
% \begin{macro}{\nocite}
%    The macro |\nocite| which is used to instruct BiB\TeX\ to
%    extract uncited references from the database.
%    \begin{macrocode}
\let\LTX@nocite\nocite
\def\nocite#1{\babel@sanitize@arg\LTX@nocite{#1}}
%    \end{macrocode}
% \end{macro}
%
% \begin{macro}{\bibcite}
%    The macro that is used in the |.aux| file to define citation
%    labels.
%    \begin{macrocode}
\let\LTX@bibcite\bibcite
\def\bibcite#1#2{\babel@sanitize@arg\LTX@bibcite{#1}{#2}}
%    \end{macrocode}
% \end{macro}
%
% \begin{macro}{\@bibitem}
%    One of the two internal \LaTeX\ macros called by |\bibitem|
%    that write the citation label on the |.aux| file.
%    \begin{macrocode}
\let\LTX@@bibitem\@bibitem
\def\@bibitem#1{\babel@sanitize@arg\LTX@@bibitem{#1}}
%    \end{macrocode}
% \end{macro}
%
% \begin{macro}{\@lbibitem}
%    The other of the two internal \LaTeX\ macros called by |\bibitem|
%    that write the citation label on the |.aux| file.
%    \begin{macrocode}
\let\LTX@@lbibitem\@lbibitem
\def\@lbibitem[#1]#2{\babel@sanitize@arg{\LTX@@lbibitem[#1]}{#2}}
%    \end{macrocode}
% \end{macro}
%
% \begin{macro}{\@testdef}
%    An internal \LaTeX\ macro used to test if the labels that have
%    been written on the |.aux| file have changed.  It is called by
%    the |\enddocument| macro.
%    \begin{macrocode}
\let\LTX@@testdef=\@testdef
\def\@testdef#1#2{\babel@sanitize@arg{\LTX@@testdef{#1}}{#2}}
%    \end{macrocode}
%</def>
% \end{macro}
%
% \Finale
%
% \clearpage
% \section{Driver files for the documented source code}
%
%    Since \babel\ version 3.4 all source files that are part of the
%    \babel\ system can be typeset separately. But in order to typeset
%    them all in one document the file \file{babel.drv} can be used.
%    If you only want the information on how to use the \babel\ system
%    and what goodies are provided by the language spcific files you
%    can run the file \file{user.drv} through \LaTeX\ to get a user
%    guide.
%
% \changes{babel~3.4b}{1994/05/18}{Use the ltxdoc class instead of
%    article}
%    \begin{macrocode}
%<*driver>
\documentclass{ltxdoc}
\DoNotIndex{\!,\',\,,\.,\-,\:,\;,\?,\/,\^,\`,\@M}
\DoNotIndex{\@,\@ne,\@m,\@afterheading,\@date,\@endpart}
\DoNotIndex{\@hangfrom,\@idxitem,\@makeschapterhead,\@mkboth}
\DoNotIndex{\@oddfoot,\@oddhead,\@restonecolfalse,\@restonecoltrue}
\DoNotIndex{\@starttoc,\@unused}
\DoNotIndex{\accent,\active}
\DoNotIndex{\addcontentsline,\advance,\Alph,\arabic}
\DoNotIndex{\baselineskip,\begin,\begingroup,\bf,\box,\c@secnumdepth}
\DoNotIndex{\catcode,\centering,\char,\chardef,\clubpenalty}
\DoNotIndex{\columnsep,\columnseprule,\crcr,\csname}
\DoNotIndex{\day,\def,\dimen,\discretionary,\divide,\dp,\do}
\DoNotIndex{\edef,\else,\empty,\end,\endgroup,\endcsname,\endinput}
\DoNotIndex{\errhelp,\errmessage,\expandafter,\fi,\filedate}
\DoNotIndex{\fileversion,\fmtname,\fnum@figure,\fnum@table,\fontdimen}
\DoNotIndex{\gdef,\global}
\DoNotIndex{\hbox,\hidewidth,\hfil,\hskip,\hspace,\ht,\Huge,\huge}
\DoNotIndex{\ialign,\if@twocolumn,\ifcase,\ifcat,\ifhmode,\ifmmode}
\DoNotIndex{\ifnum,\ifx,\immediate,\ignorespaces,\input,\item}
\DoNotIndex{\kern}
\DoNotIndex{\labelsep,\Large,\large,\labelwidth,\lccode,\leftmargin}
\DoNotIndex{\lineskip,\leavevmode,\let,\list,\ll,\long,\lower}
\DoNotIndex{\m@ne,\mathchar,\mathaccent,\markboth,\month,\multiply}
\DoNotIndex{\newblock,\newbox,\newcount,\newdimen,\newif,\newwrite}
\DoNotIndex{\nobreak,\noexpand,\noindent,\null,\number}
\DoNotIndex{\onecolumn,\or}
\DoNotIndex{\p@,par, \parbox,\parindent,\parskip,\penalty}
\DoNotIndex{\protect,\ps@headings}
\DoNotIndex{\quotation}
\DoNotIndex{\raggedright,\raise,\refstepcounter,\relax,\rm,\setbox}
\DoNotIndex{\section,\setcounter,\settowidth,\scriptscriptstyle}
\DoNotIndex{\sfcode,\sl,\sloppy,\small,\space,\spacefactor,\strut}
\DoNotIndex{\string}
\DoNotIndex{\textwidth,\the,\thechapter,\thefigure,\thepage,\thepart}
\DoNotIndex{\thetable,\thispagestyle,\titlepage,\tracingmacros}
\DoNotIndex{\tw@,\twocolumn,\typeout,\uppercase,\usecounter}
\DoNotIndex{\vbox,\vfil,\vskip,\vspace,\vss}
\DoNotIndex{\widowpenalty,\write,\xdef,\year,\z@,\z@skip}
%    \end{macrocode}
%
%     Here |\dlqq| is defined so that  an example of |"'| can be
%     given.
%    \begin{macrocode}
\makeatletter
\gdef\dlqq{{\setbox\tw@=\hbox{,}\setbox\z@=\hbox{''}%
  \dimen\z@=\ht\z@ \advance\dimen\z@-\ht\tw@
  \setbox\z@=\hbox{\lower\dimen\z@\box\z@}\ht\z@=\ht\tw@
  \dp\z@=\dp\tw@ \box\z@\kern-.04em}}
%    \end{macrocode}
%
%    The code lines are numbered within sections,
%    \begin{macrocode}
%<*!user>
\@addtoreset{CodelineNo}{section}
\renewcommand\theCodelineNo{%
  \reset@font\scriptsize\thesection.\arabic{CodelineNo}}
%    \end{macrocode}
%    which should also be visible in the index; hence this
%    redefinition of a macro from \file{doc.sty}.
%    \begin{macrocode}
\renewcommand\codeline@wrindex[1]{\if@filesw
        \immediate\write\@indexfile
            {\string\indexentry{#1}%
            {\number\c@section.\number\c@CodelineNo}}\fi}
%    \end{macrocode}
%
%    The glossary environment is used or the change log, but its
%    definition needs changing for this document.
%    \begin{macrocode}
\renewenvironment{theglossary}{%
    \glossary@prologue%
    \GlossaryParms \let\item\@idxitem \ignorespaces}%
   {}
%</!user>
\makeatother
%    \end{macrocode}
%
%    A few shorthands used in the documentation
%    \begin{macrocode}
\font\manual=logo10 % font used for the METAFONT logo, etc.
\newcommand\MF{{\manual META}\-{\manual FONT}}
\newcommand\TeXhax{\TeX hax}
\newcommand\babel{{\sf babel}}
\newcommand\ttbs{\char'134}
\newcommand\langvar{$\langle \it lang \rangle$}
%    \end{macrocode}
%
%     Some more definitions needed in the documentation.
%    \begin{macrocode}
\newcommand\note[1]{{\bf #1}}
%\newcommand\note[1]{}
\newcommand\bsl{\protect\bslash}
\newcommand\Lopt[1]{{\sf #1}}
\newcommand\file[1]{{\tt #1}}
\newcommand\languagespecific[1]{%
%<-user>  \clearpage
  \DocInput{#1}}
%    \end{macrocode}
%
%    When a full index should be generated unomment the line with
%    |\EnableCrossres|. Beware, processing may take some time.
%    Use |\DisableCrossrefs| when the index is ready.
%    \begin{macrocode}
%  \EnableCrossrefs
\DisableCrossrefs
%    \end{macrocode}
%
%    Inlude the change log.
%    \begin{macrocode}
%<-user>\RecordChanges
%    \end{macrocode}
%    The index should use the linenumbers of the code.
%    \begin{macrocode}
%<-user>\CodelineIndex
%    \end{macrocode}
%
% Set everything in |\MacroFont| instead of |\AltMacroFont|
%    \begin{macrocode}
\setcounter{StandardModuleDepth}{1}
%    \end{macrocode}
%
%    For the user guide we only want the description parts of all the
%    files.
%    \begin{macrocode}
%<+user>\OnlyDescription
%    \end{macrocode}
%    Here starts the document
%    \begin{macrocode}
\begin{document}
\DocInput{babel.dtx}
%    \end{macrocode}
%
%    All the language specific files.
% \changes{babel~3.2e}{07/07/92}{Added slovak}
% \changes{babel~3.3}{11 jul 93}{Added catalan and galician}
% \changes{babel~3.3}{11 jul 93}{Added turkish}
% \changes{babel~3.4}{1994/02/28}{Added bahasa}
%    \begin{macrocode}
%<+user>\clearpage
\languagespecific{esperant.dtx}
\languagespecific{dutch.dtx}
\languagespecific{english.dtx}
\languagespecific{germanb.dtx}
%
\languagespecific{francais.dtx}
\languagespecific{italian.dtx}
\languagespecific{portuges.dtx}
\languagespecific{spanish.dtx}
\languagespecific{catalan.dtx}
\languagespecific{galician.dtx}
%
\languagespecific{danish.dtx}
\languagespecific{norsk.dtx}
\languagespecific{swedish.dtx}
%
\languagespecific{finnish.dtx}
\languagespecific{magyar.dtx}
%
\languagespecific{croatian.dtx}
\languagespecific{czech.dtx}
\languagespecific{slovak.dtx}
\languagespecific{polish.dtx}
\languagespecific{romanian.dtx}
\languagespecific{slovene.dtx}
%%\languagespecific{russian.dtx}
%
\languagespecific{turkish.dtx}
%
\languagespecific{bahasa.dtx}
%    \end{macrocode}
%    Finally print the index and change log (not for the user guide).
%    \begin{macrocode}
%<*!user>
\clearpage
\def\filename{index}
\PrintIndex
\clearpage
\def\filename{changes}
\PrintChanges
%</!user>
\end{document}
%</driver>
%    \end{macrocode}
%%
%% \CharacterTable
%%  {Upper-case    \A\B\C\D\E\F\G\H\I\J\K\L\M\N\O\P\Q\R\S\T\U\V\W\X\Y\Z
%%   Lower-case    \a\b\c\d\e\f\g\h\i\j\k\l\m\n\o\p\q\r\s\t\u\v\w\x\y\z
%%   Digits        \0\1\2\3\4\5\6\7\8\9
%%   Exclamation   \!     Double quote  \"     Hash (number) \#
%%   Dollar        \$     Percent       \%     Ampersand     \&
%%   Acute accent  \'     Left paren    \(     Right paren   \)
%%   Asterisk      \*     Plus          \+     Comma         \,
%%   Minus         \-     Point         \.     Solidus       \/
%%   Colon         \:     Semicolon     \;     Less than     \<
%%   Equals        \=     Greater than  \>     Question mark \?
%%   Commercial at \@     Left bracket  \[     Backslash     \\
%%   Right bracket \]     Circumflex    \^     Underscore    \_
%%   Grave accent  \`     Left brace    \{     Vertical bar  \|
%%   Right brace   \}     Tilde         \~}
\endinput
