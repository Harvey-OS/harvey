%% BEGIN tp-test.tex
%%
%% LaTeX test file for textpath.tex/textpath.sty.
%%
%% To use PostScript fonts with the New Font Selection Scheme:
%%   - Include the style option "npsfont".
%%   - You may have to change the font file names given below.
%%   - "npsfont.sty" is distributed with PSTricks.

\documentstyle[pstricks,textpath,12pt,npsfont]{article}

\textwidth 6.5in
\oddsidemargin 0pt

\textheight 9in
\topmargin 0pt
\headsep 0pt
\headheight 0pt

\makeatletter
\@ifundefined{newpsfamily}{\@testfalse}{\@testtrue}
\if@test
\newpsfamily[.96]{times}{
  {m}{n}{ptmr},          %Times
  {m}{it}{ptmri},        %
  {m}{sl}{ptmri},        %
  {bx}{n}{ptmb},         %Times-Bold
  {bx}{it}{ptmbi},       %
  {bx}{sl}{ptmbi}}       %
\renewcommand{\default@family}{times}%
\renewcommand{\rmdefault}{times}%
\family\default@family\selectfont
\fi
\makeatother

\begin{document}
The first sample shows that math works. I let the line be drawn to make the sample clearer.
\begin{verbatim}
   \large
    \pstextpath[c]%
      {\pscurve[linecolor=gray](0,1)(4,3)(6,2)(9,0)(12,1)(15,1)}%
      {$S_\alpha=\Omega(\gamma_\beta)$ is a connected snarf and
      $B=(\otimes,\rightarrow,\theta)$ is Boolean left subideal.}
\end{verbatim}

\vskip 3.5cm

\begin{large}
    \pstextpath[c]%
      {\pscurve[linecolor=gray](0,1)(4,3)(6,2)(9,0)(12,1)(15,1)}%
      {$S_\alpha=\Omega(\gamma_\beta)$ is a connected snarf and
      $B=(\otimes,\rightarrow,\theta)$ is Boolean left subideal.}
\end{large}

\vskip 1cm

\begin{verbatim}
    \psset{linestyle=none}
    \pstextpath[c]{\psarcn(0,0){73pt}{180}{0}}%
      {Centre National de la}
    \pstextpath[c]{\psarc(0,0){73pt}{180}{0}}%
      {Recherche Scientifique}
\end{verbatim}

\begin{center}
    \vskip 2cm
    \Huge
    \psset{linestyle=none}
    \pstextpath[c]{\psarcn(0,0){73pt}{180}{0}}%
      {Centre National de la}
    \pstextpath[c]{\psarc(0,0){73pt}{180}{0}}%
      {Recherche Scientifique}
    \vskip 2cm
\end{center}

\begin{center}
  \pstextpath[c](0,0){\psarcn[linestyle=none](0,-6){4}{180}{0}}%
    {\parbox{3.5in}{In principle, it is possible to use
    parbox, but let's see what really happens. It seems
    hard to believe that someone would want to do this.}}
\end{center}
\end{document}
%% END tp-test.tex
