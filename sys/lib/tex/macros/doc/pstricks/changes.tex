%% BEGIN changes.tex
%%
%% Changes listing for PSTricks.
%% Run with LaTeX, with or without the NFSS.
%%
%% Change these for a4 paper:
\def\paperwidth{8.5in}
\def\paperheight{11in}

\def\FileVersion{0.93a}
\def\FileDate{March 12, 1993}

\documentstyle[12pt,fancybox]{article}

%% PAGE PARAMETERS

% Paragraphs are marked by large space rather than indentation:
\parindent 0pt
\parskip 6pt plus 1pt minus 1pt

% No headers, 1in top margin
\topmargin 0pt
\headheight 0pt
\headsep 0pt

% Total bottom margin 1in, text height 9in
\textheight 9in
\footskip .625in

% Now adjust for different paper size:
\newdimen\mydim
\mydim=\paperwidth
\advance\mydim-8.5in
\divide\mydim 2
\advance\oddsidemargin \mydim
\advance\evensidemargin \mydim
\mydim=\paperheight
\advance\mydim-11in
\divide\mydim 2
\advance\topmargin \mydim

%% OTHER

\renewcommand{\EveryVerbatimLine}[2]{}

% Short meta (works in verbatim. Can't use < for other purposes.
\catcode`\<=13 \def<#1>{{\rm\it #1\/}}    % <meta> (works in verbatim)

% Short verbatim.
\catcode`\"=13
\def"{\verb"}

\catcode`\@=12  % In case I'm using AmS-LaTeX

\begin{document}

\begin{center}
{\large\bf Changes listing for PSTricks}\\[6pt]
  Version \FileVersion\\
  \FileDate\\[6pt]
  Timothy Van Zandt\\
  tvz@Princeton.EDU
\end{center}

  The PSTricks package is still preliminary. The features list
  and user interface are unlikely to change much in the short term, but
  the internal code is not stable.

\section{VERSION 0.93 and 0.93a}

  There have been substantial changes (hopefully for the last time).

  This section describes the changes from 0.92 to 0.93a. The only differencea
between 0.93 and 0.93a are that a bug in "\nccircle" was fixed, and the
features in "\pst-old.tex" where either incorporated into the main files
("\Rput", "\Lput", "\Mput", "\Polar" and "\Cartesian") or eliminated entirely
("\NewPsput", "\OldPsput" and the old "\pscustom" commands). What is described
\subsection{Incompatible changes}

  {\bf
  These changes may require modification to files prepared with version 0.92.
Sorry about the inconvenience. To avoid letting new changes catch you by
surprise, get on the PSTricks mailing list.}
\begin{enumerate}

\item
  "\SpecialCoor" has changed completely. See User's Guide for details. Most of
these changes were announced shortly after the release of v0.92, but
"pst-beta.tex" users should note that raw PostScript coordinates are now
delimited by "!" rather than by ":".

  {\bf How to fix files:} Search for "\SpecialCoor" in your file. Change the
old syntax to the new syntax according to this table:
   \begin{center}
     \begin{tabular}{ccl}
     {\em Old} & {\em New} & {\em Type}\\[2pt]
     "(c<x>,<y>)" & "(<x>,<y>)" & Cartesian.\\
     "(p<r>,<a>)" & "(<r>;<a>)" & Polar.\\
     "(n{<node>})" & "(<node>)" & Center of <node>.\\
     "(N[<par>]{<node>})"  & "([<par>]<node>)" & Relative to <node>.\\
     "(m{<coor1>}{<coor2>})" & "(<coor1>|<coor2>)" & Mixed.
     \end{tabular}
   \end{center}
In addition, if you used the syntax "(:<ps>)" from "pst-beta.tex" for raw
postscript code, search for "(:" and replace by "(!".

\item
  Angles can no longer be specified by "{<x>,<y>}". "\SpecialCoor" lets you
use coordinates as angles, but they must be enclosed in "()". E.g.,
"{(<x>,<y>)}". See User's Guide for details.

  {\bf How to fix files:} Add the parentheses "()" to angles given by
"{<x>,<y>}" and and precede this usage by "\SpecialCoor". However, there is no
easy way to search for "{<x>,<y>}" (unless you are good at using Unix's
regular expressions). On the other hand, you probably didn't use this features
much, and it will be easier to just wait for the old usage to cause errors.

\item
  The
\begin{LVerbatim}
  arrowsize=<dim num1 num2 num3>
\end{LVerbatim}
parameter has been replaced by
\begin{LVerbatim}
  arrowsize=<dim num1>
  arrowlength=<num2>
  arrowinset=<num3>
\end{LVerbatim}

  {\bf How to fix files:}
  Search for "arrowsize", and break up your parameter change. You can also
just do nothing, because if you use the old syntax for "arrowsize", <num2> and
<num3> will simply be ignored.

\item
  The
\begin{LVerbatim}
  tbarsize=<dim num1 num2>
\end{LVerbatim}
parameter has been replaced by
\begin{LVerbatim}
  tbarsize=<dim num1>
  bracketlength=<num2>      % For square brackets.
  rbracketlength=<num2>     % For round brackets.
\end{LVerbatim}

  {\bf How to fix files:}
  Search for "tbarsize", and break up your parameter change. You can also just
do nothing, because if you use the old syntax for "tbarsize", <num2> will
simply be ignored.

\item
  "\pscustom" has changed substantially. E.g., (i) there is no  "(x,y)"
argument, (ii) plots ("\psplot", etc.) no longer run backwards, and (iii) the
treatment of the currentpoint is much different. Also, most of the special
commands for use only within "\pscustom" have changed. See the User's Guide
for details.

  {\bf How to fix files:}
  You must search for each use of "\pscustom", and make the following changes:
  \begin{itemize}
  \item Replace "\pscustom"'s old "(<x>,<y>)" argument by "\moveto(<x>,<y>)"
at the beginning of "\pscustom"'s main argument.

  \item Make the following substitutions inside "\pscustom"'s main argument:
    \begin{center}
    \begin{tabular}{ll}
    {\em Old} & {\em New}\\[2pt]
    "\pscode" & "\code"\\
    "\pscoor" & "\coor"\\
    "\psdim" & "\dim"\\
    "\psmove" & "\moveto"\\
    "\psclosepath" & "\closepath"\\
    "\psgroup{<stuff>}" & "\gsave" <stuff> "\grestore"\\
    "\psstroke" & "\stroke"\\
    "\psfill" & "\fill"
    \end{tabular}
    \end{center}
    You can instead define, e.g.,
    \begin{quote}
      "\def\pscode{\code}"  , or\\
      "\newcommand{\pscode}{\code}"
    \end{quote}
    and
    \begin{quote}
      "\def\psgroup#1{\gsave #1 \grestore}" , or\\
      "\newcommand{\psgroup}{\gsave #1 \grestore}"
    \end{quote}

  \item Check the output from your "\pscustom" command. If, after making the
above changes, things come out differently than inspected, it is probably
because the plot commands run ``forwards'' rather than ``backwards'', or
because of the new way that "\pscustom" treats the current point. Fix these on
a case-by-case basis.
  \end{itemize}

\item
  "\listplot", "\psplot" and "\parametricplot" no longer have an "(<x>,<y>)"
argument. Use the "origin" parameter instead.

  {\bf How to fix files:}
  Search for "\listplot", "\psplot" and "\parametricplot". Replace any
"(<x>,<y>)" argument by the parameter change "[origin={<x>,<y>}]".

\item
  To suppress labels with "\psaxes", use "labels=none/x/y", rather than
setting "Dx" and "Dy" to empty values.

 {\bf How to fix files:}
 Search for "=}", "=," and "=]". Remove the "Dx=," and other such parameter
settings you find, and replace instead by "labels=none/x/y", depending on
whether you want no labels, labels on the x-axes only, or labels on the y-axis
only.

\item
  The "\scalebox" macro should use a space rather than a comma to separate the
x and y scaling factors, when two scaling factors are given.

  {\bf How to fix files:}
  Wait to get error about bad numbers, or search for "\scalebox" and replace
"\scalebox{<num1>,<num2>}" by "\scalebox{<num1> <num2>}".

\item
  The "dblframewidth" parameter is gone, because "\psdblframebox" is now just
a variant of "\psframebox" with "doubleline=true". The width of each frame is
now just "linewidth"

  {\bf How to fix files:}
  Search for "dblframewidth" and replace the parameter setting be a
"linewidth" parameter setting.

\item
  The "\OldPsput" and "\Newpsput" commands are gone. These were originally
devised to retain compatibility with an older version of PSTricks that had a
"\psput" command instead of "\rput".

  {\bf How to fix files:}
  If you used "\psput" with the new syntax, then search for "\psput" and
replace by "\rput".

  If you used "\psput" with the old syntax, then either search for "\psput"
commands and replace with "\rput" commands with the new syntax, or put the
following in a file so that the "\OldPsput" command defines "\psput" with the
old syntax:
\begingroup\catcode`\<=12
\begin{LVerbatim}
  \def\old@psput{\begingroup\old@psput@}
  \def\old@psput@{%
    \def\refpoint@x{.5}\def\refpoint@y{.5}%
    \pst@ifstar{\@ifnextchar[%
      {\old@psput@i}{\def\pst@rot{}\old@psput@ii}}}
  \def\old@psput@i[#1]{\pst@getangle{#1}\pst@rot\old@psput@ii}
  \def\old@psput@ii{\@ifnextchar<{\old@psput@iii}{\old@psput@iv}}
  \def\old@psput@iii<#1>{\pst@@getref\old@psput@iv[#1]}
  \def\old@psput@iv{%
    \@ifnextchar({\end@psput\rput@i}{\end@psput\rput@i(0,0)}}
  \def\OldPsput{\let\psput\old@psput}
  \def\NewPsput{\let\psput\rput}
\end{LVerbatim}
\endgroup

\end{enumerate}


\subsection{New files with old stuff}

Remember to input these files when needed.  To have these files loaded
automatically, put an "\input" command after "\customization" in the
configuration file ("pstricks.con").

\begin{description}

\item[pst-node]
 All the node stuff has been put in "pst-node.tex" / "pst-node.sty".

\item[pst-plot]
  The plot commands ("\psplot", etc.) and the "\psaxes" command have been put
in "pst-plot.tex" / "pst-plot.sty". ("pst-plot.tex" automatically loads
"multido.tex", which is required by the axes macros.) There are also some
variants of "\listplot", "\fileplot" and "\dataplot", that are less likely to
exceed PostScript operand stack limits.

\item[colortab]
  The table coloring commands have been put in "colortab.tex" /
"colortab.sty". This is no longer PSTricks specific, but most of the old stuff
works the same as before. The only exception is that "\omit{}" is not needed
when a column is not to be colored. The documentation for "colortab.tex" is in
"colortab.doc". There are also some new features:
\begin{itemize}
\item
  "\SP" and "\RP":  These let "\LCC" ... "\ECC" work with nested arrays or
"\multicolumns" when using Mittelbach's array.sty. See "colortab.doc" for
details.

\item
  "LColors", "\LC", "\LCi", "\LCii", "\LCiii", "\LCz", for shading the cells
in the "longtable" environment. See "colortab.doc" for details.

umns when using Mittelbach's array.sty. See "colortab.doc" for details.

\item
  "LColors", "\LC", "\LCi", "\LCii", "\LCiii", "\LCz", for shading the cells
in the "longtable" environment. See "colortab.doc" for details.
\end{itemize}

\end{description}

\subsection{Obsolete but retained features}

 The features listed below are obsolete and are documented in footnotes.

\begin{itemize}
\item
  "\Polar":  Use "\SpecialCoor" and "(r;a)" instead.

\item
  "\Cartesian(x,y)"  Use "\psset{xunit=x,yunit=y}" instead.

\item
  "\Rput":  Use "\uput" instead.

\item
  "\Lput" and "\Mput":  Use "\aput", "\bput", "\Aput" and "\Bput" instead.

\end{itemize}

\subsection{New features}

This is a partial listing. See also the next section on new files.

\begin{itemize}

\item
  "\uput":  Replaces "\Rput".

\item
  "\aput", "\bput", "\Aput", "\Bput":  Replace "\Lput" and "\Mput".

\item
  "\clipbox" has optional argument "[<dim>]" that sets the clippath distance
<dim> from box. E.g., "\clipbox[1pt]{foo}".

\item
  "\newpsstyle": E.g.,
\begin{LVerbatim}
  \newpsstyle{foo}{linewidth=5pt,linestyle=dashed}
  \psline[style=foo](4,5)
\end{LVerbatim}

\item
  "\PSTricksOff" suppresses the PostScript. Useful for printing or previewing
drafts of your document with a non-PostScript driver.

\item
  New arrow styles:  {\catcode`\<=12 \tt
    >-<, <<->>, >>-<<. |*-|* }.

\item
  Many new features for the "\pscustom" command.

\item
  "dimen" parameter (for controlling whether dimensions for "\psframe",
"\pscircle", "\pswedge" and "\psellipse" refer to the inside, outside or
middle of the boundary.

\item
  "bordercolor" parameter.

\item
  "doubleline", "doublesep" and "doublecolor" parameters.

\item
  "ticks" and "labels" parameters, for suppressing ticks and labels with
"\psaxes".

\item
  "shadow" and "shadowangle" parameters.

\item
  Shadow parameters now apply to all graphics objects.

\item
  "\psmathboxtrue", "\psmathboxfalse", for controling whether box
    macros preserve math mode.

\item
  "\pslongbox": For making box environment out of box commands.

\item
  "\psverbboxtrue", "\psverbboxfalse": For controlling whether verbatim
    text is allowed in box commands.

\end{itemize}

\subsection{New files with new stuff}

\begin{description}

\item[pst-coil]
  Contains "\pscoil", "\psCoil", "\pszigzag", "\nccoil" and "\nczigzag". Uses
"pst-coil.pro" (optionally).

\item[pst2eps]
  Contains "\TeXtoEPS", to make it easier to convert \TeX\ boxes to EPS files
with dvips, and "\PSTtoEPS", for creating EPS files directly from PSTricks
graphics.

\item[textpath]
  Contains "\pstextpath" command, for typesetting text along a path. Use
"textpath.pro".

\item[gradient]
  "gradient" fillstyle. Uses "gradient.pro".

\item[charpath]
  Contains "\pscharpath" command, for stroking and filling character paths.
Also, "\pscharclip" ... "\endpscharclip" sets clipping path as well.

\item[piecharts.sh]
  A sh/awk script by Denis Girou for converting data to PSTricks piecharts.

\end{description}

\subsection{Bug fixes}

This list is incomplete.

\begin{itemize}

\item
  "\scalebox" and "\scaleboxto" now work when the vertical scaling factor
    is less than 1.

\item
  "\lput" and company now work with dvips 4.90 and later.

\item
  "\multips" can now be nested.

\item
  "\psclip" fixed.

\item
  "\clipbox" and clip option for "\pspicture" fixed.

\end{itemize}

\subsection{Other changes}

In some cases, there is a small chance these will require that you modify old
files.

\begin{itemize}

\item
  Specifying "\rput"'s <refpoint> argument as, e.g., "[.3,1]" rather than
"[br]" is now an undocumented feature.

\item
  All arguments to "\psplot", "\listplot" and "\parametric" plot are  passed
on directly as PostScript.

\item
  Dictionaries when including raw PostScript have changed. See appendix of
User's Guide for details.

\item
  "\psset{unit=dim}" always changes "\psunit", "\psxunit" and "\psyunit". To
change only "\psunit", use "runit=dim".

\item
  "border" parameter affects closed curves.

\item The "hatchsep" parameter now refers to the width of the space between
the lines, rather than the distance between the middle of the lines.
\end{itemize}

\renewcommand{\EveryVerbatimLine}{}
\renewcommand{\VerbatimFont}{\small\tt}
\renewcommand{\VerbatimFuzz}{2cm}

\section{VERSION 0.92}

\subsection{Incompatible changes}

\begin{Verbatim}
 ! V0.91 had two curve interpolation algorithsm: \pscurve and \psdoodle
   (and variants). These have been merged into a single algorithm retaining
   the names \pscurve, etc. Now the three curvature parameter has three
   numbers:
           num1 num2 num3
   When num3 is 0 (the default), you get the old \psdoodle algorithm, and
   and num1 and num2 act just like the old doodature parameter. When
   num3=-1, you get the old \pscurve algorithm, but positive values are
   usually nicer.

   If you have used \pscurve or its variants, the shape of the curves
   will change under 0.92. If you have also used the curvature
   parameter, you will get errors because the old curvature parameter is
   a single number and the new curvature parameter consists of 3 numbers.

   If you have used \psdoodle or its variants, then you can either search
   and replace doodle->curve and your use of doodature (which had 2 numbers)
   to curvature (which has 3 number), or you can put the following in the
   customization section of pstricks.con:
     \let\psdoodle\pscurve
     \let\psedoodle\psecurve
     \let\pscdoodle\psccurve
     \def\psset@doodature#1{\psset@curvature{#1 0}}

 ! In \psdblframebox: The inner frame now has \pslinewidth, and the outer
   frame now has width (dblframewidth x \pslinewidth), where dblframewidth is
   a new graphics parameter whose default value is 2.

 ! The angle, angleA and angleB parameters no longer apply to
   \ncarc, and the default values have been changed to 0. For
   \ncarc, the angle is now controlled by the arcangle, arcangleA
   and arcangleB parameters. The default is still 8.

 ! \multido has been off-loaded to the file multido.tex. The syntax has
   been changed to make it consistent with \psmultiput and LaTeX's
   \multiput: The variable declarations are now the first argument
   and the number of repetitions are now the second argument. Also,
   for number variables, the initial value and increment must now
   have the same number of digits to the right of the decimal,
   unless the initial value is an integer. There is no * version.
   Instead, the contents is never grouped, and there is even a variant
   that doesn't group the whole macro. See multido.ps, which is distributed
   with PSTricks, for details.

 ! 'diamond', 'diamond*', and 'x' dotstyles are gone. Use dotangle
   parameter instead.
\end{Verbatim}

\subsection{New features}

\begin{Verbatim}
 + New dot style: |.

 + New graphics objects: \pscustom.

 + New box framing macros: \psovalbox, \pscirclebox.

 + New box scaling macro: \scaleboxto.

 + New nodes: \ovalnode, \circlenode.

 + New node connection: \ncdiagg.

 + New loop macro: \multips.

 + New graphics parameters: dotscale, dotangle, dblframewidth.

 + New arrow styles: c and C.

 + \SpecialCoor allows using nodes as coordinates and mixing
   coordinates.
\end{Verbatim}

\subsection{Other changes}

\begin{Verbatim}
 * arrowscale parameter allows non-square scaling.

 * Documentation has been greatly improved.

 * A bug that caused problems with Arbortext's dvips (previously
   listed in pstricks.bug) was fixed.

 * Memory stats with LaTeX's article style:
     43290 words of memory
      2895 multiletter control sequences
\end{Verbatim}

\section{VERSION 0.91}

\subsection{Incompatible changes}

\begin{Verbatim}
 ! \dbox and \rotate eliminated.

 ! \psput replaced by \rput, with new syntax:
      OLD:  \psput[angle]<ref point>(x,y){stuff}
      NEW:  \rput[ref point]{angle}(x,y){stuff}
   \OldPsput sets up \psput with the old syntax (this may disappear some
   day; keep it in a safe place if you expect to need it for a long time).

 ! System for specifying angles for nested rotations with \psput has changed.
      OLD:  p0{angle}, p1{angle}, p2{angle}, etc.
      NEW:  *angle  works like p0{angle}; p1{angle} capability eliminated.
   N, S, E and W have same meaning as before.

 ! enddotsize parameter changed to dotsize.

 ! For specifying the origin as the baseline for the \pspicture
   environment, leave the optional argument [] empty (rather than
   [o]).

 ! B is not allowed as the y-coordinate when setting the reference point
   in \rput using coordinates. E.g., [Bl] and [B] are OK, but [.3,B] is
   not.

\end{Verbatim}

\subsection{New features}

\begin{Verbatim}
 + cornersize parameter added, for specifying whether radius of corners
   in \psframe and related box macros is given in relative terms (using
   framearc) or absolute terms (using linearc).

 + \psclip and \clipbox macros added.

 + \Cartesian and \Polar commands added, for switching coordinate
   systems.

 + border=dim parameter added, for giving appearance of one line
   crossing over another.

 + showpoints=bool parameter added. If true, a dot is placed at appropriate
   coordinates.

 + \psdots graphics object added.

 + \psarc graphics object added.

 + \parabola graphics object added.

 + \pscurve, \psccurve, \psecurve, \psdoodle, \psedoodle, and \pscdoodle
   graphics objects added.

 + \qline and \qdisk graphics objects added.

 + \psplot, \parametericplot and \listplot graphics objects added.

 + node connection (e.g., tree) macros added!!
\end{Verbatim}

\subsection{Other changes}

\begin{Verbatim}
 * \sunpatch not needed anymore.

 * PostScript header file (if being used) is included at the beginning,
   rather than on demand.

 * In spite of the many new features, the input file is smaller in bytes,
   and runs significantly faster (at least if used with a header file).
   The macros use up slightly more words of memory, and significantly
   more command sequences. Here are the stats with LaTeX's article
   style:
     41586 words of memory
      2966 multiletter control sequences
\end{Verbatim}

\section{VERSION 0.9}

Much internal code was changed, and additional features were added.

\end{document}
%% END changes.tex
