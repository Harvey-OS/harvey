% docultex.tex 12-11-92, Documentation for siamltex macro package
%

\documentstyle[twoside]{siamltex}

\title{USING SIAM'S \LaTeX\ MACROS\thanks{This work was
supported by the Society for Industrial and Applied
Mathematics}} 

\author{Paul Duggan\thanks{Society for Industrial and
Applied Mathematics, Philadelphia, Pennsylvania. 
({\tt duggan@siam. org}). Questions, comments, or corrections
to this document may be directed to that email address.}}


\begin{document}
\maketitle

\section{Introduction}

This file is documentation for the SIAM \LaTeX\ macros, and
provides instruction for submission of your files.

To accommodate authors who electronically typeset their manuscripts,
SIAM supports the use of \TeX. To ensure quality typesetting according
to SIAM style standards, SIAM provides a \TeX\ macro style file.
Using \TeX\ to format a manuscript should simplify the editorial process
and lessen the author's proofreading burden. It is still necessary to 
proofread with care.

Electronic files should not be submitted until the paper has been
accepted, and then not until requested to do so by someone in the SIAM
office. Once an article is slated for an issue,
someone from the SIAM office will contact the author about any or all
of the following: editorial and stylistic queries; 
supplying the source files (and any supplementary macros)
for the properly formatted article; and handling figures.

Electronic submissions (to {\tt tex@siam.org}) should be clearly
designated as to the journal and author. Authors are responsible for ensuring 
that the paper generated from the source files exactly matches the paper that
was accepted for publication by the review editor. If it does not,
that should be indicated in the transmission of the file.
When submitting a file, please be sure to include any additional
macros (other than those provided by SIAM) that will be needed to run
the paper.

Once the files are corrected here at SIAM, we will send back the revised 
proofs to be read against the original edited manuscript. We are not
set up to shuttle back and forth varying electronic versions of each
paper, so we must rely on hard copy. The author's proofreading
is an important but easily overlooked step. Even if SIAM were not
to introduce a single editorial change into your manuscript, there
would still be a need to check line and page breaks as the encoding
of Times Roman fonts would change the page makeup of a Computer
Modern file.
 

A sample file is included with this distribution to
demonstrate the standard use of SIAM's macro package. To
provide an incentive for the use of \TeX\ in paper
preparation, SIAM provides 100 free reprints of journal
articles but only to those who have made use of the SIAM
macro in proper fashion. To qualify for free reprints the
following criteria must be met:
\begin{remunerate}
\item The appropriate document style line must appear at
the beginning of the source file.
\item All top matter information (key words, affiliation, etc.) must
be present and correctly tagged. The only exception to the rule will be
the AMS subject classification.
\item All tagging conventions, as described in this documentation
and used in the example file, must be followed.
\item The format of the bibliography must strictly adhere to SIAM
guidelines. The appropriate tags must be used for all entries. All names
are to be keyed initial upper case cap and small caps. Only the first
and middle initials, followed by the last name, are to be used.
Last names should never be listed first. There will be no exceptions
to this requirement. An improperly prepared bibliography will,
in all cases, prevent an author from receiving free reprints
\end{remunerate}


The distribution contains the following items: {\tt
siamltex.sty}, the main macro package, based on {\tt
article.sty}; {\tt siam10.sty} and {\tt siam11.sty}, files
included by {\tt siamltex} depending on the point
size of the document; {\tt numinsec.sty}, a file
useable at the author's option that causes equation numbers, 
theorem-environment numbering, and figure and table numbering 
to be reset with the start of each section. Equation numbers shift from
single digit form (n) to decimal form (n.m). {\tt subeqn.sty}
is another style option for equation numbering (see \S3 for
an explanation) {\tt siam.bst}
is the style file for use with Bib\TeX. {\tt fixup.sty}, a 
separate file in previous distributions, has been
incorporated into {\tt siamltex}. Also included are this
file {\tt docultex.tex} and a sample file {\tt lexample.tex}.

The sample file is representative of the standard way to
apply the macros. The rest of this paper will emphasize
some aspects of this, as well as point out options and
special cases, and describe SIAM style standards for
authors to conform to.


\section{Headings}
The top matter of a journal paper falls into a standard
format. It begins of course with the \verb|\documentstyle| command

\begin{verbatim}

\documentstyle[twoside]{siamltex}

\end{verbatim}

Other style options (such as {\tt numinsec}) can be included
in the bracketed argument of the command, separated by commas.
SIAM publishes articles at 10 point size, and {\tt siamltex}
will do so automaticlly. If a preliminary version
at 11 point size is desired, put {\tt siam11} in the optional
argument. But when the file is submitted it must be back in
10 point size.

The title and author parts are formatted using the
\verb|\title| and \verb|\author| commands. The \verb|\date|
command is not used. \verb|\maketitle| produces the actual
output of the commands.

The addresses and support acknowledgments are put into the
\verb|\author| commands via \verb|\thanks|. If support is
overall for the authors, the support acknowledgment should
be put in a \verb|\thanks| command in the \verb|\title|.
Specific support should go following the addresses of the
individual authors in the same \verb|\thanks| command.

Sometimes authors have support or addresses in common which
necessitates having multiple \verb|\thanks| commands for
each author. Unfortunately \LaTeX\ does not normally allow this, 
so a special procedure must be used. An example of this procedure
follows. Grant information can also be run into both authors' 
footnotes.

\begin{verbatim}

\title{TITLE OF PAPER}

\author{A.~U. Thorone\footnotemark[2]\ \footnotemark[5]
\and A.~U. Thortwo\footnotemark[3]\ \footnotemark[5]
\and A.~U. Thorthree\footnotemark[4]}

\begin{document}
\maketitle

\renewcommand{\thefootnote}{\fnsymbol{footnote}}

\footnotetext[2]{Address of A.~U. Thorone}
\footnotetext[3]{Address of A.~U. Thortwo}
\footnotetext[4]{Address of A.~U. Thorthree}
\footnotetext[5]{Support in common for the first and second
authors.}

\renewcommand{\thefootnote}{\arabic{footnote}}

\end{verbatim}

Notice that the footnote marks begin with {\tt [2]}
because the first mark (the asterisk) will be used in the
title for date-received information by SIAM, even if not
already used for support data. This is just one example;
other situations follow a similar pattern.

Following the author and title is the abstract, key words
listing, and AMS subject classification number(s),
designated using the \verb|{abstract}|, \verb|{keywords}|,
and \verb|{AMS}| environments. If
there is only one AMS number, the commands
\verb|\begin{AM}| and \verb|\end{AM}| are used
instead of \verb|{AMS}|. This causes the heading to be
in the singular. Authors are responsible for providing AMS numbers.
They can be found in the Annual Index of Math Reviews, or
through {\tt e-Math} ({\tt telnet@e-math.ams.com}; login
and password are both {\tt e-math}).   

Left and right running heads should be provided in the
following way.

\begin{verbatim}

\pagestyle{myheadings}
\thispagestyle{plain}
\markboth{A.~U. THORONE AND A.~U. THORTWO}{SHORTER PAPER
TITLE} 

\end{verbatim}

\section{Equations}

One advantage of \LaTeX\ is that it can automatically number
equations and refer to these equation numbers in text. While plain \TeX's
method of equation numbering (explicit numbering using
\verb|\leqno|) works in the SIAM macro, it is not preferred
except in certain cases. SIAM style guidelines call for
aligned equations in many circumstances, and \LaTeX's 
\verb|{eqnarray}| environment is not compatible with
\verb|\leqno| and \LaTeX\ is not compatible with the plain
\TeX\ command \verb|\eqalign| and \verb|\leqalignno|. Since
SIAM may have to alter or make aligned certain groups of
equations, it is necessary to use the \LaTeX\ system of
automatic numbering.

Sometimes it is desirable to designate subequations of a larger
equation number. The subequations are designated with
(roman font) letters appended after the number. SIAM has
supplemented its macros with the {\tt subeqn.sty} style which
defines the environment \verb|{subequations}|.

\begin{verbatim}

\begin{subequations}
\begin{equation}
 y_k =  B  y_{k-1} +  f, \qquad k=1,2,3,\ldots  \label{EXk}
\end{equation}
for  any initial vector $ y_0$.   Then 
\begin{equation}
 y_k\rightarrow  u \mbox{\quad iff\quad} \rho( B)<1.
\end{equation}
\end{subequations}

\end{verbatim}

All equations within the \verb|{subequations}| environment
will keep the same overall number, but the letter
designation will increase.


\section{Special fonts}

SIAM supports the use of the AMS-\TeX\ fonts (version 2.0
and later). As described in the manual for these fonts,
they can be included by \verb|%  AMSSYM.DEF                                           September 1990
\def\fileversion{v1.0b}

%  This file contains definitions that perform the same functions as similar
%  ones in AMS-TeX, so that the file AMSSYM.TEX can be used outside of AMS-TeX.
%
%  American Mathematical Society, Technical Support Group, P. O. Box 6248,
%        Providence, RI 02940
%  800-321-4AMS or 401-455-4080;  Internet: Tech-Support@Math.AMS.com
%
%  Copyright (C) 1990, American Mathematical Society.
%  All rights reserved.  Copying of this file is authorized only if either:
%       (1) you make absolutely no changes to your copy including name; OR
%       (2) if you do make changes, you first rename it to some other name.
%
%  Instructions for using this file and the AMS symbol fonts are included in
%  the AMSFonts 2.0 User's Guide.
%
%%%%%%%%%%%%%%%%%%%%%%%%%%%%%%%%%%%%%%%%%%%%%%%%%%%%%%%%%%%%%%%%%%%%%%%%

%  Store the catcode of the @ in the csname so that it can be restored later.
\expandafter\chardef\csname pre amssym.def at\endcsname=\the\catcode`\@
%  Set the catcode to 11 for use in private control sequence names.
\catcode`\@=11

%  Include all definitions related to the fonts msam, msbm and eufm, so that
%  when this file is used by itself, the results with respect to those fonts
%  are equivalent to what they would have been using AMS-TeX.
%  Most symbols in fonts msam and msbm are defined using \newsymbol;
%  however, a few symbols that replace composites defined in plain must be
%  defined with \mathchardef.

\def\undefine#1{\let#1\undefined}
\def\newsymbol#1#2#3#4#5{\let\next@\relax
 \ifnum#2=\@ne\let\next@\msafam@\else
 \ifnum#2=\tw@\let\next@\msbfam@\fi\fi
 \mathchardef#1="#3\next@#4#5}
\def\mathhexbox@#1#2#3{\relax
 \ifmmode\mathpalette{}{\m@th\mathchar"#1#2#3}%
 \else\leavevmode\hbox{$\m@th\mathchar"#1#2#3$}\fi}
\def\hexnumber@#1{\ifcase#1 0\or 1\or 2\or 3\or 4\or 5\or 6\or 7\or 8\or
 9\or A\or B\or C\or D\or E\or F\fi}

\font\tenmsa=msam10
\font\sevenmsa=msam7
\font\fivemsa=msam5
\newfam\msafam
\textfont\msafam=\tenmsa
\scriptfont\msafam=\sevenmsa
\scriptscriptfont\msafam=\fivemsa
\edef\msafam@{\hexnumber@\msafam}
\mathchardef\dabar@"0\msafam@39
\def\dashrightarrow{\mathrel{\dabar@\dabar@\mathchar"0\msafam@4B}}
\def\dashleftarrow{\mathrel{\mathchar"0\msafam@4C\dabar@\dabar@}}
\let\dasharrow\dashrightarrow
\def\ulcorner{\delimiter"4\msafam@70\msafam@70 }
\def\urcorner{\delimiter"5\msafam@71\msafam@71 }
\def\llcorner{\delimiter"4\msafam@78\msafam@78 }
\def\lrcorner{\delimiter"5\msafam@79\msafam@79 }
\def\yen{{\mathhexbox@\msafam@55 }}
\def\checkmark{{\mathhexbox@\msafam@58 }}
\def\circledR{{\mathhexbox@\msafam@72 }}
\def\maltese{{\mathhexbox@\msafam@7A }}

\font\tenmsb=msbm10
\font\sevenmsb=msbm7
\font\fivemsb=msbm5
\newfam\msbfam
\textfont\msbfam=\tenmsb
\scriptfont\msbfam=\sevenmsb
\scriptscriptfont\msbfam=\fivemsb
\edef\msbfam@{\hexnumber@\msbfam}
\def\Bbb#1{\fam\msbfam\relax#1}
\def\widehat#1{\setbox\z@\hbox{$\m@th#1$}%
 \ifdim\wd\z@>\tw@ em\mathaccent"0\msbfam@5B{#1}%
 \else\mathaccent"0362{#1}\fi}
\def\widetilde#1{\setbox\z@\hbox{$\m@th#1$}%
 \ifdim\wd\z@>\tw@ em\mathaccent"0\msbfam@5D{#1}%
 \else\mathaccent"0365{#1}\fi}
\font\teneufm=eufm10
\font\seveneufm=eufm7
\font\fiveeufm=eufm5
\newfam\eufmfam
\textfont\eufmfam=\teneufm
\scriptfont\eufmfam=\seveneufm
\scriptscriptfont\eufmfam=\fiveeufm
\def\frak#1{{\fam\eufmfam\relax#1}}
\let\goth\frak

%  Restore the catcode value for @ that was previously saved.
\catcode`\@=\csname pre amssym.def at\endcsname

\endinput
| and
\verb|%  AMSSYM.DEF                                           September 1990
\def\fileversion{v1.0b}

%  This file contains definitions that perform the same functions as similar
%  ones in AMS-TeX, so that the file AMSSYM.TEX can be used outside of AMS-TeX.
%
%  American Mathematical Society, Technical Support Group, P. O. Box 6248,
%        Providence, RI 02940
%  800-321-4AMS or 401-455-4080;  Internet: Tech-Support@Math.AMS.com
%
%  Copyright (C) 1990, American Mathematical Society.
%  All rights reserved.  Copying of this file is authorized only if either:
%       (1) you make absolutely no changes to your copy including name; OR
%       (2) if you do make changes, you first rename it to some other name.
%
%  Instructions for using this file and the AMS symbol fonts are included in
%  the AMSFonts 2.0 User's Guide.
%
%%%%%%%%%%%%%%%%%%%%%%%%%%%%%%%%%%%%%%%%%%%%%%%%%%%%%%%%%%%%%%%%%%%%%%%%

%  Store the catcode of the @ in the csname so that it can be restored later.
\expandafter\chardef\csname pre amssym.def at\endcsname=\the\catcode`\@
%  Set the catcode to 11 for use in private control sequence names.
\catcode`\@=11

%  Include all definitions related to the fonts msam, msbm and eufm, so that
%  when this file is used by itself, the results with respect to those fonts
%  are equivalent to what they would have been using AMS-TeX.
%  Most symbols in fonts msam and msbm are defined using \newsymbol;
%  however, a few symbols that replace composites defined in plain must be
%  defined with \mathchardef.

\def\undefine#1{\let#1\undefined}
\def\newsymbol#1#2#3#4#5{\let\next@\relax
 \ifnum#2=\@ne\let\next@\msafam@\else
 \ifnum#2=\tw@\let\next@\msbfam@\fi\fi
 \mathchardef#1="#3\next@#4#5}
\def\mathhexbox@#1#2#3{\relax
 \ifmmode\mathpalette{}{\m@th\mathchar"#1#2#3}%
 \else\leavevmode\hbox{$\m@th\mathchar"#1#2#3$}\fi}
\def\hexnumber@#1{\ifcase#1 0\or 1\or 2\or 3\or 4\or 5\or 6\or 7\or 8\or
 9\or A\or B\or C\or D\or E\or F\fi}

\font\tenmsa=msam10
\font\sevenmsa=msam7
\font\fivemsa=msam5
\newfam\msafam
\textfont\msafam=\tenmsa
\scriptfont\msafam=\sevenmsa
\scriptscriptfont\msafam=\fivemsa
\edef\msafam@{\hexnumber@\msafam}
\mathchardef\dabar@"0\msafam@39
\def\dashrightarrow{\mathrel{\dabar@\dabar@\mathchar"0\msafam@4B}}
\def\dashleftarrow{\mathrel{\mathchar"0\msafam@4C\dabar@\dabar@}}
\let\dasharrow\dashrightarrow
\def\ulcorner{\delimiter"4\msafam@70\msafam@70 }
\def\urcorner{\delimiter"5\msafam@71\msafam@71 }
\def\llcorner{\delimiter"4\msafam@78\msafam@78 }
\def\lrcorner{\delimiter"5\msafam@79\msafam@79 }
\def\yen{{\mathhexbox@\msafam@55 }}
\def\checkmark{{\mathhexbox@\msafam@58 }}
\def\circledR{{\mathhexbox@\msafam@72 }}
\def\maltese{{\mathhexbox@\msafam@7A }}

\font\tenmsb=msbm10
\font\sevenmsb=msbm7
\font\fivemsb=msbm5
\newfam\msbfam
\textfont\msbfam=\tenmsb
\scriptfont\msbfam=\sevenmsb
\scriptscriptfont\msbfam=\fivemsb
\edef\msbfam@{\hexnumber@\msbfam}
\def\Bbb#1{\fam\msbfam\relax#1}
\def\widehat#1{\setbox\z@\hbox{$\m@th#1$}%
 \ifdim\wd\z@>\tw@ em\mathaccent"0\msbfam@5B{#1}%
 \else\mathaccent"0362{#1}\fi}
\def\widetilde#1{\setbox\z@\hbox{$\m@th#1$}%
 \ifdim\wd\z@>\tw@ em\mathaccent"0\msbfam@5D{#1}%
 \else\mathaccent"0365{#1}\fi}
\font\teneufm=eufm10
\font\seveneufm=eufm7
\font\fiveeufm=eufm5
\newfam\eufmfam
\textfont\eufmfam=\teneufm
\scriptfont\eufmfam=\seveneufm
\scriptscriptfont\eufmfam=\fiveeufm
\def\frak#1{{\fam\eufmfam\relax#1}}
\let\goth\frak

%  Restore the catcode value for @ that was previously saved.
\catcode`\@=\csname pre amssym.def at\endcsname

\endinput
|. The blackboard bold font in this
font package is can be used for designating number sets. 
This is preferable to other methods of combining letters
(such as I and R for the real numbers) to produce pseudo-bold
letters but this is tolerable as well. Typographicly speaking,
number sets may simply be designated using regular bold letters; 
the blackboard bold typeface was designed in response to a desire
to simulate the limitations of a chalkboard in type.

SIAM's style macros are not yet updated to make full use of the New
Font Selection Scheme (NFSS) of Mittelbach and Sch\"opf. The macros
are generally compatible with the NFSS if using the {\tt oldlfont}
style option. {\bf Note:} lines 339, 340, 348, 349, and 351 of
{\tt siamltex.sty} need to be commented out to run under the
NFSS. Those lines add additional font capability in a manner
incompatible with and redundant to the NFSS. Line 351 calls
in the {\tt cmcsc8} font, which is unanvailable in some installations.
The small caps font, along with other AMS fonts, is available from the 
American Mathematical Society.


\subsection{Punctuation}
All standard punctuation and all numerals should be set in Roman type
(upright) whether bold or plain. The only exceptions are periods and 
commas. They may be set to match the surrounding text.

\subsection{Text formatting}

SIAM style preferences do not make regular use of the \verb|{enumerate}|
and \verb|{itemize}| environments. Instead,
{\tt siamltex.sty} includes definitions of two alternate list
environments, \verb|{remunerate}| and \verb|{romannum}|.
Unlike the standard itemized lists, these environments do
not indent the secondary lines of text. The labels, whether
the defaults or optional user-defined, are always aligned
flush right.

The \verb|{remunerate}| environment consecutively numbers
each item with an arabic numeral followed by a period. This
number is always upright, even in slanted
environments.\footnote{For those wondering at the unusual
naming of this environment, it comes from Seroul and Levy's
\cite{SerLev} definition of a similar macro for plain \TeX: 
{\tt \char"5C meti} which is {\tt \char"5C item} spelled backwards. Thus
\verb|{remunerate}|, a portion of \verb|{enumerate}|
spelled backwards.}

The \verb|{romannum}| environment consecutively numbers
each item with a lower-case roman numeral enclosed in
parentheses. This number will always be upright within
slanted environments (as in theorems).


\section{Theorems and Lemmas}
Theorems, lemmas, corollaries, and propositions are covered
in the SIAM macros by the theorem-environments
\verb|{theorem}|, \verb|{lemma}|, \verb|{corollary}|, and
\verb|{proposition}|. These are all numbered in the same
sequence and produce labels in small caps with an italic
body. Other environments may be specified by the
\verb|\newtheorem| command.

Proofs are handled with the \verb|\begin{proof}|
\verb|\end{proof}| environment. A ``QED'' box is created
automatically by \verb|\end{proof}|, but this should be
preceded with a \verb|\qquad|.

Named proofs, if used, must be done independently by the
authors. SIAM style specifies that proofs which end with
displayed equations should have the QED box on line with
the equation flush right. Below is an example of how this
can be done:

\begin{verbatim}

{\em Proof}. Proof of the previous theorem 
                .
                .
                .
thus,
$$
a^2 + b^2 = c^2 \eqno\endproof
$$

\end{verbatim}

Note that the above will not work if the equation is in an
\verb|{eqnarray}| or numbered. In those instances, just include the
\verb|\endproof| command and SIAM will handle making the box
flush right in production. 


\section{Figures and tables}
Figures and tables sometimes require special consideration.
Tables in SIAM style should be in eight point size, and
written so that they do not extend beyond the text margins. Use
\verb|\footnotesize| in the body of the table to get eight
point type.

SIAM style requires that no figures or tables  appear in the
references section of the paper. \LaTeX\ is notorious for
making figure placement difficult, so it is important to
pay particular attention to figure placement near the
references in the text. All figures and tables should
be referred to in the text.

SIAM supports the use of {\tt psfig} for including {\sc PostScript}
figures. All {\sc Post\-Script} figures to be included should be sent in
separate files. See the {\tt psfig} documentation for more
details on the use of this style option. It is a good idea
to submit hardcopy of all {\sc PostScript} figures just in case
there is difficulty in reproducing the figures.

Hardcopy for non-{\sc PostScript} figures should be included in
the submission of the hardcopy of the manuscript. Space
should be left in the \verb|{figure}| command for the
hardcopy to be inserted in production.

\section{Bibliography and Bib\TeX}

If using Bib\TeX, authors need not submit the {\tt .bib} file for
their papers. Merely submit the completed {\tt .bbl} file, having used
{\tt siam.bst} as their bibliographic style file. {\tt siam.bst}
only works with Bib\TeX\ version 99i and later. The use of
Bib\TeX\ and the preparation of a {\tt .bib} file is
described in greater detail in \cite{Lamport}.

If not using Bib\TeX, SIAM bibliographic references follow
the format of the following examples:

\begin{verbatim}

\bibitem{AuTh1} {\sc A.~U. Thorone}, {\em Title of paper
with lower case letters}, SIAM J. Abbrev. Correctly, 2
(1992), pp.~000--000.

\bibitem{A1A2} {\sc A.~U. Thorone and A.~U. Thortwo}, {\it
Title of paper appearing in book}, in Book Title: With All
Initial Caps, Publisher, Location, 1992.

\bibitem{A1A22} \sameauthor, % generates the 3 em rule
{\em Title of Book{\rm :} Note Initial Caps and {\rm ROMAN
TYPE} for Punctuation and Acronyms}, Publisher,
Location, pp.~000--000, 1992.

\bibitem{AuTh3} {\sc A.~U. Thorthree}, {\em Title of paper
that's not published yet}, SIAM. J. Abbrev. Correctly, to appear.

\end{verbatim}

Other types of references fall into the same general
pattern. See the sample file or any SIAM journal for other
examples. Authors must correctly format their bibliography to
be considered as having used the macros correctly. An incorrectly 
formatted bibliography is not only time-consuming but very
likely to have errors introduced into it by keyboarders/copy editors.

As an alternative to the above style of reference, an alphanumeric
code may be used in place of the number (e.g., [AUTh90]). The same
commands are used, but \verb|\bibitem| takes an optional argument
containing the desired alphanumeric code. 

Another alternative is no number, simply the authors' names and
the year of publication following in parentheses. The rest of the
format is identical. The macros do not support this alternative
directly, but modifications to the macro definition are possible
if this reference style is preferred.



\begin{thebibliography}{1}
\bibitem{Lamport} {\sc L. Lamport}, \LaTeX: {\em A Document
Preparation System}, Addison Wesley, Reading, MA, 1986.

\bibitem{SerLev} {\sc R. Seroul and S. Levy}, {\em A
Beginner's Book of} \TeX, Springer-Verlag, Berlin, New
York, 1991.
\end{thebibliography}


\end{document}
