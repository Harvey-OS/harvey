\documentclass{article}
\usepackage{multicol}
\usepackage{float}
\usepackage{makeidx}
\makeindex
\title{Page headers and footers in \LaTeX}
\author{Piet van Oostrum\thanks{A considerable part of this article was
    written by  George
    Gr\"atzer (University of Manitoba) in \emph{Notices Amer. Math. Soc.}
    Thanks, George!}\\
  Dept.\ of Computer Science\\
  Utrecht University}
\def\latex/{\protect\LaTeX{}}
\def\tex/{\TeX}
\def\ams/{\protect\pAmS}
\def\pAmS{{\the\textfont2
        A\kern-.1667em\lower.5ex\hbox{M}\kern-.125emS}}
\def\amslatex/{\ams/-\latex/}
\newcommand{\PSNFSS}{{\sf
    PSNFSS}}
\newcommand{\bs}{\symbol{'134}}
\newcommand{\CmdIndex}[1]{\index{#1@\texttt{\bs#1}}}
\newcommand{\TTindex}[1]{\index{#1@\texttt{#1}}}
\newcommand{\PSindex}[1]{\index{page style!#1@\texttt{#1}}}
\floatstyle{ruled}
\restylefloat{figure}
\renewcommand{\topfraction}{0.9}
\renewcommand{\bottomfraction}{0.9}
\renewcommand{\textfraction}{0.05}
\makeatletter
\renewcommand\l@section      {\@dottedtocline{1}{1.5em}{2.3em}}
\makeatother
\begin{document}
\maketitle
\begin{abstract}
  Standard \latex/ offers the page styles \textsf{empty}, \textsf{plain},
  \textsf{headings} and \textsf{myheadings}. For simple page layouts these
  suffice, but they are rather rigid. A more flexible page layout may be
  obtained by using the  \textsf{fancyheadings} package.
  This article describes how to create advanced page headers and footers
  in \latex/ documents by using this package. Some of the techniques
  described are also applicable with other page styles.
\end{abstract}
\tableofcontents

\section{What is \textsf{fancyheadings}}
The \textsf{fancyheadings} macro package allows you to customize
in \latex/ your page headers and footers in an easy way.  You can
define:
\begin{itemize}
\item three-part headers and footers
\item decorative lines in headers and footers
\item headers and footers wider than the width of the text
\item multi-line headers and footers
\item separate headers and footers for even and odd pages
\item different headers and footers for chapter pages
\end{itemize}

Of course, you also have complete control over fonts, uppercase
and lowercase displays, etc.

\section{Where to get it?}\label{get}
You only need the file \verb|fancyheadings.sty| which you can find at the
\index{ftp}
CTAN sites: \verb|ftp.shsu.edu| (U.S.), \verb|ftp.tex.ac.uk| (U.K.), and
\verb|ftp.dante.de| (Germany).
You will find it in the directory\\
\verb|/tex-archive/macros/latex209/contrib/fancyheadings|. Although
\textsf{fancyheadings}
is a \latex/ 2.09 style file, it will work with \LaTeXe{}. Also this
\TTindex{fixmarks.sty}
\TTindex{extramarks.sty}
article and the two packages \texttt{fixmarks} and \texttt{extramarks} can
be found there. (The code for these packages given in this article is a
simplified version.)


\section{Use of \textsf{fancyheadings}}  To use this package in a
\LaTeXe\ document, place the file \verb|fancyheadings.sty| in a
directory/folder where \tex/ can find it (normally in the input
directory/folder), and include in the preamble of your document
after

\begin{verbatim}
\documentclass{...}
\end{verbatim}
the commands:

\PSindex{fancy}
\begin{verbatim}
\usepackage{fancyheadings}
\pagestyle{fancy}
\end{verbatim}

We can visualize the page layout we can create with \textsf{fancyheadings}
as follows:

\begin{samepage}
\vspace{8pt}
\noindent\makebox[\textwidth]{LeftHeader\hfill
CenteredHeader\hfill RightHeader}
\noindent\makebox[\textwidth]{\hrulefill}\\[\baselineskip]
\noindent\makebox[\textwidth]{\hfill page body\hfill}\\[\baselineskip]
\noindent\makebox[\textwidth]{\hrulefill}
\noindent\makebox[\textwidth]{LeftFooter\hfill
CenteredFooter\hfill RightFooter}

\end{samepage}


\vspace{8pt}

The LeftHeader and LeftFooter are left justified; the
CenteredHeader and CenteredFooter are centered; the
RightHeader and RightFooter are right justified.

We define each of the six ``fields'' and the two decorative lines
separately.

\section{A simple example}  K. Grant is writing a report to Dean
A. Smith, on ``The performance of new graduates'' with the
following page layout:

\begin{samepage}
\vspace{8pt}
\noindent\makebox[\textwidth]{\hfill\bfseries The performance of new
graduates}
\noindent\makebox[\textwidth]{\hrulefill}\\[\baselineskip]
\noindent\makebox[\textwidth]{\hfill page body\hfill}\\[\baselineskip]
\noindent\makebox[\textwidth]{\hrulefill}
\noindent\makebox[\textwidth]{From: K. Grant\phantom{3}\hfill
To: Dean A. Smith\hfill \phantom{From: K. Grant}3}

\end{samepage}

\vspace{8pt}

\noindent where ``3'' is the page number.  The title: ``The
performance of new graduates'' is bold.

This is accomplished by these commands following
\verb|\pagestyle{fancy}|:
\CmdIndex{lhead}
\CmdIndex{rhead}
\CmdIndex{chead}
\CmdIndex{lfoot}
\CmdIndex{rfoot}
\CmdIndex{cfoot}
\CmdIndex{headrulewidth}
\CmdIndex{footrulewidth}
\begin{verbatim}
\lhead{}
\chead{}
\rhead{\bfseries The performance of new graduates}
\lfoot{From: K. Grant}
\cfoot{To: Dean A. Smith}
\rfoot{\thepage}
\setlength{\headrulewidth}{0.4pt}
\setlength{\footrulewidth}{0.4pt}
\end{verbatim}
(The \verb|\thepage| macro displays the current page number.
\verb|\bfseries| is the \LaTeXe's way of selecting bold face.)

This is now fine, except that the first page does not need
all these headers and footers.  To eliminate all but the
centered page number, issue the command

\CmdIndex{thispagestyle}
\begin{verbatim}
\thispagestyle{plain}
\end{verbatim}
after the
\verb|\begin{document}|
and the
\CmdIndex{maketitle}
\verb|\maketitle|
commands.

Alternatively, issue
\begin{verbatim}
\thispagestyle{empty}
\end{verbatim}
if you do not want any headers or footers.

In fact the standard \latex/ classes have the command \verb|\maketitle|
defined in such a way that a \verb|\thispagestyle{plain}| is automatically
issued. So if you \emph{do} want the fancy layout on a page containing
\verb|\maketitle| you must issue a \verb|\thispagestyle{fancy}| after the
\verb|\maketitle|.

\section{An example of two-sided printing}\label{two-sided}

\TTindex{twoside}
Some document classes, such as \verb|book.cls|, print two-sided: the even
pages and the odd pages have different layouts; other document classes
use the \verb|twoside| option to print two-sided.

Now let us print the report two-sided.  Let the above page
layout be used for the odd (right-side) pages, and the
following for the even (left-side) pages:

\begin{samepage}

\vspace{8pt}
\noindent\makebox[\textwidth]{\bfseries The performance of new
graduates\hfill}
\noindent\makebox[\textwidth]{\hrulefill}\\[\baselineskip]
\noindent\makebox[\textwidth]{\hfill page body\hfill}\\[\baselineskip]
\noindent\makebox[\textwidth]{\hrulefill}
\noindent\makebox[\textwidth]{4\phantom{To: Dean A. Smith}\hfill
 From: K. Grant\hfill \phantom{4}To: Dean A. Smith}

\vspace{8pt}

\end{samepage}
\noindent where ``4'' is the page number.

Here are the commands:

\begin{verbatim}
\lhead[]{\bfseries The performance of new graduates}
\chead{}
\rhead[\bfseries The performance of new graduates]{}
\lfoot[\thepage]{From: K. Grant}
\cfoot[From: K. Grant]{To: Dean A. Smith}
\rfoot[To: Dean A. Smith]{\thepage}
\setlength{\headrulewidth}{0.4pt}
\setlength{\footrulewidth}{0.4pt}
\end{verbatim}

As you see, the values of the fields for the even (left-side)
pages are given in the square brackets.  Again, you may
use  \verb|\thispagestyle{plain}| for a simple page layout for
page~1.

\section{Redefining plain style}

Some \latex/ commands, like \verb|\chapter|, use the \verb|\thispagestyle|
command to automatically switch to the \texttt{plain} page style, thus
ignoring the page style currently in effect. To customize even such
pages use the \texttt{fancyplain} page style.  This page style sets up
\texttt{fancy} for normal pages and in addition redefines the page style
\texttt{plain} to conform more to the page style \texttt{fancy}.
\PSindex{plain}
\PSindex{fancyplain}
For instance, if
all the page numbers are bold, you may want this on the first page also.
You do this with the command

\PSindex{fancyplain}
\begin{verbatim}
\pagestyle{fancyplain}
\end{verbatim}
\noindent
Note that this does not issue a \verb|\pagestyle{plain}| or
\verb|\thispagestyle{plain}| command.  This is either done by \latex/ (for
instance, for the first page of a chapter in the \verb|book| class), or you
have to do it explicitly with the

\begin{verbatim}
\thispagestyle{plain}
\end{verbatim}
command.

The redefinition of the plain style is done with the command

\begin{verbatim}
\fancyplain{plain_value}{normal_value}
\end{verbatim}
inserted into all the appropriate arguments.  As an example, let us
redefine the plain style for the report in Section~\ref{two-sided} by
making the page number bold.

The leftheader definition:

\begin{verbatim}
\lhead[]{\bfseries The performance of new graduates}
\end{verbatim}
changes to

\begin{verbatim}
\lhead[\fancyplain{}{}]{\fancyplain{}{\bfseries The
   performance of new graduates}}
\end{verbatim}
Note that the first argument of both \verb|\fancyplain| commands is
empty, since the redefined plain style has an empty leftheader. (The
\verb|\fancyplain{}{}| is superfluous but for symmetry reasons it is given
in this example.)

The left- and rightfooter definitions change more.  For instance, the
leftfooter definition changes from

\begin{verbatim}
\lfoot[\thepage]{From: K. Grant}
\end{verbatim}
to
\begin{verbatim}
\lfoot[\fancyplain{}{\bfseries \thepage}]
      {\fancyplain{}{From: K. Grant}}
\end{verbatim}

Carrying out these changes for all six fields, we get the code:

\label{complicated}
\begin{verbatim}
\lhead[\fancyplain{}{}]{\fancyplain{}{\bfseries The
   performance of new graduates}}
\chead{\fancyplain{}{}}
\rhead[\fancyplain{}{\bfseries The performance of new
   graduates}]{\fancyplain{}{}}
\lfoot[\fancyplain{}{\thepage}]{\fancyplain{}{From: K. Grant}}
\cfoot[\fancyplain{\bfseries \thepage}{From: K.
   Grant}]{\fancyplain{\bfseries \thepage}{To: Dean A. Smith}}
\rfoot[\fancyplain{}{To: Dean A. Smith}]{\fancyplain{}%
   {\bfseries \thepage}}
\setlength{\headrulewidth}{0.4pt}
\setlength{\footrulewidth}{0.4pt}
\end{verbatim}

\PSindex{fancyplain}
For the ``fancy plain'' page style, the thickness of the lines
is defined by
\begin{center}
\CmdIndex{plainheadrulewidth}
\CmdIndex{plainfootrulewidth}
\verb|\plainheadrulewidth| and \verb|\plainfootrulewidth|
\end{center}
\noindent(both default to 0pt).  Observe that you may have
different ``fancy plain'' page styles for even and odd pages.

To use the ``fancy plain'' page style, define these macros; in
the preamble, include

\begin{verbatim}
\usepackage{fancyheadings}
\pagestyle{fancyplain}
\end{verbatim}
and invoke the page style ``fancy plain'' with

\begin{verbatim}
\thispagestyle{plain}
\end{verbatim}
where desired.

\section{The default layout}\label{default}
Let us use the \verb|book.cls| documentclass and the default settings for
\textsf{fancyheadings}; so we only issue the commands

\begin{verbatim}
\usepackage{fancyheadings}
\pagestyle{fancy}
\end{verbatim}
and let \textsf{fancyheadings} take care of everything.  On the
pages where new chapters start, we get a centered page number in
the footer; there is no header, and there are no decorative lines.

On an even page, we get the layout:

\begin{samepage}
\vspace{8pt}
\noindent\makebox[\textwidth]{\sl 1.2  EVALUATION\hfill
 CHAPTER 1. INTRODUCTION}
\noindent\makebox[\textwidth]{\hrulefill}\\[\baselineskip]
\noindent\makebox[\textwidth]{\hfill page body\hfill}\\[\baselineskip]
\noindent\makebox[\textwidth]{\hrulefill}
\noindent\makebox[\textwidth]{\hfill4\hfill}

\end{samepage}

\vspace{8pt}

On an odd page, we get the layout:

\begin{samepage}
\vspace{8pt}
\noindent\makebox[\textwidth]{\sl CHAPTER 1. INTRODUCTION\hfill
 1.2  EVALUATION}
\noindent\makebox[\textwidth]{\hrulefill}\\[\baselineskip]
\noindent\makebox[\textwidth]{\hfill page body\hfill}\\[\baselineskip]
\noindent\makebox[\textwidth]{\hrulefill}
\noindent\makebox[\textwidth]{\hfill
3\hfill }

\end{samepage}


\vspace{8pt}
\noindent where the header text is slanted uppercase.

This default layout is produced by the following commands:
\CmdIndex{rightmark}
\CmdIndex{leftmark}
\begin{verbatim}
\lhead[\fancyplain{}{\slshape \rightmark}]{\fancyplain{}%
   {\slshape \leftmark}}
\chead{}
\rhead[\fancyplain{}{\slshape \leftmark}]{\fancyplain{}%
   {\slshape \rightmark}}
\lfoot{}
\cfoot{\rmfamily \thepage}
\rfoot{}
\end{verbatim}

The following settings are used for the decorative lines:

\begin{tabbing}
\CmdIndex{headrulewidth}
\noindent \verb|\headrulewidth|\qquad \qquad \qquad \=0.4\=pt\\
\CmdIndex{footrulewidth}
\verb|\footrulewidth|\>0\>pt\\
\CmdIndex{plainheadrulewidth}
\verb|\plainheadrulewidth|\>0\>pt\\
\CmdIndex{plainfootrulewidth}
\verb|\plainfootrulewidth| \>0\>pt
\end{tabbing}

The header text is turned into all uppercase in \verb|book.cls|.

\section{The scoop on \latex/'s marks}\label{sec:custom}
Usually, for documents of class \verb|book| and \verb|report|, you may
want to use chapter and section information in the headings (chapter only
for one-sided printing), and for documents of class \verb|article|,
section and subsection information (section only for one-sided
printing).  \latex/ uses a marker mechanism to remember the chapter and
section (section and subsection) information for a page; this is
discussed in detail in the
\latex/ \emph{Companion}, Section 4.3.1.

There are two ways you can use and change the higher- and lower-level
sectioning information available to you.  The macros:
\CmdIndex{rightmark}
\CmdIndex{leftmark}
\verb|\leftmark|
(higher-level) and \verb|\rightmark| (lower-level) contain the information
processed by \latex/, and you can use them directly as shown in 
Section~\ref{default}.

The \verb|\leftmark| contains the left argument of the \emph{last}
\verb|\markboth| on the page, the \verb|\rightmark| contains the right
argument of the \emph{first} \verb|\markboth| or the only argument of the
\emph{first} \verb|\markright| on the page. If no marks are present on a
page they are ``inherited'' from the previous page.

You can influence how chapter, section, and subsection
information (only two of them!) is displayed by redefining the
\CmdIndex{chaptermark}
\verb|\chaptermark|,
\CmdIndex{sectionmark}
\CmdIndex{subsectionmark}
\verb|\sectionmark|, and \verb|\subsectionmark| commands\footnote{There are
  similar commands for \texttt{paragraph} and \texttt{subparagraph} but
  they are seldom used.}. You must put the redefinition
after the first call of \verb|\pagestyle{fancy}| as this
sets up the defaults.

Let us illustrate this with chapter info. It is made up of three parts:
\begin{itemize}
\CmdIndex{thechapter}
\item  the number (say, 2), displayed by the macro \verb|\thechapter|
\item  the name (in English, Chapter), displayed by the macro
\CmdIndex{chaptername}
   \verb|\chaptername|
\item  the title, contained in the first argument of
   \verb|\markboth|.
\end{itemize}

Figure~\ref{fig:markers} shows some variants for ``Chapter 2.\ Do it now''
(the last example is appropriate in some non-English languages).

\begin{figure}[tb]
\CmdIndex{chaptermark}
\CmdIndex{uppercase}
\setlength{\columnsep}{20pt}\small
\begin{multicols}{2}
\noindent Code:\\
\mbox{}\\
\verb|\renewcommand{\chaptermark}[1]{|\\
\verb| \markboth{\chaptername\ |\\
\verb| \thechapter.\ #1}{}}|\\
\mbox{}\\
\verb|\renewcommand{\chaptermark}[1]{|\\
\verb| \markboth{\uppercase{|\\
\verb| \chaptername}\ \thechapter.|\\
\verb| \ #1}{}}|\\
\mbox{}\\
\verb|\renewcommand{\chaptermark}[1]{|\\
\verb| \markboth{\uppercase{|\\
\verb| \chaptername\ \thechapter.|\\
\verb| \ #1}}{}}|\\
\mbox{}\\
\verb|\renewcommand{\chaptermark}[1]{|\\
\verb| \markboth{#1}{}}|\\
\mbox{}\\
\verb|\renewcommand{\chaptermark}[1]{|\\
\verb| \markboth{\thechapter.\ #1}{}}|\\
\mbox{}\\
\verb|\renewcommand{\chaptermark}[1]{|\\
\verb| \markboth{\thechapter.\ |\\
\verb| \chaptername.\ #1}{}}|\\
Prints:\\
\mbox{}\\
Chapter 2.\ Do it now\\
\mbox{}\\
\mbox{}\\
\mbox{}\\
CHAPTER 2.\ Do it now\\
\mbox{}\\
\mbox{}\\
\mbox{}\\
\mbox{}\\
CHAPTER 2.\ DO IT NOW\\
\mbox{}\\
\mbox{}\\
\mbox{}\\
\mbox{}\\
Do it now\\
\mbox{}\\
\mbox{}\\
2.\ Do it now\\
\mbox{}\\
\mbox{}\\
2.\ Chapter.\ Do it now\\
\mbox{}\\
\end{multicols}
\caption{Marker variants}\label{fig:markers}
\end{figure}


For the lower-level sectioning information, do the same with
\verb|\markright|.

So if ``Section 2.2.\ First steps'' is the current section,
then

\begin{verbatim}
\renewcommand{\sectionmark}[1]{
   \markright{\thesection.\ #1}}
\end{verbatim}
will give
   ``2.2.\ First steps''

Redefining the \verb|\chaptermark| and \verb|\sectionmark| commands may
not eliminate all uppercaseness. E.g.\ the bibliography will have a title
\CmdIndex{uppercase}
\index{BIBLIOGRAPHY}
of \textsc{bibliography} in the header, as the \verb|\uppercase| is
explicitly given in the definition of \verb|\thebibliography|. Similar for
\index{INDEX}
\textsc{index} etc.\ If you don't want to redefine these commands, you can
use an ugly hack like the following:

\begin{verbatim}
\lhead{\let\uppercase\relax\rightmark}
\rhead{\let\uppercase\relax\leftmark}
\end{verbatim}


It should be noted that the \latex/ marking mechanism works fine with
chapters (which always start on a new page) and sections (which are
reasonably long).  It does not work quite as well with short sections and
subsections.  This is a problem with \latex/, not with
\textsf{fancyheadings}.

As an example let's take a page layout where the leftmarks are generated by
the sections and the rightmarks by the subsections (as is default in the
\texttt{article} class). Take a page with some short sections, e.g. 

\begin{samepage}
\noindent Section 1.\\
subsection 1.1\\
subsection 1.2\\
Section 2.  
\end{samepage}

As the leftmark contains the \emph{last} mark of the page it will be
``Section 2.'', and the rightmark will be ``subsection 1.1'' as it will be
the \emph{first} mark of the page. So the page header info will combine 
section 2 with subsection 1.1 which isn't very nice.
The best you can do in these cases is use only the \verb|\rightmark|s and
redefine \verb|\sectionmark| accordingly. A \latex/ command
\CmdIndex{firstleftmark}
\verb|\firstleftmark| would also be a nice addition (see the 
\texttt{extramarks} package in section~\ref{sec:xmarks}).

Another problem with the marks in the standard \latex/ classes is that the
higher level sectioning commands (e.g. \verb|\chapter|) call
\verb|\markboth| with an empty right argument. This means that on the first
page of a chapter (or a section in article style) the \verb|\rightmark|
will be empty. If this is a problem you must manually insert extra
\verb|\markright| commands or redefine the \verb|\chaptermark|
(\verb|\sectionmark|) commands to issue a \verb|\markboth| command with
two decent parameters.

As a final remark you should also note that the \verb|*| forms of the
\verb|\chapter| etc.\ commands do \emph{not} call the mark commands. So if
you want your preface to set the header info but not be numbered nor be put in
the table of contents, you must issue the \verb|\markboth| command
yourself, e.g.

\begin{verbatim}
\chapter*{Preface\markboth{Preface}{}}
\end{verbatim}

Entering the \verb|\markboth| command inside the \verb|\chapter*| insures
that the mark will not be separated from the title by a page break. Of
course with \verb|\chapter| this wouldn't be  a problem if you put the mark
command after the chapter title, as the \verb|\chapter| command starts at a
new page. However with a \verb|\section*| it could be dangerous to say:

\begin{verbatim}
\section*{Preface}
\markboth{Preface}{}
\end{verbatim}
as a page break may occur between the two commands.

\section{Dictionary style headers}

\index{dictionary}
\index{concordance}
Dictionaries and concordances usually have a header containing the first
word defined on the page or both the first and the last words. This can
easily be accomplished with \textsf{fancyheadings} and \latex/'s
\texttt{mark} mechanism. Of course if you use the marks for dictionary
style headers, you cannot use them for chapter and section information, so
if there are also chapters and sections present, you must redefine the
\verb|\chaptermark| and \verb|\sectionmark| to make them harmless:

\begin{verbatim}
\renewcommand{\chaptermark}[1]{}
\renewcommand{\sectionmark}[1]{}
\end{verbatim}

Now you do a \verb|\markboth{#1}{#1}| for each dictionary or concordance
entry \verb|#1| and use \verb|\rightmark| for the first entry defined on
the page and \verb|\leftmark| for the last one.

If you want to use a header entry of the form \textsf{firstword--lastword}
it would be nice if this would be reduced to just the form
\textsf{firstword} if both are the same. This could happen if there is just
one entry on the page. In this case a test must be made to check if the
marks are the same.  However, \tex/'s marks are strange beasts, which
cannot be compared out of the box with the plain \tex/ \verb|\if| commands.
\TTindex{ifthen}
Fortunately the \texttt{ifthen} package works well:

\begin{verbatim}
\newcommand{\mymarks}{
  \ifthenelse{\equal{\leftmark}{\rightmark}}
    {\rightmark}
    {\rightmark--\leftmark}}
\lhead[\rm \mymarks]{\rm \thepage}
\rhead[\rm \thepage]{\rm \mymarks}
\end{verbatim}

Dictionaries are often done with two columns. Unfortunately there is a bug
in \latex/'s twocolumn option which causes some marks to be lost. If you
use the  package in figure~\ref{fig:fixmarks}, this will be
solved\footnote{This is a simplification of the  actual code. Get the real
  package from the CTAN sites.}\footnote{The
  \texttt{multicol} package uses a similar technique.}. 

\begin{figure}[tb]
\small
\TTindex{fixmarks.sty}
\begin{verbatim}
% fixmarks.sty:
% Patch LaTeX's output routine to handle marks correctly with two columns.
% Joe Pallas <pallas@edu.stanford.neon>
% Corrected by Piet van Oostrum <piet@cs.ruu.nl> on Feb 5, 1993, Oct 5, 1994

\def\@outputdblcol{\if@firstcolumn \global\@firstcolumnfalse
% Remember the marks from the first column
    \global\setbox\@leftcolumn\copy\@outputbox
    \splitmaxdepth=\maxdimen \cbaddness=10000
    \setbox\@outputbox\vsplit\@outputbox to\maxdimen
    \xdef\@firstcoltopmark{\topmark}%
    \xdef\@firstcolfirstmark{\splitfirstmark}%
    \ifx\@firstcolfirstmark\empty\global\let\@setmarks\relax\else
      \gdef\@setmarks{\let\firstmark\@firstcolfirstmark
        \let\topmark\@firstcoltopmark}%
    \fi
% End of change
  \else \global\@firstcolumntrue
    \setbox\@outputbox\vbox{\hbox to\textwidth{\hbox to\columnwidth
      {\box\@leftcolumn \hss}\hfil \vrule width\columnseprule\hfil
       \hbox to\columnwidth{\box\@outputbox \hss}}}\@combinedblfloats
% Override current first and top with those of first column if necessary
       \@setmarks
% End of change
       \@outputpage \begingroup \@dblfloatplacement \@startdblcolumn
       \@whilesw\if@fcolmade \fi{\@outputpage\@startdblcolumn}\endgroup
    \fi}
\end{verbatim}

  \caption{Fix marks in twocolumn style}
  \label{fig:fixmarks}
\end{figure}

\section{Fancy layouts}

\index{multi-line}
You can make a multi-line field with the \verb|\\| command.  It is also
possible to put extra space in a field with the \verb|\vspace| command.
Note that if you do this you will probably have to increase the height of
\CmdIndex{headheight}
\CmdIndex{footskip}
the header (\verb|\headheight|) or of the footer (\verb|\footskip|),
\index{Overfull \verb|\vbox| \ldots}
otherwise you may get error messages ``Overfull \verb|\vbox| \ldots has occurred while \verb|\output| is active''.
See Section 4.1 of the \latex/ \emph{Companion} for detail.

For instance, the following code will place  the section title and the
subsection title of an article in two lines in the upper right hand
corner:

\begin{verbatim}
\documentclass{article}
\usepackage{fancyheadings}
\pagestyle{fancy}
\addtolength{\headheight}{\baselineskip}
\renewcommand{\sectionmark}[1]{\markboth{#1}{}}
\renewcommand{\subsectionmark}[1]{\markright{#1}}
\rhead{\leftmark\\\rightmark}
\end{verbatim}

You can customize the decorative lines.  You can make the decorative
line in the header quite thick with

\CmdIndex{headrulewidth}
\begin{verbatim}
\setlength{\headrulewidth}{0.6pt}
\end{verbatim}
or you can make the decorative line in the footer disappear with
\CmdIndex{footrulewidth}
\begin{verbatim}
\setlength{\footrulewidth}{0pt}
\end{verbatim}

The decorative lines, themselves, are defined in the two macros
\verb|\headrule| and \verb|\footrule|.  For instance,
if you want a dotted line rather than a solid line in the header,
redefine the command \verb|\headrule|:

\begin{verbatim}
\renewcommand{\headrule}{\vbox to 0pt{\hbox
    to\headwidth{\dotfill}\vss}}
\end{verbatim}

\section{Two book examples}
The following definitions give an approximation of the style
used in L. Lamport's \latex/ book.

Lamport's header overhangs the outside margin.  This is done as follows.

The width of headers and footers is \verb|\headwidth|, which by default
equals the width of the text: \verb|\textwidth|. You can make the width
\CmdIndex{headwidth}
wider (or narrower) by redefining \verb|\headwidth| with the
\verb|\setlength| and \verb|\addtolength| commands.
To overhang the outside margin where the marginal notes are
\CmdIndex{marginparsep}
\CmdIndex{marginparwidth}
printed, add both \verb|\marginparsep| and \verb|\marginparwidth| to
\verb|\headwidth| with the commands:

\begin{verbatim}
\addtolength{\headwidth}{\marginparsep}
\addtolength{\headwidth}{\marginparwidth}
\end{verbatim}

\begin{flushleft}
You must issue these commands \emph{after} the first
\verb|\pagestyle{fancy}| or \verb|\pagestyle{fancyplain}| command as this
will establish the default for \verb|\headwidth|.
\end{flushleft}

And now a complete definition of Lamport's book style:

\begin{verbatim}
\documentclass{book}
\usepackage{fancyheadings}
\pagestyle{fancyplain}
\addtolength{\headwidth}{\marginparsep}
\addtolength{\headwidth}{\marginparwidth}
\renewcommand{\chaptermark}[1]{\markboth{#1}{}}
\renewcommand{\sectionmark}[1]{\markright{\thesection\ #1}}
\lhead[\fancyplain{}{\bfseries\thepage}]
   {\fancyplain{}{\bfseries\rightmark}}
\rhead[\fancyplain{}{\bfseries\leftmark}]
   {\fancyplain{}{\bfseries\thepage}}
\cfoot{}
\end{verbatim}

Notice that the \verb|\chaptermark| and \verb|\sectionmark| commands have
been redefined to eliminate the chapter numbers and the uppercaseness.

For the second example, we take the \amslatex/ book.% \cite{gG93B}.

Chapter pages have no headers or footers.  So we declare

\begin{verbatim}
\thispagestyle{empty}
\end{verbatim}
for every chapter page, and we do not need fancyplain.

Chapter and section titles appear in the form: 2.\ DO IT NOW, so we have
to redefine \verb|\chaptermark| and \verb|\sectionmark| as follows (see
Section~\ref{sec:custom}):

\begin{verbatim}
\renewcommand{\chaptermark}[1]%
   {\markboth{\uppercase{\thechapter.\ #1}}{}}
\renewcommand{\sectionmark}[1]%
   {\markright{\uppercase{\thesection.\ #1}}}
\end{verbatim}

In an even-header, the page number is printed as the LeftHeader and
the chapter info as the RightHeader; in an odd-header, the section info
is printed as the LeftHeader and the page number as the RightHeader.  The
CenteredHeaders are empty.  There are no footers.

There is a decorative line in the header.  It is 0.5pt wide, so we need
the commands:

\begin{verbatim}
\setlength{\headrulewidth}{0.5pt}
\setlength{\footrulewidth}{0pt}
\end{verbatim}

The font used in the headers is 9 pt bold Helvetica.  The \PSNFSS\ system
by Sebastian Rahtz uses the short (Karl Berry) name \verb|phv| for
Helvetica, so this font is selected with the commands:
\begin{verbatim}
\fontfamily{phv}\fontseries{b}\fontsize{9}{11}\selectfont
\end{verbatim}
(See Sections 7.6.1 and 11.9.1 of the \latex/ \emph{Companion}.)
Let us define a shorthand for this:
\begin{verbatim}
\newcommand{\helv}{%
   \fontfamily{phv}\fontseries{b}\fontsize{9}{11}\selectfont}
\end{verbatim}

Now we are ready for the page layout:

\begin{verbatim}
\lhead[\helv \thepage]{\helv \rightmark}
\rhead[\helv \leftmark]{\helv \thepage}
\cfoot{}
\end{verbatim}

Putting this all together:
\begin{verbatim}
\documentclass{book}
\usepackage{fancyheadings}
\pagestyle{fancy}
\renewcommand{\chaptermark}[1]%
   {\markboth{\uppercase{\thechapter.\ #1}}{}}
\renewcommand{\sectionmark}[1]%
   {\markright{\uppercase{\thesection.\ #1}}}
\setlength{\headrulewidth}{0.5pt}
\setlength{\footrulewidth}{0pt}
\newcommand{\helv}{%
   \fontfamily{phv}\fontseries{b}\fontsize{9}{11}\selectfont}
\lhead[\helv \thepage]{\helv \rightmark}
\rhead[\helv \leftmark]{\helv \thepage}
\cfoot{}
\end{verbatim}


\section{Special page layout for float pages}
\label{sec:float}

\index{float page}
Some people want to have a special layout for float pages (pages only
containing floats). As these pages
are generated autonomically by \latex/, the user doesn't have any control
over them. There is no \verb|\thispagestyle| for float pages and any change
of the page style will at least also affect the page before the float page.
With fancyheadings you can give the following definition in your
preamble\footnote{If you have \textsf{fancyheadings} version 1.5 or higher
  the commands described in this section are already included}:

\begin{verbatim}
\makeatletter
\def\iffloatpage#1#2{\if@fcolmade #1\else #2\fi}
\makeatother
\end{verbatim}
\CmdIndex{iffloatpage}

In each of
the header- or footer fields you can now specify

\verb|\iffloatpage{|value for float page\verb|}{|value for other pages\verb|}|

You can even use this to get rid of the decorative line on float pages only
by defining:

\begin{verbatim}
\renewcommand{\headrulewidth}{\iffloatpage{0pt}{0.4pt}}
\end{verbatim}

Note that \verb|\renewcommand| is used rather than the usual
\verb|\setlength| command to change the \verb|\headrulewidth|. Changing it
with \verb|\setlength| will not work\footnote{It was probably a dumb
decision to define the width parameters as lengths. There is no need to and
I may change this in a future version of \textsf{fancyheadings}}.

\begin{sloppy}
After this \verb|\renewcommand| it will no longer be possible to change the
parameter \verb|\headrulewidth| with \verb|\setlength|, you will have to use
\verb|\renewcommand| in the rest of the document.
\end{sloppy}

Sometimes you may want to change the layout also for pages that contain a
float on the top of the page or a float on the bottom of the page.

Enter the following definitions in a style file (or between
\verb|\makeatletter| and  \verb|\makeatother| in your preamble:

\begin{verbatim}
\let\latex@makecol\@makecol
\def\@makecol{\let\topfloat\@toplist\let\botfloat\@botlist\latex@makecol}
\def\iftopfloat#1#2{\ifx\topfloat\empty #2\else #1\fi}
\def\ifbotfloat#1#2{\ifx\botfloat\empty #2\else #1\fi}
\end{verbatim}
\CmdIndex{iftopfloat}
\CmdIndex{ifbotfloat}

You can then use the commands \verb|\iftopfloat| and  \verb|\ifbotfloat|
similar to
 \verb|\iffloatpage|.

Note: Marks in floats will not be visible in \latex/'s output routine, so
it is not useful to put marks in floats. So there is currently no way to
let a float (e.g.\ a figure caption) influence the page header or footer.

\section{Those blank pages}
\label{sec:blank}

In the \texttt{book} class when the \texttt{openany} option is not given or
in the \texttt{report} class when the \texttt{openright} option is given,
chapters start at odd-numbered pages, half of the time causing a blank page
to be inserted. Some people prefer this page to be completely empty, i.e.\ 
without headers and footers. This cannot be done with \verb|\thispagestyle|
as this command would have to be issued on the \emph{previous} page. There
is, however, no magic necessary to get this done:

\PSindex{empty}
\CmdIndex{clearpage}
\CmdIndex{cleardoublepage}
\begin{verbatim}
\clearpage{\pagestyle{empty}\cleardoublepage}
\end{verbatim}

As the \verb|\pagestyle{empty}| is enclosed in a group it only affects the
page that may be generated by the \verb|\cleardoublepage|.  You can of
course put the above in a private command. If you want to have this done
automatically at each chapter start or when you want some other text on the
page then you must redefine the \verb|\cleardoublepage| command.
\index{blank page}
\begin{verbatim}
\makeatletter
\def\cleardoublepage{\clearpage\if@twoside \ifodd\c@page\else
  \hbox{}
  \vspace*{\fill}
  \begin{center}
    This page intentionally contains only this sentence.
  \end{center}
  \vspace{\fill}
  \thispagestyle{empty}
  \newpage
  \if@twocolumn\hbox{}\newpage\fi\fi\fi}
\makeatother
\end{verbatim}

\section{\textsf{N} of \textsf{M} style page numbers}
\label{sec:nofm}

Some document writers prefer the pages to be numbered as \textsf{n} of
\textsf{m} where \textsf{m} is the number of pages in the document. There
\TTindex{nofm.sty}
is a package \texttt{nofm.sty} available, but some versions of it are
defective, and most don't work with fancyheadings because they take over the
\TTindex{lastpage.sty}
complete page layout. For \LaTeXe{} there is a package \texttt{lastpage}
available which you can use with \textsf{fancyheadings} as follows:

\begin{verbatim}
\usepackage{lastpage}
...
\cfoot{\thepage\ of \pageref{LastPage}}
\end{verbatim}

If you are still using \latex/2.09 and you are not able to switch to
\LaTeXe{} you can define a \latex/2.09 compatible \texttt{lastpage.sty} as
follows:

\begin{verbatim}
\let\origenddocument=\enddocument
\def\enddocument{\clearpage\if@filesw
  {\addtocounter{page}{-1} \immediate\write\@mainaux
    {\string\newlabel{LastPage}{{}{\thepage}}}}\origenddocument}
\end{verbatim}

With some trickery\footnote{If you have a recent version of the
  \texttt{ifthen} package the tricks are not necessary. You can then test
  the \texttt{\char`\\pageref\char`\{LastPage\char`\}} directly.}
 the value of the \texttt{LastPage} label can be used to
make different headers or footers on the last page of a document. E.g.\ if
you want the footer of every odd page, except if it is the last one, to
contain the text ``please turn over'', this can be done as follows:

\begin{verbatim}
\usepackage{lastpage}
\usepackage{ifthen}
...
\makeatletter
\def\LastPagenum{\@ifundefined{r@LastPage}{0}{\expandafter
    \expandafter\expandafter\@cdr
    \csname r@LastPage\endcsname\@nil\null} }
\makeatother
\rfoot{\ifthenelse{\isodd{\value{page}} \and \not
    \value{page}=\LastPagenum}{please turn over}{}}
\end{verbatim}


\section{When to change the headers and footers?}

Sometimes you want to change the header or footer layout in the course of a
document. Some of these changes can be accomplished by using the mark
mechanism as may be seen in section~\ref{sec:custom} and \ref{sec:xmarks}.
However, sometimes we want a more drastic change, e.g to change the page
numbering from roman to arabic (with \verb|\pagenumbering|), to change one of
\index{page style!changes}
the \textsf{fancyheadings} fields or to change to another page style.
Sometimes you may be surprised to find the change to occur too early. In
general the above mentioned changes take effect immediately, i.e.\ on the
page that is currently being built. If you want the change to take effect
at the next page you must make sure that the current page is finished. In
\CmdIndex{clearpage}
most cases this can be done by issuing a \verb|\clearpage| command before
any of the above mentioned changes. If this is not possible you can use the 
\TTindex{afterpage.sty}
\texttt{afterpage} package with:\\
\verb|\afterpage{\lhead{new value}}| or
\CmdIndex{pagenumbering}
\verb|\afterpage{\pagenumbering{roman}}|. You cannot use \verb|\afterpage|
to change the \verb|\pagestyle| as the commands issued by \verb|\afterpage|
are local in a group, and the \verb|\pagestyle| command makes only local
changes. The \verb|\pagenumbering| and the \textsf{fancyheadings} commands
make global changes so they will work, as will the \verb|\thispagestyle|
command. 

It should be noted that although the \textsf{fancyheadings} commands like
\verb|\lhead| take effect immediately, this does not mean that any
``variables'' used in these commands get the value they have at the place
where these commands are given. E.g.\ if \verb|\cfoot{\thepage}| is given
the page number that will be inserted in the footer is not the page number
of the page where this command is given, but rather the page number of
the actual page where the footer is constructed. Of course for the page
number this is what you expect, but it is also true for other commands.

So if you have a book where each chapter is written by a different author
and you want the name of the author in the lower left-hand corner you can
use the following commands:

\begin{verbatim}
\newcommand{\TheAuthor}{}
\newcommand{\Author}[1]{\renewcommand{\TheAuthor}{#1}}
\lfoot{\TheAuthor}
\end{verbatim}

\noindent and start each chapter with the command 
\verb|\Author{Real Name}|.
 If however, the author name would be changed before a page is
completed the wrong author could come in the footer. This would be the case
if you gave the above command \emph{before} the \verb|\chapter| command
rather than after it. 
Another source of problems is the fact that \tex/'s output routine processes
commands ahead, so it may already have processed some commands that produce
text that will appear on the next page. See the next section for an example.

\section{Headers and footers induced by the text}
\label{sec:xmarks}

We have seen how we can use \LaTeX's marks to get information from the
document contents to the headers and footers. The marks mechanism is the
only reliable mechanism that you can use to get changing information to the
headers or footers. This is because \latex/ may be processing your document
ahead before deciding to break the page. 

Sometimes the two marks that \latex/ offers are not enough. An example is
the following:
\begin{quote}
 If a solution to an exercise goes across a page break, then I would like
 to have ``(Continued on next page\ldots)'' at the bottom of the
\index{Continued\ldots}
 first page and ``(Continued\ldots)'' at the top in the margin of the next page.
\end{quote}

You cannot use \latex/'s mark mechanisms for this if you also want to use
chapter and section information.

The code from figure~\ref{fig:xmarks} constitutes a package that gives you
two extra marks that can be used in this situation\footnote{After I made
  this package I discovered a package \texttt{secret.sty} that does a
  similar thing to mark confidential paragraphs if they cross a page
  boundary. It does it, however, by changing the output routine.}.
\begin{figure}[tb]\small
\CmdIndex{extramarks}
\begin{verbatim}
% extramarks.sty
\def\@leftmark#1#2#3#4{#1}
\def\@rightmark#1#2#3#4{#2}

\def\markboth#1#2{{\def\protect{\noexpand\protect\noexpand}
     \let\label\relax \let\index\relax \let\glossary\relax
     \expandafter\@markboth\@themark{#1}{#2}
     \mark{\@themark}}\if@nobreak\ifvmode\nobreak\fi\fi}
\def\markright#1{{\def\protect{\noexpand\protect\noexpand}
     \let\label\relax \let\index\relax \let\glossary\relax
     \expandafter\@markright\@themark
     {#1}\mark{\@themark}}\if@nobreak\ifvmode\nobreak\fi\fi}
\def\@markright#1#2#3#4#5{\gdef\@themark{{#1}{#5}{#3}{#4}}}
\def\@markboth#1#2#3#4#5#6{\gdef\@themark{{#5}{#6}{#3}{#4}}}
\def\leftmark{\expandafter\@leftmark\botmark{}{}{}{}}
\def\rightmark{\expandafter\@rightmark\firstmark{}{}{}{}}
\def\firstleftmark{\expandafter\@leftmark\firstmark{}{}{}{}}
\def\lastrightmark{\expandafter\@rightmark\botmark{}{}{}{}}

\def\@themark{{}{}{}{}}

\def\extramarks#1#2{{\def\protect{\noexpand\protect\noexpand}
     \let\label\relax \let\index\relax \let\glossary\relax
     \expandafter\@markextra\@themark{#1}{#2}
     \mark{\@themark}}\if@nobreak\ifvmode\nobreak\fi\fi}
\def\@markextra#1#2#3#4#5#6{\gdef\@themark{{#1}{#2}{#5}{#6}}}
\def\firstxmark{\expandafter\@firstxmark\firstmark{}{}{}{}}
\def\topxmark{\expandafter\@firstxmark\topmark{}{}{}{}}
\def\lastxmark{\expandafter\@lastxmark\botmark{}{}{}{}}
\def\@firstxmark#1#2#3#4{#3}
\def\@lastxmark#1#2#3#4{#4}
\end{verbatim}
  \caption{Package for extra marks in \latex/}
  \label{fig:xmarks}
\end{figure}
Here is a way to use this package:

\begin{verbatim}
\usepackage{extramarks}
...
\pagestyle{fancy}
\lhead{\firstxmark}
\rfoot{\lastxmark}
...
\extramarks{}{Continued on next page\ldots}
Some text that may or may not cross a page boundary...
\extramarks{Continued\ldots}{}
\end{verbatim}

\CmdIndex{extramarks}
Note that the \verb|\extramarks| command must be close to the text, i.e no
empty lines (paragraph boundaries) should intervene. Otherwise the page may
be broken at that boundary and the extramarks would come on the wrong page.

There are two new marks that can be used in the page layout with this
package: If commands of the form
\verb|\extramarks{|$m_1$\verb|}{|$m_2$\verb|}| are given
\verb|\firstxmark| gives you the first $m_1$ value and
\verb|\lastxmark| gives you the last $m_2$ value
of the current page.
\CmdIndex{firstleftmark}
\CmdIndex{lastrightmark}
It also gives you the \verb|\firstleftmark| and \verb|\lastrightmark|
commands that complement the standard \latex/ marks.

To stress the point that marks are the correct way to do this, let me
give you a ``solution'' that will not work\footnote{Actually there is
  another way but it requires two \latex/ passes: you can put \texttt{\char`\\label}
  commands before and after the text and compare the \texttt{\char`\\pageref}s.}:

\begin{verbatim}
\lhead{Continued}
\rfoot{Continued on next page\ldots}
Some text that may or may not cross a page boundary...
\lhead{}
\rfoot{}
\end{verbatim}

You may be tempted to think that the first \verb|\lhead| and \verb|\rfoot|
will be in effect when \tex/ breaks the page in the middle of the text,
and the last ones when the page breaks after the text. This is not true as
the whole paragraph (including the last definitions) will be processed
\index{page break}
before \tex/ considers the page break, so at the time of the page break the
last definitions are effective, whether the page break occurs inside the text
or outside of it. Putting a paragraph boundary between the
text and the last definitions will not work either, because you don't want
the first definitions to be in effect when \tex/ decides to break the page
exactly at this boundary. Actually the marks mechanism was invented to get
rid of these kinds of problems.

In the above example the text ``Continued'' appears in the page header. It
\index{margin}
may be nicer to put it in the  margin. This can be easily
accomplished by positioning it at a fixed place relative to the page header.
In plain \tex/ you would use a concoction of \verb|\hbox to 0pt|,
\verb|\vbox to 0pt|, \verb|\hskip|,\verb|\vskip|, \verb|\hss| and
\verb|\vss| but fortunately 
\latex/'s \texttt{picture} environment gives a much cleaner way to do this.
In order not to disturb the normal header layout we put the text in a zero-sized
\texttt{picture}. Generally this is the best way to position things on fixed
places on the page. You can then also use the normal headings. See also
section~\ref{sec:thumb} for another example of 
this technique.

\TTindex{picture}
\begin{verbatim}
\setlength{\unitlength}{\baselineskip}
\lhead{\begin{picture}(0,0)
  \put(-2,-3){\makebox(0,0)[r]{\firstxmark}}
  \end{picture}\leftmark}
\end{verbatim}

This solution can of course also be used for the footer. Be sure to put the
\texttt{picture} as the first thing in left-handside entries and last in right-handside
ones.

Finally you may want to put ``(Continued\ldots)'' in the \emph{text}
rather than in the header or the margin. Then you have to use the
\TTindex{afterpage.sty}
\texttt{afterpage} package. We also decide to make a separate
environment for it.

\begin{verbatim}
\newenvironment{continued}{\par
  \extramarks{}{Continued on next page\ldots}
  \afterpage{\noindent\firstxmark\vspace{1ex}}
  }{\extramarks{(Continued\ldots)}{}\par}
\end{verbatim}

It is a bit dangerous to use \verb|\firstxmark| outside the page layout
routine, but apparently with \verb|\afterpage| this works. If you would
need the information further on in the page you must remember the state of
the marks in your own variable. You can do this in one of the
\textsf{fancyheadings} fields. For example if you want to add something
\emph{after} the broken piece of text you can use the following:

\begin{verbatim}
\newcommand{\mysaved}{}

\newenvironment{continued}{\par
  \extramarks{}{Continued on next page\ldots}
  }{\extramarks{(Continued\ldots)}{}\par\vspace{1ex}\mysaved}
\lhead{\leftmark}
\chead{\ifthenelse{\equal{\lastxmark}{}}
  {\gdef\mysaved{}}
  {\gdef\mysaved{\noindent[Continued from previous page]}}}
\end{verbatim}

If you want to include one of the marks or other varying information in the
saved text, you must use \verb|\xdef| rather than \verb|\gdef|.


\section{A movie}

\index{movie}
\TTindex{picture}
If you put at each page on the same place a picture that slightly changes
from page to page you can get a movie-like effect by flipping through the
pages. You can create such a movie easily with fancyheadings. For
simplicity we assume that each picture is in a postscript (EPS) file called
\texttt{pic}$\langle n\rangle$.\texttt{ps} where $\langle n\rangle$ is the page number and that we use
the \texttt{epsf} package. 
\TTindex{epsf}
To put the movie in the  righthandside bottom corner the following will work:

\begin{verbatim}
\rfoot{\setlength{\unitlength}{1mm}\setlength{\epsfxsize}{2cm}
  \begin{picture}(0,0)
    \put(5,0){\epsfbox{pic\thepage.ps}}
  \end{picture}}
\end{verbatim}

Notice that the \verb|\unitlength| and \verb|\epsfxsize| parameter should
be set locally in the fancyheadings field in order to avoid unwanted
interference from their values in the text.

\section{Thumb-indexes}
\label{sec:thumb}

\index{bible}
Some railroad guides and expensive bibles have so called
\index{thumb-index}
\emph{thumb-indexes}, i.e.\ there are marks on the sides of the pages that
indicate where the chapters are. You can create these by printing black
blobs in the margin of the pages. The vertical position should be
determined by the chapter number or some other counter. As the position is
independent of the contents of the page, we print these blobs as part of
the header in a zero-sized \texttt{picture} as described in the previous
section.

Of course we have to take care of two-sided printing, and we may want to
have an index page with all the blobs in the correct position. The solution
requires some hand-tuning to get the blobs nicely spaced out vertically.
For the application that I had there were 12 sections, so I made the blobs
18 mm apart, i.e. 9~mm blob separated by 9~mm whitespace. In order to avoid
calculations they are set in a \texttt{picture} environment with the
\verb|\unitlength| set to 18~mm. Page numbers are set in the headers at the
outer sides, and the blobs are attached to these. In this example the
section numbers are used to position the blobs, but you can replace this
with any numeric value. See figure~\ref{fig:overview} for the resulting
overview page and figure~\ref{fig:thumb} for the code.

\begin{figure}[htb]
\setlength{\unitlength}{9mm}
\newcommand{\blob}{\rule[-.2\unitlength]{1\unitlength}{.5\unitlength}}
\newcounter{line}
\newcommand{\secname}[1]{\addtocounter{line}{1}%
  \put(1,-\value{line}){\blob}
  \put(-7.5,-\value{line}){\arabic{line}}
  \put(-7,-\value{line}){#1}}

\newcommand{\overview}{1
  \begin{picture}(0,0)
    \secname{Introduction}
    \secname{The first year}
    \secname{Specialisation}
  \end{picture}}
 
  \begin{center}
    \leavevmode
    \begin{picture}(11.3,5)
      \put(0,0){\framebox(11.3,5)[tr]{}}
      \put(9,4.5){\overview}
    \end{picture}
  \end{center}
  \caption{Thumb-index overview page}
  \label{fig:overview}
\end{figure}

\begin{figure}[tb]\small
\begin{verbatim}
\setlength{\unitlength}{18mm}
\newcommand{\blob}{\rule[-.2\unitlength]{2\unitlength}{.5\unitlength}}

\newcommand\rblob{\thepage
  \begin{picture}(0,0)
    \put(1,-\value{section}){\blob}
  \end{picture}}

\newcommand\lblob{%
  \begin{picture}(0,0)
    \put(-3,-\value{section}){\blob}
  \end{picture}%
  \thepage}

\pagestyle{fancy}
\cfoot{}

\newcounter{line}
\newcommand{\secname}[1]{\addtocounter{line}{1}%
  \put(1,-\value{line}){\blob}
  \put(-7.5,-\value{line}){\Large \arabic{line}}
  \put(-7,-\value{line}){\Large #1}}

\newcommand{\overview}{\thepage
  \begin{picture}(0,0)
    \secname{Introduction}
    \secname{The first year}
    \secname{Specialisation}
...etc...
  \end{picture}}

\begin{document}
\rhead{\overview}\mbox{}\newpage % This produces the overview page
\rhead{} % Front matter may follow here
\clearpage
\rhead[\rightmark]{\rblob} \lhead[\lblob]{\leftmark}
...
\end{verbatim}
  \caption{Thumb-index code}
  \label{fig:thumb}
\end{figure}


\section{Final Remarks}

The commands to specify the headers and footers are quite complicated when
even- and odd-numbered pages have different layout and we want also to
redefine the \texttt{plain} page style. Therefore for version 2 of
\textsf{fancyheadings} I will create an easier user interface (the old
commands will still work). To give you the opportunity to comment, here is
a preview:

The \verb|*head| and \verb|*foot| commands will be replaced by commands
\verb|\fancyhead| and \verb|\fancyfoot|. These will have an optional
argument that selects which fields to change, e.g. L for the left field, E
for even pages, etc. See figure~\ref{fig:sel}. Selectors can be combined so
\verb|\fancyhead[LE,RO]{text}|
will define the field for both the left header on even pages and the right
header on odd pages.
\begin{figure}[tb]
  \begin{center}
    \leavevmode
    \begin{tabular}{|l|l|}
      \hline
      E & Even page         \\
      O & Odd page          \\
      \hline
      L & Left field        \\
      C & Center field      \\
      R & Right field       \\
      \hline
      T & float at Top      \\
      B & float at Bottom   \\
      F & Float page         \\
      \hline
    \end{tabular}
  \end{center}
  \caption{Selectors}
  \label{fig:sel}
\end{figure}
There will also be a new command \verb|\fancypagestyle| that allows you to
(re)define page styles. One use for this is to redefine the page style
\PSindex{plain}
\PSindex{fancyplain}
\texttt{plain} so the \verb|\fancyplain| stuff is no longer necessary.
Another use is to allow special page styles to be used with
\verb|thispagestyle| or maybe for float pages.

The complicated example in section~\ref{complicated} will then become:

\begin{verbatim}
\fancyhead{} % clear all fields
\fancyhead[LE,RO]{\bfseries The performance of new graduates}}
\fancyfoot[LO,CE]{From: K. Grant}
\fancyfoot[RE,CO]{To: Dean A. Smith}
\fancyfoot[RO,LE]{\bfseries \thepage}
\setlength{\headrulewidth}{0.4pt}
\setlength{\footrulewidth}{0.4pt}

\fancypagestyle{plain}{%
\fancyhead{}
\fancyfoot[C]{\bfseries \thepage}}
\end{verbatim}

As you can see this is much less complicated.

If you want to comment please use e-mail or regular mail to:\\
Piet van Oostrum\\
Dept.\ of Computer Science\\
Utrecht University\\
P.O. Box 80.089\\
3508 TB Utrecht, The Netherlands\\
Telephone: +31 30 531806 Telefax: +31 30 513791\\
Internet: piet@cs.ruu.nl 

\printindex

\end{document}


% Local Variables:
% mode: latex
% TeX-master: t
% End:
