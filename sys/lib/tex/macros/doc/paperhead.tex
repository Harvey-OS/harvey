\catcode`@=11
%\font\titlefont=amtitl at 14 truept
\font\titlefont=cmssbx10 scaled \magstep 2

\font\ninerm=cmr9
\font\eightrm=cmr8
\font\sixrm=cmr6

\font\ninei=cmmi9
\font\eighti=cmmi8
\font\sixi=cmmi6

\font\ninesy=cmsy9
\font\eightsy=cmsy8
\font\sixsy=cmsy6

\font\eightss=cmssq8


\font\ninebf=cmbx9
\font\eightbf=cmbx8
\font\sixbf=cmbx6

\font\ninett=cmtt9
\font\eighttt=cmtt8

\font\ninesl=cmsl9
\font\eightsl=cmsl8

\font\nineit=cmti9
\font\eightit=cmti8

\font\tenu=cmu10 % unslanted italic
\font\tenrm=cmr10
\font\tentt=cmtt10
\let\oldtenbf=\tenbf
\font\tenbf=cmbx10
\let\mainfont=\tenrm
\textfont0=\tenrm
\let\oldninerm=\ninerm
\font\ninerm=cmr9
\newskip\ttglue
\def\tenpoint{\def\rm{\fam0\tenrm}%
  \textfont0=\tenrm \scriptfont0=\sevenrm \scriptscriptfont0=\fiverm
  \textfont1=\teni \scriptfont1=\seveni \scriptscriptfont1=\fivei
  \textfont2=\tensy \scriptfont2=\sevensy \scriptscriptfont2=\fivesy
  \textfont3=\tenex \scriptfont3=\tenex \scriptscriptfont3=\tenex
  \def\it{\fam\itfam\tenit}%
  \textfont\itfam=\tenit
  \def\sl{\fam\slfam\tensl}%
  \textfont\slfam=\tensl
  \def\bf{\fam\bffam\tenbf}%
  \textfont\bffam=\tenbf \scriptfont\bffam=\sevenbf
   \scriptscriptfont\bffam=\fivebf
  \def\tt{\fam\ttfam\tentt}%
  \textfont\ttfam=\tentt
  \tt \ttglue=.5em plus.25em minus.15em
  \normalbaselineskip=12pt
  \let\sc=\eightrm
  \let\big=\tenbig
  \setbox\strutbox=\hbox{\vrule height8.5pt depth3.5pt width\z@}%
  \normalbaselines\rm}
\def\eightpoint{\def\rm{\fam0\eightrm}%
  \textfont0=\eightrm \scriptfont0=\sixrm \scriptscriptfont0=\fiverm
  \textfont1=\eighti \scriptfont1=\sixi \scriptscriptfont1=\fivei
  \textfont2=\eightsy \scriptfont2=\sixsy \scriptscriptfont2=\fivesy
  \textfont3=\tenex \scriptfont3=\tenex \scriptscriptfont3=\tenex
  \def\it{\fam\itfam\eightit}%
  \textfont\itfam=\eightit
  \def\sl{\fam\slfam\eightsl}%
  \textfont\slfam=\eightsl
  \def\bf{\fam\bffam\eightbf}%
  \textfont\bffam=\eightbf \scriptfont\bffam=\sixbf
   \scriptscriptfont\bffam=\fivebf
  \def\tt{\fam\ttfam\eighttt}%
  \textfont\ttfam=\eighttt
  \tt \ttglue=.5em plus.25em minus.15em
  \normalbaselineskip=9pt
  \let\sc=\sixrm
  \let\big=\eightbig
  \setbox\strutbox=\hbox{\vrule height7pt depth2pt width\z@}%
  \normalbaselines\rm}
%
%
\def\tenmath{\tenpoint\fam-1 } % use after $ in ninepoint sections
\def\tenbig#1{{\hbox{$\left#1\vbox to8.5pt{}\right.\n@space$}}}
\def\ninebig#1{{\hbox{$\textfont0=\tenrm\textfont2=\tensy
  \left#1\vbox to7.25pt{}\right.\n@space$}}}
\def\eightbig#1{{\hbox{$\textfont0=\ninerm\textfont2=\ninesy
  \left#1\vbox to6.5pt{}\right.\n@space$}}}
%
% macros for verbatim scanning (from Knuth's manhdr format)
%
\chardef\other=12
\def\ttverbatim{\begingroup
  \catcode`\\=\other
  \catcode`\{=\other
  \catcode`\}=\other
  \catcode`\$=\other
  \catcode`\&=\other
  \catcode`\#=\other
  \catcode`\%=\other
  \catcode`\~=\other
  \catcode`\_=\other
  \catcode`\^=\other
  \parskip=0pt
  \obeyspaces \obeylines \tt}

{\obeyspaces\gdef {\ }} % \obeyspaces now gives \ , not \space
\outer\def\begintt{$$\let\par=\endgraf \ttverbatim%\parskip=0pt
  \catcode`\|=0 \rightskip-5pc \ttfinish}
{\catcode`\|=0 |catcode`|\=\other % | is temporary escape character
  |obeylines % end of line is active
  |gdef|ttfinish#1^^M#2\endtt{|vbox{#2}|endgroup$$}}

\catcode`\|=\active
{\obeylines \gdef|{\ttverbatim \spaceskip\ttglue \let^^M=\  \let|=\endgroup}}
%\def\ttspace{{\tt\hskip\ttglue}}

\def\vrt{\hbox{\tt\char`\|}} % vertical line
\def\dn{\hbox{\tt\char'176}} % downward arrow
\def\up{\hbox{\tt\char'136}} % upward arrow
\def\]{\hbox{\tt\char`\ }} % visible space
\def\\{\hbox{\tt\char'134}} % visible backslash

% Counts one through 9 can be used for `subpage'
% numbers (we're told).  I'm going to depend on this.

\newcount\levelcount
\newbox\sectnum
%\newswitch{toc}\let\Z=\relax\def\SP{ }
\newif\iftoc\let\Z=\relax\def\SP{ }
\def\sectionlevel#1.#2{\message{Sect, level #1:`#2'}
	\if1#1\goodbreak\bigskip%	Level 1 section
	\else\ifhmode\goodbreak\bigskip%	or lower if not vert mode.
		\else\nobreak\smallskip\nobreak\fi\fi % Skip a bit
	\advance\count#1by1
	\levelcount=#1
	\loop\advance\levelcount by1 % All lower levels get zeroed
		\ifnum\levelcount<10\count\levelcount=0\repeat
	\levelcount=1
	\setbox\sectnum=\hbox{\number\count1}
	\global\def\Sectnum{\number\count1}
	\loop\ifnum\levelcount<#1		% Glue together higher levels
		\advance\levelcount by 1
		\ifnum0=\count\levelcount % Higher level 0 => 1
			\set\count\levelcount=1\fi
		\setbox\sectnum=\hbox
			{\unhbox\sectnum.\number\count\levelcount}
		\global\edef\Sectnum{\Sectnum.\number\count\levelcount}
	\repeat
	\setbox0=\hbox{\if1#1\bf\else\it\fi\copy\sectnum.\hskip1em#2}
	\def\text{#2}\def\level{#1}
		\line{\box0\hfil}
		\nobreak\smallskip
		\nobreak\ignorespaces % Should leave me in v mode
	\iftoc %
	{\let\the=0\xdef\ixout{\write\inx{\Z\level!\Sectnum.\SP!\text!\SP\the\pageno.}}\ixout}\ignorespaces%
	\fi\ignorespaces}
\newcount\Siz
\newwrite\inx
\def\Appendix#1#2{
	\vfil\eject\centerline{\twelveb Appendix #1}
	\centerline{\twelveb #2}
	\iftoc %
	{\let\the=0\xdef\ixout{\write\inx{\Z1!#1.\SP!#2!\SP\the\pageno.}}\ixout}\ignorespaces%
	\fi\ignorespaces
}

\def\DoContents{\toctrue\immediate\openout\inx=TOC}
\def\PrintTOC{\vfill\supereject\immediate\closeout\inx
	\def\LeadFill{\leaders\hbox to 1.5em{\hss.\hss}\hfill}
	\def\Z##1!##2!##3!##4.{
	    \Siz=##1\count1=\Siz
	    \advance\Siz by 1\multiply\Siz by \count1
	    \divide\Siz by 2\advance\Siz by -1
	    \dimen0=1em\multiply\dimen0 by \Siz
	    \Siz=##1\advance\Siz by 1
	    \dimen1=1em\multiply\dimen1 by \Siz
	    \line{\hskip\dimen0\hbox to \dimen1{##2\hfil}{##3}\LeadFill\hbox{##4}}}
	\centerline{\titlefont Table of Contents}\vskip .75in
	\global\count0=-1
	\input TOC}

\newcount\notecount
\def\resetnotes{\notecount=0}
\resetnotes
\def\note#1{\advance\notecount by1
    {\eightpoint\footnote{\tenpoint$^{\the\notecount}$\eightpoint}{\eightpoint #1}}}

\def\Section#1{\sectionlevel 1.{#1}}
\def\SubSection#1{\sectionlevel 2.{#1}}
\def\SubSubSection#1{\sectionlevel 3.{#1}}

\parskip=5pt plus 2pt minus 1pt
\topskip=10pt plus 2pt
\font\csc=cmcsc10
\font\twelveb=cmbx10 scaled \magstep1
\def\Unix{{\csc Unix{}}}

\def\Month {\ifcase\month Yuletide
\or January
\or February
\or March
\or April
\or May
\or June
\or July
\or August
\or September
\or October
\or November
\or December
\else Randomember \fi}
\def\Year {{\count0=\year \advance\count0 by -1900 \number\count0}}
\def\Date {\Month \number\day, \number\year}
\tenpoint
\catcode`@=12
%
% That's the macro package so far.
%
