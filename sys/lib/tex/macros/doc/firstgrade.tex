% This is the "TEX Beginner's Manual" source text

\font\ninerm=cmr9  \font\eightrm=cmr8  \font\sixrm=cmr6 % \font\mc=cmcsc
\font\ttt=cmtt9

\def\TeX{T\kern-.1667em\lower.5ex\hbox{E}\kern-.125em X}
\def\TEX{T\kern-.1667em\lower.5ex\hbox{E}\kern-.125em X}
\def\La{L\kern-.1667em\raise.5ex\hbox{a}\kern-.1em}

\magnification=1200

\def\undertext#1{$\underline{\hbox{#1}}$}
\def\\{{\tt \char'134}}
\def\up{{\sy \char'42}}
\def\down{{\sy \char'43}}
\def\caret{{\char'17}}
\parskip 10pt plus 1pt
\parindent 0pt

\def\cs{{\sl control sequence}} 
\def\css{{\sl control sequences}} 
\def\Css{{\sl Control sequences}}
\def\lbr{{\tt \char'173}}  \def\rbr{{\tt \char'175}}  \def\rrbr{{\tt \char'175\char'175}}
\def\und{\tt \char'137}
\def\til{\tt \char'176}
\def\sup{\tt \char'136}
\pageno=-1
\null\vfill
\centerline {\bf A Beginner's Preface}

\vfill
The beginner can easily be confused by the different
versions of \TeX82 that seem to exist on different computers
and by the conflicting claims for different {\sl Macro Packages\/}
that are supposed to making \TeX\ `user friendly'.
This preface attempts to clarify the situation.

In the first place, there is only one official version of \TeX82 and users
are specifically cautioned not to make any changes to the basic program
itself.  A `{\sl change file\/}' mechanism is provided to allow
the program to be compiled differently as dictated by the hardware and
system software constraints that exist at any particular computer
installation, but these changes should not modify \TeX's formatting 
capabilities.

It is customary, and indeed desirable, to supply \TeX\ with a fairly large
amount of additional information, either in the form of a separate
input file or as information that is preloaded with \TeX\ so that
it becomes available automatically when the \TeX\ program is used.  Such
files are usually called {\sl macro packages\/} and you, as a user, may
have to concern yourself with the number and kinds of {\sl macro
packages\/} that are available at your installation.

{\sl Macro packages\/} are used to supply \TeX\ with  several quite different
kinds of information, much of it being fairly standard and of no immediate concern.
{\sl Macro packages\/} often assign values to a fair number of the approximately
eighty \TeX\ {\sl primitives\/} that can be so preassigned and you may wish
to assign different values to these.
Of major concern are the fairly large number of {\sl
control sequences\/} that are predefined.  These {\sl control
sequences\/} attach names to certain logical combinations of selected {\sl
primitives\/} (and of other {\sl control sequences\/}) that together perform
frequently-used formatting functions. The availability of these predefined sequences
can make your task very much easier when you wish to do the
specific things for which they were designed.

This manual is written on the assumption that you will be using the {\tt
PLAIN.TEX} {\sl macro package}.  This package simplifies many formatting
tasks without interfering with the separate use of any of \TeX's {\sl
primitives\/}.  Having mastered plain \TeX\ to the extent that it is
explained in this manual, you may later want to define some additional
{\sl control sequences\/} that will help you in specifying the
formatting conventions that you employ.

Some {\sl macro packages} are written from quite a different point of view
from that used for plain \TeX,
where the aim is to make certain global formatting conventions very easy to
use at the expense (and sometimes with the deliberate intent) of making it more
difficult for other formatting conventions to be specified.  Some of
these packages actually redefine terms that \TeX\ uses for {\sl primitives\/}
so that these {\sl primitives} can no longer be called directly.
If one of these packages completely meets your needs then you should
use it, but beware.

One final word---there is a certain amount of circularity in the language
used in this manual. If you do not understand everything the first time
through, plow straight ahead until you bog down and then start all
over again at the beginning. You will be surprised how much clearer
everything will be on the second reading.

\eject

\pageno=1
\null \vfil
\centerline{\bf First Grade \TeX} 
\vskip .1in 
\centerline{\bf A Beginner's \TeX\ Manual}
{\bf Introduction}

This is an introductory ready-reference \TeX82 manual for the beginner
who would like to do First Grade \TeX\ work.
Only the most basic features of the \TeX\
system are discussed in detail.  Other features are summarized in an
appendix and references are given to the more complete documentation
available elsewhere.

\TeX\ is a computerized typesetting system.  As \TeX\ is normally used,
the original text is typed (into an input file) very much as it would be
typed for submission to an old-fashioned printer except that this input file
must now contain all of the instructions that are needed to describe the
desired format for the printed output.  Given such a description, still in
rather general terms, the \TeX\ compiler is able to specify in precise
detail the font (i.e., the size and kind of typeface)
and the location for each character that is to be printed.
Your final output can meet the very best publishing standards.

Defining book quality text is not an easy task.  While the \TeX\ system
takes care of many of the tedious details, the wealth of facilities that
\TeX\ provides can be very confusing to the beginner and sometimes even to
the experienced user.  These facilities include the handling of such
matters as: 1)~ligature replacements (for example, fi for f{i}),
2)~kerning (different spacings between certain letter pairs), 3)~automatic
hyphenation, 4)~line justification, 5)~centering, 6)~flushing right or
left, 7)~tabular aligning, 8)~the formatting of complicated mathematical
expressions, 9)~section and page numbering, 10)~the introduction of
running heads, 11)~the numbering and placing of footnotes, and 12) the
preparation of a table of contents and an index, to name but a few.

It will be assumed that the reader is already acquainted with the use of a
computer and with at least one text editor that can be used while typing
the \TeX\ input file.  One minor warning at this point: Do not use an
editor that requires or leaves its own special formatting marks (line numbers,
word processing commands, etc.)
in the
file that it produces, unless your version of \TeX\ has been specially tailored to 
tolerate them. Such marks will be assumed to be a part of the text
by \TeX . The text in the input file should be broken up into
reasonably short lines.  \TeX\ will ignore 1)~the way you break your
paragraphs into lines, 2)~extra
spaces between words, and 3)~extra blank lines between paragraphs,
although it does accept one or more blank lines as indicating a paragraph break.

\TeX\ is usually preloaded with one or more special files that define many
very useful commands.  This manual assumes that you will be using a basic
file called {\tt PLAIN.TEX}, and \TeX\ when so loaded will be refered to
as {\sl plain} \TeX . We will not discuss the details as to how one
actually types the input information, how this is saved as a file on the
system, how one evokes the \TeX\ compiler, and how one instructs the
computer system to print the final document, as these features are highly
system dependent.

\eject
\noindent {\bf Toward Book Quality}

Book quality text differs from ordinary typing in a number of simple but
far from trivial ways.  In the first place, a distinction is made between
the upper case O and the number 0 and between the lower case l and the number 1,
then there is a distinction between hyphens, dashes, and the minus sign.
Also, your terminal should contain two varieties of `single' quotes and
when you want a ``double'' quote you simply type two of the appropriate
single quotes and \TeX\ will space these correctly, as shown below.
$$\vbox{\halign {\hfil #\hfil\quad &\hfil #\hfil\quad &\hfil #\hfil\quad&
\hfil #\hfil\quad &\hfil #\hfil\quad&\hfil #\hfil\quad \cr
Name&Hyphen&En-dash&Em-dash&Minus sign&Double quotes\cr
To print&-&--&---&$-$&``text''\cr
You type&\tt -&\tt -{}-&\tt -{}-{}-{}&$\$${\tt -}$\$$&\tt `{}`text'{}'\cr}}$$
A word about hyphenation---\TeX\ can usually be depended upon to use
hyphens correctly (when this is necessary to achieve right justification) and
to avoid excessive hyphenation (by increasing or slightly decreasing
the inter-word spacings within pre-assigned limits).
There are exceptions,
and there are words like pres-ent and
pre-sent where the positioning of a hyphen depends upon how the word is
used.  When \TeX\ either refuses to hyphenate or makes a mistake, you can
gain hyphenation control over the word in question by inserting one or
more {\sl discretionary hyphens\/} at acceptable locations, using the {\sl
control sequence\/} `{\tt \\-}', as an example in the word {\tt re\\-cord}
or {\tt rec\\-ord}.  These will be ignored when not needed and they will
print as normal hyphens when needed.

\TeX\ provides for the automatic substitution of ligatures when available,
replacing f\/f by ff, f\/i by fi, and f\/l by fl, unless
you specifically disallow it by typing either {\tt \hbox\lbr f\lbr \rbr f\rbr}
to get \hbox{f{}f} or {\tt f\\/f} to get f\/f (slightly more space between).

A nice feature of plain \TeX, is the ease with
which special accents and a few special letters may be produced.
$$\vbox{\tabskip 1em plus 2em minus .5em
\halign{\hfil #\hfil &&\hfil #\hfil \cr
Name&grave&acute&``hat''&umlaut&tilde&``bar''&dot\cr
To print&\`a&\'e&\^o&\"u&\~n&\=y&\.p\cr
You type&{\tt \\`a}&{\tt \\'e}&{\tt \\\^{}o}&{\tt \\"u}&{\tt
\\\~{}n}&{\tt \\=y}&{\tt \\.p}\cr
\noalign{\bigskip}
Name&check&breve&long&tie-after&cedilla&bar-under&dot-under\cr
To print&\v s&\u\i&\H \j&\t\i u&\c c&\b k&\d h\cr
You type&{\tt \\v s}&{\tt \\u\\i}&{\tt \\H\\j}&{\tt \\t\\i u}&{\tt
\\c c}&{\tt \\b k}&{\tt \\d h}\cr
\noalign{\bigskip}
To print&\oe ,\OE&\ae ,\AE&\aa ,\AA&\o ,\O&\l ,\L&\ ?`\qquad !`&\ss\cr
You type&{\tt \\oe,\\OE}&{\tt \\ae,\\AE}&{\tt \\aa,\\AA}&{\tt
\\o,\\O}&{\tt \\l,\\L}&\tt\ ?{}`\quad !{}`&{\tt \\ss}\cr}}$$

Note that {\tt \\i} and {\tt \\j} give dotless i and j for use with accents.
All of the above
special symbols and characters work equally well in roman, italic,
and bold type fonts.
A few do not work in the fixed-width typewriter font.
Many other symbols, mainly mathematical, are listed in the Appendix.
{\bf Special Symbols}

Plain \TeX\ assigns special meanings to ten infrequently used but normally
available typewriter symbols. These symbols are used to simplify the task
of issuing commands to \TeX.  Should you wish these symbols to be printed
in your final document they must appear in your input file as shown below.
$$\vbox{\halign{\tt \hfil #\hfil\quad &\tt \hfil #\hfil\quad &
\vtop{\hsize=26em\strut #\strut}\cr
\rm Symbol&\rm To print, type&\qquad\quad Special \TeX\ meaning when used directly\cr
\noalign{\smallskip}
\\&\$\\backslash\$& Used to indicate the start of a \TeX\ {\sl
control sequence}, usually refered to as a {\sl backslash}\cr
\noalign{\smallskip}
\lbr\ and \rbr& $\$$\\\lbr$\$$ and $\$$\\\rbr$\$$&
Grouping symbols, to indicate range of action of a {\sl control sequence} to the enclosed text\cr
\noalign{\smallskip}
\$&\\{}\$& Used to initiate and to terminate a portion of the text that is to be in Mathematics Mode\cr
\noalign{\smallskip}
\&&\\{}\&& Alignment tab, used to delineate fields within a table \cr
\noalign{\smallskip}
\#&\\{}\#& Parameter, used to signify a field within a table or in a \cs\cr
\noalign{\smallskip}
\sup{}&\\\sup{}$\{\}$&Superscript (also used as an accent, see previous page)\cr
\noalign{\smallskip}
\und&\\{\und}& Subscript\cr
\noalign{\smallskip}
\%&\\\%& Comment symbol, the rest of the line is ignored by \TeX \cr
\noalign{\smallskip}
\til{}&\\\til{}$\{\}$& 
Used to introduce a single space in locations where a line break is not to
be allowed.\cr}}$$
{\bf Issuing Commands to \TeX}

Commands to \TeX, as distinguished from the text itself, consist of the
{\sl backslash}, `{\tt \\}', then the name of the command (without an
intervening space) and then, in some cases, the parameters that are
associated with the command.  The command name can be of two types, either
1) a single non-alphabetic character,
or 2) one or more alphabetic characters that must be
terminated by a space if the name is to be followed by other alphabetic
characters.  These alphabetic characters are often in lower case, but
please note that the correct case, as specified, must always be used.
For example, `{\tt \$\\Omega\$}' produces `$\Omega$', while
`{\tt \$\\omega\$}' produces `$\omega$'. 
Also note the basic distinction between non-alphabetic-character command names which
are not to be followed by a space (unless a space is actually wanted
in the output) and alphabetic-character command names which require a terminating space
if they are followed by other alphabetic characters.

Commands are of two types. Over 300 commands, called {\sl primitives},
are a part of \TeX 's built-in vocabulary. Then
there are {\sl control sequences}, sometimes called {\sl macros}, that are
constructed from these {\sl primitives\/} (or from other {\sl control
sequences\/}).

Many of the {\sl primitives\/} are normally used only to define \css\/\ and
need not concern the beginner, but others specify
basic characteristics of \TeX 's output, such as, for example, the
page size (`{\tt \\hsize=6.5in \\vsize=8.9in}'),
the desired margins, and the spacing between paragraphs
(`{\tt \\parskip=10pt plus 1pt}').  All
of these latter are preassigned values in plain \TeX\
although they can be changed by the user.


{\sl Control sequences\/} are used to simplify the typing of commands to
\TeX, since they are
shorter and much easier to remember than the sequence of {\sl primitives}
that they specify.
 Many very useful {\sl control sequences\/} are to be found in the
file {\tt PLAIN.TEX} and these {\sl control sequences\/} are, in effect, added
to the basic vocabulary that \TeX\ understands when this file is preloaded
with \TeX . In addition, there exist a number of special collections of
{\sl macros\/} that \TeX\ experts have found useful for special purposes
such as writing business letters, preparing internal
memoranda, and preparing manuscripts for publication. The very first line
appearing in your main input file may well be the control sequence \ \\{\tt
input} \ followed by the name of the file that contains the desired macro
collection. Incidentally, \TeX\ interprets a file name that is so used
without a
file name extension to mean a file with the extension of {\tt .TEX}, for example,
{\tt \\input MYMACS} will cause the file {\tt MYMACS.TEX} to be loaded.

Plain \TeX\ understands well over 900 {\sl control
sequences\/} but many of these are self-evident from their names and
others fall into a relatively few catagories, so that they can be easily
learned when needed.  Some of the more important of these are listed in
the Appendix.

\noindent {\bf Fonts}

\TeX\ provides for the use of as many as 256 different fonts, each
containing as many as 256 different characters. Plain \TeX, as loaded,
has 16 fonts that are assigned special control-sequence names.
The default font, called roman, is requested
by the use of the {\sl control sequence\/} `{\tt \\rm}'.
If you wished the entire text to be typed in boldface you would type
`{\tt \\bf}' before your text.  If, on the other hand, you
want a single word or a phrase to appear in a different font you should
use the grouping symbols `{\tt \lbr}' and `{\tt \rbr}' to delimit the range.

For example, typing \ `{\tt to be \lbr\\bf bold\rbr\ is to \lbr\\sl
emphasize\rbr\ something}' \ will produce `to be {\bf bold} is to {\sl
emphasize\/} something'.
Incidentally, an {\sl italic correction} command `{\tt \\/}' introduces a
little extra space when used after letters in the the {\tt \\it} and {\tt \\sl}
fonts to compensate for a change in slope of the letters.  Typing
`{\tt \lbr\\it italicized\\/\rbr} text' produces `{\it italicized\/} text' as
compared to `{\it italicized} text' when the correction is not used.

The following {\sl control sequences\/} are for the type fonts as shown:

$$\vbox{\tt \halign{\hfil#\hfil&& \quad \hfil #\hfil \cr
\\rm&\\sl&\\it&\\tt&\\bf\cr
\rm Roman&\sl Slanted&\it Italic&\tt Typewriter&\bf Boldface\cr
}}$$

Type fonts also come in different sizes. To demonstrate, if you type:

\halign{#\cr
{\tt \lbr\\{tenrm smaller and}\rbr\ \lbr\\{ninerm smaller and}\rbr\ 
\lbr\\{eightrm smaller and}\rbr}\cr
{\tt \lbr\\{sevenrm smaller and}\rbr\ \lbr\\{sixrm smaller}\rbr}{\rm
 ,\quad you get} \hfil\cr}

{\tenrm smaller and} {\ninerm smaller and} {\eightrm smaller and}
{\sevenrm smaller and} {\sixrm smaller}, all lined up to a common {\sl
baseline}.

This manual uses {\sl 10~point\/} type, as called for by {\tt PLAIN.TEX},
and defined to be
{\tt \\tenrm}, except that the author has had everything
magnified by 1.2 (for beginners), by having the command `\\{\tt
magnification=1200}' \ appear ahead of any text in the input file. So the
actual
sizes, in the above example, are all multiplied by 1.2.  You may have also
noticed that this manual uses `{\sl slanted type\/}' for emphasis and `{\tt
typewriter type}' for things that you are told to type and for the messages
that the computer displays.

The characters within a font, including those not on your keyboard, can be
gotten by typing `{\tt \\char<number>}'\ where {\tt <number>} is the
decimal number of the character position, or by typing `{\tt \\char'<octal
number>}'\ where {\tt <octal number>} is an octal number.  For example, typing
`{\tt \\char'35}' produces {\char'35}.  Incidentally, the normal
alphanumeric characters are at their {\sl ascii} positions regardless of
the code that your computer may use externally.

Another way to get symbols that are not on your keyboard is to define
{\sl control sequences} for them. For example, {\tt PLAIN.TEX} defines 
`{\tt \$\\ne\$}',
`{\tt \$\\le\$}', and `{\tt \$\\ge\$}' for `$\ne$', `$\le$' and
`$\ge$', respectively.
Over 300 such {\tt PLAIN.TEX} definitions are listed in the Appendix.


\noindent {\bf Dimensions and Keywords}

\TeX\ understands a variety of dimensional units as
{\sl keywords} that are used without the `{\tt \\}'. These 
convey fixed meanings to \TeX\ when they are used in the proper context.
These units and their meanings are:

$$\vbox{\halign{\tt \hfil#\hfil\quad&#\quad&#\qquad\qquad\hfil
&\tt \hfil#\hfil\quad&#\hfil\quad&#\cr
{\rm Unit}&Meaning&Per inch&{\rm Unit}&Meaning&Per inch\hfil\cr
\noalign{\smallskip}
pt&point&72.27&mm&millimeter&25.4\cr
pc&pica&6.023&dd&Didot point&67.54\cr
in&inch&1&bp&big point&72\cr
cm&centimeter&2.54&
sp&scaled point&4736286.72\cr}}$$

Sometimes it is convenient to use dimensions that are relative to the
character sizes in the font being used at the time. Two such units are
the {\tt em} and the {\tt ex}.
The {\tt em}, used for measuring horizontal distances, is, by tradition,
the width of the upper case `M'  but is an arbitrarily assigned value in
\TeX\ that is the width of a `quad'.  The {\tt ex} is used for measuring
vertical distances and is approximately the height of a lower case `x'.

{\bf To Use \TeX }

Having written a simple input file, say, {\tt MYFILE.TEX}, which contains the desired
text together with a few {\sl control sequences\/} that define your
wishes, you may now call on \TeX\ to operate on your file.
On some computers you do this 
by typing:  \ \ `{\tt run~tex;MYFILE}'. \
Since the correct form 
can vary from computer to computer, the best thing to do is to ask someone who
knows what to do on your particular computer.
Usually, \TeX\ will begin by prompting you with a 
double asterisk `{\tt **}', to indicate that you can then type the name of your
input file. 
If, on the other hand, the prompt is a single asterisk
`{\tt *}', (or if you are calling for another file within a file) then you
must precede the file name by the `{\tt \\input}' command thus:  `{\tt
\\input MYFILE}'. The \cs\/\ {\tt \\input} must usually be typed in lower case
and typing `{\tt MYFILE}' implies `{\tt MYFILE.TEX}'.

The input file itself may begin by referencing one or more additional
files (this time using the `{\tt \\input}' command).  These files may
define some special fonts that are to be used and they may contain the
definitions of some specialized, frequently-used, constructs ({\sl
control sequences\/} or {\sl macros\/}) that are used to simplify the
typing of the desired instructions.  Once defined, these constructs can
then be called by name as needed in the text file itself. Caution: Do not load up your
working space with macros that you never use.

The \TeX\ program produces a {\tt DVI} (\undertext{d}e\undertext{v}ice
\undertext{i}ndependent) file, that specifies each character that is
to be printed, together with its font and its exact location in the printed document.
The {\tt DVI} file may require
some further translation before it is acceptable as input to the available
printer but on most systems this further action is usually automatic,
requiring, at most, a carriage return confirmation. 
The {\tt DVI} file is usually saved and it can then be reprinted at a later date,
perhaps on a different printer, if this should be desired.

It should be noted that \TeX\ does not need to know the exact shape
of each character, but it does need to know the overall size and the
details as to how the individual characters may interact with adjacent
characters (resulting in ligature replacements or in kerning). This
information is usually contained in what are known as {\tt TFM} files.  The
printer, on the other hand, does need to have access to the detailed
information as to the exact configuration of each character that is to be
printed and this information is usually contained in what are known as
{\tt PXL} files. The two sets of files must agree as to the fonts that they
describe, or chaos will result.

{\bf Boxing with Glue}

There are a few facts about the inner working of \TeX\ that you will
need to know.
\TeX\ approaches the task of formatting a page of text much as a mason
might build a brick wall, which is to be laid in courses to a fixed
width using bricks that vary in size (and that have to be used in a fixed
order), with limits as to the minimum and maximum amount of mortar that
can be used between bricks.

\TeX\ happens to use different terms, which you should learn to recognize.
Instead of {\sl bricks}, \TeX\
talks about `{\sl hboxes\/}' (initially, the individual
words in your text), which it assembles
out of simple `{\sl boxes\/}' (the individual letters). No {\sl mortar}
is used between letters
so words always look the same.
There may be some
`{\sl kerning}', e.g., there will be less space between an `o' and an `x', as in 
box, than
there will be between two `o's as in book.
\TeX\ then puts these {\sl hboxes\/} together to form
larger {\sl hboxes} (this time, a collection of words), using `{\sl glue\/}', in place of
{\sl mortar} for the inter-word and inter-sentence spacings.
The {\sl glue\/} has properties of being {\sl
stretchable} and, to a lesser extent, {\sl shrinkable}.  This list of 
{\sl hboxes\/} and {\sl glue} can then be expanded or compressed to meet
the `{\sl hsize\/}' dimension that has been preset by {\tt PLAIN.TEX} to 6.5 in, or that you
have changed by typing {\tt \\hsize 4.25in}, for example.  

Plain \TeX\ takes care of such details as putting slightly more space after commas
and after periods and by allowing
these globs of glue to stretch more and shrink less than that allowed for
the normal glue between words. While this is desirable in most cases,
there are times when it is not desired and then you should use the {\sl
backslash-space control sequence}, `{\tt \\ }', (that is a backslash
followed by a space) to guarantee a normal
space, or perhaps use the {\sl tie} symbol,
` $\tilde{ }$ ', in place of the space after
an abbreviation such as in `{\sl Fig.~23\/}' or {\sl Mr.~Smith}, where it is
also desirable to prevent the sentence from being broken between the
abbreviation and the following number or word.

Actually, \TeX\ treats an entire paragraph as a unit and tries to distribute the
text into as many lines as are required with the individual lines meeting
the {\sl hsize\/} requirement without excessive stretching or shrinking.
If this cannot be done, \TeX\ then tries to hyphenate. \TeX\
will usually find several ways that the paragraph can be broken, and it
will then pick the way that it thinks is best.
The paragraph will then be broken up into a sequence of {\sl hboxes}, each of which
should meet the specified line width.

\TeX\ will complain with an `{\tt overfull hbox}' \ message, if it is
unable to meet the tolerances that have been specified.
Your task then is to inspect the offending line or lines to see if there
is some simple way to overcome the difficulty; perhaps a discretionary
hyphen is needed.  Lacking such a remedy, you can force a line break at an
earlier place in the offending line or earlier in the paragraph,
but this is seldom desirable. It is usually better
simply to reconstruct the sentence or paragraph so as to avoid the
trouble. Actually, there are a number of ways that you can anticipate
trouble and provide for it in advance, as you will learn through
experience.

\noindent {\bf An Example}

Perhaps, it is time to stop and show you an example, and what better
example than the title page of this  manual itself, which was typed using plain \TeX.
This example may seem a bit too complicated for a beginner but it does
provide a convenient vehicle to illustrate a number of \TeX's
idiosyncrasies that you need to know about and that are hard to explain in
abstract terms. So do go through it line by line with the explanation that
follows.

My input file is divided into pages, although
this is not required and some installations may not permit this nicety.
\TeX, of course, ignores page breaks in the input, just as it ignores line breaks.

The first part of the input file defines some \css\/\ and contains: \medskip
\vbox{\halign{\tt #\hfil&\hfil #\cr
\\font\\ninerm=cmr9\ \ \\font\\eightrm=cmr8\ \ \\font\\sixrm=cmr6\ \ \\font\\csc=cmcsc\cr
\\def\\TeX\lbr T\\kern-.1667em\\lower.5ex\\hbox\lbr E\rbr\\kern-.125em X\rbr\cr
\cr
\\magnification=1200 \ \%\ This magnifies everything by 1.2. \cr
\\parskip 10pt plus 1pt \ \%\ This puts some empty space between paragraphs\cr
\\parindent 0pt \ \%\ Paragraphs are not to be indented\cr}}

The next part of the input file contains: \medskip
\vbox{\frenchspacing \halign{\tt #\hfil&\hfil #\cr
\\nopagenumbers \%\ The title page is not to be numbered.\cr
\\null\\vskip-46pt \% Put the first line 46 points higher than normal\cr
\\hbox to 6.5truein \lbr November 1983 \\hfil Report No. STAN-CS-83-985\rbr\cr
\ \ \ \% A convenient way to make a box of specific size.\cr
\\vskip .1in \% Skip down 0.1 inch\cr
\\line \lbr Stanford Department of Computer Science\\hfil (Version 1.1)\rbr\cr
\\vfill \% This and similar commands later, will divide the space evenly.\cr
\\centerline\lbr\\bf First-Grade \\TeX\rbr\cr
\\vskip .1in\cr 
\\centerline\lbr\\bf A Beginner's \\TeX\\ Manual\rbr\cr
\\vskip .25in\cr
\\centerline\lbr by\rbr\cr
\\centerline\lbr Arthur L. Samuel\rbr\cr
\\vfill\cr
This manual is based on the publications of Donald~E.~Knuth who originated\cr
the \\TEX\\ system and on the recent work of Professor Knuth and his many\cr
students and collaborators who have helped bring the \TEX82 system to its\cr
present advanced state of development. The \\TEX\\ logo that is used in this\cr
manual is a trademark of The American Mathematical Society. The preparation\cr
of this report was supported in part by National Science Foundation grant\cr\
IST-820/926 and by the System Development Foundation.\cr
\\eject\\end \% \\eject, only, would be used if other pages were to follow.\cr}}
\medskip
So let us consider this example in detail.
We first note that the \cs\/\ `{\tt \\font}' is used to assign names to
several fonts that {\tt PLAIN.TEX} had loaded but did not name.  Actually,
only one of these fonts was used on the title page but it is good practice
to start your input file by naming all of the other  fonts that you intend to
use beside those in plain \TeX's standard set.  Note that this same \cs\/\
would have instructed \TeX\ to copy the font metrics (the dimensional
information that \TeX\ actually needs) from the named file, if this {\sl
font} information had not been preloaded, as it had been by {\tt
PLAIN.TEX}.

Next comes a definition for the \TeX\ logo. You will probably not need to
cause letters to be artificially displaced from their normal positions,
but if you do, here is an example.  It also serves as a model for other
\cs\/\ definitions that you may find occasion to use.  The `{\tt
\\magnification}' {\sl control sequence} has already been explained (under {\bf Fonts}).

The \cs\/\ `{\tt \\parskip=10pt plus 1pt}' is an example of a general class
of \css\/\ that take dimensions.
I could have written it as
{\tt parskip10pt plus 1pt minus 1pt} had I been willing for the space to
be shrunk by as much as 1pt on some occasions and stretched on others.
Incidentally, plain \TeX\ sets
`{\tt \\parindent=20pt}', \ but this has been redefined for this manual by
`{\tt \\parindent 0pt}'. \ \ Note that the use of the `{\tt =}' sign is optional but that
a unit of measure must be specified even when the value is zero.
So much for the first part.

After telling \TeX\ that the title page is not to be numbered, the
information for the title page starts with the \cs\/\ `{\tt \\null}',
which, as its name implies, normally does nothing.  So why use it?  One of
\TeX's idiosyncrasies is that it gobbles up extra spaces that you leave
between words, extra space between paragraphs, and any extra space that
might be left over when it has just introduced a page break.  So, without
the `{\tt \\null}', \TeX\ will assume that the negative space called for by
the `{\tt \\vskip-46pt}' \cs\/\ was left over from a previous page and
simply gobble it up.  The `{\tt \\null}' \cs\/\ starts the new page by
putting almost nothing (actually an {\sl hbox} that contained nothing) on
the first line.  Now since the page has been started, the negative-space
\cs\/\ is honored.  The {\tt \\vskip} {\sl primitive} is an example of a
\cs\/\ that takes a dimension.  Looking down the page, you will see other
examples.

The next \cs, `{\tt \\hbox to 6.5truein}', also takes a dimension, but in this case
using the extra word `{\tt to}'. The dimension in this case
is in {\sl true} inches since I did not want the value to be subject to the
{\tt \\magnification} command.  Two lines below this is another \cs, `{\tt
\\line}', that does essentially the same thing, in the context of this
example, since {\sl \\line} is defined in {\tt PLAIN.TEX} to produce an {\sl \\hbox}
to the current dimension of {\sl \\hsize}, which
happens to be 6.5 true inches.
Were I later to alter
the page width by changing {\sl \\hsize}, I would have to find and
alter all \css\/\ of the first variety if I wanted the lines that they produced
to still line up with the rest of the text, so the {\tt \\line} form is better in
this case.

The text that is to be put in each of these two {\sl hboxes} is enclosed
in {\sl grouping} symbols and contains still another \cs, to wit, {\tt
\\hfil}. This tells \TeX\ where to put the extra space needed to fill out
the text to exactly 6.5 inches, with the text before the {\tt \\hfil}
{\sl flush left} and the text to the right of it {\sl flush right}.
Without this {\tt \\hfil} \cs, \TeX\
complains {\tt (Underfull box (badness 10000)}), and then it would print the
first of these two lines as:\hfil\break
\line{November\hfil 1983\hfil Report\hfil No.\hfil STAN-CS-83-985}

But to continue with the example: {\tt \\centerline} \ is
another \cs\/\ that
requires the use of {\sl grouping} symbols to enclose the text to be
centered, that is, if you want more than the single next character to be
centered.  Then there is an embedded \cs\/\ within the {\sl grouping
symbols} which specifies that the centered text is to be in {\bf bold face
type}.

It can be a bit confusing to observe that some \css\/\ take arguments
within {\sl grouping} symbols while others are embedded within the {\sl
grouping} symbols along with the text that they affect. There is always a
logical reason for this difference in treatment.  When \TeX\ has
interpreted the centering command, it must look ahead to determine how
much text there is to be centered before taking any action, and it expects
to find this information within a pair of grouping symbols.  By way of
contrast, \TeX\ can start putting text in a different font without needing
to know the range over which this command is to be effective. So, if you
want the range of action to be limited, you put the font designation
within the grouping symbols and the former font choice is reinstated when
the closing symbol is reached.  For a more detailed
explanation, see ``The \TeX book'' by Don Knuth.

Next note the use of three {\tt \\vfill} \css\/\ to divide the excess
vertical space that remains available on the page into three equal
portions.  The fact that I used one `{\sl l\/}' in {\tt \\hfil} and two 
in {\tt \\vfill} was deliberate, to call attention to the fact that
both forms work for both horizontal or vertical {\sl fills}.  The
double-{\sl l\/} variety simply is more stretchable than the single-{\sl l\/}
variety and will do all of the stretching if both forms are used together.

Finally, there is some normal text at the bottom of the page and then the
commands {\tt \\eject \\end} that cause \TeX\ to finish off the page, to
close out the {\tt DVI} file and to terminate the session.  Had this page
been a normal text page, I would have used the \cs\/\ {\tt \\bye},
the recommended way to stop, defined as
{\tt \\par\\vfill\\supereject\\end}. This closes the paragraph, checks to see
that there is no text yet to be processed and causes the last page to be
filled out with blank space, if necessary, instead of spreading out the
text to occupy the entire page, as we wanted in this case.

It will help your understanding if you make
a file containing a title page for some paper that you have written or
intend to write by copying this sample and modifying it as needed.  Then
try to run \TeX\ on this file.  Unless you are extremely lucky, you will
be apt to get an error message that you will not understand---so read on.
If you have been lucky you might try changing the page width by putting
`{\tt \\hsize=4truein}' at the top of your input file, or better yet start \TeX\
so that you get the `{\tt **}' prompt and then type `{\tt \\hsize=4truein}',
a space, and then {\tt \\input} and the name of your file.

{\bf Understanding Error Messages}

Most of the errors that the beginner is apt to make, other than simple
typos, will have to do with 1)~missing or misplaced {\sl grouping} symbols,
2)~attempts to use \css\/\ in the wrong context, and 3)~a failure to understand
some of the principles that \TeX\ uses in deciding how to break the text
into lines and these lines into pages with the result that you ask it to do
some quite impossible things.

\TeX\ will report that \ ``{\tt \\end occurred inside group at level 1}''
\ if a single `\rbr' is missing and \ ``{\tt too many \rbr's}'' \ if the
contrary condition exists. Usually, it will have reported all sorts of
strange things due to a mismatch of {\sl grouping} symbols.  So find and
fix such troubles before checking further.  A somewhat similar problem may
arise if you do not having matching `{\tt\$}' signs. You will usually get the
message ``{\tt Missing \$ inserted}'' but only after some otherwise normal
text has been improperly handled.

\TeX\ may get confused if you tell it to do something out of context, for
example, to do something that relates to the formatting of words into
sentences when it is busy putting sentences onto pages. 
 At any given time,
\TeX\ will be operating in any one of six different {\sl modes}, as will
be explained  below.
If you type `{\tt h}' in response to an error message that you do not
understand, \TeX\ will try to help you by explaining what it thought that it
was doing. Often the help message will suggest a way to recover from the error.

Finally, you need to know that
\TeX\ decides how to format the text by
considering the amount of `{\sl badness\/}' that is charged
against each possible arrangement. {\sl Penalties\/} are assigned to each
possible line break point, usually automatically, although you can specify
a penalty if you wish.  These {\sl penalties\/}
measure the undesirability of a break occurring at each particular place.
{\sl Demerits\/} are assigned  1)~to each line for different features as
to its departure from the ideal, 2)~for the presence of adjacent lines
that differ from each other too much in their inter-word spacing
or that
are too similar in certain obnoxious ways, for example, in both ending
with a hyphen, and 3)~for poor paragraphing, such as
leaving a {\sl widow\/} word to appear at the top of a new page.

If \TeX\ is unable to find a solution that will meet certain tolerance limits,
it will complain, usually about the `{\tt badness}', and
expect you to do something to fix the difficulty. Should you want \TeX\ to be more
tolerant, you can increase the value of {\tt \\tolerance} from 200 to, say, 1000.

{\bf The Six Modes}

When \TeX\ is processing your text, it is constantly switching between six
different {\sl modes} of operation.  The details need
concern you only when \TeX\ reports an error on your part
or when you are doing something special like making a table or
displaying an equation.

The two most obvious {\sl modes} are 1)~the {\sl horizontal mode} when
\TeX\ is putting letters together to form words, and when it is putting
spaces or glue between words, in preparation for making {\sl hboxes} after the
line breaks are chosen,
and 2)~the {\sl vertical mode} when it is stacking these
{\sl hboxes} (or already prepared {\sl vboxes}) vertically to go on a
page. \TeX\ switches from {\sl horizontal mode} to {\sl vertical mode} when it
encounters something that clearly indicates that it should be in the {\sl vertical
mode}, such, for example, as 1)~an empty line or the {\sl control sequence}
{\tt \\par}, both of which tell it to start a new paragraph, or 2)~a v-type
{\sl control sequence}, such as {\tt \\vfill} or {\tt \\vskip .1in}, both of which were
used in the example on page~8.  \TeX\ switches from {\sl vertical mode} to {\sl 
horizontal mode} when it encounters an ordinary character or any one of several 
{\sl horizontal mode control sequences} such, for example, as {\tt \\indent} or
{\tt \\noindent} with obvious meanings.

The six {\sl modes} are:
$$\vbox{\halign{#\quad\hfil&#\hfil\cr
\quad Mode&\qquad Used when building\cr
\noalign{\smallskip}
Vertical&Main vertical list for a page\cr
Horizontal&Horizontal list for a paragraph\cr
Internal vertical& Vertical list for a vbox\cr
Restricted horizontal&Horizontal list for an hbox\cr
Math&Mathematical formula for a horizontal list\cr
Displayed math&Math formula on a separate line.\cr}}$$

You signal the entering and leaving of the {\sl math mode} by the use of a
single `{\tt\$}' sign as in `{\tt \$<math~formula>\$}', and the entering and
leaving of the {\sl display math mode} by the use of a pair of `{\tt\$}' signs as in
`{\tt \${}\$<formula~to~be~displayed>\${}\$}'.  The display math mode is also useful
for tables that may not actually involve mathematical material, because it
centers material on the page and introduces a space above and below the
material.

A typical use of {\sl restricted horizontal} mode occurs when you center some
material on the page by writing `{\tt \\centerline\lbr\\bf\ First Grade \\TeX \rbr}'.
\TeX\ will be in the {\sl vertical} mode on encountering the
`{\tt \\centerline}' \ control
sequence; it will shift to the {\sl restricted horizontal} mode while
processing the  material delineated by the braces and then back to
the {\sl vertical} mode to add this to the material that is to form the page.
Restricted horizontal mode differs from regular horizontal mode because {\tt \\par}
and {\tt \$\$} do not end a paragraph or start a display; a single line is
always produced.
{\bf Making Tables}

\TeX\ provides two different methods for typesetting tables.  The first
method apes the typewriter's tab-setting ability, and is preferred for very
large tables that may extend over several pages.  Tabs, set by the {\tt
\\settabs} command, are preserved when you introduce some ordinary text
(or even with a {\sl grouping-symbol\/}-delineated insertion of a table
with a different set of tab settings). This makes it possible to have
several different tables, on the same page or even on different pages,
that are similar in their columnar alignments.  The disadvantage is that
you must specify where these tab settings are to be, either by giving the
number of columns, assuming that they are to be evenly spaced, or by
supplying a set of sample entries made up of the largest entry for each
individual column.  This works well for numerical tables but it
can be a bit difficult to do if the table contains a mixture of fonts and
if the characters used are of varying widths.

The more general method, which we will consider first, gives \TeX\ the
task of setting the tab positions for each column to meet the possibly
varying maximum widths of the material that goes into these different
columns.  This {\tt \\halign}\ method allows you to achieve an optimum
design for each table, but with a possible lack of uniformity
between different tables. Most of the tables in this
manual were made using this method. Having made these tables,
were I to find it necessary to replace an entry or even to add another column,
I could depend upon \TeX\ to make the necessary adjustments.  This method
should not be used for large tables that span many pages, as \TeX\ must read the entire table in
order to determine the column widths.

{\bf The `\\halign' Alignment Method}

The table near the bottom of page~4 will be used as an example. 
This was typed as:

\vskip 10pt
\vbox{\tt \halign{#\hfil\cr
\$\$\\vbox \lbr\\tt \\halign \lbr\\hfil \#\\hfil \&\& \\quad \\hfil \#\\hfil \\cr\cr
\\\\rm\& \\\\sl\& \\\\it\& \\\\tt\& \\\\bf\\cr\cr % \\\\produces a backslash
\\rm Roman\& \\sl Slanted\& \\it Italic\& \\tt Typewriter\& \\bf Boldface\\cr\cr
\rrbr\$\$ \cr}}

Note that the first line and the last line are concerned with setting up
the table, while the material that actually goes into the table appears in
the second and third lines only.  Had there been more lines to the table,
this portion would be have been expanded accordingly.

The `{\tt \$\$}' at the start and end of this input-text sample is used
here to center the table and to introduce some space above and below it.
As explained earlier, the use of the double-dollar-sign construct normally
causes \TeX\ to enter the {\sl displayed math} mode.  By following the
`{\tt \$\$}' with the {\tt \\vbox} control sequence, the other math-mode
effects are temporarily suspended so that the table will appear in normal
non-math characters.  The {\tt \\vbox} has the additional effect of
preventing \TeX\ from splitting the table between pages.  The `{\tt
\lbr}', that follows, and the very last `{\tt \rbr}' on the last line are
the {\sl grouping} symbols that define the material that is to be put in
the {\sl vbox}.  Since much of the material is to be in `{\tt typewriter}'
type, we start with the `{\tt \\tt}' \cs, and then comes the `{\tt
\\halign}' control sequence that requires a second set of matching braces
to define the scope of the table. After the opening brace there follows a
series of statements, each ending with the \cs\ `{\tt \\cr}'.

The rest of the first line contains the first statement, sometimes called
the {\sl preamble}. This is a {\sl template} that specifies how the
different columns of the table are to be treated. It does this by
referring to the material that is to go into any specific column by the
parameter symbol `{\tt \#}', while the different columns are separated by
the symbol `{\tt \&}'.  

After the {\tt \\halign \lbr} and before the first {\tt \&} we find {\tt
\\hfil \#\\hfil}, which tells \TeX\ that the material in the first
column is to be centered (by putting the same amount of stretchable
glue on each side).
Normally, this would be followed by a
single {\tt \&} symbol, then another column specification and so on for
as many columns as desired, each of which could differ in some way from
the other columns, finally terminating with a {\tt \\cr}.  In this case,
all of the subsequent columns are to be treated the
same and \TeX\ provides a shortcut, this being the use of two adjacent {\tt
\&} symbols to tell \TeX\ to use as many columns as are later supplied, using
the same pattern that appears between the {\tt \&\&} and the terminating {\tt \\cr}.
A {\tt \\quad} of extra space is put before the second and subsequent
columns, to keep the columns separate.
Incidentally, the {\tt \&} symbol
has the useful property of ignoring any extra spaces that immediately
follow it, so I have added spaces after these symbols (as well as after
the \css\/) to make the example easier to read.

After the {\sl preamble} there follow the two lines of data (each ending with a
{\tt \\cr}) for the two lines of the resulting table. There is nothing
unusual about the first text line except for the use of backslashes in
pairs to tell \TeX\ that the backslashes are to be printed
(after defining `{\tt\\\\}' to mean {\tt\$\\backslash\$}). The
second line is a bit unusual since each column is to be printed in a
different font.  You will note that we have not had to use braces to
delimit the range of action of the font-specifying control sequences. This
is because the {\tt \\halign} control sequence automatically limits the
range of action of control sequences to the individual table entries where
they are used.  If no font specification appears for any specific entry,
then the default font is used. This table has but two lines, but for
longer tables one simply adds as many lines as desired, always ending each
line with a {\tt \\cr}.

A slightly more complicated situation is illustrated by the table near the
bottom of page~2. Here, the amount of material to go on each line is such
that it is difficult to decide how much space to leave between the
different columns.  So we leave this decision to \TeX\ by removing the
{\tt \\quad} command from the {\sl preamble} and by specifying a value for the
stretchable glue that is to be put between columns (and at the left and
right) by a {\tt \\tabskip} \cs.  The table
specification (which follows the usual `{\tt \$\$\\vbox\lbr}') then becomes:
\smallskip {\tt \obeylines \parskip 0pt 
\\tabskip 1em plus 2em minus .5em 
\\halign to \\hsize \lbr\\hfil \#\\hfil \&\& \\hfil \#\\hfil \\cr}
\smallskip
The specification {\tt \\halign to \\hsize} tells \TeX\ to use the 
stretchability and shrinkability of {\tt \\tabskip} so that the width of
the line is the value of {\tt \\hsize}.

We will not be concerned with the details of typing the text material itself but
one additional feature is worth  noting.  Extra vertical space has been inserted
between the different sections of the table by extra lines in the source file
that read `{\tt \\noalign~\lbr\\bigskip\rbr}', where `{\tt \\bigskip}'
\ is defined below.  In general any desired extra (vertical mode) material
may be so introduced, including one or more lines of text.  {\tt
PLAIN.TEX} defines three convenient vertical skip instructions and it is
usually better to use these, for reasons of uniformity, than to specify any
specific {\tt \\vskip}. These control sequences are:
$$\vbox{\halign{#\quad\hfil&#\hfil\cr
\quad Macro&\qquad \qquad Equivalent to\cr
\noalign{\smallskip}
\tt \\smallskip &\tt \\vskip 3pt plus 1pt minus 1pt\cr
\tt \\medskip &\tt \\vskip 6pt plus 2pt minus 2pt\cr
\tt \\bigskip &\tt \\vskip 12pt plus 4pt minus 4pt\cr}}$$
The table on page~3 is an example of a still more
complicated situation in which the material to
go into one column is too long to be contained in one line.
This requires several new features in the {\tt preamble} 
(following `{\tt \$\$\\vbox\lbr\\halign\lbr}')
which reads:
\smallskip 
{\tt \obeylines \parskip 0pt 
\\tt \\hfil \#\\hfil \\quad \& \\tt \\hfil \#\\hfil \\quad \&
\\vtop \lbr\\hsize=26em \\strut \#\\strut\rbr\\cr}
\smallskip
The first thing to note is the use of a {\tt \\vbox}, this time a special
one called {\tt \\vtop}, to contain the text for the last column.  The
{\tt \\vtop} differs from a {\tt \\vbox} in that it is aligned at the
baseline of its top line.
The width of this box is defined by {\tt \\hsize=26em}, a figure that was
determined by cut and try.
The {\tt \\struts} were introduced to make sure that
the space allowed for the {\tt \\vtop} would not be adversely affected if
the first line of text in the box had no tall letters, or if
the last line of text happened to have no descenders,
since {\tt \\vbox}es are normally made only high enough to enclose the material
that they contain.  A {\tt \\strut} produces a zero-width invisible box of
the correct height and depth for the font being used. Incidentally, a
character can be assigned a zero width and still not be invisible. It
would still print, but then it would be over-printed by the next character.

If you want to make tables that are significantly more complicated than
those used so far, you need to be an expert, so you will have to study
The \TeX book. The following example illustrates still other features.
This example is taken from The \TeX
book and attributed to Michael Lesk as published in the Bell
Laboratories Computing Science Technical Report {\bf 49} (1976).
$$\vbox{\tabskip=0pt \offinterlineskip
\def\tablerule{\noalign{\hrule}}
\halign to150pt{\strut#&\vrule#\tabskip=1em plus2em&
  \hfil#&\vrule#&\hfil#\hfil&\vrule#&
  \hfil#&\vrule#\tabskip=0pt\cr\tablerule
&&\multispan5\hfil AT\&T Common Stock\hfil&\cr\tablerule
&&\omit\hidewidth Year\hidewidth&&
 \omit\hidewidth Price\hidewidth&&
 \omit\hidewidth Dividend\hidewidth&\cr\tablerule
&&1971&&41--54&&\$2.60&\cr\tablerule
&&   2&&41--54&&2.70&\cr\tablerule
&&   3&&46--55&&2.87&\cr\tablerule
&&   4&&40--53&&3.24&\cr\tablerule
&&   5&&45--52&&3.40&\cr\tablerule
&&   6&&51--59&&.95\rlap*&\cr\tablerule
\noalign{\smallskip}
&\multispan7* (first quarter only)\hfil\cr}}$$
This table was produced by typing:
\medskip
\vbox{\tt \halign{#\hfil\cr
\$\$\\vbox\lbr\\tabskip=0pt \\offinterlineskip\cr
\\halign to150pt\lbr\\strut\#\& \\vrule\#\\tabskip=1em plus2em\& \\hfil\#\& \\vrule\#\& \cr
\\hfil\#\\hfil\& \\vrule\#\& \\hfil\#\& \\vrule\#\\tabskip=0pt\\cr \\noalign\lbr\\hrule\rbr\cr
\& \& \\multispan5\\hfil AT\\\&T Common Stock\\hfil\& \\cr \\noalign\lbr\\hrule\rbr\cr
\& \& \\omit\\hidewidth Year\\hidewidth\& \& \\omit\\hidewidth Price\\hidewidth\& \& \cr
 \\omit\\hidewidth Dividend\\hidewidth\& \\cr \\noalign\lbr\\hrule\rbr\cr
\& \& 1971\& \& 41--54\& \& \\\$2.60\& \\cr \\noalign\lbr\\hrule\rbr\cr
\& \&    2\& \& 41--54\& \& 2.70\& \\cr \\noalign\lbr\\hrule\rbr\cr
\& \&    3\& \& 46--55\& \& 2.87\& \\cr \\noalign\lbr\\hrule\rbr\cr
\& \&    4\& \& 40--53\& \& 3.24\& \\cr \\noalign\lbr\\hrule\rbr\cr
\& \&    5\& \& 45--52\& \& 3.40\& \\cr \\noalign\lbr\\hrule\rbr\cr
\& \&    6\& \& 51--59\& \& .95\\rlap*\& \\cr \\noalign\lbr\\hrule\rbr\cr
\\noalign\lbr\\smallskip\rbr\cr
\& \\multispan7* (first quarter only)\\hfil\\cr\rrbr\$\$\cr}}

The most obvious difference, between this table and the ones discussed so far,
is the use of so-called `{\sl rules}', to enclose and separate the entries.
You will also note that one entry extends over several columns within the
area defined by the outside {\sl rules\/} and the last entry also extends
over several columns.  Finally, while the numerical items in the Year
column and in the Dividend column seem to be `{\sl flush right}', the
words `Year' and `Dividend' and entry `.95*' are certainly not.

Referring to the listing, we first reset some parameters.  Setting {\tt
\\tabskip=0pt}\ may not be needed since {\sl \\tabskip\/} is usually set
to zero, but this is a precaution.  \TeX\ normally puts some {\sl
tabskip\/} glue before the first column, between columns, and after the
last column in each line of an alignment (if {\sl \\tabskip\/} is not
zero), and we will want to take advantage of this feature later to help
center the entries.

The {\tt \\offinterlineskip} \cs\/\ is used to set the usual interline
spacing to zero.  \ This prevents \TeX\ from interposing glue between the
the individual {\tt \\vrule} segments, glue that would prevent them from
abutting each other properly. Having done this, we must 
then specify the vertical space assigned to these {\tt \\vrule}s (and to
the text as well) by using a {\tt \\strut}.  As explained earlier, this
produces a zero-width invisible box of the correct height for the font
being used and the {\tt \\vrule}s are, of course, produced to this height.

The {\tt \\halign to <dimen>} comes next, then the {\sl preamble}.  The
first thing to note about the {\sl preamble} is that
eight columns are specified, not just the three that contain the data.
A template is specified for each column. The first template
contains  {\tt \\strut\#}, but no values are
specified to match the {\tt \#} sign so only the {\tt \\strut} is ever put in
this column, but this does fix the vertical space allowed for the {\tt
\\vrule} segments and for the text.  The initial {\tt \&} in the listings
of the row data that follow the {\sl preamble} is all that is necessary to
cause the {\tt \\strut} to apply to the entries in each row.  The four
{\tt \\vrule} segments in each row after the first are similarly called
into action by the second, fourth, and sixth {\tt \&} symbol in
the row specification and by the final {\tt \\cr}.

The first row of data containing `AT\&T Common Stock' is an
exception to the general rule, in that this caption spans five columns as
signalled by the \cs\/\ {\tt \\multispan 5}.  The next row also
illustrates how you {\sl omit} the application of the formatting rules
specified in the {\sl preamble} by using {\tt \\omit} \cs\/\ and how you
prevent the widths of the entries from being used to determine column
width by the {\tt \\hidewidth} \css.  Finally, the {\tt \\rlap} command is
used to overlap the * symbol in one entry so that its width is not
considered in making the alignment.  One final detail: note the {\tt
\\noalign\lbr\\hrule\rbr} statements that define the horizontal {\sl
rules}.
{\bf The Fixed-Column-Width `\\settabs' Method}

The fixed-column-width {\tt \\settabs} method can be used in simple cases
when the fixed width restriction is of no consequence. It also
functions best where the entries can all be left aligned.  The table on
page 4 does not meet these restrictions but it will reveal most of the
complications that you are apt to encounter. So you type the following,
noting the use of the `{\tt \\+}' \cs\/\ to start each row and the `{\tt
\\cr}' \cs\/\ to end it:
\medskip
\vbox{\tt \settabs 1 \columns
\+\$\$\\vbox\lbr\\tt \\settabs 5 \\columns\cr
\+\\+\\\\rm\& \\\\sl\& \\\\it\& \\\\tt\& \\\\bf\\cr\cr
\+\\+\\rm Roman\& \\sl Slanted\& \\it Italic\& \\tt Typewriter\& \\bf Boldface\\cr\cr
\+\rbr\$\$\cr}

You get five columns, each flush left:
$$\vbox{\tt \settabs 5 \columns
\+\\rm& \\sl& \\it& \\tt& \\bf\cr
\+\rm Roman& \sl Slanted& \it Italic& \tt Typewriter& \bf Boldface\cr
}$$
You can make the columns narrower by calling for an extra unused column.
You can center the text in the columns by introducing `{\tt \\hfill}'
\css\/\ both before and after each entry (using `{\tt \\hfil}' \css\/\
will not work) and by adding an extra `{\tt \&}' symbol after the last
entries (to force \TeX\ to handle the last column like all of the rest).
All of this gets to be more trouble than to use the `{\tt \\halign}'\ 
method but you can type:

\vbox{\tt \settabs 1 \columns
\+\$\$\\vbox\lbr\\tt \\settabs 6 \\columns\cr
\+\\+\\hfill\\\\rm\\hfill\&\\hfill\\\\sl\\hfill\&\cr
\+\\hfill\\\\it\\hfill\&\\hfill\\\\tt\\hfill\&\\hfill\\\\bf\\hfill\&\\cr\cr
\+\\+\\rm\\hfill Roman\\hfill\&\\sl\\hfill Slanted\\hfill\&\\it\\hfill\cr
\+Italic\\hfill\&\\tt\\hfill Typewriter\\hfill\&\\bf\\hfill Boldface\\hfill\&\\cr\rbr\$\$\cr}

{\bf The Variable-Column-Width `\\settabs' Method}

The variable-column-width method line does a somewhat better job.  This
method requires you to supply a sample row for the table, which you type in place
of the `{\tt 5 \\columns}' on the first line (this sample is used for
dimensions only and is not printed). You use the largest entry taken from
each column and you add some desired amount of space between entries.
In this case the sample line will have all of its entries taken from the
second row of the table, which is a bit tricky, since these entries are all
in different typefaces.

Here you type:
\medskip
\vskip 10pt
\vbox{\tt \settabs 1 \columns
\+\$\$\\vbox\lbr\\tt\\settabs\\+\\rm Roman\&\\quad\\sl Slanted\&\cr
\+\\quad\\it Italic\&\\quad\\tt Typewriter\&\\quad\\bf Boldface\&\\cr\cr
\+\\+\\hfill\\\\rm\\hfill\&\\hfill\\\\sl\\hfill\&\cr
\+\\hfill\\\\it\\hfill\&\\hfill\\\\tt\\hfill\&\\hfill\\\\bf\\hfill\&\\cr\cr
\+\\+\\rm\\hfill Roman\\hfill\&\\sl\\hfill Slanted\\hfill\&\\it\\hfill\cr
\+Italic\\hfill\&\\tt\\hfill Typewriter\\hfill\&\\bf\\hfill Boldface\\hfill\&\\cr\rbr\$\$\cr}

to get a table similar to the one on page~4 except for
a slight difference in centering.

{\bf Typing Mathematical Formulas}

Math formulas are, by tradition, printed using different conventions from
those used for printing ordinary text. Some of these differences may seem
quite trivial but they contribute greatly to the legibility and yes even
to the beauty of well-printed mathematical texts.  Fortunately, \TeX\
knows about these math conventions and can usually be trusted to follow
them.

Some of these conventions are:
1) Alphabetic characters are
printed in $math$ $italics$ rather than in a more normal font and
$math$ $italics$ differs in minor detail from normal {\it italics}.  2)
Many special symbols are used that are not available in the normal text mode,
such as Greek letters and conventional
mathematical symbols, $\ge$, $\le$, $\ne$, $\approx$, etc. 3) The
spacing conventions are different from those used in ordinary text. In fact,
\TeX\ completely ignores the spaces that you use in typing
formulas and applies its own spacing rules (e.g.,  $x+y$ instead of
$x$ + $y$ or $x$+$y$). 4) The alignment rules are also quite special as you will
observe in later examples, and many symbols are frequently printed in enlarged forms.
5) Superscripts and subscripts are used
and these are normally automatically reduced in size, such
as in $x^{a^b}$ (typed as {\tt \$x\^{}\lbr a\^{}b\rbr\$}) or
as in $x_{a_b}$ (typed as {\tt \$x\und{}\lbr a\und{}b\rbr\$}).\quad
6) It is also customary to use a somewhat different style for printing formulas
that are displayed on separate lines from the style used for formulas within text.

A {\sl math mode} formula is enclosed within either single dollar signs
`{\tt \$$\ldots$\$}', if the formula is to appear in a line with ordinary
text, or within double dollar signs `{\tt \$\$$\ldots$\$\$}', if the
formula is to be typed on a separate line, in so-called {\sl display math
mode}.

Eight different styles of math typesetting 
are used,
four regular styles and four ``cramped'' variations.  The letters and numbers
are typeset in three different sizes, these being 
text size
(\hbox{$x+y-z$}), script size {$\scriptstyle (x+y-z)$}, and scriptscript size
{$\scriptscriptstyle (x+y-z)$}.  If you want to enforce a
size that \TeX\ might not otherwise use, as was just done, you use the
control sequences {\tt~\\scriptstyle} and {\tt~\\scriptscriptstyle}.  The
display style and the text style use the same
size letters and numbers
but differ in the sizes used for {\sl large operators}, 
in the positioning of exponents and in the way they handle fractions.
Should you want to enforce the use
of either the text style or the display style,
 you can use the control sequences
{\tt \\textstyle} or {\tt \\displaystyle}. A discussion of how these styles are
used in fractions will be deferred until later.

Symbols classed as {\sl large operators} (including the `summation'
and `integration' signs $\sum$ and $\int$)  are printed in a larger
size when in display math mode from the way they appear in normal math
mode.
If you type {\tt \$\\sum\und{}\lbr n=1\rbr\^{}m\$} you get $\sum_{n=1}^m$ and
if you type {\tt \$\\int\und{}\lbr -\\infty\rbr\^{}\lbr +\\infty\rbr\$}
 you get $\int_{-\infty}^{+\infty}$.  On the
other hand, these same expressions typed as display formulas yield:
$$\hbox
{{$\displaystyle\sum_{n=1}^m$} \quad
  and \quad {$\displaystyle\int_{-\infty}^{+\infty}$}}$$
The  control sequence {\tt \\nolimits} will cause the summation sign to have
subscripts and superscripts just as in the text mode, and the
control sequence {\tt \\limits}, if typed after the \\int will cause
the integration limits to appear above and below the integral sign thus:
$$\hbox
{{$\displaystyle\sum\nolimits_{n=1}^m$} \quad
  and \quad {$\displaystyle\int\limits_{-\infty}^{+\infty}$}}$$
The following example illustrates the way in which \TeX\ will increase
the size of certain symbols and will make the horizontal lines long enough
to extend over the subformula to which they apply and high enough or low
enough not to bump into it. Typing:
{\tt \$\$\\sqrt\lbr 1+\\sqrt\lbr 1+\\sqrt\lbr 1+%
\\sqrt\lbr 1+\\sqrt\lbr 1+\\sqrt\lbr 1+x\rbr\rbr\rbr\rbr\rbr\rbr\$\$} \quad produces:
$$\sqrt{1+\sqrt{1+\sqrt{1+\sqrt{1+\sqrt{1+\sqrt{1+x}}}}}}$$
The \cs\/ `{\tt \\over}' is used to specify fractions, and the fraction line
is made as long as needed. \TeX\ assumes that
everything within the same sub-formula grouping before the
`{\tt \\over}' \cs\/\ is to go above the
line and everything that follows,
 again within the same sub-formula grouping, is
to go below the line that `{\tt \\over}' \ generates.
A variant, called
`{\tt \\above}', allows you to specify the weight of the line, thus:\goodbreak
$$\hbox{{\tt \$\$\lbr a\\over 1+b\rbr+c\$\$}\quad yields\quad}
{a\over1+b}+c \hbox{\quad
and\quad {\tt \$\$\lbr a\\above2pt1+b\rbr\$\$}\quad yields\quad}{a\above2pt1+b}$$
As an aside, there are four related control sequences (while in display math mode):
$$\vbox{\halign{\hfil#\hfil&&\quad \hfil#\hfil\cr
{\tt \lbr a\\atop 1+b\rbr}&
{\tt \lbr n+1\\choose k\rbr}&
{\tt \lbr m\\brack n\rbr}&
{\tt \lbr m\\brace n-1\rbr}&\ that yield:\cr
\noalign{\smallskip}
{$\displaystyle{a\atop 1+b}$}&
{$\displaystyle{n+1\choose k}$}&
{$\displaystyle{m\brack n}$}&
{$\displaystyle{m\brace n-1}$}\cr}}$$
\TeX\ provides a somewhat similar extendability with respect to parentheses and
other {\sl delimiters}, this time to extend them vertically as required.
By using the \css\/ `{\tt \\left}'\ before a left
delimiter and `{\tt \\right}' \ before the corresponding right delimiter,
you can let \TeX\ decide as to the correct size of delimiter to use.  The
`{\tt \\left}' \ and `{\tt \\right}' \ \css\/ have an auxiliary function
in that they also act as {\sl grouping} symbols so it is not necessary to
use the `{\tt \lbr}' \ and `{\tt \rbr}' \ symbols with them (unless, of course,
these are the delimiters to be printed, and then they must be typed
as {\tt \\\lbr} and {\tt \\\rbr}).
It {\bf is} necessary to use the {\tt \\left} and {\tt \\right} \css\/\ 
in matching pairs (although the delimiters themselves need not match).

The following example illustrates these features:
$$\hbox{You type \qquad
{\tt \$\$1+\\left(1\\over 1-x{}\^{}2\\right)\^{}3\$\$\qquad}to get}\qquad
1+\left(1\over1-x^2\right)^3$$
This example introduces yet another feature, namely the handling of
exponents, which was alluded to briefly once before.
The `{\tt
\^{}}' is used to indicate a superscript and its companion the `{\tt
\und}' is used to indicate a subscript. Both normally apply only to the
next character so you must use {\sl grouping\/} symbols for those cases
where a multi-character superscript or subscript is to be shown. Thus if
you type:
$$\hbox
{{\tt \$\$1+\\left(1\\over 
1-x\^{}\lbr 21\rbr\und{}3\\right)\^{}\lbr 32\rbr\$\$}\qquad
you get}\qquad 1+\left(1\over1-x^{21}\right)^{32}$$ 
A series of \css\/ for specifying delimiter sizes are available
for use in situations where \TeX\ may not make the desired choice.
For opening delimiters these are, in order of increasing
size:  {\tt \\bigl}, {\tt \\Bigl}, {\tt \\biggl}, and {\tt \\Biggl} and
for closing delimiters they are {\tt \\bigr}, {\tt \\Bigr}, {\tt \\biggr},
and {\tt \\Biggr}. It should be noted that the {\tt \\bigl} and {\tt
\\bigr} delimiters are larger than ordinary ones so that the
difference can be perceived, yet small enough to be used within the text of a
paragraph.  Even larger delimiters than those named can be made by
using the {\tt \\left} and the {\tt \\right} conventions and by adding an
empty {\tt \\vbox} of the appropriate size within the field when
necessary. For example, you can type:
$$\vbox{\tt \halign{#\hfil&#\cr
\$\$\\left(\\vbox to 27pt\lbr\rbr\\left(\\vbox to 24pt\lbr\rbr\\left(\\vbox to 21pt\lbr\rbr\cr
\\Biggl(\\biggl(\\Bigl(\\bigl((\lbr\\scriptstyle(\lbr\\scriptscriptstyle\cr
(\\hskip3pt)\rbr)\rbr)\\bigr)\\Bigr)\\biggr)\\Biggr)\\right)\\right)\\right)\cr
\qquad{\rm (With sets of three similar lines added for brackets and for braces)}\$\$\cr}}$$
to get
$$\left(\vbox to 27pt{}\left(\vbox to 24pt{}\left(\vbox to 21pt{}
\Biggl(\biggl(\Bigl(\bigl(({\scriptstyle({\scriptscriptstyle(\hskip3pt
)})})\bigr)\Bigr)\biggr)\Biggr)\right)\right)\right)
\left[\vbox to 27pt{}\left[\vbox to 24pt{}\left[\vbox to 21pt{}
\Biggl[\biggl[\Bigl[\bigl[[{\scriptstyle[{\scriptscriptstyle[\hskip3pt
]}]}]\bigr]\Bigr]\biggr]\Biggr]\right]\right]\right]
\left\{\vbox to 27pt{}\left\{\vbox to 24pt{}\left\{\vbox to 21pt{}
\Biggl\{\biggl\{\Bigl\{\bigl\{\{{\scriptstyle\{{\scriptscriptstyle\{\hskip3pt
\}}\}}\}\bigr\}\Bigr\}\biggr\}\Biggr\}\right\}\right\}\right\}$$

\TeX\ recognizes 16 other basic
delimiters that can be obtained in different sizes, either by using the {\tt
\\left} and {\tt \\right} technique or by specifying their size as
illustrated above for parentheses, brackets, and braces.
There are restrictions as to the sizes available for {\tt \\langle}, {\tt
\\rangle}, {\tt /}, and {\tt \\backslash}, but not for the other extendable
delimiters.
$$\vbox{\halign{\hfil$#$\hfil\quad&\tt #\hfil\qquad\quad&
\hfil$#$\hfil\quad&\tt #\hfil\qquad&
\hfil$#$\hfil\quad&\tt #\hfil\qquad\quad&
\hfil$#$\hfil\quad&\tt #\hfil\cr
\noalign{\centerline{Additional extendable delimiters for use in {\sl Display Math} mode:}}
\noalign{\vskip3pt}
\hbox{For}&\hbox{You type}&\hbox{For}&\hbox{You type}&
\hbox{For}&\hbox{You type}&\hbox{For}&\hbox{You type}\cr
\noalign{\vskip3pt}
\lfloor&\\lfloor& \langle&\\langle& \vert&\\vert& \downarrow&\\downarrow\cr
\rfloor&\\rfloor& \rangle&\\rangle& \Vert&\\Vert& \Downarrow&\\Downarrow\cr
\lceil&\\lceil& /&/& \uparrow&\\uparrow& \updownarrow&\\updownarrow\cr
\rceil&\\rceil& \\&\\backslash& \Uparrow&\\Uparrow& \Updownarrow&\\Updownarrow\cr
}}$$


Before considering matrices and other uses of large delimiters, let us
dispose of the matter of fractions. It is usually wise to avoid the use of
more than one {\tt \\over} in an equation and
remember that there cannot be more than one {\tt \\over} in any one
subexpression. In fact, it is often preferable to use the ``slashed'' fraction
form, particularly for mathematical expressions that are not printed as separate displays.

When you do use more than one {\tt \\over},
you probably should introduce our old
friends {\tt \\strut} and {\tt \\displaystyle} to keep the spacings
and character sizes from shrinking alarmingly. Certainly,

$$a_0+{ a\over\displaystyle a_1+
{\strut 1\over\displaystyle a_2+
{\strut 1\over\displaystyle a_3+
{\strut 1\over\displaystyle a_4}}}}
\hbox{\quad typed as\quad}\vtop{\tt \halign{#\hfil\cr
\$\$a\und{}0+\lbr  a\\over\\displaystyle a\und{}1+\cr
\lbr\\strut 1\\over\\displaystyle a\und{}2+\cr
\lbr\\strut 1\\over\\displaystyle a\und{}3+\cr
\lbr\\strut 1\\over\\displaystyle a\und{}4\rbr\rbr\rbr\$\$\cr}}$$
looks better than
$$a_0+{a\over a_1+{ 1\over a_2+{ 1\over a_3+{ 1\over a_4}}}}$$

which is what you get without {\tt \\strut} and {\tt \\displaystyle}.

As an exercise, you might reproduce this example and make several copies,
one without using {\tt \\strut} and another without {\tt \\displaystyle}, to reveal the
exact effect of these refinements. \goodbreak

Perhaps the most common use of large delimiters is in printing matrices,
where the {\tt \\matrix} macro can be used. This macro is similar in many
respects to the {\tt \\halign} macro that was discussed starting on page~12 except
that {\tt \\matrix} only works in math mode, and no template need be supplied.
To get:
$$A=\left(\matrix{x-\lambda&1&0\cr
 0&x-\lambda&1\cr
 0&0&x-\lambda\cr}\right)
\hbox{\quad you type \quad}\vcenter
{\halign{\tt #\cr
\$\$A=\\left(\\matrix\lbr  x-\\lambda\&1\&0\\cr\cr
0\&x-\\lambda\&1\\cr\cr
0\&0\&x-\\lambda\\cr\rbr\\right)\$\$\cr}}
$$
It helps in typing a matrix such as this if you line up the columns,
which you can do since \TeX\ pays no attention to the spaces that you leave.
So you might type:

\smallskip
{\tt \parskip=0pt\obeylines \obeyspaces
{\$\$A=\\left|\\matrix\lbr 
x-\\mu \& 1     \& 0     \\cr
0     \& x-\\mu \& 1     \\cr
0     \& 0     \& x-\\mu \\cr
\rbr\\right|\$\$}}\qquad This, of course, produces:
$$A=\left|\matrix{x-\mu&1&0\cr
 0&x-\mu&1\cr
 0&0&x-\mu\cr}\right|$$
since I used vertical bars instead of parentheses, and I substituted {\tt \\mu}
for {\tt \\lambda} just to be different.
If you are wondering how I was able to preserve the spacings
in showing you what I typed,
I used plain \TeX's macros, {\tt \\parskip=0pt \\obeylines \\obeyspaces}.

Ellipses (i.e., dots) are used in many places but they appear to good
advantage in matrices where you might type:
$$\vcenter{\openup 2pt
\halign{\tt #\cr
\$\$A=\\pmatrix\lbr\cr
 a\und{}\lbr11\rbr\&a\und{}\lbr 12\rbr\&\\ldots\&a\und{}\lbr 1n\rbr\\cr\cr
 a\und{}\lbr 21\rbr\&a\und{}\lbr 22\rbr\&\\ldots\&a\und{}\lbr 2n\rbr\\cr\cr
\\vdots\&\\vdots\&\\ddots\&\\vdots\\cr\cr
 a\und{}\lbr m1\rbr\&a\und{}\lbr m2\rbr\&\\ldots\&a\und{}\lbr mn\rbr\\cr\rbr\$\$\cr
\strut\cr
}}
\hbox{\qquad to get\qquad} 
A=\pmatrix{
a_{11}&a_{12}&\ldots&a_{1n}\cr
a_{21}&a_{22}&\ldots&a_{2n}\cr
\vdots&\vdots&\ddots&\vdots\cr
a_{m1}&a_{m2}&\ldots&a_{mn}\cr}$$ \vskip-28pt
Here {\tt \\pmatrix} is like {\tt \\matrix}, but it puts parentheses in for you.

The {\tt \\ldots} \cs\ (as in the above matrix) is used between letters
and commas while another \cs\/,
`{\tt \\cdots}', is used between signs and similar operators as
 in  $a+b+\cdots+z$.
Incidentally, a single centered dot is a {\tt \\cdot}.
  Still another \cs, 
`{\tt \\dots}', is used for ellipses in non-math text \dots,
when needed.

Single braces are often used in referring to different cases and plain
\TeX\ provides a convenient macro, you guessed it, called {\tt \\cases}.
This bears a remarkable resemblance to {\tt \\pmatrix} except that, 1)~only
one delimiter, a brace, is printed, 2)~the entries are adjusted flush left
in each of two columns, and
3)~the material that appears after the `{\tt \&}' symbol is not in math mode
unless specifically surrounded with `{\tt \$}' symbols. For example, to get:
$$|x|=\cases{x,&if $x\ge0$;\cr -x,&otherwise.\cr}$$
you type
$$\vbox{\tt
\$\$|x|=\\cases \lbr x,\&if \$x\\ge0\$;\\cr -x,\&otherwise.\\cr\rbr\$\$
}$$
If a `{\tt \\left}'-specified left delimiter
without a matching right delimiter is needed elsewhere,
you can use `{\tt \\right.}' as a null right delimiter,
where the period is not printed but acts to terminate the
extent of the grouping (as it affects the size of the beginning delimiter).
Similarly, `{\tt left.}' may be used to specify a null left delimiter.

Roman characters and words may be printed within mathematical
formulas using three quite different mechanisms. For only an occasional
letter or word, you can switch to roman and type:
$$\hbox{{\tt
\$\\exp(x+\lbr \\rm constant\rbr)\$}\qquad to get \qquad }\exp(x+{\rm constant})$$
This will still work for several words but, since spaces are ignored,
you have to use the control sequence `{\tt \\ }' to preserve the spacings
between words. 

The second way is to use a {\tt \\hbox}. Thus  to get:
$$x^3+\hbox{lower order terms \qquad you type \qquad {\tt 
\$x\^{}3+\\hbox\lbr lower order terms\rbr\$}}$$
This scheme was used to type this very example and it
has been used extensively throughout this manual.  It has two obvious
disadvantages, 1) the current (text) font will be used, and it may not be the
font you want (but this can be fixed) and 2) the content of the box will
always be in the same size unless extra precautions are taken when the
words are wanted in a different size, perhaps for use in a superscript or
subscript.

Finally, if you plan to use a word or a fixed sequence of roman typed words
frequently in different mathematical formulas, for example the sequence just
used, then you can assign a name to it as one of your own macros thus:
$$\hbox{\tt \\def\\loterms\lbr \\hbox\lbr \\rm lower order terms\rrbr.}$$
Thereafter when you want the sequence `lower order terms' you simply type
{\tt \\loterms}. The above example would then be typed as \quad
`{\tt \$x\^{}3+\\loterms\$}'.

Since the names of the common mathematical function are always set in
roman type, the following control sequences have been predefined in plain
\TeX:
$$\vbox{\tt \halign{#\hfil\quad&&#\hfil\quad\cr
\\arccos  &\\cos   &\\csc   &\\exp   &\\ker     &\\limsup  &\\min  &\\sinh\cr
\\arcsin  &\\cosh  &\\deg   &\\gcd   &\\lg      &\\ln      &\\Pr   &\\sup\cr
\\arctan  &\\cot   &\\det   &\\hom   &\\lim     &\\log     &\\sec  &\\tan\cr
\\arg     &\\coth  &\\dim   &\\inf   &\\liminf  &\\max     &\\sin  &\\tanh\cr
}}$$
These \css\/ lead to roman type with appropriate spacing.
Certain of these are treated as {\sl large operators} (just like {\tt \\sum}), to wit:
{\tt \\det}, {\tt \\gcd}, {tt \\inf},  {\tt \\lim}, {\tt \\liminf}, 
{\tt \\limsup}, {\tt \\max}, {\tt \\min}, {\tt \\Pr}, and {\tt \\sup}.  The following
examples are taken from The \TeX book but rearranged to use the {\tt \\cases}
{\sl control sequence}, to demonstrate that {\tt \\cases} will work for a larger 
array than previously shown.
$$\cases{\rm To\ get&You type (while in math mode)\cr 
\noalign{\vskip2pt}
\sin2\theta=2\sin\theta\cos\theta&\tt \\sin2\\theta=2\\sin\\theta\\cos\\theta\cr
\noalign{\vskip2pt}
O(n\log n\log\log n)&\tt O(n\\log n\\log\\log n)\cr
\noalign{\vskip2pt}
\Pr(X>x)=\exp(-x/\mu)&\tt \\Pr(X>x)=\\exp(-x/\\mu)\cr
\noalign{\vskip2pt}
\displaystyle{\max_{1\le n\le m}\log_2P_n}&\tt 
\\displaystyle\\max\und{}\lbr 1\\le n\\le m\rbr\\log\und{}2P\und{}n\cr
\noalign{\vskip2pt}
\displaystyle{\lim_{x\to0}{\sin x\over x}=1}&\tt 
\\displaystyle\\lim\und{}\lbr x\\to0\rbr\lbr\\sin x\\over x\rbr=1\cr
}$$
Incidentally, I wanted these examples to be spaced a little farther apart
than they would be with the spacing set by {\tt \\cases}. So I inserted
{\tt \\noalign\lbr \\vskip2pt\rbr} after every {\tt \\cr} (except the last one).

Punctuation should be used with caution.  When a formula is followed by
punctuation, such as a period or a comma, put the punctuation after the
terminating {\tt \$} sign when the formula is in text (even in the extreme case
when you type something like {\tt `\$x=a\$, \$b\$, or \$c\$.'}); but put the
punctuation before
the terminating {\tt \$\$} when the formula is displayed.  Note: Punctuation
symbols have been omitted from the display equations used so far in this
manual (so as not to confuse you), but I will start using them now.

Numbering isolated display formulas poses no special 
problem as plain \TeX\ provides two control sequences, {\tt \\eqno} for numbers
that are to go on the right, and {\tt \\leqno} for numbers that are to go on the 
left. In both cases, these control sequences go after the formula to be numbered.
They act much like our old friend {\tt \&} in that they separate two fields;
everything to the left is part of the formula and
everything to the right up to the
terminating {\tt \$\$} will appear as the equation number. 

Thus, {\tt \$\$(x+y)(x-y)=x\^{}2-y\^{}2.\\eqno(21)\$\$}  will produce:
$$(x+y)(x-y)=x^2-y^2.\eqno(21)$$
If you type,
{\tt \$\$(x+y)(x-y)=x\^{}2-y\^{}2.\\leqno[21a]\$\$} \ you will get:
$$(x+y)(x-y)=x^2-y^2.\leqno[21a]$$
The formulas are centered in both cases (without
regard for the presence of the formula numbers), and the formula numbers are in
math style unless you specify otherwise.

Several display formulas, appearing together, can be aligned at any desired
location (on an equal sign perhaps) by using the control sequence 
{\tt \\eqalign}, which works with {\tt \&} and {\tt \\cr}
in a manner somewhat similar to the use of these markers in {\tt \\matrix} and
{\tt \\cases}. For example, if you type:

{\tt \$\$\\eqalign\lbr ax\^{}2+bx+c\&=0\\cr
x\&=\lbr -b\\pm\\sqrt\lbr b\^{}2-4ac\rbr\\over2a\rbr.\\cr\rbr\$\$}

you get
$$\eqalign{ax^2+bx+c&=0\cr
x&={-b\pm\sqrt{b^2-4ac}\over2a}.\cr}$$

If you want the equations individually numbered, you use {\tt \\eqalignno}
and add a second {\tt \&} with the number added as usual for each
equation that is to be numbered.
  Typing:
\smallskip {\tt \obeylines \parskip 0pt 
{\$\$\\eqalignno\lbr ax\^{}2+bx+c\&=0\&(1)\\cr
x\&=\lbr -b\\pm\\sqrt\lbr b\^{}2-4ac\rbr\\over2a\rbr.\&(2)\\cr\rbr\$\$}}
\ \ produces
$$\eqalignno{ax^2+bx+c&=0&(1)\cr
x&={-b\pm \sqrt {b^2-4ac}\over 2a}.&(2)\cr }$$

Using {\tt \\eqalign} (not {\tt \\eqalignno}) and adding {\tt \\eqno(3)},
for example, will cause the set of formulas to be
numbered as a group, with the number centered vertically
with respect to the group, producing:
$$\eqalign{ax^2+bx+c&=0\cr
x&={-b\pm\sqrt{b^2-4ac}\over2a}.\cr}\eqno(3)$$
It is also possible to introduce extra text lines between the different
formula lines without disturbing the alignment, for example by including
{\tt \\noalign\lbr \\hbox\lbr This is the text to be introduced\rrbr}.
Finally, using {\tt \\leqno} and {\tt \\leqalignno}, with an initial
letter `{\tt l}' and with no other changes, will cause the formula numbers
to be placed at the left.

Some fundamental differences between {\tt \\eqalign} and {\tt \\eqalignno}
should be noted.

{\tt \\eqalign} makes a single vertically-centered vbox that is no wider
than necessary, that cannot be broken between pages, and that can only
take a single vertically-centered equation number.  More than one {\tt
\\eqalign} can be put on a line (if space permits).

{\tt \\eqalignno} generates full-width lines.  These lines can be broken
between pages and the individual lines can take individual line numbers.
{\tt \\eqno} cannot be used with {\tt \\eqalignno} to assign a group
equation number.  Lines of normal text can, however, be placed between the
{\tt\\eqalignno} equation lines by using the {\tt
\\noalign\lbr \\hbox\lbr\ \dots\rbr\rbr} construction.

Long equations pose a difficult problem, particularly if they must still
be aligned in some way with other equations in a group, and you may have
to help \TeX\ to do a satisfactory job.  One solution is illustrated
below.  
$$\eqalignno{x_nu_1+\cdots+x_{n+t-1}u_t
&=x_nu_1+(ax_n+c)u_2+\cdots\cr
&\phantom{=x_nu_1}\;+\bigl(a^{t-1}x_n+c(a^{t-2}+\cdots+1)\bigr)u_t\cr
&=(u_1+au_2+\cdots+a^{t-1}u_t)x_n+h(u_1,\ldots,u_t).  \quad&(47)\cr}$$
The first and third lines were aligned on the $=$ signs as usual but then
we wanted the second line to be aligned with the $+$ sign as shown, and
the \\eqalignno control sequence makes no provision for secondary
alignments.  The solution is to use the control sequence {\tt
\\phantom$\{\ldots\}$} which causes \TeX\ to leave the same amount of
space that would be taken by the indicated text without printing it.
So we put a {\tt \\phantom\lbr =x\_{}nu\_{}1\rbr} in front of the second
line and align it with the other two lines on the $=$ sign.  This
does not work perfectly because the {\tt \\phantom} command creates an empty box
of the size necessary to contain the indicated characters without regard for
their surroundings and the space allowed before the $=$ and $+$ signs depends
on the context.  The net result is that
we have to add a bit of space defined by `{\tt \\;}'. So what
we type is:  
$$\vbox{\tt \halign{#\hfill\cr
$$\\eqalignno\lbr x\und{}nu\und{}1+\\cdots+x\und{}\lbr n+t-1\rbr u\und{}t\cr
\&=x\und{}nu\und{}1+(ax\und{}n+c)u\und{}2+\\cdots\\cr\cr
\&\\phantom\lbr =x\und{}nu\und{}1\rbr\\;+\\bigl(a\^{}\lbr t-1\rbr
x\und{}n+c(a\^{}\lbr t-2\rbr +\\cdots+1)\\bigr)u\und{}t\\cr\cr
\&=(u\und{}1+au\und{}2+\\cdots+a\^{}\lbr t-1\rbr u\und{}t)x\und{}n+h(u\und{}1,\\ldots,u\und{}t)\.\quad\&(47)\\cr\rbr$$\cr
}}.$$
Normally, you can depend on \TeX\ to space things correctly, but when \TeX\ needs
your help while in math mode, you can use the following:
\thinspace
$$\vbox{\halign{\hfil#\hfil\quad&&\hfil#\hfil\quad\cr
You type&\tt \\!&\tt \\,&\tt \\>&\tt \\;\cr
To get&$-1/6$ quad&\ $1/6$ quad&\ $2/9$ quad&$5/18$ quad\cr
}}$$

Some other math features, not yet covered, are the use of {\tt \\prime} to produce
prime superscripts and subscripts where you type:
$$\hbox{{\tt\$\ y\und{}1\^{}\\prime+y\und{}2\^{}\lbr\\prime\\prime\rbr+y\und{}3\^{}\lbr
\\prime\\prime\\prime\rbr\$} \ to get \ }
y_1^\prime+y_2^{\prime\prime}+y_3^{\prime\prime\prime},$$
\leftline{the use of {\tt \\root} where,\quad
{\tt \$\\root 3 \\of \lbr x\^{}2+y\^{}2\rbr}\quad produces \quad $\root 3 \of {x^2+y^2}\,$,}
\smallskip
and the use of {\tt \\mathstrut} to enforce uniformity in the positioning,
say, of the square root signs by typing
{\tt \$\\sqrt\lbr\\mathstrut a\rbr+\\sqrt\lbr\\mathstrut d\rbr+\\sqrt\lbr\\mathstrut y\rbr\$} to
get $\sqrt{\mathstrut a}+\sqrt{\mathstrut d}+\sqrt{\mathstrut y}$
instead of $\sqrt a+\sqrt d+\sqrt y\,$.

There are, also, two variations on {\tt \\phantom} that you will find
useful, {\tt \\vphantom} with zero width and {\tt \\hphantom} with zero
height and depth, and there is {\tt \\smash} which tells \TeX\ to print 
a subexpression but to assume that it has zero height and depth.

{\bf Some Odds and Ends}

\nobreak
You are about ready to undertake ordinary typing on your own but you still
do not know how to produce footnotes,\footnote{*}{Like this, which was
produced by typing `{\tt \\footnote\lbr *$\rbr\lbr$Like this, which was produced by
typing
\dots \rbr'} right along following the macro itself.
 \TeX\ will usually do the right thing
like putting the indicated mark (which is typed in the first set of braces)
in the text where the macro {\tt \\footnote} appears and putting the footnote
itself (which was typed in a second pair of braces) at the bottom of the
same page and even dividing an extra long footnote between pages.}
how to allow for insertions, and how to change the page format if you do not 
want the pages numbered at the bottom.

\goodbreak
\newcount\notenumber
\def\clearnotenumber{\notenumber=0}
\def\note{\advance\notenumber by1 \footnote{$^{\the\notenumber}$}}
\clearnotenumber
If you should like your footnotes to be numbered automatically, even this
can be done by defining a new control sequence which might be called
{\tt\\note}.\note{This should be footnote number 1.} Before the first footnote\note{You can
also number equations automatically (but that's a different story).}  to
be so numbered you write (as I have done):
$$\vbox{\tt \halign{#\hfil\cr
\\newcount\\notenumber\cr
\\def\\clearnotenumber\lbr\\notenumber=0\rbr\cr
\\def\\note\lbr\\advance\\notenumber by1 \\footnote\lbr\$\^\lbr\\the\\notenumber\rbr\$\rbr\rbr\cr
\\clearnotenumber\cr
}}$$
There is an art to inserting illustrations or other independently derived
material into a text.  A number of people are working on supplements to
\TeX\ to allow for the direct introduction of computer derived graphics
into \TeX\ output, but I will assume that you will be content to add
photographs and graphical material manually.

Plain \TeX\ provides for three basic types of insertions, {\tt
\\topinsert}, {\tt \\midinsert}, and {\tt \\pageinsert}. These can only be
given between paragraphs and not inside boxes or other insertions. The
general form for these is:
{\tt \\topinsert <vertical mode material that can have embedded paragraphs>
\\endinsert.}

\TeX\ tries to put a {\tt \\topinsert} at the top of the current page, if
there is still room when the {\tt \\topinsert} is encountered, otherwise
it will be put at the top of the next page.  If several {\tt \\topinsert}
commands are given close together, some may be carried to still later
pages.  The {\tt \\midinsert} is put on the page in the position where it
appears,
if this is possible, otherwise it is handled as a {\tt \\topinsert}. A {\tt
\\pageinsert} is automatically expanded to fill an entire page and put on
the next page.
Complications requiring human intervention
may arise if \TeX\ is asked to put
pageinserts and footnotes on the same page and if both
topinserts and footnote extensions are carried over to a following page.

I will now type a {\tt \\topinsert}, which, as you can see, appears at the top
of the next page. Perhaps you should read it now, if you have not already done so.
\newdimen\varunit
\varunit=\hsize \advance\varunit by-2\parindent
\divide\varunit by 58 % illustrations in Chapter 12
\def\sampleglue#1#2{% #1=width, #2=text
  \vtop{\hbox to #1{\xleaders\hbox to .5\varunit{\hss\copy\smalldot\hss}\hfil}
    \kern3pt
    \tabskip \z@ plus 1fil
    \halign to #1{\hfil##\cr#2\cr}}}
\varunit=2.9pt % getting ready to make semicircular insert
\setbox0=\vtop{\null
\baselineskip3.8\varunit
\parfillskip0pt
\parshape 10
-18.25\varunit 36.50\varunit
-30.74\varunit 61.48\varunit
-38.54\varunit 77.07\varunit
-44.19\varunit 88.39\varunit
-48.47\varunit 96.93\varunit
-51.70\varunit 103.40\varunit
-54.08\varunit 108.17\varunit
-55.72\varunit 111.45\varunit
-56.68\varunit 113.37\varunit
-57.00\varunit 114.00\varunit
\ninerm
\noindent
\hbadness 6000
\tolerance 9999
\pretolerance 0
This is a {\ttt \\topinsert}.  This text is printed in this shape both to
set it apart from the rest of the text, as befitting a {\ttt \\topinsert},
and to demonstrate that it is possible to specify an essentially
arbitrary paragraph shape by saying {\ttt \\parshape=n}, where n is the
number of lines, and by following this with the n sets of dimensions for
these lines, each specified as one number for the indentation and a second
number for the length of the line.  For a more detailed explanation, see
The \TeX book, chap.~14, page 101.  In this case, these lines, so
specified, were put into a box and this box was used as a \hbox{\ttt
\\topinsert}.
}
\topinsert 
\moveright 16.3pc\box0
\vskip 10pt
\centerline{\bf Fig. E. This is a \\topinsert}\endinsert

This example may be a bit too complicated to explain in detail but in
essence it involved the creation of a box by the command {\tt
\\setbox0=\\vtop\lbr\ \dots\rbr}, where the dots stand for the command {\tt
\\parshape 10} followed by the specification of the line indentations
and lengths and then by the text itself.
Having defined such a box it was then only necessary to give the commands:
\medskip
{\obeylines \tt \parskip=0pt
\\topinsert
\\box0
\\vskip10pt
\\centerline\lbr \\bf Fig. E. This is a \\\\topinsert\rbr
\\endinsert
\medskip
\rm to tell \TeX\ to generate the {\tt\\topinsert}.}

If there is any danger of an insertion that does not get made before the end of the
appropriate section, you can force its printing by typing  {\tt \\vfill\\supereject}.

\def\leaderfill{\leaders\hbox to 1em{\hss.\hss}\hfill}
You should also know how to produce so-called {\sl leaders}, such as these,
which are often used in Tables of Contents.
\smallskip
\nobreak\penalty10000
\line{\qquad\qquad Introduction\leaderfill 1\qquad\qquad}
\line{\qquad\qquad Toward Book Quality\leaderfill 2\qquad\qquad}
\smallskip\goodbreak
The examples shown here were typed as:
$$\vbox{\tt \halign{#\hfill\cr
\\def\\leaderfill\lbr \\leaders\\hbox to 1em\lbr \\hss.\\hss\rbr\\hfill\rbr\cr
\\line\lbr\\qquad\\qquad Introduction\\leaderfill 1\\qquad\\qquad\rbr\cr
\\line\lbr\\qquad\\qquad Toward Book Quality\\leaderfill 2\\qquad\\qquad\rbr\cr
}}$$
{\parindent=35pt
\narrower
Still another feature,
with {\tt\\parindent} set to some non-zero amount (it is set to 20pt in plain
\TeX\ and to 35pt for this example), the control sequence {\tt \\narrower}
followed by \lbr\ the desired contents of the paragraph ending with a {\tt
\\smallskip} (also within the enclosing braces) \rbr, will cause a paragraph, such as this one, to be narrowed
by the {\tt\\parindent} amount.  \smallskip} \goodbreak

{\bf Output Routines}\nobreak

Defining variant forms of output routines is properly in the domain of
the \TeX pert, but you may like to have a choice of at least one other format
which I am now using, starting with this page. This change was introduced by
typing:
$$\vbox{\tt {\halign{#\hfil\cr
\\nopagenumbers \% suppress footlines\cr
\\headline=\lbr\\ifodd\\pageno\\rightheadline \\else\\leftheadline\\fi\rbr\cr
\\def\\rightheadline\lbr\\tenrm\\hfil A Beginner's \\TeX\\ Manual\\hfil\\folio\rbr\cr
\\def\\leftheadline\lbr\\tenrm\\folio\\hfil First Grade \\TeX \\hfil\rbr\cr
}}}$$
Of course, if you want to get fancy, you can have \TeX\ automatically use your
current chapter or section headings as running heads, but this may be too much
for the beginner.
\nopagenumbers % suppress footlines
\headline={\ifodd\pageno\rightheadline \else\leftheadline\fi}
\def\rightheadline{\tenrm\hfil A Beginner's \TeX\ Manual\hfil\folio}
\def\leftheadline{\tenrm\folio\hfil First Grade \TeX \hfil}

{\bf Defining Macros}

Several macros have been defined and used in this manual, a typical one being:
\medskip
\centerline{\tt \\def\\loterms\lbr \\hbox\lbr \\rm lower order terms\rrbr.}
\smallskip
Such macros all
begin with the control sequence
`{\tt \\def}' followed by the new name and then the meaning to be assigned to
this new name (enclosed in braces that are not a part of the definition).
If braces are wanted as a part of the definition they must be added.

As useful as these simple macros are, you will come across numerous
situations where you will want to define a macro that can take parameters that
are to be defined at the time the macro is used. You have already been introduced
to control sequences of this sort that were defined in {\tt PLAIN.TEX}, and that you
have learned to use.

One such definition is:\quad
{\tt\\def\\centerline\#1\lbr \\line\lbr \\hss\#1\\hss\rrbr.}

This says that if \TeX\ encounters the command {\tt \\centerline}, it is
next to look for a parameter that is here designated as {\tt \#1} which you
should have typed in braces following the {\tt \\centerline} command.
\TeX\ is then to apply the control sequence {\tt \\line\/} to the braced
expression {\tt \\hss\#1\\hss}, where 
{\tt\\line\/} is itself a derived control
sequence defined as \quad  {\tt\\def\\line\lbr \\hbox to\\hsize\rbr}.
In other words---{\bf Center It on a Line!}

Two things are to be noted: 1) It is possible to use other control
sequences within definitions and 2) The symbol {\tt\#1} is used to
designate a parameter that is given to \TeX\ at the time that the control
sequence is invoked.  Actually, up to 9 parameters can be so specified.
The definition might then take the form \quad {\tt\\def\\zzz\#1\#2\#3}\lbr a
complex expression in which {\tt\#1}, {\tt\#2}, and {\tt\#3} can be used in any order and
repeatedly if needed\rbr. The parameters, as listed following the name of
the control sequence, must, however, be in serial order.

It is also possible to write macros that are conditional.  A
simple example of a conditional macro was used to define the current page
format, as explained earlier on this page,
when we typed:
{\tt \\headline=\lbr\\ifodd\\pageno\\rightheadline\\else\\leftheadline\\fi\rbr}.
Note particularly the use of {\tt \\fi} ({\tt if} spelled backward) to end the conditional.

To round out our discussion of {\tt \\def}, we will note that
there exists a {\tt \\let} primitive that is somewhat
analogous to {\tt \\def} but differs from it in the timing of its
execution. The difference can most easily be explained by noting that the expression
{\tt \\let\\b=\\a} sets the value of {\tt \\b} to the value of {\tt \\a} at the
time when this expression is read by \TeX, while {\tt \\def\\b\lbr \\a\rbr}
does not set the value of {\tt \\b} to the value of {\tt \\a} until this
macro is executed.

Before naming a new macro, it is always wise to make sure that the
proposed name is not already in use either by \TeX\ itself or by the macro
package that you intend to use. You can do this by interrogating \TeX\
directly. Simply run \TeX82 without specifying a file name and
in response to the asterisk prompt type {\tt \\show} followed by the
proposed name. If your macro package is not preloaded, you will want to load it
before typing the {\tt \\show}. 
$$\vbox{\halign{#\hfil\quad&\tt #\hfil\qquad&\tt #\hfil\qquad&\tt #\hfil\cr
For&\rm A Macro&\rm A Primitive&\rm An undefined name\cr
\noalign {\vskip 3pt}
Typing&\\show\\centerline&\\show\\hbox&\\show\\zzzz\cr
Shows&> \\centerline=macro:&> \\hbox=\\hbox.&> \\zzzz=undefined.\cr
&\#1->\\line\lbr \\hss\#1\\hss\rbr.&<*> \\show\\hbox&<*> \\show\\zzzz\cr
&<*> \\show\\centerline&?&?\cr
&?\cr
}}$$
The line beginning with {\tt \#1->} for the {\tt \\centerline} case, shows that the
macro expects one parameter and this parameter is then used in the
topression that follows.  The next to last line is to show you what
\TeX\ has read and the last line is asking you what you want to do about
it. A carriage return will restore the asterisk prompt,
and you can then type another {\tt \\show}, or give a {\tt \\input} command.

By way of summary, there are several ways to assign meaning to a control 
sequence:
\vskip 6pt
\halign{\qquad#\hfil\quad& #\hfil\quad& \hfil #\quad& #\hfil\cr
Typing&\tt \\font\\cs=\\<font name>& makes {\tt \\cs}& a font identifier\cr
& \tt \\chardef\\cs=\\<num.>& & a character code\cr
& \tt \\countdef\\cs=<num.>& & a {\tt\\count} register\cr
&  \tt \\def\\cs\dots\lbr \dots\rbr& & a macro.\cr
Typing&\tt \\let\\cs=<token>& gives {\tt \\cs}&the token's current meaning.\cr}

You will need to know quite a bit more before you can be an efficient
macro designer. The whole story appears in Chapter 20 of The \TeX book.

Two special macro packages are soon to
be available.  \La\TeX\ by Leslie Lamport probably will appeal to the
computer programmer, while AMS-\TeX\ by Michael Spivak probably will
appeal to the mathematician. Both packages allow the user to
specify one of several formatting styles by name and in so doing they
greatly simplify the task of using \TeX.  F\'acil \TeX\ by Max D\'\i az is also
a contender but no date is available for its conversion to \TeX82.

\vfill\eject
\headline={\hfil} % headline is normally blank
\footline={\hss\tenrm\folio\hss}
\vsize=9.0 truein
\null \vskip-50pt
\centerline {\bf Appendix}
%\input manmac.tex[tex,dek]
\def\sep{\medskip\hrule width\hsize\medskip}
\def \dsep{\vskip-3pt\line{\dotfill}\smallskip}
%\medskip\smallskip
%\hrule height .61803pt
%\kern 1pt
%\hrule
%medskip
\vfill
\line{\strut Characters that are reserved for special purposes:\hfil\tt
  \\ \hfil \lbr  \hfil \rbr \hfil \$ \hfil \& \hfil \# \hfil \% \hfil \^ \hfil \_{} \hfil \~{} }
\sep
\vbox{\halign {\hfil #\hfil\hskip7.5pt &&\hfil #\hfil\hskip7.5pt \cr
To print&-&--&---&$-$&``text''&?`&!`&\$&\#&\&&\%&\ae&\AE&\oe&\OE&\aa\cr
You type&\tt -&\tt -{}-&\tt -{}-{}-{}&$\$${\tt -}$\$$&\tt `{}`text'{}'&
\tt ?{}`&\tt !{}`&\tt \\\$&\tt \\\#&\tt \\\&&\tt \\\%&
\tt \\ae&\tt \\AE&\tt \\oe&\tt \\OE&\tt \\aa\cr
}}
\dsep
\vbox{\halign {\hfil #\hfil\hskip7.8pt &&\hfil #\hfil\hskip7.8pt \cr
\AA&\ss&\o&\O&
\`a&\'e&\^o&\"u&\=y&\~n&\.p&\u\i&\v s&\H\j&\t\i u&\b k\cr
\tt \\AA&\tt \\ss&\tt \\o&\tt \\O&
\tt \\\`{}a&\tt \\\'{}e&\tt \\\^{}o&\tt \\\"{}u&\tt \\={}y&\tt \\\~{}n&
\tt \\.p&\tt \\\u\\i&\tt \\v{}s&\tt \\H\\j&\tt \\t\\i u&\tt \\b k\cr
}}
\dsep
\vbox{\halign {\hfil #\hfil\hskip7.2pt &&\hfil #\hfil\hskip7.2pt \cr
\c c&\d h&\l&\L&\dag&\ddag&\S&\P&
\rlap/c&\it \$&\it \& &\copyright&\dots\cr
\tt \\c c&\tt \\d h&\tt \\l&\tt \\L&\tt \\dag&\tt \\ddag&\tt \\S&\tt \\P&
\tt \\rlap/c&\tt \\it\\\$ &\tt \\it\\\&&\tt \\copyright&\tt \\dots\cr}}
\sep
\def\]{\leavevmode\hbox{\tt\char`\ }} % visible space
\line{\strut Line break controls:\hfil
\tt \\break\hfil\\allowbreak\hfil\\nobreak\hfil\\hbox\lbr unbreakable\rbr}
\line{\strut \tt dis\\-cre\\-tion\\-ary hy\\-phens \hfil
{\rm virgule\slash  breakpoint:}\ \ \\slash}
\dsep
\settabs 2\columns
\+\strut Breakable horizontal spaces:& Unbreakable horizontal spaces:\cr
\+{\tt \\\]} \ normal interword space& {\tt \~{}} \ normal interword space\cr
\+{\tt \\enskip} \ this\enskip much& {\tt \\enspace} \ this\enspace much\cr
\+{\tt \\quad} \ this\quad much& {\tt \\thinspace} \ this\thinspace much\cr
\+{\tt \\qquad} \ this\qquad much& {\tt \\negthinspace} \ this\negthinspace much\cr
\+\strut{\tt \\hskip \ <arbitrary dimen>}& {\tt \\kern \ <arbitrary dimen>}\cr
\sep
\+\strut Vertical spaces:\hfill
{\tt \\smallskip} $\vcenter{\hrule width2em\smallskip\hrule}$\hfill&
\\medskip $\vcenter{\hrule width3em\medskip\hrule}$\hfill
\\bigskip $\vcenter{\hrule width4em\bigskip\hrule}$&\cr
\dsep
\line{Page break controls:\hfill {\tt \\eject \hfill\\supereject\hfill
\\nobreak\hfill\\goodbreak\hfill\\filbreak}}
\+\strut Vertical spaces and good breakpoints:&
{\tt \\smallbreak\hfill\\medbreak\hfill\\bigbreak}&\cr
\line{\downbracefill}
\line{\tt \\line\lbr \\downbracefill\rbr\quad
\\hrulefill \hrulefill\quad\\dotfill \dotfill\quad
\\line\lbr \\upbracefill\rbr~}
\vskip-5pt
\line{\upbracefill}\smallskip
\medskip
\line{\hfill{\tt \\rm} {\rm Roman}\hfill
{\tt \\sl} {\sl Slant}\hfill
{\tt \\it} {\it Italic\/}\hfill
{\tt \\tt} {\tt Typewriter}\hfill
{\tt \\bf} {\bf Boldface}\hfill}
%\smallskip
Typical font table ({\tt \\rm} for amr10). If necessary, one can type
 {\tt \\char'10} to get \char'10, etc.
\medskip
\vbox{\rm \halign{\hfil#\hfil\kern3.7pt&&\hfil#\hfil\kern3.7pt\cr
\char'000&\char'001&\char'002&\char'003&\char'004&\char'005&\char'006&\char'007&&&
\char'010&\char'011&\char'012&\char'013&\char'014&\char'015&\char'016&\char'017&&&
\char'020&\char'021&\char'022&\char'023&\char'024&\char'025&\char'026&\char'027&&&
\char'030&\char'031&\char'032&\char'033&\char'034&\char'035&\char'036&\char'037\cr
\char'040&\char'041&\char'042&\char'043&\char'044&\char'045&\char'046&\char'047&&&
\char'050&\char'051&\char'052&\char'053&\char'054&\char'055&\char'056&\char'057&&&
\char'060&\char'061&\char'062&\char'063&\char'064&\char'065&\char'066&\char'067&&&
\char'070&\char'071&\char'072&\char'073&\char'074&\char'075&\char'076&\char'077\cr
\char'100&\char'101&\char'102&\char'103&\char'104&\char'105&\char'106&\char'107&&&
\char'110&\char'111&\char'112&\char'113&\char'114&\char'115&\char'116&\char'117&&&
\char'120&\char'121&\char'122&\char'123&\char'124&\char'125&\char'126&\char'127&&&
\char'130&\char'131&\char'132&\char'133&\char'134&\char'135&\char'136&\char'137\cr
\char'140&\char'141&\char'142&\char'143&\char'144&\char'145&\char'146&\char'147&&&
\char'150&\char'151&\char'152&\char'153&\char'154&\char'155&\char'156&\char'157&&&
\char'160&\char'161&\char'162&\char'163&\char'164&\char'165&\char'166&\char'167&&&
\char'170&\char'171&\char'172&\char'173&\char'174&\char'175&\char'176&\char'177\cr}}
Greek Letters, in Math mode ({\tt \$\\gamma\$}, or {\tt \$\\Gamma\$} for upper case where shown)
\smallskip
\vbox{\halign{\hfil#\hfil\quad&\hfil#\hfil\quad&\tt #\hfil\qquad\qquad&
        \hfil#\hfil\quad&\hfil#\hfil\quad&\tt #\hfil\qquad\qquad&
        \hfil#\hfil\quad&\hfil#\hfil\quad&\tt #\hfil\cr
&$\alpha$&\\alpha&&$\iota$&\\iota&&$\varrho$&\\varrho\cr
&$\beta$&\\beta&&$\kappa$&\\kappa&$\Sigma$&$\sigma$&\\sigma\cr
$\Gamma$&$\gamma$&\\gamma&$\Lambda$&$\lambda$&\\lambda&&$\varsigma$&\\varsigma\cr
$\Delta$&$\delta$&\\delta&&$\mu$&\\mu&&$\tau$&\\tau\cr
&$\epsilon$&\\epsilon&&$\nu$&\\nu&$\Upsilon$&$\upsilon$&\\upsilon\cr
&$\varepsilon$&\\varepsilon&$\Xi$&$\xi$&\\xi&$\Phi$&$\phi$&\\phi\cr
&$\zeta$&\\zeta&&$o$&o&&$\varphi$&\\varphi\cr
&$\eta$&\\eta&$\Pi$&$\pi$&\\pi&&$\chi$&\\chi\cr
$\Theta$&$\theta$&\\theta&&$\varpi$&\\varpi&$\Psi$&$\psi$&\\psi\cr
&$\vartheta$&\\vartheta&&$\rho$&\\rho&$\Omega$&$\omega$&\\omega\cr
}}

\eject
\def\smallerskip{\vskip 2pt plus 1pt minus 1pt}
\null\vskip-30pt
%\vsize=9.2 truein
Miscellaneous special symbols, available in Math mode
\smallerskip
\halign{\hfil#\hfil\quad&\tt #\hfil\qquad\qquad&
        \hfil#\hfil\quad&\tt #\hfil\qquad\qquad&
        \hfil#\hfil\quad&\tt #\hfil\cr
$\aleph$&{\tt \\aleph}&$\prime$&{\tt \\prime}&$\forall$&{\tt \\forall}\cr
$\hbar$&{\tt \\hbar}&$\emptyset$&{\tt \\emptyset}&$\exists$&{\tt \\exists}\cr
$\imath$&{\tt \\imath}&$\nabla$&{\tt \\nabla}&$\neg$&{\tt \\neg}\cr
$\jmath$&{\tt \\jmath}&$\surd$&{\tt \\surd}&$\flat$&{\tt \\flat}\cr
$\ell$&{\tt \\ell}&$\top$&{\tt \\top}&$\natural$&{\tt \\natural}\cr
$\wp$&{\tt \\wp}&$\bot$&{\tt \\bot}&$\sharp$&{\tt \\sharp}\cr
$\Re$&{\tt \\Re}&$\Vert$&{\tt \\Vert}&$\clubsuit$&{\tt \\clubsuit}\cr
$\Im$&{\tt \\Im}&$\angle$&{\tt \\angle}&$\diamondsuit$&{\tt \\diamondsuit}\cr
$\partial$&{\tt \\partial}&$\triangle$&{\tt \\triangle}&$\heartsuit$&{\tt \\heartsuit}\cr
$\infty$&{\tt \\infty}&$\backslash$&{\tt \\backslash}&$\spadesuit$&{\tt \\spadesuit}\cr
\noalign{\vskip -2pt}
%\noalign{\smallerskip}
\noalign{Binary operators (in addition to $+$, $-$, and $*$) available in Math mode}
\noalign{\smallerskip}
$\pm$&{\tt \\pm}&$\cap$&{\tt \\cap}&$\vee$&{\tt \\vee}\cr
$\mp$&{\tt \\mp}&$\cup$&{\tt \\cup}&$\wedge$&{\tt \\wedge}\cr
$\setminus$&{\tt \\setminus}&$\uplus$&{\tt \\uplus}&$\oplus$&{\tt \\oplus}\cr
$\cdot$&{\tt \\cdot}&$\sqcap$&{\tt \\sqcap}&$\ominus$&{\tt \\ominus}\cr
$\times$&{\tt \\times}&$\sqcup$&{\tt \\sqcup}&$\otimes$&{\tt \\otimes}\cr
$\ast$&{\tt \\ast}&$\triangleleft$&{\tt \\triangleleft}&$\oslash$&{\tt \\oslash}\cr
$\star$&{\tt \\star}&$\triangleright$&{\tt \\triangleright}&$\odot$&{\tt \\odot}\cr
$\diamond$&{\tt \\diamond}&$\wr$&{\tt \\wr}&$\dagger$&{\tt \\dagger}\cr
$\circ$&{\tt \\circ}&$\bigcirc$&{\tt \\bigcirc}&$\ddagger$&{\tt \\ddagger}\cr
$\bullet$&{\tt \\bullet}&$\bigtriangleup$&{\tt \\bigtriangleup}&$\amalg$&{\tt \\amalg}\cr
$\div$&{\tt \\div}&$\bigtriangledown$&{\tt \\bigtriangledown}\cr
\noalign{\vskip -2pt}
\noalign{Relations (in addition to $<$, $>$, and $=$), available in Math mode}
\noalign{\smallerskip}
$\leq$&{\tt \\leq}&$\geq$&{\tt \\geq}&$\equiv$&{\tt \\equiv}\cr
$\prec$&{\tt \\prec}&$\succ$&{\tt \\succ}&$\sim$&{\tt \\sim}\cr
$\preceq$&{\tt \\preceq}&$\succeq$&{\tt \\succeq}&$\simeq$&{\tt \\simeq}\cr
$\ll$&{\tt \\ll}&$\gg$&{\tt \\gg}&$\asymp$&{\tt \\asymp}\cr
$\subset$&{\tt \\subset}&$\supset$&{\tt \\supset}&$\approx$&{\tt \\approx}\cr
$\subseteq$&{\tt \\subseteq}&$\supseteq$&{\tt \\supseteq}&$\cong$&{\tt \\cong}\cr
$\sqsubseteq$&{\tt \\sqsubseteq}&$\sqsupseteq$&{\tt \\sqsupseteq}&$\bowtie$&{\tt \\bowtie}\cr
$\in$&{\tt \\in}&$\ni$&{\tt \\ni}&$\propto$&{\tt \\propto}\cr
$\vdash$&{\tt \\vdash}&$\dashv$&{\tt \\dashv}&$\models$&{\tt \\models}\cr
$\smile$&{\tt \\smile}&$\mid$&{\tt \\mid}&$\doteq$&{\tt \\doteq}\cr
$\frown$&{\tt \\frown}&$\parallel$&{\tt \\parallel}&$\perp$&{\tt \\perp}\cr
\noalign{\vskip -2pt}
%\noalign{\smallerskip}
\noalign{Negated relations (the {\tt\\not} symbol is considered to have zero width)}
\noalign{\smallerskip}
$\not<$&{\tt \\not<}&$\not>$&{\tt \\not>}&$\not=$&{\tt \\not=}\cr
$\not\leq$&{\tt \\not\\leq}&$\not\geq$&{\tt \\not\\geq}&
  $\not\equiv$&{\tt \\not\\equiv}\cr
$\not\prec$&{\tt \\not\\prec}&$\not\succ$&{\tt \\not\\succ}&
  $\not\sim$&{\tt \\not\\sim}\cr
$\not\preceq$&{\tt \\not\\preceq}&$\not\succeq$&{\tt \\not\\succeq}&
  $\not\simeq$&{\tt \\not\\simeq}\cr
$\not\subset$&{\tt \\not\\subset}&$\not\supset$&{\tt \\not\\supset}&
  $\not\approx$&{\tt \\not\\approx}\cr
$\not\subseteq$&{\tt \\not\\subseteq}&$\not\supseteq$&{\tt \\not\\supseteq}&
  $\not\cong$&{\tt \\not\\cong}\cr
$\not\sqsubseteq$&{\tt \\not\\sqsubseteq}&$\not\sqsupseteq$&{\tt \\not\\sqsupseteq}&
  $\not\asymp$&{\tt \\not\\asymp}\cr}
\eject
Arrows for use in Math mode
\smallskip
\halign{\hfil#\hfil\quad&\tt #\hfil\quad&
        \hfil#\hfil\quad&\tt #\hfil\qquad&
        \hfil#\hfil\quad&\tt #\hfil\cr
$\leftarrow$&{\tt \\leftarrow}&$\longleftarrow$&{\tt \\longleftarrow}&
  $\uparrow$&{\tt \\uparrow}\cr
$\Leftarrow$&{\tt \\Leftarrow}&$\Longleftarrow$&{\tt \\Longleftarrow}&
  $\Uparrow$&{\tt \\Uparrow}\cr
$\rightarrow$&{\tt \\rightarrow}&$\longrightarrow$&{\tt \\longrightarrow}&
  $\downarrow$&{\tt \\downarrow}\cr
$\Rightarrow$&{\tt \\Rightarrow}&$\Longrightarrow$&{\tt \\Longrightarrow}&
  $\Downarrow$&{\tt \\Downarrow}\cr
$\leftrightarrow$&{\tt \\leftrightarrow}&$\longleftrightarrow$&{\tt \\longleftrightarrow}&
  $\updownarrow$&{\tt \\updownarrow}\cr
$\Leftrightarrow$&{\tt \\Leftrightarrow}&$\Longleftrightarrow$&{\tt \\Longleftrightarrow}&
  $\Updownarrow$&{\tt \\Updownarrow}\cr
$\mapsto$&{\tt \\mapsto}&$\longmapsto$&{\tt \\longmapsto}&
  $\nearrow$&{\tt \\nearrow}\cr
 $\searrow$&{\tt \\searrow}&$\swarrow$&{\tt \\swarrow}&$\nwarrow$&{\tt \\nwarrow}\cr
$\hookleftarrow$&{\tt \\hookleftarrow}&$\hookrightarrow$&{\tt \\hookrightarrow}\cr}
%$\lefttophalfarrow$&{\tt \\lefttophalfarrow}&$\righttophalfarrow$&{\tt \\righttophalfarrow}&
%$\leftbothalfarrow$&{\tt \\leftbothalfarrow}&$\rightbothalfarrow$&{\tt \\rightbothalfarrow}&
%$\rightlefthalfarrows$&{\tt \\rightlefthalfarrows}\cr}
%smallskip
Some alternate names used in Math mode
\smallskip
\halign{\hfil#\hfil\quad&#\hfil\qquad\qquad&&
	\hfil#\hfil\quad&#\hfil\qquad\qquad\cr
$\ne$&{\tt \\ne}&\lbr &{\tt \\\lbr }&$\land$&{\tt \\land}&
$\to$&{\tt \\to}&$\vert$&{\tt \\vert}\cr
$\le$&{\tt \\le}&\rbr&{\tt \\\rbr}&$\lor$&{\tt \\lor}&
$\gets$&{\tt \\gets}&$\Vert$&{\tt \\Vert}\cr
$\ge$&{\tt \\ge}&$\owns $&{\tt \\owns}&$\lnot$&{\tt \\lnot}\cr}

%smallskip
Large operators as used in Math ({\tt\$}\dots{\tt\$})
and in Math Display ({\tt\$\$} \dots{\tt\$\$}) modes
\smallskip
{\displayindent=16pt \openup3pt
\halign{&\qquad\hbox to10pt{\hss$#$\hss}\quad&
  \hbox to10pt{\hss$\displaystyle#$\hss}\quad&
  \hbox to65pt{#\hss}\cr
\sum&\sum&{\tt \\sum}&\bigcap&\bigcap&{\tt \\bigcap}&
  \bigodot&\bigodot&{\tt \\bigodot}\cr
\prod&\prod&{\tt \\prod}&\bigcup&\bigcup&{\tt \\bigcup}&
  \bigotimes&\bigotimes&{\tt \\bigotimes}\cr
\coprod&\coprod&{\tt \\coprod}&\bigsqcup&\bigsqcup&{\tt \\bigsqcup}&
  \bigoplus&\bigoplus&{\tt \\bigoplus}\cr
\int&\int&{\tt \\int}&\bigvee&\bigvee&{\tt \\bigvee}&
  \biguplus&\biguplus&{\tt \\biguplus}\cr
\oint&\oint&{\tt \\oint}&\bigwedge&\bigwedge&{\tt \\bigwedge}\cr
}}
%\smallskip
Extendable delimiters, in addition to ( \ ) \ $\{$  \ $\}$ \ [ \ and ], used in Display Math mode
\smallskip
{\halign{\hfil$#$\hfil\quad&\tt #\hfil\qquad\quad&
\hfil$#$\hfil\quad&\tt #\hfil\qquad&
\hfil$#$\hfil\quad&\tt #\hfil\qquad\quad&
\hfil$#$\hfil\quad&\tt #\hfil\cr
\lfloor&\\lfloor& \langle&\\langle& \vert&\\vert& \downarrow&\\downarrow\cr
\rfloor&\\rfloor& \rangle&\\rangle& \Vert&\\Vert& \Downarrow&\\Downarrow\cr
\lceil&\\lceil& /&/& \uparrow&\\uparrow& \updownarrow&\\updownarrow\cr
\rceil&\\rceil& \\&\\backslash& \Uparrow&\\Uparrow& \Updownarrow&\\Updownarrow\cr
}}
%\smallskip
Function names that print in roman type, for use when in Math mode
\smallskip
{\tt \halign{#\hfil\quad&&#\hfil\quad\cr
\\arccos  &\\cos   &\\csc   &\\exp   &\\ker     &\\limsup  &\\min  &\\sinh\cr
\\arcsin  &\\cosh  &\\deg   &\\gcd   &\\lg      &\\ln      &\\Pr   &\\sup\cr
\\arctan  &\\cot   &\\det   &\\hom   &\\lim     &\\log     &\\sec  &\\tan\cr
\\arg     &\\coth  &\\dim   &\\inf   &\\liminf  &\\max     &\\sin  &\\tanh\cr
}}
%\smallskip
Dimensions, preset by {\tt PLAIN.TEX}, that you may want to change
\smallskip
{\ttt \halign{#\hfil\qquad&&#\hfil\qquad\cr
\\hsize=6.5in&
\\parindent=20pt&
\\abovedisplayskip=12pt plus 3pt minus 9pt\cr
\\vsize=8.9in&
\\parskip=0pt plus 1pt&
\\belowdisplayskip=12pt plus 3pt minus 9pt\cr}}
\vfill
\eject
\leftline{Macros for setting ordinary text}
\vskip-20pt
$$\vbox{\halign{\ttt #\hfil\quad&&#\hfil\quad\cr
\\frenchspacing&\\break&\\bigbreak&\\hidewidth&\\ttraggedright\cr
\\nonfrenchspacing&\\nobreak&\\line&\\multispan\#1&\\leavevmode\cr
\\normalbaselines&\\allowbreak&\\leftline\#1&\\cleartabs&\\hrulefill\cr
\\null&\\filbreak&\\rightline\#1&\\hang&\\dotfill\cr
\\obeyspaces&\\goodbreak&\\centerline\#1&\\textindent\#1&\\downbracefill\cr
\\loop\#1\\repeat&\\eject&\\rlap\#1&\\item&\\upbracefill\cr
\\iterate&\\supereject&\\llap\#1&\\itemitem\cr
\\nointerlineskip&\\smallbreak&\\underbar\#1&\\narrower\cr
\\offinterlineskip&\\medbreak&\\strut&\\raggedright\cr}}$$
\leftline{Macros for setting math}
$$\vbox{\ttt \halign{#\hfil\qquad&&#\hfil\qquad\cr
\\joinrel&\\vphantom&\\openup&\\cases\#1&\\ldots\cr
\\relbar&\\hphantom&\\eqalign\#1&\\matrix\#1&\\cdots\cr
\\Relbar&\\phantom&\\displaylines\#1&\\pmatrix\#1&\\vdots\cr
\\bowtie&\\mathstrut&\\eqalignno\#1&\\bordermatrix\#1&\\ddots\cr
\\models&\\smash&\\leqalignno\#1\cr}}$$
\leftline{Parameters preset by {\tt PLAIN.TEX} that may be reset, with caution}
$$\vbox{\ttt \halign{#\hfil\qquad&&#\hfil\qquad\cr
\\vbadness=1000&\\tracinglostchars=1&\\delimitershortfall=5pt\cr
\\linepenalty=10&\\uchyph=1&\\nulldelimiterspace=1.2pt\cr
\\hyphenpenalty=50&\\newlinechar=-1&\\scriptspace=0.5pt\cr
\\exhyphenpenalty=50&\\delimiterfactor=901&\\parindent=20pt\cr
\\binoppenalty=700&\\showboxbreadth=5&\\parskip=0pt plus 1pt\cr
\\relpenalty=500&\\showboxdepth=3&\\parfillskip=0pt plus 1fil\cr
\\hbadness=1000&\\adjdemerits=10000&\\topskip=10pt\cr
\\clubpenalty=150&\\hfuzz=0.1pt&\\maxdepth=4pt\cr
\\widowpenalty=150&\\vfuzz=0.1pt&\\normalbaselineskip=12pt\cr
\\displaywidowpenalty=50&\\overfullrule=5pt&\\normallineskip=1pt\cr
\\brokenpenalty=100&\\hsize=6.5in&\\jot=3pt\cr
\\predisplaypenalty=10000&\\vsize=8.9in&\\tolerance=200\cr}}$$
$$\vbox{\ttt \halign{#\hfil\quad&&#\hfil\quad\cr
\\doublehyphendemerits=10000&\\belowdisplayskip=12pt plus 3pt minus 9pt\cr
\\finalhyphendemerits=5000&\\belowdisplayshortskip=7pt plus 3pt minus 4pt\cr
\\thinmuskip=3mu&\\smallskipamount=3pt plus 1pt minus 1pt\cr
\\medmuskip=4mu plus 2mu minus 4mu&\\medskipamount=6pt plus 2pt minus 2pt\cr
\\thickmuskip=5mu plus 5mu&\\bigskipamount=12pt plus 4pt minus 4pt\cr
\\interdisplaylinepenalty=100&\\abovedisplayskip=12pt plus 3pt minus 9pt\cr
\\interfootnotelinepenalty=100&\\abovedisplayshortskip=0pt plus 3pt\cr}}$$
\leftline{Unassigned parameters, set to zero automatically}
$$\vbox{\ttt \halign{#\hfil\qquad&&#\hfil\qquad\cr
\\pausing&\\tracingparagraphs&\\tracingrestores&\\spaceskip\cr
\\tracingonline&\\tracingpages&\\leftskip&\\hoffset\cr
\\tracingmacros&\\tracingoutput&\\rightskip&\\voffset\cr
\\tracingstats&\\tracingcommands&\\tabskip\cr}}$$

\leftline{Fonts preloaded and named by {\tt PLAIN.TEX}.}
\leftline{Can be magnified by $\sqrt{1.2}$, $1.2$,
$(1.2)^2$, $(1.2)^3$, $(1.2)^4$, and $(1.2)^5$.}
\vskip-20pt
$$\vbox{\halign{#\hfil \qquad&#\hfil\enskip&#\hfil\qquad&
#\hfil\enskip&#\hfil\qquad&#\hfil&\enskip#\hfil\cr
\rm roman text&\\tenrm&amr10&\\sevenrm&amr7&\\fiverm&amr5\cr
\bf boldface extended&\\tenbf&ambx10&\\sevenbf&ambx7&\\fivebf&ambx5\cr
\teni math$\;$italic&\\teni&ammi10&\\seveni&ammi7&\\fivei&ammi5\cr
math symbols&\\tensy&amsy10&\\sevensy&amsy7&\\fivesy&amsy5\cr
math extension&\\tenex&amex10\cr
\tt typewriter&\\tentt&amtt10\cr
\sl slanted roman&\\tensl&amsl10\cr
\it text italic&\\tenit&amti10\cr}}$$
\leftline{Fonts preloaded by {\tt PLAIN.TEX} but un-named.}
\leftline{Can be magnified by $\sqrt{1.2}$, $1.2$, and $(1.2)^2$
(\\magstephalf, \\magstep1, and \\magstep2).}
$$\vbox{\halign{#\hfil\qquad&#\hfil\enskip&#\hfil\enskip&#\hfil\qquad\enskip&
#\hfil\qquad&#\hfil \enskip&#\hfil\cr
roman text&amr9&amr8&amr6&sans serif&amss10&amssq8\cr
math italic&ammi9&ammi8&ammi6&sans serif italic&amssi10&amssqi8\cr
math symbols&amsy9&amsy8&amsy6&slanted roman&amsl9&amsl8\cr
bold face&ambx9&ambx8&ambx6&typewriter&amtt9&amtt8\cr
text italic&amti9&amti8&amti7&slanted typewriter&amsltt10 \cr}}$$
$$\vbox{\halign{#\hfil\quad&#\hfil\qquad\quad&#\hfil\quad&#\hfil\cr
unslanted text italic&amu10&Dunhill style&amdunh10\cr
bold math italic&ambi10&
sans serif bold extended&amssbx10\cr
bold math symbols&ambsy10&
caps and small caps&amcsc10\cr}}$$\medskip
\leftline{Preloaded in magnified form for titles but un-named.}
\vskip -20pt
$$\vbox{\halign{#\hfil\qquad&#\hfil\qquad&#\hfil\cr
amr7 scaled \\magstep4&
amtt10 scaled \\magstep2&
amssbx10 scaled \\magstep2\cr}}$$

\vfill\eject
\nopagenumbers
\def\leaderfill{\leaders\hbox to 1em{\hss.\hss}\hfill}
\null
\vfill
\centerline{\bf Table of Contents}
\smallskip
\line{\qquad\qquad Introduction\leaderfill 1\qquad\qquad}
\smallskip
\line{\qquad\qquad Toward Book Quality\leaderfill 2\qquad\qquad}
\smallskip
\line{\qquad\qquad Special Symbols\leaderfill 3\qquad\qquad}
\smallskip
\line{\qquad\qquad Issuing Commands to \TeX\leaderfill 3\qquad\qquad}
\smallskip
\line{\qquad\qquad Fonts\leaderfill 4\qquad\qquad}
\smallskip
\line{\qquad\qquad Dimensions and Keywords\leaderfill 5\qquad\qquad}
\smallskip
\line{\qquad\qquad To Use \TeX \leaderfill 5\qquad\qquad}
\smallskip
\line{\qquad\qquad Boxing with Glue\leaderfill 6\qquad\qquad}
\smallskip
\line{\qquad\qquad An Example\leaderfill 7\qquad\qquad}
\smallskip
\line{\qquad\qquad Understanding Error Messages\leaderfill 10\qquad\qquad}
\smallskip
\line{\qquad\qquad The Six Modes\leaderfill 11\qquad\qquad}
\smallskip
\line{\qquad\qquad Making Tables\leaderfill 12\qquad\qquad}
\smallskip
\line{\qquad\qquad The `\\halign' Alignment Method\leaderfill 12\qquad\qquad}
\smallskip
\line{\qquad\qquad The Fixed-Column-Width `\\settabs' Method\leaderfill 16\qquad\qquad}
\smallskip
\line{\qquad\qquad The Variable-Column-Width `\\settabs' Method\leaderfill 17\qquad\qquad}
\smallskip
\line{\qquad\qquad Typing Mathematical Formulas\leaderfill 17\qquad\qquad}
\smallskip
\line{\qquad\qquad Some Odds and Ends\leaderfill 26\qquad\qquad}
\smallskip
\line{\qquad\qquad Output Routines\leaderfill 28\qquad\qquad}
\smallskip
\line{\qquad\qquad Defining Macros\leaderfill 28\qquad\qquad}
\smallskip
\line{\qquad\qquad Appendix\leaderfill 30\qquad\qquad}

\vfill
\centerline{\bf Acknowledgements}
\smallskip
{\parindent=39pt
\narrower
It is impractical to mention all of the people who have
proofread this manual and who have contributed many valuable
suggestions.  The detailed contributions made by Don Knuth,
Mrs.~Knuth, Dave Fuchs, Arthur Keller, Ron Bracewell, and Howard Trickey
were particularly helpful. Needlessly to say, much of the basic material in this
manual came directly or indirectly from The~\TeX book by Don Knuth. \smallskip}

 \vfill \eject
%This is the title page, put last so as to permit the \end termination.
\vfill
\eject
\nopagenumbers
\null\vskip-46pt
\hbox to 6.5truein {November 1983\hfil Report No. STAN-CS-83-985}
\vskip .1in
\line{Stanford Department of Computer Science\hfil (Version 1.1)}
\vfill
\centerline{\bf First Grade \TEX}
\vskip .1in
\centerline{\bf A Beginner's \TEX\ Manual}
\vskip .25in
\centerline{by}
\centerline{Arthur L. Samuel}
\vfill
This manual is based on the publications of Donald~E.~Knuth who originated
the \TEX\ system and on the recent work of Professor Knuth and his many
students and collaborators who have helped bring the \TEX82 system to its
present advanced state of development. The \TEX\ logo that is used in this
manual is a trademark of The American Mathematical Society. The preparation
of this report was supported in part by National Science Foundation grant
IST-820/926 and by the System Development Foundation.

\eject\end
