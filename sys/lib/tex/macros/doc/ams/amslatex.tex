%%%%%%%%%%%%%%%%%%%%%%%%%%%%%%%%%%%%%%%%%%%%%%%%%%%%%%%%%%%%%%%%%%%%%%%%%%%%
% AMSLATEX.TEX						    August 1990    %
%                                                                          %
% This file is part of the AMS-LaTeX Version 1.0 distribution              %
%   American Mathematical Society, Technical Support Group,                %
%   P. O. Box 6248, Providence, RI 02940                                   %
%   800-321-4AMS (321-4267) or 401-455-4080                                %
%   Internet: Tech-Support@Math.AMS.com                                    %
%%%%%%%%%%%%%%%%%%%%%%%%%%%%%%%%%%%%%%%%%%%%%%%%%%%%%%%%%%%%%%%%%%%%%%%%%%%%
%% A file for the User's Guide for the AMS-LaTeX macro package.
%%
\ifx\undefined\selectfont
% If we are using an old LaTeX format file that does not incorporate
% the Mittelbach--Sch"opf font selection scheme, we will
% use the following documentstyle line:
\documentstyle{article}
%  Otherwise we will use the following documentstyle declaration:
\else
\documentstyle[oldlfont]{article}
\fi

\makeindex % write index information to the file amslatex.idx

% Some special definitions used in producing this documentation:
%%%%%%%%%%%%%%%%%%%%%%%%%%%%%%%%%%%%%%%%%%%%%%%%%%%%%%%%%%%%%%%%%%%%%%%%%%%%
% EXTRADEF.TEX						    July 1990      %
%                                                                          %
% This file is part of the AMS-LaTeX Version 1.0 distribution              %
%   American Mathematical Society, Technical Support Group,                %
%   P. O. Box 6248, Providence, RI 02940                                   %
%   800-321-4AMS (321-4267) or 401-455-4080                                %
%   Internet: Tech-Support@Math.AMS.com                                    %
%%%%%%%%%%%%%%%%%%%%%%%%%%%%%%%%%%%%%%%%%%%%%%%%%%%%%%%%%%%%%%%%%%%%%%%%%%%%
%%
%% Special definitions for use in producing some parts of the
%% AMS-LaTeX User's Guide.
% We allow some slop at the right margin because we have some
% long control sequence names and verbatim text to deal with.
\hfuzz2pc

\makeatletter

% Change hyphenation inside \tt text back to normal:
\let\-=\@@hyph  \let\@dischyph=\@@hyph  \let\@nohyphens\@gobbletwo
{\footnotesize\tt \hyphenchar\the\font=`\- 
\small\tt \hyphenchar\the\font=`\- 
\normalsize\tt \hyphenchar\the\font=`\-
\large\tt \hyphenchar\the\font=`\-
}

\chardef\bslash=`\\ % p. 424, TeXbook
% \kern\z@ here inhibits hyphenation except at \-

% control sequence
\def\cs#1{{\tt\bslash\kern\z@#1}\index{#1@{\tt\bslash#1}}}

% LaTeX documentstyle name
\def\sty#1{{\tt\kern\z@#1}\index{#1@{\tt{}#1} documentstyle}}

% LaTeX option name
\def\opt#1{{\tt\kern\z@#1}\index{#1@{\tt{}#1} option}}

% environment name
\def\env#1{{\tt\kern\z@#1}\index{#1@{\tt{}#1} environment}}

% file name
\def\fn#1{{\tt\kern\z@#1}\index{#1@{\tt{}#1}}}

% to index a control sequence without printing it
\def\indexcs#1{\index{#1@{\tt\bslash#1}}}

% Macros for the various macro package names.  \AmS and \LaTeX are
% defined using \protect to avoid writing long strings to the .aux
% file, which would be a problem on some computers.
\def\AmS{\protect\pAmS}            \def\LaTeX{\protect\pLaTeX}
\def\pAmS{{\the\textfont2
        A\kern-.1667em\lower.5ex\hbox{M}\kern-.125emS}}
\def\pLaTeX{{\rm L\kern-.36em\raise.3ex\hbox{\the\scriptfont0 A}\kern-.15em
    T\kern-.1667em\lower.7ex\hbox{E}\kern-.125emX}}

\def\amstex/{\AmS-\TeX}         \def\amslatex/{\AmS-\LaTeX{}}
\def\latex/{\LaTeX{}}           \def\pictex/{PIC\TeX}
\def\tex/{\TeX}                 \def\jt/{{\it Joy of \TeX}}
\def\bibtex/{{\sc Bib\kern-.1em\TeX}}     \def\tugboat/{{\it TUGboat\/}}
\def\amsfonts/{AMSFonts}

% `Meta' macro.
\def\<#1>{{$\langle$\it#1\/$\rangle$}}

% Indent a little on the left in the verbatim environment.
\def\verbatim{\interlinepenalty\@M \@verbatim
  \leftskip\@totalleftmargin\advance\leftskip2pc
  \frenchspacing\@vobeyspaces \@xverbatim}

% To introduce permissible breakpoints for line breaks in verbatim text:
\def\5{\penalty500 }

% A modified form of \sloppypar, to be used at the end of a paragraph:
\def\sloppypar{{\tolerance9999\par}}

\makeatother
\endinput


% Used in the first appendix, for listings of files:
\newenvironment{filelist}{\par
  \addvspace{\medskipamount}\hrule\nobreak\addvspace{\medskipamount}\noindent
  \begin{tabular*}{\columnwidth}[t]{p{7pc}@{\extracolsep{\fill}}p{20pc}}}%
{\end{tabular*}\par}

\begin{document}
\begin{titlepage}
\pagestyle{empty}
\title{\amslatex/ Version 1.0\\User's Guide}
\author{American Mathematical Society}
\date{August 1990}
\maketitle
\end{titlepage}

% By not using arabic numbers for the table of contents pages, we
% avoid having to run this file three times to get correct page
% numbers.
\pagenumbering{roman}
\tableofcontents

\newpage \pagenumbering{arabic}
\part{General}
\section{Introduction}

The necessary documentation for using the \amslatex/ package has two
parts: this {\em User's Guide\/}, and some sample files illustrating 
the features available in the
\amslatex/ package.  The file used to produce this {\it User's Guide}
is \fn{amslatex.tex}; the sample files are named
\fn{testart.tex} and \fn{testbook.tex}.
Installation instructions for the
\amslatex/ package are found in a separate file, \fn{amltinst.tex}. 
As explained in the \fn{amltinst.tex} file, installation requires
making a new \latex/ format file.
This {\em User's Guide}, however, 
can be typeset without the new format file,
so that users can read it before proceeding further if they wish.
As a consequence, though, it was
impractical in many cases to show sample output for commands from the
\opt{amstex} option; this is done instead in the sample file
\fn{testart.tex}.  In the {\em User's Guide\/}
approximate output has been shown for the purposes
of illustration when it was practical to do so in ordinary \latex/.

For best understanding, you should be reasonably familiar with the
\latex/ manual: {\sl\latex/: A document preparation system}, by Leslie
Lamport \cite{lm}. Reading the \jt/ \cite{jt} (the manual for \amstex/)
will help you get the most out of the \amslatex/ software, but is not
mandatory. For users whose background is in \amstex/ rather than \latex/,
there is an appendix describing the ways in which the \latex/ \opt{amstex}
option differs from \amstex/ 2.0.

\subsection{Notes}

The notation \<dimension>, \<number>, and the like will be used to
indicate that an arbitrary dimension or number or whatever
is to be substituted by the user.  By {\it dimension\/} we mean a number
followed by one of \tex/'s standard units {\tt pt}, {\tt pc},
{\tt in}, {\tt mm}, {\tt cm}, and so forth.

It is important in this {\it User's Guide} that we distinguish between the
original, non-\latex/ implementation of \amstex/ and the modified form
of it that constitutes the \latex/ option \opt{amstex}.
Typewriter type will be used for the \latex/ option
\opt{amstex}, and the standard logo \amstex/ will be used for the
original non-\latex/ version.

\section{The \amslatex/ project}

\amstex/ was originally released for general use in 1982.  Its main
strength is that it makes it easy for the user to typeset mathematics,
while taking care of the many details necessary to make the output
satisfy the high standards of mathematical publishing.  It provides a
predefined set of natural commands such as \cs{matrix} and \cs{text} that
make complicated mathematics reasonably convenient to type.  These
commands incorporate the typesetting experience and standards of the
American Mathematical Society, to handle
problematic possibilities without burdening the user: matrices within matrices,
or a word of text within a subscript, and so on.

\amstex/, unlike \latex/ does not have certain features that are
very convenient for authors---automatic numbering that adjusts to
addition or deletion of material being the primary one.  There are
also labor-saving ways provided in \latex/ for preparing such items as indexes,
bibliographies, tables, and simple diagrams.  These
features are such a convenience for authors that the use of \latex/
spread rapidly in the mid-80s (a
reasonably mature version of \latex/ was available by the end of
1983), and the American Mathematical Society began to be asked by its
authors to accept electronic submissions in \latex/.

The obvious question to ask was whether the strengths of \amstex/ could
be combined with the strengths of \latex/, and in 1987 the American
Mathematical Society began to investigate the possibility of doing just
that.  Work on the \amslatex/ project was carried out over
the next three years by Romesh Kumar, a \tex/
consultant in the Chicago area, and by West German \latex/ experts Frank
Mittelbach and Rainer Sch\"opf, with assistance from Michael Downes
of the American Mathematical Society Technical Support staff.

The overall philosophy was to provide \amstex/ commands to the \latex/ user
without deviating from standard \latex/ syntax whenever possible.
Thus, to make their syntax more like normal \latex/ syntax, \amstex/
commands having the form \cs{something}\5$\ldots$\5\cs{endsomething} were
converted to \latex/ environments, so that they now have the form
\cs{begin}\verb"{something}"\5$\ldots$\5\cs{end}\verb"{something}".  For
example, a matrix is typed as
\cs{begin}\verb"{matrix}"\5\dots\5\cs{end}\verb"{matrix}" instead of
\cs{matrix}\5$\ldots$\5\cs{endmatrix}.  Also, some commands that have top
and bottom options were changed so that the option is specified using
\verb"[t]" or \verb"[b]" instead of by a prefix \verb"top" or \verb"bot"
in the command name.  See Appendix~\ref{s:diff} for more details.

A good part of the original \amstex/ was whittled off in the creation of the
\opt{amstex} option.  Many commands were redundant and were simply
dropped; others seemed only marginally useful and were omitted in order
to conserve control sequence memory.  Some internal control sequences
were eliminated by restructuring the code.

\amslatex/ is different enough from the original \amstex/ that using the
\jt/ as documentation would be unsatisfactory.  Instead, this
{\it User's Guide\/} aims to be more or less self-sufficient.  The \jt/
is still recommended reading because it provides background information
that helps explain why some things are handled the way they are.

\section{Major components of the \amslatex/ package}

The first major part of the \amslatex/ package is an extensive modification
of \amstex/ 2.0 that allows it to be used in \latex/ as a documentstyle
option.  In other words, if you are writing an article,
your documentstyle declaration should look like this:
\begin{verbatim}
\documentstyle[amstex]{article}
\end{verbatim}

The second major part of the \amslatex/ package is a pair of documentstyles
called \sty{amsart} and \sty{amsbook}, parallel to \latex/'s \sty{article}
and \sty{book}, which are designed to be used in preparing manuscripts for
submission to the AMS.
As alternatives to the standard \latex/ article and book styles,
the AMS style files also offer the general \latex/
user style files which match the style of AMS publications.
When the \sty{amsart} and \sty{amsbook} style files
are used the \opt{amstex} option will be automatically included, so
that the documentstyle declarations would simply be
\cs{documentstyle}\verb"{amsart}" or \cs{documentstyle}\verb"{amsbook}".

The analog in \amstex/ of the \sty{amsart} documentstyle is the
documentstyle \sty{amsppt}  (``AMS preprint'').  In \sty{amsart} and
\sty{amsbook} the document structure commands of the {\tt amsppt} style
described in Appendix~A of the \jt/ have been superseded by their \latex/
equivalents, where equivalents existed, and otherwise have been
reimplemented in \latex/ form.  In order to save a significant amount of
memory, the bibliography commands described in
Appendix~C of the \jt/ have been dropped in favor of \bibtex/.

\newpage
\part{Font considerations}
\label{s:fonts}

\section{The font selection scheme of Mittelbach and Sch\"opf}

In order to provide not only access to the \amsfonts/ currently
available but a general, reliable mechanism for making new math fonts
accessible to the user, the Society enlisted Frank Mittelbach and
Rainer Sch\"opf to adapt their new \latex/ font selection scheme
to accommodate the needs of the \amslatex/ project.  This new scheme has
a couple of distinctive features: (1) fonts (even math fonts) need not
be preloaded but can be loaded on demand; (2) font switches work a bit
differently---attributes are independent, and only one is changed at a
time.  In \latex/ terms this means that, for example, \verb"\bf\Large"
has the same effect as \verb"\Large\bf".

At the present time the files for the new font selection scheme are
being distributed along with the \amslatex/ package, with the
permission of Mittelbach and Sch\"opf; in the future the new scheme
will become part of \latex/, in place of the current scheme.  A
detailed description of the workings of the font selection scheme can
be found in an article by Mittelbach and Sch\"opf that appeared in
\tugboat/, June 1990 (vol.~11, no.~2): {\it The new font family
selection---user interface to standard \latex/\/}
\cite{msf}.  If you don't have access to that article, contact the
\tex/ User's Group (see Appendix~D for the address).

\section{Basic concepts}

In normal use, the ordinary \latex/ commands \cs{rm},
\cs{it}, \cs{tt}, \cs{bf} are defined in terms of more primitive
commands \cs{family} etc., and  still function in much the same way as
before.  Knowledge of the more primitive commands will not be
essential except in documentstyle design or similar tasks.

The Mittelbach--Sch\"opf font selection scheme classifies fonts
based on the attributes
{\em shape}, {\em series}, {\em size}, and {\em family}.
Each attribute can be changed independently using the commands
\cs{shape}, \cs{series}, \cs{size}, and \cs{family}.
For example, to change the family attribute to {\tt cmr} (Computer
Modern roman),
the command would be \cs{family}\verb={cmr}=.
Note that these commands do
not actually select the new font, because it's not uncommon
for you to want to change several attributes at a time
before actually switching to the new font.  The command
for putting the new attributes into effect is \cs{selectfont}.
For example, if the current font is family {\tt cmr},
size 10/12
(10-point type with 12-point baselineskip), series {\tt m} (medium
weight and width), and shape {\tt it} (italic), then the command
\begin{verbatim}
\family{cmtt}\shape{n}\selectfont
\end{verbatim}
would switch to a Computer Modern typewriter font
in the ``normal,'' i.e., upright, shape.  The current size and series
values would be used in the selection of the new font.

\subsection{Shape}
The {\em shape} attribute is one of normal ({\tt n}),
italic ({\tt it}), small caps ({\tt sc}),
slanted (or ``sloped'') ({\tt sl}), and upright italic ({u}).  The
first three of these are the shapes that were typically found
together in the same font case, in the days of manual typesetting.
The latter two are somewhat unusual
variant shapes that are present in the Computer
Modern fonts.

The command to switch to a particular shape, say {\tt sc},
without changing other font attributes would be
\begin{verbatim}
\shape{sc}\selectfont
\end{verbatim}
but there are abbreviations for the most common shape changes:
\cs{sc}, \cs{it}, \cs{sl}, and \cs{normalshape}.  These are
the same as in the previous font selection scheme, except for
\cs{normalshape}, which may be understood as a replacement for
\cs{rm}.  In the new font selection scheme \cs{rm} is a family-changing
command, not a shape-changing command.  If you are dismayed at
the prospect of typing many instances of \cs{normalshape}, which
is obviously much longer than \cs{rm}, don't be.  As you shall
see, many former uses of \cs{rm}, especially in mathematics,
are better handled by other means.  With astute use of grouping,
most documents can be done without using \cs{normalshape} at all.

\subsection{Series}

The series attribute is actually a combination of two related
attributes, weight and width.  The font charts
of type manufacturers typically show weights of light,
medium, and bold, and widths of condensed, medium, and expanded,
with intermediate and extreme variations such as semibold,
extra bold, and ultra bold.  The full list of the weights and
widths allowed for in the Mittelbach--Sch\"opf scheme are as shown
in Table~\ref{weight-width} (adapted from
Table~1 in \cite{msf}), along with their corresponding
 abbreviations for use with the \cs{series} command.
Examples:
\begin{description}
\item[\cs{series}{\tt\char`\{ux\char`\}}\cs{selectfont}]
Switches to an ultra expanded version of the current font.

\item[\cs{series}{\tt\char`\{sbc\char`\}}\cs{selectfont}]
Switches to a semibold condensed version of the current font.

\item[\cs{series}{\tt\char`\{m\char`\}}\cs{selectfont}]
Switches to a medium weight, medium width version of the current font.
\end{description}
Only two series changes are common enough to require abbreviations:
\cs{bf} and \cs{mediumseries} are abbreviations for,
respectively,
\begin{verbatim}
\series{bx}\selectfont   \series{m}\selectfont
\end{verbatim}
or, in other words, ``bold'' and ``not bold''.

\begin{table}[htp]
\caption{Font weights and widths, and their abbreviations.
For use in the {\tt\bslash series\-} command, combine the
weight and width abbreviations, dropping any
{\tt m}'s (for ``medium''), except in the case where
both weight and width are medium: then use a single
{\tt m}.  Examples: Ultra Bold Condensed: {\tt ubc};
Medium Condensed:~{\tt c}.}
\label{weight-width}
\begin{center}
\begin{tabular}{|ll@{\hspace{2\tabcolsep}}ll|}
\hline
Weight&& Width&\\ \hline
\vrule width0pt height10pt
Ultra Light& {\tt ul}&       Ultra Condensed& {\tt uc}\\
Extra Light& {\tt el}&       Extra Condensed& {\tt ec}\\
Light& {\tt l}&              Condensed& {\tt c}\\
Semilight& {\tt sl}&         Semicondensed& {\tt sc}\\
Medium (normal)& {\tt m}&    Medium& {\tt m}\\
Semibold& {\tt sb}&          Semiexpanded& {\tt sx}\\
Bold& {\tt b}&               Expanded& {\tt x}\\
Extra Bold& {\tt eb}&        Extra Expanded& {\tt ex}\\
Ultra Bold& {\tt ub}&        Ultra Expanded& {\tt ux}\\[3pt]
\hline
\end{tabular}
\end{center}
\end{table}

\subsection{Size}
Because a change in font size is usually accompanied by a  change in
baselineskip, the \cs{size} command is designed to take two arguments,
the new size and the new baselineskip.  To switch to 14-point type
with a baselineskip of 18 points, the command would be
\begin{verbatim}
\size{14}{18pt}\selectfont
\end{verbatim}
All the usual \latex/ size-changing commands from \cs{tiny}
to \cs{Huge} have suitable definitions based on the \cs{size} command.

{\em Note}.\ In the specification for the baselineskip, it is necessary
to give the units, because in some situations a unit other than {\tt
pt} may be desirable.

\subsection{Family}\label{s:families}
We define a font {\em family} as a group
of fonts of various shapes, widths, and weights, that
share distinctive design features, such as
x-height, the relative thickness of horizontal and vertical strokes,
distinctive shapes of particular letters, and so forth.
In other words,
fonts in the same family share a resemblance that fonts from
different families don't share (though in some cases the
resemblance is obvious only to an experienced eye).
Table~\ref{fams} gives a classification of some of
the Computer Modern fonts according to family.

\begin{table}[htp]
\caption{Computer Modern font families}
\label{fams}
\begin{center}
\begin{tabular}{|p{12pc}|l|}
\hline
Font file name& Family (and abbreviation)\\
\hline
\tt \raggedright
cmr10, cmti10, cmsl10, cmcsc10, cmu10,
cmbx10, cmbxti, cmbxsl, cmb10&
Computer modern roman ({\tt cmr})\\
\hline
\tt \raggedright
cmss10, cmssi10, cmssbx10, cmssdc10&
Computer modern sans serif ({\tt cmss})\\
\hline
\tt \raggedright
cmtt10, cmitt10, cmsltt, cmtcsc10&
   Computer modern typewriter ({\tt cmtt})\\
\hline
\end{tabular}
\end{center}
\end{table}

The abbreviations \cs{rm}, \cs{tt}, and \cs{sf} are provided for
switching to the Computer Modern roman, typewriter, and sans serif
families.  (The definition of \cs{sf}, for example, is
\verb=\family{cmss}\selectfont=.)

\subsection{Using other font families}
If the base family of a document is Computer Modern roman, with other
families used only sporadically, the other families would be
selected using the \cs{family} command as described in
\S\ref{s:families}.  If you want to change the {\it base family\/}
of the document, however, say to Times Roman or Baskerville,
then the best way is to change the default family settings.
In a canonical setup with all Computer Modern fonts, the following
definitions are in effect:
\begin{verbatim}
\newcommand\rmdefault{cmr}
\newcommand\sfdefault{cmss}
\newcommand\ttdefault{cmtt}
\end{verbatim}
Some or all of these default settings can be changed using
{\tt\bslash renewcommand}.  For example, if you have families
{\tt pstr}, {\tt pshel}, and {\tt pstt} for respectively
PostScript Times Roman, PostScript Helvetica, and Postscript
Typewriter fonts, then you could make them the default
via the commands
\begin{verbatim}
\renewcommand{\rmdefault}{pstr}
\renewcommand{\sfdefault}{pshel}
\renewcommand{\ttdefault}{pstt}
\end{verbatim}
either in the preamble of an individual document, or in an
option file (which then could be used by more than one document).
After these changes, the commands \cs{rm}, \cs{sf}, and \cs{tt}
will select the PostScript families rather than Computer Modern families.
Computer Modern families would still be accessible through explicit
use of the \cs{family} command, e.g., 
\begin{verbatim}
\family{cmtt}\selectfont
\end{verbatim}

Note that in order to
use such alternate families you must
have on your computer system
a fontdef file that defines which fonts belong to the
families {\tt pstr}, {\tt pshel}, and {\tt pstt},
as well as what sizes, shapes, and weights
are available on your particular system;
see the file \fn{fontdef.max} for more details.

In addition to the family defaults, there are defaults for some other
font attributes: \cs{bfdefault}, \cs{itdefault}, \cs{scdefault}, and
\cs{sldefault}. These give further control over fonts. I.e., if you
wanted to have all the slanted fonts in a document come out in italic,
it could be done like this: 
\begin{verbatim}
\renewcommand{\sldefault}{it}
\end{verbatim}
The normal values for these defaults are 
\begin{verbatim}
\bfdefault      bx
\itdefault      it
\scdefault      sc
\sldefault      sl
\end{verbatim}

Notice that by default bold fonts come from the Bold Expanded series
rather than the Bold series.  A comparison of the bold Computer Modern
fonts provided in standard distributions of \tex/ shows why:
\[\begin{tabular}{|l|ll|}
\hline
Bold& \multicolumn{2}{c|}{Bold Expanded}\\[2pt]
\hline
{\tt cmb10}&    {\tt cmbx5}&    {\tt cmbxsl8}\\
        &       {\tt cmbx6}&    {\tt cmbxsl10}\\
        &       {\tt cmbx7}&    {\tt cmbxti7}\\
        &       {\tt cmbx8}&    {\tt cmbxti10}\\
        &       {\tt cmbx9}&\\
        &       {\tt cmbx10}&\\
        &       {\tt cmbx12}&\\
\hline
\end{tabular}\]


\subsection{The {\tt oldlfont} option}
\label{s:oldlfont}

When the Mittelbach--Sch\"opf font selection
scheme is in use, emulation of the old font selection scheme can
be obtained by adding the option \opt{oldlfont} to the documentstyle
options list.  When the \opt{oldlfont} option is used, size-changing
commands return to normal shape and medium series in addition
to changing the font size; \cs{rm} gives normal shape and medium
series; \cs{tt} gives the normal shape and medium series
of the typewriter font; and \cs{sf} gives the normal shape and
medium series of sans serif.

\subsection{Warnings}

When using the Mittelbach--Sch\"opf scheme, the font names listed in
an ``overfull hbox'' message won't look the same as before.  Each font
name will have the family, series, shape, and size, separated by
slashes. For example, 10-point Computer Modern bold extended will
appear as {\tt \bslash cmr/bx/n/10}.  Formerly it would have appeared
as \cs{tenbf}.

Many combinations of font attributes are not available at the
present time because the corresponding fonts do not exist.
The combination
\begin{verbatim}
\family{cmr}\series{bx}\shape{sl}
\end{verbatim}
happens to be available, because the corresponding font file,
{\tt cmbxsl10}, is part of the standard \tex/ distribution.
However, for the combination
\begin{verbatim}
\family{cmss}\series{sbux}\shape{sc}
\end{verbatim}
``Computer Modern sans serif semibold ultra expanded small caps,''
no font file currently exists.

When a combination of font attributes is selected that is
not available, the nearest available font will be substituted,
and a warning message---not an error message, just a warning
message---will appear on-screen during the processing of
the document file.  The warning message will indicate which
font was substituted.

Once in a while,
you may find surprising results from a few commands in standard
\latex/  because they do not reset all the font attributes in the new
font selection scheme.  For example, if the \cs{footnote} command
appears within italic text (e.g., in a theorem), then the text of the
footnote will also be italic, because the standard definition of
\cs{footnote} resets only the {\em size\/} attribute, not the {\em
shape\/} or {\em family\/} or {\em series\/} attributes.  Problems
of this nature
will be rectified in the next version of \latex/ (version 2.10); 
in the meantime, you can add
explicit font commands where needed: to get a normal
footnote in italic text, type\indexcs{normalshape}
\begin{verbatim}
\footnote{\normalshape ...}
\end{verbatim}
instead of just \cs{footnote}.

\section{Names of math font commands}
\label{s:mathfonts}

The single biggest issue in the integration of \amstex/ and \latex/ font
usage was that in \amstex/ math font commands work
differently than text font commands and have different names. Instead
of being a simple switch, whose scope is bounded by curly braces, a math
font command in \amstex/ is a command with one argument.  This means that
in \amslatex/, to obtain a single bold letter in math you type 
\cs{bold}\verb"{A}" rather than \verb"{\bf A}", and two bold letters would
be typed \verb"\bold{A}"\5\verb"\bold{B}" instead of \verb"{\bf AB}".
(A similar distinction between text accents and math accents
 already existed in \latex/.)
Having the font command apply only to a single letter in this way
is more natural in math formulas, because
letters are usually single variables rather
than components of a word, and different fonts are mixed in all
combinations; four consecutive letters might be from four different
fonts.

The full list of math font commands in the \opt{amstex} option is
\cs{mathrm}, \cs{bold}, \cs{cal}, with the addition of \cs{frak}
(Fraktur) and \cs{Bbb} (blackboard bold) if \amsfonts/ are available. 
Math italic, the default font for letters in math, also has a name,
\cs{mit}, but this is never needed in ordinary use.  Tables
\ref{fonttable} and~\ref{mathfonts} give a comprehensive listing of
font change commands for convenient reference.

\begin{table}[htp]
\chardef\{=`\{ \chardef\}=`\}
\caption{Font commands used in text}
\label{fonttable}
\begin{center}
\begin{tabular}{|lll|}
\hline
\multicolumn{1}{|l}{Font command}&   Equivalent&   Font selected\\
\hline
\cs{normalshape}&  \tt\cs{shape}\{n\}&      normal, upright, ``roman''\\
\cs{it}&   \tt\cs{shape}\{it\}&             italic\\
\cs{em}&   \tt\cs{shape}\{it\}$^*$ &          emphasis\\
\cs{sl}&   \tt\cs{shape}\{sl\}&     slanted\\
\cs{sc}&   \tt\cs{shape}\{sc\}&     small caps\\
\cs{mediumseries}&  \tt\cs{series}\{m\}&     medium weight\\
\cs{bf}&            \tt\cs{series}\{bx\}&    bold extended weight\\
\cs{tt}&    \tt\cs{family}\{cmtt\}&          typewriter style\\
\cs{sf}&    \tt\cs{family}\{cmss\}&          sans serif\\
\cs{rm}&    \tt\cs{family}\{cmr\}&           roman\\
\hline
\multicolumn{3}{|l|}{\parbox{20pc}{$^*$\strut The command \cs{em} selects
shape {\tt it} if the current font is upright, otherwise it selects
shape {\tt n} (normal).}}\\[6pt]
\hline
\end{tabular}
\end{center}
\bigskip
\caption{Font commands used in math}
\label{mathfonts}
\begin{center}
\begin{tabular}{|lp{20pc}|}
\hline
\cs{bold}&  Used to obtain bold letters from the English alphabet.\\
\cs{boldsymbol}& Used to obtain bold numbers and other nonalphabetic
        symbols, as well as bold Greek letters.\\
\cs{pmb}&   ``Poor man's bold,'' used for math symbols when
        bold versions don't exist in the currently available fonts.\\
\cs{cal}&   Calligraphic letters. Only uppercase is available.\\
\cs{mit}&   Math italic.  This font is automatically selected
        in math mode, so the command \cs{mit} is not needed in
        normal use.\\
\cs{mathrm}& Roman, normal shape.  Note: most of the time, \cs{text}
        or \cs{operatorname} should be used instead of
        \cs{mathrm} to produce this font in math.\\
\cs{frak}&  Euler Fraktur alphabet.\\
\cs{Bbb}&   Blackboard bold alphabet.  Only uppercase is
        available.\\
\hline
\end{tabular}
\end{center}
\end{table}

To gain access to a new math alphabet, you use the
\cs{new\-math\-alpha\-bet} command in the preamble of your document.
If you have the \amsfonts/ 2.0 package, for example, and you want to use
Russian letters in math, taking them from the University of Washington
Cyrillic fonts, then you need
to find out the family name assigned to the fonts and
the shapes and weights available.  See Table~\ref{fonts}
to see what family names are included in the standard font definition
file \fn{fontdef.max}.  If you made a custom fontdef file
to match your available fonts, look in that file to find
the information.  If you are running \LaTeX{} at a larger
institution where some technical person has been assigned
to handle arcane font matters, you may need to consult that
person.

\begin{table}[htp]
\caption{Font name assignments made in {\tt fontdef.max}}
\label{fonts}
\medskip
\begin{tabular}{|lllp{17pc}|}
\hline
\multicolumn{1}{|c}{Family}&     \multicolumn{1}{c}{Series}&
        \multicolumn{1}{c}{Shape}&  \\
\hline
cmr&   m&     n& Computer Modern Roman\\
cmr&   m&     sl& CM slanted\\
cmr&   m&     it& CM italic\\
cmr&   m&     sc& CM small caps\\
cmr&   m&     u& CM upright italic\\
cmr&   b&     n& CM bold\\
cmr&   bx&    n& CM bold extended\\
cmr&   bx&    sl& CM bold extended slanted\\
cmr&   bx&    it& CM bold extended italic\\
cmss&  m&     n& CM sans serif\\
cmss&  m&     sl& CM sans serif slanted\\
cmss&  sbc&   n& CM sans serif semibold condensed\\
cmss&  bx&    n& CM sans serif bold extended\\
cmtt&  m&     n& CM typewriter\\
cmtt&  m&     it& CM typewriter italic\\
cmtt&  m&     sl& CM typewriter slanted\\
cmtt&  m&     sc& CM typewriter small caps\\
cmm&   m&     it& CM math italic\\
cmm&   b&     it& CM bold math italic\\
cmsy&  m&     n& CM math symbols\\
cmsy&  b&     n& CM bold math symbols\\
lasy&  m&     n& \latex/ extra symbols\\
lasy&  b&     n& \latex/ bold extra symbols\\
msa&   m&     n& AMS extra symbols~A\\
msb&   m&     n& AMS extra symbols~B\\
euf&   m&     n& Euler fraktur\\
euf&   b&     n& Euler fraktur bold\\
eur&   m&     n& Euler roman\\
eur&   b&     n& Euler bold roman\\
eus&   m&     n& Euler script\\
eus&   b&     n& Euler bold script\\
euex&  m&     n& Euler math extension symbols\\
UWCyr& m&     n& University of Washington Cyrillic\\
UWCyr& m&     it& UW Cyrillic italic\\
UWCyr& m&     sc& UW Cyrillic small caps\\
UWCyr& b&     n& UW Cyrillic bold\\
UWCyss& m&    n& UW Cyrillic sans serif\\
ccr&   m&     n& Concrete Roman\\
ccr&   m&     it& Concrete italic\\
ccr&   m&     sc& Concrete small caps\\
ccr&   c&     sl& Concrete condensed slanted\\
ccm&   m&     it& Concrete math italic\\
\hline
\end{tabular}
\end{table}

Suppose, then, that the family name for the University of
Washington fonts is {\tt UWCyr}.  Decide on the name of
the command you'd like to use for Cyrillic,
let's say \cs{cy}.  In the preamble area
of your document, add the line
\begin{verbatim}
\newmathalphabet*{\cy}{UWCyr}{m}{n}
\end{verbatim}
Thenceforth \verb=\cy{A}=, \verb=\cy{d}=, and so on will give you
a Russian A, d, or whatever in math.  Since there is not a one-to-one
correspondence between the Russian alphabet and the English
alphabet, you may need to refer to your documentation to
find out how to obtain certain letters.  The {\it \amsfonts/ User's Guide\/}
\cite{amsfonts} gives a complete table.

If you also want to use bold Russian letters, you could define
another math alphabet and name it, say, \verb=\boldcy=.
Alternatively, you could set things up so that bold Russian
letters are accessible through the commands \cs{boldsymbol}
and \cs{boldmath}.
If you add the line
\begin{verbatim}
\addtoversion{bold}{\cy}{UWCyr}{b}{n}
\end{verbatim}
in your document's preamble,
then \verb=\cy{A}= would produce a normal-weight Russian A
and
\begin{verbatim}
{\boldmath  $ ... \cy{A} ... $ }
\end{verbatim}
would produce a bold Russian A (with the rest of the formula being
made bold as well).  Furthermore, you could then obtain
a bold Russian A in the midst of normal math using \cs{boldsymbol}:
\begin{verbatim}
$ ... \boldsymbol{\cy{A}} ... $
\end{verbatim}

In the \opt{amstex} option \cs{boldsymbol} is to be used for
individual bold math symbols and bold Greek letters---everything in
math except for letters (where you would use \cs{bold}).  For example,
to obtain a bold $\infty$, $+$, $\pi$, or $0$, you would use the
commands \verb"\boldsymbol{\infty}", \verb"\boldsymbol{+}",
\verb"\boldsymbol{\pi}", or \verb"\boldsymbol{0}".   Because they are
not included in the  standard distribution of \tex/ fonts,  sizes
other than 10-point of bold fonts for math symbols, Greek, and math
italic ({\sc CMBSY} and {\sc CMMIB}) are provided in the \amsfonts/
2.0 distribution.

Since \cs{boldsymbol} takes rather a lot of typing, you would usually
put some definitions in the preamble of the form
\begin{verbatim}
\newcommand{\bpi}{\boldsymbol{\pi}}
\newcommand{\binfty}{\boldsymbol{\infty}}
\end{verbatim}
for any bold symbols you're going to use frequently.

For some math symbols \cs{boldsymbol} will not have any effect because
bold versions of those symbols do not exist in the currently available
fonts.  These include
extension symbols and large operator symbols from the font CMEX, as well
as the AMS extra math symbols from the fonts MSAM and MSBM.
``Poor man's bold'' (\cs{pmb}) can be used for some of the
things that aren't handled properly by \cs{boldsymbol}.  It works by
typesetting three copies of the symbol with slight offsets.
With large operators and extension symbols, however, \cs{pmb} does not
currently work very well because the proper spacing and treatment of
limits is not preserved.

To make an entire math formula bold (or as much of it as possible,
depending on the available fonts), use \verb=\boldmath= preceding
the formula, as described in the \latex/ manual.

The sequence \verb={\bf\hat{a}}= (in ordinary \latex/) or
\verb=\bold{\hat{a}}= (in the \opt{amstex} option) produces a bold accent
character over the {\bf a}, as you would expect.
However, combinations like
\verb"{\cal\hat{a}}" will not work because the \cs{cal} font does
not have its own accents.  In the \opt{amstex} option the font change
commands are defined in such a way that accent characters will be
taken from the \cs{rm} font if they are not available in
the current font (in addition to the \cs{cal} font, the \cs{Bbb} and
\cs{frak} fonts don't contain accents).

In ordinary \latex/ uppercase Greek can be made bold by, e.g.,
\verb"{\bf\Gamma}".  In the \opt{amstex} option uppercase
Greek can be made bold only by using \cs{boldsymbol} (in other words,
uppercase Greek is handled the same as lowercase Greek).
\sloppypar

\section{The command {\tt\bslash newsymbol}}
\label{s:newsym}

The command \cs{newsymbol} is presently used only for symbols from the
AMS extra symbol fonts, MSAM and MSBM.  \cs{newsymbol} allows you to create a
control sequence that will properly produce a symbol from the extra symbol
fonts.  The use of \cs{newsymbol} is
explained in the {\em\amsfonts/ User's Guide\/}.
In a \latex/ document
there is one main difference in usage, which is only applicable
if you want to use \amsfonts/ without using the \opt{amstex}
option: instead of using the additional
setting-up commands \cs{loadmsam} and \cs{loadmsbm},
you should put
``\opt{amsfonts}'' in the documentstyle options list.
Otherwise \cs{newsymbol} commands can be used exactly as
shown in the {\em\amsfonts/ User's Guide\/}.  Like \cs{newcommand}'s,
they should be placed in the preamble.

The \opt{amsfonts} option is geared to the current release of
\amsfonts/ (version 2.0).  In this version, some rearranging has been done
and some font names are different than in earlier versions.
If you have an earlier version,
you would need to contact the AMS for an upgrade to version 2.0
in order to use the \opt{amsfonts} option successfully.  See Appendix~D for how
to obtain \amsfonts/.

\section{The {\tt amssymb} option}

If you are running a version of \latex/ with extra memory available
for control sequence names, and you use quite a few of the extra math
symbols from the \amsfonts/, it may be more convenient for you to use
the \opt{amssymb} documentstyle option, which will define all the
symbol names (about 200), so you won't have to include an individual
\cs{newsymbol} command in your document for each one.  You may prefer
to include it in the construction of a format file (see the
installation instructions) to save processing time; it is a stand-alone
option, so it can be included in the format file without including the
\opt{amstex} option.

\newpage
\part{Features of the {\tt amstex} option}

\section{Math spacing commands}
Both\indexcs{,}\indexcs{:}\indexcs{;}\indexcs{thinspace}%
\indexcs{negthinspace}\indexcs{medspace}\indexcs{negmedspace}%
\indexcs{thickspace}\indexcs{negthickspace}
the spelled-out and abbreviated forms of these commands are robust, and
in addition they can also be used outside of math.
The primary math spacing commands are:
\begin{center}
\begin{tabular}{|l|l||l|l|}
Abbrev.&Spelled out&Abbrev.&Spelled out\\
\verb+\,+&\verb+\thinspace+&\verb+\!+&\verb+\negthinspace+\\
\verb+\:+&\verb+\medspace+&&\verb+\negmedspace+\\
\verb+\;+&\verb+\thickspace+&&\verb+\negthickspace+\\
\verb+@,+&&\verb+@!+&\\
&\verb+\quad+&&\\
&\verb+\qquad+&&
\end{tabular}
\end{center}
\verb"@,"\index{"@,@{\tt{}"@,}} and
\verb"@!"\index{"@"!@{\tt{}"@"!}} give
one-tenth the space of \cs{,} and {\tt\bslash !}\index{"!@{\tt\bslash
"!}} respectively, for extra fine tuning where necessary.

\section{Multiple integral signs}
\cs{iint}, \cs{iiint}, and \cs{iiiint} give multiple
integral signs with the spacing between them nicely adjusted,  in both
text and display style.  \cs{idotsint} is an extension of the same
idea that gives two integral signs with dots between them.

\section{Over and under arrows}

There are some additional
over and under arrow operations provided in the \opt{amstex} option:
{\samepage
\begin{tabbing}
\qquad\={\tt\bslash overleftrightarrow\qquad}\=\kill
\> \cs{underleftarrow} \> \cs{underrightarrow} \+\\
\cs{overleftrightarrow}         \> \cs{underleftrightarrow}
\end{tabbing}
}
All over and under operations, including the previously
available ones, have been modified to
scale properly in subscript  sizes.
(After you have installed \amslatex/, you can process and print the
sample file \fn{testart.tex} to see  examples of the arrows.)

\section{Dots} In the \opt{amstex} option dots should be typed as
\cs{dots}; placement (on the baseline or centered) is selected
according to whatever follows after the \cs{dots}.  If the next thing is
a plus sign, the dots will be centered; if it's a comma, they will be on
the baseline---that is, if you are using the \sty{amsart}
documentstyle.
Dot placement can be changed in other documentstyles if different
conventions are wanted.

If the dots fall at the end of a math formula, the next thing is
something like \verb"\end" or \verb"\)" or \verb"$", which does not give any
information about how to place the dots.  Then you must help by using
\cs{dotsc} for ``dots with commas,'' or \cs{dotsb} for ``dots with
binary operators/relations,'' or \cs{dotsm} for ``multiplication dots,''
or \cs{dotsi} for ``dots with integrals.'' For example, the input
\begin{verbatim}
Then we have the series $A_1,A_2,\dotsc$,
the regional sum $A_1+A_2+\dotsb$,
the orthogonal product $A_1A_2\dotsm$,
and the infinite integral
\[\int_{A_1}\int_{A_2}\dotsi\].
\end{verbatim}
will produce low dots in the first instance and centered dots
in the others, with the spacing on either side of the dots
nicely adjusted.
\begin{quotation}
Then we have the series $A_1,A_2,\ldots\,$,
the regional sum $A_1+A_2+\cdots\,$,
the orthogonal product $A_1A_2\cdots\,$,
and the infinite integral
\[\int_{A_1}\int_{A_2}\cdots\,.\]
\end{quotation}

Specifying dots this way, in terms of their meaning rather than in terms
of their visual placement, is in keeping with the general philosophy of
\latex/ and makes documents more portable between places where different
conventions prevail.  The control sequences \cs{ldots} and \cs{cdots}
are still available, however, for compatibility.

\section{Accents in math}

The following accent commands automatically
give good positioning of double accents:
\begin{verbatim}
 \Hat    \Check  \Tilde  \Acute  \Grave  \Dot    \Ddot
 \Breve  \Bar    \Vec
\end{verbatim}
In ordinary \latex/ the second
accent will usually be askew:
$\hat{\hat A}$ (\verb"\hat{\hat A}").
In the \opt{amstex} option, if you type \verb"\Hat"\5\verb"{\Hat A}"
(using the capitalized form for both accents) the second accent
will be properly positioned (see \fn{testart.tex} for examples).

As explained in the \jt/, this double accent operation is complicated
and tends to slow down the processing of a \tex/ file.
If your document contains many double accents, you can
use \cs{accentedsymbol} in the preamble of your document to
help speed things up.  It stores the result of the double accent
command in a box register, for quick retrieval.  \cs{accented\-symbol}
is used like \cs{newcommand}:
\begin{verbatim}
\accentedsymbol{\Ahathat}{\Hat{\Hat A}}
\end{verbatim}

Some accents have a wide form: typing \verb"$\widehat{xy},\widetilde{xy}$"
produces $\widehat{xy},\widetilde{xy}$.  Because these wide accents
have a certain maximum size, extremely long expressions are better
handled by a different notation:
$(AmBD)^{\widehat{\hphantom{x}}}$ instead of $\widehat{AmBD}$.  But getting
an accent into a superscript is a little tricky (try it),
so \opt{amstex} has the following control sequences
to make it easier:
\begin{verbatim}
 \sphat     \spcheck   \sptilde   \spdot
 \spddot    \spdddot   \spbreve
\end{verbatim}
The example above would be typed \verb"(AmBD)\sphat".

Finally, \cs{dddot} and \cs{ddddot} are available to
produce triple and quadruple dot accents
in addition to the \cs{dot} and \cs{ddot} accents already available
in \latex/.

\section{Roots}

In ordinary \latex/ the placement of root indices is sometimes not so
good: $\sqrt[\beta]{k}$ (\verb"\sqrt"\5\verb"[\beta]{k}").  In the
\opt{amstex} option \cs{leftroot} and \cs{uproot} allow you to adjust
the position of the root:
\verb"\sqrt"\5\verb"[\leftroot{-2}"\5\verb"\uproot{2}"\5\verb"\beta]{k}"
will move the beta up and to the right.
(See the sample file \fn{testart.tex}.) The negative argument
used with \cs{leftroot} moves the $\beta$ to the right. The units are
a small dimension that is a useful size for such
adjustments.

\section{Boxed formulas} The command \cs{boxed} puts a box around its
argument, like \cs{fbox} except that the contents are in math mode.

\section{Extensible arrows}  \verb"@>>>" and \verb"@<<<" produce
arrows that extend automatically to accommodate unusually wide
subscripts or superscripts.  The text of a superscript is typed in
between the first and second \verb+>+ or \verb+<+ symbols, and for a
subscript, it's typed between the second and third symbols. For example,
\verb+@>\xi F_k\Gamma_k\alpha>>+ would have a superscript $\xi
F_k\Gamma_k\alpha$ placed above the arrow.   These arrows were
originally developed for use in commutative diagrams but can be used
elsewhere also.  However, the \opt{amscd} option must be loaded for extensible
arrows to function.  (See section \ref{s:commdiag} for more information about
the \opt{amscd} option.)

\section{{\tt\bslash overset}, {\tt\bslash underset} and
{\tt\bslash sideset}} \latex/ provides \cs{stackrel} for
placing a superscript above a binary relation.  In \opt{amstex}
there are somewhat more general commands, \cs{overset} and
\cs{underset}, that can be used to place one symbol above or
below another symbol, whether it's a relation or something
else.  The input \verb"\overset{*}{X}" will place a
superscript-size $*$ above the  $X$; \cs{underset} performs
the parallel operation that you'd expect.

There's also a command called \cs{sideset}, for a rather special
purpose: putting symbols at the subscript and superscript
corners of a large operator symbol such as $\sum$ or $\prod$.
The prime example is the case when
you want to put a prime on a sum symbol.  If there are no
limits above or below the sum, you could just use \cs{nolimits}:
here's \verb"\sum\nolimits' E_n" in display mode:
\begin{equation}\sum\nolimits' E_n.\end{equation}
But if you want not only the prime but also something below or
above the sum symbol, it's not so easy.  If you have
\begin{equation}\sum_{n<k,\;n\ \rm odd}nE_n\end{equation}
and you want to add a prime
on the sum symbol, use \cs{sideset} like this:
\begin{verbatim}
\sideset{}{'}\sum_{...}nE_n
\end{verbatim}
The extra pair of empty braces is explained by the fact that
\cs{sideset} has the capability of putting an extra symbol
or symbols at each corner of a large operator; to put an asterisk
at each corner of a product symbol, you would type
\begin{verbatim}
\sideset{_*^*}{_*^*}\prod
\end{verbatim}
(After you have installed \amslatex/, you can
typeset and print the sample file \fn{testart.tex} to see
examples of the output.)

\section{The {\tt\bslash text} command}
The main use of the command \cs{text} is for words or phrases in a
display.  It is very similar to the \latex/ command \cs{mbox} in
its effects, but has a couple of advantages.  If you want a word
or phrase of text in a subscript, you can type
\verb"..._{\text{word or phrase}}", which is slightly easier
than the \cs{mbox} equivalent: \verb"..._{\mbox{\scriptsize word
or phrase}}".  The other advantage is the more descriptive
name.

\section{Operator names}\label{s:opname}
Math functions such as $\log$, $\sin$, and $\lim$ are traditionally
typeset in roman type to help avoid confusion with single math
variables, set in math italic.  The more common ones have predefined
names, \cs{log}, \cs{sin}, \cs{lim}, and so forth, but new ones come up
all the time in mathematical papers, so \opt{amstex} provides a general
mechanism for producing such names: \cs{operatorname}\verb"{xxx}"
produces {\rm xxx} in the proper font and automatically adds proper
spacing on either side when necessary, so that you get $A\,{\rm xxx}\,B$
instead of $A{\rm xxx}B$.

Since \cs{operatorname} takes rather a lot of typing, you would usually
put some definitions in the preamble of the form
\begin{verbatim}
\newcommand{\xxx}{\operatorname{xxx}}
\newcommand{\yyy}{\operatorname{yyy}}
\end{verbatim}
for any operator names you're going to use frequently.

Something like \cs{lim} has been defined as an
\cs{operatornamewithlimits} rather than an \cs{operatorname}, because in
displayed formulas if there is a subscript on \cs{lim} it is conventionally
placed underneath, like the limits on sums:
\begin{equation}
C_+f(x)=\lim_{t\to0}C(f)(x+it)
\end{equation}
You can use \cs{operatornamewithlimits} just like \cs{operatorname}; the
only difference is the placement of subscripts and superscripts.
A few special operator names with limits are defined for you in the
\opt{amstex} option:  \cs{varinjlim}, \cs{varprojlim}, \cs{varliminf},
and \cs{varlimsup}; there are some examples in the sample file
\fn{testart.tex}.

\section{{\tt\bslash mod} and its relatives} Commands \cs{mod},
\cs{bmod}, \cs{pmod}, \cs{pod} are provided to deal with the rather
special spacing conventions of ``mod'' notation.  \cs{bmod} and \cs{pmod}
are available in \latex/, but in the \opt{amstex} option the spacing of
\cs{pmod} will adjust to a smaller value if it's used in a
non-display-mode formula.  \cs{mod} and \cs{pod} are variants of
\cs{pmod} preferred by some authors; \cs{mod} omits the parentheses,
whereas \cs{pod} omits the ``mod'' and retains the parentheses.

\section{Fractions and related constructions}
\label{fracs}

In addition to \cs{frac} (which was already available in \latex/),
\opt{amstex} provides \cs{dfrac} and \cs{tfrac} as convenient
abbreviations for \verb"{\displaystyle\frac" \verb"..." \verb"}" and
\verb"{\textstyle\frac" \verb"..." \verb"}". Furthermore, the
thickness of the fraction line can be varied, using
a new square-bracket option of the \cs{frac} command.
\cs{frac}\5\verb"["\<dimension>\verb"]"\5\verb"{...}"\5\verb"{...}"
makes a fraction where the thickness of the horizontal rule is determined by
the given dimension.  The sample file \fn{testart.tex} shows an
example using a thickness of \verb=1.5pt=.

\cs{fracwithdelims}\5\<left delimiter>\5\<right
delimiter>\5\verb"["\<dimension>\verb"]" is an extension of the same idea,
with delimiters on either side specified by the user.%
\setbox0=\hbox{\footnotesize
\verb"\left"\5\verb"(\frac"\5\verb"{...}"\5\verb"{...}"\5\verb"\right)"}%
\footnote{The perceptive reader may wonder why this command is necessary
when you can type things like \unhbox0.
The answer is that \cs{fracwithdelims} provides slightly better
spacing.}

For binomial expressions such as $\bigl({n\atop k}\bigr)$
\opt{amstex} has \cs{binom}, \cs{dbinom} and \cs{tbinom}.  \cs{binom} is
an abbreviation for \cs{fracwithdelims}\verb"()[0pt]".

After you have installed \amslatex/, you can
typeset and print the sample file \fn{testart.tex} to see
examples of \cs{frac} and \cs{binom}.

\section{Continued fractions}
The continued fraction
\begin{equation}
\def\cfrac#1#2{{\displaystyle\strut#1\over\displaystyle#2}%
 \kern-\nulldelimiterspace}
\cfrac{1}{\sqrt{2}+
 \cfrac{1}{\sqrt{2}+
  \cfrac{1}{\sqrt{2}+
   \cfrac{1}{\sqrt{2}+
    \cfrac{1}{\sqrt{2}+\cdots
}}}}}
\end{equation}
can be obtained by typing
{\samepage
\begin{verbatim}
\cfrac{1}{\sqrt{2}+
 \cfrac{1}{\sqrt{2}+
  \cfrac{1}{\sqrt{2}+
   \cfrac{1}{\sqrt{2}+
    \cfrac{1}{\sqrt{2}+\dotsb
}}}}}
\end{verbatim}
}Left or right placement of any of the numerators is accomplished by using
\cs{lcfrac} or \cs{rcfrac} instead of \cs{cfrac}.

\section{Smash options}

The plain \tex/ command \cs{smash} is used to typeset a subformula and
give it an effective height and depth of zero, which is sometimes
useful in adjusting the subformula's position with respect to adjacent
symbols.   In \opt{amstex} there are optional arguments \verb"t" and
\verb"b" for \cs{smash}, because sometimes it is advantageous to be
able to ``smash'' only the top or only the bottom of something while
retaining the natural depth or height.  For example, to smash only the
part of a subformula that extends below the baseline, you would type
\verb"\smash[b]{"\5\<whatever>\5\verb"}".

\section{New  \latex/ environments}
\subsection{The ``cases'' environment}
``Cases'' constructions like the following are common in
mathematics:
\begin{equation} P_{r-j}=
  \left\{
  \begin{array}{ll}
    0                &\mbox{if $r-j$ is odd},\\
    r!\,(-1)^{(r-j)/2} &\mbox{if $r-j$ is even}.
  \end{array}
  \right.
\end{equation}
and in the \opt{amstex} option there is a \env{cases} environment:
\begin{verbatim}
\begin{equation} P_{r-j}=
  \begin{cases}
    0&  \text{if $r-j$ is odd},\\
    r!\,(-1)^{(r-j)/2}&  \text{if $r-j$ is even}.
  \end{cases}
\end{equation}
\end{verbatim}
Notice the use of \cs{text} and the embedded math.

\subsection{Matrix}

In the creation of the \opt{amstex} option, \amstex/'s \cs{matrix}
could have been discarded, since \latex/'s \env{array} environment has
the same function.  But we wanted to keep \amstex/'s \cs{pmatrix},
\cs{bmatrix}, \cs{vmatrix} and \cs{Vmatrix} commands, with delimiters
built in; for consistency, the basic \cs{matrix} has been retained
also.  It and the other matrix commands have been changed into \latex/
environments that work like \env{array}, except that they don't have
an argument specifying the format of the columns. Instead a default
format is provided: up to 10 centered columns. The maximum number of
columns is determined by the counter \verb"MaxMatrixCols", which you
can change if necessary using \latex/'s \cs{setcounter} or
\cs{addtocounter} commands.  I.e., suppose you have a big
matrix with 19 or 20 columns.  Then you'd do something like this:
\begin{verbatim}
\begin{equation}
\setcounter{MaxMatrixCols}{20}
A=\begin{pmatrix}
...&...&...&...&...&...&...&...&...&...&...&...&  ... \\
  ...  \\
  ...
\end{pmatrix}
\end{equation}
\end{verbatim}

To produce a small matrix suitable for use in text, use the
\env{smallmatrix} environment.
\begin{verbatim}
\begin{math}
  \bigl( \begin{smallmatrix}
      a&b\\ c&d
    \end{smallmatrix} \bigr)
\end{math}
\end{verbatim}

\cs{hdotsfor}\verb"{"\<number>\verb"}" produces a row of dots in a matrix
spanning the given number of columns.
\begin{verbatim}
\begin{matrix} a&b&c&d\\
e&\hdotsfor{3} \end{matrix}
\end{verbatim}
would give dots spanning the last three columns in the second row.
The spacing of the dots can be varied through use of a square-bracket
option, for example, \verb"\hdotsfor[1.5]{3}".  The number in square brackets
will be used as a multiplier; the normal value is 1.

\subsection{The {\tt Sb} and {\tt Sp} environments}

The \env{Sb} and \env{Sp} environments can be used to typeset several
lines as a subscript or superscript:
for example
\begin{verbatim}
\begin{equation}
  \sum\begin{Sb}
        0\le i\le m\\ 0<j<n
      \end{Sb}
    P(i,j)
\end{equation}
\end{verbatim}
produces a two-line subscript underneath the sum:
\begin{equation}
  \sum_{0\le i\le m\atop 0<j<n}
    P(i,j)
\end{equation}
\env{Sb} and \env{Sp} can be used anywhere that an ordinary subscript or
superscript can be used.

\subsection{Commutative diagrams}
\label{s:commdiag}

To save memory, the commutative diagram commands of \amstex/ are not
included in the \opt{amstex} option, but are available as a separate
option, \opt{amscd}.  The \env{picture} environment
can be used for complex commutative diagrams but for
simple diagrams without diagonal arrows the \opt{amscd}  commands are more
convenient.

The commutative diagram
\begin{equation}\label{e:cd}
\begin{array}{ccc}
S^{{\cal W}_\Lambda}\otimes T&
   \stackrel{j}{\longrightarrow}&  T\\
\Big\downarrow&
       &\Big\downarrow\vcenter{\rlap{$\scriptstyle{\rm End}\,P$}}\\
(S\otimes T)/I& =&  (Z\otimes T)/J
\end{array}
\end{equation}
can be produced in ordinary \latex/ by
\begin{verbatim}
\begin{array}{ccc}
S^{{\cal W}_\Lambda}\otimes T&
    \stackrel{j}{\longrightarrow}&  T\\
\Big\downarrow&
        &\Big\downarrow\vcenter{%
          \rlap{$\scriptstyle{\rm End}\,P$}}\\
(S\otimes T)/I& =&  (Z\otimes T)/J
\end{array}
\end{verbatim}
When the \opt{amscd} option is used you would type instead
\begin{verbatim}
\begin{CD}
S^{{\cal W}_\Lambda}\otimes T   @>j>>   T\\
@VVV                                    @VV{\End P}V\\
(S\otimes T)/I                  @=      (Z\otimes T)/J
\end{CD}
\end{verbatim}
(with \cs{End} defined as \cs{operatorname}\verb={End}=; see
\S\ref{s:opname}).
This would give longer horizontal arrows than in (\ref{e:cd})
and improved spacing between
elements of the diagram (see \fn{testart.tex}). In the \env{CD} environment
the commands \verb=@>>>=, \verb=@<<<=,
\verb=@VVV=, and \verb=@AAA= give respectively right, left, down, and
up arrows.\index{"@VVV@{\tt{}"@VVV}}%
\index{"@AAA@{\tt{}"@AAA}}%
\index{"@<<<@{\tt{}"@<<<}}%
\index{"@>>>@{\tt{}"@>>>}}%
For the horizontal arrows, material between the first and second
\verb=>= or \verb=<= symbols will be typeset as a superscript,
and material between the second and third will be typeset as
a subscript.  Similarly, material between the first and second
or second and third {\tt A}s or {\tt V}s of vertical arrows will be typeset as
left or right ``sidescripts''.

\section{Alignment structures for equations}
\label{s:displays}

In the \opt{amstex} option
several environments exist for creating multi-line displayed
equations.
They are similar in function to \latex/'s \env{equation} and
\env{eqnarray} environments.
These environments are:
\begin{verbatim}
 align      gather     alignat    xalignat   xxalignat
 multline   split
\end{verbatim}
Each environment, except for \env{split}, has both starred
and unstarred forms, where the unstarred forms have automatic
numbering, using \latex/'s \env{equation} counter.  You can suppress
the number on any particular line by putting \cs{notag} before the
\cs\bslash; you can also override it with a tag of your own using
\cs{tag}\verb"{"\<label>\verb"}", where \<label> means arbitrary text
such as \verb"$*$" or \verb"ii" used to ``number'' the equation.  There
is also a \cs{tag*} command that causes the tag to be typeset
absolutely literally, without putting parentheses around it.  \cs{tag}
and \cs{tag*} can also be used in the starred versions of all the
\opt{amstex} alignment structures.  See \fn{testart.tex}, Appendix~B,
for examples of the use of \cs{tag}.

\subsection{The {\tt align} environment}

The \env{align} environment is used for two or more equations when
vertical alignment is desired (usually binary relations such as equal
signs are aligned).  The term ``equation'' is used rather loosely here
to mean any math formula that is intended by the author as a
self-contained subdivision of the larger display,  usually, but not
always, containing a binary relation.

\subsection{The {\tt gather} environment} Like the \env{align}
environment, the \env{gather} environment is  used for two or more
equations, but when there is no alignment desired among them; each one
is centered separately between the left and right margins.

\subsection{The {\tt alignat} environment}
The \env{alignat} environment is for multiple ``align'' structures
side by side.  There is one required argument, to specify the number
of ``align'' structures.%
%
\footnote{For an argument of $n$, the number of {\tt\char`\&}'s
per line is $2n-1$ (one ampersand for alignment within each
``align'' structure, and ampersands to separate the ``align'' structures
from one another).}
%
The \env{xalignat} and \env{xxalignat} environments
are forms of the \env{alignat} environment with expanded spacing
between the component align structures.  If we consider each
``align'' structure to be a column, \env{xalignat} has equal
spacing between columns and at the margins; \env{xxalignat}
has equal spacing between columns and zero spacing at the
margins.

\subsection{The {\tt multline} environment} The \env{multline}
environment is a variation of the \env{equation} environment used for
equations that don't fit on a single line.  The first line of a
\env{multline} will be at the left margin and the last line at the right
margin, except for an indention on both sides whose amount is equal to
\cs{multlinegap}.  The value of \cs{multlinegap} can be changed using
\latex/'s \cs{setlength} and \cs{addtolength} commands.  If the
\env{multline} contains more than two lines, any lines other than the
first and last will be centered individually between the margins.

\subsection{The {\tt split} environment} The \env{split} environment
is for {\em single} equations that are too long to fit on one line and
hence must be split into multiple lines.  Unlike the other
\opt{amstex} equation structures, it provides no numbering, because it
is used only inside some other displayed equation structure, usually
an \env{equation}, \env{align}, or \env{gather} environment, which
provides the numbering.  Unlike
the \env{multline} environment, the \env{split} environment provides
for alignment among the split lines, using \verb"&" to mark alignment
points, as usual.

\subsection{Alignment environments that don't constitute an entire display}
In addition to the \env{split} environment, there are some other
equation alignment environments that do not constitute an entire
display.  They are self-contained units that
can be used inside of other formulas, or set side-by-side.  The
environment names are \env{aligned}, \env{gathered}, and
\env{alignedat}.  These environments take an optional argument
to specify their vertical positioning with respect to the material
on either side.  The default is {\tt[c]}.  A \env{gathered}
environment with the first line level with the material on either
side would be done like this.
\begin{verbatim}
\begin{gathered}[t]
...\\
...
\end{gathered}
\end{verbatim}


\subsection{Vertical spacing and page breaks  in the {\tt amstex}
equation structures} You can use the
\cs\bslash\verb"["\<dimension>\verb"]" command to get extra vertical
space between lines in all the \opt{amstex} displayed equation
environments, as is usual in \latex/\@.  Unlike \env{eqnarray},  the
\opt{amstex} environments don't allow page breaks between lines, unless
\cs{display\-break} or \cs{allow\-display\-breaks} is used.  The
philosophy is that page breaks in such situations should receive
individual attention from the author.  \cs{display\-break} must go
before the \cs\bslash\ where it is supposed to take effect.  Like
\latex/'s \cs{page\-break}, \cs{displaybreak} takes an optional argument
between 0 and 4 denoting the desirability of the pagebreak.
\verb"\displaybreak[0]" means ``it is permissible to break here''
without encouraging a break; \verb"\displaybreak" with no optional
argument is the same as \verb"\displaybreak[4]" and forces a break.

There is also an optional argument for \cs{allow\-display\-breaks}.
\cs{allow\-display\-breaks} obeys the usual \latex/ scoping rules; the
normal way of limiting its scope would be to put
\verb"{"\cs{allow\-display\-breaks} at the beginning and \verb"}" at the end
of the desired range.  Within the scope of an \cs{allow\-display\-breaks}
command, the \verb"\\*" command can be used to prohibit a pagebreak, as
usual.

\subsection{The {\tt\bslash intertext} command} The command \cs{intertext}
is used for a short interjection of a few lines in the middle of a
display alignment.  Its salient feature is preservation of the
alignment, which would not be possible if you simply ended the display
and then started it up again afterwards. \cs{intertext} may only appear
right after a \verb"\\" or \verb"\\*" command.

\subsection{Equation numbering}

In \latex/ if you wanted to have equations numbered within
sections---that is, have
equation numbers (1.1), (1.2), \dots, (2.1), (2.2),
\dots, in sections 1, 2, and so forth---you would probably redefine
\verb"\theequation":  \begin{verbatim}
\renewcommand{\theequation}{\thesection.\arabic{equation}}
\end{verbatim}
This works fine except that the equation counter won't be reset to
zero at the beginning of a new section or chapter, unless you do it
yourself using \verb"\setcounter".  To make this a little more
convenient, the \opt{amstex} option provides a command
\cs{numberwithin}.  To have equation numbering tied to section
numbering, with automatic reset of the equation counter,
the command would be
\begin{verbatim}
\numberwithin{equation}{section}
\end{verbatim}
As the name implies, \cs{numberwithin} can be applied to other
counters besides the equation counter, but the results are not
guaranteed because of potential complications.  Normal \latex/ methods
should be used where available, e.g., in \cs{newtheorem}.

To make cross-references to equations easier, an \cs{eqref}
command is provided.  This automatically supplies the parentheses
around the equation number, and adds an italic correction if necessary
(see Section~\ref{s:rom}).  To refer to an equation that
was labeled with the label {\tt e:baset}, the usage would be
\verb=\eqref{e:baset}=.

\subsection{Error messages}

One kind of error message in particular should be mentioned, since it
follows from a mistake that is easy to make.

\begin{verbatim}
Runaway argument?
 \left |\frac {\hat v(s)-\hat v(t)}{|\widetilde {D} \ETC.
! Paragraph ended before \multline* was complete.
<to be read again>
                   \par
l.17

? h
I suspect you've forgotten a `}', causing me to apply this
control sequence to too much text. How can we recover?
My plan is to forget the whole thing and hope for the best.

? e
\end{verbatim}

This usually means one of two things: Either you have an
equation alignment environment where the
end doesn't match the beginning---perhaps something
like
\begin{verbatim}
\begin{multline*}
...
\end{multline}
\end{verbatim}
(as in this case)---or else you have a missing \verb=}= or \verb=\right=
delimiter inside of the environment.  A \verb=}= is rather easy
to leave off accidentally when using certain commands,
such as \verb=\frac=.

\section{Miscellaneous}

In the \opt{amstex} option \verb"~"\index{~@{\tt\string~}}, \cs{/}, and
\cs{slash} will remove superfluous spaces on either side of them, as a
convenience to the user (in the case of \cs{/}, only a space on the
left will be removed).  For example, if you have typed
% \tt space was too wide, so used ~ instead:
\verb"p."~\verb"63" and then realize you should add a \verb"~", you can
insert the \verb"~" without deleting the space.

In ordinary \latex/ \cs{big}, \cs{bigg}, \cs{Big}, and \cs{Bigg}
delimiters don't adjust properly over the full range of \latex/ font
sizes.  In the \opt{amstex} option they do.

\section{New documentstyle options available}
\label{s:options}

Several new documentstyle
options have been created.  About half of them
have to do with the positioning of ``limits'' or \cs{tag}s.
The abbreviation of the names reflects the MS-DOS limitation
of eight characters for file names,\footnote{Not including the file
extension.} which we need to allow for.%
\begin{center}
\begin{tabular}{lp{20pc}}
 \opt{nosumlim}&      No limits on sums\\
 \opt{intlim}&        Limits on integrals\\
 \opt{nonamelm}&      No limits on operatornames\\
 \opt{ctagsplt}&      Vertically centered tags
                        on the \env{split} environment\\
 \opt{righttag}&      Equation tags on the right
\end{tabular}
\end{center}

Some of the component parts of the \opt{amstex} option are also
available individually, that is, they can be used in the options list
of the \cs{documentstyle} command:
\begin{center}
\begin{tabular}{lp{20pc}}
\opt{amstext}&  defines \cs{text}\\
\opt{amsbsy}&   defines \cs{boldsymbol} and \cs{pmb}\\
\opt{amsfonts}& defines \cs{frak} and \cs{Bbb}
and  sets up the fonts {\sc msam} (extra math symbols A), {\sc msbm}
(extra math symbols B, and blackboard bold), {\sc eufm} (Euler Fraktur),
as well as extra sizes of {\sc cmmib} (bold math italic and bold
lowerCase Greek), and {\sc cmbsy} (bold math symbols and bold script),
for use in mathematics.
\end{tabular}
\end{center}

\subsection{The {\tt amssymb} option} The \opt{amssymb} option  defines
names for all the symbols in the AMS math symbol fonts.  (Note: The
\opt{amssymb} and \opt{amsfonts} options aren't much use unless you have
version 2.0 of the \amsfonts/ package.)

\subsection{Comments}
\label{s:comment}

A new \opt{verbatim} style option written by Rainer Sch\"opf (and
distributed along with the \amslatex/ package) provides a \env{comment}
environment;
anything you write between \verb"\begin{comment}" and
% Splitting up \end%{comment} here allows this part of the documentation to be
% commented out! mjd
\verb"\end{"\verb"comment}" is totally ignored by \latex/\@.  The
\verb"\end{"\verb"comment}" should be on a line by itself: any text
after \verb"\end{"\verb"comment}" on the same line would be ignored (and
you would receive a warning message that it was lost).  Inside the
\env{comment} environment \latex/ is in a special state that is
ended by the first occurrence of the \verb"\end{"\verb"comment}"
command; you cannot have one \env{comment} environment nested inside
another.  The \opt{verbatim} option provides some
other nice features; see \fn{verbatim.doc} for further details.

\subsection{Syntax checking}

Another new style option is called \opt{syntonly}; if you include this
in your document options list, then you can put \cs{syntaxonly} in the
document preamble to run the file with syntax check only.  No output
will be produced, but any \latex/ errors will be uncovered.  The
advantage of this is that \latex/ will run significantly faster when
\cs{syntaxonly} is in effect.  How much faster depends on the particular
computer being used and other variables but 30\%--40\% is typical.

\section{Protecting fragile commands}

Many of the commands added by the \opt{amstex} option are fragile and
will need to be \cs{protect}ed in commands with ``moving
arguments''---\cs{section} and other sectioning commands, \cs{caption},
\cs{addcontentsline}, \cs{addtocontents}, \cs{markboth}, \cs{markright},
\verb"@"-expressions in an \env{array} or \env{tabular} environment, and
others (see the \latex/ manual, Section~C.1.3).
\sloppypar

\section{Differences the \latex/ user should note}

In \opt{amstex} the \verb"@" character has a special use, in the extensible
arrows \verb"@>>>" and \verb"@<<<" and in the math microspacing commands
\verb"@," and \verb"@!".  In order to get an ordinary printed @
character, type \verb"@@" instead of \verb"@".

With the various alignment environments available in the \opt{amstex}
option, the \env{eqnarray} environment is no longer needed.  Furthermore,
since it does not prevent overlapping of the equation numbers with wide
formulas, as most of the \opt{amstex} alignments do, using the
\opt{amstex} alignments seems better.

\cs{nonumber} is interchangeable with \cs{notag}; the
latter seems slightly preferable, for consistency with the name of \cs{tag}.

In math \cs{bf}, \cs{rm}, and other text font commands
should not be used for single math variables; \cs{bold},
\cs{mathrm}, etc.\ should be used instead.
(See section \ref{s:mathfonts} for details.)

\newpage
\part{The {\tt amsart} and {\tt amsbook} documentstyles}
\label{artbook}

\section{General remarks}

Two considerations controlled the development of the \sty{amsart} and
\sty{amsbook} documentstyles.   First of all, their intended use is
for articles and books submitted for publication to the American
Mathematical Society (in addition to giving the \latex/ user some additional
output styles).  And second, because \sty{amsart} and
\sty{amsbook} not only load the \opt{amstex} option, but also add
several features not found in the standard \latex/ styles, they don't
have much spare memory to work with (if used with a ``small''
implementation of \tex/).

Therefore some features of lesser usefulness found in the standard
\latex/ styles have been omitted or minimized in an effort to conserve
memory. No special provisions have been made for setting up marginal
notes or two-column format, for example. And the 11pt and 12pt options
have been reduced to a minimal kernel: they do nothing except reset
the margins and a few font sizes. More sophisticated adjustments that
are done in the standard \latex/ \sty{article} and \sty{book} styles
are omitted.

The \opt{fleqn} option and the \opt{openbib} bibliography style are
not used in AMS publications, and therefore the necessary
work to make them available has not been done.

No provision is made for fonts in sizes larger than \cs{large}; the
\latex/ commands  \cs{Large}, \cs{LARGE}, \cs{huge}, and \cs{Huge}
still function normally but the size they produce is the same as for
\cs{large}. The design of \sty{amsart} does not use internally
anything larger than \cs{normalsize}.

The \opt{amstex} option, which is part of the \sty{amsart}
and \sty{amsbook} styles, does the necessary setting up to allow the
use of fonts from the \amsfonts/ 2.0 collection, but it is perfectly
possible to use \sty{amsart} and \sty{amsbook} without having \amsfonts/.

\section{The {\tt amsart} documentstyle}

\subsection{Top matter}
We use the term ``top matter'' for the information found at the
beginning of an article, such as the title, author, addresses,
and abstract.
Compared to the standard \sty{article} documentstyle, the
\sty{amsart} documentstyle has a significantly expanded top
matter section.
\latex/'s \sty{article} style provides \cs{title}, \cs{author},
\cs{thanks}, \cs{date}, and an \env{abstract} environment.  The
complete list of top matter commands provided by the \sty{amsart}
style is:

\begin{center}
\begin{tabular}{@{\hspace{3em}}ll}
\cs{title}&   \cs{keywords}\\
\cs{author}&  \cs{subjclass}\\
\cs{address}& \cs{translator}\\
\cs{email}&   \cs{dedicatory}\\
\cs{thanks}&  \cs{date}
\end{tabular}
\end{center}

All of these commands
should precede the \cs{maketitle}
command.  If the\linebreak[2] \env{abstract} environment is used,
it should follow immediately after \cs{maketitle}.
The address, e-mail address,
and translator information prints at the end of the document;
the key words, subject classification, and thanks information
print as footnotes at the bottom of the first page of the document.

An \cs{author} command should be used for each individual author,
when a paper has multiple authors.
Things like \cs{address}, \cs{email}, and
\cs{thanks} that pertain only to one author  should be placed after
the \cs{author} command that they go with (and before any other
\cs{author} commands).  The AMS custom is to list author names in
alphabetical order. (See {\bf Author names and addresses} in section
\ref{sss:variations} for further details.)

In giving an e-mail address remember that \verb=@= characters
should be doubled in order for them to print.

\subsection{Memory conservation measures}
To free up valuable memory, commands that are needed only at the
beginning of a document are undefined when they are no longer needed.
This includes the top matter commands \cs{title}, \cs{author}, etc. and the
\env{abstract} environment.

\subsection{Running heads}

Running heads on odd-numbered pages (right-hand pages) in the
\sty{amsart} style contain the text of the article title, and on
even-numbered pages they contain the author's name.  If the title is
too long to fit within the page width, a shorter version
for the running head text can be
specified with a square-bracket option of the \cs{title}
command:
\begin{verbatim}
\title[Short Version Here]{Long Version of the Title Here,\\
  Perhaps with Multiple Lines}
\end{verbatim}
The \cs{author} command has also been given the same
kind of square-bracket option.

\subsection{Non-English versions of automatically generated text}

If the base language of an article is some language other than
English, the user may wish to change some pieces of text that
are generated automatically.  To change ``Abstract'' to ``R\'esum\'e'',
use \verb=\renewcommand= to redefine \verb=\abstractname=:
\begin{verbatim}
\renewcommand{\abstractname}{R\'esum\'e}
\end{verbatim}
The user can change the following in the same way:%
\footnote{The names of the control sequences
were chosen to match the names used in \fn{babel.sty}.}
\begin{verbatim}
\abstractname   Abstract
\partname       Part
\indexname      Index
\figurename     Figure
\tablename      Table
\proofname      Proof
\refname        References
\appendixname   Appendix
\tocname        Contents
\end{verbatim}
This also allows the user to substitute,
e.g., ``Diagram'' instead of ``Figure'' for the labels of
figure environments.
In the \sty{amsbook} style, there are some additional
names available for changing:
\begin{verbatim}
\chaptername    Chapter
\listfigurename List of Figures
\listtablename  List of Tables
\bibname        Bibliography
\end{verbatim}
(The environment \env{thebibliography} uses \cs{bibname} in the
\sty{amsbook} style, and \cs{refname} in the
\sty{amsart} style.)

\subsection{Theorems, definitions, and similar structures}

\latex/ provides \cs{newtheorem} to create theorem
environments.  The \sty{amsart} and \sty{amsbook} styles make
use of the \opt{theorem} documentstyle option to provide more
flexibility in the design of theorems, definitions, proofs,
remarks, and the like (for full details, see Frank
Mittelbach's article in \tugboat/, vol.~10, no.~3,
November 1989, pp.~416--426).  Three levels of
theorem-type environments are provided through
three \cs{theoremstyle}s: {\tt plain}, {\tt definition},
and {\tt remark}.  The different styles receive different
typographical treatment that gives them visual emphasis
corresponding to their relative importance in the
author's exposition.

To create new theorem-type environments in the
different styles, use the\linebreak[1] \cs{newtheorem} command
in the normal way, but divide your \cs{newtheorem} commands
into groups and preface each group with the appropriate
\cs{theoremstyle}.
If no \cs{theoremstyle} command is given, the style used will
be \fn{plain}.  The\linebreak[1] \cs{theorembodyfont} and
\cs{theoremheaderfont} commands described in Mittelbach's article
are unnecessary in the AMS documentstyles.

Here is an example of a rather comprehensive \cs{newtheorem} section:
\begin{verbatim}
% theorem style plain --- default
\newtheorem{thm}{Theorem}[section]
\newtheorem{lem}{Lemma}[section]
\newtheorem{cor}{Corollary}[section]
\newtheorem{prop}{Proposition}[section]
\newtheorem{crit}{Criterion}[section]
\newtheorem{alg}{Algorithm}[section]

\theoremstyle{definition}

\newtheorem{defn}{Definition}[section]
\newtheorem{conj}{Conjecture}[section]
\newtheorem{exmp}{Example}[section]
\newtheorem{prob}{Problem}[section]

\theoremstyle{remark}

\newtheorem{rem}{Remark}        \renewcommand{\theremark}{}
\newtheorem{note}{Note}         \renewcommand{\thenote}{}
\newtheorem{claim}{Claim}
\newtheorem{summ}{Summary}      \renewcommand{\thesumm}{}
\newtheorem{case}{Case}
\newtheorem{ack}{Acknowledgment}  \renewcommand{\theack}{}
\end{verbatim}
If you would like an unnumbered environment, use \cs{renewcommand}
to undefine\linebreak[1] {\tt\bslash thexxxx} (where {\tt xxxx}
stands for the environment name), as shown in the
``remark'' section.  If you have a theorem with a special name,
such as ``Klein's Theorem,'' use a separate \cs{newtheorem}
command just for that theorem and make it unnumbered:
\begin{verbatim}
\newtheorem{kthm}{Klein's Theorem}
\renewcommand{\thekthm}{}
\end{verbatim}
This will give the normal format for a theorem in all respects
except the automatic numbering.

\subsection{Proofs}

A predefined \env{pf} environment and a starred form \env{pf*}
are provided for
proofs, and produce the heading ``Proof'' with appropriate
spacing and punctuation.  A ``Q.E.D.'' symbol,
$\vcenter{\hrule\hbox{\vrule height.6em\kern.6em\vrule}\hrule}$,
is automatically appended at the end of a proof.
To substitute
a different end-of-proof symbol, use \cs{renewcommand}
to redefine the command \cs{qedsymbol}.
The proof environment is primarily intended for short proofs, less than
a page in length; longer proofs should probably be done as a separate
section or subsection in your document.  For a long
proof that doesn't use the \env{pf} environment,
you can obtain the symbol and the usual amount of preceding space
by using \cs{qed}.

The starred form, \env{pf*}, of the proof environment takes an
argument in curly braces, which allows you to
substitute a different name for the standard ``Proof''.
If you want to substitute, say, ``Proof (sufficiency)'', then write
\begin{verbatim}
\begin{pf*}{Proof (sufficiency)}
\end{verbatim}
and that's all there is to it.

\subsection{Miscellaneous notes}
\subsubsection{Variations from standard \latex/}
\label{sss:variations}

Variations to \latex/ (like \verb=\subjclass= for subject
classification numbers) that are simple additions will not be
specially pointed out. However, a couple of variations that involve
contradictions of statements in the \latex/ manual need to be noted.

\paragraph{Starred forms of sectioning commands}
In the {\tt amsart} and {\tt amsbook} documentstyles starred forms of
the \cs{chapter} and \cs{section} commands produce a table of contents
entry.  This is a variation from standard \latex/
(see the \latex/ manual, \S C.3.1), but more in keeping with
usual publishing practice.

\paragraph{Author names and addresses}
The standard \latex/ format for specifying the names and addresses
 of a document's authors is this:
\begin{verbatim}
\author{First Author\\Address, Line 1\\Address, Line 2
        \and Second Author\\Address, Line 1\\Address, Line 2}
\end{verbatim}
In the \sty{amsart} and \sty{amsbook} documentstyles
there is a separate
\cs{address} command for addresses, and the author names and addresses are
specified individually like all the other elements of the
top matter:
\begin{verbatim}
\author{First Author}
\address{Address, Line 1\\Address, Line 2}
...
\author{Second Author}
\address{Address, Line 1\\Address, Line 2}
\end{verbatim}

Addresses and similar elements will be associated with the nearest
preceding \cs{author} command to determine where they should be
printed.

Author names and addresses typed in the standard \latex/ format
will still print fine, without error messages, but the addresses
may not fall in the proper place (at the end of the document, in
the \sty{amsart} style).

\subsubsection{Numbers and punctuation in italic text}
\label{s:rom}

Mathematical typesetting poses a problem that rarely arises in
nonmathematical typesetting.  In mathematical publishing,
for consistency,  parentheses and other punctuation, as
well as numbers, are always set in  an upright font rather than
varying with the surrounding text.   But then when a math formula with
non\-italic numbers and punctuation occurs in the middle of italicized
text---e.g., in a theorem---with italic numbers and punctuation
nearby, the visual discrepancy rises up to smite the reader
in the eye.
Therefore it is conventional in mathematical publishing to use the
same upright style for numbers and punctuation in italic text as is
used in the mathematics. 

At the present time, italic fonts with upright numbers and
punctuation aren't available.
The work of getting upright numbers and punctuation in italic text
therefore must be done by the author.   In order to
help the author, the \sty{amsart} and \sty{amsbook} documentstyles
take two steps.  First, they do as much as possible automatically,
behind the scenes.  For example, \cs{ref} usually produces a number
of some sort; the definition of \cs{ref} has been changed so
that the number will never print in italic.  Second, they provide
a control sequence \cs{rom}, with one argument, that can be used by the author
when necessary to make an individual punctuation mark or number
nonitalic.  For example:
\begin{verbatim}
... formal \rom{(} in the previous year, \rom{1989)} ...
\end{verbatim}
Italic corrections are inserted automatically by \cs{rom}.

The use of \cs{rom} is unnecessary for punctuation marks that are not
high enough to have a noticeable slant, such as commas and periods.

\section{The {\tt amsbook} documentstyle}

The \sty{amsbook} style has much in common with the \sty{amsart}
style; everything in the previous sections about \sty{amsart}
holds true for \sty{amsbook}, excepting some details
such as the placement of author addresses and other top matter
information.

\subsection{Front matter}
The ``top matter'' information for a book (more commonly called the front matter,
when discussing a book)
is usually specially made up on a title page, with the format
varying widely from book to book.  In the \sty{amsbook} design
\cs{maketitle} produces a simple title page with the title
and author; subject classification numbers, abstract, or
key words, if supplied, will print on the next page.
The style is too plain for actual publication;
it's purpose is to make it convenient for authors to provide
the necessary information to a publisher.

For submissions to the
American Mathematical Society, please provide as a minimum the
following information: title, author, addresses, mathematics
subject classification numbers, translator (if applicable),
and acknowledgments of funding support (\cs{thanks}),

\subsection{Running heads}

Right-hand running heads in the \sty{amsbook} style contain
the text of the current section heading; left-hand running
heads contain the current chapter title.  For special chapters
such as a preface or bibliography that don't have sections,
the right running head will be the same as the left.
Square-bracket options can be used, as normal, to change
the text used for running heads.

\section{Bibliography styles for use with \bibtex/}

The \amslatex/ distribution includes two bibliography styles,
\fn{amsplain} and \fn{amsalpha}, analogous to the standard \latex/
\fn{plain} and \fn{alpha} bibliography styles.  In the AMS
styles an extra field ``language'' is provided, for giving the
original language of a reference, as an indication to the
reader that the title, author name, and so on are translated.
(Because that means locating the reference in a library will
require a little extra care.)

Also included is a file \fn{mrabbrev.bib} containing  standardized
abbreviations used by {\it Mathematical Reviews\/} for journal names
in the mathematical sciences and related fields.
Because the full list is too big to be handled by the current
version of \bibtex/, individual users should use it as a resource,
extracting abbreviations for the journals
that they cite in their particular bibliography database
and adding them to their database.

\newpage
\part{Appendixes}
\appendix

\section{Files included in the \amslatex/ distribution}

The total number of files in the \amslatex/ package, including
documentation and option files, is more than sixty.
A majority of these files are for the
Mittelbach--Sch\"opf font selection scheme and other \latex/
option files written and maintained by Mittelbach and Sch\"opf.  They
are used by various parts of the \amslatex/ package but are not
inherently part of the \amslatex/ distribution; they are included at
the present time because they have not yet become widely available in
the United States.

\subsection{Files maintained by the American Mathematical Society}

\begin{filelist}
\fn{amltinst.tex}&

Instructions on how to install the \amslatex/ package.
\end{filelist}

\begin{filelist}
\fn{amslatex.tex}\newline\fn{amslatex.toc}&

This {\it User's Guide}, describing the \amslatex/ package, and the auxiliary
file for the table of contents.
\end{filelist}

\begin{filelist}
\fn{testart.tex}\newline \fn{test.bib}\newline \fn{testart.bbl}&

A sample file illustrating the use of commands from the \opt{amstex}
option, as well as the \opt{amsart} documentstyle.
\end{filelist}

\begin{filelist}
\fn{testbook.tex}\newline \fn{pref.tex}\newline \fn{chap1.tex}\newline
\fn{chap2.tex}\newline \fn{app.tex}\newline \fn{testbook.bbl}&

These files are sample files illustrating the use of the \opt{amsbook}
documentstyle.\sloppy
\end{filelist}

\begin{filelist}
\fn{amstex.sty}&

The \opt{amstex} documentstyle option. Defines
special \amstex/ structures (like multiline display alignments) with
\latex/ syntax.  It is a copy of \fn{amstex.tex}, version 2.0,
modified as necessary to make it usable from within \latex/.
\end{filelist}

\begin{filelist}
\fn{amstext.sty}\newline \fn{amsbsy.sty}\newline
\fn{amsfonts.sty}\newline \fn{amssymb.sty}&

These are extra option files that can be used apart from the
\opt{amstex} option.
The file \fn{amsbsy.sty} defines the \cs{boldsymbol} command
and the \cs{pmb} command (``poor man's
bold'').  The file
\fn{amstext.sty} defines the \amstex/ \cs{text} command.
The files \fn{amsfonts.sty} and \fn{amssymb.sty} are for
use with the \amsfonts/ package (version 2.0).   \fn{amsfonts.sty}
defines commands, including \cs{newsymbol},
for using fonts in the \amsfonts/ collection,
and \fn{amssymb.sty} defines the names of
all the math symbols available in the \amsfonts/
collection.\sloppy
\end{filelist}

\begin{filelist}
\fn{amscd.sty}& Commutative diagrams\\

&The \opt{amscd}  option
contains extra mathematical commands that can be used as add-ons with the
\opt{amstex} option. (It can also be used without the
\opt{amstex} option.)
\end{filelist}

\begin{filelist}
\fn{intlim.sty}\newline \fn{nonamelm.sty}\newline
\fn{nosumlim.sty}\newline \fn{righttag.sty}\newline
\fn{ctagsplt.sty}&

Extra math style options that affect, for example, left or right
placement of equation numbers.  They are for use only with
the \opt{amstex} option. The \opt{intlim} option provides for
integral subscripts to be placed above and below rather than on the
side. The \opt{nosumlim} option provides for sum subscripts to be
placed on the side rather than above and below. The \opt{nonamelm}
option provides for ``operator name'' subscripts to be placed on the
side rather than above and below. The \opt{righttag} option puts
equation numbers on the right instead of on the left. The
\opt{ctagsplt} option gives equation numbers  vertically centered on
the height of a displayed formula that uses the \env{split}
environment.
\end{filelist}

\begin{filelist}
\fn{amsart.sty}\newline \fn{amsbook.sty}\newline
\fn{amsart.doc}\newline \fn{amsbook.doc}&

Primary documentstyles for submissions to the AMS, for articles and
books respectively, and technical documentation files.
Auxiliary files for 10-point, 11-point and
12-point options are also distributed: \fn{amsart10.sty},
\fn{amsart11.sty}, \fn{amsart12.sty},\newline \fn{amsbk10.sty},
\fn{amsbk11.sty}, \fn{amsbk12.sty},\newline \fn{amsbk10.doc},
\fn{amsart10.doc}.\sloppy\hbadness9999
\end{filelist}

\begin{filelist}
\fn{amsplain.bst}\newline \fn{amsalpha.bst}\newline \fn{mrabbrev.bib}&

Bibliography style files for use with \bibtex/, and a file
containing the {\it Mathematical Reviews\/} standard abbreviations
for the names of mathematical journals.
\end{filelist}

\subsection{Files maintained by Mittelbach and Sch\"opf}

The official copies of the remaining files in this distribution are maintained
by Frank Mittelbach and Rainer Sch\"opf, who have given permission for the
American Mathematical Society to distribute them.

\begin{filelist}
\fn{theorem.sty}\newline
\fn{theorem.doc}\newline and related files&

Option for special treatment of theorems and similar structures,
written by Frank Mittelbach, and auxiliary files; used by \fn{amsart.sty}
and \fn{amsbook.sty}.
\end{filelist}

\begin{filelist}
\fn{verbatim.sty}\newline \fn{verbatim.doc}&
Option file implementing an improved \env{verbatim}
environment and a \env{comment} environment.
\end{filelist}

\begin{filelist}
\fn{lfonts.new}\newline
\fn{preload.min}\newline
\fn{fontdef.max}\newline
\fn{newlfont.sty}\newline
and related files&

The files that implement the Mittelbach--Sch\"opf
font selection scheme.
\end{filelist}


\section{Differences between \amstex/ (version 2.0)
and the {\tt amstex} option}
\label{s:diff}

This section describes the parts of \amstex/ that were removed
during the creation of the \opt{amstex} option.  It will probably
be of interest primarily to users with \amstex/ experience.

In general, \amstex/ commands that were redundant with \latex/
commands were simply dropped.  Other commands were reimplemented
as documentstyle options or otherwise changed in form.

\subsection{Document structure commands}
These commands have all been
superseded by their \latex/ equivalents (some of which
have the same name but function slightly differently):

\begin{center}
\begin{tabular}{|l|l|}
\amstex/&\latex/\\[3pt]
\noalign{\hrule\vskip3pt}
\cs{document}&\verb+\begin{document}+\\
\cs{midspace}&\verb+\beginfigure[htp]...\endfigure+\\
\cs{footnote}&\cs{footnote}\\
\cs{cite}&\cs{cite}\\
\cs{pagewidth}&\cs{textwidth}\\
\cs{pagebreak}&\cs{pagebreak}\\
\end{tabular}
\end{center}

For more information on document structure commands
refer to Part~\ref{artbook}, which
describes the \sty{amsart} and \sty{amsbook} documentstyles.

\subsection{Math font commands}

The names for \amstex/ math font commands couldn't simply be carried over
to \latex/ because there is a conflict with \cs{roman}, which is
preempted by \latex/ for another use.  Therefore, in the \opt{amstex}
option, \amstex/'s \cs{roman} has been renamed \cs{mathrm}.  In
addition, the \cs{italic} and \cs{slanted} math font commands have been
dropped in \opt{amstex}, since their usefulness is in question and memory
space for control sequence names is in short supply.  It appears that
\verb"\text"\5\verb"{\it...}" will serve everywhere that \cs{italic}
might be used, and the same goes for \cs{sl} and \cs{slanted}.

In \amstex/ the text font commands \cs{bf}, \cs{rm}, \cs{sl}, etc.,
cause an error message if used in math mode, but in the \opt{amstex}
option this has been disabled.  This is intended to make it easier for
users who might want to add the \opt{amstex} option to a \latex/ document
that has already been written or partially written.  However, using
these commands in math mode will have no effect on font changes.

In \amstex/ 2.0 there is a command \cs{boldkey} used to obtain bold
versions of math symbols such as \verb+=+ and \verb=+= that are
present on keyboard keys.  In the \opt{amstex} option the use of the
new font selection scheme made it possible to generalize
\cs{boldsymbol} so that \cs{boldkey} is not needed.

\subsection{Matrices}
The \cs{format} option of \amstex/'s \cs{matrix} is not available for
\env{matrix}, \env{pmatrix} and related
environments; just use the \env{array} environment
instead if you need an unusual format for the columns.

\amstex/'s \cs{smallmatrix} command has also been reimplemented
as an environment:
\begin{verbatim}
\begin{math}
  \bigl( \begin{smallmatrix}
      a&b\\ c&d
    \end{smallmatrix} \bigr)
\end{math}
\end{verbatim}

\subsection{Displayed equation structures}

 In the \latex/ \opt{amstex} option, commands for creating
multiple-line displays
have been converted to environments similar to \latex/'s
\env{eqnarray} and \env{equation}---they use
\cs{begin} and \cs{end}, and the \verb"$$" that would have been used
in \amstex/ should not be used.  See Section~\ref{s:displays}
for more details.

\subsection{Math style commands}
As a matter of convenience, \amstex/ provided
the abbreviations \cs{dsize}, \cs{tsize},
\cs{ssize}, and \cs{sssize} for
\cs{display\-style},
\cs{text\-style}, \cs{script\-style}, and \cs{script\-script\-style}.
In order to conserve control sequence names, these have
been dropped in \opt{amstex}, since they are merely synonyms.
If you need to use a math style command frequently because of the nature
of your material, you can add an abbreviation using
\cs{newcommand} in the preamble of your document,
and call it whatever you choose:
\begin{verbatim}
\newcommand{\sst}{\scriptstyle}
\end{verbatim}

\subsection{{\tt\bslash thickfrac}}

The \cs{thickfrac} and \cs{thickfracwithdelims} commands of
\amstex/ have been replaced by square-bracket options on
the \cs{frac} and \cs{fracwithdelims} commands.  See
Section~\ref{fracs}.

\subsection{Commenting out a  large section of text}

The \cs{comment} command of \amstex/ is replaced by the
\env{comment} environment of the \opt{verbatim} documentstyle
option.  See the description in Section~\ref{s:comment}.

\subsection{Page breaks inside a display}

In the \opt{amstex} option, \cs{displaybreak} should
{\em precede\/} the \cs\bslash\ where it is supposed to take effect.
In the original \amstex/ it follows
immediately after the \cs\bslash.

\subsection{Special colons in math}
\latex/ and \amstex/ have different definitions for the \cs{:} command.
In \latex/ it is a medium math space, whereas in \amstex/ it is a colon
with spacing appropriate in certain notation for mappings:
$S\,{:}\;s\to s^t$ (\verb"$S\:s\to" \5\verb"s^t$"). The \latex/ version
is the one that has been retained, in order to avoid compatibility
problems.  \verb"\colon" is available as a substitute for the \amstex/
\verb"\:".

\subsection{Paragraphed text within a displayed equation}

\amstex/ has a \cs{foldedtext} command  for handling
a piece of text within a display that needs to be typeset
as a paragraph (perhaps to keep it from running over the right
margin).  In the \opt{amstex} option this was dropped because
it's redundant; \latex/'s \cs{parbox} command
can be used instead.

\subsection{Commutative diagrams}

In order to conserve memory, commutative diagram commands are a separate
option, \opt{amscd}, that must be loaded in the documentstyle options list
if it is desired.

The \cs{pretend}\dots\cs{haswidth} command is not available in the
\opt{amscd} option. Approximately the same results can be gotten by
inserting blank space using \cs{hspace} in the subscript or
superscript fields of the extensible arrow commands (\verb=@>>>= and
\verb=@<<<=).\index{"@<<<@{\tt{}"@<<<}}\index{"@>>>@{\tt{}"@>>>}}

\subsection{Footnotes}

The \cs{footnote} command of the \sty{amsppt} documentstyle
is superseded by \latex/'s command of
the same name.  Instead of the \amstex/ \cs{adjustfootnotemark}
command use \cs{addtocounter}\verb"{footnote}{"\<number>\verb"}".  The
literal footnote mark feature of \sty{amsppt} is not available.

\subsection{Vertical spacing}

The \amstex/ use of \cs{vspace} in alignment structures
\cs{align}, \cs{split}, etc. is superfluous in \latex/
because the same function is available
through the optional
argument of the \verb"\\" or \verb"\\*" commands.  Therefore
\amstex/'s version of \cs{vspace} has been dropped.

\subsection{Blank space for figures}

\amstex/'s \cs{midspace}, \cs{topspace}, \cs{caption}
and\index{figures}\index{floats}\index{floating environments}
\cs{captionwidth} are superfluous in the \opt{amstex} option and have
been dropped; use \latex/'s \env{figure} environment and \cs{caption}
command.

\subsection{{\tt\bslash hdotsfor}}

The original \amstex/ syntax of \cs{hdotsfor} has been simplified
somewhat in the \opt{amstex} option; \amstex/'s \cs{innerhdotsfor} is
not needed.  The spacing between
dots is adjusted via a square-bracket
option rather than through a separate command \cs{spacehdotsfor} or
\cs{spaceinnerhdotsfor}.

\subsection{{\tt\bslash topsmash} and {\tt\bslash botsmash}}
These have been changed in the \opt{amstex} option to square-bracket
options of the \cs{smash} command.

\subsection{{\tt\bslash spreadlines} and other display options}

Some of the \amstex/ options used inside displays, 
such as \cs{spreadlines} and \cs{nopagebreak}, have been dropped.  For the
most part their effect can be obtained by other means available in
standard \latex/.

\subsection{The {\tt\bslash and} command}
There is a name conflict between \amstex/'s \cs{and} and
\latex/'s \cs{and}.  The function of \amstex/'s \cs{and}
can be obtained in the \opt{amstex} option by using
\cs{And}.

\subsection{Global options}
There are several \amstex/ commands that change the global setting of
certain aspects of the document style.  In the \opt{amstex} option we've
done the natural thing, which is to make them into \latex/
documentstyle options (see Section~\ref{s:options}).  These commands are
\cs{Tags\-On\-Right}, \cs{Centered\-Tags\-On\-Splits}, \cs{Limits\-On\-Ints},
\cs{No\-Limits\-On\-Names}, and \cs{No\-Limits\-On\-Sums}.  The corresponding
opposites of these commands have been dropped because they
describe the default conditions in the \opt{amstex} option.
Because it seems to be of only marginal usefulness,
\cs{TagsAsMath} has been dropped.

\section{Memory statistics}
\label{memstats}

Combining all of \amstex/ with all of \latex/, even after eliminating
redundancies, produces a large macro package that strains the current
limits of personal computer versions of \tex/\@.  After the
zealous application of efficiency measures, the current version of
\amslatex/ is probably more compact than anyone anticipated;
nevertheless, for some documents and some implementations of \tex/
it will still be too big to run.  Among other things, a
large number of bibliography items, cross-references, or personal
definitions will tend to cause an overrun in a particular area of
\tex/'s memory: the maximum limit on the number of control sequence
names.   Also, you are more likely to run out of main memory if your
document includes a large table or Pic\tex/ diagram.

\begin{table}[htp]
\caption{Memory statistics}
\label{memtable}
\medskip
\footnotesize
\setlength{\tabcolsep}{.2\tabcolsep}
\noindent
\begin{tabular*}{\columnwidth}{|r|c*{6}{@{\extracolsep{\fill}}c}|}
\hline
&{\tt article}$^*$&
{\tt\begin{tabular}[b]{c}[M--S\\fonts]\\article\end{tabular}}&
{\tt\begin{tabular}[b]{c}[amstex]$^{\dag}$\\article\end{tabular}}&
{\tt\begin{tabular}[b]{c}[amstex,\\amscd,\\amssymb]\\article\end{tabular}}&
{\tt\begin{tabular}[b]{c}[amsfonts,\\amsbsy]\\article\end{tabular}}&
{\tt\begin{tabular}[b]{c}[amstex]$^{\dag}$\\amsart\end{tabular}}&
\begin{tabular}[b]{c}represen-\\tative\\maxima\end{tabular}\\
\hline
strings& 286& 386& 942& 1170& 500& 933& 7032\\ \hline
\begin{tabular}{r}string\\charac-\\ters\end{tabular}&
        2421& 3183& 8444& 10505& 4218& 8419& 20798\\ \hline
\begin{tabular}{r}main\\memory\end{tabular}&
        51376& 50126& 63506& 63880& 51059& 65445& 65501\\ \hline
\begin{tabular}{r}control\\sequences\end{tabular}&
        2234& 2410& 2938& 3160& 2498& 2946& 5000\\ \hline
\begin{tabular}{r}font\\infor-\\mation\end{tabular}&
        18941& 12721& 16842& 16842& 16842& 11462& 65504\\ \hline
\begin{tabular}{r}number\\of fonts\end{tabular}&
        72& 50& 70& 70& 70& 47& 220\\ \hline
\begin{tabular}{r}hyphen-\\ation\\excep-\\tions\end{tabular}&
        14& 14& 14& 14& 14& 14& 307\\ \hline
\begin{tabular}{r}input\\stack\end{tabular}&
        12& 16& 16& 16& 16& 19& 200\\ \hline
\begin{tabular}{r}nesting\\levels\end{tabular}&
        9& 9& 14& 14& 9& 14& 40\\ \hline
\begin{tabular}{r}param-\\eter\\stack\end{tabular}&
        25& 25& 25& 25& 25& 25& 60\\ \hline
\begin{tabular}{r}input\\buffer\end{tabular}&
        361& 361& 436& 436& 400& 436& 1500\\ \hline
\begin{tabular}{r}save\\stack\end{tabular}&
        233& 173& 192& 192& 187& 264& 2000\\
\hline
\end{tabular*}%
\par
\smallskip

\parbox{\columnwidth}{$^*$This column is the control,
using standard \latex/ with the standard \sty{article} documentstyle,
without the Mittelbach--Sch\"opf font selection scheme.  The
adjacent column is the same, except for the font selection
scheme.

$^{\dag}$Recall that
the \opt{amstex} option includes the
\opt{amsfonts}, \opt{amsbsy}, and \opt{amstext} options, and
that the \sty{amsart} and \sty{amsbook}
documentstyles automatically include the \opt{amstex} option.
}% end of parbox
\end{table}

For those who might be interested in the details, Table~\ref{memtable}
shows memory statistics
from \latex/ runs using various combinations of option files
from the \amslatex/ distribution.  For comparison purposes,
the statistics in the first column are from
a sample run using the current standard \latex/ without
the Mittelbach--Sch\"opf font scheme, with the
\sty{article} documentstyle, and the last column, ``representative
maxima'', shows the available memory in each category in the
implementation of \tex/ used for testing (VAX/VMS Version 2.98a.0 (AMS)).
The test document used in each case
was a medium-size article with about 20 bibliography
entries, 50 author-defined commands, and 50 cross-reference
labels.

It can be seen that in all of the tests with the \opt{amstex} option
loaded the upper limit of 65536 words of main memory is nearly
exceeded.  And use of the \opt{amssymb} option in the fourth test
would cause control sequence memory to be exceeded for implementations
of \tex/ that have a maximum of 3000 (currently true for
many implementations on small computer systems).

The font base used in all the tests, except for the control, was
\fn{fontdef.max}, \fn{preload.min}, and \fn{newlfont.sty}.  Obviously,
preloading more fonts by using a different preload file would
tend to increase font memory usage.  The reason for the comparatively
small number of fonts used by the \sty{amsart} documentstyle is that
no fonts larger than \cs{normalsize} are used in the title and
section headings.

Memory statistics for the \sty{amsbook} documentstyle are comparable
to the statistics for \sty{amsart}.  But bear in mind that books
will usually have larger bibliographies and more cross-references,
which means greater usage of control sequence memory.

\newpage
\section{Getting help}

Comments or questions on the \amslatex/ package should be sent to:
\medskip

\begin{raggedright}

\leftskip=4.25pc
American Mathematical Society\\
Technical Support\\
P. O. Box 6248\\
Providence, RI 02940\\[3pt]
Phone: 800-321-4AMS (321-4267) \quad or \quad 401-455-4080\\
Internet: {\tt tech-support@Math.AMS.com}\\
\end{raggedright}
\medskip

\noindent If you are reporting a problem you should include
the following information:
\begin{enumerate}
\item the source file---either in electronic form or printed---where
  the problem occurred, preferably with irrelevant
  material removed.
\item a \latex/ transcript (log) file showing the error
 message (if applicable) and the version numbers of
 the documentstyle and option files being used.
\end{enumerate}

\medskip

If you wish to obtain the article by Mittelbach and Sch\"opf that
appeared in \tugboat/, June 1990 (vol.~11, no.~2): {\it The new font
family selection---user interface to standard \latex/\/}, contact:

\medskip
\begin{raggedright}

\leftskip=4.25pc
\tex/ User's Group\\
P. O. Box 9506\\
Providence, RI 02940\\[3pt]
Phone: (401) 751-7760\\
Internet: {\tt TUG@Math.AMS.com}\\
\end{raggedright}


\begin{thebibliography}{9}

\bibitem{amsfonts} {\it AMSFonts version 2.0---user's guide},
American Mathematical Society, Providence, R.I., 1990; distributed
with the \amsfonts/ package.

\bibitem{lm} Leslie Lamport, {\it\latex/: A document preparation
system}, Addison-Wesley, 1985.

\bibitem{msf} Frank Mittelbach and Rainer Sch\"opf,
{\it The new font family
selection---user interface to standard \latex/},
\tugboat/ {\bf11}, no.~2 (June 1990), pp.~297--305.

\bibitem{jt} Michael Spivak, {\it The joy of \tex/}, 2nd ed.,
American Mathematical Society, Providence, R.I., 1990.

\end{thebibliography}

%%%%%%%%%%%INDEX
%% The following index is the output file from running the program MakeIndex
%% on the .idx file produced by running amslatex.tex through LaTeX. It is
%% included here for the benefit of users who do
%% not have use of the program MakeIndex.
%%%%%%%%%%%%
\begin{theindex}

\small

  \item {\tt\bslash !}, 15
  \item {\ptt \bslash /}, 27
  \item {\ptt \bslash :}, 15, 41
  \item {\ptt \bslash ;}, 15
  \item {\tt{}@!}, 15
  \item {\tt{}@,}, 15
  \item {\tt{}@<<<}, 23, 42
  \item {\tt{}@>>>}, 23, 42
  \item {\tt{}@AAA}, 23
  \item {\tt{}@VVV}, 23
  \item {\ptt \bslash \bslash }, 24, 25, 41
  \item {\tt\string~}, 27

  \indexspace

  \item {\ptt {}abstract} environment, 31
  \item {\ptt \bslash accentedsymbol}, 17
  \item {\ptt \bslash addcontentsline}, 28
  \item {\ptt \bslash address}, 31, 34
  \item {\ptt \bslash addtocontents}, 28
  \item {\ptt \bslash addtocounter}, 21, 42
  \item {\ptt \bslash addtolength}, 24
  \item {\ptt \bslash adjustfootnotemark}, 42
  \item {\ptt \bslash align}, 42
  \item {\ptt {}align} environment, 24, 25
  \item {\ptt {}alignat} environment, 24
  \item {\ptt {}aligned} environment, 25
  \item {\ptt {}alignedat} environment, 25
  \item {\ptt \bslash allowdisplaybreaks}, 25
  \item {\ptt {}alpha}, 36
  \item {\ptt {}amltinst.tex}, 37
  \item {\ptt {}amsalpha}, 36
  \item {\ptt {}amsalpha.bst}, 38
  \item {\ptt {}amsart} documentstyle, 16, 30--32, 34, 35, 39, 44, 45
  \item {\ptt {}amsart} option, 37
  \item {\ptt {}amsart.doc}, 38
  \item {\ptt {}amsart.sty}, 38, 39
  \item {\ptt {}amsart10.doc}, 38
  \item {\ptt {}amsart10.sty}, 38
  \item {\ptt {}amsart11.sty}, 38
  \item {\ptt {}amsart12.sty}, 38
  \item {\ptt {}amsbk10.doc}, 38
  \item {\ptt {}amsbk10.sty}, 38
  \item {\ptt {}amsbk11.sty}, 38
  \item {\ptt {}amsbk12.sty}, 38
  \item {\ptt {}amsbook} documentstyle, 3, 30, 32, 34--36, 39, 44, 45
  \item {\ptt {}amsbook} option, 37
  \item {\ptt {}amsbook.doc}, 38
  \item {\ptt {}amsbook.sty}, 38, 39
  \item {\ptt {}amsbsy} option, 28, 44
  \item {\ptt {}amsbsy.sty}, 38
  \item {\ptt {}amscd} option, 18, 22, 23, 38, 42
  \item {\ptt {}amscd.sty}, 38
  \item {\ptt {}amsfonts} option, 14, 28, 44
  \item {\ptt {}amsfonts.sty}, 38
  \item {\ptt {}amslatex.tex}, 1, 37
  \item {\ptt {}amslatex.toc}, 37
  \item {\ptt {}amsplain}, 36
  \item {\ptt {}amsplain.bst}, 38
  \item {\ptt {}amsppt} documentstyle, 3, 42
  \item {\ptt {}amssymb} option, i, 14, 28, 45
  \item {\ptt {}amssymb.sty}, 38
  \item {\ptt {}amstex} option, 1--3, 10, 13--30, 38--45
  \item {\ptt {}amstex.sty}, 37
  \item {\ptt {}amstex.tex}, 37
  \item {\ptt {}amstext} option, 28, 44
  \item {\ptt {}amstext.sty}, 38
  \item {\ptt \bslash And}, 43
  \item {\ptt \bslash and}, 43
  \item {\ptt {}app.tex}, 37
  \item {\ptt {}array} environment, 21, 28, 40
  \item {\ptt {}article} documentstyle, 3, 30, 43, 44
  \item {\ptt \bslash author}, 30, 31, 35

  \indexspace

  \item {\ptt {}babel.sty}, 32
  \item {\ptt \bslash Bbb}, 10, 11, 13, 28
  \item {\ptt \bslash begin}, 2, 40
  \item {\ptt \bslash bf}, 4, 5, 11, 29, 40
  \item {\ptt \bslash bfdefault}, 8
  \item {\ptt \bslash bibname}, 32
  \item {\ptt \bslash Big}, 27
  \item {\ptt \bslash big}, 27
  \item {\ptt \bslash Bigg}, 27
  \item {\ptt \bslash bigg}, 27
  \item {\ptt \bslash binom}, 20
  \item {\ptt \bslash bmatrix}, 21
  \item {\ptt \bslash bmod}, 19
  \item {\ptt \bslash bold}, 9--11, 13, 29
  \item {\ptt \bslash boldkey}, 40
  \item {\ptt \bslash boldmath}, 10
  \item {\ptt \bslash boldsymbol}, 10, 11, 13, 28, 38, 40
  \item {\ptt {}book} documentstyle, 3, 30
  \item {\ptt \bslash boxed}, 17

  \indexspace

  \item {\ptt \bslash cal}, 10, 11
  \item {\ptt \bslash caption}, 28, 42
  \item {\ptt \bslash captionwidth}, 42
  \item {\ptt {}cases} environment, 21
  \item {\ptt {}CD} environment, 23
  \item {\ptt \bslash cdots}, 16
  \item {\ptt \bslash CenteredTagsOnSplits}, 43
  \item {\ptt \bslash cfrac}, 20
  \item {\ptt {}chap1.tex}, 37
  \item {\ptt {}chap2.tex}, 37
  \item {\ptt \bslash chapter}, 34
  \item {\ptt \bslash cite}, 39
  \item {\ptt \bslash comment}, 41
  \item {\ptt {}comment} environment, 28, 39, 41
  \item {\ptt {}ctagsplt} option, 27, 38
  \item {\ptt {}ctagsplt.sty}, 38
  \item {\ptt \bslash cy}, 10

  \indexspace

  \item {\ptt \bslash date}, 31
  \item {\ptt \bslash dbinom}, 20
  \item {\ptt \bslash ddddot}, 17
  \item {\ptt \bslash dddot}, 17
  \item {\ptt \bslash ddot}, 17
  \item {\ptt \bslash dedicatory}, 31
  \item {\ptt \bslash dfrac}, 19
  \item {\ptt \bslash displaybreak}, 25, 41
  \item {\ptt \bslash displaystyle}, 41
  \item {\ptt \bslash document}, 39
  \item {\ptt \bslash documentstyle}, 3, 27
  \item {\ptt \bslash dot}, 17
  \item {\ptt \bslash dots}, 15
  \item {\ptt \bslash dotsb}, 16
  \item {\ptt \bslash dotsc}, 16
  \item {\ptt \bslash dotsi}, 16
  \item {\ptt \bslash dotsm}, 16
  \item {\ptt \bslash dsize}, 41

  \indexspace

  \item {\ptt \bslash em}, 11
  \item {\ptt \bslash email}, 31
  \item {\ptt \bslash End}, 23
  \item {\ptt \bslash end}, 2, 40
  \item {\ptt \bslash endmatrix}, 2
  \item {\ptt \bslash endsomething}, 2
  \item {\ptt {}eqnarray} environment, 23, 25, 29, 40
  \item {\ptt \bslash eqref}, 26
  \item {\ptt {}equation} environment, 23--25, 40

  \indexspace

  \item {\ptt \bslash family}, 4, 7, 8
  \item {\ptt \bslash fbox}, 17
  \item {\ptt {}figure} environment, 42
  \item figures, 42
  \item {\ptt {}fleqn} option, 30
  \item floating environments, 42
  \item floats, 42
  \item {\ptt \bslash foldedtext}, 41
  \item {\ptt {}fontdef.max}, 8, 10, 39, 45
  \item {\ptt \bslash footnote}, 9, 39, 42
  \item {\ptt \bslash format}, 40
  \item {\ptt \bslash frac}, 19, 20, 41
  \item {\ptt \bslash fracwithdelims}, 20, 41
  \item {\ptt \bslash frak}, 10, 11, 13, 28

  \indexspace

  \item {\ptt {}gather} environment, 24, 25
  \item {\ptt {}gathered} environment, 25

  \indexspace

  \item {\ptt \bslash haswidth}, 42
  \item {\ptt \bslash hdotsfor}, 22, 42
  \item {\ptt \bslash hspace}, 42
  \item {\ptt \bslash Huge}, 6, 30
  \item {\ptt \bslash huge}, 30

  \indexspace

  \item {\ptt \bslash idotsint}, 15
  \item {\ptt \bslash iiiint}, 15
  \item {\ptt \bslash iiint}, 15
  \item {\ptt \bslash iint}, 15
  \item {\ptt \bslash innerhdotsfor}, 42
  \item {\ptt \bslash intertext}, 26
  \item {\ptt {}intlim} option, 27, 38
  \item {\ptt {}intlim.sty}, 38
  \item {\ptt \bslash it}, 4, 5, 11
  \item {\ptt \bslash italic}, 40
  \item {\ptt \bslash itdefault}, 8

  \indexspace

  \item {\ptt \bslash keywords}, 31

  \indexspace

  \item {\ptt \bslash LARGE}, 30
  \item {\ptt \bslash Large}, 30
  \item {\ptt \bslash large}, 30
  \item {\ptt \bslash lcfrac}, 20
  \item {\ptt \bslash ldots}, 16
  \item {\ptt \bslash leftroot}, 17
  \item {\ptt {}lfonts.new}, 39
  \item {\ptt \bslash lim}, 19
  \item {\ptt \bslash LimitsOnInts}, 43
  \item {\ptt \bslash loadmsam}, 14
  \item {\ptt \bslash loadmsbm}, 14
  \item {\ptt \bslash log}, 19

  \indexspace

  \item {\ptt \bslash maketitle}, 31, 36
  \item {\ptt \bslash markboth}, 28
  \item {\ptt \bslash markright}, 28
  \item {\ptt \bslash mathrm}, 10, 11, 29, 40
  \item {\ptt \bslash matrix}, 2, 21, 40
  \item {\ptt {}matrix} environment, 40
  \item {\ptt \bslash mbox}, 18
  \item {\ptt \bslash mediumseries}, 5, 11
  \item {\ptt \bslash medspace}, 15
  \item {\ptt \bslash midspace}, 39, 42
  \item {\ptt \bslash mit}, 10, 11
  \item {\ptt \bslash mod}, 19
  \item {\ptt {}mrabbrev.bib}, 36, 38
  \item {\ptt {}multline} environment, 24, 25
  \item {\ptt \bslash multlinegap}, 24

  \indexspace

  \item {\ptt \bslash negmedspace}, 15
  \item {\ptt \bslash negthickspace}, 15
  \item {\ptt \bslash negthinspace}, 15
  \item {\ptt \bslash newcommand}, 14, 17, 41
  \item {\ptt {}newlfont.sty}, 39, 45
  \item {\ptt \bslash newmathalphabet}, 10
  \item {\ptt \bslash newsymbol}, i, 13, 14, 38
  \item {\ptt \bslash newtheorem}, 26, 32, 33
  \item {\ptt \bslash nolimits}, 18
  \item {\ptt \bslash NoLimitsOnNames}, 43
  \item {\ptt \bslash NoLimitsOnSums}, 43
  \item {\ptt {}nonamelm} option, 27, 38
  \item {\ptt {}nonamelm.sty}, 38
  \item {\ptt \bslash nonumber}, 29
  \item {\ptt \bslash nopagebreak}, 43
  \item {\ptt \bslash normalshape}, 5, 9, 11
  \item {\ptt \bslash normalsize}, 30, 45
  \item {\ptt {}nosumlim} option, 27, 38
  \item {\ptt {}nosumlim.sty}, 38
  \item {\ptt \bslash notag}, 24, 29
  \item {\ptt \bslash numberwithin}, 26

  \indexspace

  \item {\ptt {}oldlfont} option, 8
  \item {\ptt {}openbib} option, 30
  \item {\ptt \bslash operatorname}, 11, 19, 23
  \item {\ptt \bslash operatornamewithlimits}, 19
  \item {\ptt \bslash overleftrightarrow}, 15
  \item {\ptt \bslash overset}, 18

  \indexspace

  \item {\ptt \bslash pagebreak}, 25, 39
  \item {\ptt \bslash pagewidth}, 39
  \item {\ptt \bslash parbox}, 41
  \item {\ptt {}pf} environment, 33, 34
  \item {\ptt {}pf*} environment, 33, 34
  \item {\ptt {}picture} environment, 22
  \item {\ptt {}plain}, 33, 36
  \item {\ptt \bslash pmatrix}, 21
  \item {\ptt {}pmatrix} environment, 40
  \item {\ptt \bslash pmb}, 11, 13, 28, 38
  \item {\ptt \bslash pmod}, 19
  \item {\ptt \bslash pod}, 19
  \item {\ptt {}pref.tex}, 37
  \item {\ptt {}preload.min}, 39, 45
  \item {\ptt \bslash pretend}, 42
  \item {\ptt \bslash protect}, 28

  \indexspace

  \item {\ptt \bslash qed}, 34
  \item {\ptt \bslash qedsymbol}, 34

  \indexspace

  \item {\ptt \bslash rcfrac}, 20
  \item {\ptt \bslash ref}, 35
  \item {\ptt \bslash refname}, 32
  \item {\ptt \bslash renewcommand}, 33, 34
  \item {\ptt {}righttag} option, 27, 38
  \item {\ptt {}righttag.sty}, 38
  \item {\ptt \bslash rm}, 4, 5, 7, 8, 11, 13, 29, 40
  \item {\ptt \bslash rom}, 35
  \item {\ptt \bslash roman}, 40

  \indexspace

  \item {\ptt {}Sb} environment, 22
  \item {\ptt \bslash sc}, 5, 11
  \item {\ptt \bslash scdefault}, 8
  \item {\ptt \bslash scriptscriptstyle}, 41
  \item {\ptt \bslash scriptstyle}, 41
  \item {\ptt \bslash section}, 28, 34
  \item {\ptt \bslash selectfont}, 4, 5
  \item {\ptt \bslash series}, 4, 5, 11
  \item {\ptt \bslash setcounter}, 21
  \item {\ptt \bslash setlength}, 24
  \item {\ptt \bslash sf}, 8, 11
  \item {\ptt \bslash shape}, 4, 11
  \item {\ptt \bslash sideset}, 18
  \item {\ptt \bslash sin}, 19
  \item {\ptt \bslash size}, 4, 6
  \item {\ptt \bslash sl}, 5, 11, 40
  \item {\ptt \bslash slanted}, 40
  \item {\ptt \bslash slash}, 27
  \item {\ptt \bslash sldefault}, 8
  \item {\ptt \bslash smallmatrix}, 40
  \item {\ptt {}smallmatrix} environment, 21
  \item {\ptt \bslash smash}, 20, 42
  \item {\ptt \bslash something}, 2
  \item {\ptt {}Sp} environment, 22
  \item {\ptt \bslash spacehdotsfor}, 42
  \item {\ptt \bslash spaceinnerhdotsfor}, 42
  \item {\ptt \bslash split}, 42
  \item {\ptt {}split} environment, 23, 25, 27, 38
  \item {\ptt \bslash spreadlines}, 43
  \item {\ptt \bslash ssize}, 41
  \item {\ptt \bslash sssize}, 41
  \item {\ptt \bslash stackrel}, 18
  \item {\ptt \bslash subjclass}, 31
  \item {\ptt \bslash syntaxonly}, 28
  \item {\ptt {}syntonly} option, 28

  \indexspace

  \item {\ptt {}tabular} environment, 28
  \item {\ptt \bslash tag}, 24, 27, 29
  \item {\ptt \bslash tag*}, 24
  \item {\ptt \bslash TagsAsMath}, 43
  \item {\ptt \bslash TagsOnRight}, 43
  \item {\ptt \bslash tbinom}, 20
  \item {\ptt \bslash tenbf}, 9
  \item {\ptt {}test.bib}, 37
  \item {\ptt {}testart.bbl}, 37
  \item {\ptt {}testart.tex}, 1, 15--20, 23, 24, 37
  \item {\ptt {}testbook.bbl}, 37
  \item {\ptt {}testbook.tex}, 1, 37
  \item {\ptt \bslash text}, 2, 11, 18, 21, 28, 38
  \item {\ptt \bslash textstyle}, 41
  \item {\ptt \bslash textwidth}, 39
  \item {\ptt \bslash tfrac}, 19
  \item {\ptt \bslash thanks}, 31, 36
  \item {\ptt {}thebibliography} environment, 32
  \item {\ptt {}theorem} option, 32
  \item {\ptt {}theorem.doc}, 39
  \item {\ptt {}theorem.sty}, 39
  \item {\ptt \bslash theorembodyfont}, 33
  \item {\ptt \bslash theoremheaderfont}, 33
  \item {\ptt \bslash theoremstyle}, 32, 33
  \item {\ptt \bslash thickfrac}, 41
  \item {\ptt \bslash thickfracwithdelims}, 41
  \item {\ptt \bslash thickspace}, 15
  \item {\ptt \bslash thinspace}, 15
  \item {\ptt \bslash tiny}, 6
  \item {\ptt \bslash title}, 30, 31
  \item {\ptt \bslash topspace}, 42
  \item {\ptt \bslash translator}, 31
  \item {\ptt \bslash tsize}, 41
  \item {\ptt \bslash tt}, 4, 7, 8, 11

  \indexspace

  \item {\ptt \bslash underleftarrow}, 15
  \item {\ptt \bslash underleftrightarrow}, 15
  \item {\ptt \bslash underrightarrow}, 15
  \item {\ptt \bslash underset}, 18
  \item {\ptt \bslash uproot}, 17

  \indexspace

  \item {\ptt \bslash varinjlim}, 19
  \item {\ptt \bslash varliminf}, 19
  \item {\ptt \bslash varlimsup}, 19
  \item {\ptt \bslash varprojlim}, 19
  \item {\ptt {}verbatim} environment, 39
  \item {\ptt {}verbatim} option, 28, 41
  \item {\ptt {}verbatim.doc}, 28, 39
  \item {\ptt {}verbatim.sty}, 39
  \item {\ptt \bslash Vmatrix}, 21
  \item {\ptt \bslash vmatrix}, 21
  \item {\ptt \bslash vspace}, 42

  \indexspace

  \item {\ptt {}xalignat} environment, 24
  \item {\ptt {}xxalignat} environment, 24

\end{theindex}
%%%%%%%%%%%INDEX ENDS HERE

\end{document}

\endinput

