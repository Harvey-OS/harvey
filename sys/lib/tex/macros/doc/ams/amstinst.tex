%%@texfile{%
%% filename="amstinst.tex",
%% version="2.1a",
%% date="4-SEP-1991",
%% filetype="AMS-TeX: documentation",
%% copyright="Copyright (C) American Mathematical Society, all rights
%%   reserved.  Copying of this file is authorized only if either:
%%   (1) you make absolutely no changes to your copy, including name;
%%   OR (2) if you do make changes, you first rename it to some other
%%   name.",
%% author="American Mathematical Society",
%% address="American Mathematical Society,
%%   Technical Support Department,
%%   P. O. Box 6248,
%%   Providence, RI 02940,
%%   USA",
%% telephone="401-455-4080 or (in the USA) 800-321-4AMS",
%% email="Internet: Tech-Support@Math.AMS.org",
%% checksumtype="line count",
%% checksum="449",
%% codetable="ISO/ASCII",
%% keywords="amstex, ams-tex",
%% abstract="This file is part of the AMS-\TeX{} package, version 2.1.
%%   It contains installation instructions. The file amsguide.tex
%%   inputs this file, but this file is also designed so that it
%%   can be processed separately, using only Plain \TeX{}."
%%}
%%% end of file header
%
%      This file is input by amsguide.tex, which requires AMS-TeX 2.0+.
%      However it can also be typeset separately using Plain \TeX{}, by
%      means of the mechanism below: we check to see if amsppt.sty has
%      been loaded earlier; if so, we set the catcode of the ~ character
%      to 14 so that the following section of definitions will be
%      ignored (since the definitions would be redundant); otherwise
%      we set the catcode to 9 ("ignore") so that the definitions
%      will be carried out.
%
%      Enclosing the definitions section within the initial \if ...
%      \else ... \fi would be problematic because of the outerness
%      of \head and \subhead in the amsppt documentstyle.
%
\expandafter\ifx\csname amsppt.sty\endcsname\relax
  \catcode`\~=9 \else \catcode`\~=14 \let\BYE\relax \fi
~  \let\BYE\bye
~  \hsize=30pc \vsize=47pc
~  \def\head#1\endhead{\bigskip{\sc\noindent
~    \leftskip0pt plus.5\hsize \rightskip=\leftskip\parfillskip=0pt 
~    \def\\{\break}#1\par}\nobreak\smallskip}
~  \font\sc=cmcsc10
~  \def\subhead#1\endsubhead{\removelastskip\medskip{\bf\noindent
~    #1\par}\nobreak\smallskip}
~  \def\AmSTeX{{\the\textfont2 A\kern-.1667em%
~    \lower.5ex\hbox{M}\kern-.125emS}-\TeX}
~  \font\tenss=cmss10
~  \long\def\usertype#1{\smallskip \moveright2pc\vbox{\def\par{\crcr}\halign{%
~     \setbox0\hbox{\tt##}%
~     \hbox\ifdim\wd0<10pc to10pc\fi{\unhbox0\hfil}%
~          \kern1pc \it $\langle$return$\rangle$\hss
~     \cr#1\crcr}}%
~    \smallskip}
~  \long\def\systype#1{{\rightskip=4pc\leftskip=4pc\noindent\tt #1\par}}
~  \hfuzz1pc % to suppress reporting of overfull boxes.
~  \hyphenation{amsppt}
~  \def\Textures{{\it Textures\/}}
~  \def\AMS{American Mathematical Society}
~  \def\filnam#1{{\tt\def\\{\char`\\}\ignorespaces#1\unskip}}
~  \hyphenchar\tentt=-1 % to prohibit hyphenation in tt text
~  \def\cs#1{\leavevmode
~    \skip0\lastskip\unskip\penalty0
~    \ifdim\skip0>0pt \hskip\skip0\fi
~    {\tt\def\{{\char`\{}\def\}{\char`\}}%
~      \def\\{\char`\\}\char`\\\ignorespaces#1\unskip}}
~  \newcount\rostercount
~  \def\roster{\par\smallskip\begingroup \rostercount=0
~    \def\par{{\endgraf}}\hangindent3pc
~    \def\item{\futurelet\next\iitem}%
~    \def\iitem{\ifx\next"\expandafter\iiitem
~      \else\advance\rostercount 1 \iiitem"(\the\rostercount)"\fi}%
~    \def\iiitem"##1"{\par \noindent\hbox to\hangindent{\hss##1\enspace}}%
~  }
~  \def\endroster{\par\smallskip\endgroup}
~  \def\newpage{\vfil\eject}
\catcode126=\active % restore ~ to normal; using `\~ here wouldn't work!

\head Appendix B\\ Installation Procedures -- PC\endhead

\subhead B.1. Introduction\endsubhead

The \AmSTeX{} software can be used with any full implementation of \TeX{}. 
Sections B.2--B.5 of this appendix describe the installation of \AmSTeX{} for
DOS implementations of \TeX{} on an IBM PC or compatible machine from a DOS
diskette or diskettes provided by the \AMS{}. PC\TeX{} (from Personal \TeX{},
Inc.) is used as an example.  If your operating system is not DOS, or if you
obtained \AmSTeX{} through other channels, you won't be able to use the
\filnam{install} program that is provided. Instead, follow the generic
installation instructions in section B.6. You will have to refer to the
documentation for your particular implementation of \TeX{} to work out the
details of any necessary variations. Users who encounter difficulties should
seek help from the manufacturer of their implementation of \TeX{}, or from the
\AMS{}'s technical support staff.

The TFM files for some of AMSFonts 2.1 (\filnam{msam*}, \filnam{msbm*}, and
\filnam{eufm*}) are needed to run \AmSTeX{} with the AMSPPT documentstyle, even
if you don't plan to actually print anything using the AMSFonts. In the
diskette distribution TFM files for all AMSFonts  are provided  in a directory
named \filnam{\\tfm}. They are also available from the AMS Internet archive,
\filnam{e-MATH.ams.org}, by anonymous FTP.

\subhead B.2. Getting Started\endsubhead

The installation procedures consist of moving files to the proper
directories, and creating format files.  The first step is determining
the names of the proper directories, which will vary with different
implementations of \TeX{}.  You need to know the name of your {\bf \TeX{}
inputs directory}, your {\bf \TeX{} TFM directory}, and your {\bf \TeX{}
formats directory}. Check your directory structure or  consult the
documentation for your implementation of \TeX{} to see what these names are.  
If you have any difficulty determining these directories: 
\roster
\item"(a)"Search for
the plain \TeX{} input file, \filnam{plain.tex}. The directory where it
is located will be your \TeX{} inputs directory. 
\item"(b)"Search for the font
file \filnam{cmr10.tfm}. The directory where it is found will be your
\TeX{} TFM directory. 
\item"(c)"Search for the plain \TeX{} format file,
\filnam{plain.fmt}. The directory where it is found will be your \TeX{}
formats directory.
\endroster
For example, for PC\TeX{} the directory names are  \filnam{\\pctex\\texinput},
\filnam{\\pctex\\textfms},  and
\filnam{\\pctex\\texfmts}.{\tolerance9999\hbadness\tolerance\par}


Note: If you currently have any  of the following files from earlier
releases of \AmSTeX{}, backup and delete them before installing the new
version.  They are either irrelevant or superseded in the new version of
\AmSTeX{}, and it is best to remove them to avoid confusion.  All of
them except for the last one would be found in your \TeX{} inputs
directory;  \filnam{amsplain.fmt} would be found in your \TeX{} formats
directory. 

\settabs5\columns {\tt
\+\ \ amsfil.chg& amsplain.tex& amsppt.sty& amsppt.mor& amstex.tex\cr
\+\ \ amstex.chg& cyracc.def& cyrmemo.def& cyrmemo.tex& amsplain.fmt\cr
} 

Users who did not receive \AmSTeX{} on diskette from the AMS, or those
using non-DOS implementations of \TeX{}, should  proceed now to section B.6.

\subhead B.3. Installing \AmSTeX{} (DOS/diskette)\endsubhead

The following files are used in the installation process
for DOS implementations of \TeX{}, for \AmSTeX{} 2.1:
\par\nobreak
\smallskip
\settabs3\columns
\setbox0\vbox{\tt\+install.exe& amsinst.bat& amstex.ini\cr}
\moveright\parindent\box0
\smallskip

For DOS installation, you need to know on which drive to install
\AmSTeX{}. This should be the drive on which you already have installed
\TeX{}. 

{\bf For PC\TeX{},} you will
probably want to select the first choice for each of the questions which
the installation programs asks you. This will result in placing the
\AmSTeX{} files in the directory \filnam{\\pctex\\texinput} and the TFM
files in the directory \filnam{\\pctex\\textfms}.

{\bf For other DOS implementations of \TeX{},}  you will
need to know the names of your \TeX{} inputs directory and
\TeX{} TFM directory, as described earlier, so that you can
enter them when prompted.

\smallskip When you have the disk and directory information ready, place
the disk labeled ``\AmSTeX{} 2.1'' in your floppy disk drive  and type
the following commands (if the disk is in drive B, substitute ``b'' for
``a'' in the first line):

\usertype{a:

install }

This will run the \filnam{install} program, which will ask you some
questions.  Enter the answers which you determined from reading the
above paragraphs.   

\subhead B.4. Running INITEX to Create Format Files (DOS)\endsubhead

To complete the installation procedure, you should create a
format file.  This will enable you to run \TeX{} with \AmSTeX{}, or
\AmSTeX{} and the AMSPPT preprint style, preloaded. This preloading
will save quite a bit of startup time on slower systems.
Note: Each format file takes up 150K--300K of disk space
(depending on your implementation of \TeX{}).

\smallskip 

{\bf Warning:} INITEX requires much more memory to run than
regular \TeX{}. The first time you try to run INITEX, you may get the
message ``{\tt Not enough memory to run TeX}'' or something similar (or
with simpler implementations, it may just crash). If this happens, you
must remove as many memory-resident programs as possible (such as
communications software and memory-resident utilities [TSRs]) and reboot your
system to create enough memory to run INITEX. Consult the documentation
for your implementation of \TeX{} for more information on running INITEX.
\smallskip

You are now ready to create a format file so that \AmSTeX{}, or \AmSTeX{} and
the AMSPPT preprint style, can be preloaded when you typeset a document. Before
creating your format file, you will want to consider whether you habitually use
the AMSPPT documentstyle. If you use other documentstyles  rarely or never,
then you would benefit from the use of a format file with \filnam{amsppt.sty}
preloaded.  If you are likely to use other documenstyles periodically, then you
probably do not want to preload \filnam{amsppt.sty}. To make a simple \AmSTeX{}
format file, proceed with the next paragraph.  To make a format file with
\filnam{amsppt.sty} preloaded, edit the file \filnam{amstex.ini} and remove the
percent sign (comment character) at the beginning of the line
\cs{documentstyle\{amsppt\}}, just before the \cs{dump} command.

The \TeX{} file named \filnam{amstex.ini} should now be installed in your
\TeX{} inputs directory.  Once you have decided whether to create an \AmSTeX{}
or an AMSPPT format file (see previous paragraph), run
INITEX on \filnam{amstex.ini}. E.g., for PC\TeX{}, you would give the command

\usertype{tex amstex.ini -i}

\noindent   This creates an \AmSTeX{} format file named
\filnam{amstex.fmt}.  
For other implementations of \TeX{} the form of the INITEX command
may be different, e.g., {\tt tex/i amstex.ini}.

{\bf Moving the format files to the right directory.}
Some implementations of \TeX{}, including PC\TeX{}, will automatically
place the format file in the proper directory.  Otherwise you should now
move the file manually into your \TeX{} formats directory.

\subhead B.5. Using \AmSTeX{} 2.0+ (DOS)\endsubhead

On the distribution diskettes, a DOS batch file \filnam{amstex.bat} is
provided, to make use of the format file more convenient. 
\filnam{amstex.bat} will have been placed by the installation procedures
in the root directory of the drive where \TeX{} is located.  If you are
connected to that directory or if it is in your system path, you would
run \AmSTeX{} on a file called \filnam{filename.tex} by typing

\usertype{amstex filename}

\subhead B.6. Installing \AmSTeX{} (Generic)\endsubhead

Use these instructions if you didn't receive \AmSTeX{} on diskettes from
the AMS, but obtained it by other methods, or if you have a non-DOS
implementation of \TeX{}.  You will need to know the names of your
\TeX{} inputs directory, \TeX{} TFM directory, and \TeX{} formats
directory, as explained in section~B.2.
\roster
\item Copy the following files into your \TeX{} inputs directory:
\filnam{amstex.tex}, \filnam{amsppt.sty}, \filnam{amssym.tex},
\filnam{amstex.ini}, and \filnam{amsppt1.tex}.
{\tolerance9999\par}

\item There are three documentation files: \filnam{amsguide.tex},
\filnam{joyerr.tex}, and \filnam{amsppt.doc}. If you have a documentation
directory, or wish to create one, then put these files there; otherwise
they can be put in the \TeX{} inputs directory.
{\tolerance9999\par}

\item Copy the AMSFonts TFM files to your \TeX{} TFM directory. If
you don't plan to use any of the AMSFonts, you should still copy
\filnam{dummy.tfm}, which is needed for \AmSTeX{}'s syntax check option, 
and \filnam{msam*.tfm}, \filnam{msbm*.tfm}, and \filnam{eufm*.tfm},
which are needed for typesetting the {\it User's Guide}.

\item 
You are now ready to create a format file so that \AmSTeX{}, or \AmSTeX{} and
the AMSPPT preprint style, can be preloaded when you typeset a document. Before
creating your format file, you will want to consider whether you habitually use
the AMSPPT documentstyle. If you use other documentstyles  rarely or never,
then you would benefit from the use of a format file with \filnam{amsppt.sty}
preloaded.  If you are likely to use other documenstyles periodically, then you
probably do not want to preload \filnam{amsppt.sty}. To make a simple \AmSTeX{}
format file, proceed with the next paragraph.  To make a format file with
\filnam{amsppt.sty} preloaded, edit the file \filnam{amstex.ini} and remove the
percent sign (comment character) at the beginning of the line
\cs{documentstyle\{amsppt\}}, just before the \cs{dump} command.

\item  Check the documentation for your implementation of \TeX{} to find
out how to run INITEX and create format files.  Format files greatly
speed up processing when you are using a large macro package such as
\AmSTeX{}.  If your implementation of \TeX{} doesn't automatically place format
files in the \TeX{} formats directory (check your documentation), you
will have to either go to the \TeX{} formats directory before running
INITEX, or move the format files there after they are created.  The
warning in section B.4 will also be relevant, for most PC users. Once you have
decided whether to include AMSPPT in your format file (see above paragraph),
run the file \filnam{amstex.ini} through INITEX, to create the file
\filnam{amstex.fmt}.  This is a preloaded form of \AmSTeX{}. If you included
AMSPPT, you may wish to rename it \filnam{amsppt.fmt}.

\item Now move the format file that you just created into your
\TeX{} formats directory, if you didn't create it there.
\endroster

\medskip For details of how to use format files with your implementation of
\TeX{}, see your documentation.  Typically, you use a format file by
specifying it on the command line preceded by an ampersand, e.g., {\tt
tex \&amstex filename}.
If you did not receive a printed copy of this {\it User's Guide\/} with
your distribution, you can use your newly created \AmSTeX{} or AMSPPT
format file to typeset the file \filnam{amsguide.tex}.

\newpage
\csname firstpage\string @true\endcsname

\head Appendix C\\ Installation Procedures -- Macintosh\endhead

\subhead C.1. Introduction\endsubhead

These instructions describe the installation of \AmSTeX{} for use with
\Textures{}, on the Macintosh. There is one disk in the distribution.
The files \filnam{amstex.tex} and  \filnam{amsppt.sty} are \Textures{}
documents to be input by \Textures{}.  The file \filnam{amsppt.doc} is
technical documentation for the macros defined in  \filnam{amsppt.sty}.  
The file \filnam{amstex.ini} is used in the installation to create
format files.  The file \filnam{amsguide.tex} is the \Textures{} source
for this {\it User's Guide.}{\tolerance9999\par}

The file \filnam{joyerr.tex} is a list of errata to the 1986 edition of
{\it The Joy of \TeX{}.} You may typeset it in \Textures{} using your
new \AmSTeX{} format file (once it has been created).
If you do not have AMSFonts 2.1, the
characters in this file which come from the AMS symbol fonts will appear
in the default system font.  The file \filnam{AMSFonts 2.1 metrics}
contains the \TeX{} metrics information for AMSFonts 2.1. This file is
required to use \filnam{amsppt.sty} even if you do not have AMSFonts
2.1. ({\bf Note: }Even if you currently have AMSFonts 2.0, we {\it strongly\/}
recommend that you obtain version 2.1, as the metrics have changed.)

\AmSTeX{} Versions 2.0+ can be used with \Textures{} without AMSFonts. 
However,  \AmSTeX{} Versions 2.0+ will not work with releases of AMSFonts
previous to Version 2.0. Additionally, if AMSFonts 2.1 are to be used,
they require \Textures{} version 1.2 or higher. If you have an earlier
release of \Textures{}, you must upgrade before using these fonts. To
upgrade \Textures{}, contact the manufacturer: Blue Sky Research, 534 SW
Third Ave., Portland, OR 97204; 800-622-8398 or
503-222-9571. 

\subhead C.2. If You Have a Previous Version of \AmSTeX{}\endsubhead

Before installing the new version of \AmSTeX{}, you should backup and
delete files from your hard disk which are related to previous releases
of \AmSTeX{}. In particular, you should remove the following files from
your \filnam{Textures} folder (or any folders contained in your
\filnam{Textures} folder):

\settabs5\columns {\tt
\+\ \ amsfil.chg&amsplain.tex&amsppt.sty&amsppt.mor&amstex.tex\cr
\+\ \ amstex.chg&cyracc.def&cyrmemo.def&cyrmemo.tex\cr }

\noindent
and you should remove any previous \AmSTeX{} format files from
your \filnam{TeX formats} folder.
These files are either irrelevant or superseded in the new version of
\AmSTeX{}, and it is best to remove them from your hard disk to avoid
confusion.

\subhead C.3. Installing \AmSTeX{}\endsubhead

If you do not have a folder named \filnam{TeX inputs} inside the
\filnam{Textures} folder on your hard disk, create a new folder inside
your \filnam{Textures} folder and name it \filnam{TeX inputs}. Also, if
you do not have a  \filnam{TeX formats} folder, create a new folder
inside your \filnam{Textures}  folder and name it \filnam{TeX formats}.
Likewise, create a \filnam{TeX fonts} folder if you do not already
have one.

To install the new version of \AmSTeX{} on your system, copy the
following files into the  \filnam{TeX inputs} folder inside the
\filnam{Textures} folder on your hard disk:

{\tt \quad amstex.tex\quad amssym.tex\quad amsppt.sty\quad
amsppt1.tex\quad amstex.ini} 

Then copy the file \filnam{amsfonts 2.1 metrics} into the \filnam{TeX
fonts} folder inside of the \filnam{Textures} folder on your hard disk.

You are now ready to create a format file so that \AmSTeX{}, or \AmSTeX{} and
the AMSPPT preprint style, can be preloaded when you typeset a document. Before
creating your format file, you will want to consider whether you habitually use
the AMSPPT documentstyle. If you use other documentstyles  rarely or never,
then you would benefit from the use of a format file with \filnam{amsppt.sty}
preloaded.  If you are likely to use other documenstyles periodically, then you
probably do not want to preload \filnam{amsppt.sty}. To make a simple \AmSTeX{}
format file, proceed with the next paragraph.  To make a format file with
\filnam{amsppt.sty} preloaded, edit the file \filnam{amstex.ini} and remove the
percent sign (comment character) at the beginning of the line
\cs{documentstyle\{amsppt\}}, just before the \cs{dump} command.

Open the \filnam{TeX inputs} folder (inside your \filnam{Textures} folder) and
double-click on the file \filnam{amstex.ini} to start running \Textures{}. Make
sure that the line \cs{input plain} is commented out (begins with a percent
sign). If there is not a percent sign at the beginning of this line, insert
one. Check under the Typeset menu to make sure that the Plain format is
selected (there is a check mark next to it). Select ``Typeset'' from the
Typeset menu. When the dialog box appears asking you what to name the format
file, go through the folder hierarchy to place yourself inside the \filnam{TeX
formats} folder inside your \filnam{Textures} folder. You may name the file
what you wish, but \filnam{2.1} is a good idea.  If you included
\filnam{amsppt.sty} in your format file, name the  format file
\filnam{AMS-TeX/AMSPPT} or something similarly descriptive.

For more information about using formats in \Textures{}, see the
\Textures{} {\it User's Guide\/}.

If you did not receive a printed copy of the {\it \AmSTeX{} User's Guide\/}
with your distribution, you can use your newly created \AmSTeX{} or AMSPPT
format file to typeset the file \filnam{amsguide.tex}.

\medskip\noindent
Questions concerning \Textures{} should be directed to:
\smallskip

\begingroup
\obeylines\parindent1in \parskip0pt
Blue Sky Research
534 Southwest Third Avenue
Portland, OR 97204
Phone: 800-622-8398 \quad or\quad 503-222-9571
\endgroup

\BYE % This is = \relax if this file is input by amsguide.tex
%% \CharacterTable
%%  {Upper-case    \A\B\C\D\E\F\G\H\I\J\K\L\M\N\O\P\Q\R\S\T\U\V\W\X\Y\Z
%%   Lower-case    \a\b\c\d\e\f\g\h\i\j\k\l\m\n\o\p\q\r\s\t\u\v\w\x\y\z
%%   Digits        \0\1\2\3\4\5\6\7\8\9
%%   Exclamation   \!     Double quote  \"     Hash (number) \#
%%   Dollar        \$     Percent       \%     Ampersand     \&
%%   Acute accent  \'     Left paren    \(     Right paren   \)
%%   Asterisk      \*     Plus          \+     Comma         \,
%%   Minus         \-     Point         \.     Solidus       \/
%%   Colon         \:     Semicolon     \;     Less than     \<
%%   Equals        \=     Greater than  \>     Question mark \?
%%   Commercial at \@     Left bracket  \[     Backslash     \\
%%   Right bracket \]     Circumflex    \^     Underscore    \_
%%   Grave accent  \`     Left brace    \{     Vertical bar  \|
%%   Right brace   \}     Tilde         \~}
