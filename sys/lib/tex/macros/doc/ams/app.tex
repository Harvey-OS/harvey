%%%%%%%%%%%%%%%%%%%%%%%%%%%%%%%%%%%%%%%%%%%%%%%%%%%%%%%%%%%%%%%%%%%%%%%%%%%%
% APP.TEX						    July 1990      %
%                                                                          %
% This file is part of the AMS-LaTeX Version 1.0 distribution              %
%   American Mathematical Society, Technical Support Group,                %
%   P. O. Box 6248, Providence, RI 02940                                   %
%   800-321-4AMS (321-4267) or 401-455-4080                                %
%   Internet: Tech-Support@Math.AMS.com                                    %
%%%%%%%%%%%%%%%%%%%%%%%%%%%%%%%%%%%%%%%%%%%%%%%%%%%%%%%%%%%%%%%%%%%%%%%%%%%%
\appendix
\chapter[Nonselfadjoint Equations]%
{On the Eigenvalues and Eigenfunctions\\
of Certain Classes of Nonselfadjoint Equations}

\section{Compact operators} In an appropriate Hilbert space, all 
the equations considered below can be reduced to the 
form
\begin{equation}
y=L(\lambda)y+f,\qquad L(\lambda)=K_0+\lambda K_1+\dots+\lambda^n
K_n,
\end{equation}
where $y$ and $f$ are elements of the Hilbert space, 
$\lambda$ is a complex parameter, and the $K_i$ are
compact operators.

A compact operator $R(\lambda)$ is the resolvent of 
$L(\lambda)$ if $(E+R)(E-L)=E$.  If the resolvent exists
for some $\lambda=\lambda_0$, it is a meromorphic function
of $\lambda$ on the whole plane.  We say that $y$ is
an eigenelement for the eigenvalue $\lambda=c$, and that
$y_1,\dots,y_k$ are elements associated with it (or
associated elements) if 
\begin{equation}
y=L(c)y,\quad y_k=L(c)y_k+\frac{1}{1!}\,\frac{\partial L(c)}{\partial c}
y_{k-1}+\dots+\frac{1}{k!}\,\frac{\partial^kL(c)}{\partial c^k}y.
\end{equation}
Note that if $y$ is an eigenelement and $y_1,\dots,y_k$
are elements associated with it, then $y(t)=e^{ct}(y_k
+y_{k-1}t/1!+\dots+yt^k/k!)$ is a solution of the equation
$y=K_0y+K_1\partial y/\partial t+\dots+K_n\partial^ny/
\partial t^n$.

\endinput
