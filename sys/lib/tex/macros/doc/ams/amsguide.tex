%  AMSGUIDE.TEX                                         February 1990
%
%  This file is the User's Guide describing the use of AMS-TeX 2.0.
%
%  American Mathematical Society, Technical Support Group, P. O. Box 6248,
%        Providence, RI 02940
%  800-321-4AMS or 401-455-4080;  Internet: Tech-Support@Math.AMS.com
%
%  TeXing this file requires the following files and fonts:
%       AMSTEX.TEX (version 2.0)
%       AMSPPT.STY (version 2.0)
%       AMSSYM.TEX (loaded by AMSPPT.STY)
%       MSAM10
%       MSBM10
%       EUFM10
%       CMBXTI10
%
%%%%%%%%%%%%%%%%%%%%%%%%%%%%%%%%%%%%%%%%%%%%%%%%%%%%%%%%%%%%%%%%%%%%%%%%


\input amstex
\documentstyle{amsppt}

%  Change default dimensions and fonts
\parindent=10pt
\hfuzz1pc % to suppress reporting of overfull boxes.
\aboveheadskip=3\bigskipamount
\belowheadskip=\medskipamount
\subheadskip=\bigskipamount
\fontdimen3\tenrm=2.0pt
\font\oldnos=cmmi10
\font\tenbfit=cmbxti10
\let\bfit=\tenbfit
\loadbold
\addto\tenpoint{\abovedisplayskip=6pt plus2pt minus3pt
    \belowdisplayskip=\abovedisplayskip}

%  Define macros for text substitution and for presentation of examples
\define\AMS{American Mathematical Society}
\define\JAMS{{\it Journal of the \AMS}}
\define\JoT{{\it The Joy of \TeX}}
\define\Joy{{\it Joy}}
\def\LaTeX{{\rm L\kern-.36em\raise.3ex\hbox{\smc a}\kern-.15em
    T\kern-.1667em\lower.7ex\hbox{E}\kern-.125emX}}

\def\filnam#1{\hbox{\tt\ignorespaces#1\unskip}}
\newdimen\exindent
\exindent=3\parindent
% Add a high penalty to discourage line breaks within an example
% without absolutely prohibiting them.
{\obeylines
 \gdef^^M{\par\penalty9999}%
 \gdef\beginexample#1{\medskip\bgroup %
   \def~{\char`\~}%
   \NoBlackBoxes\tt\frenchspacing %
   \parindent=0pt#1\leftskip=\exindent\obeylines}
}%  end \obeylines
\def\endexample{\endgraf\egroup\medskip}
\newdimen\exboxwidth
\exboxwidth=3in
\def\exbox#1#2{\noindent \hangindent=\exboxwidth
  \leavevmode\llap{\null\rm#1\unskip\enspace}%
  \hbox to\exboxwidth{\tt\ignorespaces#2\hss}\rm\ignorespaces}
\def\\{\char`\\}
\def\ttcs#1{\leavevmode\hbox{\tt\\\ignorespaces#1\unskip}}
\def\{{\char`\{\relax}
\def\}{\char`\}\relax}
\def\<#1>{{\rm$\langle$#1$\rangle$}}
\def\Dimen{\<dimen>}
\def\tab{{\smc tab}}

\catcode`\@=11

%  Redefine the \subhead macro to be on a line by itself and omit period.
\outer\def\subhead#1\endsubhead{\par\penaltyandskip@{-100}\subheadskip
  \noindent{\subheadfont@\ignorespaces#1\unskip\endgraf}\nobreak\noindent}

%  Define macros for presentation of tables of symbols.
\def\BBB#1{\par\bigbreak
  \leavevmode\llap{$\bullet$\enspace}{\bf#1}}
\newdimen\biggest
\setbox0\hbox{$\dashrightarrow$}\biggest=\wd0
\def\1#1{\hbox to\biggest{\hfill$\csname#1\endcsname$\hfill}\ \ %
  \ttcs{#1}}
\def\fudge{\hbox to\biggest{}\ \ \hphantom{\tt\char'134 }}

\def\getID@#1{\edef\next@{\expandafter\meaning\csname#1\endcsname}%
 \expandafter\getID@@\next@0\getID@@}
\def\getID@@#1"#2#3#4#5#6\getID@@{\def\next@{#6}%
  \ifx\next@\empty
   \def\next@{#2}%
    \ifx\next@\msafam@
     \def\ID@{10#3#4}%
    \else
     \def\ID@{20#3#4}%
    \fi
  \else 
   \def\next@{#3}%
    \ifx\next@\msafam@
     \def\ID@{1#2#4#5}%
    \else
     \def\ID@{2#2#4#5}%
    \fi
  \fi}
\def\2#1{\hbox to.5\hsize
  {\hbox to\biggest{\hfill$\csname#1\endcsname$\hfill}\ \ %
    \getID@{#1}{\tt\ID@}\ \ \ttcs{#1}\hfill}}
\def\3#1#2{\hbox to.5\hsize
  {\hbox to\biggest{\hfil$\csname#1\endcsname$\hfil}\ \ %
    \getID@{#1}{\tt\ID@}\ \ \ttcs{#1}, \ttcs{#2}\hss}}
\def\4#1{\hbox to.5\hsize
  {\hbox to\biggest{\hfill$\csname#1\endcsname$\hfill}\ \ %
    \getID@{#1}{\tt\ID@}\ \ \ttcs{#1}\ \ {\eightpoint(U)}\hfill}}

\catcode`\@=\active

%%%%%%%%%%%%%%%%%%%%%%%%%%%%%%%%%%%%%%%%%%%%%%%%%%%%%%%%%%%%%%%%%%%%%%%%

\topmatter
\title\nofrills User's Guide to \AmSTeX{} Version 2.0\endtitle
\leftheadtext\nofrills{User's Guide to \AmSTeX{} Version 2.0}
\rightheadtext\nofrills{User's Guide to \AmSTeX{} Version 2.0}
\date January 1990\enddate

\toc
\head 1. Overview\endhead
\head 2. Formatting Features\endhead
\head 3. Mathematical Constructions\endhead
\head 4. Fonts\endhead
\head 5. Symbol Names\endhead
\head 6. Other Things You Ought to Know\endhead
\head 7. Getting Help\endhead
\head {} Appendix. Sample Bibliography Input and Output\endhead
\endtoc

\endtopmatter
\document

%%%%%%%%%%%%%%%%%%%%%%%%%%%%%%%%%%%%%%%%%%%%%%%%%%%%%%%%%%%%%%%%%%%%%%%%

\head 1. Overview
\endhead

\AmSTeX{} is a macro package for \TeX, designed to simplify the input
of mathematical material and format the output according to preset
style specifications.  Although \AmSTeX{} is copyright by the \AMS{},
its use is not restricted, but is encouraged for the preparation of
manuscripts intended for publication both in the Society's books and
journals, and also in other mathematical literature.  In recognition of
the copyright, the Society requests that published documents prepared
with \AmSTeX{} include an acknowledgment of its use.  The suggested
forms for acknowledgments are given in the section {\bf Other Things
You Ought to Know}.

Version 2.0 of \AmSTeX{} contains numerous minor improvements
and bug fixes, as well as some major changes involving additional fonts.
This User's Guide describes all the new and changed features and how
to use them.  Topics are grouped by type, and then presented in roughly
the same order as they appear in \JoT{}.

This User's Guide assumes that you already have a copy of \JoT{}.
It contains references to specific pages that probably won't make
sense if you don't have a copy.
It also assumes that you will be using the ``preprint style,'' a set
of macros that provides features specific to the formatting of a
document, such as headings, page numbers, and the like.
If you are planning to use the preprint style, you will also need to
have a copy of AMSFonts Version~2.0.  \JoT{} and AMSFonts 2.0 are
available from the \AMS{} and other distributors.

\vfil\eject                     %%%%%%%%%%%%%%%%%%%%

\subhead Files comprising the \AmSTeX{} Version~2.0 package
\endsubhead

The following files are contained in the \AmSTeX{} Version~2.0 package
distributed by AMS:
\beginexample{\exboxwidth=1.125in}
\exbox{}{AMSTEX.TEX} the \AmSTeX{} Version 2.0 macros
\exbox{}{AMSSYM.TEX} macros defining the symbols in fonts \filnam{msam} %
        and \filnam{msbm}
\exbox{}{AMSPPT.STY} the preprint style for \AmSTeX{} Version 2.0
\exbox{}{AMSPPT.DOC} technical documentation for \filnam{AMSPPT.STY}
\exbox{}{AMSGUIDE.TEX} the source file for this User's Guide
\exbox{}{JOYERR.TEX} errata to \JoT{} prior to \AmSTeX{} Version 2.0
\endexample

The file \filnam{AMSPPT.DOC} is an  {\smc ascii} file, and is
not intended to be processed with \TeX{}\null.  This  documentation file
is arranged in the same order as the macro file that it describes, and
explains the intent and mechanics of the macros in detail.  A separate
file (\filnam{AMSTEX.DOC}), documenting the file \filnam{AMSTEX.TEX},
is available on request.

For instructions on installing the \AmSTeX{} macros and preprint style,
see the separate Installation Guide provided with the package.

\subhead General description of changes
\endsubhead

\AmSTeX{} 2.0, the preprint style, and their technical documentation are
the result of a joint effort begun by Michael Spivak and extended
by the Composition Technical Support group of the \AMS.

In this new version of \AmSTeX, the following changes have been installed:
\roster
\item All known bugs have been eliminated.
\item Messages identifying the current versions of \filnam{AMSTEX.TEX}
  and \filnam{AMSPPT.STY} will be displayed on your terminal screen and
  in the log file.
\item Some error and help messages have been changed for the sake of
  clarity or to provide more information.
\item Refinements have been made to conserve memory space.
\item The CM versions of the Computer Modern fonts have replaced the
  AM versions.
\item Support for additional fonts has been added.
{%
  \itemitem{(a)} \filnam{AMSTEX.TEX} provides the mechanism for accessing
    the Euler and extra symbol fonts of the AMSFonts collection.
  \itemitem{(b)} The preprint style assumes that fonts \filnam{msam},
    \filnam{msbm}, and \filnam{eufm} are installed and available.
\par}
\item Changes have been made to the preprint style to make it conform
  more closely to the style of AMS publications, in particular, the \JAMS.
{%
  \itemitem{(a)} Running heads are automatic; they can be suppressed
    if desired.
  \itemitem{(b)} Additional elements are recognized in both the top matter
    and the body of a document, and the input syntax has been regularized.
  \itemitem{(c)} Footnotes are now indented.\footnote{Like this.}
  \itemitem{(d)} The style of the references has changed considerably.
\par}
\item In the preprint style, mathematics-oriented hyphenation exceptions
  have been added.  (These follow American, not British, rules.)
\item A new option in the preprint style allows documents to be formatted
  as chapters of a monograph rather than as separate papers.
\item Finally, some optional formatting features requested by
  \AmSTeX{} users have been added to the preprint style.
\endroster


\subhead This User's Guide
\endsubhead

This User's Guide has been prepared using \AmSTeX{} Version~2.0 with the
preprint style.  Some changes have been made: font and dimension settings
have been reset, the macros for headings have been redefined to produce a
result more suited to documentation, and some {\it ad hoc\/} macros have
been defined to simplify the presentation of particular information.
However, in general, this document and the file from which it was produced
illustrate the general appearance and input for a preprint with running
heads.  Printing the output of \TeX{} for this Guide requires AMSFonts
Version~2.0.


%%%%%%%%%%%%%%%%%%%%%%%%%%%%%%%%%%%%%%%%%%%%%%%%%%%%%%%%%%%%%%%%%%%%%%%%

\head 2. Formatting Features
\endhead

Formatting documents prepared with \AmSTeX{} is accomplished by a
``style file.''  The features described here are part of the preprint
style.  The preprint style for \AmSTeX{} Version~2.0 will, unless it
is told to use the style of a \ttcs{Monograph}, format an input file
in a manner suitable for a paper in a journal.  Unless noted otherwise,
that is the style described below.

\setbox1=\hbox{\tt\\dedicatory...\\enddedicatory\ }
\setbox0=\vbox{\hsize=\wd1\parindent=0pt\tt\obeylines
  \strut\\title...\\endtitle
  \\author...\\endauthor
  \\affil...\\endaffil
  \\address...\\endaddress
  \\email...\\endemail
  \\dedicatory...\\enddedicatory
  \\date...\\enddate
  \\thanks...\\endthanks
  \\translator...\\endtranslator
  \\keywords...\\endkeywords
  \\subjclass...\\endsubjclass
  \strut\\abstract...\\endabstract\endgraf}
\setbox1=\hbox{$\dsize\left\lbrace\,\vcenter{\vphantom{\copy0}}\right.
   \nulldelimiterspace=0pt$}%
\setbox2=\hbox{\kern-\wd1$\dsize\left\lbrace\,\vcenter{\copy0}\right\rbrace$}

\subhead Top matter {\rm (\Joy, p.~40)}
\endsubhead

Some commands affect the appearance of a whole document.  Such commands
should go at the top of your input file, right after the \ttcs{documentstyle}
line and before the \ttcs{topmatter} line.  This area will be referred to as
the ``preamble.''  Commands that should be in the preamble include the
existing tags
\ttcs{define}, \ttcs{TagsOnRight}, etc., and the new options
\ttcs{NoPageNumbers}, \ttcs{NoRunningHeads}, \ttcs{Monograph},
\ttcs{pagewidth}, \ttcs{pageheight}, \ttcs{pageno},
and commands that load fonts.
The preamble is relevant to any document, whether paper or monograph.

The syntax of elements of the top matter has been changed.  All
now require both a beginning and an ending tag.  The beginning of an
\AmSTeX{} file should thus look something like this:
\beginexample{}
\\input amstex
\\documentstyle\{...\}
\medskip
\<preamble commands, such as \ttcs{define}, \ttcs{pageno},
\ttcs{NoRunningHeads}, \ttcs{Monograph}, \ttcs{loadbold}, etc.>
\medskip
\\topmatter
\indent\box2
\strut\\endtopmatter
\\document
\endexample

If any \ttcs{end...} tag is omitted, an error message will appear at
\ttcs{endtopmatter}, if not before.  If the forgotten tag was
\ttcs{endtitle}, \ttcs{endkeywords}, or another \ttcs{end...} for an
item that has the \ttcs{nofrills} option, the error message will be
``{\tt !~Paragraph ended before \ttcs{next\@} was complete}.''  
If you omit \ttcs{endtopmatter}, on the other hand, none of the
topmatter material will print (as before).

For multi-line titles, affiliations, authors, or dedications (basically
everywhere that lines are centered individually rather than being
set in paragraphs), line breaks are obtained by
using \ttcs{\\}.  In other parts of the topmatter, which are set in paragraph
form, line breaks are obtained by \ttcs{linebreak}.

The title will be set in uppercase.  To turn off the automatic
uppercasing, use the \ttcs{nofrills} option:
\ttcs{title\\nofrills...\\endtitle}.

The \ttcs{overlong} option that used to be available for \ttcs{title},
\ttcs{author}, and \ttcs{head} has been removed, mainly because publishers
as a rule would rather introduce extra line breaks than allow text to hang
over into the margins.

Electronic mail addresses can be typed using \ttcs{email...\\endemail}.
Every \ttcs{email} address must be preceded by a regular \ttcs{address}.
Multiple \ttcs{email} addresses may be used, but each
\ttcs{email...\\endemail} must be paired with the \ttcs{address} of the
same author.  The \ttcs{email} address will be printed at the end of
the paper, as ``{\it E-mail:\/} \<net-address>'', following the address
it is paired with.  If no preceding \ttcs{address} exists, the e-mail
address will not print.

The \ttcs{dedicatory} is new, to be used for such things as
``Dedicated to Professor X on the occasion of his eightieth birthday.''
A dedication will appear in italics, before the abstract.

Unlike \ttcs{address} and \ttcs{email}, \ttcs{thanks} cannot appear more
than once in the topmatter.  It is sometimes desirable, however, to
acknowledge support separately for separate authors of a joint paper. 
Since \ttcs{thanks} creates an unnumbered footnote, separate
acknowledgments should be made separate paragraphs within the footnote.
But this brings up a special problem: It's fairly easy to forget the
ending~{\tt\}} for a \ttcs{footnote}, so an error-detecting mechanism
built into \TeX{} looks for the end of a paragraph within a footnote, and
if it sees one, assumes that you've forgotten the~{\tt\}} and issues an
error message.  In order to end a paragraph without setting off the
error-detector, you must use
\ttcs{endgraf}\footnote{\ttcs{endgraf} is just an abbreviation for ``end
paragraph.''}
instead of a blank line or \ttcs{par}.

In case a paper has not only an author but a translator, \ttcs{translator}
is provided.  This information will be printed at the end of the paper in
eight-point roman, as ``Translated by'' followed by the translator's name
in uppercase.

The information for \ttcs{keywords} and \ttcs{subjclass} now appear as
unnumbered footnotes at the foot of the first page, as in AMS journals.
(They used to appear at the end of an article.)  In a monograph chapter
they will not print at all, since they should be handled separately, as
part of the front matter for the monograph.

The abstract heading, ``{\eightpoint\smc Abstract.}'', now appears in caps
and small caps, in the same size (eight point) as the abstract itself.

A simple table of contents setup is now available.  Tables of contents
are typed in the topmatter along with everything else (except for
monographs---see the section {\bf Book Formatting} below), using
\ttcs{toc...\\endtoc}.
\beginexample{}
\\toc
\\specialhead...\\endspecialhead
\\head...\\endhead
\\subhead...\\endsubhead
\\subsubhead...\\endsubsubhead
\\endtoc
\endexample
\noindent
The syntax of the parts is intentionally identical to the syntax used
for headings within the document (see the sections {\bf Headings} and
{\bf Book Formatting} below), so that for those who wish to do so
and have a capable text editor, the table of contents can be constructed
by extracting the relevant lines from the main text.\footnote{But note that
the original line breaks in multi-line headings would not be appropriate
for the table of contents, so you'd want to remove any \ttcs{\\}'s that
might be present.}

Page numbers aren't usually appropriate for the short table of contents
that might appear in a journal article, but if desired, page numbers can
be entered in a manner similar to that for a monograph; see the section
{\bf Book Formatting} below.

The hanging indentation within a table of contents for \ttcs{head} and
\ttcs{subhead} is preset to accommodate numbers of the form
``1.''\ and ``1.1.''\ respectively; the amount of indentation can
be adjusted by using \ttcs{widestnumber}:
\beginexample{}
\\toc
\\widestnumber\\head\{10\}
\\widestnumber\\subhead\{10.1\}
...
\endexample
\noindent This can be done more than once within different sections of
the table of contents, if desired.

If the ``section number'' of a \ttcs{head} happens to be something like
``Appendix'' (as actually happens in this User's Guide), the input should
be modified as follows:
\beginexample{}
\\head \{\}\ Appendix. Sample bibliography input ...\\endhead
\endexample
\noindent
Insertion of {\tt\{\}} followed by a space at the beginning of the heading
text will cause the entire entry to be set flush left as a unit.

If you are preparing a monograph, the format and content of the top matter
will be different.  See the section below on {\bf Book Formatting} for
details.


\subhead Headings {\rm (\Joy, p.~42)}
\endsubhead

There are now four levels of headings instead of two (not counting
\ttcs{title}'s), and the names have been changed to have
consistent syntax.  The old form was
\ttcs{heading...\\endheading} and \ttcs{subheading\{...\}}.  The new
headings are:
\beginexample{}
\\specialhead...\\endspecialhead
\\head...\\endhead
\\subhead...\\endsubhead
\\subsubhead...\\endsubsubhead
\endexample
\noindent The heading of
this section was typed as
\beginexample{}
\\head 2. Formatting Features
\\endhead
\endexample

\noindent And the subheading for this subsection was typed as
\beginexample{}
\\subhead Headings \{\\rm (\{\\it Joy\}, p.\~{ }42)\}
\\endsubhead
\endexample
\noindent
Ordinarily, subheadings in the preprint style are run into the text, but for
this User's Guide, the style varies slightly.

\indent\ttcs{specialhead} is for long articles that need extra divisions at
a level above the \ttcs{head} level. In the preprint style
\ttcs{specialhead} uses boldface type and is set ragged right; \ttcs{head}
is small caps, centered; \ttcs{subhead} is boldface, flush left,
run in with the following text; and
\ttcs{subsubhead} is italic, indented as for an ordinary paragraph, and run
into the text.

Explicit line breaks are obtained by a \ttcs{\\} in a \ttcs{head} or a
\ttcs{specialhead}, but for \ttcs{subhead} and \ttcs{subsubhead},
which are part of their paragraph, just use \ttcs{linebreak} as
you would in normal paragraphed text.

If you are preparing a monograph, the styles of headings will be different.
See the section below on {\bf Book Formatting} for details.


\subhead Theorems and Proofs {\rm (\Joy, pp.~42--43)}
\endsubhead

In accordance with the style of the \JAMS, the labels on \ttcs{proclaim}'s
and similar constructions are now printed in boldface type (\ttcs{bf}).
However, unlike the \JAMS, the preprint style uses slanted type (\ttcs{sl})
for the text of a \ttcs{proclaim}, rather than italic.
(Most \AMS{} publications currently use Times Roman fonts, for which no
slanted form was designed.)

In addition to the usual proclamations and demonstrations, mathematicians
may pose other kinds of propositions, which editors may prefer to see
presented in different styles.  The following have been provided in
the preprint style.
\beginexample{}
\\definition...\\enddefinition
\\example...\\endexample
\\remark...\\endremark
\endexample
\noindent
In the preprint style \ttcs{definition} and \ttcs{example} have
the spacing and heading font of \ttcs{proclaim}, but are in roman.
\ttcs{remark} resembles \ttcs{demo}
except that extra space added at the end of a proof by \ttcs{enddemo}
is not added by \ttcs{endremark}.


\subhead Other Devices \rm(\Joy, pp.~43--45)
\endsubhead

For a list produced by \ttcs{roster}, the amount of indentation can be
adjusted to accommodate wide item numbers.  Just before beginning the
\ttcs{roster}, type, for example, \ttcs{widestnumber\\item\{(viii)\}}.
This adjustment is temporary.  The default will be reinstated by
\ttcs{endroster}.

Formerly \ttcs{cite} produced the reference citation in boldface
type, within square brackets: \cite{\bf21}; 
the current AMS style retains the
brackets but uses roman type instead of boldface: \cite{21}.

A new structure, \ttcs{block...\\endblock}, has been added for quotations.
It is intended for use in the middle of a paragraph to quote an
extract from another source.


\subhead Book Formatting {\rm (new entry, \Joy, p.~143)}
\endsubhead

If you are preparing a monograph, several features are available in the
preprint style that will make your output look like chapters rather than
individual papers.

First of all, you must signal your intentions by typing \ttcs{Monograph}
in the preamble, right after the \ttcs{documentstyle} line.

A typical topmatter section for a monograph chapter would be typed
like this:
\beginexample{}
\\documentstyle\{amsppt\}
\\Monograph
\\topmatter
\\title\\chapter\{4\} Matrix Algebras\\endtitle
\\endtopmatter
\endexample
\noindent which produces a chapter heading that looks like this:
\bigskip
\vbox{
\centerline{\eightpoint CHAPTER IV}
\bigskip
\centerline{\bf MATRIX ALGEBRAS}}
\bigskip
\noindent
Notice that the number is converted automatically to roman numerals and
the word ``{\eightpoint CHAPTER}'' is added.  For a chapter title that
needs a different sort of treatment, \ttcs{nofrills} can be used:
\beginexample{}
\\topmatter
\\title\\chapter\\nofrills\{APPENDIX B\} The Poisson Integral\\endtitle
\\endtopmatter
\endexample
\noindent This produces
\bigskip
\vbox{
\centerline{\eightpoint APPENDIX B}
\bigskip
\centerline{\bf THE POISSON INTEGRAL}}
\bigskip
\noindent The replacement \ttcs{chapter} text will appear exactly as
typed.

Finally, for things like a preface or introduction which has no
pretitle text at all, omit the \ttcs{chapter} command:
\par\eject                      %%%%%%%%%%%%%%%%%%%%
\beginexample{}
\\topmatter
\\title Preface\\endtitle
\\endtopmatter
\endexample

In monographs, the table of contents is usually treated as a separate
chapter.  Start by typing the title ``Contents'' as for a preface or
introduction, and then use the \ttcs{toc...\\endtoc} structure as the
body of the document (rather than putting it in the topmatter, as you
would for a journal article).
\beginexample{}
\\topmatter
\\title Contents\\endtitle
\\endtopmatter
\bigskip
\\document
\\toc
\\title Preface\\page\{vii\}\\endtitle
\\title\\chapter\{1\} Matrix Algebras\\page\{1\}\\endtitle
\\head Continuous complex-valued functions\\page\{1\}\\endhead
...
\\title Bibliography\\page\{307\}\\endtitle
\\endtoc
\\enddocument
\endexample

The chapter titles listed in the table of contents are typed in the same
way as in actual use.  To get page numbers in the table of contents, use
\ttcs{page} as shown, just before the ending of an element.  This option
is available for all levels of headings.

In a monograph using the preprint style, the chapter title is used for
the left running head and the text of section headings
(from \ttcs{head}) appears as the right running head.  
It's not uncommon for the text of a heading to be too long to fit
in the running head width; in such a case use \ttcs{rightheadtext} to
specify a shortened form of the heading for use in the running heads:
\beginexample{}
\\head Fourier coefficients of continuous periodic functions
of bounded entropy norm\\endhead
\\rightheadtext\{Fourier coefficients of periodic functions\}
\endexample
\noindent This should follow immediately after the \ttcs{head}, to ensure
that both take effect on the same page.  If the chapter title is too long
to fit as a running head, a shortened form can be supplied in a similar
way with \ttcs{leftheadtext} immediately after the \ttcs{title}.
See also the section~{\bf Running Heads}.

The style for a chapter of a monograph differs in some particulars from the
style for a paper.  The text of a \ttcs{head} will be boldface instead of
small caps; headings of theorems, propositions, definitions, remarks, etc\.
will be small caps instead of boldface, and indented rather than flush 
left.


\subhead Inserts with Captions {\rm (new entry, \Joy, p.~168)}
\endsubhead

Figures, tables, and some other kinds of objects are often handled as
inserts.  These objects may be prepared separately from the main document
and pasted in, in which case space must be left for them.
These objects usually have captions; a caption may be positioned above (for
a table) or below (for a figure).

An insert may be specified for the top or ``middle'' of a page, i.e.,
right where the input for the insert occurs in the text.  These are
typed as \ttcs{topinsert} and \ttcs{midinsert} respectively.
Furthermore, a caption may be placed at the top or the bottom of the insert,
using the tags \ttcs{topcaption} and \ttcs{botcaption} respectively.

The general structure used to specify an insert with a caption at the top is:
\beginexample{}
\\topinsert\quad{\rm or}\quad \\midinsert
\ttcs{captionwidth}\{\Dimen\}\quad{\rm(optional)}
\\topcaption\{\<caption label>\}
\ \<optional caption text>
\\endcaption
\ttcs{vspace}\{\Dimen\}\quad{\rm or}\quad%
  \<optional code for the insertion body>
\\endinsert
\endexample

If a bottom caption was desired, \ttcs{topcaption} would be replaced by
\ttcs{botcaption}, and
the \ttcs{vspace\{\Dimen\}} option (or the
optional code for the insertion body) would be moved before the
\ttcs{botcaption} macro.

The \ttcs{vspace\{\Dimen\}} option would be used to leave blank space for an
object to be pasted into place.
If a \Dimen{} is specified, its value should be the exact height of the
object to be pasted in.  Extra space around the object and the caption
are dependent on the document style, and will be provided automatically.

The \ttcs{captionwidth\{\Dimen\}} option may be used to override the default
caption width specified by the document style.

The \<caption label> is something like ``Figure~1'' or ``Table~2a''.
Do not type any final punctuation; it will be provided.  The caption
label will be set in caps and small caps.

The \<optional caption text> is any descriptive text that may be desired. 
The preprint style will set this in roman.  Even if there is no text, the
\ttcs{endcaption} tag must be present.

If you choose to include the \TeX{} code for a figure, table, or other
captioned object in the input, then omit the \ttcs{vspace\{\Dimen\}} line
and type the code after the \ttcs{endcaption} and before \ttcs{endinsert}.
The size will be calculated automatically, and the caption set in the
appropriate location above or below the object.

Sometimes a table is small enough that it is not necessary to put it in an
insert.  If the caption is to appear above it, input can be typed as follows:
\beginexample{}
\\topcaption\{\<caption label>\}
\ \<optional caption text>
\\endcaption
\ \<code for the table body>
\endexample
\noindent
The form of the input would be the following if the caption is to appear below:
\beginexample{}
\ \<code for the table body>
\\botcaption\{\<caption label>\}
\ \<optional caption text>
\\endcaption
\endexample
\noindent
This form of ``insertion'' should be used only for very small objects.


\subhead Page Numbers {\rm (\Joy, p.~178)}
\endsubhead

If you are using the preprint style, page numbers will appear in the running
heads, at the outside margin, except for the first page, where the running head
will be omitted and the page number will be centered at the bottom of the page.

If you wish to omit page numbers, type \ttcs{NoPageNumbers} at the
beginning of the document (after the \ttcs{documentstyle} line).
The running head text will remain; see also {\bf Running Heads}.

In the previous version of the preprint style you could not get roman
numeral page numbers, e.g.\ for a table of contents or preface.  Now you
can, using the normal \TeX{} convention of \ttcs{pageno} plus a negative
number.


\subhead Page Size {\rm (\Joy, p.~178)}
\endsubhead

In the preprint style, the default page width is 30pc, and the default
height is 47.5pc. 
You can change the size of the page by typing
\beginexample{}
\\pagewidth\{\Dimen\}\newline
\\pageheight\{\Dimen\}
\endexample
\noindent using suitable \Dimen{}s, where by this notation we mean
a valid \TeX{} dimension as described in the {\bf Dimensions} section of
\Joy, Chapter 20, pp.~154--55.


\subhead QED {\rm (\Joy, p.~181)}
\endsubhead

In the preprint style, \ttcs{qed} now gives an open box `$\square$',
separated from what precedes it by a quad of space.  It used to
give a solid black box.


\subhead Running Heads {\rm (new entry, \Joy, p.~183)}
\endsubhead

If you are using the preprint style, running heads similar to those in
\Joy{} will appear, with text in the center and page numbers to the
outside.  (On the first page, as usual, the running head is omitted, and
the page number is placed at the bottom.)

If you do nothing to define the text of the running heads, the author's
name will be used on the left-hand and the title on the right-hand pages.
(This is the style for papers; for monographs, see below.)
If you want some other values, say a shortened title, you
can redefine the text to appear on left- and right-hand pages by typing
\beginexample{}
\\leftheadtext\{\<left running head text>\}
\\rightheadtext\{\<right running head text>\}
\endexample

\noindent These instructions can appear anywhere in the file, but the most
common place to use them is immediately after a \ttcs{title} or
\ttcs{author} or \ttcs{head} to override the automatic running head text. 
If \ttcs{rightheadtext} or \ttcs{leftheadtext} is specified above the
topmatter, \ttcs{title} and \ttcs{author} will not override them.

If you are doing a monograph rather than a journal article, and use
the \ttcs{Monograph} switch, it affects the running heads as follows:
The chapter title appears in the left-hand running heads, and the text of
the current section heading (from \ttcs{head}) appears in the right-hand
running heads.  In chapters that don't contain any \ttcs{head}'s---for
example, a foreword---both the left- and right-hand running heads
will contain the chapter title.

By default, running heads will be uppercase.  This is a frill that
can be turned off (as it was in this Guide), by \ttcs{nofrills}, e.g.,
\beginexample{}
\\rightheadtext\\nofrills\{Text of Running Head\}
\endexample

If for some reason you don't want running heads at all, type
\ttcs{NoRunningHeads} at the beginning of the document (after the
\ttcs{documentstyle} line).  When running heads are omitted, page numbers
will appear centered at the bottom of the page.  (And even those can be
turned off using \ttcs{NoPageNumbers}.)

In a monograph, if you don't want
the text from the section \ttcs{head}'s to appear in the running heads
you must redefine the internal command, \ttcs{headmark}, that is used
by \ttcs{head} to set the right-hand running head. To do this, put the
following line in your document file, after \ttcs{Monograph} and
before \ttcs{topmatter}:
\beginexample{}
\\redefine\\headmark\#1\{\}
\endexample
\noindent (where the {\tt\#1} is an argument number as explained in \Joy,
p.~127ff).


\subhead Tables {\rm (new entry, \Joy, p.~186)}
\endsubhead

There are no special macros to support the creation of tables in \AmSTeX{}.
Table macro packages are available from other sources.
See {\bf Inserts with Captions} above for more information.


\subhead Appendix B: Bibliographies {\rm (\Joy, pp.~247--48)}
\endsubhead

The references section of a paper begins with \ttcs{Refs}.  Previously it
was not required that the end of the references be marked, and no command
existed to do so.  Although the references comprise the last section of most
papers, sometimes a survey paper will have small bibliography sections
scattered throughout.  Since \ttcs{Refs} switches to eight-point type and
makes a few other changes behind the scenes, it is hard to reestablish the
main style unless each such reference section is ended cleanly.  The
references section of a paper now needs \ttcs{endRefs} at the end.

The references are set with hanging indentation.  The amount of indentation
is preset to accommodate the most common case, two-digit numbers.
It can be increased (or decreased) by specifying the widest
number or key used in the references. For example,
\beginexample{}
\\widestnumber\\no\{999\}
\\widestnumber\\key\{GHMaR\}
\endexample
\noindent
will increase the indentation to accommodate a three-digit number or the key
\hbox{[GHMaR]} respectively.  Note that the parts of the formatting that
depend on the current journal style are taken into account automatically
(the period after a number or the [\dots]\ around a key, plus the usual
space). You can also specify \ttcs{widestnumber\\no\{9\}} to reduce the
indentation from two digits' worth to one, if your bibliography has
fewer than ten entries.

For letter labels typed using \ttcs{key} it is no longer necessary to
type anything other than the letters themselves.  A label that would
formerly have been typed \ttcs{key[\{\\bf C1\}]}
is now typed \ttcs{key C1}.

Formerly there were three ways of entering an author name:
\ttcs{by}, \ttcs{manyby}, and \ttcs{bysame}.  For consecutive references by
the same author(s), you were required to use \ttcs{manyby} for the first one,
and \ttcs{bysame} for the rest.  Now \ttcs{manyby} is obsolete, and will
result in an error message if used.  Instead, \ttcs{by} is used for the first
reference by an author, and \ttcs{bysame}, as before, is used for subsequent
ones.  The horizontal line produced by \ttcs{bysame} formerly varied
according to the typeset width of the author's name; now it has a fixed
length of three ems.

There are four new options: \ttcs{ed}, \ttcs{eds}, \ttcs{lang},
and \ttcs{transl}.

Two variations are provided for entering editor names, as with \ttcs{page}
and \ttcs{pages}, because the note ``ed.''\ or ``eds.''\ is part of the
automatic formatting.

\ttcs{lang} is used to indicate the original language for papers where
bibliographic information has been translated or there is some other reason
to believe that the original language cannot be correctly identified from
information in the reference.

In the new version of the preprint style
\ttcs{nofrills} can be used to keep automatic punctuation
from appearing.  \ttcs{nofrills\\bookinfo} suppresses the comma that would
normally precede the \ttcs{bookinfo} information, while
\ttcs{bookinfo\\nofrills...}\ suppresses the comma or other punctuation
that would normally appear at the end of the \ttcs{bookinfo} information. 
In keeping with this idea, \ttcs{finalinfo} now inserts the ending period
automatically.  It can be suppressed with \ttcs{finalinfo\\nofrills}.
Automatic punctuation will no longer be added if the pertinent field was
included but left blank.  For instance, the sequence
\beginexample{}
...
\\book Title of Book
\\bookinfo
\\publ Publisher Name
...
\endexample
\noindent
would formerly have given an extra comma after the book title.

Some examples will illustrate these new tags.  See the appendix for
samples of input and output.


%%%%%%%%%%%%%%%%%%%%%%%%%%%%%%%%%%%%%%%%%%%%%%%%%%%%%%%%%%%%%%%%%%%%%%%%

\head 3. Mathematical Constructions
\endhead

\subhead Wide Accents in Math Mode {\rm (\Joy, p.~134)}
\endsubhead

There are now wider versions of the \ttcs{widehat} and \ttcs{widetilde}
accents; they appear on lines (5) and (6):
\beginexample{}
\exbox{(1)}{\$\\hat x, \\tilde x\$} $\hat x, \tilde x$
\exbox{(2)}{\$\\widehat x, \\widetilde x\$} $\widehat x, \widetilde x$
\exbox{(3)}{\$\\widehat\{xy\}, \\widetilde\{xy\}\$} %
  $\widehat{xy}, \widetilde{xy}$
\exbox{(4)}{\$\\widehat\{xyz\}, \\widetilde\{xyz\}\$} %
  $\widehat{xyz}, \widetilde{xyz}$
\exbox{(5)}{\$\\widehat\{xyzu\}, \\widetilde\{xyzu\}\$} %
  $\widehat{xyzu}, \widetilde{xyzu}$
\exbox{(6)}{\$\\widehat\{xyzuv\}, \\widetilde\{xyzuv\}\$} %
  $\widehat{xyzuv}, \widetilde{xyzuv}$
\endexample
\noindent
These new accents are in the \filnam{msbm} family.  If \filnam{msbm}
has been loaded, \ttcs{widehat} and \ttcs{widetilde} will automatically
select these wider versions when required; otherwise, the characters
on line (4) will be the largest available.  If you are using the
preprint style, \filnam{msbm} is loaded automatically; otherwise,
see the section entitled {\bf Fonts} for instructions on loading it.


\subhead Tabs in Matrices {\rm (\Joy, pp.~175--76)}
\endsubhead

The \tab{} key can no longer be used to separate
columns of a matrix.  Only {\tt\&} can be used for that purpose.

The \tab{} is treated exactly like a space by \AmSTeX, so it is still
possible to type a matrix that will look like a matrix on your screen,
as long as you include an {\tt\&} along with the spaces.

\ttcs{enabletabs}, \ttcs{disabletabs}, \ttcs{Enabletabs} and
\ttcs{Disabletabs} are undefined in \AmSTeX~2.0 and will result in an
error message if they occur in your input file.


%%%%%%%%%%%%%%%%%%%%%%%%%%%%%%%%%%%%%%%%%%%%%%%%%%%%%%%%%%%%%%%%%%%%%%%%

\head 4. Fonts
\endhead

\subhead Additional fonts for \AmSTeX
\endsubhead

A number of new fonts have been created for use with \AmSTeX{} 2.0, both
Computer Modern fonts in sizes not previously available and new fonts
of alphabets and symbols intended to be used for mathematical notation.
These fonts are in the collection AMSFonts Version~2.0.  They must be
installed on your computer before you can use \AmSTeX's preprint style
or otherwise refer to them.
Note that AMSFonts Version~2.0 cannot be used with versions of \AmSTeX{}
earlier than Version~2.0, and vice versa.

Several of these fonts are made available automatically by the preprint
style and others can be loaded on demand.  The fonts available and the
commands used to load them are described below.

\subsubhead Fonts loaded with the preprint style
\endsubsubhead
Several fonts are loaded automatically for general use.
\roster
\item"--" \filnam{cmcsc8} is a new size of the Computer Modern small caps font.
\item"--" \filnam{cmex8} and \filnam{cmex7} are new sizes of the Computer
        Modern math extension font.  \filnam{cmex8} is used by the preprint
        style in abstracts and other eight-point environments; \filnam{cmex7}
        is used for all sub- and superscripts.
\endroster


\subsubhead Math fonts loaded with the preprint style
\endsubsubhead
\roster
\item"--" \filnam{msam} and \filnam{msbm} contain extra symbols.  The symbols
        and the names that will produce them are shown in the section 
        {\bf Symbol Names} below.  If you are not using the preprint style,
        each can be loaded separately by \ttcs{loadmsam} or \ttcs{loadmsbm}
        as appropriate.
\item"--" \filnam{eufm} is the medium-weight Euler Fraktur (German) font.
        It can also be loaded by \ttcs{loadeufm} if the preprint style is not
        being used.
\endroster


\subsubhead Math fonts loaded by \ttcs{loadbold}
\endsubsubhead
See the sections below on {\bf Bold Characters in Math Mode} and
{\bf Bold Greek Letters} for details on accessing particular characters
in these fonts.
\roster
\item"--" \filnam{cmmib} is Computer Modern bold math italic.
        It also contains bold Greek.
\item"--" \filnam{cmbsy} contains Computer Modern bold math symbols.
\endroster


\subsubhead Additional Euler fonts, for use in math
\endsubsubhead
\roster
\item"--" \filnam{eufb}, \ttcs{loadeufb}, is bold Fraktur.
\item"--" \filnam{eusm}, \ttcs{loadeusm}, is medium-weight script.
\item"--" \filnam{eusb}, \ttcs{loadeusb}, is bold script.
\item"--" \filnam{eurm}, \ttcs{loadeurm}, is medium-weight ``cursive roman.''
\item"--" \filnam{eurb}, \ttcs{loadeurb}, is bold ``cursive roman.''
\endroster


\subsubhead Considerations and warnings
\endsubsubhead
The commands to load these font files should be typed in the preamble area
between the 
\ttcs{documentstyle\{...\}} line and the \ttcs{topmatter}.
Once any of these files is loaded, it will be available
automatically for use in math mode.

\TeX{} can accommodate only sixteen font families in math mode; seven are
already defined before \AmSTeX{} begins, and the preprint style loads three
more, for a total of ten.  For this reason, you should load additional fonts
with care, requesting only those you know for certain you will need.

All the fonts described here, and some others as well, are included in the
collection AMSFonts Version~2.0, which is available from the AMS and other
distributors.  The math fonts mentioned here are all supplied in sizes from
five through ten point, suitable for use in mathematical text.


\subhead Bold Characters in Math Mode {\rm (\Joy, pp.~160--62)}
\endsubhead

Bold letters are still obtained by \ttcs{bold} as described in \Joy{}.
In addition, bold symbols, italic, and lowercase Greek can now be
obtained once \ttcs{loadbold} appears in the file.  Two control sequences
are used for different kinds of bold symbols:
\beginexample{\exboxwidth=1.25in}
\exbox{}{\\boldkey} for symbols that actually appear on the keyboard
\exbox{}{\\boldsymbol} for symbols specified by a single control sequence
\endexample
\noindent
For example,
$$\hbox{\tt\$\\bold x \\boldsymbol\\in \\boldsymbol\\varGamma\$}$$
gives
$$\bold x \boldsymbol\in \boldsymbol\varGamma$$
[and {\tt\$\\boldsymbol\\lbrack a \\boldsymbol\\rbrack\$} gives
$\boldsymbol\lbrack a \boldsymbol\rbrack$, if you need to use
\ttcs{lbrack} and \ttcs{rbrack} instead of the {\tt[} and {\tt]} keys].

More precisely, \ttcs{boldkey} can be used in math formulas in the
following combinations:
\roster
\item"$\bullet$" With any of the symbols
$$ +\ \ -\ \ =\ \ <\ \ >\ \ (\ \ )\ \ [\ \ ]\ \ |\ \ /\ \ *
    \ \ .\ \ ,\ \ :\ \ ;\ \ !\ \ ?$$
to give
$$
\boldkey+\ \ \boldkey-\ \ \boldkey=\ \ \boldkey<\ \ \boldkey>\ \ 
\boldkey(\ \ \boldkey)\ \ \boldkey[\ \ \boldkey]\ \ \boldkey|\ \ 
\boldkey/\ \ \boldkey*\ \ \boldkey.\ \ \boldkey,\ \ \boldkey:\ \ 
\boldkey;\ \ \boldkey!\ \ \boldkey?
$$
But \ttcs{bold} cannot be used to get bold versions of these symbols.
{\tt\$\\bold+\$} will give only the ordinary $+$, etc.

The bold $\boldkey+$ and $\boldkey-$ will be  binary operators,
like the ordinary $+$ and $-$ symbols;
the bold $\boldkey=$ will be a binary relation, like the ordinary $=$, etc.

\medskip
\item"$\bullet$" With letters:
\beginexample{\exboxwidth=3.75in}
\exbox{}{\$\\boldkey a\$, ..., \$\\boldkey z\$} %
        $\boldkey a, \dots, \boldkey z$
\exbox{}{\$\\boldkey A\$, ..., \$\\boldkey Z\$} %
        $\boldkey A, \dots, \boldkey Z$
\endexample
\noindent
Notice that these are {\bfit bold math italic\/} letters, as opposed to the
bold text letters $\bold a, \dots, \bold z$, $\bold A, \dots, \bold Z$ that
you get by using \ttcs{bold} in math mode.

\medskip
\item"$\bullet$" With numbers:
\beginexample{\exboxwidth=3.75in}
\exbox{}{\$\\boldkey 0\$, ..., \$\\boldkey 9\$} %
        $\boldkey 0, \dots, \boldkey 9$
\endexample
\noindent
However, these combinations simply give the same numerals that you get with
{\tt\$\\bold0\$}, \dots, {\tt\$\\bold9\$}.
\endroster

\medskip
The \ttcs{boldsymbol} construction can be used in any of the following
combinations:
\roster
\item"$\bullet$" With uppercase and lowercase Greek letters
\beginexample{\exboxwidth=3.75in}
\exbox{}{\$\\boldsymbol\\Gamma\$, ..., \$\\boldsymbol\\Omega\$} %
        $\boldsymbol\Gamma$, \dots, $\boldsymbol\Omega$
\exbox{}{\$\\boldsymbol\\varGamma\$, ..., \$\\boldsymbol\\varOmega\$} %
        $\boldsymbol\varGamma$, \dots, $\boldsymbol\varOmega$
\exbox{}{\$\\boldsymbol\\alpha\$, ..., \$\\boldsymbol\\omega\$} %
        $\boldsymbol\alpha$, \dots, $\boldsymbol\omega$
\endexample
\noindent
In previous versions of \AmSTeX, bold unslanted uppercase Greek letters
$\boldsymbol\Gamma$, \dots, $\boldsymbol\Omega$ were specified by
\ttcs{boldGamma}, \dots, \ttcs{boldOmega}; these control sequences have now
disappeared. 

\medskip
\item"$\bullet$"
For convenience, \ttcs{boldsymbol} may also be followed by a letter (but
not by a number or other character), giving the same result as
\ttcs{boldkey}.
  
\medskip
\item"$\bullet$"
You can also apply \ttcs{boldsymbol} to all the other standard symbols that
are specified by single control sequences. For example, to get bold primes:
\beginexample{\exboxwidth=3.75in}
\exbox{}{\$\\boldsymbol\\prime\$} $\boldsymbol\prime$
\exbox{}{\$\\boldsymbol A\^{ }\{\\boldsymbol\\prime\}\$} %
        $\boldsymbol A^{\boldsymbol\prime}$
\endexample
\noindent
(But \ttcs{boldsymbol'}, using the shorthand notation for \ttcs{prime},
won't work.)

\medskip
\item"$\bullet$"
You can apply \ttcs{boldsymbol} to ``delimiters,'' such as
\beginexample{\exboxwidth=3.75in}
\exbox{}{\$\\boldsymbol\\\{ ... \\boldsymbol\\\}\$} %
        $\boldsymbol\{ \dots \boldsymbol\}$
\exbox{}{\$\\boldsymbol\\langle ... \\boldsymbol\\rangle\$} %
        $\boldsymbol\langle \dots \boldsymbol\rangle$
\exbox{}{\$\char`\|, \\boldkey\char`\|, \\\char`\|, \\boldsymbol\\\char`\|\$} %
        $|,\ \boldkey|,\ \|,\ \boldsymbol\|$
\exbox{}{\$\\vert, \\boldsymbol\\vert, \\Vert, \\boldsymbol\\Vert\$} %
        $\vert,\ \boldsymbol\vert,\ \Vert,\ \boldsymbol\Vert$
\endexample
\noindent
However, you can't use \ttcs{boldsymbol} after \ttcs{left} and \ttcs{right}.
In particular, typing
\hbox{\tt\\left\\boldsymbol\char`\|\ ...\ \\right\\boldsymbol\char`\|}
will produce only error messages.

\medskip
\item"$\bullet$"
Certain symbols on the bold fonts can't be accessed at all via \ttcs{boldkey}
or \ttcs{boldsymbol}: These include bold versions 
${\fam\cmbsyfam A}$, \dots, ${\fam\cmbsyfam Z}$
 of the ``calligraphic letters'' $\Cal A$,~\dots, $\Cal Z$ that you type
as \ttcs{Cal A}, \dots, \ttcs{Cal Z},
and bold versions\linebreak {\tencmmib0}, \dots, {\tencmmib9}
 of the oldstyle numbers
{\oldnos0}, \dots, {\oldnos9} that you get with \ttcs{oldnos}.  If 
you really need to have these symbols, you will have to enlist the aid of a
\TeX{}nician, or use \ttcs{pmb}.
\endroster


\subhead Fraktur Font {\rm (\Joy, p.~162)}
\endsubhead

The German Fraktur font, which is defined only in math mode, can be made
available by typing \ttcs{loadeufm} at the top of your paper, before
\ttcs{documentstyle}.  If you are using the preprint style,
medium-weight Fraktur is loaded automatically.
To produce a Fraktur letter, type
\beginexample{\exboxwidth=3.75in}
\exbox{}{\$\\frak g\$} $\frak g$
\exbox{}{\$\\frak A\$, ..., \$\\frak Z\$} $\frak A$, \dots, $\frak Z$
\endexample


\subhead Blackboard Bold {\rm (\Joy, p.~162)}
\endsubhead

\AmSTeX{} now has a ``blackboard bold'' font, \ttcs{Bbb}.  Like \ttcs{Cal},
it will work only in math mode, and only when applied to uppercase
letters.  This alphabet is part of the \filnam{msbm} font, and can be
made available by typing \ttcs{loadmsbm} at the top of your file.  (It is
loaded automatically with the preprint style.)
\beginexample{\exboxwidth=3.75in}
\exbox{}{\$\\Bbb A, \\Bbb C, \\Bbb R\$, etc.} $\Bbb A, \Bbb C, \Bbb R$, etc.
\endexample


\subhead Poor Man's Bold {\rm (\Joy, p.~181)}
\endsubhead

\AmSTeX{} now has boldface versions of most math symbols.  However, if you
need only one or two bold symbols and have run out of \TeX{} capacity for
new fonts or font families, you can always get a poor man's bold version
of bold with \ttcs{pmb}, as described in \Joy{}.


\subhead Bold Greek Letters {\rm (\Joy, p.~255)}
\endsubhead

Bold Greek letters, both lowercase and uppercase, can be obtained by using the
\ttcs{boldsymbol} construction, as described in {\bf Bold Characters in Math
Mode}.

The old control sequences, \ttcs{boldGamma}, \dots, \ttcs{boldOmega},
have now disappeared.
The upright uppercase bold Greek letters are part of the ordinary bold
font.  However, the lowercase and slanted letters are not loaded
automatically, so you must specify \ttcs{loadbold} before using them.


%%%%%%%%%%%%%%%%%%%%%%%%%%%%%%%%%%%%%%%%%%%%%%%%%%%%%%%%%%%%%%%%%%%%%%%%

\head 5. Symbol Names
\endhead

The symbols in the \filnam{msam} and \filnam{msbm} fonts have been
assigned ``standard'' control sequence names as shown below.  All
the symbol names are loaded automatically by the preprint style; if
you are not using the preprint style, the command \ttcs{UseAMSsymbols}
will have the same effect.  (Both fonts must already have been installed
or error messages will result.)  This will add about 200 new
control sequences to \TeX's internal table.  If you are short on
space, or need only a few of the symbols, you can use a different
approach to access just the ones you need.  See the section {\bf The
\ttcs{newsymbol} command} below.


\subhead Special symbols and Blackboard Bold letters
\endsubhead

Certain symbols from the \filnam{msam} family can be specified by
control sequences that will be defined as soon as the command
\ttcs{loadmsam} has appeared in the file.

First there are four symbols that are normally used outside of math mode:
$$\vcenter{\halign to\hsize{\1{#}\hfil\tabskip\centering&
   \hbox to.5\hsize{\1{#}\hfil}\tabskip0pt\cr
checkmark&circledR\cr
maltese&yen\cr}}
$$
These symbols, like \P, \S, \dag, and \ddag, can also be used in
math mode, and will change sizes correctly in subscripts and superscripts.

Next are four symbols that are ``delimiters'' (although there are
no larger versions obtainable with \ttcs{left} and \ttcs{right}), so they
must be used in math mode:
$$\vcenter{\halign to\hsize{\1{#}\hfil\tabskip\centering&
   \hbox to.5\hsize{\1{#}\hfil}\tabskip0pt\cr
 ulcorner&urcorner\cr
 llcorner&lrcorner\cr}}$$

Finally, two dashed arrows are constructed from symbols in this family.
Note that one of them has two names; it can be accessed by either one:
$$\vcenter{\halign to\hsize{\1{#}\hfil\tabskip\centering&
   \hbox to.5\hsize{\1{#}\hfil}\tabskip0pt\cr
 \omit\hbox to.5\hsize{\hbox to\biggest{\hfil$\dashrightarrow$\hfil}\ \ %
    \ttcs{dashrightarrow}, \ttcs{dasharrow}\hss}&dashleftarrow\cr}}$$

The Blackboard Bold letters $\Bbb A, \dots, \Bbb Z$
appear in the \filnam{msbm} family.  Once \ttcs{loadmsbm} has appeared
in the file, they can be typed (in math mode) as \ttcs{Bbb A}, \dots,
\ttcs{Bbb Z}.

The \filnam{msbm} family also contains wider versions of the \ttcs{widehat}
and \ttcs{widetilde} as described in Chapter 20, ``Wide accents in math
mode.''


\subhead The \ttcs{newsymbol} command
\endsubhead

All other symbols of the \filnam{msam} and \filnam{msbm} fonts must be named
by control sequences so that they can be used (in math mode only) when the
fonts are loaded.  This can be done all at once by typing the instruction
\ttcs{UseAMSsymbols}, which will load in the file \filnam{AMSSYM.TEX}\null.
This instruction is included in the preprint style, so the names are
assigned automatically, which requires over~200 control sequences.

If you are very short on space for control sequence names, and need only
a few of these symbols, you can omit \ttcs{UseAMSsymbols}.  Instead,
assign only the names you will need by using a new \AmSTeX\ control
sequence \ttcs{newsymbol} to create a control sequence that will
properly produce this symbol.  The control sequence can be either the
``standard'' name, as listed below, or one of your own choosing.

The list of symbols below shows for each symbol the symbol itself, a
four-character~``ID,'' and the ``standard'' name of the symbol. 
(The first character of the ID identifies the font family in which a
symbol resides.  Symbols from the \filnam{msam} family have {\tt1} as the
first character; symbols from the \filnam{msbm} family have {\tt2} as the
first character.)
For example, the symbol $\nleqslant$ appears as
\medskip
\noindent\kern3\parindent\2{nleqslant}
\medskip
\noindent
To produce a control sequence with this name, the instruction
\medskip
\noindent\kern3\parindent\ttcs{newsymbol}\ttcs{nleqslant 230A}
\medskip
\noindent
appears in the file \filnam{AMSSYM.TEX}\null.  This same instruction can
be typed by a user who is not using the preprint style and has chosen not
to load all the symbol names by \ttcs{UseAMSsymbols}.  Thereafter, the
control sequence \ttcs{nleqslant} will produce the symbol $\nleqslant$
(in math mode), and will act properly as a ``binary relation.''

A few symbols in these fonts replace symbols defined in \filnam{PLAIN.TEX}
by combinations of symbols available in the Computer Modern fonts.  These
are \ttcs{angle}~($\angle$) and \ttcs{hbar}~($\hbar$) from the group
``Miscellaneous symbols,'' and \ttcs{rightleftharpoons}~($\rightleftharpoons$)
from the group ``Arrows'' below (and \Joy, p.~257).  The new symbols will
change sizes correctly in subscripts and superscripts, provided that you
are using appropriate redefinitions.  In order to use \ttcs{newsymbol} to
replace an existing definition, the name must first be ``undefined.''
Here are the lines you must put in your file if you are not using the
preprint style or \ttcs{UseAMSsymbols} (which perform the redefinition
automatically):
\medskip
\begingroup
\parindent=3\parindent
\obeylines
\ttcs{undefine}\ttcs{angle}
\ttcs{newsymbol}\ttcs{angle 105C}
\ttcs{undefine}\ttcs{hbar}
\ttcs{newsymbol}\ttcs{hbar 207E}
\ttcs{undefine}\ttcs{rightleftharpoons}
\ttcs{newsymbol}\ttcs{rightleftharpoons 130A}
\endgroup
\medskip
\noindent
These symbols are flagged in the tables below with a ``{\eightpoint(U)}''
as a reminder that they must be undefined.

Note in the tables that some symbols are shown with two names.  In such
cases, either one can be used to access the symbol.

%  since the symbol tables are set in displays, decrease the skip
%  above, so that the space between a section heading and table is
%  not so large.
\abovedisplayskip=3pt plus 3pt minus 0pt

\BBB{Lowercase Greek letters}
$$\halign{\hbox to.5\hsize{\2{#}}&\2{#}\cr
digamma&varkappa\cr}$$

\BBB{Hebrew letters}
$$\halign{\hbox to.5\hsize{\2{#}}&\2{#}\cr
beth&gimel\cr 
daleth\cr
}$$

\BBB{Miscellaneous symbols}
$$\halign{\hbox to.5\hsize{\2{#}}&\2{#}\cr
\omit\4{hbar}&backprime\cr
hslash&varnothing\cr
vartriangle&blacktriangle\cr
triangledown&blacktriangledown\cr
square&blacksquare\cr
lozenge&blacklozenge\cr
circledS&bigstar\cr
\omit\4{angle}&sphericalangle\cr
measuredangle&\omit\cr
nexists&complement\cr
mho&eth\cr
Finv&diagup\cr
Game&diagdown\cr
Bbbk&\omit\cr
}$$

\BBB{Binary operators}
$$\halign{\hbox to.5\hsize{\2{#}}&\2{#}\cr
dotplus&ltimes\cr
smallsetminus&rtimes\cr
\omit\3{Cap}{doublecap}&leftthreetimes\cr
\omit\3{Cup}{doublecup}&rightthreetimes\cr
barwedge&curlywedge\cr
veebar&curlyvee\cr
doublebarwedge\cr
boxminus&circleddash\cr
boxtimes&circledast\cr
boxdot&circledcirc\cr
boxplus&centerdot\cr
divideontimes&intercal\cr}
$$

\BBB{Binary relations}
$$\halign{\hbox to.5\hsize{\2{#}}&\2{#}\cr
leqq&geqq\cr
leqslant&geqslant\cr
eqslantless&eqslantgtr\cr
lesssim&gtrsim\cr
lessapprox&gtrapprox\cr
approxeq\cr
lessdot&gtrdot\cr
\omit\3{lll}{llless}&\omit\3{ggg}{gggtr}\cr
lessgtr&gtrless\cr
lesseqgtr&gtreqless\cr
lesseqqgtr&gtreqqless\cr
\omit\3{doteqdot}{Doteq}&eqcirc\cr
risingdotseq&circeq\cr
fallingdotseq&triangleq\cr
backsim&thicksim\cr
backsimeq&thickapprox\cr
subseteqq&supseteqq\cr
Subset&Supset\cr
sqsubset&sqsupset\cr
preccurlyeq&succcurlyeq\cr
curlyeqprec&curlyeqsucc\cr
precsim&succsim\cr
precapprox&succapprox\cr
vartriangleleft&vartriangleright\cr
trianglelefteq&trianglerighteq\cr
vDash&Vdash\cr
Vvdash\cr
smallsmile&shortmid\cr
smallfrown&shortparallel\cr
bumpeq&between\cr
Bumpeq&pitchfork\cr
varpropto&backepsilon\cr
blacktriangleleft&blacktriangleright\cr
therefore&because\cr}$$
\bigbreak
\BBB{Negated relations}
$$\halign{\hbox to.5\hsize{\2{#}}&\2{#}\cr
nless&ngtr\cr
nleq&ngeq\cr
nleqslant&ngeqslant\cr
nleqq&ngeqq\cr
lneq&gneq\cr
lneqq&gneqq\cr
lvertneqq&gvertneqq\cr
lnsim&gnsim\cr
lnapprox&gnapprox\cr
nprec&nsucc\cr
npreceq&nsucceq\cr
precneqq&succneqq\cr
precnsim&succnsim\cr
precnapprox&succnapprox\cr
nsim&ncong\cr
nshortmid&nshortparallel\cr
nmid&nparallel\cr
nvdash&nvDash\cr
nVdash&nVDash\cr
ntriangleleft&ntriangleright\cr
ntrianglelefteq&ntrianglerighteq\cr
nsubseteq&nsupseteq\cr
nsubseteqq&nsupseteqq\cr
subsetneq&supsetneq\cr
varsubsetneq&varsupsetneq\cr
subsetneqq&supsetneqq\cr
varsubsetneqq&varsupsetneqq\cr}$$

\overfullrule=0pt

\BBB{Arrows}
$$\halign{\hbox to.5\hsize{\2{#}}&\2{#}\cr
leftleftarrows&rightrightarrows\cr
leftrightarrows&rightleftarrows\cr
Lleftarrow&Rrightarrow\cr
twoheadleftarrow&twoheadrightarrow\cr
leftarrowtail&rightarrowtail\cr
looparrowleft&looparrowright\cr
leftrightharpoons&\omit\4{rightleftharpoons}\cr
curvearrowleft&curvearrowright\cr
circlearrowleft&circlearrowright\cr
Lsh&Rsh\cr
upuparrows&downdownarrows\cr
upharpoonleft&\omit\3{upharpoonright}{restriction}\cr
downharpoonleft&downharpoonright\cr
multimap&rightsquigarrow\cr
leftrightsquigarrow\cr}$$

\BBB{Negated arrows}
$$\halign{\hbox to.5\hsize{\2{#}}&\2{#}\cr
leftarrow&nrightarrow\cr
nLeftarrow&nRightarrow\cr
nleftrightarrow&nLeftrightarrow\cr}$$


%%%%%%%%%%%%%%%%%%%%%%%%%%%%%%%%%%%%%%%%%%%%%%%%%%%%%%%%%%%%%%%%%%%%%%%%

\head 6. Other Things You Ought to Know
\endhead

\subhead Errata to \JoT{} prior to \AmSTeX{} 2.0
\endsubhead

The file \filnam{JOYERR.TEX} contains the full list of corrections to
\Joy\ that preceded the release of \AmSTeX{} Version 2.0.  A user who
desires a typeset copy of this file may run it through \TeX{} and print
out the \filnam{.dvi} file.  This will require Version~2.0 of \AmSTeX{}
and \filnam{AMSPPT.STY}, and also AMSFonts Version~2.0.


\subhead Acknowledging the use of \AmSTeX{}
\endsubhead

The following are suggested as appropriate statements of acknowledgment
that \AmSTeX{} has been used to format a document for publication.

A single paper should include the following at the bottom of the first
page:
\beginexample{}
\rm{}Typeset by \AmSTeX{}
\endexample
\noindent
(This notation is provided automatically by the \AmSTeX{} preprint style.)

If an entire journal or book is prepared with \AmSTeX{}, the following
statement should appear on its copyright page:
\beginexample{}
\rm{}This [journal/book] was typeset by \AmSTeX{}, the \TeX{} macro %
system of the \AMS{}.
\endexample

If only selected papers in a journal or book are set with \AmSTeX{}, these
papers should be identified as shown above, and the following should
appear on the copyright page:
\beginexample{}
\rm{}\AmSTeX{} is the \TeX\ macro system of the \AMS{}.
\endexample


%%%%%%%%%%%%%%%%%%%%%%%%%%%%%%%%%%%%%%%%%%%%%%%%%%%%%%%%%%%%%%%%%%%%%%%%

\head 7. Getting Help
\endhead

If you should find any bugs in the macros or documentation,
send a Problem Report to:
\beginexample{\rm}
Technical Support Group
\AMS{}
P. O. Box 6248
Providence, RI 02940
\vskip 2pt %
Phone: 800-321-4AMS \quad or \quad 401-455-4080
Internet: tech-support\@Math.AMS.com
\endexample

A Problem Report should contain the following information:
\roster
\item version of \filnam{AMSTEX.TEX} and of \filnam{AMSPPT.STY} with which
  the problem occurred;
\item a detailed description of the problem, including the input code for
  one or more examples that illustrate the problem;
\item a log file of a \TeX{} session showing the problem.
\endroster


%%%%%%%%%%%%%%%%%%%%%%%%%%%%%%%%%%%%%%%%%%%%%%%%%%%%%%%%%%%%%%%%%%%%%%%%

\newpage

%  Arrange for the sample references to be set broadside, with the output
%  pasted up next to the corresponding input.  The section heading should
%  be full-width, and the running heads should be the normal page width.
%  To accomplish the latter, we must redefine plain's \makeheadline.

\catcode`\@=11
\newdimen\headlinewidth
\headlinewidth=\hsize
\def\makeheadline{\vbox to\z@{\vskip-22.5\p@
  \hbox to\headlinewidth{\vbox to8.5\p@{}\the\headline}\vss}\nointerlineskip}
\catcode`\@=\active

\pagewidth{47.5pc}
\pageheight{30pc}

\head Appendix.\quad Sample Bibliography Input and Output
\endhead

\pagewidth{23pc}
\beginexample{\exindent=0pt}
\\Refs
\\ref\\no 4
\\by V. I. Arnol\$'\$d, A. N. Varchenko,
\ and S. M. Guse\\u\\i n-Zade
\\book Singularities of differentiable maps.~{\\rm I}
\\publ ``Nauka'' \\publaddr Moscow \\yr 1982
\\lang Russian
\\endref
\ {}
\\ref\\no 5\\bysame 
\\book Singularities of differentiable maps.~{\\rm II}
\\publ ``Nauka'' \\publaddr Moscow \\yr 1984
\\lang Russian
\\endref
\ {}
\\ref\\no 6
\\by O. A. Ladyzhenskaya
\\book Mathematical problems in the dynamics
\ of a viscous incompressible fluid 
\\bookinfo 2nd rev. aug. ed.
\\publ ``Nauka'' \\publaddr Moscow \\yr 1970
\\lang Russian
\\transl English transl. of 1st ed.
\\book The mathematical theory of viscous
\ incompressible flow
\\publ Gordon and Breach \\publaddr New York
\\yr 1963; rev. 1969
\\endref
\endexample

\newpage

\beginexample{\exindent=0pt}
\\ref\\no 7
\\by P. D. Lax and C. D. Levermore
\\paper The small dispersion limit for the
\ KdV equation.~{\\rm I}
\\jour Comm. Pure Appl. Math. \\vol 36 \\yr 1983
\\pages 253--290 \\nofrills\\finalinfo (overview)
\\moreref\\paper {\\rm II}
\\jour Comm. Pure Appl. Math. 
\\vol 36 \\yr 1983 \\pages 571--594
\\moreref\\paper {\\rm III}
\\jour Comm. Pure Appl. Math. 
\\vol 36 \\yr 1983 \\pages 809--829 \\endref
\ {}
\\ref\\no10 \\by S. Osher
\\paper Shock capturing algorithms for equations of
\ mixed type
\\inbook Numerical Methods for Partial Differential
\ Equations \\eds S. I. Hariharan and T. H. Moulton
\\publ Longman \\publaddr New York \\yr 1986
\\pages 305--322
\\endref
\ {}
\\ref\\no 17 \\by G. S. Petrov
\\paper Elliptic integrals and their nonoscillatory
\ behavior
\\jour Funktsional. Anal. i Prilozhen.
\\vol 20 \\yr 1986 \\pages 46--49
\\transl\\nofrills English transl. in
\\jour Functional Anal. Appl. \\vol 20\\yr 1986
\\endref
\endexample

\newpage

\beginexample{\exindent=0pt}
\\widestnumber\\key\{GHMaR\}
\ {}
\\ref\\key C1
\\by B. Coomes
\\book Polynomial flows, symmetry groups, and
\ conditions sufficient for injectivity of maps
\\bookinfo Ph.D. thesis, Univ. Nebraska--Lincoln
\\yr 1988
\\endref
\ {}
\\ref\\key C2
\\bysame \% B. Coomes
\\paper The Lorenz system does not have a
\ polynomial flow
\\jour J. Differential Equations
\\toappear
\\endref
\ {}
\\ref\\key GHMaR
\\by J. Guckenheimer, P. Holmes, M. Martineau,
\ and L. P. Robinson
\\book Nonlinear oscillations, dynamical systems,
\ and bifurcations of vector fields
\\bookinfo
\\publ Springer-Verlag \\publaddr New York
\\yr 1983
\\endRefs
\endexample

\newpage

\begingroup
\aboveheadskip=\abovedisplayskip

\Refs
\ref\no 4
\by V. I. Arnol$'$d, A. N. Varchenko, and S. M. Guse\u\i n-Zade
\book Singularities of differentiable maps.~{\rm I}
\publ ``Nauka'' \publaddr Moscow
\yr 1982
\lang Russian
\endref

\ref\no 5
\bysame 
\book Singularities of differentiable maps.~{\rm II}
\publ ``Nauka'' \publaddr Moscow
\yr 1984
\lang Russian
\endref

\ref\no 6
\by O. A. Ladyzhenskaya
\book Mathematical problems in the dynamics of a
 viscous incompressible fluid 
\bookinfo 2nd rev. aug. ed.
\publ ``Nauka'' \publaddr Moscow
\yr 1970
\lang Russian
\transl English transl. of 1st ed.
\book The mathematical theory of viscous
incompressible flow
\publ Gordon and Breach \publaddr New York
\yr 1963; rev. 1969
\endref

\bigskip

\ref\no 7
\by P. D. Lax and C. D. Levermore
\paper The small dispersion limit for the KdV equation.~{\rm I}
\jour Comm. Pure Appl. Math. 
\vol 36 \yr 1983 \pages 253--290
\nofrills\finalinfo (overview)
\moreref\paper {\rm II}
\jour Comm. Pure Appl. Math. 
\vol 36 \yr 1983 \pages 571--594
\moreref\paper {\rm III}
\jour Comm. Pure Appl. Math. 
\vol 36 \yr 1983 \pages 809--829 \endref

\ref\no10 \by S. Osher
\paper Shock capturing algorithms for equations of mixed type
\inbook Numerical Methods for Partial Differential Equations
\eds S. I. Hariharan and T. H. Moulton
\publ Longman \publaddr New York \yr 1986 \pages 305--322
\endref

\ref\no 17 \by G. S. Petrov
\paper Elliptic integrals and their nonoscillatory behavior
\jour Funktsional. Anal. i Pri\-lo\-zhen.
\vol 20 \yr 1986 \pages 46--49
\transl\nofrills English transl. in \jour Functional Anal. Appl.
\vol 20\yr 1986
\endref

\bigskip

\widestnumber\key{GHMaR}

\ref\key C1
\by B. Coomes
\book Polynomial flows, symmetry groups, and conditions
 sufficient for injectivity of maps
\bookinfo Ph.D. thesis, Univ. Nebraska--\penalty0\hskip0pt Lincoln
\yr 1988
\endref

\ref\key C2
\bysame % B. Coomes
\paper The Lorenz system does not have a polynomial flow
\jour J. Differential Equations
\toappear
\endref

\ref\key GHMaR
\by J. Guckenheimer, P. Holmes, M. Martineau, and L. P. Robinson
\book Nonlinear oscillations, dynamical systems, and
 bifurcations of vector fields
\bookinfo
\publ Springer-Verlag \publaddr New York
\yr 1983
\endref
\endRefs

\endgroup       % end special value of \aboveheadskip


\enddocument

%%%%%%%%%%%%%%%%%%%%%%%%%%%%%%%%%%%%%%%%%%%%%%%%%%%%%%%%%%%%%%%%%%%%%%%%
