%% @texfile{
%%     filename="amsguide.tex",
%%     version="2.1b",
%%     date="24-DEC-1991",
%%     filetype="AMS-TeX: user documentation",
%%     copyright="Copyright (C) American Mathematical Society,
%%            all rights reserved.  Copying of this file is
%%            authorized only if either:
%%            (1) you make absolutely no changes to your copy 
%%                including name; OR
%%            (2) if you do make changes, you first rename it to some
%%                other name.",
%%     author="American Mathematical Society",
%%     address="American Mathematical Society,
%%            Technical Support Department,
%%            P. O. Box 6248,
%%            Providence, RI 02940,
%%            USA",
%%     telephone="401-455-4080 or (in the USA) 800-321-4AMS",
%%     email="Internet: Tech-Support@Math.AMS.org",
%%     codetable="ISO/ASCII",
%%     checksumtype="line count",
%%     checksum="1983",
%%     keywords="amstex, ams-tex, tex",
%%     abstract="This file is the User's Guide describing the use of
%%       AMS-TeX 2.0+. Installation instructions are found in appendixes
%%       B and C."
%%     }
%
%      In addition to plain TeX and standard CM fonts, TeXing this file
%      requires the following files and fonts:
%
% AMSTEX.TEX (version 2.1)
% AMSPPT.STY (version 2.1)
% AMSSYM.TEX (loaded by AMSPPT.STY)
% MSAM10
% MSBM10
% EUFM10
% CMEX8
% CMEX7
% CMBSY7
% CMCSC8
%
%%%%%%%%%%%%%%%%%%%%%%%%%%%%%%%%%%%%%%%%%%%%%%%%%%%%%%%%%%%%%%%%%%%%%%%%

\input amstex
\documentstyle{amsppt}

\font\tenss=cmss10

\long\def\usertype#1{\smallskip \moveright2pc\vbox{\def\par{\crcr}\halign{%
   \setbox0\hbox{\tt##}%
   \hbox\ifdim\wd0<10pc to10pc\fi{\unhbox0\hfil}%
        \kern1pc \it $\langle$return$\rangle$\hss
   \cr#1\crcr}}%
  \smallskip}
\long\def\systype#1{{\rightskip=4pc\leftskip=4pc\noindent\tt #1\par}}

%      Change default dimensions and fonts
\hfuzz1pc % to suppress reporting of overfull boxes.
\aboveheadskip=3\bigskipamount
\belowheadskip=\medskipamount
\subheadskip=\bigskipamount
\addto\tenpoint{\abovedisplayskip=6pt plus2pt minus3pt
    \belowdisplayskip=\abovedisplayskip}
\loadbold
%      Prevent hyphenation of "amsppt":
\hyphenation{amsppt}

%      Macros for text substitution and for presentation of examples.
%
\define\Textures{{\it Textures\/}}
\define\AMS{American Mathematical Society}
\define\JAMS{{\it Journal of the \AMS}}
\define\JoT{{\it The Joy of \TeX{}}}
\define\Joy{{\it Joy}}

%      Notice that the LaTeX name is always roman: it does not change
%      font if it occurs inside italic text.  This is the way it is
%      defined by latex.tex.  We substitute a subscript size A instead
%      of a small caps A, however, to avoid possible font problems for
%      users who don't happen to have odd sizes or magnifications of
%      cmcsc10 available.
\define\LaTeX{{\rm L\kern-.36em\raise.3ex\hbox{\the\scriptfont0 A}\kern-.15em
    T\kern-.1667em\lower.7ex\hbox{E}\kern-.125emX}}

\def\filnam#1{{\tt\ignorespaces#1\unskip}}
\hyphenchar\tentt=-1 % to prohibit hyphenation in tt text

\newdimen\exindent   \exindent=2\parindent
% Add a high penalty to discourage line breaks within an example
% without absolutely prohibiting them.
{\obeylines
 \gdef^^M{\par\penalty9999}%
 \gdef\beginexample#1{\medskip\bgroup %
   \def~{\char`\~}%
   \NoBlackBoxes\tt\frenchspacing %
   \parindent=0pt#1\leftskip=\exindent\obeylines}
}%  end \obeylines
\def\endexample{\endgraf\egroup\medskip}
\newdimen\exboxwidth
\exboxwidth=3in
\def\exbox#1#2{\noindent \hangindent=\exboxwidth
  \leavevmode\llap{\null\rm#1\unskip\enspace}%
  \hbox to\exboxwidth{\tt\ignorespaces#2\hss}\rm\ignorespaces}
\chardef\\=`\\       \chardef\{=`\{       \chardef\}=`\}
\def\<#1>{{\it$\langle$#1\/$\rangle$}}
\def\Dimen{\<dimen>}

\catcode`\@=11
\def\cs#1{\leavevmode
%      Save the previous skip and put it back after the penalty 0
%      so that the penalty0 won't cause a blank at the end of a line.
  \skip@\lastskip\unskip\penalty\z@
  \ifdim\skip@>\z@ \hskip\skip@\fi
  {\tt\char`\\\ignorespaces#1\unskip}}

%  Redefine the \subhead macro to be on a line by itself and omit period.
\outer\def\subhead#1\endsubhead{\par\penaltyandskip@{-100}\subheadskip
  \noindent{\subheadfont@\ignorespaces#1\unskip\endgraf}\nobreak\noindent}

%  Define macros for presentation of tables of symbols.
\def\BBB#1{\par\bigbreak
  \leavevmode\llap{$\bullet$\enspace}{\bf#1}}
\newdimen\biggest
\setbox0\hbox{$\dashrightarrow$}\biggest=\wd0
\def\1#1{\hbox to\biggest{\hfill$\csname#1\endcsname$\hfill}\ \ %
  \cs{#1}}
\def\fudge{\hbox to\biggest{}\ \ \hphantom{\tt\char'134 }}

\def\getID@#1{\edef\next@{\expandafter\meaning\csname#1\endcsname}%
 \expandafter\getID@@\next@0\getID@@}
\def\getID@@#1"#2#3#4#5#6\getID@@{\def\next@{#6}%
  \ifx\next@\empty
   \def\next@{#2}%
    \ifx\next@\msafam@
     \def\ID@{10#3#4}%
    \else
     \def\ID@{20#3#4}%
    \fi
  \else 
   \def\next@{#3}%
    \ifx\next@\msafam@
     \def\ID@{1#2#4#5}%
    \else
     \def\ID@{2#2#4#5}%
    \fi
  \fi}
\def\2#1{\hbox to.5\hsize
  {\hbox to\biggest{\hfill$\csname#1\endcsname$\hfill}\ \ %
    \getID@{#1}{\tt\ID@}\ \ \cs{#1}\hfill}}
\def\3#1#2{\hbox to.5\hsize
  {\hbox to\biggest{\hfil$\csname#1\endcsname$\hfil}\ \ %
    \getID@{#1}{\tt\ID@}\ \ \cs{#1}, \cs{#2}\hss}}
\def\4#1{\hbox to.5\hsize
  {\hbox to\biggest{\hfill$\csname#1\endcsname$\hfill}\ \ %
    \getID@{#1}{\tt\ID@}\ \ \cs{#1}\ \ {\eightpoint(U)}\hfill}}

\catcode`\@=\active

\define\thismonth{\ifcase\month % case 0 --- impossible!
  \or January\or February\or March\or April\or May\or June%
  \or July\or August\or September\or October\or November%
  \or December\fi}

%%%%%%%%%%%%%%%%%%%%%%%%%%%%%%%%%%%%%%%%%%%%%%%%%%%%%%%%%%%%%%%%%%%%%%%%
%% TITLE PAGE

\font\fourtn=cmr10 scaled \magstep2
\font\fourtnsy=cmsy10 scaled \magstep3
\font\fourtnbf=cmbx10 scaled \magstep3
\textfont2=\fourtnsy

\shipout\vbox to\vsize{%
\parindent=0pt 
\vskip5pc
\rightline{\fourtnbf User's Guide to \AmSTeX{}}
\bigskip
\rightline{\fourtn Version 2.1}
\medskip
\rightline{\fourtn August 1991}

\vfill

\tenpoint

This publication was typeset using \AmSTeX{}, the American
Mathematical\newline \quad Society's \TeX{} macro system.

Copyright \copyright{} 1991 by the \AMS{}.

All rights reserved. Any material in this guide may be reproduced or
duplicated for personal or educational use.

\medskip
\begingroup\obeylines
IBM PC is a registered trademark of International Business Machines, Inc.
Personal \TeX{} and PC\TeX{} are registered trademarks of Personal \TeX{}, Inc.
\Textures{} is a trademark of Blue Sky Research.
Macintosh is a trademark of Apple Computer, Inc.
\TeX{} is a trademark of the \AMS{}.
\endgroup
}% End of vbox being shipped out

%% END TITLE PAGE

%%%%%%%%%%%%%%%%%%%%%%%%%%%%%%%%%%%%%%%%%%%%%%%%%%%%%%%%%%%%%%%%%%%%%%%%
\topmatter
\title\nofrills User's Guide to \AmSTeX{} Version 2.1\endtitle

\date {\thismonth} {\number\year}\enddate

\toc
\widestnumber\head{7}
\head 1. Overview\endhead
\head 2. Formatting Features\endhead
\head 3. Mathematical Constructions\endhead
\head 4. Fonts\endhead
\head 5. Symbol Names\endhead
\head 6. Other Things You Ought to Know\endhead
\head 7. Getting Help\endhead
\head {} References\endhead
\head {} Appendix A. Sample Bibliography Input and Output\endhead
\head {} Appendix B. Installation Procedures -- PC\endhead
\head {} Appendix C. Installation Procedures -- Macintosh\endhead
\endtoc

\endtopmatter
\document

%%%%%%%%%%%%%%%%%%%%%%%%%%%%%%%%%%%%%%%%%%%%%%%%%%%%%%%%%%%%%%%%%%%%%%%%

\head 1. Overview\endhead

\AmSTeX{} is a macro package for \TeX{}, designed to simplify the input
of mathematical material and format the output according to preset style
specifications.  Although the \AMS{} holds the copyright for \AmSTeX{},
its use is not restricted, but is encouraged for the preparation of
manuscripts intended for publication both in the Society's books and
journals, and also in other mathematical literature.  In recognition of
the copyright, the Society requests that published documents prepared
with \AmSTeX{} include an acknowledgment of its use.  The suggested
forms for acknowledgments are given in the section {\bf Other Things You
Ought to Know}.

Version 2.0 of \AmSTeX{} contained numerous minor improvements
and bug fixes, as well as some major changes involving additional fonts.
This User's Guide describes all the new and changed features and how
to use them, with further additions and revisions pertaining
to version 2.1.  Topics are grouped by type, and then presented in roughly
the same order as they appear in \JoT{}.

This User's Guide assumes that you already have a copy of \JoT{}. It
contains references to specific sections that won't help you much if you
don't have a copy. It also assumes for the most part that you will be
using the ``preprint style,'' a set of macros that provides features
specific to the formatting of a document, such as headings, page
numbers, and the like. If you are planning to use the preprint style,
you will also need to have a copy of AMSFonts Version~2.1.  \JoT{} and
AMSFonts 2.1 are available from the \AMS{} and other distributors.

\subhead Files Comprising the \AmSTeX{} Version~2.1 package
\endsubhead

The following files are contained in the \AmSTeX{} Version~2.1 package
distributed by the \AMS{}:
\medskip
\settabs\+\indent&\filnam{AMSGUIDE.TEX}\qquad&\kern.6\hsize\cr
\+&\filnam{AMSTEX.TEX}&
        the \AmSTeX{} Version 2.1 macros\cr
\+&\filnam{AMSSYM.TEX}&
        macros defining the symbols in fonts \filnam{msam} and
       \filnam{msbm}\cr
\+&\filnam{AMSPPT.STY}&
       the preprint style for \AmSTeX{} Version 2.1\cr
\+&\filnam{AMSPPT.DOC}&
       technical documentation for \filnam{AMSPPT.STY}\cr
\+&\filnam{AMSGUIDE.TEX}&
       the source file for this User's Guide\cr
\+&\filnam{AMSPPT1.TEX}&
       a backward compatibility file for use with documents\cr
\+&&   already completed using \AmSTeX{} versions earlier than 2.0\cr
\+&\filnam{JOYERR.TEX}&
       errata to \JoT{} (first edition)\cr
\+&\filnam{*.TFM}     &
       TFM files for AMSFonts Version 2.1\cr
\+&\filnam{AMSTEX.INI}&
       Used in creating format files\cr
\+&\filnam{AMSTEX.BAT}&
       DOS installations only\cr
\medskip

The file \filnam{AMSPPT.DOC} is an {\smc ascii} file, and is
not intended to be processed with \TeX{}\null.  This  documentation file
is arranged in the same order as the macro file that it describes, and
explains the intent and mechanics of the macros in detail.  A separate
file (\filnam{AMSTEX.DOC}), documenting the file \filnam{AMSTEX.TEX},
is available on request.

In addition, other files are used during installation from diskettes.
For instructions on installing the \AmSTeX{} macros and preprint style,
see Appendixes B and C.  These appendixes describe the
installation process for two common systems: PC\TeX{} on an IBM PC or
compatible, and \Textures{} on a Macintosh.  They also give suggestions
for installation on other systems.

\subhead General Description of Changes\endsubhead

\AmSTeX{} 2.0+, the preprint style, and their technical
documentation are the result of a joint effort begun by Michael Spivak
and extended by the Composition Technical Support group of the \AMS{}.

In version 2.0 of \AmSTeX{}, the following changes were made:
\widestnumber\item{10}
\roster 
\item All known bugs were eliminated.
\item Messages were added identifying the current versions of
  \filnam{AMSTEX.TEX} and \filnam{AMSPPT.STY}, to be displayed on your
  terminal screen and in the log file.
\item Some error and help messages were changed for the sake of
  clarity or to provide more information.
\item Refinements were made to conserve memory space.
\item The CM versions of the Computer Modern fonts were substituted for the
  older AM versions.
\item Support for additional fonts was added.

  \itemitem{(a)} \filnam{AMSTEX.TEX} provides the mechanism for accessing
    the Euler and extra symbol fonts of the AMSFonts collection.
  \itemitem{(b)} The preprint style assumes that fonts \filnam{msam},
    \filnam{msbm}, and \filnam{eufm} are installed and available.

\item Changes were made to the preprint style to make it conform
  more closely to the style of AMS publications, in particular, the \JAMS.

  \itemitem{(a)} Running heads were made automatic; they can be suppressed
    if desired.
  \itemitem{(b)} Additional elements are recognized in both the top matter
    and the body of a document, and the input syntax was regularized.
  \itemitem{(c)} Footnotes were changed to have normal indentation.%
       \footnote{Like this.}
  \itemitem{(d)} The style of the references was changed considerably.

\item The ability to produce roman-numeral page numbers using the
  plain \TeX{} convention (negative \cs{pageno}) was added.
\item In the preprint style, mathematics-oriented hyphenation exceptions
  were added.  (These follow American, not British, rules.)
\item An option was added in the preprint style that
  allows documents to be formatted
  as chapters of a monograph rather than as separate papers.
\item Finally, some optional formatting features requested by
  \AmSTeX{} users were added to the preprint style.
\endroster

\medskip
In version 2.1, the following changes were made:
\roster
\item \cs{curraddr} and \cs{rom} were added.
\item Additional error conditions were identified and supplied with more
  informative messages.
\item A sporadic line-breaking problem in the preprint style references
  section was remedied.
\item Some bugs introduced into the preprint style during the
 version 2.0 changes were found and eliminated.
\item Use of the Euler fonts other than medium Euler Fraktur was
  made more convenient.
\item The ability to use multiple \cs{thanks} commands was added.
\item The installation instructions were revised and augmented.
\item The backward compatibility file \filnam{AMSPPT1.TEX} was added.
\endroster

\subhead This User's Guide\endsubhead

This User's Guide has been prepared using \AmSTeX{} Version~2.1 with the
preprint style.  Some changes have been made: font and dimension settings
have been reset, the macros for headings have been redefined to produce a
result more suited to documentation, and some {\it ad hoc\/} macros have
been defined to simplify the presentation of particular information.
However, in general, this document and the file from which it was produced
illustrate the general appearance and input for a preprint with running
heads.  Printing the output of \TeX{} for this Guide requires AMSFonts
Version~2.0 or later
(users of AMSFonts Version~2.0 are strongly encouraged to upgrade to
Version~2.1).

%%%%%%%%%%%%%%%%%%%%%%%%%%%%%%%%%%%%%%%%%%%%%%%%%%%%%%%%%%%%%%%%%%%%%%%%

\head 2. Formatting Features
\endhead

Formatting documents prepared with \AmSTeX{} is accomplished by a
``style file.''  The features described here are part of the preprint
style.  The \AmSTeX{} preprint style, Version~2.1, will format an input
file in a manner suitable for a paper in a journal, unless  the style of
a \cs{Monograph} is explicitly selected.  Unless noted otherwise, 
the journal style is the style described below.

\setbox1=\hbox{\tt\\dedicatory...\\enddedicatory\ }
\setbox0=\vbox{\hsize=\wd1\parindent=0pt\tt\obeylines
  \strut\\title...\\endtitle
  \\author...\\endauthor
  \\affil...\\endaffil
  \\address...\\endaddress
  \\curraddr...\\endcurraddr
  \\email...\\endemail
  \\dedicatory...\\enddedicatory
  \\date...\\enddate
  \\thanks...\\endthanks
  \\translator...\\endtranslator
  \\keywords...\\endkeywords
  \\subjclass...\\endsubjclass
  \strut\\abstract...\\endabstract\endgraf}
\setbox1=\hbox{$\dsize\left\lbrace\,\vcenter{\vphantom{\copy0}}\right.
   \nulldelimiterspace=0pt$}%
\setbox2=\hbox{\kern-\wd1$\dsize\left\lbrace\,\vcenter{\copy0}\right\rbrace$}

\subhead Top Matter\endsubhead

Some commands affect the appearance of a whole document.  Such commands
should go at the top of your input file, right after the \cs{documentstyle}
line and before the \cs{topmatter} line.  This area will be referred to as
the ``preamble.''  Commands that should be in the preamble include 
\cs{define}, \cs{TagsOnRight} and the like,
\cs{NoPageNumbers}, \cs{NoRunningHeads}, \cs{Monograph},
\cs{pagewidth}, \cs{pageheight}, \cs{pageno},
and commands that load fonts.
The preamble is relevant to any document, whether paper or monograph.

The beginning of an
\AmSTeX{} file should look something like this:
\beginexample{}
\\input amstex
\\documentstyle\{...\}
\medskip
\<preamble commands, such as \cs{define}, \cs{pageno}, \cs{Monograph}, 
\leavevmode\hbox to20pt{}\cs{NoRunningHeads}, \cs{loadbold}, etc.>
\medskip
\\topmatter
\indent\box2
\strut\\endtopmatter
\\document
\endexample

If any \cs{end...} tag is omitted (or misspelled), an error message will
appear at \cs{endtopmatter} or at the next blank line: for example, if
you misspell \cs{endtitle}, the message will be something like ``{\tt
!~Paragraph ended before \cs{title} was
com\discretionary{-}{}{}plete}.'' If you omit \cs{endtopmatter}, there
won't be any error message, but none of the topmatter material will
print.{\tolerance2000\par}

If you have documents that were prepared for versions of \AmSTeX{}
earlier than version 2.0, you may find the file \filnam{amsppt1.tex}
useful.  By including the line \cs{input amsppt1} immediately after the
\cs{documentstyle} line, the topmatter commands and the sectioning
commands that changed in form will work in the original way. Other than
that, there should be few incompatibility problems with previous
versions. Note: The use of \filnam{amsppt1.tex} is discouraged except
for processing preexisting files.

For multi-line titles, affiliations, authors, or dedications (basically
everywhere that lines are centered individually rather than being set in
paragraphs), line breaks are obtained by using \cs{\\}.  In other parts
of the topmatter, which are set in paragraph form, line breaks are
obtained by \cs{linebreak}.

The title will be set in uppercase.  To turn off the automatic
uppercasing, use the \cs{nofrills} option:
\cs{title\\nofrills...\\endtitle}.

Electronic mail addresses can be keyed using \cs{email}\dots\cs{endemail}.
Every \cs{email} address must be preceded by a regular \cs{address},
otherwise the e-mail address will not print.
Multiple \cs{email} addresses may be used, but each
\cs{email}\dots\cs{endemail} must be paired with the \cs{address}
of the same author.  The \cs{email} address will be printed at the end
of the paper, as ``{\it E-mail address:\/} \<net-address>,'' following
the address with which it is paired.

Normally, the address given in \cs{address} is 
the address of the author at the time the research
was being done; if the author's address at the time of
publication is different, the current address should be
given in \cs{curraddr}.  This should be entered between
\cs{address} and \cs{email} in the document file.
Like \cs{email}, multiple \cs{curraddr}'s can be used, if
each one is keyed after the \cs{address} of the author to which it
pertains.  If it is not preceded by an \cs{address}, the
current address will not print.

The \cs{dedicatory} command is used for such things as
``Dedicated to Professor X on the occasion of his eightieth birthday.''
The dedication will appear in italics, before the abstract.

The \cs{thanks} command is provided for acknowledgments of grant support
and other kinds of support for the author's research, or other general
information not covered by one of the predefined tags such as
\cs{keywords} or \cs{subjclass}. The information will be printed as an
unnumbered footnote at the bottom of the first page.  Like \cs{address},
\cs{thanks} can be used more than once.

In case a paper has not only an author but a translator, \cs{translator}
is provided.  This information will be printed at the end of the paper in
eight-point roman, as ``Translated by'' followed by the translator's name
in uppercase.

The information for \cs{keywords} and \cs{subjclass} appears as
unnumbered footnotes at the foot of the first page, as in AMS journals.
In a monograph chapter they will not print at all, since they should be
handled separately, as part of the front matter for the monograph.

The abstract heading ``{\eightpoint\smc Abstract.}'' appears in caps
and small caps, in the same size (eight point) as the abstract itself.

A simple table of contents setup is available.  Tables of contents
are typed in the topmatter along with everything else (except for
monographs---see the section {\bf Book Formatting} below), using
\cs{toc...\\endtoc}.
\beginexample{}
\\toc
\\specialhead...\\endspecialhead
\\head...\\endhead
\\subhead...\\endsubhead
\\subsubhead...\\endsubsubhead
\\endtoc
\endexample
\noindent
The syntax of the parts is identical to the syntax used
for headings within the document (see the sections {\bf Headings} and
{\bf Book Formatting} below), so that for those who wish to do so
and have a capable text editor, the table of contents can be constructed
by extracting the relevant lines from the main text.\footnote{But note that
the original line breaks in multi-line headings would not be appropriate
for the table of contents, so you'd want to remove any \cs{\\}'s that
might be present.}

Page numbers aren't usually appropriate for the short table of contents
that might appear in a journal article, but if desired, page numbers can
be entered in a manner similar to that for a monograph; see the section
{\bf Book Formatting} below.

The hanging indentation within a table of contents for \cs{head} and
\cs{subhead} is preset to accommodate numbers of the form
``1.''\ and ``1.1.''\ respectively; the amount of indentation can
be adjusted by using \cs{widestnumber}:
\beginexample{}
\\toc
\\widestnumber\\head\{10\}
\\widestnumber\\subhead\{10.1\}
...
\endexample
\noindent This can be done more than once within different sections of
the table of contents, if desired.

If the ``section number'' of a \cs{head} happens to be something like
``Appendix'' (as actually happens in this User's Guide), a pair
of empty braces should be entered before it, as follows:
\beginexample{}
\\head \{\}\ Appendix. Sample bibliography input ...\\endhead
\endexample
\noindent
Insertion of {\tt\{\}} followed by a space at the beginning of the heading
text will cause the entire entry to be set flush left as a unit.

If you are preparing a monograph, the format and content of the top matter
will be different.  See the section below on {\bf Book Formatting} for
details.


\subhead Headings \endsubhead

There are four levels of headings (not counting
\cs{title}'s):
\beginexample{}
\\specialhead...\\endspecialhead
\\head...\\endhead
\\subhead...\\endsubhead
\\subsubhead...\\endsubsubhead
\endexample
\noindent The heading of
this section was typed as
\beginexample{}
\\head 2. Formatting Features
\\endhead
\endexample

\noindent And the subheading for this subsection was typed as
\beginexample{}
\\subhead Headings\\endsubhead
\endexample
\noindent
Ordinarily, subheadings in the preprint style are run into the text, but for
this User's Guide, the style varies slightly.

\indent\cs{specialhead} is for long articles that need extra divisions at
a level above the \cs{head} level. In the preprint style
\cs{specialhead} uses boldface type and is set ragged right; \cs{head}
is small caps, centered; \cs{subhead} is boldface, flush left,
run in with the following text; and
\cs{subsubhead} is italic, indented as for an ordinary paragraph, and run
into the text.

Explicit line breaks are obtained by a \cs{\\} in a \cs{head} or a
\cs{specialhead}, but for \cs{subhead} and \cs{subsubhead},
which are part of their paragraph, just use \cs{linebreak} as
you would in normal paragraphed text.

If you are preparing a monograph, the styles of headings will be different.
See the section below on {\bf Book Formatting} for details.


\subhead Theorems and Proofs \endsubhead

In addition to the usual proclamations and demonstrations, mathematicians
may pose other kinds of propositions, which editors may prefer to see
presented in different styles.  The following have been provided in
the preprint style.
\beginexample{}
\\definition...\\enddefinition
\\example...\\endexample
\\remark...\\endremark
\endexample
\noindent
In the preprint style \cs{definition} and \cs{example} have
the spacing and heading font of \cs{proclaim}, but are in roman.
\cs{remark} resembles \cs{demo}
except that extra space added at the end of a proof by \cs{enddemo}
is not added by \cs{endremark}.

In accordance with the style of the \JAMS, the labels on \cs{proclaim}'s
and similar constructions are now printed in boldface type (\cs{bf}).
However, unlike the \JAMS, the preprint style uses slanted type (\cs{sl})
for the text of a \cs{proclaim}, rather than italic.
(Most \AMS{} publications currently use Times Roman fonts, for which no
slanted form was designed.)

It is conventional in mathematical publishing to use roman,  upright
numbers and punctuation even in the midst of
italic text, to avoid visual conflicts with numbers and punctuation in
adjacent math formulas.  Since dedicated ``mathematical text italic''
fonts containing roman numbers and punctuation are not currently
available, the \filnam{amsppt} preprint style provides a command
\cs{rom} to be applied  inside theorems and other stretches of
italic text, to give the desired results.
For example, to~produce\-

\proclaim{Proposition 2.5} Let $S_1,\dots,S_m$ be the components
of a $J$-holomorphic cusp-curve $S$ and suppose that each
component $S_i$ is \rom(a multiple covering of\rom) a regular
curve and that Assumption \rom{(1.4a)} is satisfied. \dots
\endproclaim

\noindent you would use \cs{rom} in the following places:
\beginexample{}
each component \$S\char`\_i\$ is \\rom(a multiple covering of\\rom) a
regular curve and that Assumption \\rom\{(1.4a)\} is satisfied
\endexample 

As you can see, \cs{rom} is used like the math font command \cs{roman}:
it applies to the next single character or the next group enclosed in
braces.

\subhead Other Devices \endsubhead

For a list produced by \cs{roster}, the amount of indentation can be
adjusted to accommodate wide item numbers.  Just before beginning the
\cs{roster}, type, for example, \cs{widestnumber\\item\{(viii)\}}.
This adjustment is temporary.  The default will be reinstated by
\cs{endroster}.

The command \cs{cite} produces
a reference citation in roman type, within square brackets: \cite{21}.

A structure \cs{block...}\cs{endblock} is provided for quotations.
It is intended for use in the middle of a paragraph to quote an
extract from another source.


\subhead Book Formatting \endsubhead

If you are preparing a monograph, several features are available in the
preprint style that will make your output look like chapters rather than
individual papers.

First of all, you must signal your intentions by typing \cs{Monograph}
in the preamble, right after the \cs{documentstyle} line.

A typical topmatter section for a monograph chapter would be typed
like this:
\beginexample{}
\\documentstyle\{amsppt\}
\\Monograph
\\topmatter
\\title\\chapter\{4\} Matrix Algebras\\endtitle
\\endtopmatter
\endexample
\noindent which produces a chapter heading that looks like this:
\bigskip
\vbox{
\centerline{\eightpoint CHAPTER IV}
\bigskip
\centerline{\bf MATRIX ALGEBRAS}}
\bigskip
\noindent
Notice that the number is converted automatically to roman numerals and
the word ``{\eightpoint CHAPTER}'' is added.  For a chapter title that
needs a different sort of treatment, \cs{nofrills} can be used:

\beginexample{}
\\topmatter
\\title\\chapter\\nofrills\{APPENDIX D\} The Poisson Integral\\endtitle
\\endtopmatter
\endexample
\noindent This produces
\bigskip
\vbox{
\centerline{\eightpoint APPENDIX D}
\bigskip
\centerline{\bf THE POISSON INTEGRAL}}
\bigskip
\noindent The replacement \cs{chapter} text will appear exactly as
typed.

Finally, for things like a preface or introduction which have no
pretitle text at all, omit the \cs{chapter} command:
\beginexample{}
\\topmatter
\\title Preface\\endtitle
\\endtopmatter
\endexample

In monographs, the table of contents is usually treated as a separate
chapter.  Start by typing the title ``Contents'' as for a preface or
introduction, and then use the \cs{toc...\\endtoc} structure as the
body of the document (rather than putting it in the topmatter, as you
would for a journal article).
\beginexample{}
\\topmatter
\\title Contents\\endtitle
\\endtopmatter
\bigskip
\\document
\\toc
\\title Preface\\page\{vii\}\\endtitle
\\title\\chapter\{1\} Matrix Algebras\\page\{1\}\\endtitle
\\head \{\} Continuous complex-valued functions\\page\{1\}\\endhead
...
\\title Bibliography\\page\{307\}\\endtitle
\\endtoc
\\enddocument
\endexample

The chapter titles listed in the table of contents are typed in the same
way as in actual use.  To get page numbers in the table of contents, use
\cs{page} as shown, just before the ending of an element.  This option
is available for all levels of headings.

In a monograph using the preprint style, the chapter title is used for
the left running head and the text of section headings
(from \cs{head}) appears as the right running head.  
It's not uncommon for the text of a heading to be too long to fit
in the running head width; in such a case use \cs{rightheadtext} to
specify a shortened form of the heading for use in the running heads:
\beginexample{}
\\head Fourier coefficients of continuous periodic functions
of bounded entropy norm\\endhead
\\rightheadtext\{Fourier coefficients of periodic functions\}
\endexample
\noindent This should follow immediately after the \cs{head}, to ensure
that both take effect on the same page.  If the chapter title is too long
to fit as a running head, a shortened form can be supplied in a similar
way with \cs{leftheadtext} immediately after the \cs{title}.
See also the section~{\bf Running Heads}.

The style for a chapter of a monograph differs in some particulars from the
style for a paper.  The text of a \cs{head} will be boldface instead of
small caps; headings of theorems, propositions, definitions, remarks, etc\.
will be small caps instead of boldface, and indented rather than flush 
left.


\subhead Inserts with Captions \endsubhead

Figures, tables, and some other kinds of objects are often handled as
inserts.  These objects may be prepared separately from the main document
and pasted in, in which case space must be left for them.
These objects usually have captions; a caption may be positioned above (for
a table) or below (for a figure).

An insert may be specified for the top or ``middle'' of a page, i.e.,
right where the input for the insert occurs in the text.  These are
typed as \cs{topinsert} and \cs{midinsert} respectively.
Furthermore, a caption may be placed at the top or the bottom of the insert,
using the tags \cs{topcaption} and \cs{botcaption} respectively.

The general structure used to specify an insert with a caption at the top is:
\beginexample{}
\\topinsert\quad{\rm or}\quad \\midinsert
\cs{captionwidth}\{\Dimen\}\quad{\rm(optional)}
\\topcaption\{\<caption label>\}
\ \<optional caption text>
\\endcaption
\cs{vspace}\{\Dimen\}\quad{\rm or}\quad%
  \<optional code for the insertion body>
\\endinsert
\endexample

Here the notation \Dimen{} means a valid \TeX{} dimension as
described in the {\bf Dimensions} section of \JoT{}.
If a bottom caption was desired, \cs{topcaption} would be replaced by
\cs{botcaption}, and
the \cs{vspace} command (or the
optional code for the insertion body) would be moved before the
\cs{botcaption} macro.

The \cs{vspace\{\Dimen\}} option would be used to leave blank space for
an object to be pasted into place.  The value of the \Dimen{} should be
the exact height of the object to be pasted in, because extra space
around the object and the caption are dependent on the document style,
and will be provided automatically.

The \cs{captionwidth\{\Dimen\}} option may be used to override the default
caption width specified by the document style.

The \<caption label> is something like ``Figure~1'' or ``Table~2a.''
Do not type any final punctuation; it will be provided.  The caption
label will be set in caps and small caps.

The \<optional caption text> is any descriptive text that may be desired. 
The preprint style will set this in roman.  Even if there is no text, the
\cs{endcaption} tag must be present.

If you choose to include the \TeX{} code for a figure, table, or other
captioned object in the input, then omit the \cs{vspace} command
and type the code where appropriate (before \cs{botcaption}
or after the \cs{endcaption} of \cs{topcaption}).

Sometimes a table is small enough that it is not necessary to put it in an
insert.  If the caption is to appear above it, input can be typed as follows:
\beginexample{}
\\topcaption\{\<caption label>\}
\ \<optional caption text>
\\endcaption
\ \<code for the table body>
\endexample
\noindent
The form of the input would be the following if the caption is to appear below:
\beginexample{}
\ \<code for the table body>
\\botcaption\{\<caption label>\}
\ \<optional caption text>
\\endcaption
\endexample
\noindent
To avoid page-breaking problems,
this form of ``insertion'' should be used only for very small objects.


\subhead Page Numbers \endsubhead

If you are using the preprint style, page numbers will appear in the running
heads, at the outside margin, except for the first page, where the running head
will be omitted and the page number will be centered at the bottom of the page.

If you wish to omit page numbers, type \cs{NoPageNumbers} at the
beginning of the document (after the \cs{documentstyle} line).
The running head text will remain; see also {\bf Running Heads}.

You can get roman numeral page numbers, e.g.\ for a table of contents or
preface, using the normal \TeX{} convention of \cs{pageno} plus
a negative number.


\subhead Page Size \endsubhead

In the preprint style, the default page width is 30pc, and the default
height is 47.5pc. 
You can change the size of the page by typing
\beginexample{}
\\pagewidth\{\Dimen\}\newline
\\pageheight\{\Dimen\}
\endexample
\noindent using suitable \Dimen{}s, where by this notation we mean a
valid \TeX{} dimension as described in the {\bf Dimensions} section of
\JoT{}.


\subhead QED \endsubhead

In the preprint style, \cs{qed} gives an open box `$\square$',
separated from what precedes it by a quad of space.


\subhead Running Heads \endsubhead

If you are using the preprint style, running heads similar to those in
\Joy{} will appear, with text in the center and page numbers to the
outside.  (On the first page, as usual, the running head is omitted, and
the page number is placed at the bottom.)

If you do nothing to define the text of the running heads, the author's
name will be used on the left-hand and the title on the right-hand pages.
(This is the style for papers; for monographs, see below.)
If you want some other values, say a shortened title, you
can redefine the text to appear on left- and right-hand pages by typing
\beginexample{}
\\leftheadtext\{\<left running head text>\}
\\rightheadtext\{\<right running head text>\}
\endexample

\noindent These instructions can appear anywhere after the
\cs{documentstyle} command, but the most
common place to use them is immediately after a \cs{title} or
\cs{author} or \cs{head} to override the automatic running head text. 
If \cs{rightheadtext} or \cs{leftheadtext} is specified above the
topmatter, \cs{title} and \cs{author} will not override them.

If you are doing a monograph rather than a journal article, and use
the \cs{Monograph} switch, it affects the running heads as follows:
The chapter title appears in the left-hand running heads, and the text of
the current section heading (from \cs{head}) appears in the right-hand
running heads.  In chapters that don't contain any \cs{head}'s---for
example, a foreword---both the left- and right-hand running heads
will contain the chapter title.

By default, running heads will be uppercase.  This is a frill that
can be turned off by \cs{nofrills}, e.g.,
\beginexample{}
\\rightheadtext\\nofrills\{Text of Running Head\}
\endexample

If for some reason you don't want running heads at all, type
\cs{NoRunningHeads} at the beginning of the document (after the
\cs{documentstyle} line).  When running heads are omitted, page numbers
will appear centered at the bottom of the page.  (And even those can be
turned off using \cs{NoPageNumbers}.)

In a monograph, if you don't want
the text from the section \cs{head}'s to appear in the running heads
you must redefine the internal command, \cs{headmark}, that is used
by \cs{head} to set the right-hand running head. To do this, put the
following line in your document file, after \cs{Monograph} and
before \cs{topmatter}:
\beginexample{}
\\redefine\\headmark\#1\{\}
\endexample
\noindent (where the {\tt\#1} is an argument number as explained in \Joy,
in the description of \cs{define} and related commands).

\subhead Tables \endsubhead

There are no special macros to support the creation of tables in
\AmSTeX{}. Plain \TeX{}'s \cs{settabs}
command and \cs{halign} can also be used (see {\it The \TeX{}book} for
documentation of their usage). More sophisticated table macro packages
are available from other sources. See also the section {\bf Inserts with
Captions} above.

\subhead Bibliographies \endsubhead

The references section of a paper begins with \cs{Refs} and must have
\cs{endRefs} at the end.  Each entry in the references begins with
\cs{ref} and ends with \cs{endref}.  The individual elements between
\cs{ref} and \cs{endref} can be specified in any order. However,
following \cs{ref} is usually a number or other label identifying the
particular reference. This label is produced using \cs{key}.
The format of the labels is determined by the current
{\it references style}, which is set by the \cs{refstyle}
command. The preprint documentstyle provides three reference
styles denoted A, B, and C, corresponding to
letter labels, no labels, and arabic numbers respectively.
The form of the \cs{cite} and \cs{key} commands for each
style, and the output they produce, is as follows:
$$
\vbox{\offinterlineskip\def\strut{\vrule depth.35\normalbaselineskip
  width0pt height.75\normalbaselineskip}\tabskip0pt
  \halign{{\tt\strut#}\hfil&\quad#\hfil&\quad\vrule\quad{\tt#}\hfil&
       \quad#\hfil&\quad\vrule\quad{\tt#}\hfil&\quad#\hfil\cr
\multispan2\strut depth.7\normalbaselineskip\hfil\cs{refstyle\{A\}}\hfil&
       \multispan2\quad\vrule\hfil\cs{refstyle\{B\}}\hfil&
              \multispan2\quad\vrule\hfil\cs{refstyle\{C\}}\hfil\cr
%
\noalign{\hrule}
%
 height1.1\normalbaselineskip\cs{cite\{DK\}}& [DK]&
       \cs{cite\{Smith 1989\}}& [Smith 1989]&
              \cs{cite\{19\}}& [19]\cr
%
\cs{key DK}& [DK]&
       \omit\quad\vrule\quad(no key)\hfil & (no label)&
              \cs{key 19}& 19.\cr
}}$$
The \cs{refstyle} command is normally placed in the preamble of
a document.

The references are set with hanging indentation.  The amount of indentation
is preset to accommodate the most common case, two-digit numbers.
It can be increased (or decreased) by specifying the widest
label used in the references. For example,
\beginexample{}
\\widestnumber\\key\{GHMR\} \% refstyle A
\\widestnumber\\key\{999\} \% refstyle C --- 3 digits
\endexample
\noindent
will increase the indentation to accommodate the key \hbox{[GHMR]}, or a
three-digit number, respectively.   You could also specify
\cs{widestnumber}\cs{key\{9\}} to reduce the indentation from two digits'
worth to one, if your bibliography has fewer than ten entries. As the
examples show, you do not include square brackets, periods, font
commands, or other such formatting when using \cs{widestnumber}.  The 
indentation will be adjusted for these things automatically.

For consecutive references by the same author(s),  \cs{by} is used for
the first reference, with the author name(s) given in full, and
\cs{bysame} is used for subsequent ones---just the command \cs{bysame}
without repetition of the name(s). The horizontal line produced by
\cs{bysame} has a fixed length of three ems.

Two variations, \cs{ed} and \cs{eds}, are provided for entering editor
names, as with \cs{page} and \cs{pages}, because the note ``ed.''\ or
``eds.''\ is part of the automatic formatting.  If \cs{by} is absent,
the editor name(s) will be used in place of the author name.

For a proceedings volume, the place and date of the meeting can
be recorded in the \cs{procinfo} field.  Parentheses will be
added.

There are two options for miscellaneous notes at the end of a reference,
\cs{finalinfo} and \cs{miscnote}.  \cs{miscnote} differs only by
automatically adding parentheses; it would typically be used for
a note such as ``(preprint)'' or ``(submitted)'' or ``(to appear)''.
Because it's fairly common, the latter has its own command \cs{toappear}
that is equivalent to \cs{miscnote} {\tt to appear}.

\cs{lang} is used to indicate the original language for papers where
bibliographic information has been translated or there is some other reason
to believe that the original language cannot be correctly identified from
information in the reference.

Sometimes several references are combined into one---for example, parts
of a long paper that have been published separately.  Another type of
compound reference is a work cited both in the original and in
translation.  There are commands \cs{moreref} and \cs{transl} to handle
such situations.  After \cs{moreref} and \cs{transl}, any of the normal
reference tags can be used again.

\cs{moreref} is used for citing, e.g., ``part II'' of an article; the
\cs{moreref} command is followed by the desired additional tags and
data.  For example:
\beginexample{}
...\\moreref\\paper\\rom\{II\}
\\jour Comm. Pure Appl. Math. \\vol 36
\\yr 1983 \\pages 571--594\\endref
\endexample

When using \cs{transl}, a note  that describes the translation is
normally entered between \cs{transl} and the next tag. The tags and data
for the translated work then follow. For example: 
\beginexample
...\\transl English transl. \\publ Birkh\\"auser 
\\publaddr Basel \\yr 1985 \\endref
\endexample

Automatic punctuation will be omitted if the pertinent field was
included but left blank.  Otherwise, the command \cs{nofrills} can be
used to keep automatic punctuation from appearing.  For example,
\cs{bookinfo\\nofrills...}\ suppresses the comma or other punctuation
that would normally be added at the end of the \cs{bookinfo}
information. \cs{nofrills} also suppresses other automatic formatting
such as the word ``eds.'' for \cs{eds}, the word ``vol.'' for book volumes,
or the parentheses around the year for journal articles. The ending
period of a reference can be suppressed with
\cs{finalinfo}\cs{nofrills}.

Some examples will illustrate the use of these tags.  See Appendix~A for
samples of input and output. See also Appendix~C of \JoT{} (first
edition: Appendix~B) for more information on references.

%%%%%%%%%%%%%%%%%%%%%%%%%%%%%%%%%%%%%%%%%%%%%%%%%%%%%%%%%%%%%%%%%%%%%%%%

\head 3. Mathematical Constructions
\endhead

\subhead Wide Accents in Math Mode \endsubhead

In version 2.0+ of the AMSFonts,
there are wider versions of the \cs{widehat} and \cs{widetilde}
accents; they appear on lines (5) and (6):
\beginexample{}
\exbox{(1)}{\$\\hat x, \\tilde x\$} $\hat x, \tilde x$
\exbox{(2)}{\$\\widehat x, \\widetilde x\$} $\widehat x, \widetilde x$
\exbox{(3)}{\$\\widehat\{xy\}, \\widetilde\{xy\}\$} %
  $\widehat{xy}, \widetilde{xy}$
\exbox{(4)}{\$\\widehat\{xyz\}, \\widetilde\{xyz\}\$} %
  $\widehat{xyz}, \widetilde{xyz}$
\exbox{(5)}{\$\\widehat\{xyzu\}, \\widetilde\{xyzu\}\$} %
  $\widehat{xyzu}, \widetilde{xyzu}$
\exbox{(6)}{\$\\widehat\{xyzuv\}, \\widetilde\{xyzuv\}\$} %
  $\widehat{xyzuv}, \widetilde{xyzuv}$
\endexample
\noindent
These wider accents are in the \filnam{msbm} family.  If \filnam{msbm}
has been loaded, \cs{widehat} and \cs{widetilde} will automatically
select these wider versions when required; otherwise, the characters
on line (4) will be the largest available.  If you are using the
preprint style, \filnam{msbm} is loaded automatically; otherwise,
see the section entitled {\bf Fonts} for instructions on loading it.


%%%%%%%%%%%%%%%%%%%%%%%%%%%%%%%%%%%%%%%%%%%%%%%%%%%%%%%%%%%%%%%%%%%%%%%%

\head 4. Fonts\endhead

\subhead Additional fonts for \AmSTeX{}\endsubhead

A number of fonts were created for use with \AmSTeX{} 2.0+, both
Computer Modern fonts in sizes not previously available and new fonts
of alphabets and symbols intended to be used for mathematical notation.
These fonts are in the collection AMSFonts Version~2.1.  They must be
installed on your computer before you can use \AmSTeX{}'s preprint style
or otherwise refer to them.
Note that AMSFonts Version~2.1 cannot be used with versions of \AmSTeX{}
earlier than Version~2.0, and \AmSTeX{} Version~2.1 cannot be used with
versions of AMSFonts earlier than Version~2.0
(users of AMSFonts Version~2.0 are strongly encouraged to upgrade to
Version~2.1).

Several of these fonts are loaded automatically by the preprint
style and others can be loaded on demand.  The fonts available and the
commands used to load them are described below.

\subsubhead Fonts loaded with the preprint style
\endsubsubhead
Several fonts are loaded automatically for general use.
\roster
\item"--" \filnam{cmcsc8} is a new size of the Computer Modern small caps font.
\item"--" \filnam{cmex8} and \filnam{cmex7} are new sizes of the Computer
        Modern math extension font.  \filnam{cmex8} is used by the preprint
        style in abstracts and other eight-point environments; \filnam{cmex7}
        is used for all sub- and superscripts.
\endroster


\subsubhead Math fonts loaded with the preprint style
\endsubsubhead
\roster
\item"--" \filnam{msam} and \filnam{msbm} contain extra symbols.  The symbols
        and the names that will produce them are shown in the section 
        {\bf Symbol Names} below.  If you are not using the preprint style,
        each can be loaded separately by \cs{loadmsam} or \cs{loadmsbm}
        as appropriate.
\item"--" \filnam{eufm} is the medium-weight Euler Fraktur (German) font.
        It can also be loaded by \cs{loadeufm} if the preprint style is not
        being used.
\endroster


\subsubhead Math fonts loaded by \cs{loadbold}
\endsubsubhead
See the sections below on {\bf Bold Characters in Math Mode} and
{\bf Bold Greek Letters} for details on accessing particular characters
in these fonts.
\roster
\item"--" \filnam{cmmib} is Computer Modern bold math italic.
        It also contains bold Greek.
\item"--" \filnam{cmbsy} contains Computer Modern bold math symbols.
\endroster


\subsubhead Additional Euler fonts, for use in math, loaded by
\cs{loadeu...}\tt\endsubsubhead
\roster
\item"--" \filnam{eufb} is bold Fraktur (\cs{loadeufb}).
\item"--" \filnam{eusm} is medium-weight script (\cs{loadeusm}).
\item"--" \filnam{eusb} is bold script (\cs{loadeusb}).
\item"--" \filnam{eurm} is medium-weight ``cursive roman'' (\cs{loadeurm}).
\item"--" \filnam{eurb} is bold ``cursive roman'' (\cs{loadeurb}).
\endroster


\subsubhead Considerations and warnings\endsubsubhead
The commands to load these font files should be typed in the preamble area
between the 
\cs{documentstyle\{...\}} line and the \cs{topmatter}.
Each \cs{load...} command loads the pertinent fonts (including
subscript sizes), assigns a ``math
family'' for them, and defines a math font command.
The names of the commands are the same as the font names:
\cs{eufm}, \cs{eufb}, \cs{eusm}, \cs{eusb}, \cs{eurm}, and
\cs{eurb}.  These are used in the same way as
\cs{roman} or \cs{bold}, e.g., \cs{eufb\{M\}} or \cs{eufb M}@.
\AmSTeX{} also defines a couple of synonyms,
\cs{frak} and \cs{goth}, for \cs{eufm} (medium Euler Fraktur).

\TeX{} can accommodate only sixteen font families in math mode; eight
are already defined by plain \TeX{} before \AmSTeX{} begins, and the
preprint style loads three more (\filnam{msam}, \filnam{msbm}, and
\filnam{eufm}), for a total of eleven.  For this reason, you should load
additional fonts with care, requesting only those you know for certain
you will need.

All the fonts described here, and some others as well, are included in the
collection AMSFonts Version~2.1, which is available from the AMS and other
distributors.  The math fonts mentioned here are all supplied in sizes from
five through ten point, suitable for use in mathematical text.


\subhead Bold Characters in Math Mode \endsubhead

Bold letters are obtained by \cs{bold} as described in \Joy{}.
In addition, bold symbols, italic, and lowercase Greek can be
obtained once \cs{loadbold} appears in the file (this
requires version 2.0+ of \AmSTeX{} and AMSFonts).  Two control sequences
are used for different kinds of bold symbols:
\beginexample{\exboxwidth=1.25in}
\exbox{}{\\boldkey} for symbols that actually appear on the keyboard
\exbox{}{\\boldsymbol} for symbols specified by a single control sequence
\endexample
\noindent
For example,
$$\hbox{\tt\$\\bold x \\boldsymbol\\in \\boldsymbol\\varGamma\$}$$
gives
$$\bold x \boldsymbol\in \boldsymbol\varGamma$$
[and {\tt\$\\boldsymbol\\lbrack a \\boldsymbol\\rbrack\$} gives
$\boldsymbol\lbrack a \boldsymbol\rbrack$, if you need to use
\cs{lbrack} and \cs{rbrack} instead of the {\tt[} and {\tt]} keys].

More precisely, \cs{boldkey} can be used in math formulas in the
following combinations:
\roster
\item"$\bullet$" With any of the symbols
$$ +\ \ -\ \ =\ \ <\ \ >\ \ (\ \ )\ \ [\ \ ]\ \ |\ \ /\ \ *
    \ \ .\ \ ,\ \ :\ \ ;\ \ !\ \ ?$$
to give
$$
\boldkey+\ \ \boldkey-\ \ \boldkey=\ \ \boldkey<\ \ \boldkey>\ \ 
\boldkey(\ \ \boldkey)\ \ \boldkey[\ \ \boldkey]\ \ \boldkey|\ \ 
\boldkey/\ \ \boldkey*\ \ \boldkey.\ \ \boldkey,\ \ \boldkey:\ \ 
\boldkey;\ \ \boldkey!\ \ \boldkey?
$$
But \cs{bold} cannot be used to get bold versions of these symbols.
{\tt\$\\bold+\$} will give only the ordinary $+$, etc.

The bold $\boldkey+$ and $\boldkey-$ will be  binary operators,
like the ordinary $+$ and $-$ symbols;
the bold $\boldkey=$ will be a binary relation, like the ordinary $=$, etc.

\medskip
\item"$\bullet$" With letters:
\beginexample{\exboxwidth=3.75in}
\exbox{}{\$\\boldkey a\$, ..., \$\\boldkey z\$} %
        $\boldkey a, \dots, \boldkey z$
\exbox{}{\$\\boldkey A\$, ..., \$\\boldkey Z\$} %
        $\boldkey A, \dots, \boldkey Z$
\endexample
\noindent
Notice that these are $\fam\cmmibfam bold\ math\ italic$ letters, as
opposed to the bold text letters $\bold a, \dots, \bold z$, $\bold A,
\dots, \bold Z$ that you get by using \cs{bold} in math mode.

\medskip
\item"$\bullet$" With numbers:
\beginexample{\exboxwidth=3.75in}
\exbox{}{\$\\boldkey 0\$, ..., \$\\boldkey 9\$} %
        $\boldkey 0, \dots, \boldkey 9$
\endexample
\noindent
However, these combinations simply give the same numerals that you get with
{\tt\$\\bold0\$}, \dots, {\tt\$\\bold9\$}.
\endroster

\medskip
The \cs{boldsymbol} construction can be used in any of the following
combinations:
\roster
\item"$\bullet$" With uppercase and lowercase Greek letters
\beginexample{\exboxwidth=3.75in}
\exbox{}{\$\\boldsymbol\\Gamma\$, ..., \$\\boldsymbol\\Omega\$} %
        $\boldsymbol\Gamma$, \dots, $\boldsymbol\Omega$
\exbox{}{\$\\boldsymbol\\varGamma\$, ..., \$\\boldsymbol\\varOmega\$} %
        $\boldsymbol\varGamma$, \dots, $\boldsymbol\varOmega$
\exbox{}{\$\\boldsymbol\\alpha\$, ..., \$\\boldsymbol\\omega\$} %
        $\boldsymbol\alpha$, \dots, $\boldsymbol\omega$
\endexample
\noindent
In versions of \AmSTeX{} earlier than 2.0, bold unslanted uppercase
Greek letters $\boldsymbol\Gamma$, \dots, $\boldsymbol\Omega$ were
specified by \cs{boldGamma}, \dots, \cs{boldOmega}; these control
sequences have now disappeared. 

\medskip
\item"$\bullet$"
For convenience, \cs{boldsymbol} may also be followed by a letter (but
not by a number or other character), giving the same result as
\cs{boldkey}.
  
\medskip
\item"$\bullet$"
You can also apply \cs{boldsymbol} to all the other standard symbols that
are specified by single control sequences. For example, to get bold primes:
\beginexample{\exboxwidth=3.75in}
\exbox{}{\$\\boldsymbol\\prime\$} $\boldsymbol\prime$
\exbox{}{\$\\boldsymbol A\^{ }\{\\boldsymbol\\prime\}\$} %
        $\boldsymbol A^{\boldsymbol\prime}$
\endexample
\noindent
(But \cs{boldsymbol'}, using the shorthand notation for \cs{prime},
won't work.)

\medskip
\item"$\bullet$"
You can apply \cs{boldsymbol} to ``delimiters,'' such as
\beginexample{\exboxwidth=3.75in}
\exbox{}{\$\\boldsymbol\\\{ ... \\boldsymbol\\\}\$} %
        $\boldsymbol\{ \dots \boldsymbol\}$
\exbox{}{\$\\boldsymbol\\langle ... \\boldsymbol\\rangle\$} %
        $\boldsymbol\langle \dots \boldsymbol\rangle$
\exbox{}{\$\char`\|, \\boldkey\char`\|, \\\char`\|, \\boldsymbol\\\char`\|\$} %
        $|,\ \boldkey|,\ \|,\ \boldsymbol\|$
\exbox{}{\$\\vert, \\boldsymbol\\vert, \\Vert, \\boldsymbol\\Vert\$} %
        $\vert,\ \boldsymbol\vert,\ \Vert,\ \boldsymbol\Vert$
\endexample
\noindent
However, you can't use \cs{boldsymbol} after \cs{left} and \cs{right}.
In particular, typing
\hbox{\tt\\left\\boldsymbol\char`\|\ ...\ \\right\\boldsymbol\char`\|}
will produce only error messages.

\medskip
\item"$\bullet$"
Certain symbols on the bold fonts can't be accessed at all via \cs{boldkey}
or \cs{boldsymbol}: These include bold versions 
${\fam\cmbsyfam A}$, \dots, ${\fam\cmbsyfam Z}$
 of the ``calligraphic letters'' $\Cal A$,~\dots, $\Cal Z$ that you type
as \cs{Cal A}, \dots, \cs{Cal Z},
and bold versions {\tencmmib0}, \dots, {\tencmmib9}
 of the oldstyle numbers
\oldnos0, \dots, \oldnos9 that you get with \cs{oldnos}.  If 
you really need to have these symbols, you will have to enlist the aid of a
\TeX{}nician, or use \cs{pmb}.
\endroster


\subhead Fraktur Font \endsubhead

The German Fraktur font, which is designed for use
only in math mode, can be made
available by typing \cs{loadeufm} in the preamble area
of your paper.  If you are using the preprint style,
medium-weight Fraktur is loaded automatically.
To produce a Fraktur letter, type
\beginexample{\exboxwidth=3.75in}
\exbox{}{\$\\frak g\$} $\frak g$
\exbox{}{\$\\frak A\$, \\dots, \$\\frak Z\$} $\frak A$, \dots, $\frak Z$
\endexample


\subhead Blackboard Bold \endsubhead

\AmSTeX{} has a ``blackboard bold'' font, \cs{Bbb}.  Like \cs{Cal},
it will work only in math mode, and only when applied to uppercase
letters.  This alphabet is part of the \filnam{msbm} font, and can be
made available by typing \cs{loadmsbm} at the top of your file.  (It is
loaded automatically with the preprint style.)
\beginexample{\exboxwidth=3.75in}
\exbox{}{\$\\Bbb A, \\Bbb C, \\Bbb R\$, etc.} $\Bbb A, \Bbb C, \Bbb R$, etc.
\endexample


\subhead Poor Man's Bold \endsubhead

\AmSTeX{} now has boldface versions of most math symbols.  However, if you
need only one or two bold symbols and have run out of \TeX{} capacity for
new fonts or font families, you can always get a poor man's bold version
of bold with \cs{pmb}, as described in \Joy{}.


\subhead Bold Greek Letters \endsubhead

Bold Greek letters, both lowercase and uppercase, can be obtained by
using the \cs{boldsymbol} construction, as described in {\bf Bold
Characters in Math Mode}.  The upright uppercase bold Greek letters are
part of the ordinary bold font and therefore extra font loading commands
do not need to be used in order to get them.  However, the lowercase and
slanted uppercase bold Greek letters are not loaded automatically, so
you must specify \cs{loadbold} before using them.


%%%%%%%%%%%%%%%%%%%%%%%%%%%%%%%%%%%%%%%%%%%%%%%%%%%%%%%%%%%%%%%%%%%%%%%%

\head 5. Symbol Names
\endhead

The symbols in the \filnam{msam} and \filnam{msbm} fonts have been
assigned ``standard'' control sequence names as shown below.  All
the symbol names are loaded automatically by the preprint style; if
you are not using the preprint style, the command \cs{UseAMSsymbols}
will have the same effect.
This will add about 200 new
control sequences to \TeX{}'s internal table.  If you are short on
space, or need only a few of the symbols, you can use a different
approach to access just the ones you need.  See the section {\bf The
\cs{newsymbol} command} below.


\subhead Special Symbols and Blackboard Bold Letters
\endsubhead

Certain symbols from the \filnam{msam} family can be specified by
control sequences that will be defined as soon as the command
\cs{loadmsam} has appeared in the file.

First there are four symbols that are normally used outside of math mode:
$$\vcenter{\halign to\hsize{\1{#}\hfil\tabskip\centering&
   \hbox to.5\hsize{\1{#}\hfil}\tabskip0pt\cr
checkmark&circledR\cr
maltese&yen\cr}}
$$
These symbols, like \P, \S, \dag, and \ddag, can also be used in
math mode, and will change sizes correctly in subscripts and superscripts.

Next are four symbols that are ``delimiters'' (although there are
no larger versions obtainable with \cs{left} and \cs{right}), so they
must be used in math mode:
$$\vcenter{\halign to\hsize{\1{#}\hfil\tabskip\centering&
   \hbox to.5\hsize{\1{#}\hfil}\tabskip0pt\cr
 ulcorner&urcorner\cr
 llcorner&lrcorner\cr}}$$

Finally, two dashed arrows are constructed from symbols in this family.
Note that one of them has two names; it can be accessed by either one:
$$\vcenter{\halign to\hsize{\1{#}\hfil\tabskip\centering&
   \hbox to.5\hsize{\1{#}\hfil}\tabskip0pt\cr
 \omit\hbox to.5\hsize{\hbox to\biggest{\hfil$\dashrightarrow$\hfil}\ \ %
    \cs{dashrightarrow}, \cs{dasharrow}\hss}&dashleftarrow\cr}}$$

The Blackboard Bold letters $\Bbb A, \dots, \Bbb Z$
appear in the \filnam{msbm} family.  Once \cs{loadmsbm} has appeared
in the file, they can be typed (in math mode) as \cs{Bbb A}, \dots,
\cs{Bbb Z}.

The \filnam{msbm} family also contains wider versions of the \cs{widehat}
and \cs{widetilde} as described in Chapter 20, ``Wide accents in math
mode.''


\subhead The \cs{newsymbol} Command\endsubhead

All other symbols of the \filnam{msam} and \filnam{msbm} fonts must be named
by control sequences so that they can be used (in math mode only) when the
fonts are loaded.  This can be done all at once by typing the instruction
\cs{UseAMSsymbols}, which will load in the file \filnam{AMSSYM.TEX}\null.
This instruction is included in the preprint style, so the names are
assigned automatically, which requires over~200 control sequences.

If you are very short on space for control sequence names, and need only
a few of these symbols, you can omit \cs{UseAMSsymbols}.  Instead,
assign only the names you will need by using a new \AmSTeX{} control
sequence \cs{newsymbol} to create a control sequence that will
properly produce this symbol.  The control sequence can be either the
``standard'' name, as listed below, or one of your own choosing.

The list of symbols below shows for each symbol the symbol itself, a
four-character~``ID,'' and the ``standard'' name of the symbol. 
(The first character of the ID identifies the font family in which a
symbol resides.  Symbols from the \filnam{msam} family have {\tt1} as the
first character; symbols from the \filnam{msbm} family have {\tt2} as the
first character.)
For example, the symbol $\nleqslant$ appears as
\medskip
\noindent\kern2\parindent\2{nleqslant}
\medskip
\noindent
To produce a control sequence with this name, the instruction
\medskip
\noindent\kern2\parindent\cs{newsymbol}\cs{nleqslant 230A}
\medskip
\noindent
appears in the file \filnam{AMSSYM.TEX}\null.  This same instruction can
be typed by a user who is not using the preprint style and has chosen not
to load all the symbol names by \cs{UseAMSsymbols}.  Thereafter, the
control sequence \cs{nleqslant} will produce the symbol $\nleqslant$
(in math mode), and will act properly as a ``binary relation.''

A few symbols in these fonts replace symbols defined in \filnam{PLAIN.TEX}
by combinations of symbols available in the Computer Modern fonts.  These
are \cs{angle}~($\angle$) and \cs{hbar}~($\hbar$) from the group
``Miscellaneous symbols,'' and \cs{rightleftharpoons}~($\rightleftharpoons$)
from the group ``Arrows'' below (and \Joy, Appendix~F).  The new symbols will
change sizes correctly in subscripts and superscripts, provided that you
are using appropriate redefinitions.  In order to use \cs{newsymbol} to
replace an existing definition, the name must first be ``undefined.''
Here are the lines you must put in your file if you are not using the
preprint style or \cs{UseAMSsymbols} (which perform the redefinition
automatically):
\medskip
\begingroup
\parindent=2\parindent
\obeylines
\cs{undefine}\cs{angle}
\cs{newsymbol}\cs{angle 105C}
\cs{undefine}\cs{hbar}
\cs{newsymbol}\cs{hbar 207E}
\cs{undefine}\cs{rightleftharpoons}
\cs{newsymbol}\cs{rightleftharpoons 130A}
\endgroup
\medskip
\noindent
These symbols are flagged in the tables below with a ``{\eightpoint(U)}''
as a reminder that they must be undefined.

Note in the tables that some symbols are shown with two names.  In such
cases, either one can be used to access the symbol.

%  since the symbol tables are set in displays, decrease the skip
%  above, so that the space between a section heading and table is
%  not so large.
\abovedisplayskip=3pt plus 3pt minus 0pt

\BBB{Lowercase Greek letters}
$$\halign{\hbox to.5\hsize{\2{#}}&\2{#}\cr
digamma&varkappa\cr}$$

\BBB{Hebrew letters}
$$\halign{\hbox to.5\hsize{\2{#}}&\2{#}\cr
beth&gimel\cr 
daleth\cr
}$$

\BBB{Miscellaneous symbols}
$$\halign{\hbox to.5\hsize{\2{#}}&\2{#}\cr
\omit\4{hbar}&backprime\cr
hslash&varnothing\cr
vartriangle&blacktriangle\cr
triangledown&blacktriangledown\cr
square&blacksquare\cr
lozenge&blacklozenge\cr
circledS&bigstar\cr
\omit\4{angle}&sphericalangle\cr
measuredangle&\omit\cr
nexists&complement\cr
mho&eth\cr
Finv&diagup\cr
Game&diagdown\cr
Bbbk&\omit\cr
}$$

\BBB{Binary operators}
$$\halign{\hbox to.5\hsize{\2{#}}&\2{#}\cr
dotplus&ltimes\cr
smallsetminus&rtimes\cr
\omit\3{Cap}{doublecap}&leftthreetimes\cr
\omit\3{Cup}{doublecup}&rightthreetimes\cr
barwedge&curlywedge\cr
veebar&curlyvee\cr
doublebarwedge\cr
boxminus&circleddash\cr
boxtimes&circledast\cr
boxdot&circledcirc\cr
boxplus&centerdot\cr
divideontimes&intercal\cr}
$$

\BBB{Binary relations}
$$\halign{\hbox to.5\hsize{\2{#}}&\2{#}\cr
leqq&geqq\cr
leqslant&geqslant\cr
eqslantless&eqslantgtr\cr
lesssim&gtrsim\cr
lessapprox&gtrapprox\cr
approxeq\cr
lessdot&gtrdot\cr
\omit\3{lll}{llless}&\omit\3{ggg}{gggtr}\cr
lessgtr&gtrless\cr
lesseqgtr&gtreqless\cr
lesseqqgtr&gtreqqless\cr
\omit\3{doteqdot}{Doteq}&eqcirc\cr
risingdotseq&circeq\cr
fallingdotseq&triangleq\cr
backsim&thicksim\cr
backsimeq&thickapprox\cr
subseteqq&supseteqq\cr
Subset&Supset\cr
sqsubset&sqsupset\cr
preccurlyeq&succcurlyeq\cr
curlyeqprec&curlyeqsucc\cr
precsim&succsim\cr
precapprox&succapprox\cr
vartriangleleft&vartriangleright\cr
trianglelefteq&trianglerighteq\cr
vDash&Vdash\cr
Vvdash\cr
smallsmile&shortmid\cr
smallfrown&shortparallel\cr
bumpeq&between\cr
Bumpeq&pitchfork\cr
varpropto&backepsilon\cr
blacktriangleleft&blacktriangleright\cr
therefore&because\cr}$$
\bigbreak
\BBB{Negated relations}
$$\halign{\hbox to.5\hsize{\2{#}}&\2{#}\cr
nless&ngtr\cr
nleq&ngeq\cr
nleqslant&ngeqslant\cr
nleqq&ngeqq\cr
lneq&gneq\cr
lneqq&gneqq\cr
lvertneqq&gvertneqq\cr
lnsim&gnsim\cr
lnapprox&gnapprox\cr
nprec&nsucc\cr
npreceq&nsucceq\cr
precneqq&succneqq\cr
precnsim&succnsim\cr
precnapprox&succnapprox\cr
nsim&ncong\cr
nshortmid&nshortparallel\cr
nmid&nparallel\cr
nvdash&nvDash\cr
nVdash&nVDash\cr
ntriangleleft&ntriangleright\cr
ntrianglelefteq&ntrianglerighteq\cr
nsubseteq&nsupseteq\cr
nsubseteqq&nsupseteqq\cr
subsetneq&supsetneq\cr
varsubsetneq&varsupsetneq\cr
subsetneqq&supsetneqq\cr
varsubsetneqq&varsupsetneqq\cr}$$

\overfullrule=0pt

\BBB{Arrows}
$$\halign{\hbox to.5\hsize{\2{#}}&\2{#}\cr
leftleftarrows&rightrightarrows\cr
leftrightarrows&rightleftarrows\cr
Lleftarrow&Rrightarrow\cr
twoheadleftarrow&twoheadrightarrow\cr
leftarrowtail&rightarrowtail\cr
looparrowleft&looparrowright\cr
leftrightharpoons&\omit\4{rightleftharpoons}\cr
curvearrowleft&curvearrowright\cr
circlearrowleft&circlearrowright\cr
Lsh&Rsh\cr
upuparrows&downdownarrows\cr
%      Some fancy tricks to avoid a lot of extra work.  MJD
upharpoonleft&upharpoonright}\setbox0\lastbox\unhbox0\unskip,\hfill{\cr
downharpoonleft&\omit\kern7em \cs{restriction}\hfil\cr
multimap&downharpoonright\cr
leftrightsquigarrow&rightsquigarrow\cr}$$

\BBB{Negated arrows}
$$\halign{\hbox to.5\hsize{\2{#}}&\2{#}\cr
nleftarrow&nrightarrow\cr
nLeftarrow&nRightarrow\cr
nleftrightarrow&nLeftrightarrow\cr}$$


%%%%%%%%%%%%%%%%%%%%%%%%%%%%%%%%%%%%%%%%%%%%%%%%%%%%%%%%%%%%%%%%%%%%%%%%

\head 6. Other Things You Ought to Know
\endhead

\subhead Errata to \JoT{} prior to \AmSTeX{} 2.0
\endsubhead

The file \filnam{JOYERR.TEX} contains the full list of errata
for the first edition of \JoT{}, for versions of \AmSTeX{}
earlier than version 2.0.  A user who
desires a typeset copy of this file may run it through \TeX{} and print
out the \filnam{.dvi} file.  This will require Version~2.0+ of \AmSTeX{}
and \filnam{AMSPPT.STY}, and also AMSFonts Version~2.0+
(users of AMSFonts Version~2.0 are strongly encouraged to upgrade to
Version~2.1).


\subhead Acknowledging the Use of \AmSTeX{}\endsubhead

The following are suggested as appropriate statements of acknowledgment
that \AmSTeX{} has been used to format a document for publication.

\penalty-9000 % to encourage a break without absolutely forcing it.
A single paper should include the following at the bottom of the first
page:
\beginexample{}
\rm{}Typeset by \AmSTeX{}
\endexample
\noindent
(This notation is provided automatically by the \AmSTeX{} preprint style.)

If an entire journal or book is prepared with \AmSTeX{}, the following
statement should appear on its copyright page:
\beginexample{}
\rm{}This [journal/book] was typeset by \AmSTeX{}, the \TeX{} macro %
system of the \AMS{}.
\endexample

If only selected papers in a journal or book are set with \AmSTeX{}, these
papers should be identified as shown above, and the following should
appear on the copyright page:
\beginexample{}
\rm{}\AmSTeX{} is the \TeX{} macro system of the \AMS{}.
\endexample


\head 7. Getting Help
\endhead

If you should find any bugs in the macros or documentation,
send a Problem Report to:
\beginexample{\rm}
Technical Support Department
\AMS{}
P. O. Box 6248
Providence, RI 02940
\vskip 2pt %
Phone: 800-321-4AMS \quad or \quad 401-455-4080
Internet: tech-support\@Math.AMS.org
\endexample

A Problem Report should contain the following information:
\roster
\item version of \filnam{AMSTEX.TEX} and of \filnam{AMSPPT.STY} with which
  the problem occurred;
\item a detailed description of the problem, including the input code for
  one or more examples that illustrate the problem;
\item a log file of a \TeX{} session showing the problem.
\endroster

\head References\endhead

\noindent\hangindent2pc Knuth, Donald E. {\it The \TeX{}book}. Reading:
Addison Wesley, 1986.

\noindent\hangindent2pc Spivak, Michael D. {\it The Joy of \TeX{}},
2nd (revised) edition, \AMS{}, Providence, 1990.


%%%%%%%%%%%%%%%%%%%%%%%%%%%%%%%%%%%%%%%%%%%%%%%%%%%%%%%%%%%%%%%%%%%%%%%%

\newpage

%  Arrange for the sample references to be set broadside, with the output
%  pasted up next to the corresponding input.  The section heading should
%  be full-width, and the running heads should be the normal page width.
%  To accomplish the latter, we must redefine plain's \makeheadline.

\begingroup % This will be ended after the broadside section

% To suppress an unimportant `overfull vbox' message (0.8 points):
\vfuzz=1pt

\newdimen\headlinewidth
\headlinewidth=\hsize
\def\makeheadline{\vbox to0pt{\vskip-22.5pt
  \hbox to\headlinewidth{\vbox to8.5pt{}\the\headline}\vss}\nointerlineskip}

\head Appendix A.\quad Sample Bibliography Input and Output
\endhead

\pageheight{30pc}

\pagewidth{23pc}
\beginexample{\exindent=0pt}
\\Refs
\\ref\\key 4 \% assuming \\refstyle\{C\} 
\\by V. I. Arnol\$'\$d, A. N. Varchenko,
\ and S. M. Guse\\u\\i n-Zade
\\book Singularities of differentiable maps.~\\rom I
\\publ ``Nauka'' \\publaddr Moscow \\yr 1982
\\lang Russian
\\endref
\ {}
\\ref\\key 5\\bysame 
\\book Singularities of differentiable maps.~\\rom\{II\}
\\publ ``Nauka'' \\publaddr Moscow \\yr 1984
\\lang Russian
\\endref
\ {}
\\ref\\key 6
\\by O. A. Ladyzhenskaya
\\book Mathematical problems in the dynamics
\ of a viscous incompressible fluid 
\\bookinfo 2nd rev. aug. ed.
\\publ ``Nauka'' \\publaddr Moscow \\yr 1970
\\lang Russian
\\transl English transl. of 1st ed.
\\book The mathematical theory of viscous
\ incompressible flow
\\publ Gordon and Breach \\publaddr New York
\\yr 1963; rev. 1969
\\endref
\endexample

\newpage

\beginexample{\exindent=0pt}
\\ref\\key 7
\\by P. D. Lax and C. D. Levermore
\\paper The small dispersion limit for the
\ KdV equation.~\\rom I
\\jour Comm. Pure Appl. Math. \\vol 36 \\yr 1983
\\pages 253--290 \\finalinfo (overview)
\\moreref\\paper \\rom\{II\}
\\jour Comm. Pure Appl. Math. 
\\vol 36 \\yr 1983 \\pages 571--594
\\moreref\\paper \\rom\{III\}
\\jour Comm. Pure Appl. Math. 
\\vol 36 \\yr 1983 \\pages 809--829 \\endref
\ {}
\\ref\\key 10 \\by S. Osher
\\paper Shock capturing algorithms for equations of
\ mixed type
\\inbook Numerical Methods for Partial Differential
\ Equations \\eds S. I. Hariharan and T. H. Moulton
\\publ Longman \\publaddr New York \\yr 1986
\\pages 305--322
\\endref
\ {}
\\ref\\key 17 \\by G. S. Petrov
\\paper Elliptic integrals and their nonoscillatory
\ behavior
\\jour Funktsional. Anal. i Prilozhen.
\\vol 20 \\yr 1986 \\pages 46--49
\\transl\\nofrills English transl. in
\\jour Functional Anal. Appl. \\vol 20\\yr 1986
\\endref
\endexample

\newpage

\beginexample{\exindent=0pt}
\% switch to a different references style
\\refstyle\{A\}
\\widestnumber\\key\{GHMR\}
\ {}
\\ref\\key C1
\\by B. Coomes
\\book Polynomial flows, symmetry groups, and
\ conditions sufficient for injectivity of maps
\\bookinfo Ph.D. thesis, Univ. Nebraska--Lincoln
\\yr 1988
\\endref
\ {}
\\ref\\key C2
\\bysame \% B. Coomes
\\paper The Lorenz system does not have a
\ polynomial flow
\\jour J. Differential Equations
\\toappear
\\endref
\ {}
\\ref\\key GHMR
\\by J. Guckenheimer, P. Holmes, M. Martineau,
\ and L. P. Robinson
\\book Nonlinear oscillations, dynamical systems,
\ and bifurcations of vector fields
\\bookinfo \% fields can be left blank
\\publ Springer-Verlag \\publaddr New York
\\yr 1983
\\endRefs
\endexample

\newpage

\begingroup
\refstyle{C}
\aboveheadskip=\abovedisplayskip

\Refs
\ref\key 4 % assuming \refstyle{C}
\by V. I. Arnol$'$d, A. N. Varchenko, and S. M. Guse\u\i n-Zade
\book Singularities of differentiable maps.~{\rm I}
\publ ``Nauka'' \publaddr Moscow
\yr 1982
\lang Russian
\endref

\ref\key 5
\bysame 
\book Singularities of differentiable maps.~{\rm II}
\publ ``Nauka'' \publaddr Moscow
\yr 1984
\lang Russian
\endref

\ref\key 6
\by O. A. Ladyzhenskaya
\book Mathematical problems in the dynamics of a
 viscous incompressible fluid 
\bookinfo 2nd rev. aug. ed.
\publ ``Nauka'' \publaddr Moscow
\yr 1970
\lang Russian
\transl English transl. of 1st ed.
\book The mathematical theory of viscous
incompressible flow
\publ Gordon and Breach \publaddr New York
\yr 1963; rev. 1969
\endref

\bigskip

\ref\key 7
\by P. D. Lax and C. D. Levermore
\paper The small dispersion limit for the KdV equation.~{\rm I}
\jour Comm. Pure Appl. Math. 
\vol 36 \yr 1983 \pages 253--290
\finalinfo (overview)
\moreref\paper {\rm II}
\jour Comm. Pure Appl. Math. 
\vol 36 \yr 1983 \pages 571--594
\moreref\paper {\rm III}
\jour Comm. Pure Appl. Math. 
\vol 36 \yr 1983 \pages 809--829 \endref

\ref\key 10 \by S. Osher
\paper Shock capturing algorithms for equations of mixed type
\inbook Numerical Methods for Partial Differential Equations
\eds S. I. Hariharan and T. H. Moulton
\publ Longman \publaddr New York \yr 1986 \pages 305--322
\endref

\ref\key 17 \by G. S. Petrov
\paper Elliptic integrals and their nonoscillatory behavior
\jour Funktsional. Anal. i Pri\-lo\-zhen.
\vol 20 \yr 1986 \pages 46--49
\transl\nofrills English transl. in \jour Functional Anal. Appl.
\vol 20\yr 1986
\endref

\bigskip

% switch to a different references style
\refstyle{A}
\widestnumber\key{GHMR}

\ref\key C1
\by B. Coomes
\book Polynomial flows, symmetry groups, and conditions
 sufficient for injectivity of maps
\bookinfo Ph.D. thesis, Univ. Nebraska--Lincoln
\yr 1988
\endref

\ref\key C2
\bysame % B. Coomes
\paper The Lorenz system does not have a polynomial flow
\jour J. Differential Equations
\toappear
\endref

\ref\key GHMR
\by J. Guckenheimer, P. Holmes, M. Martineau, and L. P. Robinson
\book Nonlinear oscillations, dynamical systems, and
 bifurcations of vector fields
\bookinfo % fields can be left blank
\publ Springer-Verlag \publaddr New York
\yr 1983
\endref
\endRefs

\endgroup       % end special value of \aboveheadskip


\newpage
\endgroup % end broadside section

% Put the page number at the bottom of the page:
\csname firstpage\string @true\endcsname
% Turn off the "Typeset by AmSTeX" logo:
\expandafter\redefine\csname logo\string @\endcsname{}

% Reset the page number because the broadside pages in
% the previous section will become three pages instead of four.
% 18-JAN-1991 mjd
\advance\pageno by -1

%      The following file contains the installation instructions
\input amstinst.tex

\enddocument

%%%%%%%%%%%%%%%%%%%%%%%%%%%%%%%%%%%%%%%%%%%%%%%%%%%%%%%%%%%%%%%%%%%%%%%%
