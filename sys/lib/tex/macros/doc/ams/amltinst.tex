%%%%%%%%%%%%%%%%%%%%%%%%%%%%%%%%%%%%%%%%%%%%%%%%%%%%%%%%%%%%%%%%%%%%%%%%%%%%
% AMLTINST.TEX						    July 1990      %
%                                                                          %
% This file is part of the AMS-LaTeX Version 1.0 distribution              %
%   American Mathematical Society, Technical Support Group,                %
%   P. O. Box 6248, Providence, RI 02940                                   %
%   800-321-4AMS (321-4267) or 401-455-4080                                %
%   Internet: Tech-Support@Math.AMS.com                                    %
%%%%%%%%%%%%%%%%%%%%%%%%%%%%%%%%%%%%%%%%%%%%%%%%%%%%%%%%%%%%%%%%%%%%%%%%%%%%
%%
%% A file for instructions for installation of AMS-LaTeX macro package.
%%
\documentstyle{report}
\textheight 46pc
\textwidth 30.5pc
\def\thesection {\arabic{section}}

\newcommand{\amstex}{{\the\textfont2 A}\kern-.1667em\lower.5ex\hbox
 {\the\textfont2 M}\kern-.125em{\the\textfont2 S}-\TeX}
\newcommand{\amslatex}{{\the\textfont2 A}\kern-.1667em\lower.5ex\hbox
 {\the\textfont2 M}\kern-.125em{\the\textfont2 S}-\LaTeX}

\newcommand{\usertype}[1]{\vspace{3pt plus 3pt}\centerline{%
                                #1}\vspace{3pt plus 3pt}}

% Define a control sequence to enable the @ sign to be typeset regardless
% of whether this document is being run under AMS-LaTeX or standard LaTeX.
{\catcode`\@=12
\gdef\atsign/{@}}

\begin{document}

\centerline{\huge Installation Guide for \amslatex\ 1.0}
\vspace{-8pt}

\section{Introduction}

The \amslatex\ Version 1.0 macro package is a group of files which are 
designed to provide some of the capabilities of \amstex\ within the 
\LaTeX\ environment.

\amslatex\ can be used with any full implementation of \TeX. See the {\it
\amslatex\ User's Guide}
for a listing of recommended \TeX\ capacities for using this package.

If you do not have a recent version of the file \verb+latex.tex+, then you 
may have difficulty using the \amslatex\ package. Checking the version
number is not sufficient; there have been several releases called Version 2.09.
At the top of the file, next to the Version number (probably version 2.09),
your \verb+latex.tex+ contains a date. That date must {\it not} be earlier than
May 1986.
If your version is earlier than May 1986, then you must obtain a more
recent version.

\section{Files included in this distribution}

The subdirectory \verb+doc+
contains the following files:

{\tt
\begin{center}
\begin{tabular}{l@{\hspace{3em}}l@{\hspace{3em}}l}
amltinst.tex&
amsart.doc&
amsart10.doc\\
amsbk10.doc&
amsbook.doc&
amslatex.tex\\
app.tex&
chap1.tex&
chap2.tex\\
extradef.tex&
pref.tex&
test.bib\\
testart.bbl&
testart.tex&
testbook.bbl\\
testbook.tex&
theorem.doc&
verbatim.doc\\
\end{tabular}
\end{center}
}

\smallskip
\noindent The subdirectory \verb+fontsel+ contains the following files:

{\tt
\begin{center}
\begin{tabular}{l@{\hspace{3em}}l@{\hspace{3em}}l}
array.sty&
basefont.tex&
concrete.doc\\
concrete.sty&
euscript.sty&
fontdef.max\\
fontdef.ori&
fontsel.bug&
fontsel.tex\\
lfonts.new&
margid.sty&
newlfont.sty\\
nomargid.sty&
oldlfont.sty&
preload.med\\
preload.min&
preload.ori&
readme.mz3\\
syntonly.sty&
tracefnt.sty&
\end{tabular}
\end{center}
}

\smallskip
\noindent The subdirectory \verb+inputs+ contains the following files:


{\tt
\begin{center}
\begin{tabular}{l@{\hspace{3em}}l@{\hspace{3em}}l}
amsalpha.bst&
amsart.sty&
amsart10.sty\\
amsart11.sty&
amsart12.sty&
amsbk10.sty\\
amsbk11.sty&
amsbk12.sty&
amsbook.sty\\
amsbsy.sty&
amscd.sty&
amsfonts.sty\\
amsplain.bst&
amssymb.sty&
amstex.sty\\
amstext.sty&
ctagsplt.sty&
fontdef.ams\\
intlim.sty&
mrabbrev.bib&
nonamelm.sty\\
nosumlim.sty&
preload.max&
righttag.sty\\
theorem.sty&
thp.sty&
verbatim.sty\\
\end{tabular}
\end{center}
}

\medskip

\goodbreak
\section{Copying files to the appropriate directories}

All the files in the \verb+doc+ subdirectory can be copied to any  documentation
directory or other directory where you would like to keep them. 

The files in the \verb+inputs+ subdirectory and the \verb+fontsel+ subdirectory
should all be copied to the directory where your implementation of \TeX\ looks
for input files. 

\section{What the various files are for}

Once the files have been copied to the appropriate places, the
main ones of interest will be:
\[\begin{tabular}{ll}
$\left.\begin{tabular}{l}
lfonts.new\\ fontdef.*\\ preload.*\\ basefont.tex
\end{tabular}\right\}$&
        Used in creating a new \LaTeX\ format file\\[6pt]
\ \ amstex.sty&
        \LaTeX\ documentstyle option file\\[3pt]
\ \ amsart.sty&
        Main documentstyle for AMS article formatting\\[3pt]
\ \ amsbook.sty&
        Main documentstyle for AMS book formatting\\[3pt]
\ \ amslatex.tex&
        {\it\amslatex\ User's Guide}\\[6pt]
$\left.\begin{tabular}{l}
app.tex\\ chap*.tex\\ pref.tex\\ test*.*
\end{tabular}\right\}$&
Examples
\end{tabular}\]

Use of the \amslatex\ package is dependent on the font selection scheme of
Mittelbach and Sch\"opf described in {\it TUGboat} {\bf 11}, no.~2, June
1990, pp.~297--305. 
This means that you need to receive a new \LaTeX\ format file based on
Mittelbach and Sch\"opf's scheme (some of the distributors of \TeX\ will
provide this), or make one yourself (instructions follow), using {\sc initex},
a version of \TeX\ with no preloaded format which is included with most
implementations of \TeX.

The {\it User's Guide} file \verb+amslatex.tex+ can be typeset using the old
\LaTeX\ format file \verb+lplain.fmt+, so if you want to you can typeset it
using \LaTeX\  in the usual way and read it before proceeding further.  It
gives a general description of the \amslatex\ project and the features
available to users.

\section{Making a new format file}

In order to create a new \LaTeX\ format file that uses the new font selection
scheme, you must first rename the file \verb+lfonts.tex+ which was distributed
with your original \LaTeX\ distribution to another name. This file is in the
directory where your implementation of \TeX\ looks for input files; e.g. for
PC\TeX\ it would be  \verb+\pctex\texinput+. You want to name it something 
such as \verb+olfonts.tex+ (``O'' for ``original'').  You may also want to
rename the file \verb+lplain.fmt+, in the directory where your \TeX\ looks for
format files, to \verb+olplain.fmt+,  if you want to continue to be able to use
the old version of \LaTeX\ as well as the new version.

Then follow the directions for creating a new \LaTeX\ format file, using  {\sc
initex}, in the documentation for your implementation of \TeX. The directions
will vary from one implementation of \TeX\ to another, but what you want to do
is \TeX\ the file \verb+lplain.tex+ using the {\sc initex} option. 
(\verb+lplain.tex+ is a standard \LaTeX\ file which should already be installed
on your system.  It is not included with this distribution.)

After some initial processing, {\sc initex} will stop and ask for another
filename because it cannot find the file \verb+lfonts.tex+. (Remember, you
renamed it according to the first paragraph of this section.) When it stops, 
just type 

\usertype{lfonts.new}

\noindent in response to the prompt (``Please type another input file name'') and
continue.  There are three other files that will be input in a similar
way. They are indicated in the following table:
\begin{eqnarray*}
\mbox{In place of:}&& \mbox{Substitute:}\\
\mbox{fontdef.tex}&& \left\{
\begin{tabular}{lp{16pc}}
% We use \rightskip here instead of \raggedright because
% \raggedright causes problems with \\ commands.
fontdef.ori& \rightskip0pt plus3em
             If you want to use only CM and \LaTeX\ fonts.\\
fontdef.ams& \rightskip0pt plus3em
             If you want to use the basic math fonts from
             the AMSFonts 2.0 collection (Euler Fraktur, bold math,
             extra math symbols MSAM and MSBM).\\
fontdef.max& \rightskip0pt plus3em 
             If you want to use other AMSFonts (Euler Roman or
             Script, cyrillic text fonts), or Concrete fonts.
\end{tabular}\right.\\ % end of second line of eqnarray
\mbox{preload.tex}&& \left\{
\begin{tabular}{lp{16pc}}
preload.min& \rightskip0pt plus3em 
             To preload fewer fonts (uses less font memory,
             but may increase processing time).\\
preload.ori& \rightskip0pt plus3em 
             To preload all basic \LaTeX\ fonts (leaves less
             memory for other fonts, faster processing time).
\end{tabular}\right.\\ % end of third line of eqnarray
\mbox{xxxlfont.sty}&& \left\{
\begin{tabular}{lp{16pc}}
basefont.tex& \rightskip0pt plus3em If you intend to use AMSFonts 2.0.\\
newlfont.sty& \rightskip0pt plus3em If you don't intend to use AMSFonts 2.0.
\end{tabular}\right.
\end{eqnarray*}
\chardef\bsl=`\\
Note: When basefont.tex is used the following math symbols will be
undefined: {\tt\bsl mho},  {\tt\bsl Join},  {\tt\bsl
Box},  {\tt\bsl Diamond},  {\tt\bsl leadsto},  {\tt\bsl lhd}, 
{\tt\bsl unlhd},  {\tt\bsl rhd}, and {\tt\bsl unrhd}.  Alternate
versions of all these symbols exist in the AMSFonts math symbol fonts
(which presumably you have, if you used basefont.tex) and they can be
defined using {\tt\bsl newsymbol}.  See the {\it User's Guide}.

After quite a bit of further processing you will wind up at a * prompt,
whereupon you type 

\vskip-6pt
\usertype{$\backslash$dump}

\noindent and return.  You will now have a file called \verb+lplain.fmt+.
Depending on your implementation of \TeX, this file may be in the directory
where \TeX\ looks for format files or it may be in the currently connected
directory. If the latter is true, you must copy it to the directory where \TeX\
will look for format files (see your \TeX's documentation for the name of that
directory).

You might also want to copy the transcript file \verb+lplain.log+ into the
directory with the format file, for future reference.

\section{Using the new format file}

The test files \verb+testart.tex+ and \verb+testbook.tex+, and the other files
in the \verb+doc+ subdirectory which they input, are useful for testing your
new \verb+lplain.fmt+ format file.

You can use your new format file to typeset them,
following the instructions for using a format file with your implementation of
\TeX. For example, with PC\TeX\ and many other implementations of \TeX, the
command would be

\usertype{tex\ \ \&lplain\ \ testart.tex\quad\quad ({\it or\/} testbook.tex)}

\noindent but it may be different for your implementation. These are examples
of using the \verb+amsart+ and \verb+amsbook+ style options.


\section{Using the old format file}

Some of your old \LaTeX\ files may be incompatible with the new format
file. If you find this to be the case, you can ordinarily typeset such
a file by adding ``{\tt oldlfont}'' to the documentstyle options list.
Alternatively, of course, you could use the old \LaTeX\ format file
instead of the new one.  Assuming you saved the previous
\verb+lplain.fmt+  file under the name \verb+olplain.fmt+, you would
use ``\&olplain'' instead of  ``\&lplain'' in the command line.


\section{Typesetting the {\it User's Guide}}

The file \verb+amslatex.tex+ is the source file for the {\it \amslatex\
User's Guide}.  If you haven't already done so, you should now typeset this file
using either of the methods described above.  (Remember that
\verb+amslatex.tex+ can be typeset using either the old or the new format
file.)  It is essential that you read through the {\it User's Guide\/} before
attempting to use the \amslatex\ package.


\section{Getting Help}

Questions or comments concerning \amslatex\ 
can be directed to:


\vskip1pc
\parindent1in\parskip0pt

Technical Support

American Mathematical Society

201 Charles Street

P.O. Box 6248

Providence, RI 02940

800-321-4AMS (321-4267)\quad or \quad 401-455-4080

Internet: tech-support\atsign/Math.AMS.com

\end{document}
