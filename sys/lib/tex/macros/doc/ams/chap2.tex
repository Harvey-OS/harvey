%%%%%%%%%%%%%%%%%%%%%%%%%%%%%%%%%%%%%%%%%%%%%%%%%%%%%%%%%%%%%%%%%%%%%%%%%%%%
% CHAP2.TEX						    July 1990      %
%                                                                          %
% This file is part of the AMS-LaTeX Version 1.0 distribution              %
%   American Mathematical Society, Technical Support Group,                %
%   P. O. Box 6248, Providence, RI 02940                                   %
%   800-321-4AMS (321-4267) or 401-455-4080                                %
%   Internet: Tech-Support@Math.AMS.com                                    %
%%%%%%%%%%%%%%%%%%%%%%%%%%%%%%%%%%%%%%%%%%%%%%%%%%%%%%%%%%%%%%%%%%%%%%%%%%%%
\chapter{Keldysh Pencils}
\section{Holomorphic operator-valued functions}

The main object of study in this book is polynomial operator pencils
(operator polynomials). However, it is more convenient for us to give
certain definitions and results for the more general case of
holomorphic operator-valued functions.

Let $U$ be a domain in $\bold C$, $\cal{B}$ be a complex Banach space,
and $f(\lambda)$ a $\cal{B}$-valued function defined in $U$. Such a
function $f(\lambda)$ is said to be {\it strongly \rom(weakly\rom)
holomorphic} in $U$ if for any $\lambda_0\in U$ the strong (weak)
limit
\begin{equation}
\lim_{\lambda\to\lambda_0}\frac{f(\lambda)-f(\lambda_0)}{\lambda-\lambda_0}\
(=f'(\lambda_0))
\end{equation}
exists.

Obviously,  $f(\lambda)$ is weakly holomorphic if and only if any function
$\psi(f(\lambda))$ is holomorphic, where $\psi\in\cal{B}^*$.

An operator-valued function $A(\lambda)$ $(\lambda\in U)$ with values in
$L(\cal H)$ is said to be {\it uniformly \rom(strongly, weakly\rom) holomorphic}
 if for any $\lambda_0\in U$ the uniform (strong, weak) limit
\begin{equation}
\lim_{\lambda\to\lambda_0}\frac{A(\lambda)-A(\lambda_0)}{\lambda-\lambda_0}
\ (=A'(\lambda_0))
\end{equation}
exists.

\endinput
