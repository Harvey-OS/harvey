%%%%%%%%%%%%%%%%%%%%%%%%%%%%%%%%%%%%%%%%%%%%%%%%%%%%%%%%%%%%%%%%%%%%%%%%%%%%
% PREF.TEX						    July 1990      %
%                                                                          %
% This file is part of the AMS-LaTeX Version 1.0 distribution              %
%   American Mathematical Society, Technical Support Group,                %
%   P. O. Box 6248, Providence, RI 02940                                   %
%   800-321-4AMS (321-4267) or 401-455-4080                                %
%   Internet: Tech-Support@Math.AMS.com                                    %
%%%%%%%%%%%%%%%%%%%%%%%%%%%%%%%%%%%%%%%%%%%%%%%%%%%%%%%%%%%%%%%%%%%%%%%%%%%%
\chapter*{Preface}

This book is a study of polynomial operator pencils,
i.e., operator polynomials of the form
\begin{equation*}
A(\lambda)=A_0+\lambda A_1+\cdots+\lambda^nA_n,
\end{equation*}
where $\lambda$ is a spectral parameter and $A_0,\dots,A_n$ are linear
operators acting in a Hilbert space $\cal H$. In the simplest cases 
$A(\lambda)=
A-\lambda I$ and $A(\lambda)=I-\lambda A$ we come to the usual (linear)
spectral problems. 

Spectral problems for polynomial pencils arise naturally in diverse
areas of mathematical physics (differential equations and boundary value
problems, controllable systems, the theory of oscillations and waves,
elasticity theory, and hydromechanics). This explains the steady interest
in these problems over the last 35 years.

A consideration of the simplest model---a matrix pencil---enables us
to see the  essential differences between nonlinear and linear
spectral problems. If the coefficients of the pencil $A_k$
$(k=0,\dots,n)$ are matrices of order $m$, $\det A_n\neq 0$, and all
the roots $\{\lambda_k\}_1^{nm}$ of the characteristic equation $\det
A(\lambda) =0$ are distinct, then the pencil $A(\lambda)$ has $nm$
eigenvectors $\{\varphi _k\}_1^{nm}$ (i.e.,
$A(\lambda_k)\varphi_k=0$). One possible approach is to single out in
the system $\{\varphi_k\}_1^{nm}$ various subsystems
$\{\varphi_{k_j}\}_{j=1}^m$ forming bases in $\bold C^m$. Another
approach leads to the consideration of the system of vectors
$\{\varphi_k,\lambda_k \varphi_k,\dots,\lambda_k^{n-1}\varphi_k)$
$(k=1,\dots,nm)$, which forms a basis in the space $\bold C^{nm}$.
Both these approaches are fruitful and admit far-reaching
generalizations.

\endinput
