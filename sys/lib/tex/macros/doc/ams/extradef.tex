%%%%%%%%%%%%%%%%%%%%%%%%%%%%%%%%%%%%%%%%%%%%%%%%%%%%%%%%%%%%%%%%%%%%%%%%%%%%
% EXTRADEF.TEX						    July 1990      %
%                                                                          %
% This file is part of the AMS-LaTeX Version 1.0 distribution              %
%   American Mathematical Society, Technical Support Group,                %
%   P. O. Box 6248, Providence, RI 02940                                   %
%   800-321-4AMS (321-4267) or 401-455-4080                                %
%   Internet: Tech-Support@Math.AMS.com                                    %
%%%%%%%%%%%%%%%%%%%%%%%%%%%%%%%%%%%%%%%%%%%%%%%%%%%%%%%%%%%%%%%%%%%%%%%%%%%%
%%
%% Special definitions for use in producing some parts of the
%% AMS-LaTeX User's Guide.
% We allow some slop at the right margin because we have some
% long control sequence names and verbatim text to deal with.
\hfuzz2pc

\makeatletter

% Change hyphenation inside \tt text back to normal:
\let\-=\@@hyph  \let\@dischyph=\@@hyph  \let\@nohyphens\@gobbletwo
{\footnotesize\tt \hyphenchar\the\font=`\- 
\small\tt \hyphenchar\the\font=`\- 
\normalsize\tt \hyphenchar\the\font=`\-
\large\tt \hyphenchar\the\font=`\-
}

\chardef\bslash=`\\ % p. 424, TeXbook
% \kern\z@ here inhibits hyphenation except at \-

% control sequence
\def\cs#1{{\tt\bslash\kern\z@#1}\index{#1@{\tt\bslash#1}}}

% LaTeX documentstyle name
\def\sty#1{{\tt\kern\z@#1}\index{#1@{\tt{}#1} documentstyle}}

% LaTeX option name
\def\opt#1{{\tt\kern\z@#1}\index{#1@{\tt{}#1} option}}

% environment name
\def\env#1{{\tt\kern\z@#1}\index{#1@{\tt{}#1} environment}}

% file name
\def\fn#1{{\tt\kern\z@#1}\index{#1@{\tt{}#1}}}

% to index a control sequence without printing it
\def\indexcs#1{\index{#1@{\tt\bslash#1}}}

% Macros for the various macro package names.  \AmS and \LaTeX are
% defined using \protect to avoid writing long strings to the .aux
% file, which would be a problem on some computers.
\def\AmS{\protect\pAmS}            \def\LaTeX{\protect\pLaTeX}
\def\pAmS{{\the\textfont2
        A\kern-.1667em\lower.5ex\hbox{M}\kern-.125emS}}
\def\pLaTeX{{\rm L\kern-.36em\raise.3ex\hbox{\the\scriptfont0 A}\kern-.15em
    T\kern-.1667em\lower.7ex\hbox{E}\kern-.125emX}}

\def\amstex/{\AmS-\TeX}         \def\amslatex/{\AmS-\LaTeX{}}
\def\latex/{\LaTeX{}}           \def\pictex/{PIC\TeX}
\def\tex/{\TeX}                 \def\jt/{{\it Joy of \TeX}}
\def\bibtex/{{\sc Bib\kern-.1em\TeX}}     \def\tugboat/{{\it TUGboat\/}}
\def\amsfonts/{AMSFonts}

% `Meta' macro.
\def\<#1>{{$\langle$\it#1\/$\rangle$}}

% Indent a little on the left in the verbatim environment.
\def\verbatim{\interlinepenalty\@M \@verbatim
  \leftskip\@totalleftmargin\advance\leftskip2pc
  \frenchspacing\@vobeyspaces \@xverbatim}

% To introduce permissible breakpoints for line breaks in verbatim text:
\def\5{\penalty500 }

% A modified form of \sloppypar, to be used at the end of a paragraph:
\def\sloppypar{{\tolerance9999\par}}

\makeatother
\endinput
