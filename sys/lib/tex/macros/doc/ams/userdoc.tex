% USERDOC.TEX						February 1990
%
% A documentation file for use of AMSFonts 2.0, called AMSFonts User's Guide.
% To TeX this file, you will need the files USERDOC.DEF, USERDOC.CYR, 
%        and USERDOC.FNT.
% 
% NOTE: This file should be typeset using Plain TeX, NOT AMS-TeX or LaTeX.
% 
% American Mathematical Society, Technical Support Group, P. O. Box 6248,
%        Providence, RI 02940
% 800-321-4AMS or 401-455-4080;  Internet: Tech-Support@Math.AMS.com
%
\input userdoc.def
\input amssym.def
\input amssym
\parskip=\smallskipamount

\def\amstexreleasedate{January 1990}
\def\fontsreleasedate{January 1990}

\pageno=1
\def\ppcontents{1}
\def\pphistory{6}
\def\ppAMSTeX{8}
\def\ppnonAMSTeX{11}
\def\ppcyrillic{13}
\def\ppsymbols{17}
\def\ppfurtherinfo{22}
\def\ppreflist{23}

\makeatletter
\def\@period{.}
\def\@gobble#1{}
\def\furtherinfo{\futurelet\test@period\check@period}
\def\check@period{the section {\bf For further information}%
  \ifx\test@period .\def\next{.\@gobble}\else \def\next{''}\fi \next }

\def\@Note #1{\def\endx{\ifhmode \unskip\endgraf \fi \endgroup }%
  \ifdim\lastskip>\medskipamount \else \removelastskip\medskip \fi
  \begingroup
  \leftskip=3pc \rightskip=\leftskip
  \interlinepenalty=10000
  \noindent
  {\bf\ignorespaces#1\unskip: } \ignorespaces }
\def\Note{\@Note{Note}}
\def\Warn{\@Note{Warning}}

\def\<#1>{$\langle${\rm #1\/}$\rangle$}

\def\filename#1{\leavevmode\hbox{\tt\ignorespaces#1\unskip}}
\def\fontname#1{\leavevmode\hbox{\ignorespaces#1\unskip}}       
\makeatother

\input cyracc.def
\font\tencyr=wncyr10
\def\cyr{\tencyr\cyracc}


%%%%%%%%%%%%%%%%%%%%%%%%%%%%%%%%%%%%%%%%%%%%%%%%%%%%%%%%%%%%%%%%%%%%%%%%

\maintitle AMSFonts Version 2.0\Dash User's Guide\\
  \fontsreleasedate<

Standard distributions of \TeX{} ordinarily come with all the fonts
specified in |PLAIN.TEX|, and they may also come with a number of
additional fonts intended for use with \LaTeX.  Additional fonts
designed for use in mathematics and defined in \AmSTeX{} are not always
included among such font collections.  For this reason, the \AMS{} has
compiled a collection, known as AMSFonts, which contains fonts of
symbols and several alphabets corresponding to symbols and alphabets
used in AMS publications, including the MathSci online database.

\Warn  The fonts in this collection are suitable for use with
  \AmSTeX{} Version~2.0, but are not compatible with earlier versions
  of \AmSTeX.  If you are using an older version of \AmSTeX{} and are
  intending to use fonts from this Version~2.0 collection, you should
  obtain a new version of \AmSTeX, if you haven't already done so.
  Similarly, earlier versions of this collection are not compatible
  with \AmSTeX{} Version~2.0.
\endx


\section Contents of this document

\begingroup
\parfillskip=\normalparindent
\parindent=\normalparindent

\item{\bull} Contents of the AMSFonts collection \dotsfill\ \ppcontents{}

\item{\bull} History of these fonts \dotsfill\ \pphistory{}

\item{\bull} How to use AMSFonts 2.0 with \AmSTeX{} 2.0 \dotsfill\ \ppAMSTeX{}

\item{\bull} How to use AMSFonts 2.0 without \AmSTeX{}
  \dotsfill\ \ppnonAMSTeX{}

\item{\bull} Installing and using cyrillic \dotsfill\ \ppcyrillic{}

\item{\bull} Installing and using the extra symbols \dotsfill\ \ppsymbols{}

\item{\bull} For further information \dotsfill\ \ppfurtherinfo{}

\item{\bull} References \dotsfill\ \ppreflist{}

\endgroup


%%%%%%%%%%%%%%%%%%%%%%%%%%%%%%%%%%%%%%%%%%%%%%%%%%%%%%%%%%%%%%%%%%%%%%%%

\title Contents of the AMSFonts collection

The AMSFonts collection contains the following fonts, in the sizes
indicated:

\begingroup
\raggedright
\parskip=0pt
\item{\bull} The Euler family, all but \fontname{EUEX} in 5, 6, 7, 8, 9,
  and 10 point:
\itemitem{--} Fraktur (German), medium-weight and bold (\fontname{EUFM}
  and \fontname{EUFB})
\itemitem{--} ``Roman'' cursive, medium-weight and bold (\fontname{EURM}
  and \fontname{EURB})
\itemitem{--} Script, medium-weight and bold (\fontname{EUSM} and
  \fontname{EUSB})
\itemitem{--} Euler-compatible extension font (\fontname{EUEX}),
  in 7, 8, 9, and 10 point

\item{\bull} Additional sizes of some Computer Modern math fonts
  (the 10-point fonts are included in standard \TeX\ distributions):
\itemitem{--} bold math italic (\fontname{CMMIB}), in 5, 6, 7, 8, and 9 point
\itemitem{--} bold math symbols (\fontname{CMBSY}), in 5, 6, 7, 8, and 9 point
\itemitem{--} math extension font (\fontname{CMEX}), in 7, 8, and 9 point
              (the 10-point font is included in standard \TeX\ distributions)

\item{\bull} Extra math symbols, in 5, 6, 7, 8, 9, and 10 point:
\itemitem{--} first series, medium-weight (\fontname{MSAM})
\itemitem{--} second series, including Blackboard Bold, medium-weight
  (\fontname{MSBM})

\item{\bull} Cyrillic, developed at the University of Washington
\itemitem{--} lightface (\fontname{WNCYR}), in 5, 6, 7, 8, 9, and 10 point
\itemitem{--} bold (\fontname{WNCYB}), in 5, 6, 7, 8, 9, and 10 point
\itemitem{--} italic (\fontname{WNCYI}), in 5, 6, 7, 8, 9, and 10 point
\itemitem{--} caps and small caps (\fontname{WNCYSC}), in 10 point
\itemitem{--} sans serif (\fontname{WNCYSS}), in 8, 9, and 10 point

\item{\bull} Computer Modern caps and small caps (\fontname{CMCSC}),
  in 8 and 9 point
 (the 10-point font is included in standard \TeX\ distributions)

\item{\bull} The ``dummy font,'' used in \AmSTeX{} for syntax checking,
  exists only as metrics (\filename{DUMMY.TFM})

\item{\bull} Other files needed to use these fonts:
\itemitem{--} \filename{AMSSYM.TEX}, a file defining the symbols in fonts
  \fontname{MSAM} and \fontname{MSBM}
\itemitem{--} \filename{AMSSYM.DEF}, a file that loads the fonts
  \fontname{MSAM}, \fontname{MSBM} and \fontname{EUFM} and defines some
  control sequences required by \filename{AMSSYM.TEX}
\itemitem{--} \filename{CYRACC.DEF}, a file containing definitions needed for
  proper access to characters in the cyrillic fonts

\item{\bull} Other useful files:
\itemitem{--} \filename{USERDOC.TEX}, the source file for this User's Guide
\itemitem{--} \filename{USERDOC.CYR}, the source file for the table showing
  cyrillic input conventions, input by \filename{USERDOC.TEX}
\itemitem{--} \filename{USERDOC.FNT}, the source file for the tables of
  the principal 10-point fonts in the AMSFonts collection, input by
  \filename{USERDOC.TEX}; this file may also be \TeX{}ed by itself
\itemitem{--} \filename{USERDOC.DEF}, the macros used to format this
  User's Guide
\endgraf
\endgroup       % end \raggedright

\noindent
Each font at a particular size is provided in seven standard \TeX{}
magnifications, magsteps 0~through~5, including magstephalf.  The AMSFonts
package for the IBM~PC and compatibles includes all magnifications.  For
use with {\it Textures\/} on the Macintosh, the Standard AMSfonts package
includes only magsteps 0~and~1; the Extended AMSFonts package includes all
seven magnifications.  All instances of every font have been newly
generated for this release of the AMSFonts collection.

The philosophy under which the Euler fonts were implemented was different
from that used for Computer Modern, and the result is a lower degree of
``meta-ness''.  For that reason, the appearance of these fonts is not
very good at small sizes when output on low-resolution devices, in
particular on screens.  Even so, the fonts are included in AMSFonts in all
the sizes and magnifications offered, on the assumption that the printed
output will be prepared on a device of higher resolution (at least 300dpi)
where this effect will not be noticeable.

Figure 1 shows the layouts for the 10-point fonts listed above.
Where both medium-weight and bold versions are listed, only the medium-weight
layout is shown.  The bold fonts have the same layout as the corresponding
medium-weight versions.


%%%%%%%%%%%%%%%%%%%%%%%%%%%%%%%%%%%%%%%%%%%%%%%%%%%%%%%%%%%%%%%%%%%%%%%%

\newpage
\normalbottom

\figuretitle Figure 1a. {\rm Layouts of the AMSFonts\Dash Euler}

\Note  The charts shown in this figure were set with the 10-point versions
  of the indicated fonts, then photographically reduced. 
\endx

\vskip 0pt plus \vsize
\begingroup
\leftskip 1in
\obeylines
  Charts to appear here:
  Euler: EUFM10, EURM10, EUSM10, EUEX10
\endgroup
\vskip 0pt plus \vsize
\DoHrule
\eject

\figuretitle Figure 1b. {\rm Layouts of the AMSFonts\Dash
    Computer Modern bold math; extra symbols}

\vskip 0pt plus \vsize
\begingroup
\leftskip 1in
\obeylines
  Charts to appear here:
  Symbols: MSAM10, MSBM10
  Computer Modern bold math: CMMIB10, CMBSY10
\endgroup
\vskip 0pt plus \vsize
\DoHrule
\eject

\figuretitle Figure 1c. {\rm Layouts of the AMSFonts\Dash 
    Computer Modern math extension; caps/small caps; cyrillic}

\vskip 0pt plus \vsize
\begingroup
\leftskip 1in
\obeylines
  Charts to appear here:
  Math extension: CMEX10
  Caps/small caps: CMCSC10
  Cyrillic: WNCYR10, WNCYI10
\endgroup
\vskip 0pt plus \vsize
\DoHrule
\eject

\raggedbottom

%%%%%%%%%%%%%%%%%%%%%%%%%%%%%%%%%%%%%%%%%%%%%%%%%%%%%%%%%%%%%%%%%%%%%%%%


\section Font naming conventions

Developers of fonts for use with \TeX, at least those using \MF,
generally try to make the names distinctive, so that a user will know
the origin of the font by the font name.  For most of the fonts in the
AMSFonts collection, the first two letters identify the font source,
as follows:

\begingroup
\raggedright
\parskip=0pt
\item{\bull} ``CM'': These fonts are based directly on the
  specifications for Knuth's Computer Modern fonts, as described
  in Volume~E of \CandT{} [DEK86E].
\item{\bull} ``EU'': These are members of the Euler family,
  described below.
\item{\bull} ``MS'': These fonts of math symbols were developed
  by or under the direction of the AMS staff.
\item{\bull} ``WN'': These cyrillic fonts were developed at or
  under the direction of the University of Washington Humanities and Arts
  Computing Center, and are distributed with their permission.
\endgraf
\endgroup

\noindent
DUMMY is a special case; it was developed as part of the Stanford
University \TeX{} Project, and follows no particular naming
convention.

For information on \fontname{CM} fonts other than those specifically named
here, and on other fonts in general, the \TUG{} is the best source.
For TUG's address, see \furtherinfo.


%%%%%%%%%%%%%%%%%%%%%%%%%%%%%%%%%%%%%%%%%%%%%%%%%%%%%%%%%%%%%%%%%%%%%%%%

\title History of these fonts

When the AMS began using \TeX{} to produce its publications, the available
complement of symbols was found to be inadequate.  Several alphabets
used extensively as symbols were not available either.  While
development of the symbols could be undertaken by in-house personnel,
using the existing \TeX{} symbol font as a model, the creation of new
Fraktur and script alphabets required the assistance of someone with
experience in font design.


\section Euler

With Donald Knuth's assistance and encouragement, Hermann Zapf, one of the
premier font designers of this century, was commissioned to create designs
for Fraktur and script, and for a somewhat experimental, upright cursive
alphabet that would represent a mathematician's handwriting on a blackboard
and that could be used in place of italic.  The designs that resulted were
named Euler, in honor of Leonhard Euler, a prominent mathematician of the
eighteenth century.  Zapf's designs were rendered in \MF{} code by graduate
students at Stanford, working under Knuth's direction; the process by which
the \MF{} fonts were implemented is described in a report by David Siegel
[DRS85].  The Euler fonts were designed to be used as symbols; they are not
intended for setting running text.

The Fraktur face of the Euler family has been used by the AMS for several
years in its production.  However, no extensive test or use had been made
of the script or cursive until Knuth decided that they should be used in a
textbook, {\sl Concrete Mathematics}, written by him and two co-authors
[GKP88].  During the course of preparing that book, a number of errors,
particularly in spacing parameters affecting the placement of sub- and
superscripts, were discovered.  All these errors have been corrected in the
new medium-weight versions of the Euler fonts (almost no boldface symbols
were used in {\sl Concrete Mathematics\/}).  Knuth also noticed that the
style of some symbols in the Computer Modern extension font, in particular
the integral sign, was too slanted to be attractive with Euler, and
consequently he prepared a new (partial) extension font for use with Euler.
Knuth has described his experience with the Euler fonts in a \TUB\/ article
[DEK89].  In the article he also identifies the macros he used and where they
can be obtained.


\section Additional Computer Modern fonts for use in math

Only the 10-point size of the Computer Modern bold math italic (which
includes Greek), symbol, and math extension fonts are included in standard
distributions of \TeX{}.  Since these symbols are often needed in
mathematics, other sizes have been constructed, using the principles
demonstrated in Knuth's {\sl Computer Modern Typefaces\/} [DEK86E], and
included in the AMSFonts collection.


\section Symbols

Two fonts of ``extra'' symbols are included in the AMSFonts collection.
These are named \fontname{MSAM} and \fontname{MSBM}, and have been newly
implemented in ``new'' \MF{} (\mf84); they replace earlier fonts (named
\fontname{MSXM} and \fontname{MSYM}) that were defined in old \MF{}
(\mf79)\null.  These fonts contain symbols needed in the publishing program
of the AMS, including the MathSci online database, and include the
uppercase letters of an alphabet known as Blackboard Bold
($\Bbb A, \dots, \Bbb Z$).


\section Cyrillic

Titles of books reviewed in \MR\/ are traditionally rendered in their
original language.  For books published in Russian or other Slavic
languages, this frequently requires use of the cyrillic alphabet.
A cyrillic font was developed at AMS using \MF\/79 with the \fontname{AM}
fonts as a model.  This font was organized in a manner suitable for use with
the transliteration scheme adopted by {\sl MR\/} in 1980, and contained only
those letters which appear in current mathematical literature.
In particular, this meant that the letters dropped from the Russian
alphabet after the Revolution of 1917, and some letters used in non-Slavic
languages now rendered in cyrillic (such as Azerbaijani, from which no
mathematical literature is currently reviewed in {\sl MR\/}) were absent.

In 1988, the Humanities and Arts Computing Center of the University of
Washington undertook a font development project for support of scholars
in Slavic languages.  The fonts developed through this project include
several different font layouts.  One layout is based on that of the
original AMS cyrillic augmented with `{\cyr \u\i}' (cyrillic short `i'),
`{\cyr\"e}' (umlauted `e'), and several pre-Revolutionary letters.
The fonts with the AMS layout are included in the AMSFonts collection
with the permission of the University of Washington developers.  For
information on cyrillic fonts with other layouts, see \furtherinfo.

The cyrillic fonts are based on Computer Modern letter shapes.
Type styles include ordinary upright, bold (based on \fontname{CM} bold
extended), caps and small caps, italic, and upright sans serif.
The principal text fonts (upright, italic and boldface) are present in
sizes from 5 through 10 point;
sans serif is in sizes 8, 9 and 10 point;
caps and small caps are in 10 point only.


\section Caps/small caps

The font \fontname{CMCSC10} is referenced in \filename{PLAIN.TEX} and
should be included in all standard \TeX{} distributions.  However, Knuth
did not generate this font in any other sizes.  The AMSFonts collection
includes 8 and 9-point sizes, generated according to the same principles as
other \fontname{CM} fonts of these sizes.


\section Dummy font

This is a pseudo-font, which exists only as a set of metrics.
Mainframe \TeX{} distributions contain this font in ``property list''
(\filename{.PL}) format, a human-readable file that contains everything in the
corresponding \filename{.TFM} file.  (Transformation between \filename{.PL}
and \filename{.TFM} formats can be accomplished by the programs |PLtoTF| and
|TFtoPL|; for convenience, the AMSFonts collection includes the file
\filename{DUMMY.TFM}\null.)  Specifically, the dummy font contains no
ligature or kerning information, and all dimensions and parameter values
are set to zero.

The dummy font is used in \AmSTeX{} to implement ``syntax checking.''
(Syntax checking is activated by the |\printoptions| command as described
in \JoT{} [MDS86].)
In this mode, the dummy font replaces all the usual ``printing'' fonts,
so that \TeX{} never accumulates any text to be set, and never tries to
write out a page, but in the process of reading the input file, checks
all control sequences for syntactic correctness.  In this mode, an input
file will be processed perhaps 30 percent faster than if it were actually
being set.  However, some errors and conditions are not detected during
a syntax check; in particular, overfull boxes will not be detected until
setting actually occurs.


%%%%%%%%%%%%%%%%%%%%%%%%%%%%%%%%%%%%%%%%%%%%%%%%%%%%%%%%%%%%%%%%%%%%%%%%

\title How to use AMSFonts 2.0 with \AmSTeX{} 2.0

In \JoT{}, Michael Spivak describes various fonts that are used in
mathematics in addition to the fonts provided with the standard
distributions of \TeX.  Two references in particular are of interest with
respect to AMSFonts: the section {\bf Fonts in math mode} in Chapter 20,
and Appendix F, {\bf Further fonts}.  The first describes the use of
letters from alphabets, including Fraktur, and the second, mostly
nonalphabetic symbols.

\Warn  This information is effective as of \AmSTeX{} version~2.0,
  \amstexreleasedate{} and the 2nd edition of \JoT\/ [MDS90].
  The handling of non-\filename{PLAIN} fonts in earlier versions is
  different from what is described here.  If you have an earlier
  version of \AmSTeX, it is highly recommended that you obtain \AmSTeX{}
  Version~2.0, which is available from AMS and most \TeX{} distributors.
\endx

Instructions for using the fonts of the AMSFonts collection with \AmSTeX{}
are also given in the {\sl User's Guide to \AmSTeX{} Version~2.0\/} [AMS90]
and in Appendix~F of editions of \JoT{} [MDS90] dated 1990 or later.

Additional fonts to be used with \AmSTeX{} should be specified at the top
of the document input file, in what is known as the ``preamble.''  The
arrangement of commands at the top of an input file is the following:
\begintt
\input amstex
|<preamble commands>
\documentstyle{...}
\endtt

\AmSTeX{} provides a simple method for accessing most of the fonts in the
AMSFonts collection.  The two extra symbol fonts and Euler Fraktur are
loaded automatically by the preprint style (\filename{AMSPPT.STY})\null.
If you are using \AmSTeX{}, but not the preprint style, the method used to
load these fonts and define the associated symbol names depends on how many
symbols will be needed.  If a lot of the symbols will be needed, or
you aren't worried about memory space and just want to do what is easiest,
all three fonts will be loaded and the symbol names defined if you type the
command |\UseAMSsymbols| in the preamble.  This will load the file
\filename{AMSSYM.TEX}, in which all the symbol names (more than 200 of them)
are defined.  If only a few symbols from these fonts are needed, the
commands |\loadmsam|, |\loadmsbm|, and |\loadeufm| will load the
medium-weight versions of the two extra symbol fonts and Euler Fraktur
respectively.  The command |\newsymbol| can then be used to define just
those symbols that are needed; its use is described below, in the section
{\bf Installing and using the extra symbols}.

\Warn  Additional fonts from the AMSFonts collection can be accessed
  easily in \AmSTeX.
  However, users should be aware that \TeX{} limits the number of
  math mode font families to 16, of which 10 are predefined in \AmSTeX.
  Only those additional families should be activated that will actually
  be used in a document, to avoid exceeding the limit.
\endx

Two sizes of fonts, suitable for body text and for passages requiring smaller
type (e.g.\ abstracts and footnotes), are incorporated in the preprint style
\filename{AMSPPT.STY}\null.  These are accessed through the control sequences
|\tenpoint| and |\eightpoint|, which are ordinarily referred to only by
commands that identify the kind of text being input (e.g.\ |\title|,
|\abstract|, |\footnote|).  Most fonts in the AMSFonts collection have
|\load...|\ instructions defined in \AmSTeX{} and will be accessed properly for
use with the preprint style when the |\load| instructions are included in the
preamble of the document input.  If you are not using the preprint style,
you can use the font definitions in \filename{AMSPPT.STY} as a model.

\Warn  Editions of \Joy\/ prior to 1990 describe an obsolete method of
  accessing these fonts.  The directions given in Appendix F of these
  older editions are superseded by the |\load...|\ instructions cited
  here for Euler, bold math italic, and symbol fonts.
\endx


\section Euler

The Euler fonts are defined only in math mode, in sizes appropriate for
text and two orders of sub- and superscripts.  They can be activated by
invoking the proper |\load| instructions before the |\documentstyle|
command, in the preamble of a paper in which the fonts are to be used.
(The medium-weight Fraktur font is activated automatically by the preprint
style.)  The Euler fonts can be activated by the following commands:

\begingroup
\smallskip
\parskip=0pt
\def\1 #1 {\item{}{\tt\bs#1}\qquad\ignorespaces}
\1 loadeufm Euler Fraktur medium (automatic with the preprint style)
\1 loadeufb Euler Fraktur bold
\1 loadeurm Euler cursive medium
\1 loadeurb Euler cursive bold
\1 loadeusm Euler script medium
\1 loadeusb Euler script bold
\endgraf
\endgroup

After the \fontname{EUFM} font has been loaded, the medium-weight Fraktur
letters can be produced by typing |\frak| followed by the desired letter.
For example, |$\frak g \frak A$| yields $\frak g \frak A$.


\section Computer Modern bold math italic and symbols

The Computer Modern bold math italic (\fontname{CMMIB}) and bold math
symbol (\fontname{CMBSY}) fonts can both be loaded by the command
|\loadbold|; there are no predefined commands to load them separately.
|\loadbold| must be invoked in the preamble of the document input file.

A rather elaborate mechanism has been defined in \AmSTeX{} to simplify
access to bold letters and symbols, in math mode only.  Three control
sequences are available, each of which affects a particular class of
characters:

\begingroup
\smallskip
\parskip=0pt
\setbox0=\hbox{\tt xboldsymbol}
\def\1 #1 {\item{}\hbox to\wd0{\tt\bs#1\hfil}\qquad\ignorespaces}
\1 bold         for a single letter or numeral
\1 boldkey      for a symbol that appears on the keyboard
\1 boldsymbol   for a symbol specified by a single control sequence
\endgraf
\endgroup
\noindent
These facilities are described in more detail in the User's Guide to
\AmSTeX{} Version~2.0 [AMS90] and editions of \Joy\/ published in 1990
or later [MDS90].


\section Computer Modern math extension font

Smaller sizes of the math extension font are appropriate for use in text
smaller than ten-point and in sub- and superscripts.  They are provided
automatically for these environments in the preprint style.  If you are
not using the preprint style, you can use the font definitions in either
\filename{AMSPPT.STY} or Appendix~E of \TB\/ [DEK86A] as a model.


\section Extra symbols

The medium-weight versions of the two extra symbol fonts are available
automatically, including all the symbol names, if you are using the preprint
style or if you have specified |\input amssym|.  If you wish to load these
fonts separately, use the appropriate control sequence |\loadmsam| or
|\loadmsbm| in the preamble of your document.  If you load the fonts
separately, a few symbols will be defined when one of the fonts is loaded,
but most must be defined using the |\newsymbol| command before they can be
used.  See the section {\bf Installing and using the extra symbols} for
information on both the symbol names and on using |\newsymbol| to define them.


\section Cyrillic

Cyrillic is not referred to in the \AmSTeX{} files as distributed.
The cyrillic fonts included in AMSFonts are intended for use mainly in text,
not as symbols in math.  Detailed instructions for loading and using cyrillic
appear below in the section {\bf Installing and using cyrillic}.


\section Caps/small caps

Caps/small caps are loaded automatically by the \AmSTeX{} preprint style
for use in ten-point and eight-point text.  If you are not using the preprint
style, you can use the font definitions in either \filename{AMSPPT.STY}
or Appendix~E of \TB\/ [DEK86A] as a model.


\section Dummy font

No special action is needed to use the dummy font with \AmSTeX.
It is already built into the syntax checking procedure.


%%%%%%%%%%%%%%%%%%%%%%%%%%%%%%%%%%%%%%%%%%%%%%%%%%%%%%%%%%%%%%%%%%%%%%%%

\title How to use AMSFonts 2.0 without \AmSTeX{}

For the most part, it is assumed that anyone using AMSFonts without
\AmSTeX{} has some experience with \TeX{} macros or has a friendly
relationship with a \TeX nician.  
Since all applications seem to refer to fonts in different
ways, no assumptions are made about how any ``average'' user is going
to use these fonts.  However, some general guidelines may be helpful.

Two models for defining fonts should be accessible to most users:
\item{\bull} Appendix E of \TB\/ contains size-specific font definitions
  for \hbox{|\tenpoint|}, \hbox{|\ninepoint|} and |\eightpoint| that permit
  size-switching, including support of mathematics.
\item{\bull} \filename{AMSPPT.STY}, the file of macros supporting the
  \AmSTeX{} preprint style, contains similar font definitions, |\tenpoint|
  and |\eightpoint|.

\noindent
Extensive size-switching font facilities are also present in \LaTeX,
but these cannot easily be copied for uses outside of \LaTeX.

Before attempting to load all available fonts into every \TeX{} job,
determine (if you can) how many fonts can be accommodated by the
implementation of \TeX{} you are using.  It is generally a good idea to
load seldom-used fonts selectively.


\section Euler

The following commands will load the medium-weight Euler Fraktur font, and
can be used as a model for accessing the other Euler fonts.
\begintt
\font\teneufm=eufm10
\font\seveneufm=eufm7
\font\fiveeufm=eufm5
\newfam\eufmfam
\textfont\eufmfam=\teneufm
\scriptfont\eufmfam=\seveneufm
\scriptscriptfont\eufmfam=\fiveeufm
\def\frak#1{{\fam\eufmfam\relax#1}}
\endtt

Individual letters in the Euler fonts are accessible by the ordinary
letters on your keyboard, once the font has been loaded and named by
a control sequence equivalent to |\frak|.

The medium-weight Fraktur font, \fontname{EUFM}, can also be loaded by
|\input amssym.def|; this loads the two extra symbol fonts as well.


\section Computer Modern bold math italic and symbols

The \fontname{CMMIB} and \fontname{CMBSY} fonts can be loaded and made
accessible to math in ten-point environments by the following code:
\begintt
\font\tencmmib=cmmib10  \skewchar\tencmmib='177
\font\sevencmmib=cmmib7 \skewchar\sevencmmib='177
\font\fivecmmib=cmmib5  \skewchar\fivecmmib='177
\newfam\cmmibfam
\textfont\cmmibfam=\tencmmib \scriptfont\cmmibfam=\sevencmmib
 \scriptscriptfont\cmmibfam=\fivecmmib

\font\tencmbsy=cmbsy10  \skewchar\tencmbsy='60
\font\sevencmbsy=cmbsy7 \skewchar\sevencmbsy='60
\font\fivecmbsy=cmbsy5  \skewchar\fivecmbsy='60
\newfam\cmbsyfam
\textfont\cmbsyfam=\tencmbsy \scriptfont\cmbsyfam=\sevencmbsy
 \scriptscriptfont\cmbsyfam=\fivecmbsy
\endtt
The \TeX{} primitive |\mathchar| must be used to access individual characters
from a font in math mode.
|\mathchar|, like the |\char| primitive, requires that you know the position in
the font of the character you are accessing.  However, |\mathchar| also
requires that you specify the ``class'' and the family of the math character
being accessed.  See Chapter 17 of \TB{} for more details on the use of
|\mathchar|, as well as |\mathchardef|, which will allow you to define your own
macro names for individual characters in these fonts.

\Note The file \filename{AMSSYM.DEF} contains a convenient macro,
|\hexnumber@|, to determine the family number of the font being accessed
through |\mathchar|.  For example, the |\mathchar| statement to properly access
the bold alpha in the CMMIB font would be:
\begintt
\mathchar"0\hexnumber@\cmmibfam0B
\endtt
\endx



\section Computer Modern math extension font

The 10-point \fontname{CMEX} font is loaded by \filename{PLAIN.TEX}.
To install the 7-point size appropriate for sub- and superscripts in
a ten-point math environment, include the following code in your file:
\begintt
\font\sevenex=cmex7
\scriptfont3=\sevenex \scriptscriptfont3=\sevenex 
\endtt
To use other sizes implies the use of switchable-size fonts,
which may be implemented according to the models cited
at the beginning of this section.


\section Extra symbols

Detailed instructions for accessing the \fontname{MSAM} and \fontname{MSBM}
fonts are given in the section {\bf Installing and using the extra
symbols}.


\section Cyrillic

See the section {\bf Installing and using cyrillic} for instructions.


\section Caps/small caps

The 10-point \fontname{CMCSC} font is loaded by \filename{PLAIN.TEX}.
To use the smaller versions implies the use of switchable-size fonts,
which may be implemented according to the models cited at the beginning
of this section.


\section Dummy font

The dummy font was designed to be used for syntax checking.  The general
technique is described in Appendix~D of \TB, p.~401.  This has been
implemented in the file \filename{AMSTEX.TEX}, which can be used as a model.


%%%%%%%%%%%%%%%%%%%%%%%%%%%%%%%%%%%%%%%%%%%%%%%%%%%%%%%%%%%%%%%%%%%%%%%%

\title Installing and using cyrillic

\def\2#1{${}\mapsto{}${\cyr#1}}

\newcount\cyrtablefigno  \cyrtablefigno=2
\def\cyrtablefig{Figure~\number\cyrtablefigno}

The cyrillic fonts in the AMSFonts collection have been designed so that
input using the transliteration conventions of {\sl Mathematical Reviews\/}
will be converted directly to cyrillic text.  The following cyrillic
fonts are included:

\item{} \fontname{WNCYR} (upright), in sizes 5, 6, 7, 8, 9, and 10 point
\item{} \fontname{WNCYB} (bold), in the same range of sizes as \fontname{WNCYR}
\item{} \fontname{WNCYI} (italic), in the same range of sizes as
  \fontname{WNCYR}
\item{} \fontname{WNCYSC} (caps and small caps), in size 10 point
\item{} \fontname{WNCYSS} (upright sans serif), in sizes 8, 9, and 10 point

\noindent
The file \filename{CYRACC.DEF}, which is included in the AMSFonts collection,
must be input to any document using the cyrillic fonts as defined with the
AMS layout.  Since the cyrillic alphabet contains more letters than the
roman alphabet, some cyrillic letters are accessed by combinations of roman
letters, accented letters, or control sequences.  \filename{CYRACC.DEF}
contains the definitions of these accents and control sequences.
If this file is not input, some cyrillic letters will be inaccessible.


\section Making cyrillic available to a document

If you are not using \AmSTeX, include the following instructions near the
top of the document input file to make the 10-point cyrillic font available
for use in text (see below for cyrillic in math):
\begintt
\input cyracc.def
\font\tencyr=wncyr10
\def\cyr{\tencyr\cyracc}
\endtt
If you require cyrillic text in more than one size, you must take a
different approach in defining |\cyr|.  An appropriate model appears in
Appendix~E of \TB\/ [DEK86A], pages 414--15.  The definition of |\cyr|
should be incorporated into size-specific macros such as |\tenpoint| and
|\eightpoint| similarly to what is done there for |\bf|.  Don't forget to
include the command |\cyracc| in the definition.

If you are using \AmSTeX{} and the preprint style, the following instructions
should be included in the preamble of your document input file to make
cyrillic available in 10-point and 8-point text:
\begintt
\input cyracc.def
\catcode`\@=11
\font@\tencyr=wncyr10
\font@\eightcyr=wncyr8
\catcode`\@=13
\addto\tenpoint{\def\cyr{\tencyr\cyracc}}
\addto\eightpoint{\def\cyr{\eightcyr\cyracc}}
\endtt
(The |\font@| command not only loads the fonts, but also makes them behave
properly during syntax checking.)
If you are not using the preprint style, you can use the font definitions
in either \filename{AMSPPT.STY} or \TB\/ Appendix~E as a model.

The macro definitions in \filename{CYRACC.DEF} govern the behavior of
cyrillic-specific control sequences, including accents, in cyrillic and
noncyrillic text.  Definitions governing noncyrillic text are activated
as soon as \filename{CYRACC.DEF} is |\input|.  This will permit text input
according to the scheme shown in \cyrtablefig{} to be typeset in
transliterated form, according to the {\sl MR\/} conventions.  To produce
actual cyrillic text, enclose the cyrillic input in a group that begins with
the instruction |\cyr| {\sl inside\/} the group, as
\begintt
...{\cyr ...} ...
\endtt
Enclosing in braces both the |\cyr| and the text to be set in cyrillic type,
in the same way that an italic phrase would be indicated in a roman text,
is particularly important for two reasons.  First, like |\it|, |\cyr| must
be explicitly terminated to return to roman text.  And second, unlike |\it|,
the special cyrillic control sequences invoked by |\cyracc| are interpreted
differently by \TeX{} depending on whether they are in a cyrillic or a
noncyrillic environment.  The ``cyrillic'' interpretation is not turned
off simply by invoking |\rm|.  Failure to follow this practice will yield
gibberish.


\section Cyrillic input

The table in \cyrtablefig{} follows the alphabetical order of the table
published in the 1983 MR author index.  The three paired columns
contain: (1)~Cyrillic; (2)~Input; (3)~Transliteration.

The letters in the Cyrillic columns will appear in the
typeset output when the corresponding codes from the Input columns
are used in the |{\cyr ...}| context described above.  The roman
letters in the Transliteration columns will appear in the output when
the corresponding codes from the Input columns are used in a noncyrillic
environment, i.e., have not been preceded by |\cyr|.

\penalty0
\topinsert
\begingroup
\input userdoc.cyr
\endgroup
\endinsert

\penalty0
\indent
Several points should be noted here.

\nobreak
\item{\bull} Input codes for uppercase cyrillic which consist of more than
  one letter, e.g. |Zh|\2{Zh}, can also be input in all caps, e.g. |ZH|\2{ZH},
  if the context is entirely in caps.

\item{\bull} Particular care is necessary when the letter t\2{t} is followed
  by s\2{s}.  The control sequence |\cydot| (``cyrillic dot'')
  is provided as a separator to keep those letters distinct:
  |t\cydot s|~(t\cydot s)\2{t\cydot s}.
  Otherwise, they will be combined as ts\2{ts}.

\item{} The t\cydot s pair appears, for example, in the word
  |sovet\cydot ski\u\i|
  (sovet\cydot ski\u\i)\2{sovet\cydot ski\u\i}
  and is not uncommon in the suffix of reflexive verbs, e.g.
  \hbox{|nakhodyat\cydot sya|}
  (nakhodyat\cydot sya)\2{nakhodyat\cydot sya}.

\item{\bull} Because there is not a one-to-one correspondence between
  cyrillic and roman letters, some cyrillic letters have been placed
  in locations where a roman letter does not have a cyrillic
  counterpart.  A user who is aware of this fact may be able to
  detect input keying that does not conform to the recommendations
  shown in \cyrtablefig, and correct it more easily than otherwise.
  The following nonstandard assignments have been made:\newline
\indent |c|\2{c}; |h|\2{h}; |q|\2{q}; |w|\2{w}; |x|\2{x}.

\item{\bull} Some very strange effects can occur in cyrillic text
  hyphenated by the default English hyphenation rules; in particular,
  a cyrillic letter input as a group of letters can be decomposed.
  (Most multiple-letter input groups are converted to a single cyrillic
  letter by way of \TeX's ligaturing mechanism.)  For example,
  |shch|\2{shch} might, in especially unlucky circumstances, be
  decomposed as {\cyr s-hch}, {\cyr sh-ch} or {\cyr shc-h}.
  In other words, if there is any chance that cyrillic text might fall
  into a position where hyphenation could occur, the results should be
  checked very carefully, and discretionary hyphens used as appropriate.

\item{\bull} Hyphenation patterns do not exist for the AMS cyrillic font
  when the input conventions shown here are used.  Furthermore, it is
  probably impracticable to attempt to develop such rules, since the
  rules to recognize control sequences and complicated ligatures, both
  used extensively by the AMS cyrillic input conventions, are not easily
  specified to \TeX's hyphenation mechanism.  Another approach to
  hyphenation, requiring some changes to the cyrillic \filename{.TFM}
  files, has been described by Dimitri Vulis in a \TUB\/ article [DLV89].


\section Cyrillic in math

Although the cyrillic fonts are intended for use as text, the need
occasionally arises to use one or two letters in math; for example,
{\cyr SH} may be used to represent the Shafarevich group.  When cyrillic
must also be made available to math mode, the following instructions
(which will support the use of cyrillic in both text and math) should
replace the definition of |\cyr| shown previously (which will work only
for text):
\begintt
\newfam\cyrfam
\font\tencyr=wncyr10
\font\sevencyr=wncyr7
\font\fivecyr=wncyr5
\def\cyr{\fam\cyrfam\tencyr\cyracc}
\textfont\cyrfam=\tencyr \scriptfont\cyrfam=\sevencyr
  \scriptscriptfont\cyrfam=\sevencyr
\endtt
If only the 10-point cyrillic font has been accessed, the references
to |\sevencyr| and |\fivecyr| can be changed to |\tencyr| to save memory.
When using \AmSTeX{} and the preprint style, use |\font@| instead of |\font|,
remembering to change the |\catcode| of the |@| appropriately, and embed the
font family specifications in |\addto\tenpoint|, as shown above.

If other base text sizes are used besides ten point, the suggestions given
above under {\bf Making cyrillic available} apply here as well.


%%%%%%%%%%%%%%%%%%%%%%%%%%%%%%%%%%%%%%%%%%%%%%%%%%%%%%%%%%%%%%%%%%%%%%%%

\makeatletter

%  Define macros for presentation of tables of symbols.
\def\BBB#1{\par\bigbreak
  \leavevmode\llap{$\bullet$\enspace}{\bf#1}}
\def\ttcs#1{\leavevmode\hbox{\tt\bs\ignorespaces#1\unskip}}
\newdimen\biggest
\setbox0\hbox{$\dashrightarrow$}\biggest=\wd0
\def\1#1{\hbox to\biggest{\hfil$\csname#1\endcsname$\hfil}\ \ %
  \ttcs{#1}}

\def\getID@#1{\edef\next@{\expandafter\meaning\csname#1\endcsname}%
 \expandafter\getID@@\next@0\getID@@}
\def\getID@@#1"#2#3#4#5#6\getID@@{\def\next@{#6}%
  \ifx\next@\empty
   \def\next@{#2}%
    \ifx\next@\msafam@
     \def\ID@{10#3#4}%
    \else
     \def\ID@{20#3#4}%
    \fi
  \else 
   \def\next@{#3}%
    \ifx\next@\msafam@
     \def\ID@{1#2#4#5}%
    \else
     \def\ID@{2#2#4#5}%
    \fi
  \fi}
\def\2#1{\hbox to.5\hsize
  {\hbox to\biggest{\hfill$\csname#1\endcsname$\hfill}\ \ %
    \getID@{#1}{\tt\ID@}\ \ \ttcs{#1}\hfill}}
\def\3#1#2{\hbox to.5\hsize
  {\hbox to\biggest{\hfil$\csname#1\endcsname$\hfil}\ \ %
    \getID@{#1}{\tt\ID@}\ \ \ttcs{#1}, \ttcs{#2}\hss}}
\def\4#1{\hbox to.5\hsize
  {\hbox to\biggest{\hfill$\csname#1\endcsname$\hfill}\ \ %
    \getID@{#1}{\tt\ID@}\ \ \ttcs{#1}\ \ {\eightpoint(U)}\hfill}}

\makeatother

%%%%%%%%%%%%%%%%%%%%%%%%%%%%%%%%%%%%%%%%%%%%%%%%%%%%%%%%%%%%%%%%%%%%%%%%

\title Installing and using the extra symbols

Most users of the extra symbol fonts will probably want to make them
accessible to their \TeX{} jobs with the least possible fuss.  For \AmSTeX{}
users, these fonts are available automatically with the preprint style, and
other methods of loading them for use with \AmSTeX{} are described above.

If you are not using \AmSTeX, the easiest method of loading these fonts and
defining the control sequences for accessing the symbols is to place the
commands
\begintt
\input amssym.def
\input amssym
\endtt
at the top of your input file.  This will load the fonts \fontname{MSAM},
\fontname{MSBM}, and \fontname{EUFM} in sizes 10, 7, and 5 point, suitable
for use in ordinary ten-point math environments, and define the names of
all the symbols in these fonts.  However, this assigns more than 200
control sequence names, so if you are limited for space, an alternative
method may be preferred.

If you type just |\input amssmy.def|, the fonts will be loaded, but only the
names of the few special symbols listed below will be defined.

First there are four symbols that are normally used outside of math mode:
$$\vcenter{\halign to\hsize{\1{#}\hfil\tabskip\centering&
   \hbox to.5\hsize{\1{#}\hfil}\tabskip0pt\cr
checkmark&circledR\cr
maltese&yen\cr}}
$$
These symbols, like \P, \S, \dag, and \ddag, can also be used in
math mode, and will change sizes correctly in subscripts and superscripts.

Next are four symbols that are ``delimiters'' (although there are
no larger versions obtainable with \ttcs{left} and \ttcs{right}), so they
must be used in math mode:
$$\vcenter{\halign to\hsize{\1{#}\hfil\tabskip\centering&
   \hbox to.5\hsize{\1{#}\hfil}\tabskip0pt\cr
 ulcorner&urcorner\cr
 llcorner&lrcorner\cr}}$$

Finally, two dashed arrows are constructed from symbols in this family
(note that one of them has two names; it can be accessed by either one):
$$\vcenter{\halign to\hsize{\1{#}\hfil\tabskip\centering&
   \hbox to.5\hsize{\1{#}\hfil}\tabskip0pt\cr
 \omit\hbox to.5\hsize{\hbox to\biggest{\hfil$\dashrightarrow$\hfil}\ \ %
    \ttcs{dashrightarrow}, \ttcs{dasharrow}\hss}&dashleftarrow\cr}}$$

The Blackboard Bold letters $\Bbb A,\dots,\Bbb Z$ can be accessed by typing
(in math mode) |\Bbb A|,\dots,|\Bbb Z|.

Wider versions of the \filename{PLAIN.TEX} |\widehat| and |\widetilde|
are now available.

Letters in the \fontname{EUFM} font can be accessed (in math mode) by typing,
for example, |\frak A \frak g| to get $\frak A \frak g$.


\section The {\tt\bs newsymbol} command

All other symbols of the \fontname{MSAM} and \fontname{MSBM} fonts must be
named by control sequences so that they can be used (in math mode only) when
the fonts are loaded.  If you are very short on space for control sequence
names, and need only a few of these symbols, you can omit the loading of
\filename{AMSSYM.TEX} and instead assign only the names you will need by
using the command |\newsymbol| for each symbol you need, to create a
control sequence that will properly produce that symbol.  The control
sequence can be either the ``standard'' name, as listed below, or one
of your own choosing.

The list of symbols below shows for each symbol the symbol itself, a
four-character~``ID,'' and the ``standard'' name of the symbol. 
(The first character of the ID identifies the font family in which a
symbol resides.  Symbols from the \fontname{MSAM} family have {\tt1} as the
first character; symbols from the \fontname{MSBM} family have {\tt2} as the
first character.)
For example, the symbol $\nleqslant$ appears as
\medskip
\noindent\kern\parindent\2{nleqslant}
\medskip
\noindent
To produce a control sequence with this name, the instruction
\begintt
\newsymbol\nleqslant 230A
\endtt
appears in the file \filename{AMSSYM.TEX}\null.  This same instruction can
be typed by a user who is not using the \AmSTeX{} preprint style and has
chosen not to load all the symbols, and thereafter the control sequence
|\nleqslant| will produce the symbol $\nleqslant$ (in math mode), and will
act properly as a ``binary relation.''

A few symbols in these fonts replace symbols defined in \filename{PLAIN.TEX}
by combinations of symbols available in the Computer Modern fonts.  These
are |\angle|~($\angle$) and |\hbar|~($\hbar$) from the group
``Miscellaneous symbols,'' and |\rightleftharpoons|~($\rightleftharpoons$)
from the group ``Arrows'' below.  The new symbols will
change sizes correctly in subscripts and superscripts, provided that you
are using appropriate redefinitions.  In order to use |\newsymbol| to
replace an existing definition, the name must first be ``undefined.''
Here are the lines you must put in your file if you are not using the
\AmSTeX{} preprint style or |\input amssym| (which perform the redefinition
automatically):
\begintt
\undefine\angle
\newsymbol\angle 105C
\undefine\hbar
\newsymbol\hbar 207E
\undefine\rightleftharpoons
\newsymbol\rightleftharpoons 130A
\endtt
\noindent
These symbols are flagged in the tables below with a ``{\eightpoint(U)},''
as a reminder that they must be undefined.

Note in the tables that some symbols are shown with two names; in such a
case, either one can be used to access the symbol.

\BBB{Lowercase Greek letters}
$$\halign{\hbox to.5\hsize{\2{#}}&\2{#}\cr
digamma&varkappa\cr}$$

\BBB{Hebrew letters}
$$\halign{\hbox to.5\hsize{\2{#}}&\2{#}\cr
beth&gimel\cr 
daleth\cr
}$$

\BBB{Miscellaneous symbols}
$$\halign{\hbox to.5\hsize{\2{#}}&\2{#}\cr
\omit\4{hbar}&backprime\cr
hslash&varnothing\cr
vartriangle&blacktriangle\cr
triangledown&blacktriangledown\cr
square&blacksquare\cr
lozenge&blacklozenge\cr
circledS&bigstar\cr
\omit\4{angle}&sphericalangle\cr
measuredangle&\omit\cr
nexists&complement\cr
mho&eth\cr
Finv&diagup\cr
Game&diagdown\cr
Bbbk&\omit\cr
}$$

\BBB{Binary operators}
$$\halign{\hbox to.5\hsize{\2{#}}&\2{#}\cr
dotplus&ltimes\cr
smallsetminus&rtimes\cr
\omit\3{Cap}{doublecap}&leftthreetimes\cr
\omit\3{Cup}{doublecup}&rightthreetimes\cr
barwedge&curlywedge\cr
veebar&curlyvee\cr
%                               %%%%%%%%%%
%\noalign{\newpage}
%                               %%%%%%%%%%
doublebarwedge\cr
boxminus&circleddash\cr
boxtimes&circledast\cr
boxdot&circledcirc\cr
boxplus&centerdot\cr
divideontimes&intercal\cr}
$$

\BBB{Binary relations}
$$\halign{\hbox to.5\hsize{\2{#}}&\2{#}\cr
leqq&geqq\cr
leqslant&geqslant\cr
eqslantless&eqslantgtr\cr
lesssim&gtrsim\cr
lessapprox&gtrapprox\cr
approxeq\cr
lessdot&gtrdot\cr
\omit\3{lll}{llless}&\omit\3{ggg}{gggtr}\cr
lessgtr&gtrless\cr
lesseqgtr&gtreqless\cr
lesseqqgtr&gtreqqless\cr
\omit\3{doteqdot}{Doteq}&eqcirc\cr
risingdotseq&circeq\cr
fallingdotseq&triangleq\cr
backsim&thicksim\cr
backsimeq&thickapprox\cr
subseteqq&supseteqq\cr
Subset&Supset\cr
sqsubset&sqsupset\cr
preccurlyeq&succcurlyeq\cr
curlyeqprec&curlyeqsucc\cr
precsim&succsim\cr
precapprox&succapprox\cr
vartriangleleft&vartriangleright\cr
trianglelefteq&trianglerighteq\cr
vDash&Vdash\cr
Vvdash\cr
smallsmile&shortmid\cr
smallfrown&shortparallel\cr
bumpeq&between\cr
Bumpeq&pitchfork\cr
varpropto&backepsilon\cr
blacktriangleleft&blacktriangleright\cr
therefore&because\cr}$$
\bigbreak
\BBB{Negated relations}
$$\halign{\hbox to.5\hsize{\2{#}}&\2{#}\cr
nless&ngtr\cr
nleq&ngeq\cr
nleqslant&ngeqslant\cr
nleqq&ngeqq\cr
lneq&gneq\cr
lneqq&gneqq\cr
lvertneqq&gvertneqq\cr
lnsim&gnsim\cr
lnapprox&gnapprox\cr
%                               %%%%%%%%%%
%\noalign{\newpage}
%                               %%%%%%%%%%
nprec&nsucc\cr
npreceq&nsucceq\cr
precneqq&succneqq\cr
precnsim&succnsim\cr
precnapprox&succnapprox\cr
nsim&ncong\cr
nshortmid&nshortparallel\cr
nmid&nparallel\cr
nvdash&nvDash\cr
nVdash&nVDash\cr
ntriangleleft&ntriangleright\cr
ntrianglelefteq&ntrianglerighteq\cr
nsubseteq&nsupseteq\cr
nsubseteqq&nsupseteqq\cr
subsetneq&supsetneq\cr
varsubsetneq&varsupsetneq\cr
subsetneqq&supsetneqq\cr
varsubsetneqq&varsupsetneqq\cr}$$

\overfullrule=0pt

\BBB{Arrows}
$$\halign{\hbox to.5\hsize{\2{#}}&\2{#}\cr
leftleftarrows&rightrightarrows\cr
leftrightarrows&rightleftarrows\cr
Lleftarrow&Rrightarrow\cr
twoheadleftarrow&twoheadrightarrow\cr
leftarrowtail&rightarrowtail\cr
looparrowleft&looparrowright\cr
leftrightharpoons&\omit\4{rightleftharpoons}\cr
curvearrowleft&curvearrowright\cr
circlearrowleft&circlearrowright\cr
Lsh&Rsh\cr
upuparrows&downdownarrows\cr
upharpoonleft&\omit\3{upharpoonright}{restriction}\cr
downharpoonleft&downharpoonright\cr
multimap&rightsquigarrow\cr
leftrightsquigarrow\cr}$$

\BBB{Negated arrows}
$$\halign{\hbox to.5\hsize{\2{#}}&\2{#}\cr
leftarrow&nrightarrow\cr
nLeftarrow&nRightarrow\cr
nleftrightarrow&nLeftrightarrow\cr}$$


%%%%%%%%%%%%%%%%%%%%%%%%%%%%%%%%%%%%%%%%%%%%%%%%%%%%%%%%%%%%%%%%%%%%%%%%


\title For further information

The AMSFonts collection was implemented and packaged by the Technical
Support Group in the \AMS's Composition Services Department.
Questions or suggestions for improvements should be directed to that
group at the following address:
\begingroup
\smallskip
\parskip=0pt
\TextAddr Technical Support Group
  \AMS{}
  \POBox 6248
  Providence, RI 02940
  \smallskip Phone: 800-321-4AMS\quad or\quad 401-455-4080
  Internet: Tech-Support@Math.AMS.com<
\endgroup

\smallskip

The cyrillic fonts included in the AMSFonts collection were developed
at the Humanities and Arts Computing Center of the University of Washington.
Questions regarding these cyrillic fonts should be directed to:
\begingroup
\smallskip
\parskip=0pt
\TextAddr Director
  Humanities and Arts Computing Center
  DR-10
  University of Washington
  Seattle, WA 98195
  \smallskip Phone: 206-543-4218<
\endgroup

\smallskip

The \TUG{} is a good source of general information about fonts for use
with \TeX.  Inquiries can be directed to:
\begingroup
\smallskip
\parskip=0pt
\TextAddr \TUG{}
  \POBox 9506
  Providence, RI 02940-9506
  \smallskip Phone: 401-751-7760
  Internet: TUG@Math.AMS.com<
\endgroup


\section Obtaining the \MF{} source files

The AMSFonts collection has been prepared for a number of different
resolutions suitable for use on what the AMS staff has determined to be
the most popular devices currently being used to prepare \TeX{} output.
More such devices continue to appear, many of them with characteristics
different from the devices that are currently supported.

Users of unsupported devices who have access to an operating version of
\MF{} and have some experience with generating \MF{} fonts are encouraged
to contact the manager of the Society's Composition Services Department
to arrange to obtain the \MF{} source files.  Some restrictions will
apply; details will be provided when the sources are requested.

\MF{} source files for the cyrillic fonts in the AMSFonts collection can
be obtained either from the Society or directly from the University of
Washington.  The sources available from Washington also include other
fonts in different layouts.  Information can be obtained from the
director of the Academic Computing Center at the address given above.


%%%%%%%%%%%%%%%%%%%%%%%%%%%%%%%%%%%%%%%%%%%%%%%%%%%%%%%%%%%%%%%%%%%%%%%%

\title References

\begingroup
\raggedright
\hyphenpenalty=10000
\exhyphenpenalty=10000

\setbox\TestBox=\hbox{[DEK86A] }
\leftskip=\wd\TestBox
\def\1 [#1] {\noindent\kern-\leftskip
    \hbox to\leftskip{[#1]\hfil}\ignorespaces}
\def\bysame{\hbox to 3em{\leaders\hrule\hfill}\thinspace, }

\1 [AMS90] {\sl User's Guide to \AmSTeX{} Version~2.0, January 1990},
  \AMS, Providence, RI, 1990; distributed with \AmSTeX{} Version~2.0.

\1 [DEK86A] Donald E. Knuth, {\sl The \TeX book},
  Volume~A of \CandT, \AW{} Publishing Co.,
  Reading, 1986.

\1 [DEK86E] \bysame {\sl Computer Modern Typefaces},
  Volume~E of \CandT, \AW{} Publishing Co.,
  Reading, 1986.

\1 [DEK89] \bysame ``Typesetting Concrete Mathematics,''
  {\sl \TUB\/} {\bf10} (1989), no.~1, 31--36; erratum,
  {\sl \TUB\/} {\bf10} (1989), no.~3, 342.

\1 [DLV89] Dimitri Vulis, ``Notes on Russian \TeX,''
  {\sl \TUB\/} {\bf10} (1989), no.~3, 332--36.

\1 [DRS86] David R Siegel, {\sl The Euler Project at Stanford},
  Computer Science Department, Stanford University, 1985.

\1 [GKP88] Ronald L. Graham, Donald E. Knuth, and Oren Patashnik,
  {\sl Concrete Mathematics}, \AW{} Publishing Co.,
  Reading, 1988.

\1 [MDS86] M. D. Spivak, \JoT, \AMS, Providence, 1986.

\1 [MDS90] \bysame \JoT, $2^{\rm nd}$ (revised) edition,
  \AMS, Providence, 1990 (to appear).

\endgroup


%%%%%%%%%%%%%%%%%%%%%%%%%%%%%%%%%%%%%%%%%%%%%%%%%%%%%%%%%%%%%%%%%%%%%%%%

\newpage
%  Increase page length to get two font charts per page.
\collgt=54pc
\resetpagelgt

%%%%%%%%%%%%%%%%%%%%%%%%%%%%%%%%%%%%%%%%%%%%%%%%%%%%%%%%%%%%%%%%%%%%%%%%

\title Font charts to be used in Figure 1

\begingroup
\let\bye=\endinput

\input userdoc.fnt

\endgroup

\bye

