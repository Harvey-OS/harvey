\input webmac
% limbo material
\font\ninerm=amr9
\let\mc=\ninerm % medium caps for names like PASCAL
\def\WEB{{\tt WEB}}
\def\PASCAL{{\mc PASCAL}}
\def\[{\ifhmode\ \fi$[\![$}
\def\]{$]\!]$\ }
\def\<{$\langle\,$}
\def\>{$\,\rangle$}
\def\Dijk{{2}} % unnecessary when combined with text of paper
\def\goto{{3}} % ditto
\hyphenation{Dijk-stra} % ditto
\def\sec{{\tensy x}}
\hsize=84mm


\N1.  Printing primes: An example of \WEB.
The following program is essentially the same as Edsger Dijkstra's
``first example of step-wise program composition,''
found on pages 26--39 of his {\sl Notes on Structured Programming},$^{\Dijk}$
but it has been translated into the \WEB\ language.

\[Double brackets will be used in what follows to enclose comments
relating to \WEB\ itself, because the chief purpose of this program
is to introduce the reader to the \WEB\ style of documentation.
\WEB\ programs are always broken into small sections, each
of which has a serial number; the present section is number~1.\]

Dijkstra's program prints a table of the first thousand prime numbers.  We
shall begin as he did, by reducing the entire program to its top-level
description. \[Every section in a \WEB\ program begins with optional {\it
commentary\/} about that section, and ends with optional {\it program
text\/} for the section. For example, you are now reading part of the
commentary in \sec1, and the program text for \sec1 immediately follows
the present paragraph. Program texts are specifications of \PASCAL\
programs; they either use \PASCAL\ language directly, or they use angle
brackets to represent \PASCAL\ code that appears in other sections. For
example, the angle-bracket notation `\X2:Program to print $\ldots$
numbers\X' is \WEB's way of saying the following: ``The \PASCAL\ text to
be inserted here is called `Program to print $\ldots$ numbers', and you
can find out all about it by looking at section~2.'' One of the main
characteristics of \WEB\ is that different parts of the program are
usually abbreviated, by giving them such an informal top-level
description.\]

\Y\P\X2:Program to print the first thousand prime numbers\X\par
\fi

\M2. This program has no input, because we want to keep it rather simple.
The result of the program will be to produce a list of the first
thousand prime numbers, and this list will appear on the \\{output} file.

Since there is no input, we declare the value $\|m=1000$ as a compile-time
constant. The program itself is capable of generating the first
\|m prime numbers for any positive \|m, as long as the computer's
finite limitations are not exceeded.

\[The program text below specifies the ``expanded meaning'' of `\X2:Program
to print $\ldots$ numbers\X'; notice that it involves the top-level
descriptions of three other sections. When those top-level descriptions
are replaced by their expanded meanings, a syntactically correct \PASCAL\
program will be obtained.\]

\Y\P$\4\X2:Program to print the first thousand prime numbers\X\S$\6
\4\&{program}\1\  \37$\\{print\_primes}(\\{output})$;\6
\4\&{const} \37$\|m=1000$;\5
\X5:Other constants of the program\X\6
\4\&{var} \37\X4:Variables of the program\X\6
\&{begin} \37\X3:Print the first \|m prime numbers\X;\6
\&{end}.\par
\U section~1.\fi


\N3.  Plan of the program.
We shall proceed to fill out the rest of the program by making whatever
decisions seem easiest at each step; the idea will be to strive for
simplicity first and efficiency later, in order to see where this leads us.
The final program may not be optimum, but we want it to be reliable,
well motivated, and reasonably fast.

Let us decide at this point to maintain a table that includes all of the
prime numbers that will be generated, and to separate the generation
problem from the printing problem.

\[The \WEB\ description you are reading once again follows a pattern that
will soon be familiar: A typical section begins with comments and
ends with program text. The comments motivate and explain noteworthy
features of the program text.\]

\Y\P$\4\X3:Print the first \|m prime numbers\X\S$\6
\X11:Fill table \|p with the first \|m prime numbers\X;\6
\X8:Print table \|p\X\par
\U section~2.\fi

\M4. How should table \|p be represented? Two possibilities suggest
themselves: We could construct a sufficiently large array
of boolean values in which the $k$th entry is \\{true} if and only if the
number~\|k is prime; or we could build an array of integers in which
the \|kth entry is the \|kth prime number. Let us choose the latter
alternative, by introducing an integer array called $\|p[1\to\|m]$.

In the documentation below, the notation `$\|p[\|k]$' will refer to the
\|kth element of array~\|p, while `$p_k$' will refer to the $k$th
prime number. If the program is correct, $\|p[\|k]$ will either be
equal to $p_k$ or it will not yet have been assigned any value.

\[Incidentally, our program will eventually make use of several
more variables as we refine the data structures. All of the sections
where variables are declared will be called `\X4:Variables of the
program\X'; the number `{\eightrm4}' in this name refers to the
present section, which is the first section to specify the
expanded meaning of `\<Variables of the program\>'.
The note `{\eightrm See also $\ldots$}' refers to all of the other
sections that have the same top-level description. The expanded meaning of
`\X4:Variables of the program\X' consists of all the program texts
for this name, not just the text found in~\sec4.\]

\Y\P$\4\X4:Variables of the program\X\S$\6
\4\|p: \37\&{array} $[1\to\|m]$ \1\&{of}\5
\\{integer};\C{the first \|m prime numbers, in increasing order}\2\par
\A sections~7, 12, 15, 17, 23, and~24.
\U section~2.\fi


\N5.  The output phase.
Let's work on the second part of the program first. It's not as interesting
as the problem of computing prime numbers; but the job of printing must be
done sooner or later, and we might as well do it sooner, since it will
be good to have it done. \[And it is easier to learn \WEB\ when reading a
program that has comparatively few distracting complications.\]

Since \|p is simply an array of integers, there is little difficulty
in printing the output, except that we need to decide upon a suitable
output format. Let us print the table on separate pages, with \\{rr} rows
and \\{cc} columns per page, where every column is \\{ww} character positions
wide. In this case we shall choose $\\{rr}=50$, $\\{cc}=4$, and $\\{ww}=10$, so
that
the first 1000 primes will appear on five pages. The program will
not assume that \|m is an exact multiple of $\\{rr}\cdot\\{cc}$.

\Y\P$\4\X5:Other constants of the program\X\S$\6
$\\{rr}=50$;\C{this many rows will be on each page in the output}\6
$\\{cc}=4$;\C{this many columns will be on each page in the output}\6
$\\{ww}=10$;\C{this many character positions will be used in each column}\par
\A section~19.
\U section~2.\fi

\M6. In order to keep this program reasonably free of notations that
are uniquely \PASCAL esque, \[and in order to illustrate more of the
facilities of \WEB,\] a few macro definitions for low-level output
instructions are introduced here. All of the output-oriented commands
in the remainder of the program will be stated in terms of five
simple primitives called \\{print\_string}, \\{print\_integer}, \\{print%
\_entry},
\\{new\_line}, and \\{new\_page}.

\[Sections of a \WEB\ program are allowed to contain {\it macro definitions\/}
between the opening comments and the closing program text. The
general format for each section is actually tripartite: commentary,
then definitions, then program. Any of the three parts may be absent;
for example, the present section contains no program text.\]

\[Simple macros simply substitute a bit of \PASCAL\ code for an
identifier. Parametric macros are similar, but they also substitute
an argument wherever `\#' occurs in the macro definition. The first three
macro definitions here are parametric; the other two are simple.\]

\Y\P\D \37$\\{print\_string}(\#)\S\\{write}(\#)$\C{put a given string into the %
\\{output} file}\par
\P\D \37$\\{print\_integer}(\#)\S\\{write}(\#:1)$\C{put a given integer into
the \\{output}     file, in decimal notation, using only as many digit
positions as necessary}\par
\P\D \37$\\{print\_entry}(\#)\S\\{write}(\#:\\{ww})$\C{like \\{print\_integer},
but \\{ww} character     positions are filled, inserting blanks at the left}\par
\P\D \37$\\{new\_line}\S\\{write\_ln}$\C{advance to a new line in the %
\\{output} file}\par
\P\D \37$\\{new\_page}\S\\{page}$\C{advance to a new page in the \\{output}
file}\par
\fi

\M7. Several variables are needed to govern the output process. When we begin
to print a new page, the variable \\{page\_number} will be the ordinal number
of that page, and \\{page\_offset} will be such that $\|p[\\{page\_offset}]$ is
the
first prime to be printed. Similarly, $\|p[\\{row\_offset}]$ will be the first
prime in a given row.

\[Notice the notation `$+\S$' below; this indicates that the present
section has the same name as a previous section, so the program text
will be appended to some text that was previously specified.\]

\Y\P$\4\X4:Variables of the program\X\mathrel{+}\S$\6
\4\\{page\_number}: \37\\{integer};\C{one more than the number of pages printed
so far}\6
\4\\{page\_offset}: \37\\{integer};\C{index into \|p for the first entry on the
current page}\6
\4\\{row\_offset}: \37\\{integer};\C{index into \|p for the first entry in the
current row}\6
\4\|c: \37$0\to\\{cc}$;\C{runs through the columns in a row}\par
\fi

\M8. Now that appropriate auxiliary variables have been introduced, the process
of outputting table~\|p almost writes itself.

\Y\P$\4\X8:Print table \|p\X\S$\6
\&{begin} \37$\\{page\_number}\K1$;\5
$\\{page\_offset}\K1$;\6
\&{while} $\\{page\_offset}\L\|m$ \1\&{do}\6
\&{begin} \37\X9:Output a page of answers\X;\6
$\\{page\_number}\K\\{page\_number}+1$;\5
$\\{page\_offset}\K\\{page\_offset}+\\{rr}\ast\\{cc}$;\6
\&{end};\2\6
\&{end}\par
\U section~3.\fi

\M9. A simple heading is printed at the top of each page.

\Y\P$\4\X9:Output a page of answers\X\S$\6
\&{begin} \37$\\{print\_string}(\.{\'The\ First\ \'})$;\5
$\\{print\_integer}(\|m)$;\6
$\\{print\_string}(\.{\'\ Prime\ Numbers\ ---\ Page\ \'})$;\5
$\\{print\_integer}(\\{page\_number})$;\5
\\{new\_line};\5
\\{new\_line};\C{there's a blank line after the heading}\6
\&{for} $\\{row\_offset}\K\\{page\_offset}\mathrel{\&{to}}\\{page\_offset}+%
\\{rr}-1$ \1\&{do}\5
\X10:Output a line of answers\X;\2\6
\\{new\_page};\6
\&{end}\par
\U section~8.\fi

\M10. The first row will contain
$$\hbox{$\|p[1]$, $\|p[1+\\{rr}]$, $\|p[1+2\ast\\{rr}]$, \dots;}$$
a similar pattern holds for each value of the \\{row\_offset}.

\Y\P$\4\X10:Output a line of answers\X\S$\6
\&{begin} \37\&{for} $\|c\K0\mathrel{\&{to}}\\{cc}-1$ \1\&{do}\6
\&{if} $\\{row\_offset}+\|c\ast\\{rr}\L\|m$ \1\&{then}\5
$\\{print\_entry}(\|p[\\{row\_offset}+\|c\ast\\{rr}])$;\2\2\6
\\{new\_line};\6
\&{end}\par
\U section~9.\fi


\N11.  Generating the primes.
The remaining task is to fill table~\|p with the correct numbers.
Let us do this by generating its entries one at a time: Assuming that
we have computed all primes that are \|j~or less, we will advance \|j
to the next suitable value, and continue doing this until the
table is completely full.

The program includes a provision to initialize the variables in certain
data structures that will be introduced later.

\Y\P$\4\X11:Fill table \|p with the first \|m prime numbers\X\S$\6
\X16:Initialize the data structures\X;\6
\&{while} $\|k<\|m$ \1\&{do}\6
\&{begin} \37\X14:Increase \|j until it is the next prime number\X;\6
$\|k\K\|k+1$;\5
$\|p[\|k]\K\|j$;\6
\&{end}\2\par
\U section~3.\fi

\M12. We need to declare the two variables \|j and~\|k that were just
introduced.

\Y\P$\4\X4:Variables of the program\X\mathrel{+}\S$\6
\4\|j: \37\\{integer};\C{all primes $\L\|j$ are in table \|p}\6
\4\|k: \37$0\to\|m$;\C{this many primes are in table \|p}\par
\fi

\M13. So far we haven't needed to confront the issue of what a prime number
is. But everything else has been taken care of, so we must delve into
a bit of number theory now.

By definition, a number is called prime if it is an integer greater
than~1 that is not evenly divisible by any smaller prime number. Stating
this another way, the integer $\|j>1$ is not prime if and only if there
exists a prime number $p_n<j$ such that \|j is a multiple of~$p_n$.

Therefore the section of the program that is called `\<Increase \|j until
it is the next prime number\>' could be coded very simply:
`\ignorespaces \&{repeat} $\|j\K\|j+1$;\unskip\
\<Give to~\\{j\_prime} the meaning: \|j~is a prime number\>;
\ignorespaces  \&{until}  \\{j\_prime}\unskip'.
And to compute the boolean value \\{j\_prime}, the following
would suffice: `\ignorespaces$\\{j\_prime}\K\\{true}$;  \&{for} $\|n\K1%
\mathrel{\&{to}}\|k$ \&{do}\unskip\
\<If $\|p[\|n]$ divides \|j, set $\\{j\_prime}\K\\{false}$\>'.

\fi

\M14. However, it is possible to obtain a much more efficient algorithm by
using more facts of number theory. In the first place, we can speed
things up a bit by recognizing that $p_1=2$ and that all subsequent
primes are odd; therefore we can let \|j run through odd values only.
Our program now takes the following form:

\Y\P$\4\X14:Increase \|j until it is the next prime number\X\S$\6
\1\&{repeat} \37$\|j\K\|j+2$;\5
\X20:Update variables that depend on~\|j\X;\6
\X22:Give to \\{j\_prime} the meaning: \|j~is a prime number\X;\6
\4\&{until}\5
\\{j\_prime}\2\par
\U section~11.\fi

\M15. The  \&{repeat}  loop in the previous section introduces a boolean
variable \\{j\_prime}, so that it will not be necessary to resort to
a \&{goto}  statement. (We are following Dijkstra,$^\Dijk$ not Knuth.$^\goto$)

\Y\P$\4\X4:Variables of the program\X\mathrel{+}\S$\6
\4\\{j\_prime}: \37\\{boolean};\C{is \|j a prime number?}\par
\fi

\M16. In order to make the odd-even trick work, we must of course initialize
the variables \|j, \|k, and $\|p[1]$ as follows.

\Y\P$\4\X16:Initialize the data structures\X\S$\6
$\|j\K1$;\5
$\|k\K1$;\5
$\|p[1]\K2$;\par
\A section~18.
\U section~11.\fi

\M17. Now we can apply more number theory in order to obtain further
economies. If \|j is not prime, its smallest prime factor $p_n$ will
be $\sqrt j$ or less. Thus if we know a number \\{ord} such that
$$p[\\{ord}]^2>j,$$ and if \|j is odd, we need only test for divisors
in the set $\{p[2], \ldots, p[\\{ord}-1]\}$. This is much faster than
testing divisibility by $\{p[2],\ldots,p[k]\}$, since \\{ord} tends
to be much smaller than~\|k. \ (Indeed, when \|k is large, the
celebrated ``prime number theorem'' implies that the value of \\{ord}
will be approximately $2\sqrt{k/\!\ln k}$.)

Let us therefore introduce \\{ord} into the data structure. A moment's
thought makes it clear that \\{ord} changes in a simple way when \|j
increases, and that another variable \\{square} facilitates the
updating process.

\Y\P$\4\X4:Variables of the program\X\mathrel{+}\S$\6
\4\\{ord}: \37$2\to\\{ord\_max}$;\C{the smallest index $\G2$ such that
$p_{ord}^2>j$}\6
\4\\{square}: \37\\{integer};\C{$\\{square}=p_{ord}^2$}\par
\fi

\M18. \P$\X16:Initialize the data structures\X\mathrel{+}\S$\6
$\\{ord}\K2$;\5
$\\{square}\K9$;\par
\fi

\M19. The value of \\{ord} will never get larger than a certain value
\\{ord\_max}, which must be chosen sufficiently large. It turns out that
\\{ord} never exceeds~30 when $\|m=1000$.

\Y\P$\4\X5:Other constants of the program\X\mathrel{+}\S$\6
$\\{ord\_max}=30$;\C{$p_{ord\_max}^2$ must exceed $p_m$}\par
\fi

\M20. When \|j has been increased by~2, we must increase \\{ord} by unity
when $j=p_{ord}^2$, i.e., when $\|j=\\{square}$.

\Y\P$\4\X20:Update variables that depend on~\|j\X\S$\6
\&{if} $\|j=\\{square}$ \1\&{then}\6
\&{begin} \37$\\{ord}\K\\{ord}+1$;\5
\X21:Update variables that depend on~\\{ord}\X;\6
\&{end}\2\par
\U section~14.\fi

\M21. At this point in the program, \\{ord} has just been increased by unity,
and we want to set $\\{square}:=p_{ord}^2$. A surprisingly subtle point
arises here: How do we know that $p_{ord}$ has already been computed,
i.e., that $\\{ord}\L\|k$? If there were a gap in the sequence of prime
numbers,
such that $p_{k+1}>p_k^2$ for some~$k$, then this part of the program would
refer to the yet-uncomputed value $\|p[\|k+1]$ unless some special test were
made.

Fortunately, there are no such gaps. But no simple proof of this fact is
known. For example, Euclid's famous demonstration that there are
infinitely many prime numbers is strong enough to prove only that
$p_{k+1}<=p_1\ldots p_k+1$. Advanced books on number theory come to our
rescue by showing that much more is true; for example, ``Bertrand's
postulate''  states that $p_{k+1}<2p_k$
for all~$k$.

\Y\P$\4\X21:Update variables that depend on~\\{ord}\X\S$\6
$\\{square}\K\|p[\\{ord}]\ast\|p[\\{ord}]$;\C{at this point $\\{ord}\L\|k$}\par
\A section~25.
\U section~20.\fi


\N22.  The inner loop.
Our remaining task is to determine whether or not a given integer~\|j is prime.
The general outline of this part of the program is quite simple,
using the value of \\{ord} as described above.

\Y\P$\4\X22:Give to \\{j\_prime} the meaning: \|j~is a prime number\X\S$\6
$\|n\K2$;\5
$\\{j\_prime}\K\\{true}$;\6
\&{while} $(\|n<\\{ord})\W\\{j\_prime}$ \1\&{do}\6
\&{begin} \37\X26:If $\|p[\|n]$ is a factor of~\|j, set $\\{j\_prime}\K%
\\{false}$\X;\6
$\|n\K\|n+1$;\6
\&{end}\2\par
\U section~14.\fi

\M23. \P$\X4:Variables of the program\X\mathrel{+}\S$\6
\4\|n: \37$2\to\\{ord\_max}$;\C{runs from 2 to \\{ord} when testing
divisibility}\par
\fi

\M24. Let's suppose that division is very slow or nonexistent on our
machine. We want to detect nonprime odd numbers, which are odd multiples
of the set of primes $\{p_2,\ldots,p_{ord}\}$.

Since \\{ord\_max} is small, it is reasonable to maintain an auxiliary table of
the smallest odd multiples that haven't already been used to show that
some~\|j is nonprime. In other words, our goal is to ``knock out'' all
of the odd multiples of each $p_n$ in the set $\{p_2,\ldots,p_{ord}\}$,
and one way to do this is to introduce an auxiliary table that serves as
a control structure for a set of knock-out procedures that are being
simulated in parallel. (The so-called ``sieve of Eratosthenes''
generates primes by a similar method, but
it knocks out the multiples of each prime serially.)

The auxiliary table suggested by these considerations is a \\{mult}
array that satisfies the following invariant condition: For $2\L\|n<\\{ord}$,
$\\{mult}[\|n]$ is an odd multiple of $p_n$ such that $\\{mult}[n]<j+2p_n$.

\Y\P$\4\X4:Variables of the program\X\mathrel{+}\S$\6
\4\\{mult}: \37\&{array} $[2\to\\{ord\_max}]$ \1\&{of}\5
\\{integer};\C{runs through multiples of primes}\2\par
\fi

\M25. When \\{ord} has been increased, we need to initialize a new element of
the \\{mult} array. At this point $j=p[\\{ord}-1]^2$, so there is no
need for an elaborate computation.

\Y\P$\4\X21:Update variables that depend on~\\{ord}\X\mathrel{+}\S$\6
$\\{mult}[\\{ord}-1]\K\|j$;\par
\fi

\M26. The remaining task is straightforward, given the data structures
already prepared. Let us recapitulate the current situation: The
goal is to test whether or not \|j~is divisible by~$p_n$, without
actually performing a division. We know that $j$~is odd, and that
$\\{mult}[\|n]$ is an odd multiple of~$p_n$ such that $\\{mult}[n]<j+2p_n$.
If $\\{mult}[\|n]<\|j$, we can increase $\\{mult}[\|n]$ by $2p_n$ and the same
conditions will hold. On the other hand if $\\{mult}[\|n]\G\|j$, the
conditions imply that \|j~is divisible by~$p_n$ if and only if
$\|j=\\{mult}[\|n]$.

\Y\P$\4\X26:If $\|p[\|n]$ is a factor of~\|j, set $\\{j\_prime}\K\\{false}$\X%
\S$\6
\&{while} $\\{mult}[\|n]<\|j$ \1\&{do}\5
$\\{mult}[\|n]\K\\{mult}[\|n]+\|p[\|n]+\|p[\|n]$;\2\6
\&{if} $\\{mult}[\|n]=\|j$ \1\&{then}\5
$\\{j\_prime}\K\\{false}$\2\par
\U section~22.\fi


\N27.  Index.
Every identifier used in this program is shown here together with a list
of the section numbers where that identifier appears. The section number
is underlined if the identifier was defined in that section. However,
one-letter identifiers are indexed only at their point of definition,
since such identifiers tend to appear almost everywhere. \[An index like
this is prepared automatically by the \WEB\ software, and it is appended
to the final section of the program. However, underlining of section
numbers is not automatic; the user is supposed to mark identifiers
at their point of definition in the \WEB\ source file.\]

This index also refers to some of the places where key elements of the
program are treated. For example, the entries for `Output format' and
`Page headings' indicate where details of the output format are
discussed. Several other topics that appear in the documentation
(e.g., `Bertrand's postulate') have also been indexed. \[Special
instructions within a \WEB\ source file can be used to insert
essentially anything into the index.\]
\fi



\inx
\:{Bertrand, Joseph, postulate}, 21.
\:\\{boolean}, 15.
\:\|{c}, \[7].
\:\\{cc}, \[5], 7, 8, 10.
\:{Dijkstra, Edsger}, 1, 15.
\:{Eratosthenes, sieve of}, 24.
\:\\{false}, 13, 26.
\:\\{integer}, 4, 7, 12, 17, 24.
\:\|{j}, \[12].
\:\\{j\_prime}, 13, 14, \[15], 22, 26.
\:\|{k}, \[12].
\:{Knuth, Donald E.}, 15.
\:\|{m}, \[2].
\:\\{mult}, \[24], 25, 26.
\:\|{n}, \[23].
\:\\{new\_line}, \[6], 9, 10.
\:\\{new\_page}, \[6], 9.
\:\\{ord}, \[17], 18, 19, 20, 21, 22, 23, 24, 25.
\:\\{ord\_max}, 17, \[19], 23, 24.
\:\\{output}, 2, 6.
\:{output format}, 5, 9.
\:\|{p}, \[4].
\:\\{page}, 6.
\:{page headings}, 9.
\:\\{page\_number}, \[7], 8, 9.
\:\\{page\_offset}, \[7], 8, 9.
\:{prime number, definition of}, 13.
\:\\{print\_entry}, \[6], 10.
\:\\{print\_integer}, \[6], 9.
\:\\{print\_primes}, \[2].
\:\\{print\_string}, \[6], 9.
\:\\{row\_offset}, \[7], 9, 10.
\:\\{rr}, \[5], 8, 9, 10.
\:\\{square}, \[17], 18, 20, 21.
\:\\{true}, 4, 13, 22.
\:\.{WEB}, 1.
\:\\{write}, 6.
\:\\{write\_ln}, 6.
\:\\{ww}, \[5], 6.

\fin
\:\X11:Fill table \|p with the first \|m prime numbers\X
\U section~3.
\:\X22:Give to \\{j\_prime} the meaning: \|j~is a prime number\X
\U section~14.
\:\X26:If $\|p[\|n]$ is a factor of~\|j, set $\\{j\_prime}\K\\{false}$\X
\U section~22.
\:\X14:Increase \|j until it is the next prime number\X
\U section~11.
\:\X16, 18:Initialize the data structures\X
\U section~11.
\:\X5, 19:Other constants of the program\X
\U section~2.
\:\X10:Output a line of answers\X
\U section~9.
\:\X9:Output a page of answers\X
\U section~8.
\:\X8:Print table \|p\X
\U section~3.
\:\X3:Print the first \|m prime numbers\X
\U section~2.
\:\X2:Program to print the first thousand prime numbers\X
\U section~1.
\:\X20:Update variables that depend on~\|j\X
\U section~14.
\:\X21, 25:Update variables that depend on~\\{ord}\X
\U section~20.
\:\X4, 7, 12, 15, 17, 23, 24:Variables of the program\X
\U section~2.
\con
