% This is DPSCOLOR.TEX and DPSCOLOR.STY in text format, as of
% March 5, 1992, written by Jim Hafner, HAFNER@ALMADEN, or
% hafner@almaden.ibm.com.  Modified by Tomas Rokicki to be a
% bit smaller and easier to add to.
%
% This style file can be used to get color in TeX, LaTeX or more
% importantly, FoilTeX as an optional style file.  It is generic in
% that the color parameters are determined by the driver's header files.
%
% It can be used with any driver which knows how to process the keywords
% in the \special commands.  Currently, only versions of dvips greater
% than 5.478 (dvips is the DVI->PostScript driver by Tom Rokicki) and
% TeXview on the NeXT interpret these specials.
%
% This file can be used in any flavor of LaTeX (e.g. LaTeX, FoilTeX,
% AMS-LaTeX) by adding the keyword "dpscolor" to the options in the
% \documentstyle command (e.g., "\documentstyle[color]{foils}") or \input
% in any flavor of TeX (e.g. Plain TeX, eplain, AMS-TeX or any of the
% above LaTeX flavors) with the command "\input dpscolor".
%
% The actual parameters for each color are determined by the color.pro
% header file prepended to the output file by dvips.  They can be
% overrriden by a user-supplied header, called by either the -h option
% or the ``h'' flag in a printer configuration file, or in the .dvipsrc
% (or its analogue on other systems).
%
% Most of the color names and their matches were based on the
% most recent (at time of writing) Crayola Crayon box of 64 colors.
%
% There are 68 predefined colors.  At the end of this file is a
% listing of the color names and their approximate PANTONE color
% match.  This is for reference purposes only.
%
% The first macro lets the user specify the background color for the
% document.  It sets the background color for the current page and all
% succeeding pages, unless changed by another command of this type.  To
% change the background color back to the default, issue
% \background{White}
%
\def\background#1{\special{background #1}}
%
% There are two types of text color commands.  The first is in the form
%    \ColorName   (note the uppercase for the color name).
%
% It is called a local color command since it takes one argument
% enclosed in brackets.  It writes the contents of its argument in the
% selected color.  This should be used for local or nested color
% changes, since it restores the original color state when it completes.
% The second type of color command is in the form
%    \textColorName
% This uses the same naming convention as before.  It is called a global
% color command since it takes no arguments and simply sets the color at
% this point.  No previous color information is saved.  IF YOU USE THIS
% INTERNAL TO ANY LOCAL COLOR COMMAND, THE NESTING HISTORY IS LOST.
%
% Here are the global color changers, with color codes defined;
% these are used to define the small region colors.
%
% This first is for user defined color.  The argument #1 is for a "CMYK"
% quadruple of intensity values between 0 and 1.  (CMYK stands for Cyan,
% Magenta, Yellow and Black.)  E.g., \textColor{.2 .3 .4 .1}
%
\def\textColor#1{\special{color #1}}
%
\def\newColor #1 {\expandafter\def\csname text#1\endcsname{\special
   {color #1}}\expandafter\def\csname #1\endcsname##1{\special
   {color push #1}##1\special{color pop}}}%
\def\Color#1#2{\special{color push #1}#2\special{color pop}}
%
% Here are the color names and their PANTONE match (approximately)
%
\newColor GreenYellow     % GreenYellow  Approximate PANTONE 388 
\newColor Yellow	  % Yellow  Approximate PANTONE YELLOW 
\newColor Goldenrod	  % Goldenrod  Approximate PANTONE 109 
\newColor Dandelion	  % Dandelion  Approximate PANTONE 123 
\newColor Apricot	  % Apricot  Approximate PANTONE 1565 
\newColor Peach		  % Peach  Approximate PANTONE 164 
\newColor Melon		  % Melon  Approximate PANTONE 177 
\newColor YellowOrange	  % YellowOrange  Approximate PANTONE 130 
\newColor Orange	  % Orange  Approximate PANTONE ORANGE-021 
\newColor BurntOrange	  % BurntOrange  Approximate PANTONE 388 
\newColor Bittersweet	  % Bittersweet  Approximate PANTONE 167 
\newColor RedOrange	  % RedOrange  Approximate PANTONE 179 
\newColor Mahogany	  % Mahogany  Approximate PANTONE 484 
\newColor Maroon	  % Maroon  Approximate PANTONE 201 
\newColor BrickRed	  % BrickRed  Approximate PANTONE 1805 
\newColor Red		  % Red  VERY-Approx PANTONE RED 
\newColor OrangeRed	  % OrangeRed  No PANTONE match 
\newColor RubineRed	  % RubineRed  Approximate PANTONE RUBINE-RED 
\newColor WildStrawberry  % WildStrawberry  Approximate PANTONE 206 
\newColor Salmon	  % Salmon  Approximate PANTONE 183 
\newColor CarnationPink	  % CarnationPink  Approximate PANTONE 218 
\newColor Magenta	  % Magenta  Approximate PANTONE PROCESS-MAGENTA 
\newColor VioletRed	  % VioletRed  Approximate PANTONE 219 
\newColor Rhodamine	  % Rhodamine  Approximate PANTONE RHODAMINE-RED 
\newColor Mulberry	  % Mulberry  Approximate PANTONE 241 
\newColor RedViolet	  % RedViolet  Approximate PANTONE 234 
\newColor Fuchsia	  % Fuchsia  Approximate PANTONE 248 
\newColor Lavender	  % Lavender  Approximate PANTONE 223 
\newColor Thistle	  % Thistle  Approximate PANTONE 245 
\newColor Orchid	  % Orchid  Approximate PANTONE 252 
\newColor DarkOrchid	  % DarkOrchid  No PANTONE match 
\newColor Purple	  % Purple  Approximate PANTONE PURPLE 
\newColor Plum		  % Plum  VERY-Approx PANTONE 518 
\newColor Violet	  % Violet  Approximate PANTONE VIOLET 
\newColor RoyalPurple	  % RoyalPurple  Approximate PANTONE 267 
\newColor BlueViolet	  % BlueViolet  Approximate PANTONE 2755 
\newColor Periwinkle	  % Periwinkle  Approximate PANTONE 2715 
\newColor CadetBlue	  % CadetBlue  Approximate PANTONE (534+535)/2 
\newColor CornflowerBlue  % CornflowerBlue  Approximate PANTONE 292 
\newColor MidnightBlue	  % MidnightBlue  Approximate PANTONE 302 
\newColor NavyBlue	  % NavyBlue  Approximate PANTONE 293 
\newColor RoyalBlue	  % RoyalBlue  No PANTONE match 
\newColor Blue		  % Blue  Approximate PANTONE BLUE-072 
\newColor Cerulean	  % Cerulean  Approximate PANTONE 3005 
\newColor Cyan		  % Cyan  Approximate PANTONE PROCESS-CYAN 
\newColor ProcessBlue	  % ProcessBlue  Approximate PANTONE PROCESS-BLUE 
\newColor SkyBlue	  % SkyBlue  Approximate PANTONE 2985 
\newColor Turquoise	  % Turquoise  Approximate PANTONE (312+313)/2 
\newColor TealBlue	  % TealBlue  Approximate PANTONE 3145 
\newColor Aquamarine	  % Aquamarine  Approximate PANTONE 3135 
\newColor BlueGreen	  % BlueGreen  Approximate PANTONE 320 
\newColor Emerald	  % Emerald  No PANTONE match 
\newColor JungleGreen	  % JungleGreen  Approximate PANTONE 328 
\newColor SeaGreen	  % SeaGreen  Approximate PANTONE 3268 
\newColor Green		  % Green  VERY-Approx PANTONE GREEN 
\newColor ForestGreen	  % ForestGreen  Approximate PANTONE 349 
\newColor PineGreen	  % PineGreen  Approximate PANTONE 323 
\newColor LimeGreen	  % LimeGreen  No PANTONE match 
\newColor YellowGreen	  % YellowGreen  Approximate PANTONE 375 
\newColor SpringGreen	  % SpringGreen  Approximate PANTONE 381 
\newColor OliveGreen	  % OliveGreen  Approximate PANTONE 582 
\newColor RawSienna	  % RawSienna  Approximate PANTONE 154 
\newColor Sepia		  % Sepia  Approximate PANTONE 161 
\newColor Brown		  % Brown  Approximate PANTONE 1615 
\newColor Tan		  % Tan  No PANTONE match 
\newColor Gray		  % Gray  Approximate PANTONE COOL-GRAY-8 
\newColor Black		  % Black  Approximate PANTONE PROCESS-BLACK 
\newColor White		  % White  No PANTONE match 



































































