% SLITEX VERSION 2.09 <4 Dec 1989>
% Copyright (C) 1986 by Leslie Lamport

\everyjob{\typeout{SliTeX Version 2.09 <8 Jun 1988>}}
\immediate\write10{SliTeX Version 2.09 <8 Jun 1988>}

% NOTES FOR DEFINING FONTS AND STYLES FOR SLIDES
%   Every font definition -- \rm, \it, etc. must \def\@currfont to itself.
%   Every size definition such as \normalsize is defined to be
%          \@normalsize
%          \def\@currsize{\@normal}\rm 
%          commands to set:
%             \baselineskip 
%             \lineskip
%             the ...displayskip and ...displayshortskip parameters
%             \strutbox
%             \parskip and \@parskip
%   where \@normal makes all the font definitions.
%   \@normal must test the switch @visible and define its 
%   fonts equal to the visible or the invisible versions accordingly.

 \message{hacks,}
%      **********************************************
%      *          HACKS FOR SLIDE MACROS            *
%      **********************************************
% 
% The macro \@getend{FOO} defines \@arg to equal all the text up to
% but excluding the next \end{FOO}, which it gobbles up.  
% Note that the characters in this text will all be interpreted with
% their current category codes, so any embedded environments (such as
% an example environment) won't work right.  NOTE--GOBBLES
% THE \fi's -- MUST BE MODIFIED LIKE \@gobbletoend BELOW.


%\def\@getend#1{\def\@arg{}\def\@argend{#1}\@addtoarg}
%
%\long\def\@addtoarg#1\end#2{\tokens{#1}%
%\long\edef\@x@add##1\@@{\long\def\@arg{##1\the\tokens}}%
%\expandafter\@x@add\@arg\@@\def\@x@a{#2}\ifx\@x@a\@argend\else
%\tokens{\end{#2}}\long\edef\@x@add##1\@@{\long\def\@arg{##1\the\tokens}}%
%\expandafter\@x@add\@arg\@@\@addtoarg\fi}

% \@gobbletoend{FOO} gobbles all text up to and including the
% next \end{FOO}.  Must be used inside an \if, right before the \fi.

\def\@gobbletoend#1{\def\@argend{#1}\@ggobtoend}

\long\def\@ggobtoend#1\end#2{\@xfi\def\@x@a{#2}%
\ifx\@x@a\@argend\else\@ggobtoend\fi}
\def\@xfi{\fi}

 \message{slides,}
%      **********************************************
%      *               SLIDE  MACROS                *
%      **********************************************
% 
% Switches:
% @bw      : true if making black and white slides
% @visible : true if visible output to be produced.
% @makingslides : true if executing \blackandwhite or \colorslides

\newif\if@bw
\newif\if@visible
\newif\if@onlyslidesw \@onlyslideswfalse
\newif\if@onlynotesw  \@onlynoteswfalse
\newif\if@makingslides

% Counters
%  slide   = slide number
%  overlay = overlay number for a slide
%  note    = note number for a slide

\countdef\c@slide=0 \c@slide=0
\def\cl@slide{}
\countdef\c@overlay=1 \c@overlay=0
\def\cl@overlay{}
\countdef\c@note=2 \c@note=0
\def\cl@note{}


\@addtoreset{overlay}{slide}
\@addtoreset{note}{slide}

% Redefine page counter to some other number.
% The page counter will always be zero except when putting out an
% extra page for a slide, note or overlay.
%
\@definecounter{page} 
\@addtoreset{page}{slide}
\@addtoreset{page}{note}
\@addtoreset{page}{overlay}


\def\theslide{\@arabic\c@slide}
\def\theoverlay{\theslide-\@alph\c@overlay}
\def\thenote{\theslide-\@arabic\c@note}

% \@setlimits \LIST \LOW \HIGH 
%
%    Assumes that \LIST = RANGE1,RANGE2,...,RANGEn  (n>0)
%    Where RANGEi = j or j-k.
%    
%    Then \@setlimits  globally sets
%        (i) \LIST := RANGE2, ... , RANGEn
%       (ii) \LOW  := p
%      (iii) \HIGH := q
%   where either RANGE1 = p-q   or  RANGE1 = p  and  q=p.

\def\@sl@getargs#1-#2-#3\relax#4#5{\xdef#4{#1}\xdef#5{#2}}
\def\@sl@ccdr#1,#2\relax#3#4{\xdef#3{#1-#1-}\xdef#4{#2}}

\def\@setlimits #1#2#3{\expandafter\@sl@ccdr#1\relax\@sl@gtmp #1%
\expandafter\@sl@getargs\@sl@gtmp\relax#2#3}

% \onlyslides{LIST} ::=
%  BEGIN
%    @onlyslidesw := true
%    \@doglslidelist :=G LIST,999999,999999
%   if @onlynotesw = true
%     else @onlynotesw := true
%          \@doglnotelist :=G LIST,999999,999999
%   fi
%   message: Only Slides LIST
%  END

\def\onlyslides#1{\@onlyslideswtrue\gdef\@doglslidelist{#1,999999,999999}%
\if@onlynotesw \else \@onlynoteswtrue\gdef\@doglnotelist{999999,999999}\fi
\typeout{Only Slides #1}}

% \onlynotes{LIST} ::=
%  BEGIN
%    @onlynotesw := true
%    \@doglnotelist :=G LIST,999999,999999
%   if @onlyslidesw = true
%     else \@onlyslidesw := true
%          \@doglslidelist{999999,999999}
%   fi
%   message: Only Notes LIST
%  END

\def\onlynotes#1{\@onlynoteswtrue\gdef\@doglnotelist{#1,999999,999999}%
\if@onlyslidesw \else \@onlyslideswtrue\gdef\@doglslidelist{999999,999999}\fi
\typeout{Only Notes #1}}


% \blackandwhite #1  ::=
%    \newpage
%    page counter := 0
%    @bw := T
%    @visible := T
%    if @onlyslidesw = true
%      then  \@doslidelist := \@doglslidelist
%            \@setlimits\@doslidelist\@doslidelow\@doslidehigh
%    fi
%    if @onlynotesw = true
%      then  \@donotelist := \@doglnotelist
%            \@setlimits\@donotelist\@donotelow\@donotehigh
%    fi
%    \normalsize    % Note, this sets font to \rm , which sets
%                     % \@currfont to \rm
%    counter slidenumber := 0
%    counter note        := 0
%    counter overlay     := 0
%    @makingslides       := T
%    input #1
%    @makingslides       := F

\def\blackandwhite#1{\newpage\setcounter{page}{0}\@bwtrue\@visibletrue
\if@onlyslidesw \xdef\@doslidelist{\@doglslidelist}%
\@setlimits\@doslidelist\@doslidelow\@doslidehigh\fi
\if@onlynotesw \xdef\@donotelist{\@doglnotelist}%
\@setlimits\@donotelist\@donotelow\@donotehigh\fi
\normalsize\setcounter{slide}{0}\setcounter{overlay}{0}%
\setcounter{note}{0}\@makingslidestrue\input #1\@makingslidesfalse}


% \colors{COLORS} ::=
%  for \@colortemp := COLORS
%     do \csname \@colortemp \endcsname == \@color{\@colortemp} od
%  if \@colorlist = empty
%     then \@colorlist := COLORS
%     else \@colorlist := \@colorlist , COLORS
%  fi
%
\def\colors#1{\@for\@colortemp:=#1\do{\expandafter
  \xdef\csname\@colortemp\endcsname{\noexpand\@color{\@colortemp}}}\ifx
  \@colorlist\@empty \gdef\@colorlist{#1}
    \else \xdef\@colorlist{\@colorlist,#1}\fi}

\def\@colorlist{}

% \colorslides{FILE} ::=
%    \newpage
%    page counter := 0
%    @bw := F
%    for \@currcolor := \@colorlist
%      do  @visible := T
%          if @onlyslidesw = true
%            then  \@doslidelist := \@doglslidelist
%                  \@setlimits\@doslidelist\@doslidelow\@doslidehigh
%          fi
%          if @onlynotesw = true
%            then  \@donotelist := \@doglnotelist
%                  \@setlimits\@donotelist\@donotelow\@donotehigh
%          fi
%          \normalsize
%          counter slide := 0
%          counter overlay := 0
%          counter note    := 0
%          type message
%          generate color layer output page
%          @makingslides := T
%          input #1
%          @makingslides := F
%      od

\def\colorslides#1{\newpage\setcounter{page}{0}\@bwfalse
\@for\@currcolor:=\@colorlist\do
{\@visibletrue
\if@onlyslidesw \xdef\@doslidelist{\@doglslidelist}%
\@setlimits\@doslidelist\@doslidelow\@doslidehigh\fi
\if@onlynotesw \xdef\@donotelist{\@doglnotelist}%
\@setlimits\@donotelist\@donotelow\@donotehigh\fi
\normalsize\setcounter{slide}{0}\setcounter{overlay}{0}%
\setcounter{note}{0}\typeout{color \@currcolor}%
\newpage
\begin{huge}
\begin{center}
COLOR LAYER\\[.75in]
\@currcolor
\end{center}
\end{huge}
\newpage
\@makingslidestrue
\input #1
\@makingslidesfalse}}


% \slide COLORS ::=
%  BEGIN
%   \stepcounter{slide}   
%   \@slidesw :=G T
%   if @onlyslidesw = true                     % set \@slidesw = T iff page to
%     then                                   % be output
%       while \c@slide > \@doslidehigh
%          do  \@setlimits\@doslidelist\@doslidelow\@doslidehigh  od
%       if \c@slide < \@doslidelow
%         then \@slidesw := F
%       fi
%   fi
%   if \@slidesw = T
%      then \@slidesw :=G F
%           \begingroup                           
%              if @bw = true  
%                then  \@slidesw :=G T   
%                else \@color{COLORS}
%                     \if@visible then \@slidesw :=G T \fi
%              fi
%            \endgroup
%  fi
%  if \@slidesw = T
%    then \newpage
%         \thispagestyle{slide}
%    else \end{slide}
%          \@gobbletoend{slide}
%  fi
% END

% \endslide ::=
%  BEGIN
%    \par\break
%  END

\def\slide#1{\stepcounter{slide}\gdef\@slidesw{T}\if@onlyslidesw
\@whilenum \c@slide > \@doslidehigh\relax
\do{\@setlimits\@doslidelist\@doslidelow\@doslidehigh}\ifnum
\c@slide < \@doslidelow\relax\gdef\@slidesw{F}\fi\fi
\if\@slidesw T\gdef\@slidesw{F}\begingroup\if@bw\gdef\@slidesw{T}\else
\@color{#1}\if@visible \gdef\@slidesw{T}\fi\fi\endgroup\fi
\if\@slidesw T\newpage\thispagestyle{slide}%
\else\end{slide}\@gobbletoend{slide}\fi}

\def\endslide{\par\break}

% \overlay COLORS ::=
%  BEGIN
%   \stepcounter{overlay}   
%   \@slidesw :=G T
%   if @onlyslidesw = T                       % set \@slidesw = T iff page to
%     then                                  % be output
%       if \c@slide < \@doslidelow
%         then \@slidesw :=G F
%       fi
%  fi
%  if \@slidesw = T
%    \@slidesw :=G F
%    \begingroup 
%      if @bw = true 
%          then  \@slidesw :=G T 
%          else  \@color{COLORS}
%                \if@visible then \@slidesw :=G T \fi
%      fi
%    \endgroup
%  fi
%  if \@slidesw = T
%     then \newpage
%          \thispagestyle{overlay}
%     else \end{overlay}
%          \@gobbletoend{overlay}
%  fi
% END

% \endoverlay ::=
%  BEGIN
%    \par\break
%  END

\def\overlay#1{\stepcounter{overlay}\gdef\@slidesw{T}%
\if@onlyslidesw\ifnum \c@slide < \@doslidelow\relax
\gdef\@slidesw{F}\fi\fi
\if\@slidesw T\gdef\@slidesw{F}\begingroup\if@bw\gdef\@slidesw{T}%
\else\@color{#1}\if@visible \gdef\@slidesw{T}\fi\fi\endgroup\fi
\if\@slidesw T\newpage\thispagestyle{overlay}%
\else\end{overlay}\@gobbletoend{overlay}\fi}

\def\endoverlay{\par\break}

% \note ::=
%  BEGIN
%   \stepcounter{note}   
%   if @bw = T
%     then 
%       \@slidesw :=G T
%       if @onlynotesw = true                  % set \@notesw = T iff page to
%         then                                 % be output
%           while \c@slide > \@donotehigh
%              do  \@setlimits\@donotelist\@donotelow\@donotehigh  od
%           if \c@slide < \@donotelow
%             then \@slidesw :=G F
%           fi
%       fi
%     else \@slidesw :=G F
%  fi
%  if \@slidesw = T
%     then \newpage
%          \thispagestyle{note}
%     else \end{note}
%          \@gobbletoend{note}
%  fi
% END

% \endnote ::=
%  BEGIN
%    \par\break
%  END

\def\note{\stepcounter{note}%
\if@bw \gdef\@slidesw{T}\if@onlynotesw\@whilenum \c@slide > \@donotehigh\relax
\do{\@setlimits\@donotelist\@donotelow\@donotehigh}\ifnum
\c@slide < \@donotelow\relax \gdef\@slidesw{F}\fi\fi
\else\gdef\@slidesw{F}\fi
\if\@slidesw T\newpage\thispagestyle{note}\else
\end{note}\@gobbletoend{note}\fi}

\def\endnote{\par\break}


% \@color{COLORS} ::=
%  BEGIN
%   if math mode
%     then type warning
%   fi
%   if @bw
%     then @visible := T
%     else @visible := F
%         for \@x@a := COLORS
%          do if \@x@a = \@currcolor
%               then @visible := T
%             fi
%          od
%   fi
%   \@currsize -- sets the visibility of the current size
%   \@currfont -- sets the visibility of the current font
%   \ignorespaces
% END

\def\@color#1{\@mmodetest
\if@bw \@visibletrue\else\@visiblefalse
  \@for \@x@a :=#1\do{\ifx\@x@a\@currcolor\@visibletrue\fi}\fi
  \@currsize\@currfont\ignorespaces}

\def\@mmodetest{\ifmmode\@warning{Color-changing command in math mode}\fi}

% \invisible ::= BEGIN type warning if math mode
%                      \@visiblefalse \@currsize\@currfont\ignorespaces END

\def\invisible{\@mmodetest\@visiblefalse\@currsize\@currfont\ignorespaces}
\let\invisible=\invisible
\let\endinvisible=\relax

 \message{picture,}
%      ****************************************
%      *          MODIFICATIONS TO            *
%      *       THE PICTURE ENVIRONMENT        *
%      ****************************************
%
%  Below are the new definitions of the picture-drawing macros
%  required for SLiTeX.  Only those commands that actually
%  draw something must be changed so that they do not produce
%  any output when the @visible switch is false.

\def\line(#1,#2)#3{\if@visible\@xarg #1\relax \@yarg #2\relax
\@linelen=#3\unitlength
\ifnum\@xarg =0 \@vline 
  \else \ifnum\@yarg =0 \@hline \else \@sline\fi
\fi\fi}

\def\vector(#1,#2)#3{\if@visible\@xarg #1\relax \@yarg #2\relax
\@linelen=#3\unitlength
\ifnum\@xarg =0 \@vvector 
  \else \ifnum\@yarg =0 \@hvector \else \@svector\fi
\fi\fi}

\def\dashbox#1(#2,#3){\leavevmode\if@visible\hbox to \z@{\baselineskip \z@%
\lineskip \z@%
\@dashdim=#2\unitlength%
\@dashcnt=\@dashdim \advance\@dashcnt 200
\@dashdim=#1\unitlength\divide\@dashcnt \@dashdim
\ifodd\@dashcnt\@dashdim=\z@%
\advance\@dashcnt \@ne \divide\@dashcnt \tw@ 
\else \divide\@dashdim \tw@ \divide\@dashcnt \tw@
\advance\@dashcnt \m@ne
\setbox\@dashbox=\hbox{\vrule \@height \@halfwidth \@depth \@halfwidth
\@width \@dashdim}\put(0,0){\copy\@dashbox}%
\put(0,#3){\copy\@dashbox}%
\put(#2,0){\hskip-\@dashdim\copy\@dashbox}%
\put(#2,#3){\hskip-\@dashdim\box\@dashbox}%
\multiply\@dashdim 3 
\fi
\setbox\@dashbox=\hbox{\vrule \@height \@halfwidth \@depth \@halfwidth
\@width #1\unitlength\hskip #1\unitlength}\@tempcnta=0
\put(0,0){\hskip\@dashdim \@whilenum \@tempcnta <\@dashcnt
\do{\copy\@dashbox\advance\@tempcnta \@ne }}\@tempcnta=0
\put(0,#3){\hskip\@dashdim \@whilenum \@tempcnta <\@dashcnt
\do{\copy\@dashbox\advance\@tempcnta \@ne }}%
\@dashdim=#3\unitlength%
\@dashcnt=\@dashdim \advance\@dashcnt 200
\@dashdim=#1\unitlength\divide\@dashcnt \@dashdim
\ifodd\@dashcnt \@dashdim=\z@%
\advance\@dashcnt \@ne \divide\@dashcnt \tw@
\else
\divide\@dashdim \tw@ \divide\@dashcnt \tw@
\advance\@dashcnt \m@ne
\setbox\@dashbox\hbox{\hskip -\@halfwidth
\vrule \@width \@wholewidth 
\@height \@dashdim}\put(0,0){\copy\@dashbox}%
\put(#2,0){\copy\@dashbox}%
\put(0,#3){\lower\@dashdim\copy\@dashbox}%
\put(#2,#3){\lower\@dashdim\copy\@dashbox}%
\multiply\@dashdim 3
\fi
\setbox\@dashbox\hbox{\vrule \@width \@wholewidth 
\@height #1\unitlength}\@tempcnta0
\put(0,0){\hskip -\@halfwidth \vbox{\@whilenum \@tempcnta < \@dashcnt
\do{\vskip #1\unitlength\copy\@dashbox\advance\@tempcnta \@ne }%
\vskip\@dashdim}}\@tempcnta0
\put(#2,0){\hskip -\@halfwidth \vbox{\@whilenum \@tempcnta< \@dashcnt
\relax\do{\vskip #1\unitlength\copy\@dashbox\advance\@tempcnta \@ne }%
\vskip\@dashdim}}}\fi\@makepicbox(#2,#3)}

\def\@oval(#1,#2)[#3]{\if@visible\begingroup \boxmaxdepth \maxdimen
  \@ovttrue \@ovbtrue \@ovltrue \@ovrtrue
  \@tfor\@tempa :=#3\do{\csname @ov\@tempa false\endcsname}\@ovxx
  #1\unitlength \@ovyy #2\unitlength
  \@tempdimb \ifdim \@ovyy >\@ovxx \@ovxx\else \@ovyy \fi
  \@getcirc \@tempdimb
  \@ovro \ht\@tempboxa \@ovri \dp\@tempboxa
  \@ovdx\@ovxx \advance\@ovdx -\@tempdima \divide\@ovdx \tw@
  \@ovdy\@ovyy \advance\@ovdy -\@tempdima \divide\@ovdy \tw@
  \@circlefnt \setbox\@tempboxa
  \hbox{\if@ovr \@ovvert32\kern -\@tempdima \fi
  \if@ovl \kern \@ovxx \@ovvert01\kern -\@tempdima \kern -\@ovxx \fi
  \if@ovt \@ovhorz \kern -\@ovxx \fi
  \if@ovb \raise \@ovyy \@ovhorz \fi}\advance\@ovdx\@ovro
  \advance\@ovdy\@ovro \ht\@tempboxa\z@ \dp\@tempboxa\z@
  \@put{-\@ovdx}{-\@ovdy}{\box\@tempboxa}%
  \endgroup\fi}

\def\@circle#1{\if@visible \begingroup \boxmaxdepth \maxdimen 
   \@tempdimb #1\unitlength
   \ifdim \@tempdimb >15.5pt\relax \@getcirc\@tempdimb
      \@ovro\ht\@tempboxa 
     \setbox\@tempboxa\hbox{\@circlefnt
      \advance\@tempcnta\tw@ \char \@tempcnta
      \advance\@tempcnta\m@ne \char \@tempcnta \kern -2\@tempdima
      \advance\@tempcnta\tw@
      \raise \@tempdima \hbox{\char\@tempcnta}\raise \@tempdima
        \box\@tempboxa}\ht\@tempboxa\z@ \dp\@tempboxa\z@
      \@put{-\@ovro}{-\@ovro}{\box\@tempboxa}%
   \else  \@circ\@tempdimb{96}\fi\endgroup\fi}

\def\@dot#1{\if@visible\@tempdimb #1\unitlength \@circ\@tempdimb{112}\fi}

\long\def\@iframebox[#1][#2]#3{\leavevmode
  \savebox\@tempboxa[#1][#2]{\kern\fboxsep #3\kern\fboxsep}\@tempdima\fboxrule
    \advance\@tempdima \fboxsep \advance\@tempdima \dp\@tempboxa
   \hbox{\lower \@tempdima\hbox
  {\vbox{ \if@visible \hrule \@height \else \vskip \fi \fboxrule 
          \hbox{\if@visible \vrule \@width \fboxrule \hskip-\fboxrule \fi
                \vbox{\vskip\fboxsep \box\@tempboxa\vskip\fboxsep}\if@visible 
                \vrule \@width \fboxrule \hskip-\fboxrule \fi}\if@visible 
         \hrule \@height \else \vskip\fi\fboxrule}}}}
         

\long\def\frame#1{\if@visible\leavevmode
\vbox{\vskip-\@halfwidth\hrule \@height\@halfwidth \@depth \@halfwidth
  \vskip-\@halfwidth\hbox{\hskip-\@halfwidth \vrule \@width\@wholewidth
  \hskip-\@halfwidth #1\hskip-\@halfwidth \vrule \@width \@wholewidth
  \hskip -\@halfwidth}\vskip -\@halfwidth\hrule \@height \@halfwidth
  \@depth \@halfwidth\vskip -\@halfwidth}\else #1\fi}

\long\def\fbox#1{\leavevmode\setbox\@tempboxa\hbox{#1}\@tempdima\fboxrule
    \advance\@tempdima \fboxsep \advance\@tempdima \dp\@tempboxa
   \hbox{\lower \@tempdima\hbox
  {\vbox{\if@visible \hrule \@height \else \vskip\fi\fboxrule 
          \hbox{\if@visible\vrule \@width \else \hskip \fi\fboxrule 
                 \hskip\fboxsep
            \vbox{\vskip\fboxsep \box\@tempboxa\vskip\fboxsep}\hskip 
                   \fboxsep \if@visible\vrule \@width\else\hskip \fi\fboxrule}
          \if@visible\hrule \@height \else \vskip \fi\fboxrule}}}}

 \message{mods,}
%      ****************************************
%      *        OTHER MODIFICATIONS TO        *
%      *        TeX AND LaTeX COMMANDS        *
%      ****************************************
%

% \rule

\def\@rule[#1]#2#3{\@tempdima#3\advance\@tempdima #1\leavevmode
 \hbox{\if@visible\vrule
  \@width#2 \@height\@tempdima \@depth-#1\else
\vrule \@width \z@ \@height\@tempdima \@depth-#1\vrule 
 \@width#2 \@height\z@\fi}}

% \_  (Added 10 Nov 86)

\def\_{\leavevmode \kern.06em \if@visible\vbox{\hrule width.3em}\else
   \vbox{\hrule height 0pt width.3em}\vbox{\hrule width 0pt}\fi}

% \overline, \underline, \frac and \sqrt
%
% \@mathbox{STYLE}{BOX}{MTEXT} : Called in math mode, typesets MTEXT and
%   stores result in BOX, using style STYLE.
%
% \@bphant{BOX}    : Creates a phantom with dimensions BOX.
% \@vbphant{BOX}   : Creates a phantom with ht of BOX and zero width.
% \@hbphant{BOX}   : Creates a phantom with width of BOX and zero ht & dp.
% \@hvsmash{STYLE}{MTEXT} : Creates a copy of MTEXT with zero height and width
%                           in style STYLE.

\def\@mathbox#1#2#3{\setbox#2\hbox{$\m@th#1{#3}$}}

\def\@vbphantom#1{\setbox\tw@\null \ht\tw@\ht #1 \dp\tw@\dp #1
   \box\tw@}

\def\@bphantom#1{\setbox\tw@\null \wd\tw@\wd #1 \ht\tw@\ht #1 \dp\tw@\dp #1
   \box\tw@}

\def\@hbphantom#1{\setbox\tw@\null \wd\tw@\wd #1 \ht\tw@\z@ \dp\tw@\z@
   \box\tw@}

\def\@hvsmash#1#2{\@mathbox#1\z@{#2}\ht\z@\z@ \dp\z@\z@ \wd\z@\z@ \box\z@}

\def\underline#1{\relax\ifmmode 
  \@xunderline{#1}\else $\@xunderline{\hbox{#1}}$\relax\fi}

\def\@xunderline#1{\mathchoice{\@xyunderline\displaystyle{#1}}{\@xyunderline
    \textstyle{#1}}{\@xyunderline\scriptstyle{#1}}{\@xyunderline
      \scriptscriptstyle{#1}}}

\def\@xyunderline#1#2{\@mathbox#1\@smashboxa{#2}\@hvsmash#1{\copy\@smashboxa}
   \if@visible \@hvsmash#1{\@@underline{\@bphantom\@smashboxa}}\fi
  \@mathbox#1\@smashboxb{\@@underline{\box\@smashboxa}}
   \@bphantom\@smashboxb}

\let\@@overline=\overline 

\def\overline#1{\mathchoice{\@xoverline\displaystyle{#1}}{\@xoverline
    \textstyle{#1}}{\@xoverline\scriptstyle{#1}}{\@xoverline
      \scriptscriptstyle{#1}}}

\def\@xoverline#1#2{\@mathbox#1\@smashboxa{#2}\@hvsmash#1{\copy\@smashboxa}
   \if@visible \@hvsmash#1{\@@overline{\@bphantom\@smashboxa}}\fi
   \@mathbox#1\@smashboxb{\@@overline{\box\@smashboxa}}
   \@bphantom\@smashboxb}


% \@frac {STYLE}{DENOMSTYLE}{NUM}{DEN}{FONTSIZE} : Creates \frac{NUM}{DENOM}
%   in style SYTLE with NUM and DENOM in style DENOMSTYLE
%   FONTSIZE should be \textfont \scriptfont or \scriptscriptfont

\def\frac#1#2{\mathchoice
   {\@frac\displaystyle\textstyle{#1}{#2}\textfont}{\@frac
          \textstyle\scriptstyle{#1}{#2}\textfont}{\@frac
          \scriptstyle\scriptscriptstyle{#1}{#2}\scriptfont}{\@frac
          \scriptscriptstyle\scriptscriptstyle{#1}{#2}\scriptscriptfont}}

\def\@frac#1#2#3#4#5{\@mathbox#2\@smashboxa{#3}\@mathbox#2\@smashboxb{#4}
   \@mathbox#1\@smashboxc{\copy\@smashboxa\over\copy\@smashboxb}   
   \@vbphantom\@smashboxc
   \vcenter{\vbox to \z@{\hsize \wd\@smashboxc
                         \vss\nointerlineskip
                         \hbox to \wd\@smashboxc{\hss\box\@smashboxa\hss}
              \hrule height \z@}
          \vskip 7\fontdimen8#53            
          \if@visible\hrule height \fontdimen8#53\else \vskip \fontdimen8#53\fi
          \vskip 7\fontdimen8#53            
          \nointerlineskip
          \vbox to \z@{\nointerlineskip
                         \hbox to \wd\@smashboxc{\hss\box\@smashboxb\hss}
              \hrule height \z@\vss}
           }}

\def\r@@t#1#2{\setbox\z@\hbox{$\m@th#1\@xysqrt#1{#2}$}
  \dimen@\ht\z@ \advance\dimen@-\dp\z@
  \mskip5mu\raise.6\dimen@\copy\rootbox \mskip-10mu \box\z@}

\def\sqrt{\@ifnextchar[{\@sqrt}{\@xsqrt}}
\def\@sqrt[#1]{\root #1\of}
\def\@xsqrt#1{\mathchoice{\@xysqrt\displaystyle{#1}}{\@xysqrt
     \textstyle{#1}}{\@xysqrt\scriptstyle{#1}}{\@xysqrt
    \scriptscriptstyle{#1}}}
\def\@xysqrt#1#2{\@mathbox#1\@smashboxa{#2}\if@visible
    \@hvsmash#1{\@@sqrt{\@bphantom\@smashboxa}}\fi
    \phantom{\@@sqrt{\@vbphantom\@smashboxa}}\box\@smashboxa}

\newbox\@smashboxa
\newbox\@smashboxb
\newbox\@smashboxc

% array and tabular environments: changes to `|', \hline, \cline, and \vline
% added 8 Jun 88

\def\@arrayrule{\if@visible\@addtopreamble{\hskip -.5\arrayrulewidth 
   \vrule \@width \arrayrulewidth\hskip -.5\arrayrulewidth}\fi}

\def\cline#1{\if@visible\@cline[#1]\fi}


\def\hline{\noalign{\ifnum0=`}\fi
    \if@visible \hrule \@height \arrayrulewidth 
      \else \hrule \@width \z@ 
    \fi
    \futurelet \@tempa\@xhline}


\def\vline{\if@visible \vrule \@width \arrayrulewidth
            \else \vrule \@width \arrayrulewidth \@height \z@ 
           \@depth \z@ \fi}


 \message{output,}
%      ****************************************
%      *   CHANGES TO LaTeX \output ROUTINE   *
%      ****************************************
%
%  \@makecol ==
%    BEGIN
% % Following test added for slides to check if extra page
%     if @makingslides = T
%     then if \c@page > 0
%             then if \c@note > 0
%                     then type 'Note \thenote too long.'
%                     else if \c@overlay > 0
%                            then type 'Overlay \theoverlay too long.'
%                            else type 'Slide \theslide too long'
%     fi     fi     fi     fi
%     ifvoid \insert\footins
%        then  \@outputbox := \box255
%        else  \@outputbox := \vbox {\unvbox255
%                                    \vskip \skip\footins
%                                    \footnoterule
%                                    \unvbox\@footinsert
%                                   }
%    fi
%    \@freelist :=G \@freelist * \@midlist
%    \@midlist  :=G empty
%    \@combinefloats
%    \@outputbox := \vbox to \@colht{\boxmaxdepth := \maxdepth
%                                     \vfil     %%\vfil added for slides
%                                     \unvbox\@outputbox
%                                     \vfil }   %%\vfil added for slides
%    \maxdepth :=G \@maxdepth
%    END

\def\@makecol{\if@makingslides\ifnum\c@page>\z@ \@extraslide\fi\fi
\ifvoid\footins \setbox\@outputbox\box\@cclv \let\@botfil\vfil
   \else\let\@botfil\relax\setbox\@outputbox
     \vbox{\unvbox\@cclv\vfil
           \vskip\skip\footins\footnoterule\unvbox\footins\vskip
            \z@ plus.1fil\relax}\fi
  \xdef\@freelist{\@freelist\@midlist}\gdef\@midlist{}\@combinefloats
     \setbox\@outputbox\vbox to\@colht{\boxmaxdepth\maxdepth
        \vfil\unvbox\@outputbox\@botfil}\global\maxdepth\@maxdepth}

\def\@extraslide{\ifnum\c@note>\z@ 
    \@warning{Note \thenote\space too long}\else 
     \ifnum\c@overlay>\z@ 
        \@warning{Overlay \theoverlay\space too long}\else 
        \@warning{Slide \theslide\space too long}\fi\fi}

 \message{init}
%      ****************************************
%      *    SPECIAL SLiTeX INITIALIZATIONS    *
%      ****************************************
%
\nofiles
\@visibletrue

